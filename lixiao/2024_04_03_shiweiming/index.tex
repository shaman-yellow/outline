% Options for packages loaded elsewhere
\PassOptionsToPackage{unicode}{hyperref}
\PassOptionsToPackage{hyphens}{url}
%
\documentclass[
  ignorenonframetext,
]{beamer}
\usepackage{pgfpages}
\setbeamertemplate{caption}[numbered]
\setbeamertemplate{caption label separator}{: }
\setbeamercolor{caption name}{fg=normal text.fg}
\beamertemplatenavigationsymbolsempty
% Prevent slide breaks in the middle of a paragraph
\widowpenalties 1 10000
\raggedbottom
\setbeamertemplate{part page}{
  \centering
  \begin{beamercolorbox}[sep=16pt,center]{part title}
    \usebeamerfont{part title}\insertpart\par
  \end{beamercolorbox}
}
\setbeamertemplate{section page}{
  \centering
  \begin{beamercolorbox}[sep=12pt,center]{part title}
    \usebeamerfont{section title}\insertsection\par
  \end{beamercolorbox}
}
\setbeamertemplate{subsection page}{
  \centering
  \begin{beamercolorbox}[sep=8pt,center]{part title}
    \usebeamerfont{subsection title}\insertsubsection\par
  \end{beamercolorbox}
}
\AtBeginPart{
  \frame{\partpage}
}
\AtBeginSection{
  \ifbibliography
  \else
    \frame{\sectionpage}
  \fi
}
\AtBeginSubsection{
  \frame{\subsectionpage}
}
\usepackage{lmodern}
\usepackage{amssymb,amsmath}
\usepackage{ifxetex,ifluatex}
\ifnum 0\ifxetex 1\fi\ifluatex 1\fi=0 % if pdftex
  \usepackage[T1]{fontenc}
  \usepackage[utf8]{inputenc}
  \usepackage{textcomp} % provide euro and other symbols
\else % if luatex or xetex
  \usepackage{unicode-math}
  \defaultfontfeatures{Scale=MatchLowercase}
  \defaultfontfeatures[\rmfamily]{Ligatures=TeX,Scale=1}
\fi
% Use upquote if available, for straight quotes in verbatim environments
\IfFileExists{upquote.sty}{\usepackage{upquote}}{}
\IfFileExists{microtype.sty}{% use microtype if available
  \usepackage[]{microtype}
  \UseMicrotypeSet[protrusion]{basicmath} % disable protrusion for tt fonts
}{}
\makeatletter
\@ifundefined{KOMAClassName}{% if non-KOMA class
  \IfFileExists{parskip.sty}{%
    \usepackage{parskip}
  }{% else
    \setlength{\parindent}{0pt}
    \setlength{\parskip}{6pt plus 2pt minus 1pt}}
}{% if KOMA class
  \KOMAoptions{parskip=half}}
\makeatother
\usepackage{xcolor}
\IfFileExists{xurl.sty}{\usepackage{xurl}}{} % add URL line breaks if available
\IfFileExists{bookmark.sty}{\usepackage{bookmark}}{\usepackage{hyperref}}
\hypersetup{
  hidelinks,
  pdfcreator={LaTeX via pandoc}}
\urlstyle{same} % disable monospaced font for URLs
\newif\ifbibliography
\usepackage{longtable,booktabs}
\usepackage{caption}
% Make caption package work with longtable
\makeatletter
\def\fnum@table{\tablename~\thetable}
\makeatother
\setlength{\emergencystretch}{3em} % prevent overfull lines
\providecommand{\tightlist}{%
  \setlength{\itemsep}{0pt}\setlength{\parskip}{0pt}}
\setcounter{secnumdepth}{-\maxdimen} % remove section numbering

\author{}
\date{\vspace{-2.5em}}

\begin{document}

\begin{frame}
\begin{titlepage} \newgeometry{top=7.5cm}
\ThisCenterWallPaper{1.12}{~/outline/lixiao//cover_page.pdf}
\begin{center} \textbf{\Huge
XX基因通过促进糖酵解促进巨噬细胞M1极化}
\vspace{4em} \begin{textblock}{10}(3,5.9) \huge
\textbf{\textcolor{white}{2024-06-03}}
\end{textblock} \begin{textblock}{10}(3,7.3)
\Large \textcolor{black}{LiChuang Huang}
\end{textblock} \begin{textblock}{10}(3,11.3)
\Large \textcolor{black}{@立效研究院}
\end{textblock} \end{center} \end{titlepage}
\restoregeometry

\pagenumbering{roman}

\tableofcontents

\listoffigures

\listoftables



\pagenumbering{arabic}
\end{frame}

\hypertarget{abstract}{%
\section{摘要}\label{abstract}}

\begin{frame}{生信需求}
\protect\hypertarget{ux751fux4fe1ux9700ux6c42}{}
疾病:类风湿性关节炎RA 物种:临床患者或者动物模型都可以 细胞:巨噬细胞

目标:筛出XX基因,XX基因满足, 1、是糖酵解相关基因
2、与巨噬细胞极化相关(M1/M2)

设想:XX基因在RA中上调,RA中M1巨噬细胞上调,其中XX基因主要分布在M1巨噬细胞而非M2巨噬细胞上,XX基因可能通过促进糖酵解促进巨噬细胞M1极化

M1标志:iNOS,CD11c,CD86等 M2标志:CD206,IL-10,TGF-beta等
\end{frame}

\begin{frame}[fragile]{结果}
\protect\hypertarget{ux7ed3ux679c}{}
\begin{itemize}
\tightlist
\item
  首先通过分析 GEO 单细胞数据,鉴定出巨噬细胞不同表型 Fig.
  @ref(fig:The-Phenotypes)。
\item
  该数据集为小鼠来源,鉴定 M0、M1、M2 的小鼠基因 Marker
  参考{[}@NovelMarkersTJablon2015{]}, 实际使用的 Marker 见 Tab.
  @ref(tab:The-markers-for-Macrophage-phenotypes-annotation)
\item
  进行差异分析 (Tab. @ref(tab:DEGs-of-the-contrasts)) :

  \begin{itemize}
  \tightlist
  \item
    XX 在RA中M1巨噬细胞上调: \texttt{GPI-day25-RA\_Macrophage\_M1} vs
    \texttt{Control\_Macrophage\_M1}
  \item
    其中XX基因主要分布在M1巨噬细胞而非M2巨噬细胞上:
    \texttt{GPI-day25-RA\_Macrophage\_M1} vs
    \texttt{GPI-day25-RA\_Macrophage\_M2}
  \end{itemize}
\item
  以上两组差异基因交集见 Fig.
  @ref(fig:Intersection-of-RA-M1-up-with-M1-not-M2)
\item
  小鼠基因映射到人类 Tab. @ref(tab:Mapped-genes)
\item
  其中糖酵解相关的基因见 Fig.
  @ref(fig:Intersection-of-RA-M1M2-related-with-Glycolysis-related)
\item
  筛选到唯一的基因: PPARG (小鼠 Pparg)。其表达特征见 Fig.
  @ref(fig:Violing-plot-of-expression-level-of-the-Pparg)
\end{itemize}
\end{frame}

\begin{frame}{进一步分析需求}
\protect\hypertarget{ux8fdbux4e00ux6b65ux5206ux6790ux9700ux6c42}{}
利用开源数据库进行生物信息学分析,筛选并验证类风湿性关节炎临床患者和动物模型中与巨噬细胞极化和糖酵解相关的关键基因XX的表达情况

\begin{itemize}
\tightlist
\item
  XX (VWF)
  表达水平与炎症因子、巨噬细胞浸润、巨噬细胞极化相关因子、糖酵解相关因子的相关性
\item
  与患者状态(例如血清类风湿因子(RF)、抗链球菌溶血素抗体(ASO)、红细胞沉降率(ESR)和C反应蛋白(CRP))的相关性
\end{itemize}
\end{frame}

\begin{frame}{进一步分析结果}
\protect\hypertarget{ux8fdbux4e00ux6b65ux5206ux6790ux7ed3ux679c}{}
\begin{itemize}
\tightlist
\item
  关联分析结果见 Fig. @ref(fig:HUMAN-correlation-heatmap), Tab.
  @ref(tab:HUMAN-correlation)。
\item
  未找到可用的 RA 表型数据集。
\end{itemize}
\end{frame}

\hypertarget{introduction}{%
\section{前言}\label{introduction}}

\hypertarget{methods}{%
\section{材料和方法}\label{methods}}

\begin{frame}{材料}
\protect\hypertarget{ux6750ux6599}{}
All used GEO expression data and their design:

\begin{itemize}
\item
  \textbf{GSE184609}: scRNA-Seq analysis of FACS-sorted live synovial
  cells isolated from naïve mice (two replicates) or from mice at day 6,
  14, or 25 of GPI-induced arthritis (one replicate per time point).
\item
  \textbf{GSE17755}: Peripheral blood was obtained from patients with RA
  (n=112), SLE (n=22), polyJIA (n=6), sJIA (n=51), HC (n=8), and HI
  (n=45). Blood samples from 8 HC and 45 HI are used as control.
\end{itemize}
\end{frame}

\begin{frame}[fragile]{方法}
\protect\hypertarget{ux65b9ux6cd5}{}
Mainly used method:

\begin{itemize}
\tightlist
\item
  The \texttt{biomart} was used for mapping genes between organism
  (e.g., mgi\_symbol to hgnc\_symbol){[}@MappingIdentifDurinc2009{]}.
\item
  The Human Gene Database \texttt{GeneCards} used for disease related
  genes prediction{[}@TheGenecardsSStelze2016{]}.
\item
  GEO \url{https://www.ncbi.nlm.nih.gov/geo/} used for expression
  dataset aquisition.
\item
  Databses of \texttt{DisGeNet}, \texttt{GeneCards}, \texttt{PharmGKB}
  used for collating disease related targets{[}@TheDisgenetKnPinero2019;
  @TheGenecardsSStelze2016; @PharmgkbAWorBarbar2018{]}.
\item
  R package \texttt{Limma} and \texttt{edgeR} used for differential
  expression analysis{[}@LimmaPowersDiRitchi2015; @EdgerDifferenChen{]}.
\item
  The data in published article of Jablonski et al used for
  distinguishing macrophage phenotypes
  (M0/M1/M2){[}@NovelMarkersTJablon2015{]}.
\item
  The R package \texttt{Seurat} used for scRNA-seq
  processing{[}@IntegratedAnalHaoY2021; @ComprehensiveIStuart2019{]}.
\item
  \texttt{SCSA} (python) used for cell type
  annotation{[}@ScsaACellTyCaoY2020{]}.
\item
  R version 4.4.0 (2024-04-24); Other R packages (eg., \texttt{dplyr}
  and \texttt{ggplot2}) used for statistic analysis or data
  visualization.
\end{itemize}
\end{frame}

\hypertarget{results}{%
\section{分析结果}\label{results}}

\hypertarget{dis}{%
\section{结论}\label{dis}}

\hypertarget{workflow}{%
\section{附:分析流程}\label{workflow}}

\begin{frame}[fragile]{scRNA-seq}
\protect\hypertarget{scrna-seq}{}
\begin{block}{数据来源}
\protect\hypertarget{ux6570ux636eux6765ux6e90}{}
这是一批小鼠的 单细胞测序数据。

\begin{center}\begin{tcolorbox}[colback=gray!10, colframe=gray!50, width=0.9\linewidth, arc=1mm, boxrule=0.5pt]
\textbf{
Data Source ID
:}

\vspace{0.5em}

    GSE184609

\vspace{2em}


\textbf{
data\_processing
:}

\vspace{0.5em}

    10X Genomics Cell Ranger v3.1

\vspace{2em}


\textbf{
data\_processing.1
:}

\vspace{0.5em}

    Gene-cell UMI matrix was generated for downstream
analyses. Low-quality cells were removed based on their
unique feature counts and mitochondrial gene content. Data
was normalized and log transformed using the default
setting of Seurat (version 3.1.4).

\vspace{2em}


\textbf{
data\_processing.2
:}

\vspace{0.5em}

    Genome\_build: mm10

\vspace{2em}


\textbf{
data\_processing.3
:}

\vspace{0.5em}

    Supplementary\_files\_format\_and\_content: For each
sample, there is one mtx file with filtered gene expressing
UMI counts for each sample, one tsv file containing gene
names, and one tsv file with cell barcodes.

\vspace{2em}
\end{tcolorbox}
\end{center}

\textbf{(上述信息框内容已保存至
\texttt{Figure+Table/GSE184609-content})}
\end{block}

\begin{block}{细胞聚类与初步注释}
\protect\hypertarget{ux7ec6ux80deux805aux7c7bux4e0eux521dux6b65ux6ce8ux91ca}{}
使用 SCSA 对细胞类型注释。

\begin{center}\vspace{1.5cm}\pgfornament[anchor=center,ydelta=0pt,width=9cm]{88}\end{center}

Figure @ref(fig:SCSA-Cell-type-annotation) (下方图) 为图SCSA Cell type
annotation概览。

\textbf{(对应文件为
\texttt{Figure+Table/SCSA-Cell-type-annotation.pdf})}

\def\@captype{figure}
\begin{center}
\includegraphics[width = 0.9\linewidth]{Figure+Table/SCSA-Cell-type-annotation.pdf}
\caption{SCSA Cell type annotation}\label{fig:SCSA-Cell-type-annotation}
\end{center}

\begin{center}\pgfornament[anchor=center,ydelta=0pt,width=9cm]{88}\vspace{1.5cm}\end{center}
\end{block}

\begin{block}{巨噬细胞重聚类}
\protect\hypertarget{ux5de8ux566cux7ec6ux80deux91cdux805aux7c7b}{}
\begin{center}\vspace{1.5cm}\pgfornament[anchor=center,ydelta=0pt,width=9cm]{88}\end{center}

Figure @ref(fig:Microphage-UMAP-Clustering) (下方图) 为图Microphage UMAP
Clustering概览。

\textbf{(对应文件为
\texttt{Figure+Table/Microphage-UMAP-Clustering.pdf})}

\def\@captype{figure}
\begin{center}
\includegraphics[width = 0.9\linewidth]{Figure+Table/Microphage-UMAP-Clustering.pdf}
\caption{Microphage UMAP Clustering}\label{fig:Microphage-UMAP-Clustering}
\end{center}

\begin{center}\pgfornament[anchor=center,ydelta=0pt,width=9cm]{88}\vspace{1.5cm}\end{center}
\end{block}

\begin{block}{巨噬细胞表型 M0、M1、M2 鉴定 Markers}
\protect\hypertarget{ux5de8ux566cux7ec6ux80deux8868ux578b-m0m1m2-ux9274ux5b9a-markers}{}
\begin{center}\vspace{1.5cm}\pgfornament[anchor=center,ydelta=0pt,width=9cm]{89}\end{center}

Table @ref(tab:The-markers-for-Macrophage-phenotypes-annotation)
(下方表格) 为表格The markers for Macrophage phenotypes annotation概览。

\textbf{(对应文件为
\texttt{Figure+Table/The-markers-for-Macrophage-phenotypes-annotation.csv})}

\begin{center}\begin{tcolorbox}[colback=gray!10, colframe=gray!50, width=0.9\linewidth, arc=1mm, boxrule=0.5pt]注:表格共有26行2列,以下预览的表格可能省略部分数据;含有3个唯一`cell'。
\end{tcolorbox}
\end{center}

\begin{longtable}[]{@{}ll@{}}
\caption{The markers for Macrophage phenotypes
annotation}\tabularnewline
\toprule
cell & markers\tabularnewline
\midrule
\endfirsthead
\toprule
cell & markers\tabularnewline
\midrule
\endhead
Macrophage\_M0 & Sh2d3c\tabularnewline
Macrophage\_M0 & Slc13a3\tabularnewline
Macrophage\_M0 & Rcan1\tabularnewline
Macrophage\_M0 & Trp53inp1\tabularnewline
Macrophage\_M0 & Slc40a1\tabularnewline
Macrophage\_M0 & Il16\tabularnewline
Macrophage\_M1 & Cfb\tabularnewline
Macrophage\_M1 & Slfn4\tabularnewline
Macrophage\_M1 & H2-Q6\tabularnewline
Macrophage\_M1 & Fpr1\tabularnewline
Macrophage\_M1 & Slfn1\tabularnewline
Macrophage\_M1 & Ccrl2\tabularnewline
Macrophage\_M1 & Fpr2\tabularnewline
Macrophage\_M1 & Cxcl10\tabularnewline
Macrophage\_M1 & Oasl1\tabularnewline
\ldots{} & \ldots{}\tabularnewline
\bottomrule
\end{longtable}

\begin{center}\pgfornament[anchor=center,ydelta=0pt,width=9cm]{89}\vspace{1.5cm}\end{center}

\begin{center}\vspace{1.5cm}\pgfornament[anchor=center,ydelta=0pt,width=9cm]{88}\end{center}

Figure @ref(fig:Heatmap-show-the-reference-genes) (下方图) 为图Heatmap
show the reference genes概览。

\textbf{(对应文件为
\texttt{Figure+Table/Heatmap-show-the-reference-genes.pdf})}

\def\@captype{figure}
\begin{center}
\includegraphics[width = 0.9\linewidth]{Figure+Table/Heatmap-show-the-reference-genes.pdf}
\caption{Heatmap show the reference genes}\label{fig:Heatmap-show-the-reference-genes}
\end{center}

\begin{center}\pgfornament[anchor=center,ydelta=0pt,width=9cm]{88}\vspace{1.5cm}\end{center}

\begin{center}\vspace{1.5cm}\pgfornament[anchor=center,ydelta=0pt,width=9cm]{88}\end{center}

Figure @ref(fig:Macrophage-phenotypes-type-annotation) (下方图)
为图Macrophage phenotypes type annotation概览。

\textbf{(对应文件为
\texttt{Figure+Table/Macrophage-phenotypes-type-annotation.pdf})}

\def\@captype{figure}
\begin{center}
\includegraphics[width = 0.9\linewidth]{Figure+Table/Macrophage-phenotypes-type-annotation.pdf}
\caption{Macrophage phenotypes type annotation}\label{fig:Macrophage-phenotypes-type-annotation}
\end{center}

\begin{center}\pgfornament[anchor=center,ydelta=0pt,width=9cm]{88}\vspace{1.5cm}\end{center}
\end{block}

\begin{block}{RA 与 Control 的巨噬细胞表型}
\protect\hypertarget{ra-ux4e0e-control-ux7684ux5de8ux566cux7ec6ux80deux8868ux578b}{}
随后,根据数据集的来源 (RA 或 Control,将巨噬细胞分类)

\begin{center}\vspace{1.5cm}\pgfornament[anchor=center,ydelta=0pt,width=9cm]{88}\end{center}

Figure @ref(fig:The-Phenotypes) (下方图) 为图The Phenotypes概览。

\textbf{(对应文件为 \texttt{Figure+Table/The-Phenotypes.pdf})}

\def\@captype{figure}
\begin{center}
\includegraphics[width = 0.9\linewidth]{Figure+Table/The-Phenotypes.pdf}
\caption{The Phenotypes}\label{fig:The-Phenotypes}
\end{center}

\begin{center}\pgfornament[anchor=center,ydelta=0pt,width=9cm]{88}\vspace{1.5cm}\end{center}
\end{block}

\begin{block}{差异分析}
\protect\hypertarget{ux5deeux5f02ux5206ux6790}{}
\begin{itemize}
\tightlist
\item
  XX 在RA中M1巨噬细胞上调: \texttt{GPI-day25-RA\_Macrophage\_M1} vs
  \texttt{Control\_Macrophage\_M1}
\item
  其中XX基因主要分布在M1巨噬细胞而非M2巨噬细胞上:
  \texttt{GPI-day25-RA\_Macrophage\_M1} vs
  \texttt{GPI-day25-RA\_Macrophage\_M2}
\end{itemize}

\begin{center}\vspace{1.5cm}\pgfornament[anchor=center,ydelta=0pt,width=9cm]{89}\end{center}

Table @ref(tab:DEGs-of-the-contrasts) (下方表格) 为表格DEGs of the
contrasts概览。

\textbf{(对应文件为 \texttt{Figure+Table/DEGs-of-the-contrasts.csv})}

\begin{center}\begin{tcolorbox}[colback=gray!10, colframe=gray!50, width=0.9\linewidth, arc=1mm, boxrule=0.5pt]注:表格共有355行7列,以下预览的表格可能省略部分数据;含有2个唯一`contrast;含有335个唯一`gene'。
\end{tcolorbox}
\end{center}

\begin{longtable}[]{@{}lllllll@{}}
\caption{DEGs of the contrasts}\tabularnewline
\toprule
contrast & p\_val & avg\_log2FC & pct.1 & pct.2 & p\_val\_adj &
gene\tabularnewline
\midrule
\endfirsthead
\toprule
contrast & p\_val & avg\_log2FC & pct.1 & pct.2 & p\_val\_adj &
gene\tabularnewline
\midrule
\endhead
GPI-day25-\ldots{} & 1.75831850\ldots{} & 2.17770295\ldots{} & 0.044 &
0.385 & 5.27495551\ldots{} & Adora3\tabularnewline
GPI-day25-\ldots{} & 2.23900708\ldots{} & 9.52901476\ldots{} & 0.069 &
0.427 & 6.71702126\ldots{} & F7\tabularnewline
GPI-day25-\ldots{} & 6.04375313\ldots{} & 9.97788827\ldots{} & 0.093 &
0.536 & 1.81312594\ldots{} & Hal\tabularnewline
GPI-day25-\ldots{} & 1.83819770\ldots{} & 13.5930067\ldots{} & 0.052 &
0.641 & 5.51459312\ldots{} & Cxcl13\tabularnewline
GPI-day25-\ldots{} & 1.00690808\ldots{} & 4.79632080\ldots{} & 0.153 &
0.583 & 3.02072424\ldots{} & Ifi44\tabularnewline
GPI-day25-\ldots{} & 2.16444867\ldots{} & 8.41136649\ldots{} & 0.153 &
0.87 & 6.49334601\ldots{} & Slc13a3\tabularnewline
GPI-day25-\ldots{} & 8.87142099\ldots{} & 7.46793928\ldots{} & 0.081 &
0.391 & 2.66142629\ldots{} & Cd4\tabularnewline
GPI-day25-\ldots{} & 1.52778766\ldots{} & 0.95745400\ldots{} & 0.141 &
0.307 & 4.58336298\ldots{} & Tnfsf14\tabularnewline
GPI-day25-\ldots{} & 4.80404528\ldots{} & 3.93765968\ldots{} & 0.169 &
0.484 & 1.44121358\ldots{} & Cd79b\tabularnewline
GPI-day25-\ldots{} & 8.42060311\ldots{} & 3.12179649\ldots{} & 0.06 &
0.651 & 2.52618093\ldots{} & Cd209e\tabularnewline
GPI-day25-\ldots{} & 8.42067724\ldots{} & 1.96704094\ldots{} & 0.145 &
0.786 & 2.52620317\ldots{} & Adgre4\tabularnewline
GPI-day25-\ldots{} & 8.24915617\ldots{} & 8.83724798\ldots{} & 0.161 &
0.766 & 2.47474685\ldots{} & Pparg\tabularnewline
GPI-day25-\ldots{} & 1.31157015\ldots{} & 8.50505864\ldots{} & 0.153 &
0.292 & 3.93471045\ldots{} & F10\tabularnewline
GPI-day25-\ldots{} & 2.28779328\ldots{} & 2.51813054\ldots{} & 0.024 &
0.411 & 6.86337986\ldots{} & Apoc4\tabularnewline
GPI-day25-\ldots{} & 2.75634208\ldots{} & 4.11075823\ldots{} & 0.073 &
0.755 & 8.26902626\ldots{} & Il10\tabularnewline
\ldots{} & \ldots{} & \ldots{} & \ldots{} & \ldots{} & \ldots{} &
\ldots{}\tabularnewline
\bottomrule
\end{longtable}

\begin{center}\pgfornament[anchor=center,ydelta=0pt,width=9cm]{89}\vspace{1.5cm}\end{center}

\begin{center}\vspace{1.5cm}\pgfornament[anchor=center,ydelta=0pt,width=9cm]{88}\end{center}

Figure @ref(fig:Intersection-of-RA-M1-up-with-M1-not-M2) (下方图)
为图Intersection of RA M1 up with M1 not M2概览。

\textbf{(对应文件为
\texttt{Figure+Table/Intersection-of-RA-M1-up-with-M1-not-M2.pdf})}

\def\@captype{figure}
\begin{center}
\includegraphics[width = 0.9\linewidth]{Figure+Table/Intersection-of-RA-M1-up-with-M1-not-M2.pdf}
\caption{Intersection of RA M1 up with M1 not M2}\label{fig:Intersection-of-RA-M1-up-with-M1-not-M2}
\end{center}

\begin{center}\pgfornament[anchor=center,ydelta=0pt,width=9cm]{88}\vspace{1.5cm}\end{center}\begin{center}\begin{tcolorbox}[colback=gray!10, colframe=gray!50, width=0.9\linewidth, arc=1mm, boxrule=0.5pt]
\textbf{
Intersection
:}

\vspace{0.5em}

    Ifi44, Adgre4, Pparg, Dppa3, Cadm1, P2ry14, Gm1673,
Vwf, Ednrb, Fam43a, Bambi, Slc28a2, Plk2, Rcn3, Rrm1,
Ifi204, Bmp2, Gfra2, Spon1, Gstm1

\vspace{2em}
\end{tcolorbox}
\end{center}

\textbf{(上述信息框内容已保存至
\texttt{Figure+Table/Intersection-of-RA-M1-up-with-M1-not-M2-content})}
\end{block}
\end{frame}

\begin{frame}[fragile]{小鼠基因映射到人类基因}
\protect\hypertarget{ux5c0fux9f20ux57faux56e0ux6620ux5c04ux5230ux4ebaux7c7bux57faux56e0}{}
\begin{center}\vspace{1.5cm}\pgfornament[anchor=center,ydelta=0pt,width=9cm]{89}\end{center}

Table @ref(tab:Mapped-genes) (下方表格) 为表格Mapped genes概览。

\textbf{(对应文件为 \texttt{Figure+Table/Mapped-genes.csv})}

\begin{center}\begin{tcolorbox}[colback=gray!10, colframe=gray!50, width=0.9\linewidth, arc=1mm, boxrule=0.5pt]注:表格共有19行2列,以下预览的表格可能省略部分数据;含有19个唯一`mgi\_symbol;含有19个唯一`hgnc\_symbol'。
\end{tcolorbox}
\end{center}
\begin{center}\begin{tcolorbox}[colback=gray!10, colframe=gray!50, width=0.9\linewidth, arc=1mm, boxrule=0.5pt]\begin{enumerate}\tightlist
\item hgnc\_symbol:  基因名 (Human)
\item mgi\_symbol:  基因名 (Mice)
\end{enumerate}\end{tcolorbox}
\end{center}

\begin{longtable}[]{@{}ll@{}}
\caption{Mapped genes}\tabularnewline
\toprule
mgi\_symbol & hgnc\_symbol\tabularnewline
\midrule
\endfirsthead
\toprule
mgi\_symbol & hgnc\_symbol\tabularnewline
\midrule
\endhead
Bmp2 & BMP2\tabularnewline
Ednrb & EDNRB\tabularnewline
Dppa3 & DPPA3\tabularnewline
Spon1 & SPON1\tabularnewline
Gfra2 & GFRA2\tabularnewline
Bambi & BAMBI\tabularnewline
Cadm1 & CADM1\tabularnewline
Slc28a2 & SLC28A2\tabularnewline
Rrm1 & RRM1\tabularnewline
Ifi44 & IFI44\tabularnewline
Gm1673 & C4orf48\tabularnewline
Ifi204 & MNDA\tabularnewline
P2ry14 & P2RY14\tabularnewline
Rcn3 & RCN3\tabularnewline
Gstm1 & GSTM1\tabularnewline
\ldots{} & \ldots{}\tabularnewline
\bottomrule
\end{longtable}

\begin{center}\pgfornament[anchor=center,ydelta=0pt,width=9cm]{89}\vspace{1.5cm}\end{center}
\end{frame}

\begin{frame}[fragile]{糖酵解相关基因}
\protect\hypertarget{ux7cd6ux9175ux89e3ux76f8ux5173ux57faux56e0}{}
\begin{center}\vspace{1.5cm}\pgfornament[anchor=center,ydelta=0pt,width=9cm]{89}\end{center}

Table @ref(tab:Glycolysis-related-genes-from-GeneCards) (下方表格)
为表格Glycolysis related genes from GeneCards概览。

\textbf{(对应文件为
\texttt{Figure+Table/Glycolysis-related-genes-from-GeneCards.xlsx})}

\begin{center}\begin{tcolorbox}[colback=gray!10, colframe=gray!50, width=0.9\linewidth, arc=1mm, boxrule=0.5pt]注:表格共有118行7列,以下预览的表格可能省略部分数据;含有118个唯一`Symbol'。
\end{tcolorbox}
\end{center}\begin{center}\begin{tcolorbox}[colback=gray!10, colframe=gray!50, width=0.9\linewidth, arc=1mm, boxrule=0.5pt]
\textbf{
The GeneCards data was obtained by querying
:}

\vspace{0.5em}

    Glycolysis

\vspace{2em}


\textbf{
Restrict (with quotes)
:}

\vspace{0.5em}

    FALSE

\vspace{2em}


\textbf{
Filtering by Score:
:}

\vspace{0.5em}

    Score > 3

\vspace{2em}
\end{tcolorbox}
\end{center}

\begin{longtable}[]{@{}lllllll@{}}
\caption{Glycolysis related genes from GeneCards}\tabularnewline
\toprule
Symbol & Description & Category & UniProt\_ID & GIFtS & GC\_id &
Score\tabularnewline
\midrule
\endfirsthead
\toprule
Symbol & Description & Category & UniProt\_ID & GIFtS & GC\_id &
Score\tabularnewline
\midrule
\endhead
TIGAR & TP53 Induc\ldots{} & Protein Co\ldots{} & Q9NQ88 & 45 &
GC12P038924 & 22.4\tabularnewline
PKM & Pyruvate K\ldots{} & Protein Co\ldots{} & P14618 & 58 &
GC15M072199 & 20.77\tabularnewline
HK2 & Hexokinase 2 & Protein Co\ldots{} & P52789 & 55 & GC02P074947 &
19.42\tabularnewline
GAPDH & Glyceralde\ldots{} & Protein Co\ldots{} & P04406 & 59 &
GC12P038965 & 17.14\tabularnewline
LDHA & Lactate De\ldots{} & Protein Co\ldots{} & P00338 & 59 &
GC11P018394 & 15.81\tabularnewline
HIF1A & Hypoxia In\ldots{} & Protein Co\ldots{} & Q16665 & 57 &
GC14P061695 & 15.1\tabularnewline
RRAD & RRAD, Ras \ldots{} & Protein Co\ldots{} & P55042 & 46 &
GC16M067483 & 15.1\tabularnewline
HK1 & Hexokinase 1 & Protein Co\ldots{} & P19367 & 59 & GC10P069269 &
14.64\tabularnewline
PKLR & Pyruvate K\ldots{} & Protein Co\ldots{} & P30613 & 55 &
GC01M155289 & 13.37\tabularnewline
ENO1 & Enolase 1 & Protein Co\ldots{} & P06733 & 56 & GC01M008861 &
13.36\tabularnewline
ENO3 & Enolase 3 & Protein Co\ldots{} & P13929 & 54 & GC17P004948 &
13.33\tabularnewline
PFKP & Phosphofru\ldots{} & Protein Co\ldots{} & Q01813 & 53 &
GC10P003066 & 13.19\tabularnewline
TPI1 & Triosephos\ldots{} & Protein Co\ldots{} & P60174 & 55 &
GC12P006867 & 13.18\tabularnewline
GLTC1 & Glycolysis\ldots{} & RNA Gene (\ldots{} & & 2 & GC11U909607 &
12.97\tabularnewline
PGK1 & Phosphogly\ldots{} & Protein Co\ldots{} & P00558 & 57 &
GC0XP078166 & 12.94\tabularnewline
\ldots{} & \ldots{} & \ldots{} & \ldots{} & \ldots{} & \ldots{} &
\ldots{}\tabularnewline
\bottomrule
\end{longtable}

\begin{center}\pgfornament[anchor=center,ydelta=0pt,width=9cm]{89}\vspace{1.5cm}\end{center}

\begin{center}\vspace{1.5cm}\pgfornament[anchor=center,ydelta=0pt,width=9cm]{88}\end{center}

Figure @ref(fig:Intersection-of-RA-M1M2-related-with-Glycolysis-related)
(下方图) 为图Intersection of RA M1M2 related with Glycolysis
related概览。

\textbf{(对应文件为
\texttt{Figure+Table/Intersection-of-RA-M1M2-related-with-Glycolysis-related.pdf})}

\def\@captype{figure}
\begin{center}
\includegraphics[width = 0.9\linewidth]{Figure+Table/Intersection-of-RA-M1M2-related-with-Glycolysis-related.pdf}
\caption{Intersection of RA M1M2 related with Glycolysis related}\label{fig:Intersection-of-RA-M1M2-related-with-Glycolysis-related}
\end{center}

\begin{center}\pgfornament[anchor=center,ydelta=0pt,width=9cm]{88}\vspace{1.5cm}\end{center}\begin{center}\begin{tcolorbox}[colback=gray!10, colframe=gray!50, width=0.9\linewidth, arc=1mm, boxrule=0.5pt]
\textbf{
Intersection
:}

\vspace{0.5em}

    PPARG

\vspace{2em}
\end{tcolorbox}
\end{center}

\textbf{(上述信息框内容已保存至
\texttt{Figure+Table/Intersection-of-RA-M1M2-related-with-Glycolysis-related-content})}
\end{frame}

\begin{frame}[fragile]{交集基因的表达 (小鼠单细胞数据)}
\protect\hypertarget{ux4ea4ux96c6ux57faux56e0ux7684ux8868ux8fbe-ux5c0fux9f20ux5355ux7ec6ux80deux6570ux636e}{}
\begin{center}\vspace{1.5cm}\pgfornament[anchor=center,ydelta=0pt,width=9cm]{88}\end{center}

Figure @ref(fig:Violing-plot-of-expression-level-of-the-Pparg) (下方图)
为图Violing plot of expression level of the Pparg概览。

\textbf{(对应文件为
\texttt{Figure+Table/Violing-plot-of-expression-level-of-the-Pparg.pdf})}

\def\@captype{figure}
\begin{center}
\includegraphics[width = 0.9\linewidth]{Figure+Table/Violing-plot-of-expression-level-of-the-Pparg.pdf}
\caption{Violing plot of expression level of the Pparg}\label{fig:Violing-plot-of-expression-level-of-the-Pparg}
\end{center}

\begin{center}\pgfornament[anchor=center,ydelta=0pt,width=9cm]{88}\vspace{1.5cm}\end{center}
\end{frame}

\hypertarget{ux8fdbux4e00ux6b65ux5206ux6790}{%
\section{进一步分析}\label{ux8fdbux4e00ux6b65ux5206ux6790}}

\begin{frame}[fragile]{数据来源}
\protect\hypertarget{ux6570ux636eux6765ux6e90-1}{}
\begin{center}\begin{tcolorbox}[colback=gray!10, colframe=gray!50, width=0.9\linewidth, arc=1mm, boxrule=0.5pt]
\textbf{
Data Source ID
:}

\vspace{0.5em}

    GSE17755

\vspace{2em}


\textbf{
data\_processing
:}

\vspace{0.5em}

    Log2 ratios of Cy3 to Cy5 were calculated and
normalized by the method of global ratio median
normalization.

\vspace{2em}
\end{tcolorbox}
\end{center}

\textbf{(上述信息框内容已保存至
\texttt{Figure+Table/HUMAN-GSE17755-content})}
\end{frame}

\begin{frame}[fragile]{炎症因子、巨噬细胞浸润、巨噬细胞极化相关因子、糖酵解相关因子}
\protect\hypertarget{ux708eux75c7ux56e0ux5b50ux5de8ux566cux7ec6ux80deux6d78ux6da6ux5de8ux566cux7ec6ux80deux6781ux5316ux76f8ux5173ux56e0ux5b50ux7cd6ux9175ux89e3ux76f8ux5173ux56e0ux5b50}{}
使用 genecards 获取相关基因 (各取前 50 基因):

\begin{itemize}
\tightlist
\item
  IF: Inflammatory factors 炎症因子
\item
  MI: Macrophage infiltration 巨噬细胞浸润
\item
  MP: Macrophage polarization 巨噬细胞极化
\item
  G: Glycolysis 糖酵解
\end{itemize}

\begin{center}\vspace{1.5cm}\pgfornament[anchor=center,ydelta=0pt,width=9cm]{89}\end{center}

Table @ref(tab:All-Factors) (下方表格) 为表格All Factors概览。

\textbf{(对应文件为 \texttt{Figure+Table/All-Factors.csv})}

\begin{center}\begin{tcolorbox}[colback=gray!10, colframe=gray!50, width=0.9\linewidth, arc=1mm, boxrule=0.5pt]注:表格共有200行2列,以下预览的表格可能省略部分数据;含有4个唯一`type'。
\end{tcolorbox}
\end{center}

\begin{longtable}[]{@{}ll@{}}
\caption{All Factors}\tabularnewline
\toprule
type & name\tabularnewline
\midrule
\endfirsthead
\toprule
type & name\tabularnewline
\midrule
\endhead
Inflammatory factors & IL6\tabularnewline
Inflammatory factors & TNF\tabularnewline
Inflammatory factors & CRP\tabularnewline
Inflammatory factors & BDNF-AS\tabularnewline
Inflammatory factors & IL1B\tabularnewline
Inflammatory factors & LINC02605\tabularnewline
Inflammatory factors & TLR4\tabularnewline
Inflammatory factors & MIR146B\tabularnewline
Inflammatory factors & ADIPOQ\tabularnewline
Inflammatory factors & LINC01672\tabularnewline
Inflammatory factors & CXCL8\tabularnewline
Inflammatory factors & IL1A\tabularnewline
Inflammatory factors & NFKB1\tabularnewline
Inflammatory factors & CERNA3\tabularnewline
Inflammatory factors & IL18\tabularnewline
\ldots{} & \ldots{}\tabularnewline
\bottomrule
\end{longtable}

\begin{center}\pgfornament[anchor=center,ydelta=0pt,width=9cm]{89}\vspace{1.5cm}\end{center}

对上述基因集去重复后,关联分析。
\end{frame}

\begin{frame}[fragile]{关联分析}
\protect\hypertarget{ux5173ux8054ux5206ux6790}{}
\begin{center}\vspace{1.5cm}\pgfornament[anchor=center,ydelta=0pt,width=9cm]{88}\end{center}

Figure @ref(fig:HUMAN-correlation-heatmap) (下方图) 为图HUMAN
correlation heatmap概览。

\textbf{(对应文件为
\texttt{Figure+Table/HUMAN-correlation-heatmap.pdf})}

\def\@captype{figure}
\begin{center}
\includegraphics[width = 0.9\linewidth]{Figure+Table/HUMAN-correlation-heatmap.pdf}
\caption{HUMAN correlation heatmap}\label{fig:HUMAN-correlation-heatmap}
\end{center}

\begin{center}\pgfornament[anchor=center,ydelta=0pt,width=9cm]{88}\vspace{1.5cm}\end{center}

\begin{center}\vspace{1.5cm}\pgfornament[anchor=center,ydelta=0pt,width=9cm]{88}\end{center}

Figure @ref(fig:HUMAN-correlation-heatmap-VWF-significant) (下方图)
为图HUMAN correlation heatmap VWF significant概览。

\textbf{(对应文件为
\texttt{Figure+Table/HUMAN-correlation-heatmap-VWF-significant.pdf})}

\def\@captype{figure}
\begin{center}
\includegraphics[width = 0.9\linewidth]{Figure+Table/HUMAN-correlation-heatmap-VWF-significant.pdf}
\caption{HUMAN correlation heatmap VWF significant}\label{fig:HUMAN-correlation-heatmap-VWF-significant}
\end{center}

\begin{center}\pgfornament[anchor=center,ydelta=0pt,width=9cm]{88}\vspace{1.5cm}\end{center}

\begin{center}\pgfornament[anchor=center,ydelta=0pt,width=9cm]{85}\vspace{1.5cm}\end{center}

`HUMAN regression VWF significant' 数据已全部提供。

\textbf{(对应文件为
\texttt{Figure+Table/HUMAN-regression-VWF-significant})}

\begin{center}\begin{tcolorbox}[colback=gray!10, colframe=gray!50, width=0.9\linewidth, arc=1mm, boxrule=0.5pt]注:文件夹Figure+Table/HUMAN-regression-VWF-significant共包含4个文件。

\begin{enumerate}\tightlist
\item 1\_Glycolysis.pdf
\item 2\_Inflammatory factors.pdf
\item 3\_Macrophage infiltration.pdf
\item 4\_Macrophage polarization.pdf
\end{enumerate}\end{tcolorbox}
\end{center}

\begin{center}\pgfornament[anchor=center,ydelta=0pt,width=9cm]{85}\vspace{1.5cm}\end{center}

\begin{center}\vspace{1.5cm}\pgfornament[anchor=center,ydelta=0pt,width=9cm]{89}\end{center}

Table @ref(tab:HUMAN-correlation) (下方表格) 为表格HUMAN
correlation概览。

\textbf{(对应文件为 \texttt{Figure+Table/HUMAN-correlation.csv})}

\begin{center}\begin{tcolorbox}[colback=gray!10, colframe=gray!50, width=0.9\linewidth, arc=1mm, boxrule=0.5pt]注:表格共有2023行9列,以下预览的表格可能省略部分数据;含有17个唯一`From'。
\end{tcolorbox}
\end{center}
\begin{center}\begin{tcolorbox}[colback=gray!10, colframe=gray!50, width=0.9\linewidth, arc=1mm, boxrule=0.5pt]\begin{enumerate}\tightlist
\item cor:  皮尔逊关联系数,正关联或负关联。
\item pvalue:  显著性 P。
\item -log2(P.value):  P 的对数转化。
\item significant:  显著性。
\item sign:  人为赋予的符号,参考 significant。
\end{enumerate}\end{tcolorbox}
\end{center}

\begin{longtable}[]{@{}lllllllll@{}}
\caption{HUMAN correlation}\tabularnewline
\toprule
From & To & cor & pvalue & -log2(\ldots{} & signif\ldots{} & sign &
Factors & Type\tabularnewline
\midrule
\endfirsthead
\toprule
From & To & cor & pvalue & -log2(\ldots{} & signif\ldots{} & sign &
Factors & Type\tabularnewline
\midrule
\endhead
EDNRB & GAPDH & -0.14 & 0.0931 & 3.4250\ldots{} & \textgreater{} 0.05 &
- & Glycol\ldots{} & Others\tabularnewline
PPARG & GAPDH & -0.06 & 0.4812 & 1.0552\ldots{} & \textgreater{} 0.05 &
- & Glycol\ldots{} & Others\tabularnewline
CADM1 & GAPDH & -0.26 & 9e-04 & 10.117\ldots{} & \textless{} 0.001 & **
& Glycol\ldots{} & Others\tabularnewline
BMP2 & GAPDH & -0.1 & 0.225 & 2.1520\ldots{} & \textgreater{} 0.05 & - &
Glycol\ldots{} & Others\tabularnewline
SLC28A2 & GAPDH & 0.06 & 0.4919 & 1.0235\ldots{} & \textgreater{} 0.05 &
- & Glycol\ldots{} & Others\tabularnewline
RRM1 & GAPDH & 0.33 & 0 & 16.609\ldots{} & \textless{} 0.001 & ** &
Glycol\ldots{} & Others\tabularnewline
BAMBI & GAPDH & -0.16 & 0.0796 & 3.6510\ldots{} & \textgreater{} 0.05 &
- & Glycol\ldots{} & Others\tabularnewline
PLK2 & GAPDH & -0.08 & 0.3648 & 1.4548\ldots{} & \textgreater{} 0.05 & -
& Glycol\ldots{} & Others\tabularnewline
P2RY14 & GAPDH & 0.43 & 0 & 16.609\ldots{} & \textless{} 0.001 & ** &
Glycol\ldots{} & Others\tabularnewline
MNDA & GAPDH & -0.02 & 0.7616 & 0.3928\ldots{} & \textgreater{} 0.05 & -
& Glycol\ldots{} & Others\tabularnewline
GSTM1 & GAPDH & -0.16 & 0.073 & 3.7759\ldots{} & \textgreater{} 0.05 & -
& Glycol\ldots{} & Others\tabularnewline
IFI44 & GAPDH & -0.33 & 0 & 16.609\ldots{} & \textless{} 0.001 & ** &
Glycol\ldots{} & Others\tabularnewline
RCN3 & GAPDH & 0.12 & 0.148 & 2.7563\ldots{} & \textgreater{} 0.05 & - &
Glycol\ldots{} & Others\tabularnewline
SPON1 & GAPDH & -0.14 & 0.0882 & 3.5030\ldots{} & \textgreater{} 0.05 &
- & Glycol\ldots{} & Others\tabularnewline
GFRA2 & GAPDH & -0.07 & 0.3662 & 1.4492\ldots{} & \textgreater{} 0.05 &
- & Glycol\ldots{} & Others\tabularnewline
\ldots{} & \ldots{} & \ldots{} & \ldots{} & \ldots{} & \ldots{} &
\ldots{} & \ldots{} & \ldots{}\tabularnewline
\bottomrule
\end{longtable}

\begin{center}\pgfornament[anchor=center,ydelta=0pt,width=9cm]{89}\vspace{1.5cm}\end{center}
\end{frame}

\end{document}
