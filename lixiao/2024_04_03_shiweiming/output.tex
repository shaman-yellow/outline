% Options for packages loaded elsewhere
\PassOptionsToPackage{unicode}{hyperref}
\PassOptionsToPackage{hyphens}{url}
%
\documentclass[
]{article}
\usepackage{lmodern}
\usepackage{amssymb,amsmath}
\usepackage{ifxetex,ifluatex}
\ifnum 0\ifxetex 1\fi\ifluatex 1\fi=0 % if pdftex
  \usepackage[T1]{fontenc}
  \usepackage[utf8]{inputenc}
  \usepackage{textcomp} % provide euro and other symbols
\else % if luatex or xetex
  \usepackage{unicode-math}
  \defaultfontfeatures{Scale=MatchLowercase}
  \defaultfontfeatures[\rmfamily]{Ligatures=TeX,Scale=1}
\fi
% Use upquote if available, for straight quotes in verbatim environments
\IfFileExists{upquote.sty}{\usepackage{upquote}}{}
\IfFileExists{microtype.sty}{% use microtype if available
  \usepackage[]{microtype}
  \UseMicrotypeSet[protrusion]{basicmath} % disable protrusion for tt fonts
}{}
\makeatletter
\@ifundefined{KOMAClassName}{% if non-KOMA class
  \IfFileExists{parskip.sty}{%
    \usepackage{parskip}
  }{% else
    \setlength{\parindent}{0pt}
    \setlength{\parskip}{6pt plus 2pt minus 1pt}}
}{% if KOMA class
  \KOMAoptions{parskip=half}}
\makeatother
\usepackage{xcolor}
\IfFileExists{xurl.sty}{\usepackage{xurl}}{} % add URL line breaks if available
\IfFileExists{bookmark.sty}{\usepackage{bookmark}}{\usepackage{hyperref}}
\hypersetup{
  hidelinks,
  pdfcreator={LaTeX via pandoc}}
\urlstyle{same} % disable monospaced font for URLs
\usepackage[margin=1in]{geometry}
\usepackage{longtable,booktabs}
% Correct order of tables after \paragraph or \subparagraph
\usepackage{etoolbox}
\makeatletter
\patchcmd\longtable{\par}{\if@noskipsec\mbox{}\fi\par}{}{}
\makeatother
% Allow footnotes in longtable head/foot
\IfFileExists{footnotehyper.sty}{\usepackage{footnotehyper}}{\usepackage{footnote}}
\makesavenoteenv{longtable}
\usepackage{graphicx}
\makeatletter
\def\maxwidth{\ifdim\Gin@nat@width>\linewidth\linewidth\else\Gin@nat@width\fi}
\def\maxheight{\ifdim\Gin@nat@height>\textheight\textheight\else\Gin@nat@height\fi}
\makeatother
% Scale images if necessary, so that they will not overflow the page
% margins by default, and it is still possible to overwrite the defaults
% using explicit options in \includegraphics[width, height, ...]{}
\setkeys{Gin}{width=\maxwidth,height=\maxheight,keepaspectratio}
% Set default figure placement to htbp
\makeatletter
\def\fps@figure{htbp}
\makeatother
\setlength{\emergencystretch}{3em} % prevent overfull lines
\providecommand{\tightlist}{%
  \setlength{\itemsep}{0pt}\setlength{\parskip}{0pt}}
\setcounter{secnumdepth}{5}
\usepackage{caption} \captionsetup{font={footnotesize},width=6in} \renewcommand{\dblfloatpagefraction}{.9} \makeatletter \renewenvironment{figure} {\def\@captype{figure}} \makeatother \@ifundefined{Shaded}{\newenvironment{Shaded}} \@ifundefined{snugshade}{\newenvironment{snugshade}} \renewenvironment{Shaded}{\begin{snugshade}}{\end{snugshade}} \definecolor{shadecolor}{RGB}{230,230,230} \usepackage{xeCJK} \usepackage{setspace} \setstretch{1.3} \usepackage{tcolorbox} \setcounter{secnumdepth}{4} \setcounter{tocdepth}{4} \usepackage{wallpaper} \usepackage[absolute]{textpos} \tcbuselibrary{breakable} \renewenvironment{Shaded} {\begin{tcolorbox}[colback = gray!10, colframe = gray!40, width = 16cm, arc = 1mm, auto outer arc, title = {R input}]} {\end{tcolorbox}} \usepackage{titlesec} \titleformat{\paragraph} {\fontsize{10pt}{0pt}\bfseries} {\arabic{section}.\arabic{subsection}.\arabic{subsubsection}.\arabic{paragraph}} {1em} {} []
\newlength{\cslhangindent}
\setlength{\cslhangindent}{1.5em}
\newenvironment{cslreferences}%
  {}%
  {\par}

\author{}
\date{\vspace{-2.5em}}

\begin{document}

\begin{titlepage} \newgeometry{top=7.5cm}
\ThisCenterWallPaper{1.12}{~/outline/lixiao//cover_page.pdf}
\begin{center} \textbf{\Huge
XX基因通过促进糖酵解促进巨噬细胞M1极化}
\vspace{4em} \begin{textblock}{10}(3,5.9) \huge
\textbf{\textcolor{white}{2024-04-15}}
\end{textblock} \begin{textblock}{10}(3,7.3)
\Large \textcolor{black}{LiChuang Huang}
\end{textblock} \begin{textblock}{10}(3,11.3)
\Large \textcolor{black}{@立效研究院}
\end{textblock} \end{center} \end{titlepage}
\restoregeometry

\pagenumbering{roman}

\tableofcontents

\listoffigures

\listoftables

\newpage

\pagenumbering{arabic}

\hypertarget{abstract}{%
\section{摘要}\label{abstract}}

\hypertarget{ux751fux4fe1ux9700ux6c42}{%
\subsection{生信需求}\label{ux751fux4fe1ux9700ux6c42}}

疾病:类风湿性关节炎RA
物种:临床患者或者动物模型都可以
细胞:巨噬细胞

目标:筛出XX基因,XX基因满足,
1、是糖酵解相关基因
2、与巨噬细胞极化相关(M1/M2)

设想:XX基因在RA中上调,RA中M1巨噬细胞上调,其中XX基因主要分布在M1巨噬细胞而非M2巨噬细胞上,XX基因可能通过促进糖酵解促进巨噬细胞M1极化

M1标志:iNOS,CD11c,CD86等
M2标志:CD206,IL-10,TGF-beta等

\hypertarget{ux7ed3ux679c}{%
\subsection{结果}\label{ux7ed3ux679c}}

\begin{itemize}
\tightlist
\item
  首先通过分析 GEO 单细胞数据,鉴定出巨噬细胞不同表型 Fig. \ref{fig:The-Phenotypes}。
\item
  该数据集为小鼠来源,鉴定 M0、M1、M2 的小鼠基因 Marker 参考\textsuperscript{\protect\hyperlink{ref-NovelMarkersTJablon2015}{1}},
  实际使用的 Marker 见 Tab. \ref{tab:The-markers-for-Macrophage-phenotypes-annotation}
\item
  进行差异分析 (Tab. \ref{tab:DEGs-of-the-contrasts}) :

  \begin{itemize}
  \tightlist
  \item
    XX 在RA中M1巨噬细胞上调: \texttt{GPI-day25-RA\_Macrophage\_M1} vs \texttt{Control\_Macrophage\_M1}
  \item
    其中XX基因主要分布在M1巨噬细胞而非M2巨噬细胞上: \texttt{GPI-day25-RA\_Macrophage\_M1} vs \texttt{GPI-day25-RA\_Macrophage\_M2}
  \end{itemize}
\item
  以上两组差异基因交集见 Fig. \ref{fig:Intersection-of-RA-M1-up-with-M1-not-M2}
\item
  小鼠基因映射到人类 Tab. \ref{tab:Mapped-genes}
\item
  其中糖酵解相关的基因见 Fig. \ref{fig:Intersection-of-RA-M1M2-related-with-Glycolysis-related}
\item
  筛选到唯一的基因: PPARG (小鼠 Pparg)。其表达特征见 Fig. \ref{fig:Violing-plot-of-expression-level-of-the-Pparg}
\end{itemize}

\hypertarget{introduction}{%
\section{前言}\label{introduction}}

\hypertarget{methods}{%
\section{材料和方法}\label{methods}}

\hypertarget{ux6750ux6599}{%
\subsection{材料}\label{ux6750ux6599}}

All used GEO expression data and their design:

\begin{itemize}
\tightlist
\item
  \textbf{GSE184609}: scRNA-Seq analysis of FACS-sorted live synovial cells isolated from naïve mice (two replicates) or from mice at day 6, 14, or 25 of GPI-induced arthritis (one replicate per time point).
\end{itemize}

\hypertarget{ux65b9ux6cd5}{%
\subsection{方法}\label{ux65b9ux6cd5}}

Mainly used method:

\begin{itemize}
\tightlist
\item
  The \texttt{biomart} was used for mapping genes between organism (e.g., mgi\_symbol to hgnc\_symbol)\textsuperscript{\protect\hyperlink{ref-MappingIdentifDurinc2009}{2}}.
\item
  The Human Gene Database \texttt{GeneCards} used for disease related genes prediction\textsuperscript{\protect\hyperlink{ref-TheGenecardsSStelze2016}{3}}.
\item
  GEO \url{https://www.ncbi.nlm.nih.gov/geo/} used for expression dataset aquisition.
\item
  The data in published article of Jablonski et al used for distinguishing macrophage phenotypes (M0/M1/M2)\textsuperscript{\protect\hyperlink{ref-NovelMarkersTJablon2015}{1}}.
\item
  The R package \texttt{Seurat} used for scRNA-seq processing\textsuperscript{\protect\hyperlink{ref-IntegratedAnalHaoY2021}{4},\protect\hyperlink{ref-ComprehensiveIStuart2019}{5}}.
\item
  \texttt{SCSA} (python) used for cell type annotation\textsuperscript{\protect\hyperlink{ref-ScsaACellTyCaoY2020}{6}}.
\item
  R version 4.3.2 (2023-10-31); Other R packages (eg., \texttt{dplyr} and \texttt{ggplot2}) used for statistic analysis or data visualization.
\end{itemize}

\hypertarget{results}{%
\section{分析结果}\label{results}}

\hypertarget{dis}{%
\section{结论}\label{dis}}

\hypertarget{workflow}{%
\section{附:分析流程}\label{workflow}}

\hypertarget{scrna-seq}{%
\subsection{scRNA-seq}\label{scrna-seq}}

\hypertarget{ux6570ux636eux6765ux6e90}{%
\subsubsection{数据来源}\label{ux6570ux636eux6765ux6e90}}

这是一批小鼠的 单细胞测序数据。

\begin{center}\begin{tcolorbox}[colback=gray!10, colframe=gray!50, width=0.9\linewidth, arc=1mm, boxrule=0.5pt]
\textbf{
Data Source ID
:}

\vspace{0.5em}

    GSE184609

\vspace{2em}


\textbf{
data\_processing
:}

\vspace{0.5em}

    10X Genomics Cell Ranger v3.1

\vspace{2em}


\textbf{
data\_processing.1
:}

\vspace{0.5em}

    Gene-cell UMI matrix was generated for downstream
analyses. Low-quality cells were removed based on their
unique feature counts and mitochondrial gene content. Data
was normalized and log transformed using the default
setting of Seurat (version 3.1.4).

\vspace{2em}


\textbf{
data\_processing.2
:}

\vspace{0.5em}

    Genome\_build: mm10

\vspace{2em}


\textbf{
data\_processing.3
:}

\vspace{0.5em}

    Supplementary\_files\_format\_and\_content: For each
sample, there is one mtx file with filtered gene expressing
UMI counts for each sample, one tsv file containing gene
names, and one tsv file with cell barcodes.

\vspace{2em}
\end{tcolorbox}
\end{center}

\textbf{(上述信息框内容已保存至 \texttt{Figure+Table/GSE184609-content})}

\hypertarget{ux7ec6ux80deux805aux7c7bux4e0eux521dux6b65ux6ce8ux91ca}{%
\subsubsection{细胞聚类与初步注释}\label{ux7ec6ux80deux805aux7c7bux4e0eux521dux6b65ux6ce8ux91ca}}

使用 SCSA 对细胞类型注释。

Figure \ref{fig:SCSA-Cell-type-annotation} (下方图) 为图SCSA Cell type annotation概览。

\textbf{(对应文件为 \texttt{Figure+Table/SCSA-Cell-type-annotation.pdf})}

\def\@captype{figure}
\begin{center}
\includegraphics[width = 0.9\linewidth]{Figure+Table/SCSA-Cell-type-annotation.pdf}
\caption{SCSA Cell type annotation}\label{fig:SCSA-Cell-type-annotation}
\end{center}

\hypertarget{ux5de8ux566cux7ec6ux80deux91cdux805aux7c7b}{%
\subsubsection{巨噬细胞重聚类}\label{ux5de8ux566cux7ec6ux80deux91cdux805aux7c7b}}

Figure \ref{fig:Microphage-UMAP-Clustering} (下方图) 为图Microphage UMAP Clustering概览。

\textbf{(对应文件为 \texttt{Figure+Table/Microphage-UMAP-Clustering.pdf})}

\def\@captype{figure}
\begin{center}
\includegraphics[width = 0.9\linewidth]{Figure+Table/Microphage-UMAP-Clustering.pdf}
\caption{Microphage UMAP Clustering}\label{fig:Microphage-UMAP-Clustering}
\end{center}

\hypertarget{ux5de8ux566cux7ec6ux80deux8868ux578b-m0m1m2-ux9274ux5b9a-markers}{%
\subsubsection{巨噬细胞表型 M0、M1、M2 鉴定 Markers}\label{ux5de8ux566cux7ec6ux80deux8868ux578b-m0m1m2-ux9274ux5b9a-markers}}

Table \ref{tab:The-markers-for-Macrophage-phenotypes-annotation} (下方表格) 为表格The markers for Macrophage phenotypes annotation概览。

\textbf{(对应文件为 \texttt{Figure+Table/The-markers-for-Macrophage-phenotypes-annotation})}

\begin{center}\begin{tcolorbox}[colback=gray!10, colframe=gray!50, width=0.9\linewidth, arc=1mm, boxrule=0.5pt]注:表格共有26行2列,以下预览的表格可能省略部分数据;含有3个唯一`cell'。
\end{tcolorbox}
\end{center}

\begin{longtable}[]{@{}ll@{}}
\caption{\label{tab:The-markers-for-Macrophage-phenotypes-annotation}The markers for Macrophage phenotypes annotation}\tabularnewline
\toprule
cell & markers\tabularnewline
\midrule
\endfirsthead
\toprule
cell & markers\tabularnewline
\midrule
\endhead
Macrophage\_M0 & Sh2d3c\tabularnewline
Macrophage\_M0 & Slc13a3\tabularnewline
Macrophage\_M0 & Rcan1\tabularnewline
Macrophage\_M0 & Trp53inp1\tabularnewline
Macrophage\_M0 & Slc40a1\tabularnewline
Macrophage\_M0 & Il16\tabularnewline
Macrophage\_M1 & Cfb\tabularnewline
Macrophage\_M1 & Slfn4\tabularnewline
Macrophage\_M1 & H2-Q6\tabularnewline
Macrophage\_M1 & Fpr1\tabularnewline
Macrophage\_M1 & Slfn1\tabularnewline
Macrophage\_M1 & Ccrl2\tabularnewline
Macrophage\_M1 & Fpr2\tabularnewline
Macrophage\_M1 & Cxcl10\tabularnewline
Macrophage\_M1 & Oasl1\tabularnewline
\ldots{} & \ldots{}\tabularnewline
\bottomrule
\end{longtable}

Figure \ref{fig:Heatmap-show-the-reference-genes} (下方图) 为图Heatmap show the reference genes概览。

\textbf{(对应文件为 \texttt{Figure+Table/Heatmap-show-the-reference-genes.pdf})}

\def\@captype{figure}
\begin{center}
\includegraphics[width = 0.9\linewidth]{Figure+Table/Heatmap-show-the-reference-genes.pdf}
\caption{Heatmap show the reference genes}\label{fig:Heatmap-show-the-reference-genes}
\end{center}

Figure \ref{fig:Macrophage-phenotypes-type-annotation} (下方图) 为图Macrophage phenotypes type annotation概览。

\textbf{(对应文件为 \texttt{Figure+Table/Macrophage-phenotypes-type-annotation.pdf})}

\def\@captype{figure}
\begin{center}
\includegraphics[width = 0.9\linewidth]{Figure+Table/Macrophage-phenotypes-type-annotation.pdf}
\caption{Macrophage phenotypes type annotation}\label{fig:Macrophage-phenotypes-type-annotation}
\end{center}

\hypertarget{ra-ux4e0e-control-ux7684ux5de8ux566cux7ec6ux80deux8868ux578b}{%
\subsubsection{RA 与 Control 的巨噬细胞表型}\label{ra-ux4e0e-control-ux7684ux5de8ux566cux7ec6ux80deux8868ux578b}}

随后,根据数据集的来源 (RA 或 Control,将巨噬细胞分类)

Figure \ref{fig:The-Phenotypes} (下方图) 为图The Phenotypes概览。

\textbf{(对应文件为 \texttt{Figure+Table/The-Phenotypes.pdf})}

\def\@captype{figure}
\begin{center}
\includegraphics[width = 0.9\linewidth]{Figure+Table/The-Phenotypes.pdf}
\caption{The Phenotypes}\label{fig:The-Phenotypes}
\end{center}

\hypertarget{ux5deeux5f02ux5206ux6790}{%
\subsubsection{差异分析}\label{ux5deeux5f02ux5206ux6790}}

\begin{itemize}
\tightlist
\item
  XX 在RA中M1巨噬细胞上调: \texttt{GPI-day25-RA\_Macrophage\_M1} vs \texttt{Control\_Macrophage\_M1}
\item
  其中XX基因主要分布在M1巨噬细胞而非M2巨噬细胞上: \texttt{GPI-day25-RA\_Macrophage\_M1} vs \texttt{GPI-day25-RA\_Macrophage\_M2}
\end{itemize}

Table \ref{tab:DEGs-of-the-contrasts} (下方表格) 为表格DEGs of the contrasts概览。

\textbf{(对应文件为 \texttt{Figure+Table/DEGs-of-the-contrasts.csv})}

\begin{center}\begin{tcolorbox}[colback=gray!10, colframe=gray!50, width=0.9\linewidth, arc=1mm, boxrule=0.5pt]注:表格共有355行7列,以下预览的表格可能省略部分数据;含有2个唯一`contrast;含有335个唯一`gene'。
\end{tcolorbox}
\end{center}

\begin{longtable}[]{@{}lllllll@{}}
\caption{\label{tab:DEGs-of-the-contrasts}DEGs of the contrasts}\tabularnewline
\toprule
contrast & p\_val & avg\_log2FC & pct.1 & pct.2 & p\_val\_adj & gene\tabularnewline
\midrule
\endfirsthead
\toprule
contrast & p\_val & avg\_log2FC & pct.1 & pct.2 & p\_val\_adj & gene\tabularnewline
\midrule
\endhead
GPI-day25-\ldots{} & 1.75831850\ldots{} & 2.17770295\ldots{} & 0.044 & 0.385 & 5.27495551\ldots{} & Adora3\tabularnewline
GPI-day25-\ldots{} & 2.23900708\ldots{} & 9.52901476\ldots{} & 0.069 & 0.427 & 6.71702126\ldots{} & F7\tabularnewline
GPI-day25-\ldots{} & 6.04375313\ldots{} & 9.97788827\ldots{} & 0.093 & 0.536 & 1.81312594\ldots{} & Hal\tabularnewline
GPI-day25-\ldots{} & 1.83819770\ldots{} & 13.5930067\ldots{} & 0.052 & 0.641 & 5.51459312\ldots{} & Cxcl13\tabularnewline
GPI-day25-\ldots{} & 1.00690808\ldots{} & 4.79632080\ldots{} & 0.153 & 0.583 & 3.02072424\ldots{} & Ifi44\tabularnewline
GPI-day25-\ldots{} & 2.16444867\ldots{} & 8.41136649\ldots{} & 0.153 & 0.87 & 6.49334601\ldots{} & Slc13a3\tabularnewline
GPI-day25-\ldots{} & 8.87142099\ldots{} & 7.46793928\ldots{} & 0.081 & 0.391 & 2.66142629\ldots{} & Cd4\tabularnewline
GPI-day25-\ldots{} & 1.52778766\ldots{} & 0.95745400\ldots{} & 0.141 & 0.307 & 4.58336298\ldots{} & Tnfsf14\tabularnewline
GPI-day25-\ldots{} & 4.80404528\ldots{} & 3.93765968\ldots{} & 0.169 & 0.484 & 1.44121358\ldots{} & Cd79b\tabularnewline
GPI-day25-\ldots{} & 8.42060311\ldots{} & 3.12179649\ldots{} & 0.06 & 0.651 & 2.52618093\ldots{} & Cd209e\tabularnewline
GPI-day25-\ldots{} & 8.42067724\ldots{} & 1.96704094\ldots{} & 0.145 & 0.786 & 2.52620317\ldots{} & Adgre4\tabularnewline
GPI-day25-\ldots{} & 8.24915617\ldots{} & 8.83724798\ldots{} & 0.161 & 0.766 & 2.47474685\ldots{} & Pparg\tabularnewline
GPI-day25-\ldots{} & 1.31157015\ldots{} & 8.50505864\ldots{} & 0.153 & 0.292 & 3.93471045\ldots{} & F10\tabularnewline
GPI-day25-\ldots{} & 2.28779328\ldots{} & 2.51813054\ldots{} & 0.024 & 0.411 & 6.86337986\ldots{} & Apoc4\tabularnewline
GPI-day25-\ldots{} & 2.75634208\ldots{} & 4.11075823\ldots{} & 0.073 & 0.755 & 8.26902626\ldots{} & Il10\tabularnewline
\ldots{} & \ldots{} & \ldots{} & \ldots{} & \ldots{} & \ldots{} & \ldots{}\tabularnewline
\bottomrule
\end{longtable}

Figure \ref{fig:Intersection-of-RA-M1-up-with-M1-not-M2} (下方图) 为图Intersection of RA M1 up with M1 not M2概览。

\textbf{(对应文件为 \texttt{Figure+Table/Intersection-of-RA-M1-up-with-M1-not-M2.pdf})}

\def\@captype{figure}
\begin{center}
\includegraphics[width = 0.9\linewidth]{Figure+Table/Intersection-of-RA-M1-up-with-M1-not-M2.pdf}
\caption{Intersection of RA M1 up with M1 not M2}\label{fig:Intersection-of-RA-M1-up-with-M1-not-M2}
\end{center}
\begin{center}\begin{tcolorbox}[colback=gray!10, colframe=gray!50, width=0.9\linewidth, arc=1mm, boxrule=0.5pt]
\textbf{
Intersection
:}

\vspace{0.5em}

    Ifi44, Adgre4, Pparg, Dppa3, Cadm1, P2ry14, Gm1673,
Vwf, Ednrb, Fam43a, Bambi, Slc28a2, Plk2, Rcn3, Rrm1,
Ifi204, Bmp2, Gfra2, Spon1, Gstm1

\vspace{2em}
\end{tcolorbox}
\end{center}

\textbf{(上述信息框内容已保存至 \texttt{Figure+Table/Intersection-of-RA-M1-up-with-M1-not-M2-content})}

\hypertarget{ux5c0fux9f20ux57faux56e0ux6620ux5c04ux5230ux4ebaux7c7bux57faux56e0}{%
\subsection{小鼠基因映射到人类基因}\label{ux5c0fux9f20ux57faux56e0ux6620ux5c04ux5230ux4ebaux7c7bux57faux56e0}}

Table \ref{tab:Mapped-genes} (下方表格) 为表格Mapped genes概览。

\textbf{(对应文件为 \texttt{Figure+Table/Mapped-genes.csv})}

\begin{center}\begin{tcolorbox}[colback=gray!10, colframe=gray!50, width=0.9\linewidth, arc=1mm, boxrule=0.5pt]注:表格共有19行2列,以下预览的表格可能省略部分数据;含有19个唯一`mgi\_symbol;含有19个唯一`hgnc\_symbol'。
\end{tcolorbox}
\end{center}
\begin{center}\begin{tcolorbox}[colback=gray!10, colframe=gray!50, width=0.9\linewidth, arc=1mm, boxrule=0.5pt]\begin{enumerate}\tightlist
\item hgnc\_symbol:  基因名 (Human)
\item mgi\_symbol:  基因名 (Mice)
\end{enumerate}\end{tcolorbox}
\end{center}

\begin{longtable}[]{@{}ll@{}}
\caption{\label{tab:Mapped-genes}Mapped genes}\tabularnewline
\toprule
mgi\_symbol & hgnc\_symbol\tabularnewline
\midrule
\endfirsthead
\toprule
mgi\_symbol & hgnc\_symbol\tabularnewline
\midrule
\endhead
Bmp2 & BMP2\tabularnewline
Ednrb & EDNRB\tabularnewline
Dppa3 & DPPA3\tabularnewline
Spon1 & SPON1\tabularnewline
Gfra2 & GFRA2\tabularnewline
Bambi & BAMBI\tabularnewline
Cadm1 & CADM1\tabularnewline
Slc28a2 & SLC28A2\tabularnewline
Rrm1 & RRM1\tabularnewline
Ifi44 & IFI44\tabularnewline
Gm1673 & C4orf48\tabularnewline
Ifi204 & MNDA\tabularnewline
P2ry14 & P2RY14\tabularnewline
Rcn3 & RCN3\tabularnewline
Gstm1 & GSTM1\tabularnewline
\ldots{} & \ldots{}\tabularnewline
\bottomrule
\end{longtable}

\hypertarget{ux7cd6ux9175ux89e3ux76f8ux5173ux57faux56e0}{%
\subsection{糖酵解相关基因}\label{ux7cd6ux9175ux89e3ux76f8ux5173ux57faux56e0}}

\begin{center}\begin{tcolorbox}[colback=gray!10, colframe=gray!50, width=0.9\linewidth, arc=1mm, boxrule=0.5pt]
\textbf{
The GeneCards data was obtained by querying
:}

\vspace{0.5em}

    Glycolysis

\vspace{2em}


\textbf{
Restrict (with quotes)
:}

\vspace{0.5em}

    FALSE

\vspace{2em}


\textbf{
Filtering by Score:
:}

\vspace{0.5em}

    Score > 3

\vspace{2em}
\end{tcolorbox}
\end{center}

Table \ref{tab:Glycolysis-related-genes-from-GeneCards} (下方表格) 为表格Glycolysis related genes from GeneCards概览。

\textbf{(对应文件为 \texttt{Figure+Table/Glycolysis-related-genes-from-GeneCards.xlsx})}

\begin{center}\begin{tcolorbox}[colback=gray!10, colframe=gray!50, width=0.9\linewidth, arc=1mm, boxrule=0.5pt]注:表格共有118行7列,以下预览的表格可能省略部分数据;含有118个唯一`Symbol'。
\end{tcolorbox}
\end{center}

\begin{longtable}[]{@{}lllllll@{}}
\caption{\label{tab:Glycolysis-related-genes-from-GeneCards}Glycolysis related genes from GeneCards}\tabularnewline
\toprule
Symbol & Description & Category & UniProt\_ID & GIFtS & GC\_id & Score\tabularnewline
\midrule
\endfirsthead
\toprule
Symbol & Description & Category & UniProt\_ID & GIFtS & GC\_id & Score\tabularnewline
\midrule
\endhead
TIGAR & TP53 Induc\ldots{} & Protein Co\ldots{} & Q9NQ88 & 45 & GC12P038924 & 22.4\tabularnewline
PKM & Pyruvate K\ldots{} & Protein Co\ldots{} & P14618 & 58 & GC15M072199 & 20.77\tabularnewline
HK2 & Hexokinase 2 & Protein Co\ldots{} & P52789 & 55 & GC02P074947 & 19.42\tabularnewline
GAPDH & Glyceralde\ldots{} & Protein Co\ldots{} & P04406 & 59 & GC12P038965 & 17.14\tabularnewline
LDHA & Lactate De\ldots{} & Protein Co\ldots{} & P00338 & 59 & GC11P018394 & 15.81\tabularnewline
HIF1A & Hypoxia In\ldots{} & Protein Co\ldots{} & Q16665 & 57 & GC14P061695 & 15.1\tabularnewline
RRAD & RRAD, Ras \ldots{} & Protein Co\ldots{} & P55042 & 46 & GC16M067483 & 15.1\tabularnewline
HK1 & Hexokinase 1 & Protein Co\ldots{} & P19367 & 59 & GC10P069269 & 14.64\tabularnewline
PKLR & Pyruvate K\ldots{} & Protein Co\ldots{} & P30613 & 55 & GC01M155289 & 13.37\tabularnewline
ENO1 & Enolase 1 & Protein Co\ldots{} & P06733 & 56 & GC01M008861 & 13.36\tabularnewline
ENO3 & Enolase 3 & Protein Co\ldots{} & P13929 & 54 & GC17P004948 & 13.33\tabularnewline
PFKP & Phosphofru\ldots{} & Protein Co\ldots{} & Q01813 & 53 & GC10P003066 & 13.19\tabularnewline
TPI1 & Triosephos\ldots{} & Protein Co\ldots{} & P60174 & 55 & GC12P006867 & 13.18\tabularnewline
GLTC1 & Glycolysis\ldots{} & RNA Gene (\ldots{} & & 2 & GC11U909607 & 12.97\tabularnewline
PGK1 & Phosphogly\ldots{} & Protein Co\ldots{} & P00558 & 57 & GC0XP078166 & 12.94\tabularnewline
\ldots{} & \ldots{} & \ldots{} & \ldots{} & \ldots{} & \ldots{} & \ldots{}\tabularnewline
\bottomrule
\end{longtable}

Figure \ref{fig:Intersection-of-RA-M1M2-related-with-Glycolysis-related} (下方图) 为图Intersection of RA M1M2 related with Glycolysis related概览。

\textbf{(对应文件为 \texttt{Figure+Table/Intersection-of-RA-M1M2-related-with-Glycolysis-related.pdf})}

\def\@captype{figure}
\begin{center}
\includegraphics[width = 0.9\linewidth]{Figure+Table/Intersection-of-RA-M1M2-related-with-Glycolysis-related.pdf}
\caption{Intersection of RA M1M2 related with Glycolysis related}\label{fig:Intersection-of-RA-M1M2-related-with-Glycolysis-related}
\end{center}
\begin{center}\begin{tcolorbox}[colback=gray!10, colframe=gray!50, width=0.9\linewidth, arc=1mm, boxrule=0.5pt]
\textbf{
Intersection
:}

\vspace{0.5em}

    PPARG

\vspace{2em}
\end{tcolorbox}
\end{center}

\textbf{(上述信息框内容已保存至 \texttt{Figure+Table/Intersection-of-RA-M1M2-related-with-Glycolysis-related-content})}

\hypertarget{ux4ea4ux96c6ux57faux56e0ux7684ux8868ux8fbe-ux5c0fux9f20ux5355ux7ec6ux80deux6570ux636e}{%
\subsection{交集基因的表达 (小鼠单细胞数据)}\label{ux4ea4ux96c6ux57faux56e0ux7684ux8868ux8fbe-ux5c0fux9f20ux5355ux7ec6ux80deux6570ux636e}}

Figure \ref{fig:Violing-plot-of-expression-level-of-the-Pparg} (下方图) 为图Violing plot of expression level of the Pparg概览。

\textbf{(对应文件为 \texttt{Figure+Table/Violing-plot-of-expression-level-of-the-Pparg.pdf})}

\def\@captype{figure}
\begin{center}
\includegraphics[width = 0.9\linewidth]{Figure+Table/Violing-plot-of-expression-level-of-the-Pparg.pdf}
\caption{Violing plot of expression level of the Pparg}\label{fig:Violing-plot-of-expression-level-of-the-Pparg}
\end{center}

\hypertarget{bibliography}{%
\section*{Reference}\label{bibliography}}
\addcontentsline{toc}{section}{Reference}

\hypertarget{refs}{}
\begin{cslreferences}
\leavevmode\hypertarget{ref-NovelMarkersTJablon2015}{}%
1. Jablonski, K. A. \emph{et al.} Novel markers to delineate murine m1 and m2 macrophages. \emph{PloS one} \textbf{10}, (2015).

\leavevmode\hypertarget{ref-MappingIdentifDurinc2009}{}%
2. Durinck, S., Spellman, P. T., Birney, E. \& Huber, W. Mapping identifiers for the integration of genomic datasets with the r/bioconductor package biomaRt. \emph{Nature protocols} \textbf{4}, 1184--1191 (2009).

\leavevmode\hypertarget{ref-TheGenecardsSStelze2016}{}%
3. Stelzer, G. \emph{et al.} The genecards suite: From gene data mining to disease genome sequence analyses. \emph{Current protocols in bioinformatics} \textbf{54}, 1.30.1--1.30.33 (2016).

\leavevmode\hypertarget{ref-IntegratedAnalHaoY2021}{}%
4. Hao, Y. \emph{et al.} Integrated analysis of multimodal single-cell data. \emph{Cell} \textbf{184}, (2021).

\leavevmode\hypertarget{ref-ComprehensiveIStuart2019}{}%
5. Stuart, T. \emph{et al.} Comprehensive integration of single-cell data. \emph{Cell} \textbf{177}, (2019).

\leavevmode\hypertarget{ref-ScsaACellTyCaoY2020}{}%
6. Cao, Y., Wang, X. \& Peng, G. SCSA: A cell type annotation tool for single-cell rna-seq data. \emph{Frontiers in genetics} \textbf{11}, (2020).
\end{cslreferences}

\end{document}
