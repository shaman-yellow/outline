% Options for packages loaded elsewhere
\PassOptionsToPackage{unicode}{hyperref}
\PassOptionsToPackage{hyphens}{url}
%
\documentclass[
]{article}
\usepackage{lmodern}
\usepackage{amssymb,amsmath}
\usepackage{ifxetex,ifluatex}
\ifnum 0\ifxetex 1\fi\ifluatex 1\fi=0 % if pdftex
  \usepackage[T1]{fontenc}
  \usepackage[utf8]{inputenc}
  \usepackage{textcomp} % provide euro and other symbols
\else % if luatex or xetex
  \usepackage{unicode-math}
  \defaultfontfeatures{Scale=MatchLowercase}
  \defaultfontfeatures[\rmfamily]{Ligatures=TeX,Scale=1}
\fi
% Use upquote if available, for straight quotes in verbatim environments
\IfFileExists{upquote.sty}{\usepackage{upquote}}{}
\IfFileExists{microtype.sty}{% use microtype if available
  \usepackage[]{microtype}
  \UseMicrotypeSet[protrusion]{basicmath} % disable protrusion for tt fonts
}{}
\makeatletter
\@ifundefined{KOMAClassName}{% if non-KOMA class
  \IfFileExists{parskip.sty}{%
    \usepackage{parskip}
  }{% else
    \setlength{\parindent}{0pt}
    \setlength{\parskip}{6pt plus 2pt minus 1pt}}
}{% if KOMA class
  \KOMAoptions{parskip=half}}
\makeatother
\usepackage{xcolor}
\IfFileExists{xurl.sty}{\usepackage{xurl}}{} % add URL line breaks if available
\IfFileExists{bookmark.sty}{\usepackage{bookmark}}{\usepackage{hyperref}}
\hypersetup{
  hidelinks,
  pdfcreator={LaTeX via pandoc}}
\urlstyle{same} % disable monospaced font for URLs
\usepackage[margin=1in]{geometry}
\usepackage{longtable,booktabs}
% Correct order of tables after \paragraph or \subparagraph
\usepackage{etoolbox}
\makeatletter
\patchcmd\longtable{\par}{\if@noskipsec\mbox{}\fi\par}{}{}
\makeatother
% Allow footnotes in longtable head/foot
\IfFileExists{footnotehyper.sty}{\usepackage{footnotehyper}}{\usepackage{footnote}}
\makesavenoteenv{longtable}
\usepackage{graphicx}
\makeatletter
\def\maxwidth{\ifdim\Gin@nat@width>\linewidth\linewidth\else\Gin@nat@width\fi}
\def\maxheight{\ifdim\Gin@nat@height>\textheight\textheight\else\Gin@nat@height\fi}
\makeatother
% Scale images if necessary, so that they will not overflow the page
% margins by default, and it is still possible to overwrite the defaults
% using explicit options in \includegraphics[width, height, ...]{}
\setkeys{Gin}{width=\maxwidth,height=\maxheight,keepaspectratio}
% Set default figure placement to htbp
\makeatletter
\def\fps@figure{htbp}
\makeatother
\setlength{\emergencystretch}{3em} % prevent overfull lines
\providecommand{\tightlist}{%
  \setlength{\itemsep}{0pt}\setlength{\parskip}{0pt}}
\setcounter{secnumdepth}{5}
\usepackage{caption} \captionsetup{font={footnotesize},width=6in} \renewcommand{\dblfloatpagefraction}{.9} \makeatletter \renewenvironment{figure} {\def\@captype{figure}} \makeatother \@ifundefined{Shaded}{\newenvironment{Shaded}} \@ifundefined{snugshade}{\newenvironment{snugshade}} \renewenvironment{Shaded}{\begin{snugshade}}{\end{snugshade}} \definecolor{shadecolor}{RGB}{230,230,230} \usepackage{xeCJK} \usepackage{setspace} \setstretch{1.3} \usepackage{tcolorbox} \setcounter{secnumdepth}{4} \setcounter{tocdepth}{4} \usepackage{wallpaper} \usepackage[absolute]{textpos} \tcbuselibrary{breakable} \renewenvironment{Shaded} {\begin{tcolorbox}[colback = gray!10, colframe = gray!40, width = 16cm, arc = 1mm, auto outer arc, title = {R input}]} {\end{tcolorbox}} \usepackage{titlesec} \titleformat{\paragraph} {\fontsize{10pt}{0pt}\bfseries} {\arabic{section}.\arabic{subsection}.\arabic{subsubsection}.\arabic{paragraph}} {1em} {} []
\newlength{\cslhangindent}
\setlength{\cslhangindent}{1.5em}
\newenvironment{cslreferences}%
  {}%
  {\par}

\author{}
\date{\vspace{-2.5em}}

\begin{document}

\begin{titlepage} \newgeometry{top=7.5cm}
\ThisCenterWallPaper{1.12}{~/outline/lixiao//cover_page.pdf}
\begin{center} \textbf{\Huge
菌群+对应代谢产物介导+机制研究} \vspace{4em}
\begin{textblock}{10}(3,5.9) \huge
\textbf{\textcolor{white}{2024-02-23}}
\end{textblock} \begin{textblock}{10}(3,7.3)
\Large \textcolor{black}{LiChuang Huang}
\end{textblock} \begin{textblock}{10}(3,11.3)
\Large \textcolor{black}{@立效研究院}
\end{textblock} \end{center} \end{titlepage}
\restoregeometry

\pagenumbering{roman}

\tableofcontents

\listoffigures

\listoftables

\newpage

\pagenumbering{arabic}

\hypertarget{abstract}{%
\section{摘要}\label{abstract}}

\hypertarget{ux9700ux6c42ux6982ux8981}{%
\subsection{需求概要}\label{ux9700ux6c42ux6982ux8981}}

数据分组:

\begin{itemize}
\tightlist
\item
  con: Control
\item
  A: colitis
\item
  B: colon precancerous lesions
\end{itemize}

肠道菌群测序结果+生信分析,得出:菌群+对应代谢产物介导+机制研究+再闭环回到临床。

具体:

溃疡性结肠炎和结肠癌的肠道菌群之间的区别和关联,进而研究其对应的机制,研究结肠炎向结肠癌发展的关键机制,为临床早期筛查提供理论支持

\hypertarget{ux5206ux6790ux7ed3ux679c}{%
\subsection{分析结果}\label{ux5206ux6790ux7ed3ux679c}}

\begin{itemize}
\tightlist
\item
  基本分析:

  \begin{itemize}
  \tightlist
  \item
    alpha、beta 多样性,A、B、C 组均无显著性差异 (\ref{alpha}, \ref{beta})。
  \item
    差异菌筛选 (level 6, Species) ,筛得差异菌 (Fig. \ref{fig:Ancom-test-group-level-6-volcano}):d\_\_Bacteria;p\_\_Proteobacteria;c\_\_Alphaproteobacteria;o\_\_Rhizobiales;f\_\_Beijerinckiaceae;g\_\_Methylobacterium-Methylorubrum
    该差异菌主要存在于 A、B 组,不存在 (或少量于) 于 C (对照) 组。
    含量见 Fig. \ref{fig:Ancom-test-group-level-6-Percentile-abundance}
  \item
    \ldots{}
  \item
    差异菌 (level 2, Phylum) (Fig. \ref{fig:Ancom-test-group-level-2-volcano}), 同样的有:d\_\_Bacteria;p\_\_Proteobacteria
    该差异菌主要存在于 A、B 组,不存在 (或少量于) 于 C (对照) 组。
    含量见 Fig. \ref{fig:Ancom-test-group-level-2-Percentile-abundance}
  \end{itemize}
\item
  从肠道菌到相关代谢物:

  \begin{itemize}
  \tightlist
  \item
    使用 gutMDisorder 未发现相关代谢物。
  \item
    从一孟德尔随机化相关研究中\textsuperscript{\protect\hyperlink{ref-MendelianRandoLiuX2022}{1}},发现了与差异菌相关的代谢物,见
    Tab. \ref{tab:MendelianRandoLiuX2022-matched-data}。
    这些代谢物为 (详细信息见 Tab. \ref{tab:compounds-ID}):
    5-methyltetrahydrofolic acid, selenium, L-cystine, Glutamic acid
  \item
    用 MetaboAnalystR 对相关代谢物进行富集分析,富集到两条通路 (见 Fig. \ref{fig:MetaboAnalyst-kegg-enrichment})
  \item
    用 FELLA 对相关代谢物富集分析, 可以发现相关联的更多通路或反应模块
    (结果见 Fig. \ref{fig:Enrichment-with-algorithm-PageRank},
    Tab. \ref{tab:Data-of-enrichment-with-algorithm-PageRank})
  \end{itemize}
\item
  尝试从已有的关于结肠炎或结肠癌的研究中验证上述发现:

  \begin{itemize}
  \tightlist
  \item
    从结肠癌相关研究中匹配到\textsuperscript{\protect\hyperlink{ref-LossOfSymbiotSadegh2024}{2}} (Tab.
    \ref{tab:LossOfSymbiotSadegh2024-matched-Phylum-microbiota}):d\_\_Bacteria;p\_\_Proteobacteria
    (注:在 Phylum 水平上得到验证)
  \item
    未从其它文献中匹配到代谢物或差异肠道菌 (见 \ref{valids}。
  \end{itemize}
\item
  结肠炎向结肠癌之间的转化:

  \begin{itemize}
  \tightlist
  \item
    A 为结肠炎,B 为结肠癌前病变; A 与 B 组间无显著差异菌,因此无法从这一批数据探究可能的发展机制 (A -\textgreater{} B)。
  \end{itemize}
\end{itemize}

\hypertarget{introduction}{%
\section{前言}\label{introduction}}

\hypertarget{methods}{%
\section{材料和方法}\label{methods}}

\hypertarget{ux6750ux6599}{%
\subsection{材料}\label{ux6750ux6599}}

Other data obtained from published article (e.g., supplementary tables):

\begin{itemize}
\tightlist
\item
  Supplementary file from article refer to\textsuperscript{\protect\hyperlink{ref-MendelianRandoLiuX2022}{1}}.
\end{itemize}

\hypertarget{ux65b9ux6cd5}{%
\subsection{方法}\label{ux65b9ux6cd5}}

Mainly used method:

\begin{itemize}
\tightlist
\item
  R package \texttt{FELLA} used for metabolite enrichment analysis\textsuperscript{\protect\hyperlink{ref-FellaAnRPacPicart2018}{3}}.
\item
  \texttt{Fastp} used for Fastq data preprocessing\textsuperscript{\protect\hyperlink{ref-UltrafastOnePChen2023}{4}}.
\item
  Database \texttt{gutMDisorder} used for finding associations between gut microbiota and metabolites\textsuperscript{\protect\hyperlink{ref-GutmdisorderACheng2019}{5}}.
\item
  R package \texttt{MicrobiotaProcess} used for microbiome data visualization\textsuperscript{\protect\hyperlink{ref-MicrobiotaproceXuSh2023}{6}}.
\item
  \texttt{MetaboAnalyst} used for metabolomic data analysis\textsuperscript{\protect\hyperlink{ref-Metaboanalyst4Chong2018}{7}}.
\item
  \texttt{Qiime2} used for gut microbiome 16s rRNA analysis\textsuperscript{\protect\hyperlink{ref-ReproducibleIBolyen2019}{8}--\protect\hyperlink{ref-MicrobialCommuHamday2009}{12}}.
\item
  Other R packages (eg., \texttt{dplyr} and \texttt{ggplot2}) used for statistic analysis or data visualization.
\end{itemize}

\hypertarget{results}{%
\section{分析结果}\label{results}}

\hypertarget{dis}{%
\section{结论}\label{dis}}

\hypertarget{workflow}{%
\section{附:分析流程}\label{workflow}}

\hypertarget{microbiota-16s-rna}{%
\subsection{Microbiota 16s RNA}\label{microbiota-16s-rna}}

\hypertarget{fastp-qc}{%
\subsubsection{Fastp QC}\label{fastp-qc}}

原始数据质控:

`Fastp QC' 数据已全部提供。

\textbf{(对应文件为 \texttt{./fastp\_report/})}

\begin{center}\begin{tcolorbox}[colback=gray!10, colframe=gray!50, width=0.9\linewidth, arc=1mm, boxrule=0.5pt]注:文件夹./fastp\_report/共包含23个文件。

\begin{enumerate}\tightlist
\item A1.338F\_806R..html
\item A2.338F\_806R..html
\item A3.338F\_806R..html
\item A4.338F\_806R..html
\item A5.338F\_806R..html
\item ...
\end{enumerate}\end{tcolorbox}
\end{center}

\hypertarget{ux5143ux6570ux636e}{%
\subsubsection{元数据}\label{ux5143ux6570ux636e}}

Table \ref{tab:microbiota-metadata} (下方表格) 为表格microbiota metadata概览。

\textbf{(对应文件为 \texttt{Figure+Table/microbiota-metadata.csv})}

\begin{center}\begin{tcolorbox}[colback=gray!10, colframe=gray!50, width=0.9\linewidth, arc=1mm, boxrule=0.5pt]注:表格共有22行7列,以下预览的表格可能省略部分数据;表格含有22个唯一`SampleName'。
\end{tcolorbox}
\end{center}
\begin{center}\begin{tcolorbox}[colback=gray!10, colframe=gray!50, width=0.9\linewidth, arc=1mm, boxrule=0.5pt]\begin{enumerate}\tightlist
\item group:  分组名称
\end{enumerate}\end{tcolorbox}
\end{center}

\begin{longtable}[]{@{}lllllll@{}}
\caption{\label{tab:microbiota-metadata}Microbiota metadata}\tabularnewline
\toprule
SampleName & group & dirs & reports & Run & forward-ab\ldots{} & reverse-ab\ldots{}\tabularnewline
\midrule
\endfirsthead
\toprule
SampleName & group & dirs & reports & Run & forward-ab\ldots{} & reverse-ab\ldots{}\tabularnewline
\midrule
\endhead
A1 & A & ./material\ldots{} & ./material\ldots{} & rawData & /home/echo\ldots{} & /home/echo\ldots{}\tabularnewline
A2 & A & ./material\ldots{} & ./material\ldots{} & rawData & /home/echo\ldots{} & /home/echo\ldots{}\tabularnewline
A3 & A & ./material\ldots{} & ./material\ldots{} & rawData & /home/echo\ldots{} & /home/echo\ldots{}\tabularnewline
A4 & A & ./material\ldots{} & ./material\ldots{} & rawData & /home/echo\ldots{} & /home/echo\ldots{}\tabularnewline
A5 & A & ./material\ldots{} & ./material\ldots{} & rawData & /home/echo\ldots{} & /home/echo\ldots{}\tabularnewline
A6 & A & ./material\ldots{} & ./material\ldots{} & rawData & /home/echo\ldots{} & /home/echo\ldots{}\tabularnewline
A7 & A & ./material\ldots{} & ./material\ldots{} & rawData & /home/echo\ldots{} & /home/echo\ldots{}\tabularnewline
A8 & A & ./material\ldots{} & ./material\ldots{} & rawData & /home/echo\ldots{} & /home/echo\ldots{}\tabularnewline
B1 & B & ./material\ldots{} & ./material\ldots{} & rawData & /home/echo\ldots{} & /home/echo\ldots{}\tabularnewline
B2 & B & ./material\ldots{} & ./material\ldots{} & rawData & /home/echo\ldots{} & /home/echo\ldots{}\tabularnewline
B3 & B & ./material\ldots{} & ./material\ldots{} & rawData & /home/echo\ldots{} & /home/echo\ldots{}\tabularnewline
B4 & B & ./material\ldots{} & ./material\ldots{} & rawData & /home/echo\ldots{} & /home/echo\ldots{}\tabularnewline
B5 & B & ./material\ldots{} & ./material\ldots{} & rawData & /home/echo\ldots{} & /home/echo\ldots{}\tabularnewline
B6 & B & ./material\ldots{} & ./material\ldots{} & rawData & /home/echo\ldots{} & /home/echo\ldots{}\tabularnewline
B7 & B & ./material\ldots{} & ./material\ldots{} & rawData & /home/echo\ldots{} & /home/echo\ldots{}\tabularnewline
\ldots{} & \ldots{} & \ldots{} & \ldots{} & \ldots{} & \ldots{} & \ldots{}\tabularnewline
\bottomrule
\end{longtable}

\hypertarget{qiime2-ux5206ux6790}{%
\subsubsection{Qiime2 分析}\label{qiime2-ux5206ux6790}}

Microbiota 数据经 Qiime2 分析后,由 \texttt{MicrobiotaProcess} 下游分析和可视化。

\hypertarget{microbiotaprocess-ux5206ux6790}{%
\subsubsection{MicrobiotaProcess 分析}\label{microbiotaprocess-ux5206ux6790}}

\hypertarget{ux6837ux672cux805aux7c7b}{%
\paragraph{样本聚类}\label{ux6837ux672cux805aux7c7b}}

Figure \ref{fig:PCoA} (下方图) 为图PCoA概览。

\textbf{(对应文件为 \texttt{Figure+Table/PCoA.pdf})}

\def\@captype{figure}
\begin{center}
\includegraphics[width = 0.9\linewidth]{Figure+Table/PCoA.pdf}
\caption{PCoA}\label{fig:PCoA}
\end{center}

\hypertarget{alpha}{%
\paragraph{Alpha 多样性}\label{alpha}}

三组 alpha 多样性没有显著差异。

Figure \ref{fig:Alpha-diversity} (下方图) 为图Alpha diversity概览。

\textbf{(对应文件为 \texttt{Figure+Table/Alpha-diversity.pdf})}

\def\@captype{figure}
\begin{center}
\includegraphics[width = 0.9\linewidth]{Figure+Table/Alpha-diversity.pdf}
\caption{Alpha diversity}\label{fig:Alpha-diversity}
\end{center}

`Taxonomy abundance' 数据已全部提供。

\textbf{(对应文件为 \texttt{Figure+Table/Taxonomy-abundance})}

\begin{center}\begin{tcolorbox}[colback=gray!10, colframe=gray!50, width=0.9\linewidth, arc=1mm, boxrule=0.5pt]注:文件夹Figure+Table/Taxonomy-abundance共包含6个文件。

\begin{enumerate}\tightlist
\item 1\_Phylum.pdf
\item 2\_Class.pdf
\item 3\_Order.pdf
\item 4\_Family.pdf
\item 5\_Genus.pdf
\item ...
\end{enumerate}\end{tcolorbox}
\end{center}

\hypertarget{alpha-ux7a00ux758fux66f2ux7ebf}{%
\paragraph{Alpha 稀疏曲线}\label{alpha-ux7a00ux758fux66f2ux7ebf}}

Figure \ref{fig:Alpha-rarefaction} (下方图) 为图Alpha rarefaction概览。

\textbf{(对应文件为 \texttt{Figure+Table/Alpha-rarefaction.pdf})}

\def\@captype{figure}
\begin{center}
\includegraphics[width = 0.9\linewidth]{Figure+Table/Alpha-rarefaction.pdf}
\caption{Alpha rarefaction}\label{fig:Alpha-rarefaction}
\end{center}

\hypertarget{beta}{%
\paragraph{Beta 多样性}\label{beta}}

Beta 多样性无显著差异。

Figure \ref{fig:Beta-diversity-group-test} (下方图) 为图Beta diversity group test概览。

\textbf{(对应文件为 \texttt{Figure+Table/Beta-diversity-group-test.pdf})}

\def\@captype{figure}
\begin{center}
\includegraphics[width = 0.9\linewidth]{Figure+Table/Beta-diversity-group-test.pdf}
\caption{Beta diversity group test}\label{fig:Beta-diversity-group-test}
\end{center}

`Taxonomy hierarchy' 数据已全部提供。

\textbf{(对应文件为 \texttt{Figure+Table/Taxonomy-hierarchy})}

\begin{center}\begin{tcolorbox}[colback=gray!10, colframe=gray!50, width=0.9\linewidth, arc=1mm, boxrule=0.5pt]注:文件夹Figure+Table/Taxonomy-hierarchy共包含6个文件。

\begin{enumerate}\tightlist
\item 1\_Phylum.pdf
\item 2\_Class.pdf
\item 3\_Order.pdf
\item 4\_Family.pdf
\item 5\_Genus.pdf
\item ...
\end{enumerate}\end{tcolorbox}
\end{center}

\hypertarget{ux5deeux5f02ux5206ux6790}{%
\paragraph{差异分析}\label{ux5deeux5f02ux5206ux6790}}

MicrobiotaProcess 的差异分析 (\texttt{MicrobiotaProcess::mp\_diff\_analysis}) 未发现差异菌,因此这里主要用的
\texttt{Qiime2} 的差异分析结果 (\texttt{accom\ test})。

注:关于 \texttt{ancom\ test} 的结果的解释,可以参考:

\begin{enumerate}
\def\labelenumi{\arabic{enumi}.}
\tightlist
\item
  \url{https://forum.qiime2.org/t/how-to-interpret-ancom-results/1958}
\item
  \url{https://forum.qiime2.org/t/specify-w-cutoff-for-anacom/1844}
\end{enumerate}

Figure \ref{fig:Ancom-test-group-level-2-volcano} (下方图) 为图Ancom test group level 2 volcano概览。

\textbf{(对应文件为 \texttt{Figure+Table/Ancom-test-group-level-2-volcano.pdf})}

\def\@captype{figure}
\begin{center}
\includegraphics[width = 0.9\linewidth]{Figure+Table/Ancom-test-group-level-2-volcano.pdf}
\caption{Ancom test group level 2 volcano}\label{fig:Ancom-test-group-level-2-volcano}
\end{center}

Figure \ref{fig:Ancom-test-group-level-2-Percentile-abundance} (下方图) 为图Ancom test group level 2 Percentile abundance概览。

\textbf{(对应文件为 \texttt{Figure+Table/Ancom-test-group-level-2-Percentile-abundance.pdf})}

\def\@captype{figure}
\begin{center}
\includegraphics[width = 0.9\linewidth]{Figure+Table/Ancom-test-group-level-2-Percentile-abundance.pdf}
\caption{Ancom test group level 2 Percentile abundance}\label{fig:Ancom-test-group-level-2-Percentile-abundance}
\end{center}

Figure \ref{fig:Ancom-test-group-level-6-volcano} (下方图) 为图Ancom test group level 6 volcano概览。

\textbf{(对应文件为 \texttt{Figure+Table/Ancom-test-group-level-6-volcano.pdf})}

\def\@captype{figure}
\begin{center}
\includegraphics[width = 0.9\linewidth]{Figure+Table/Ancom-test-group-level-6-volcano.pdf}
\caption{Ancom test group level 6 volcano}\label{fig:Ancom-test-group-level-6-volcano}
\end{center}

Figure \ref{fig:Ancom-test-group-level-6-Percentile-abundance} (下方图) 为图Ancom test group level 6 Percentile abundance概览。

\textbf{(对应文件为 \texttt{Figure+Table/Ancom-test-group-level-6-Percentile-abundance.pdf})}

\def\@captype{figure}
\begin{center}
\includegraphics[width = 0.9\linewidth]{Figure+Table/Ancom-test-group-level-6-Percentile-abundance.pdf}
\caption{Ancom test group level 6 Percentile abundance}\label{fig:Ancom-test-group-level-6-Percentile-abundance}
\end{center}

`level 2' 对应 Ontology 中的 Phylum。
`level 6' 对应 Ontology 中的 Species。

其余结果的可视化见:

`Ancom test visualization' 数据已全部提供。

\textbf{(对应文件为 \texttt{Figure+Table/Ancom-test-visualization})}

\begin{center}\begin{tcolorbox}[colback=gray!10, colframe=gray!50, width=0.9\linewidth, arc=1mm, boxrule=0.5pt]注:文件夹Figure+Table/Ancom-test-visualization共包含3个文件。

\begin{enumerate}\tightlist
\item 1\_ancom\_test\_group\_level\_4.pdf
\item 2\_ancom\_test\_group\_level\_5.pdf
\item 3\_ancom\_test\_group\_level\_6.pdf
\end{enumerate}\end{tcolorbox}
\end{center}

`Ancom test Percentile abundance' 数据已全部提供。

\textbf{(对应文件为 \texttt{Figure+Table/Ancom-test-Percentile-abundance})}

\begin{center}\begin{tcolorbox}[colback=gray!10, colframe=gray!50, width=0.9\linewidth, arc=1mm, boxrule=0.5pt]注:文件夹Figure+Table/Ancom-test-Percentile-abundance共包含5个文件。

\begin{enumerate}\tightlist
\item 1\_ancom\_test\_group\_level\_2.pdf
\item 2\_ancom\_test\_group\_level\_3.pdf
\item 3\_ancom\_test\_group\_level\_4.pdf
\item 4\_ancom\_test\_group\_level\_5.pdf
\item 5\_ancom\_test\_group\_level\_6.pdf
\end{enumerate}\end{tcolorbox}
\end{center}

`Ancom test results' 数据已全部提供。

\textbf{(对应文件为 \texttt{Figure+Table/Ancom-test-results})}

\begin{center}\begin{tcolorbox}[colback=gray!10, colframe=gray!50, width=0.9\linewidth, arc=1mm, boxrule=0.5pt]注:文件夹Figure+Table/Ancom-test-results共包含3个文件。

\begin{enumerate}\tightlist
\item 1\_ancom\_test\_group\_level\_4.csv
\item 2\_ancom\_test\_group\_level\_5.csv
\item 3\_ancom\_test\_group\_level\_6.csv
\end{enumerate}\end{tcolorbox}
\end{center}

\hypertarget{ux5deeux5f02ux83ccux5173ux8054ux5230ux4ee3ux8c22ux7269}{%
\subsubsection{差异菌关联到代谢物}\label{ux5deeux5f02ux83ccux5173ux8054ux5230ux4ee3ux8c22ux7269}}

\hypertarget{ux4ece-gutmdisorder-ux6570ux636eux5e93ux68c0ux7d22ux5173ux8054ux4ee3ux8c22ux7269}{%
\paragraph{从 gutMDisorder 数据库检索关联代谢物}\label{ux4ece-gutmdisorder-ux6570ux636eux5e93ux68c0ux7d22ux5173ux8054ux4ee3ux8c22ux7269}}

使用的数据库如下:

Table \ref{tab:GutMDisorder-database} (下方表格) 为表格GutMDisorder database概览。

\textbf{(对应文件为 \texttt{Figure+Table/GutMDisorder-database.xlsx})}

\begin{center}\begin{tcolorbox}[colback=gray!10, colframe=gray!50, width=0.9\linewidth, arc=1mm, boxrule=0.5pt]注:表格共有724行12列,以下预览的表格可能省略部分数据;表格含有289个唯一`Gut.Microbiota'。
\end{tcolorbox}
\end{center}

\begin{longtable}[]{@{}llllllll@{}}
\caption{\label{tab:GutMDisorder-database}GutMDisorder database}\tabularnewline
\toprule
Gut.Mi\ldots\ldots1 & Gut.Mi\ldots\ldots2 & Gut.Mi\ldots\ldots3 & Classi\ldots{} & Substrate & Substr\ldots\ldots6 & Substr\ldots\ldots7 & \ldots{}\tabularnewline
\midrule
\endfirsthead
\toprule
Gut.Mi\ldots\ldots1 & Gut.Mi\ldots\ldots2 & Gut.Mi\ldots\ldots3 & Classi\ldots{} & Substrate & Substr\ldots\ldots6 & Substr\ldots\ldots7 & \ldots{}\tabularnewline
\midrule
\endhead
Christ\ldots{} & NA & gm0883 & strain & D-Glucose & 5793 & HMDB00\ldots{} & \ldots{}\tabularnewline
Christ\ldots{} & NA & gm0883 & strain & Salicin & 439503 & HMDB00\ldots{} & \ldots{}\tabularnewline
Christ\ldots{} & NA & gm0883 & strain & D-Xylose & 135191 & HMDB00\ldots{} & \ldots{}\tabularnewline
Christ\ldots{} & NA & gm0883 & strain & L-Arab\ldots{} & 439195 & HMDB00\ldots{} & \ldots{}\tabularnewline
Christ\ldots{} & NA & gm0883 & strain & L-Rham\ldots{} & 25310 & HMDB00\ldots{} & \ldots{}\tabularnewline
Christ\ldots{} & NA & gm0883 & strain & D-Mannose & 18950 & HMDB00\ldots{} & \ldots{}\tabularnewline
Christ\ldots{} & NA & gm0883 & strain & D-Glucose & 5793 & HMDB00\ldots{} & \ldots{}\tabularnewline
Christ\ldots{} & NA & gm0883 & strain & Salicin & 439503 & HMDB00\ldots{} & \ldots{}\tabularnewline
Christ\ldots{} & NA & gm0883 & strain & D-Xylose & 135191 & HMDB00\ldots{} & \ldots{}\tabularnewline
Christ\ldots{} & NA & gm0883 & strain & L-Arab\ldots{} & 439195 & HMDB00\ldots{} & \ldots{}\tabularnewline
Christ\ldots{} & NA & gm0883 & strain & L-Rham\ldots{} & 25310 & HMDB00\ldots{} & \ldots{}\tabularnewline
Christ\ldots{} & NA & gm0883 & strain & D-Mannose & 18950 & HMDB00\ldots{} & \ldots{}\tabularnewline
Entero\ldots{} & 1343173 & gm0884 & species & Orientin & 5281675 & HMDB00\ldots{} & \ldots{}\tabularnewline
Clostr\ldots{} & 29347 & gm0885 & strain & Bile acid & 439520 & & \ldots{}\tabularnewline
Clostr\ldots{} & 29347 & gm0885 & strain & Cholic\ldots{} & 221493 & HMDB00\ldots{} & \ldots{}\tabularnewline
\ldots{} & \ldots{} & \ldots{} & \ldots{} & \ldots{} & \ldots{} & \ldots{} & \ldots{}\tabularnewline
\bottomrule
\end{longtable}

使用差异肠道菌匹配:

\begin{center}\begin{tcolorbox}[colback=gray!10, colframe=gray!50, width=0.9\linewidth, arc=1mm, boxrule=0.5pt]
\textbf{
Content
:}

\vspace{0.5em}

    Proteobacteria, Alphaproteobacteria, Rhizobiales,
Beijerinckiaceae, Methylobacterium-Methylorubrum

\vspace{2em}
\end{tcolorbox}
\end{center}

未找到相关代谢物。

\hypertarget{mr-match}{%
\paragraph{尝试从已发表研究 (孟德尔随机化相关) 中寻找关联代谢物}\label{mr-match}}

请参考\textsuperscript{\protect\hyperlink{ref-MendelianRandoLiuX2022}{1}}

匹配到 Phylum 水平的菌群关联的代谢物:

\begin{center}\begin{tcolorbox}[colback=gray!10, colframe=gray!50, width=0.9\linewidth, arc=1mm, boxrule=0.5pt]
\textbf{
Content
:}

\vspace{0.5em}

    5-methyltetrahydrofolic acid, selenium, Cystine,
Glutamic acid

\vspace{2em}
\end{tcolorbox}
\end{center}

Table \ref{tab:MendelianRandoLiuX2022-matched-data} (下方表格) 为表格MendelianRandoLiuX2022 matched data概览。

\textbf{(对应文件为 \texttt{Figure+Table/MendelianRandoLiuX2022-matched-data.csv})}

\begin{center}\begin{tcolorbox}[colback=gray!10, colframe=gray!50, width=0.9\linewidth, arc=1mm, boxrule=0.5pt]注:表格共有4行25列,以下预览的表格可能省略部分数据;表格含有1个唯一`X1'。
\end{tcolorbox}
\end{center}

\begin{longtable}[]{@{}llllllllll@{}}
\caption{\label{tab:MendelianRandoLiuX2022-matched-data}MendelianRandoLiuX2022 matched data}\tabularnewline
\toprule
\begin{minipage}[b]{0.07\columnwidth}\raggedright
X1\strut
\end{minipage} & \begin{minipage}[b]{0.07\columnwidth}\raggedright
X2\strut
\end{minipage} & \begin{minipage}[b]{0.07\columnwidth}\raggedright
beta\strut
\end{minipage} & \begin{minipage}[b]{0.07\columnwidth}\raggedright
se\strut
\end{minipage} & \begin{minipage}[b]{0.07\columnwidth}\raggedright
p\strut
\end{minipage} & \begin{minipage}[b]{0.07\columnwidth}\raggedright
beta.1\strut
\end{minipage} & \begin{minipage}[b]{0.07\columnwidth}\raggedright
se.1\strut
\end{minipage} & \begin{minipage}[b]{0.07\columnwidth}\raggedright
p.1\strut
\end{minipage} & \begin{minipage}[b]{0.07\columnwidth}\raggedright
p\_MRPR\ldots{}\strut
\end{minipage} & \begin{minipage}[b]{0.07\columnwidth}\raggedright
beta.2\strut
\end{minipage}\tabularnewline
\midrule
\endfirsthead
\toprule
\begin{minipage}[b]{0.07\columnwidth}\raggedright
X1\strut
\end{minipage} & \begin{minipage}[b]{0.07\columnwidth}\raggedright
X2\strut
\end{minipage} & \begin{minipage}[b]{0.07\columnwidth}\raggedright
beta\strut
\end{minipage} & \begin{minipage}[b]{0.07\columnwidth}\raggedright
se\strut
\end{minipage} & \begin{minipage}[b]{0.07\columnwidth}\raggedright
p\strut
\end{minipage} & \begin{minipage}[b]{0.07\columnwidth}\raggedright
beta.1\strut
\end{minipage} & \begin{minipage}[b]{0.07\columnwidth}\raggedright
se.1\strut
\end{minipage} & \begin{minipage}[b]{0.07\columnwidth}\raggedright
p.1\strut
\end{minipage} & \begin{minipage}[b]{0.07\columnwidth}\raggedright
p\_MRPR\ldots{}\strut
\end{minipage} & \begin{minipage}[b]{0.07\columnwidth}\raggedright
beta.2\strut
\end{minipage}\tabularnewline
\midrule
\endhead
\begin{minipage}[t]{0.07\columnwidth}\raggedright
p\_Prot\ldots{}\strut
\end{minipage} & \begin{minipage}[t]{0.07\columnwidth}\raggedright
5-meth\ldots{}\strut
\end{minipage} & \begin{minipage}[t]{0.07\columnwidth}\raggedright
-0.15312\strut
\end{minipage} & \begin{minipage}[t]{0.07\columnwidth}\raggedright
0.0313679\strut
\end{minipage} & \begin{minipage}[t]{0.07\columnwidth}\raggedright
1.0532\ldots{}\strut
\end{minipage} & \begin{minipage}[t]{0.07\columnwidth}\raggedright
-0.097\ldots{}\strut
\end{minipage} & \begin{minipage}[t]{0.07\columnwidth}\raggedright
0.0417018\strut
\end{minipage} & \begin{minipage}[t]{0.07\columnwidth}\raggedright
0.0188806\strut
\end{minipage} & \begin{minipage}[t]{0.07\columnwidth}\raggedright
0.1756\strut
\end{minipage} & \begin{minipage}[t]{0.07\columnwidth}\raggedright
-0.166\ldots{}\strut
\end{minipage}\tabularnewline
\begin{minipage}[t]{0.07\columnwidth}\raggedright
p\_Prot\ldots{}\strut
\end{minipage} & \begin{minipage}[t]{0.07\columnwidth}\raggedright
selenium\strut
\end{minipage} & \begin{minipage}[t]{0.07\columnwidth}\raggedright
-0.122431\strut
\end{minipage} & \begin{minipage}[t]{0.07\columnwidth}\raggedright
0.0289698\strut
\end{minipage} & \begin{minipage}[t]{0.07\columnwidth}\raggedright
2.3772\ldots{}\strut
\end{minipage} & \begin{minipage}[t]{0.07\columnwidth}\raggedright
-0.032\ldots{}\strut
\end{minipage} & \begin{minipage}[t]{0.07\columnwidth}\raggedright
0.0412256\strut
\end{minipage} & \begin{minipage}[t]{0.07\columnwidth}\raggedright
0.431953\strut
\end{minipage} & \begin{minipage}[t]{0.07\columnwidth}\raggedright
0.189\strut
\end{minipage} & \begin{minipage}[t]{0.07\columnwidth}\raggedright
-0.192\ldots{}\strut
\end{minipage}\tabularnewline
\begin{minipage}[t]{0.07\columnwidth}\raggedright
p\_Prot\ldots{}\strut
\end{minipage} & \begin{minipage}[t]{0.07\columnwidth}\raggedright
Cystine\strut
\end{minipage} & \begin{minipage}[t]{0.07\columnwidth}\raggedright
-0.097\ldots{}\strut
\end{minipage} & \begin{minipage}[t]{0.07\columnwidth}\raggedright
0.0221977\strut
\end{minipage} & \begin{minipage}[t]{0.07\columnwidth}\raggedright
1.0141\ldots{}\strut
\end{minipage} & \begin{minipage}[t]{0.07\columnwidth}\raggedright
-0.045\ldots{}\strut
\end{minipage} & \begin{minipage}[t]{0.07\columnwidth}\raggedright
0.0323047\strut
\end{minipage} & \begin{minipage}[t]{0.07\columnwidth}\raggedright
0.16131\strut
\end{minipage} & \begin{minipage}[t]{0.07\columnwidth}\raggedright
0.4752\strut
\end{minipage} & \begin{minipage}[t]{0.07\columnwidth}\raggedright
-0.101\ldots{}\strut
\end{minipage}\tabularnewline
\begin{minipage}[t]{0.07\columnwidth}\raggedright
p\_Prot\ldots{}\strut
\end{minipage} & \begin{minipage}[t]{0.07\columnwidth}\raggedright
Glutam\ldots{}\strut
\end{minipage} & \begin{minipage}[t]{0.07\columnwidth}\raggedright
0.175275\strut
\end{minipage} & \begin{minipage}[t]{0.07\columnwidth}\raggedright
0.0392992\strut
\end{minipage} & \begin{minipage}[t]{0.07\columnwidth}\raggedright
8.1952\ldots{}\strut
\end{minipage} & \begin{minipage}[t]{0.07\columnwidth}\raggedright
0.0363522\strut
\end{minipage} & \begin{minipage}[t]{0.07\columnwidth}\raggedright
0.0368746\strut
\end{minipage} & \begin{minipage}[t]{0.07\columnwidth}\raggedright
0.324216\strut
\end{minipage} & \begin{minipage}[t]{0.07\columnwidth}\raggedright
0.0082\strut
\end{minipage} & \begin{minipage}[t]{0.07\columnwidth}\raggedright
0.1929\ldots{}\strut
\end{minipage}\tabularnewline
\bottomrule
\end{longtable}

\hypertarget{ux4ee3ux8c22ux7269ux7684ux5bccux96c6ux5206ux6790}{%
\subsubsection{代谢物的富集分析}\label{ux4ee3ux8c22ux7269ux7684ux5bccux96c6ux5206ux6790}}

将匹配到的代谢物 (\ref{mr-match}) 进行代谢物富集分析。

以下是代谢物的数据库匹配:

Table \ref{tab:compounds-ID} (下方表格) 为表格compounds ID概览。

\textbf{(对应文件为 \texttt{Figure+Table/compounds-ID.csv})}

\begin{center}\begin{tcolorbox}[colback=gray!10, colframe=gray!50, width=0.9\linewidth, arc=1mm, boxrule=0.5pt]注:表格共有4行7列,以下预览的表格可能省略部分数据;表格含有4个唯一`Query'。
\end{tcolorbox}
\end{center}

\begin{longtable}[]{@{}lllllll@{}}
\caption{\label{tab:compounds-ID}Compounds ID}\tabularnewline
\toprule
Query & Match & HMDB & PubChem & KEGG & SMILES & Comment\tabularnewline
\midrule
\endfirsthead
\toprule
Query & Match & HMDB & PubChem & KEGG & SMILES & Comment\tabularnewline
\midrule
\endhead
5-methylte\ldots{} & 5-Methylte\ldots{} & HMDB0001396 & 439234 & C00440 & CN1C(CNC2=\ldots{} & 1\tabularnewline
selenium & Selenium & HMDB0001349 & NA & C01529 & {[}Se++{]} & 1\tabularnewline
L-cystine & L-Cystine & HMDB0000192 & 67678 & C00491 & N{[}\href{mailto:C@@H}{\nolinkurl{C@@H}}{]}(CS\ldots{} & 1\tabularnewline
Glutamic acid & Glutamic acid & HMDB0000148 & 33032 & C00302 & N{[}\href{mailto:C@@H}{\nolinkurl{C@@H}}{]}(CC\ldots{} & 1\tabularnewline
\bottomrule
\end{longtable}

Figure \ref{fig:MetaboAnalyst-kegg-enrichment} (下方图) 为图MetaboAnalyst kegg enrichment概览。

\textbf{(对应文件为 \texttt{Figure+Table/metabolites\_ORA\_dot\_kegg\_pathway\_dpi72.pdf})}

\def\@captype{figure}
\begin{center}
\includegraphics[width = 0.9\linewidth]{figs/metabolites_ORA_dot_kegg_pathway_dpi72.pdf}
\caption{MetaboAnalyst kegg enrichment}\label{fig:MetaboAnalyst-kegg-enrichment}
\end{center}

Figure \ref{fig:Enrichment-with-algorithm-PageRank} (下方图) 为图Enrichment with algorithm PageRank概览。

\textbf{(对应文件为 \texttt{Figure+Table/Enrichment-with-algorithm-PageRank.pdf})}

\def\@captype{figure}
\begin{center}
\includegraphics[width = 0.9\linewidth]{Figure+Table/Enrichment-with-algorithm-PageRank.pdf}
\caption{Enrichment with algorithm PageRank}\label{fig:Enrichment-with-algorithm-PageRank}
\end{center}

Table \ref{tab:Data-of-enrichment-with-algorithm-PageRank} (下方表格) 为表格Data of enrichment with algorithm PageRank概览。

\textbf{(对应文件为 \texttt{Figure+Table/Data-of-enrichment-with-algorithm-PageRank.xlsx})}

\begin{center}\begin{tcolorbox}[colback=gray!10, colframe=gray!50, width=0.9\linewidth, arc=1mm, boxrule=0.5pt]注:表格共有39行7列,以下预览的表格可能省略部分数据;表格含有39个唯一`name'。
\end{tcolorbox}
\end{center}

\begin{longtable}[]{@{}lllllll@{}}
\caption{\label{tab:Data-of-enrichment-with-algorithm-PageRank}Data of enrichment with algorithm PageRank}\tabularnewline
\toprule
name & com & NAME & label & input & abbrev.name & type\tabularnewline
\midrule
\endfirsthead
\toprule
name & com & NAME & label & input & abbrev.name & type\tabularnewline
\midrule
\endhead
hsa00220 & 1 & Arginine b\ldots{} & Arginine b\ldots{} & Others & Arginine b\ldots{} & Pathway\tabularnewline
hsa00250 & 1 & Alanine, a\ldots{} & Alanine, a\ldots{} & Others & Alanine, a\ldots{} & Pathway\tabularnewline
hsa00270 & 1 & Cysteine a\ldots{} & Cysteine a\ldots{} & Others & Cysteine a\ldots{} & Pathway\tabularnewline
hsa00380 & 1 & Tryptophan\ldots{} & Tryptophan\ldots{} & Others & Tryptophan\ldots{} & Pathway\tabularnewline
hsa00450 & 1 & Selenocomp\ldots{} & Selenocomp\ldots{} & Others & Selenocomp\ldots{} & Pathway\tabularnewline
hsa00670 & 1 & One carbon\ldots{} & One carbon\ldots{} & Others & One carbon\ldots{} & Pathway\tabularnewline
hsa01200 & 1 & Carbon met\ldots{} & Carbon met\ldots{} & Others & Carbon met\ldots{} & Pathway\tabularnewline
hsa01523 & 1 & Antifolate\ldots{} & Antifolate\ldots{} & Others & Antifolate\ldots{} & Pathway\tabularnewline
hsa02010 & 1 & ABC transp\ldots{} & ABC transp\ldots{} & Others & ABC transp\ldots{} & Pathway\tabularnewline
hsa04216 & 1 & Ferroptosi\ldots{} & Ferroptosi\ldots{} & Others & Ferroptosi\ldots{} & Pathway\tabularnewline
hsa04974 & 1 & Protein di\ldots{} & Protein di\ldots{} & Others & Protein di\ldots{} & Pathway\tabularnewline
hsa04975 & 1 & Fat digest\ldots{} & Fat digest\ldots{} & Others & Fat digest\ldots{} & Pathway\tabularnewline
M00017 & 2 & Methionine\ldots{} & Methionine\ldots{} & Others & Methionine\ldots{} & Module\tabularnewline
M00170 & 2 & C4-dicarbo\ldots{} & C4-dicarbo\ldots{} & Others & C4-dicarbo\ldots{} & Module\tabularnewline
M00171 & 2 & C4-dicarbo\ldots{} & C4-dicarbo\ldots{} & Others & C4-dicarbo\ldots{} & Module\tabularnewline
\ldots{} & \ldots{} & \ldots{} & \ldots{} & \ldots{} & \ldots{} & \ldots{}\tabularnewline
\bottomrule
\end{longtable}

\hypertarget{valids}{%
\subsubsection{从结肠炎或结肠癌已发表的代谢物研究中验证}\label{valids}}

\hypertarget{depressionandyuan2021-ux7ed3ux80a0ux708e-ux80a0ux9053ux83cc}{%
\paragraph{DepressionAndYuan2021 结肠炎 (肠道菌)}\label{depressionandyuan2021-ux7ed3ux80a0ux708e-ux80a0ux9053ux83cc}}

Depression and anxiety in patients with active ulcerative colitis: crosstalk of gut microbiota, metabolomics and proteomics\textsuperscript{\protect\hyperlink{ref-DepressionAndYuan2021}{13}}

以下是整理自该文献的差异肠道菌汇总:

Table \ref{tab:DepressionAndYuan2021-published-data-significant-microbiota} (下方表格) 为表格DepressionAndYuan2021 published data significant microbiota概览。

\textbf{(对应文件为 \texttt{Figure+Table/DepressionAndYuan2021-published-data-significant-microbiota.xlsx})}

\begin{center}\begin{tcolorbox}[colback=gray!10, colframe=gray!50, width=0.9\linewidth, arc=1mm, boxrule=0.5pt]注:表格共有91行4列,以下预览的表格可能省略部分数据;表格含有91个唯一`Taxonomy'。
\end{tcolorbox}
\end{center}

\begin{longtable}[]{@{}llll@{}}
\caption{\label{tab:DepressionAndYuan2021-published-data-significant-microbiota}DepressionAndYuan2021 published data significant microbiota}\tabularnewline
\toprule
\begin{minipage}[b]{0.22\columnwidth}\raggedright
Taxonomy\strut
\end{minipage} & \begin{minipage}[b]{0.22\columnwidth}\raggedright
p.value\strut
\end{minipage} & \begin{minipage}[b]{0.22\columnwidth}\raggedright
MRA.in.UC\strut
\end{minipage} & \begin{minipage}[b]{0.22\columnwidth}\raggedright
MRA.in.HC\strut
\end{minipage}\tabularnewline
\midrule
\endfirsthead
\toprule
\begin{minipage}[b]{0.22\columnwidth}\raggedright
Taxonomy\strut
\end{minipage} & \begin{minipage}[b]{0.22\columnwidth}\raggedright
p.value\strut
\end{minipage} & \begin{minipage}[b]{0.22\columnwidth}\raggedright
MRA.in.UC\strut
\end{minipage} & \begin{minipage}[b]{0.22\columnwidth}\raggedright
MRA.in.HC\strut
\end{minipage}\tabularnewline
\midrule
\endhead
\begin{minipage}[t]{0.22\columnwidth}\raggedright
p\_\_Gemmatimonadetes\strut
\end{minipage} & \begin{minipage}[t]{0.22\columnwidth}\raggedright
0.00049553252446827\strut
\end{minipage} & \begin{minipage}[t]{0.22\columnwidth}\raggedright
0.00171119907683608\strut
\end{minipage} & \begin{minipage}[t]{0.22\columnwidth}\raggedright
0.000160190585722501\strut
\end{minipage}\tabularnewline
\begin{minipage}[t]{0.22\columnwidth}\raggedright
p\_\_Actinobacteria\strut
\end{minipage} & \begin{minipage}[t]{0.22\columnwidth}\raggedright
0.00834784591709094\strut
\end{minipage} & \begin{minipage}[t]{0.22\columnwidth}\raggedright
0.0617033057472256\strut
\end{minipage} & \begin{minipage}[t]{0.22\columnwidth}\raggedright
0.0389112516772091\strut
\end{minipage}\tabularnewline
\begin{minipage}[t]{0.22\columnwidth}\raggedright
p\_\_Firmicutes\strut
\end{minipage} & \begin{minipage}[t]{0.22\columnwidth}\raggedright
0.00478686441960355\strut
\end{minipage} & \begin{minipage}[t]{0.22\columnwidth}\raggedright
0.589610871565727\strut
\end{minipage} & \begin{minipage}[t]{0.22\columnwidth}\raggedright
0.499806265231797\strut
\end{minipage}\tabularnewline
\begin{minipage}[t]{0.22\columnwidth}\raggedright
p\_\_Bacteroidetes\strut
\end{minipage} & \begin{minipage}[t]{0.22\columnwidth}\raggedright
9.60337999115942e-05\strut
\end{minipage} & \begin{minipage}[t]{0.22\columnwidth}\raggedright
0.250752809840626\strut
\end{minipage} & \begin{minipage}[t]{0.22\columnwidth}\raggedright
0.403687121772228\strut
\end{minipage}\tabularnewline
\begin{minipage}[t]{0.22\columnwidth}\raggedright
p\_\_unidentified\strut
\end{minipage} & \begin{minipage}[t]{0.22\columnwidth}\raggedright
1.40986578996555e-07\strut
\end{minipage} & \begin{minipage}[t]{0.22\columnwidth}\raggedright
0.000640301130981834\strut
\end{minipage} & \begin{minipage}[t]{0.22\columnwidth}\raggedright
2.05372545798078e-06\strut
\end{minipage}\tabularnewline
\begin{minipage}[t]{0.22\columnwidth}\raggedright
c\_\_Actinobacteria\strut
\end{minipage} & \begin{minipage}[t]{0.22\columnwidth}\raggedright
0.00353055167754535\strut
\end{minipage} & \begin{minipage}[t]{0.22\columnwidth}\raggedright
0.046725454341533\strut
\end{minipage} & \begin{minipage}[t]{0.22\columnwidth}\raggedright
0.0237116295626934\strut
\end{minipage}\tabularnewline
\begin{minipage}[t]{0.22\columnwidth}\raggedright
c\_\_Longimicrobia\strut
\end{minipage} & \begin{minipage}[t]{0.22\columnwidth}\raggedright
7.16632518063251e-07\strut
\end{minipage} & \begin{minipage}[t]{0.22\columnwidth}\raggedright
0.00145496100157866\strut
\end{minipage} & \begin{minipage}[t]{0.22\columnwidth}\raggedright
0\strut
\end{minipage}\tabularnewline
\begin{minipage}[t]{0.22\columnwidth}\raggedright
c\_\_Bacteroidia\strut
\end{minipage} & \begin{minipage}[t]{0.22\columnwidth}\raggedright
6.38198131772293e-05\strut
\end{minipage} & \begin{minipage}[t]{0.22\columnwidth}\raggedright
0.24518920382048\strut
\end{minipage} & \begin{minipage}[t]{0.22\columnwidth}\raggedright
0.403287329883075\strut
\end{minipage}\tabularnewline
\begin{minipage}[t]{0.22\columnwidth}\raggedright
c\_\_Deinococci\strut
\end{minipage} & \begin{minipage}[t]{0.22\columnwidth}\raggedright
3.17450175630301e-06\strut
\end{minipage} & \begin{minipage}[t]{0.22\columnwidth}\raggedright
0.00131123215732429\strut
\end{minipage} & \begin{minipage}[t]{0.22\columnwidth}\raggedright
0\strut
\end{minipage}\tabularnewline
\begin{minipage}[t]{0.22\columnwidth}\raggedright
c\_\_Bacilli\strut
\end{minipage} & \begin{minipage}[t]{0.22\columnwidth}\raggedright
0.00330587139663399\strut
\end{minipage} & \begin{minipage}[t]{0.22\columnwidth}\raggedright
0.0649165901033199\strut
\end{minipage} & \begin{minipage}[t]{0.22\columnwidth}\raggedright
0.0154563377967633\strut
\end{minipage}\tabularnewline
\begin{minipage}[t]{0.22\columnwidth}\raggedright
c\_\_Cytophagia\strut
\end{minipage} & \begin{minipage}[t]{0.22\columnwidth}\raggedright
2.10636789003367e-05\strut
\end{minipage} & \begin{minipage}[t]{0.22\columnwidth}\raggedright
0.0016664310495747\strut
\end{minipage} & \begin{minipage}[t]{0.22\columnwidth}\raggedright
4.24436594649361e-05\strut
\end{minipage}\tabularnewline
\begin{minipage}[t]{0.22\columnwidth}\raggedright
c\_\_unidentified\strut
\end{minipage} & \begin{minipage}[t]{0.22\columnwidth}\raggedright
2.64154914758447e-08\strut
\end{minipage} & \begin{minipage}[t]{0.22\columnwidth}\raggedright
0.0016652538833675\strut
\end{minipage} & \begin{minipage}[t]{0.22\columnwidth}\raggedright
8.83101946931736e-05\strut
\end{minipage}\tabularnewline
\begin{minipage}[t]{0.22\columnwidth}\raggedright
c\_\_Flavobacteriia\strut
\end{minipage} & \begin{minipage}[t]{0.22\columnwidth}\raggedright
5.26714172360768e-05\strut
\end{minipage} & \begin{minipage}[t]{0.22\columnwidth}\raggedright
0.00167703418517471\strut
\end{minipage} & \begin{minipage}[t]{0.22\columnwidth}\raggedright
5.95580382814426e-05\strut
\end{minipage}\tabularnewline
\begin{minipage}[t]{0.22\columnwidth}\raggedright
o\_\_Oceanospirillales\strut
\end{minipage} & \begin{minipage}[t]{0.22\columnwidth}\raggedright
0.00495867783478051\strut
\end{minipage} & \begin{minipage}[t]{0.22\columnwidth}\raggedright
0.000461817587592187\strut
\end{minipage} & \begin{minipage}[t]{0.22\columnwidth}\raggedright
2.05372545798077e-06\strut
\end{minipage}\tabularnewline
\begin{minipage}[t]{0.22\columnwidth}\raggedright
o\_\_Bifidobacteriales\strut
\end{minipage} & \begin{minipage}[t]{0.22\columnwidth}\raggedright
0.00354779153342515\strut
\end{minipage} & \begin{minipage}[t]{0.22\columnwidth}\raggedright
0.0443044462799416\strut
\end{minipage} & \begin{minipage}[t]{0.22\columnwidth}\raggedright
0.0224465346805772\strut
\end{minipage}\tabularnewline
\begin{minipage}[t]{0.22\columnwidth}\raggedright
\ldots{}\strut
\end{minipage} & \begin{minipage}[t]{0.22\columnwidth}\raggedright
\ldots{}\strut
\end{minipage} & \begin{minipage}[t]{0.22\columnwidth}\raggedright
\ldots{}\strut
\end{minipage} & \begin{minipage}[t]{0.22\columnwidth}\raggedright
\ldots{}\strut
\end{minipage}\tabularnewline
\bottomrule
\end{longtable}

未从 Fig. \ref{tab:DepressionAndYuan2021-published-data-significant-microbiota} 中匹配到
客户数据筛选出的肠道菌。

\hypertarget{alterationsinscovil2018-ux7ed3ux80a0ux708e-ux4ee3ux8c22ux7269}{%
\paragraph{AlterationsInScovil2018 结肠炎 (代谢物)}\label{alterationsinscovil2018-ux7ed3ux80a0ux708e-ux4ee3ux8c22ux7269}}

Alterations in Lipid, Amino Acid, and Energy Metabolism Distinguish Crohn's Disease from Ulcerative Colitis and Control Subjects by Serum Metabolomic Profiling.\textsuperscript{\protect\hyperlink{ref-AlterationsInScovil2018}{14}}

以下是整理自该文献的代谢物汇总:

Table \ref{tab:AlterationsInScovil2018-published-data-metabolites} (下方表格) 为表格AlterationsInScovil2018 published data metabolites概览。

\textbf{(对应文件为 \texttt{Figure+Table/AlterationsInScovil2018-published-data-metabolites.xlsx})}

\begin{center}\begin{tcolorbox}[colback=gray!10, colframe=gray!50, width=0.9\linewidth, arc=1mm, boxrule=0.5pt]注:表格共有565行22列,以下预览的表格可能省略部分数据;表格含有565个唯一`Metabolite'。
\end{tcolorbox}
\end{center}

\begin{longtable}[]{@{}llllllllll@{}}
\caption{\label{tab:AlterationsInScovil2018-published-data-metabolites}AlterationsInScovil2018 published data metabolites}\tabularnewline
\toprule
Metabo\ldots{} & Pathway & Sub-Pa\ldots{} & Platform & RI & Mass & MSI Id\ldots{} & PUBCHEM & KEGG & HMDB\tabularnewline
\midrule
\endfirsthead
\toprule
Metabo\ldots{} & Pathway & Sub-Pa\ldots{} & Platform & RI & Mass & MSI Id\ldots{} & PUBCHEM & KEGG & HMDB\tabularnewline
\midrule
\endhead
alanine & Amino \ldots{} & Alanin\ldots{} & LC/MS \ldots{} & 2780.3 & 88.0404 & 1 & 5950 & C00041 & HMDB00161\tabularnewline
aspara\ldots{} & Amino \ldots{} & Alanin\ldots{} & LC/MS \ldots{} & 2951.1 & 131.0462 & 1 & 6267 & C00152 & HMDB00168\tabularnewline
aspartate & Amino \ldots{} & Alanin\ldots{} & LC/MS Neg & 640 & 132.0302 & 1 & 5960 & C00049 & HMDB00191\tabularnewline
N-acet\ldots{} & Amino \ldots{} & Alanin\ldots{} & LC/MS \ldots{} & 1564.8 & 130.051 & 1 & 88064 & C02847 & HMDB00766\tabularnewline
N-acet\ldots{} & Amino \ldots{} & Alanin\ldots{} & LC/MS \ldots{} & 785 & 175.0713 & 1 & 99715 & NA & HMDB06028\tabularnewline
N-acet\ldots{} & Amino \ldots{} & Alanin\ldots{} & LC/MS \ldots{} & 3143 & 174.0408 & 1 & 65065 & C01042 & HMDB00812\tabularnewline
creatine & Amino \ldots{} & Creati\ldots{} & LC/MS \ldots{} & 2920 & 130.0622 & 1 & 586 & C00300 & HMDB00064\tabularnewline
creati\ldots{} & Amino \ldots{} & Creati\ldots{} & LC/MS \ldots{} & 2055 & 114.0662 & 1 & 588 & C00791 & HMDB00562\tabularnewline
guanid\ldots{} & Amino \ldots{} & Creati\ldots{} & LC/MS \ldots{} & 2884 & 116.0466 & 1 & 763 & C00581 & HMDB00128\tabularnewline
glutamate & Amino \ldots{} & Glutam\ldots{} & LC/MS \ldots{} & 1500 & 148.0604 & 1 & 611 & C00025 & HMDB00148\tabularnewline
glutamine & Amino \ldots{} & Glutam\ldots{} & LC/MS \ldots{} & 1291 & 147.0764 & 1 & 5961 & C00064 & HMDB00641\tabularnewline
N-acet\ldots{} & Amino \ldots{} & Glutam\ldots{} & LC/MS \ldots{} & 1035 & 305.098 & 1 & 5255 & C12270 & HMDB01067\tabularnewline
N-acet\ldots{} & Amino \ldots{} & Glutam\ldots{} & LC/MS \ldots{} & 3106 & 188.0564 & 1 & 70914 & C00624 & HMDB01138\tabularnewline
N-acet\ldots{} & Amino \ldots{} & Glutam\ldots{} & LC/MS Neg & 771 & 187.0724 & 1 & 182230 & C02716 & HMDB06029\tabularnewline
pyrogl\ldots{} & Amino \ldots{} & Glutam\ldots{} & LC/MS \ldots{} & 1900 & 129.0659 & 2 & 134508 & NA & NA\tabularnewline
\ldots{} & \ldots{} & \ldots{} & \ldots{} & \ldots{} & \ldots{} & \ldots{} & \ldots{} & \ldots{} & \ldots{}\tabularnewline
\bottomrule
\end{longtable}

未从 Tab. \ref{tab:AlterationsInScovil2018-published-data-metabolites} 中匹配到
\ref{mr-match} 中的关联代谢物。

\hypertarget{valid}{%
\paragraph{LossOfSymbiotSadegh2024 结肠癌 (肠道菌)}\label{valid}}

Loss of symbiotic and increase of virulent bacteria through microbial networks
in Lynch syndrome colon carcinogenesis\textsuperscript{\protect\hyperlink{ref-LossOfSymbiotSadegh2024}{2}}

以下是整理自该文献的关联肠道菌汇总:

Table \ref{tab:LossOfSymbiotSadegh2024-published-data-microbiota} (下方表格) 为表格LossOfSymbiotSadegh2024 published data microbiota概览。

\textbf{(对应文件为 \texttt{Figure+Table/LossOfSymbiotSadegh2024-published-data-microbiota.csv})}

\begin{center}\begin{tcolorbox}[colback=gray!10, colframe=gray!50, width=0.9\linewidth, arc=1mm, boxrule=0.5pt]注:表格共有34行6列,以下预览的表格可能省略部分数据;表格含有34个唯一`taxon'。
\end{tcolorbox}
\end{center}

\begin{longtable}[]{@{}llllll@{}}
\caption{\label{tab:LossOfSymbiotSadegh2024-published-data-microbiota}LossOfSymbiotSadegh2024 published data microbiota}\tabularnewline
\toprule
taxon & proportion\ldots{} & position.i\ldots{} & mean.auc\ldots.. & sd.auc..test. & Value.with\ldots{}\tabularnewline
\midrule
\endfirsthead
\toprule
taxon & proportion\ldots{} & position.i\ldots{} & mean.auc\ldots.. & sd.auc..test. & Value.with\ldots{}\tabularnewline
\midrule
\endhead
k\_Bacteria\ldots{} & 0.992 & numerator & 1 & 0 & stool\tabularnewline
k\_Bacteria\ldots{} & 0.998 & numerator & 1 & 0 & stool\tabularnewline
k\_Bacteria\ldots{} & 0.956 & numerator & 1 & 0 & stool\tabularnewline
k\_Bacteria\ldots{} & 1 & numerator & 1 & 0 & stool\tabularnewline
k\_Bacteria\ldots{} & 0.982 & numerator & 1 & 0 & stool\tabularnewline
k\_Bacteria\ldots{} & 1 & numerator & 1 & 0 & stool\tabularnewline
k\_Bacteria\ldots{} & 1 & numerator & 1 & 0 & stool\tabularnewline
k\_Bacteria\ldots{} & 1 & numerator & 1 & 0 & stool\tabularnewline
k\_Bacteria\ldots{} & 1 & numerator & 1 & 0 & stool\tabularnewline
k\_Bacteria\ldots{} & 1 & numerator & 1 & 0 & stool\tabularnewline
k\_Bacteria\ldots{} & 0.996 & numerator & 1 & 0 & stool\tabularnewline
k\_Bacteria\ldots{} & 1 & numerator & 1 & 0 & stool\tabularnewline
k\_Bacteria\ldots{} & 1 & numerator & 1 & 0 & stool\tabularnewline
k\_Bacteria\ldots{} & 1 & numerator & 1 & 0 & stool\tabularnewline
k\_Bacteria\ldots{} & 1 & numerator & 1 & 0 & stool\tabularnewline
\ldots{} & \ldots{} & \ldots{} & \ldots{} & \ldots{} & \ldots{}\tabularnewline
\bottomrule
\end{longtable}

匹配到的肠道菌 (Phylum 水平):

Table \ref{tab:LossOfSymbiotSadegh2024-matched-Phylum-microbiota} (下方表格) 为表格LossOfSymbiotSadegh2024 matched Phylum microbiota概览。

\textbf{(对应文件为 \texttt{Figure+Table/LossOfSymbiotSadegh2024-matched-Phylum-microbiota.csv})}

\begin{center}\begin{tcolorbox}[colback=gray!10, colframe=gray!50, width=0.9\linewidth, arc=1mm, boxrule=0.5pt]注:表格共有3行6列,以下预览的表格可能省略部分数据;表格含有3个唯一`taxon'。
\end{tcolorbox}
\end{center}

\begin{longtable}[]{@{}llllll@{}}
\caption{\label{tab:LossOfSymbiotSadegh2024-matched-Phylum-microbiota}LossOfSymbiotSadegh2024 matched Phylum microbiota}\tabularnewline
\toprule
taxon & proportion\ldots{} & position.i\ldots{} & mean.auc\ldots.. & sd.auc..test. & Value.with\ldots{}\tabularnewline
\midrule
\endfirsthead
\toprule
taxon & proportion\ldots{} & position.i\ldots{} & mean.auc\ldots.. & sd.auc..test. & Value.with\ldots{}\tabularnewline
\midrule
\endhead
k\_Bacteria\ldots{} & 1 & denominator & 1 & 0 & stool\tabularnewline
k\_Bacteria\ldots{} & 1 & denominator & 1 & 0 & stool\tabularnewline
k\_Bacteria\ldots{} & 1 & denominator & 1 & 0 & stool\tabularnewline
\bottomrule
\end{longtable}

\hypertarget{integratedanalchen2022-ux7ed3ux80a0ux764c-ux80a0ux9053ux83ccux548cux4ee3ux8c22ux7269}{%
\paragraph{IntegratedAnalChen2022 结肠癌 (肠道菌和代谢物)}\label{integratedanalchen2022-ux7ed3ux80a0ux764c-ux80a0ux9053ux83ccux548cux4ee3ux8c22ux7269}}

Integrated analysis of the faecal metagenome and serum metabolome reveals the
role of gut microbiome-associated metabolites in the detection of colorectal
cancer and adenoma\textsuperscript{\protect\hyperlink{ref-IntegratedAnalChen2022}{15}}

以下是整理自该文献的肠道菌和代谢物数据 (PDF 识别结果):

Table \ref{tab:IntegratedAnalChen2022-published-data-microbiota} (下方表格) 为表格IntegratedAnalChen2022 published data microbiota概览。

\textbf{(对应文件为 \texttt{Figure+Table/IntegratedAnalChen2022-published-data-microbiota.csv})}

\begin{center}\begin{tcolorbox}[colback=gray!10, colframe=gray!50, width=0.9\linewidth, arc=1mm, boxrule=0.5pt]注:表格共有782行4列,以下预览的表格可能省略部分数据;表格含有3个唯一`Type'。
\end{tcolorbox}
\end{center}
\begin{center}\begin{tcolorbox}[colback=gray!10, colframe=gray!50, width=0.9\linewidth, arc=1mm, boxrule=0.5pt]\begin{enumerate}\tightlist
\item pvalue:  显著性 P。
\end{enumerate}\end{tcolorbox}
\end{center}

\begin{longtable}[]{@{}llll@{}}
\caption{\label{tab:IntegratedAnalChen2022-published-data-microbiota}IntegratedAnalChen2022 published data microbiota}\tabularnewline
\toprule
Type & Species & Metabolites & pvalue\tabularnewline
\midrule
\endfirsthead
\toprule
Type & Species & Metabolites & pvalue\tabularnewline
\midrule
\endhead
Tumor\_promoting\_b\ldots{} & Alistipes\_finegoldii & X23.4\_476.011mz\_pos & 3.87e-07\tabularnewline
Tumor\_promoting\_b\ldots{} & Alistipes\_finegoldii & X21.4\_494.68mz\_pos & 0.000232816\tabularnewline
Tumor\_promoting\_b\ldots{} & Alistipes\_finegoldii & X24.5\_504.692mz\_pos & 0.000162744\tabularnewline
Tumor\_promoting\_b\ldots{} & Alistipes\_finegoldii & X26.1\_509.03mz\_pos & 0.000793675\tabularnewline
Tumor\_promoting\_b\ldots{} & Alistipes\_finegoldii & X26.3\_514.705mz\_pos & 0.000881926\tabularnewline
Tumor\_promoting\_b\ldots{} & Bilophila\_wadswor\ldots{} & X23.4\_476.011mz\_pos & 4.45e-05\tabularnewline
Tumor\_promoting\_b\ldots{} & Fusobacterium\_nuc\ldots{} & X21.2\_512.336mz\_neg & 0.000581862\tabularnewline
Tumor\_promoting\_b\ldots{} & Fusobacterium\_nuc\ldots{} & X19.2\_536.299mz\_neg & 0.000431165\tabularnewline
Tumor\_promoting\_b\ldots{} & Fusobacterium\_sp\ldots. & X21.2\_512.336mz\_neg & 0.000850936\tabularnewline
Tumor\_promoting\_b\ldots{} & Fusobacterium\_sp\ldots. & X19.2\_536.299mz\_neg & 9.61e-05\tabularnewline
Tumor\_promoting\_b\ldots{} & Odoribacter\_splan\ldots{} & X23.4\_476.011mz\_pos & 1.93e-06\tabularnewline
Tumor\_promoting\_b\ldots{} & Odoribacter\_splan\ldots{} & X21.4\_494.68mz\_pos & 3.38e-05\tabularnewline
Tumor\_promoting\_b\ldots{} & Odoribacter\_splan\ldots{} & X24.8\_495.024mz\_pos & 0.000618015\tabularnewline
Tumor\_promoting\_b\ldots{} & Odoribacter\_splan\ldots{} & X22.9\_499.686mz\_pos & 0.000395014\tabularnewline
Tumor\_promoting\_b\ldots{} & Odoribacter\_splan\ldots{} & X24.5\_504.692mz\_pos & 0.000898428\tabularnewline
\ldots{} & \ldots{} & \ldots{} & \ldots{}\tabularnewline
\bottomrule
\end{longtable}

Table \ref{tab:IntegratedAnalChen2022-published-data-metabolites} (下方表格) 为表格IntegratedAnalChen2022 published data metabolites概览。

\textbf{(对应文件为 \texttt{Figure+Table/IntegratedAnalChen2022-published-data-metabolites.xlsx})}

\begin{center}\begin{tcolorbox}[colback=gray!10, colframe=gray!50, width=0.9\linewidth, arc=1mm, boxrule=0.5pt]注:表格共有969行19列,以下预览的表格可能省略部分数据;表格含有905个唯一`V1'。
\end{tcolorbox}
\end{center}

\begin{longtable}[]{@{}llllllllll@{}}
\caption{\label{tab:IntegratedAnalChen2022-published-data-metabolites}IntegratedAnalChen2022 published data metabolites}\tabularnewline
\toprule
V1 & V2 & V3 & V4 & V5 & V6 & V7 & V8 & V9 & V10\tabularnewline
\midrule
\endfirsthead
\toprule
V1 & V2 & V3 & V4 & V5 & V6 & V7 & V8 & V9 & V10\tabularnewline
\midrule
\endhead
Supple\ldots{} & material & placed & on & this & supple\ldots{} & material & which & has & been\tabularnewline
Table & S2. & List & of & colore\ldots{} & abnormal & associ\ldots{} & metabo\ldots{} & &\tabularnewline
Metabo\ldots{} & feature & Metabo\ldots{} & annota\ldots{} & approach & F\_meanN & F\_meanC & F\_meanA & anova\_\ldots{} & anova\_\ldots{}\tabularnewline
X10.3\_\ldots{} & (-)-Fo\ldots{} & exact & mass & match & 7748.8\ldots{} & 5053.5\ldots{} & 5527.3\ldots{} & 0.0000\ldots{} & 7.36E-08\tabularnewline
X12.4\_\ldots{} & (-)-Or\ldots{} & exact & mass & match & 45226\ldots. & 31316\ldots. & 27721\ldots. & 0.0000\ldots{} & 0.0000327\tabularnewline
X10.9\_\ldots{} & (-)-tr\ldots{} & glucos\ldots{} & mass & match & 364.39\ldots{} & 8542.0151 & 9965.7\ldots{} & 0.0026\ldots{} & 0.0009\ldots{}\tabularnewline
X15\_26\ldots{} & (+)-2,\ldots{} & Goniot\ldots{} & mass & match & 32595\ldots. & 22986\ldots. & 25975\ldots. & 0.0012\ldots{} & 0.0013\ldots{}\tabularnewline
X20.8\_\ldots{} & (+/-)-\ldots{} & mass & match & 6812.3\ldots{} & 10133\ldots. & 3573.2\ldots{} & 0.0042\ldots{} & 0.0002535 & 0.1164\ldots{}\tabularnewline
X12.5\_\ldots{} & (±)15-\ldots{} & exact & mass & match & 11931\ldots. & 8063.3\ldots{} & 8309.9\ldots{} & 3.99E-10 & 3.99E-10\tabularnewline
X14.9\_\ldots{} & (±)-Go\ldots{} & exact & mass & match & 24851\ldots. & 16540\ldots. & 15696\ldots. & 0.0022\ldots{} & 0.0001\ldots{}\tabularnewline
X15.1\_\ldots{} & (±)-Pa\ldots{} & exact & mass & match & 78154\ldots. & 37693\ldots. & 61456\ldots. & 0.0000\ldots{} & 0.0004\ldots{}\tabularnewline
X24\_35\ldots{} & (1S)-1\ldots{} & mass & match & 2067.0\ldots{} & 6783.2\ldots{} & 10613\ldots. & 0.0000\ldots{} & 0.0001\ldots{} & 0.5843\ldots{}\tabularnewline
X20.7\_\ldots{} & (20R)-\ldots{} & Rh2 & exact & mass & match & 19519\ldots. & 12756\ldots. & 12327\ldots. & 4.39E-10\tabularnewline
X11\_42\ldots{} & (22E)-\ldots{} & & & & & & & &\tabularnewline
exact & mass & match & 97120\ldots. & & & & & &\tabularnewline
\ldots{} & \ldots{} & \ldots{} & \ldots{} & \ldots{} & \ldots{} & \ldots{} & \ldots{} & \ldots{} & \ldots{}\tabularnewline
\bottomrule
\end{longtable}

未从上述数据中匹配到客户数据的差异肠道菌或其关联的代谢物。

\hypertarget{bibliography}{%
\section*{Reference}\label{bibliography}}
\addcontentsline{toc}{section}{Reference}

\hypertarget{refs}{}
\begin{cslreferences}
\leavevmode\hypertarget{ref-MendelianRandoLiuX2022}{}%
1. Liu, X. \emph{et al.} Mendelian randomization analyses support causal relationships between blood metabolites and the gut microbiome. \emph{Nature Genetics} \textbf{54}, (2022).

\leavevmode\hypertarget{ref-LossOfSymbiotSadegh2024}{}%
2. Sadeghi, M. \emph{et al.} Loss of symbiotic and increase of virulent bacteria through microbial networks in lynch syndrome colon carcinogenesis. \emph{Frontiers in oncology} \textbf{13}, (2024).

\leavevmode\hypertarget{ref-FellaAnRPacPicart2018}{}%
3. Picart-Armada, S., Fernandez-Albert, F., Vinaixa, M., Yanes, O. \& Perera-Lluna, A. FELLA: An r package to enrich metabolomics data. \emph{BMC Bioinformatics} \textbf{19}, 538 (2018).

\leavevmode\hypertarget{ref-UltrafastOnePChen2023}{}%
4. Chen, S. Ultrafast one-pass fastq data preprocessing, quality control, and deduplication using fastp. \emph{iMeta} \textbf{2}, (2023).

\leavevmode\hypertarget{ref-GutmdisorderACheng2019}{}%
5. Cheng, L., Qi, C., Zhuang, H., Fu, T. \& Zhang, X. GutMDisorder: A comprehensive database for dysbiosis of the gut microbiota in disorders and interventions. \emph{Nucleic Acids Research} \textbf{48}, (2019).

\leavevmode\hypertarget{ref-MicrobiotaproceXuSh2023}{}%
6. Xu, S. \emph{et al.} MicrobiotaProcess: A comprehensive r package for deep mining microbiome. \emph{The Innovation} \textbf{4}, 100388 (2023).

\leavevmode\hypertarget{ref-Metaboanalyst4Chong2018}{}%
7. Chong, J. \emph{et al.} MetaboAnalyst 4.0: Towards more transparent and integrative metabolomics analysis. \emph{Nucleic Acids Research} \textbf{46}, W486--W494 (2018).

\leavevmode\hypertarget{ref-ReproducibleIBolyen2019}{}%
8. Bolyen, E. \emph{et al.} Reproducible, interactive, scalable and extensible microbiome data science using qiime 2. \emph{Nature Biotechnology} \textbf{37}, 852--857 (2019).

\leavevmode\hypertarget{ref-TheBiologicalMcdona2012}{}%
9. McDonald, D. \emph{et al.} The biological observation matrix (biom) format or: How i learned to stop worrying and love the ome-ome. \emph{GigaScience} \textbf{1}, 7 (2012).

\leavevmode\hypertarget{ref-Dada2HighResCallah2016}{}%
10. Callahan, B. J. \emph{et al.} DADA2: High-resolution sample inference from illumina amplicon data. \emph{Nature methods} \textbf{13}, 581 (2016).

\leavevmode\hypertarget{ref-ErrorCorrectinHamday2008}{}%
11. Hamday, M., Walker J., J., Harris, J. K., Gold J., N. \& Knight, R. Error-correcting barcoded primers allow hundreds of samples to be pyrosequenced in multiplex. \emph{Nature Methods} \textbf{5}, 235--237 (2008).

\leavevmode\hypertarget{ref-MicrobialCommuHamday2009}{}%
12. Hamday, M. \& Knight, R. Microbial community profiling for human microbiome projects: Tools, techniques, and challenges. \emph{Genome Research} \textbf{19}, 1141--1152 (2009).

\leavevmode\hypertarget{ref-DepressionAndYuan2021}{}%
13. Yuan, X. \emph{et al.} Depression and anxiety in patients with active ulcerative colitis: Crosstalk of gut microbiota, metabolomics and proteomics. \emph{Gut microbes} \textbf{13}, (2021).

\leavevmode\hypertarget{ref-AlterationsInScovil2018}{}%
14. Scoville, E. A. \emph{et al.} Alterations in lipid, amino acid, and energy metabolism distinguish crohn's disease from ulcerative colitis and control subjects by serum metabolomic profiling. \emph{Metabolomics : Official journal of the Metabolomic Society} \textbf{14}, (2018).

\leavevmode\hypertarget{ref-IntegratedAnalChen2022}{}%
15. Chen, F. \emph{et al.} Integrated analysis of the faecal metagenome and serum metabolome reveals the role of gut microbiome-associated metabolites in the detection of colorectal cancer and adenoma. \emph{Gut} \textbf{71}, 1315--1325 (2022).
\end{cslreferences}

\end{document}
