% Options for packages loaded elsewhere
\PassOptionsToPackage{unicode}{hyperref}
\PassOptionsToPackage{hyphens}{url}
%
\documentclass[
]{article}
\usepackage{amsmath,amssymb}
\usepackage{iftex}
\ifPDFTeX
  \usepackage[T1]{fontenc}
  \usepackage[utf8]{inputenc}
  \usepackage{textcomp} % provide euro and other symbols
\else % if luatex or xetex
  \usepackage{unicode-math} % this also loads fontspec
  \defaultfontfeatures{Scale=MatchLowercase}
  \defaultfontfeatures[\rmfamily]{Ligatures=TeX,Scale=1}
\fi
\usepackage{lmodern}
\ifPDFTeX\else
  % xetex/luatex font selection
\fi
% Use upquote if available, for straight quotes in verbatim environments
\IfFileExists{upquote.sty}{\usepackage{upquote}}{}
\IfFileExists{microtype.sty}{% use microtype if available
  \usepackage[]{microtype}
  \UseMicrotypeSet[protrusion]{basicmath} % disable protrusion for tt fonts
}{}
\makeatletter
\@ifundefined{KOMAClassName}{% if non-KOMA class
  \IfFileExists{parskip.sty}{%
    \usepackage{parskip}
  }{% else
    \setlength{\parindent}{0pt}
    \setlength{\parskip}{6pt plus 2pt minus 1pt}}
}{% if KOMA class
  \KOMAoptions{parskip=half}}
\makeatother
\usepackage{xcolor}
\usepackage[margin=1in]{geometry}
\usepackage{longtable,booktabs,array}
\usepackage{calc} % for calculating minipage widths
% Correct order of tables after \paragraph or \subparagraph
\usepackage{etoolbox}
\makeatletter
\patchcmd\longtable{\par}{\if@noskipsec\mbox{}\fi\par}{}{}
\makeatother
% Allow footnotes in longtable head/foot
\IfFileExists{footnotehyper.sty}{\usepackage{footnotehyper}}{\usepackage{footnote}}
\makesavenoteenv{longtable}
\usepackage{graphicx}
\makeatletter
\def\maxwidth{\ifdim\Gin@nat@width>\linewidth\linewidth\else\Gin@nat@width\fi}
\def\maxheight{\ifdim\Gin@nat@height>\textheight\textheight\else\Gin@nat@height\fi}
\makeatother
% Scale images if necessary, so that they will not overflow the page
% margins by default, and it is still possible to overwrite the defaults
% using explicit options in \includegraphics[width, height, ...]{}
\setkeys{Gin}{width=\maxwidth,height=\maxheight,keepaspectratio}
% Set default figure placement to htbp
\makeatletter
\def\fps@figure{htbp}
\makeatother
\setlength{\emergencystretch}{3em} % prevent overfull lines
\providecommand{\tightlist}{%
  \setlength{\itemsep}{0pt}\setlength{\parskip}{0pt}}
\setcounter{secnumdepth}{5}
\newlength{\cslhangindent}
\setlength{\cslhangindent}{1.5em}
\newlength{\csllabelwidth}
\setlength{\csllabelwidth}{3em}
\newlength{\cslentryspacingunit} % times entry-spacing
\setlength{\cslentryspacingunit}{\parskip}
\newenvironment{CSLReferences}[2] % #1 hanging-ident, #2 entry spacing
 {% don't indent paragraphs
  \setlength{\parindent}{0pt}
  % turn on hanging indent if param 1 is 1
  \ifodd #1
  \let\oldpar\par
  \def\par{\hangindent=\cslhangindent\oldpar}
  \fi
  % set entry spacing
  \setlength{\parskip}{#2\cslentryspacingunit}
 }%
 {}
\usepackage{calc}
\newcommand{\CSLBlock}[1]{#1\hfill\break}
\newcommand{\CSLLeftMargin}[1]{\parbox[t]{\csllabelwidth}{#1}}
\newcommand{\CSLRightInline}[1]{\parbox[t]{\linewidth - \csllabelwidth}{#1}\break}
\newcommand{\CSLIndent}[1]{\hspace{\cslhangindent}#1}
\usepackage{caption} \captionsetup{font={footnotesize},width=6in} \renewcommand{\dblfloatpagefraction}{.9} \makeatletter \renewenvironment{figure} {\def\@captype{figure}} \makeatother \newenvironment{Shaded}{\begin{snugshade}}{\end{snugshade}} \definecolor{shadecolor}{RGB}{230,230,230} \usepackage{xeCJK} \usepackage{setspace} \setstretch{1.3} \usepackage{tcolorbox} \setcounter{secnumdepth}{4} \setcounter{tocdepth}{4} \usepackage{wallpaper} \usepackage[absolute]{textpos} \tcbuselibrary{breakable} \renewenvironment{Shaded} {\begin{tcolorbox}[colback = gray!10, colframe = gray!40, width = 16cm, arc = 1mm, auto outer arc, title = {Input}]} {\end{tcolorbox}} \usepackage{titlesec} \titleformat{\paragraph} {\fontsize{10pt}{0pt}\bfseries} {\arabic{section}.\arabic{subsection}.\arabic{subsubsection}.\arabic{paragraph}} {1em} {} []
\ifLuaTeX
  \usepackage{selnolig}  % disable illegal ligatures
\fi
\IfFileExists{bookmark.sty}{\usepackage{bookmark}}{\usepackage{hyperref}}
\IfFileExists{xurl.sty}{\usepackage{xurl}}{} % add URL line breaks if available
\urlstyle{same}
\hypersetup{
  hidelinks,
  pdfcreator={LaTeX via pandoc}}

\author{}
\date{\vspace{-2.5em}}

\begin{document}

\begin{titlepage} \newgeometry{top=7.5cm}
\ThisCenterWallPaper{1.12}{../cover_page.pdf}
\begin{center} \textbf{\Huge
菌群+对应代谢产物介导+机制研究} \vspace{4em}
\begin{textblock}{10}(3,5.9) \huge
\textbf{\textcolor{white}{2024-01-12}}
\end{textblock} \begin{textblock}{10}(3,7.3)
\Large \textcolor{black}{LiChuang Huang}
\end{textblock} \begin{textblock}{10}(3,11.3)
\Large \textcolor{black}{@立效研究院}
\end{textblock} \end{center} \end{titlepage}
\restoregeometry

\pagenumbering{roman}

\tableofcontents

\listoffigures

\listoftables

\newpage

\pagenumbering{arabic}

\hypertarget{abstract}{%
\section{摘要}\label{abstract}}

生信分析(8个con)+(8个A)+(8个B)(盲筛,不提供具体分组信息)。

\hypertarget{introduction}{%
\section{前言}\label{introduction}}

\hypertarget{methods}{%
\section{材料和方法}\label{methods}}

\hypertarget{ux6750ux6599}{%
\subsection{材料}\label{ux6750ux6599}}

\hypertarget{ux65b9ux6cd5}{%
\subsection{方法}\label{ux65b9ux6cd5}}

Mainly used method:

\begin{itemize}
\tightlist
\item
  \texttt{Fastp} used for Fastq data preprocessing\textsuperscript{\protect\hyperlink{ref-UltrafastOnePChen2023}{1}}.
\item
  R package \texttt{MicrobiotaProcess} used for microbiome data visualization\textsuperscript{\protect\hyperlink{ref-MicrobiotaproceXuSh2023}{2}}.
\item
  \texttt{Qiime2} used for gut microbiome 16s rRNA analysis\textsuperscript{\protect\hyperlink{ref-ReproducibleIBolyen2019}{3}--\protect\hyperlink{ref-MicrobialCommuHamday2009}{7}}.
\item
  Other R packages (eg., \texttt{dplyr} and \texttt{ggplot2}) used for statistic analysis or data visualization.
\end{itemize}

\hypertarget{results}{%
\section{分析结果}\label{results}}

\begin{itemize}
\tightlist
\item
  A、B 组 Alpha 和 Beta 多样性无显著差异 (见 \ref{alpha} 和 \ref{beta})。
\item
  A、B 组差异分析,未找到差异菌。
\end{itemize}

\hypertarget{dis}{%
\section{结论}\label{dis}}

\hypertarget{workflow}{%
\section{附:分析流程}\label{workflow}}

\hypertarget{microbiota-16s-rna}{%
\subsection{Microbiota 16s RNA}\label{microbiota-16s-rna}}

\hypertarget{fastp-qc}{%
\subsubsection{Fastp QC}\label{fastp-qc}}

原始数据质控:

`Fastp QC' 数据已全部提供。

\textbf{(对应文件为 \texttt{./fastp\_report/})}

\begin{center}\begin{tcolorbox}[colback=gray!10, colframe=gray!50, width=0.9\linewidth, arc=1mm, boxrule=0.5pt]注:文件夹./fastp\_report/共包含17个文件。

\begin{enumerate}\tightlist
\item A1.338F\_806R..html
\item A2.338F\_806R..html
\item A3.338F\_806R..html
\item A4.338F\_806R..html
\item A5.338F\_806R..html
\item ...
\end{enumerate}\end{tcolorbox}
\end{center}

\hypertarget{ux5143ux6570ux636e}{%
\subsubsection{元数据}\label{ux5143ux6570ux636e}}

Table \ref{tab:microbiota-metadata} (下方表格) 为表格microbiota metadata概览。

\textbf{(对应文件为 \texttt{Figure+Table/microbiota-metadata.csv})}

\begin{center}\begin{tcolorbox}[colback=gray!10, colframe=gray!50, width=0.9\linewidth, arc=1mm, boxrule=0.5pt]注:表格共有16行7列,以下预览的表格可能省略部分数据;表格含有16个唯一`SampleName'。
\end{tcolorbox}
\end{center}
\begin{center}\begin{tcolorbox}[colback=gray!10, colframe=gray!50, width=0.9\linewidth, arc=1mm, boxrule=0.5pt]\begin{enumerate}\tightlist
\item group:  分组名称
\end{enumerate}\end{tcolorbox}
\end{center}

\begin{longtable}[]{@{}
  >{\raggedright\arraybackslash}p{(\columnwidth - 12\tabcolsep) * \real{0.1358}}
  >{\raggedright\arraybackslash}p{(\columnwidth - 12\tabcolsep) * \real{0.0741}}
  >{\raggedright\arraybackslash}p{(\columnwidth - 12\tabcolsep) * \real{0.1728}}
  >{\raggedright\arraybackslash}p{(\columnwidth - 12\tabcolsep) * \real{0.1728}}
  >{\raggedright\arraybackslash}p{(\columnwidth - 12\tabcolsep) * \real{0.0988}}
  >{\raggedright\arraybackslash}p{(\columnwidth - 12\tabcolsep) * \real{0.1728}}
  >{\raggedright\arraybackslash}p{(\columnwidth - 12\tabcolsep) * \real{0.1728}}@{}}
\caption{\label{tab:microbiota-metadata}Microbiota metadata}\tabularnewline
\toprule\noalign{}
\begin{minipage}[b]{\linewidth}\raggedright
SampleName
\end{minipage} & \begin{minipage}[b]{\linewidth}\raggedright
group
\end{minipage} & \begin{minipage}[b]{\linewidth}\raggedright
dirs
\end{minipage} & \begin{minipage}[b]{\linewidth}\raggedright
reports
\end{minipage} & \begin{minipage}[b]{\linewidth}\raggedright
Run
\end{minipage} & \begin{minipage}[b]{\linewidth}\raggedright
forward-ab\ldots{}
\end{minipage} & \begin{minipage}[b]{\linewidth}\raggedright
reverse-ab\ldots{}
\end{minipage} \\
\midrule\noalign{}
\endfirsthead
\toprule\noalign{}
\begin{minipage}[b]{\linewidth}\raggedright
SampleName
\end{minipage} & \begin{minipage}[b]{\linewidth}\raggedright
group
\end{minipage} & \begin{minipage}[b]{\linewidth}\raggedright
dirs
\end{minipage} & \begin{minipage}[b]{\linewidth}\raggedright
reports
\end{minipage} & \begin{minipage}[b]{\linewidth}\raggedright
Run
\end{minipage} & \begin{minipage}[b]{\linewidth}\raggedright
forward-ab\ldots{}
\end{minipage} & \begin{minipage}[b]{\linewidth}\raggedright
reverse-ab\ldots{}
\end{minipage} \\
\midrule\noalign{}
\endhead
\bottomrule\noalign{}
\endlastfoot
A1 & A & ./material\ldots{} & ./material\ldots{} & rawData & /home/echo\ldots{} & /home/echo\ldots{} \\
A2 & A & ./material\ldots{} & ./material\ldots{} & rawData & /home/echo\ldots{} & /home/echo\ldots{} \\
A3 & A & ./material\ldots{} & ./material\ldots{} & rawData & /home/echo\ldots{} & /home/echo\ldots{} \\
A4 & A & ./material\ldots{} & ./material\ldots{} & rawData & /home/echo\ldots{} & /home/echo\ldots{} \\
A5 & A & ./material\ldots{} & ./material\ldots{} & rawData & /home/echo\ldots{} & /home/echo\ldots{} \\
A6 & A & ./material\ldots{} & ./material\ldots{} & rawData & /home/echo\ldots{} & /home/echo\ldots{} \\
A7 & A & ./material\ldots{} & ./material\ldots{} & rawData & /home/echo\ldots{} & /home/echo\ldots{} \\
A8 & A & ./material\ldots{} & ./material\ldots{} & rawData & /home/echo\ldots{} & /home/echo\ldots{} \\
B1 & B & ./material\ldots{} & ./material\ldots{} & rawData & /home/echo\ldots{} & /home/echo\ldots{} \\
B2 & B & ./material\ldots{} & ./material\ldots{} & rawData & /home/echo\ldots{} & /home/echo\ldots{} \\
B3 & B & ./material\ldots{} & ./material\ldots{} & rawData & /home/echo\ldots{} & /home/echo\ldots{} \\
B4 & B & ./material\ldots{} & ./material\ldots{} & rawData & /home/echo\ldots{} & /home/echo\ldots{} \\
B5 & B & ./material\ldots{} & ./material\ldots{} & rawData & /home/echo\ldots{} & /home/echo\ldots{} \\
B6 & B & ./material\ldots{} & ./material\ldots{} & rawData & /home/echo\ldots{} & /home/echo\ldots{} \\
B7 & B & ./material\ldots{} & ./material\ldots{} & rawData & /home/echo\ldots{} & /home/echo\ldots{} \\
\ldots{} & \ldots{} & \ldots{} & \ldots{} & \ldots{} & \ldots{} & \ldots{} \\
\end{longtable}

\hypertarget{qiime2-ux5206ux6790}{%
\subsubsection{Qiime2 分析}\label{qiime2-ux5206ux6790}}

Microbiota 数据经 Qiime2 分析后,由 \texttt{MicrobiotaProcess} 下游分析和可视化。

\hypertarget{microbiotaprocess-ux5206ux6790}{%
\subsubsection{MicrobiotaProcess 分析}\label{microbiotaprocess-ux5206ux6790}}

\hypertarget{ux6837ux672cux805aux7c7b}{%
\paragraph{样本聚类}\label{ux6837ux672cux805aux7c7b}}

在预分析中,根据 PCoA 去除离群样本:

Figure \ref{fig:All-samples-PCoA} (下方图) 为图All samples PCoA概览。

\textbf{(对应文件为 \texttt{Figure+Table/All-samples-PCoA.pdf})}

\def\@captype{figure}
\begin{center}
\includegraphics[width = 0.9\linewidth]{Figure+Table/All-samples-PCoA.pdf}
\caption{All samples PCoA}\label{fig:All-samples-PCoA}
\end{center}

以下为除去样本后的 PCoA:

Figure \ref{fig:Filtered-PCoA} (下方图) 为图Filtered PCoA概览。

\textbf{(对应文件为 \texttt{Figure+Table/Filtered-PCoA.pdf})}

\def\@captype{figure}
\begin{center}
\includegraphics[width = 0.9\linewidth]{Figure+Table/Filtered-PCoA.pdf}
\caption{Filtered PCoA}\label{fig:Filtered-PCoA}
\end{center}

随后的分析以去除离群样本后进行。

\hypertarget{alpha}{%
\paragraph{Alpha 多样性}\label{alpha}}

A、B 组 alpha 多样性没有显著差异。

Figure \ref{fig:Alpha-diversity} (下方图) 为图Alpha diversity概览。

\textbf{(对应文件为 \texttt{Figure+Table/Alpha-diversity.pdf})}

\def\@captype{figure}
\begin{center}
\includegraphics[width = 0.9\linewidth]{Figure+Table/Alpha-diversity.pdf}
\caption{Alpha diversity}\label{fig:Alpha-diversity}
\end{center}

`Taxonomy abundance' 数据已全部提供。

\textbf{(对应文件为 \texttt{Figure+Table/Taxonomy-abundance})}

\begin{center}\begin{tcolorbox}[colback=gray!10, colframe=gray!50, width=0.9\linewidth, arc=1mm, boxrule=0.5pt]注:文件夹Figure+Table/Taxonomy-abundance共包含6个文件。

\begin{enumerate}\tightlist
\item 1\_Phylum.pdf
\item 2\_Class.pdf
\item 3\_Order.pdf
\item 4\_Family.pdf
\item 5\_Genus.pdf
\item ...
\end{enumerate}\end{tcolorbox}
\end{center}

\hypertarget{alpha-ux7a00ux758fux66f2ux7ebf}{%
\paragraph{Alpha 稀疏曲线}\label{alpha-ux7a00ux758fux66f2ux7ebf}}

Figure \ref{fig:Alpha-rarefaction} (下方图) 为图Alpha rarefaction概览。

\textbf{(对应文件为 \texttt{Figure+Table/Alpha-rarefaction.pdf})}

\def\@captype{figure}
\begin{center}
\includegraphics[width = 0.9\linewidth]{Figure+Table/Alpha-rarefaction.pdf}
\caption{Alpha rarefaction}\label{fig:Alpha-rarefaction}
\end{center}

\hypertarget{beta}{%
\paragraph{Beta 多样性}\label{beta}}

A、B 组 Beta 多样性无显著差异。

Figure \ref{fig:Beta-diversity-group-test} (下方图) 为图Beta diversity group test概览。

\textbf{(对应文件为 \texttt{Figure+Table/Beta-diversity-group-test.pdf})}

\def\@captype{figure}
\begin{center}
\includegraphics[width = 0.9\linewidth]{Figure+Table/Beta-diversity-group-test.pdf}
\caption{Beta diversity group test}\label{fig:Beta-diversity-group-test}
\end{center}

`Taxonomy hierarchy' 数据已全部提供。

\textbf{(对应文件为 \texttt{Figure+Table/Taxonomy-hierarchy})}

\begin{center}\begin{tcolorbox}[colback=gray!10, colframe=gray!50, width=0.9\linewidth, arc=1mm, boxrule=0.5pt]注:文件夹Figure+Table/Taxonomy-hierarchy共包含6个文件。

\begin{enumerate}\tightlist
\item 1\_Phylum.pdf
\item 2\_Class.pdf
\item 3\_Order.pdf
\item 4\_Family.pdf
\item 5\_Genus.pdf
\item ...
\end{enumerate}\end{tcolorbox}
\end{center}

\hypertarget{ux5deeux5f02ux5206ux6790}{%
\paragraph{差异分析}\label{ux5deeux5f02ux5206ux6790}}

注:MicrobiotaProcess 的差异分析和 Qiime2 差异分析结果相同,未发现差异肠道菌。

\hypertarget{bibliography}{%
\section*{Reference}\label{bibliography}}
\addcontentsline{toc}{section}{Reference}

\hypertarget{refs}{}
\begin{CSLReferences}{0}{0}
\leavevmode\vadjust pre{\hypertarget{ref-UltrafastOnePChen2023}{}}%
\CSLLeftMargin{1. }%
\CSLRightInline{Chen, S. \href{https://doi.org/10.1002/imt2.107}{Ultrafast one-pass FASTQ data preprocessing, quality control, and deduplication using fastp}. \emph{iMeta} \textbf{2}, (2023).}

\leavevmode\vadjust pre{\hypertarget{ref-MicrobiotaproceXuSh2023}{}}%
\CSLLeftMargin{2. }%
\CSLRightInline{Xu, S. \emph{et al.} \href{https://doi.org/10.1016/j.xinn.2023.100388}{MicrobiotaProcess: A comprehensive r package for deep mining microbiome}. \emph{The Innovation} \textbf{4}, 100388 (2023).}

\leavevmode\vadjust pre{\hypertarget{ref-ReproducibleIBolyen2019}{}}%
\CSLLeftMargin{3. }%
\CSLRightInline{Bolyen, E. \emph{et al.} \href{https://doi.org/10.1038/s41587-019-0209-9}{Reproducible, interactive, scalable and extensible microbiome data science using QIIME 2}. \emph{Nature Biotechnology} \textbf{37}, 852--857 (2019).}

\leavevmode\vadjust pre{\hypertarget{ref-TheBiologicalMcdona2012}{}}%
\CSLLeftMargin{4. }%
\CSLRightInline{McDonald, D. \emph{et al.} \href{https://doi.org/10.1186/2047-217X-1-7}{The biological observation matrix (BIOM) format or: How i learned to stop worrying and love the ome-ome}. \emph{GigaScience} \textbf{1}, 7 (2012).}

\leavevmode\vadjust pre{\hypertarget{ref-Dada2HighResCallah2016}{}}%
\CSLLeftMargin{5. }%
\CSLRightInline{Callahan, B. J. \emph{et al.} \href{https://doi.org/10.1038/nmeth.3869}{DADA2: High-resolution sample inference from illumina amplicon data}. \emph{Nature methods} \textbf{13}, 581 (2016).}

\leavevmode\vadjust pre{\hypertarget{ref-ErrorCorrectinHamday2008}{}}%
\CSLLeftMargin{6. }%
\CSLRightInline{Hamday, M., Walker J., J., Harris, J. K., Gold J., N. \& Knight, R. \href{https://doi.org/10.1038/nmeth.1184}{Error-correcting barcoded primers allow hundreds of samples to be pyrosequenced in multiplex}. \emph{Nature Methods} \textbf{5}, 235--237 (2008).}

\leavevmode\vadjust pre{\hypertarget{ref-MicrobialCommuHamday2009}{}}%
\CSLLeftMargin{7. }%
\CSLRightInline{Hamday, M. \& Knight, R. \href{https://doi.org/10.1101/gr.085464.108}{Microbial community profiling for human microbiome projects: Tools, techniques, and challenges}. \emph{Genome Research} \textbf{19}, 1141--1152 (2009).}

\end{CSLReferences}

\end{document}
