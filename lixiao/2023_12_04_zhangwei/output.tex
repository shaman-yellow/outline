% Options for packages loaded elsewhere
\PassOptionsToPackage{unicode}{hyperref}
\PassOptionsToPackage{hyphens}{url}
%
\documentclass[
]{article}
\usepackage{lmodern}
\usepackage{amssymb,amsmath}
\usepackage{ifxetex,ifluatex}
\ifnum 0\ifxetex 1\fi\ifluatex 1\fi=0 % if pdftex
  \usepackage[T1]{fontenc}
  \usepackage[utf8]{inputenc}
  \usepackage{textcomp} % provide euro and other symbols
\else % if luatex or xetex
  \usepackage{unicode-math}
  \defaultfontfeatures{Scale=MatchLowercase}
  \defaultfontfeatures[\rmfamily]{Ligatures=TeX,Scale=1}
\fi
% Use upquote if available, for straight quotes in verbatim environments
\IfFileExists{upquote.sty}{\usepackage{upquote}}{}
\IfFileExists{microtype.sty}{% use microtype if available
  \usepackage[]{microtype}
  \UseMicrotypeSet[protrusion]{basicmath} % disable protrusion for tt fonts
}{}
\makeatletter
\@ifundefined{KOMAClassName}{% if non-KOMA class
  \IfFileExists{parskip.sty}{%
    \usepackage{parskip}
  }{% else
    \setlength{\parindent}{0pt}
    \setlength{\parskip}{6pt plus 2pt minus 1pt}}
}{% if KOMA class
  \KOMAoptions{parskip=half}}
\makeatother
\usepackage{xcolor}
\IfFileExists{xurl.sty}{\usepackage{xurl}}{} % add URL line breaks if available
\IfFileExists{bookmark.sty}{\usepackage{bookmark}}{\usepackage{hyperref}}
\hypersetup{
  hidelinks,
  pdfcreator={LaTeX via pandoc}}
\urlstyle{same} % disable monospaced font for URLs
\usepackage[margin=1in]{geometry}
\usepackage{longtable,booktabs}
% Correct order of tables after \paragraph or \subparagraph
\usepackage{etoolbox}
\makeatletter
\patchcmd\longtable{\par}{\if@noskipsec\mbox{}\fi\par}{}{}
\makeatother
% Allow footnotes in longtable head/foot
\IfFileExists{footnotehyper.sty}{\usepackage{footnotehyper}}{\usepackage{footnote}}
\makesavenoteenv{longtable}
\usepackage{graphicx}
\makeatletter
\def\maxwidth{\ifdim\Gin@nat@width>\linewidth\linewidth\else\Gin@nat@width\fi}
\def\maxheight{\ifdim\Gin@nat@height>\textheight\textheight\else\Gin@nat@height\fi}
\makeatother
% Scale images if necessary, so that they will not overflow the page
% margins by default, and it is still possible to overwrite the defaults
% using explicit options in \includegraphics[width, height, ...]{}
\setkeys{Gin}{width=\maxwidth,height=\maxheight,keepaspectratio}
% Set default figure placement to htbp
\makeatletter
\def\fps@figure{htbp}
\makeatother
\setlength{\emergencystretch}{3em} % prevent overfull lines
\providecommand{\tightlist}{%
  \setlength{\itemsep}{0pt}\setlength{\parskip}{0pt}}
\setcounter{secnumdepth}{5}
\usepackage{caption} \captionsetup{font={footnotesize},width=6in} \renewcommand{\dblfloatpagefraction}{.9} \makeatletter \renewenvironment{figure} {\def\@captype{figure}} \makeatother \definecolor{shadecolor}{RGB}{242,242,242} \usepackage{xeCJK} \usepackage{setspace} \setstretch{1.3} \usepackage{tcolorbox} \setcounter{secnumdepth}{4} \setcounter{tocdepth}{4} \usepackage{wallpaper} \usepackage[absolute]{textpos}
\newlength{\cslhangindent}
\setlength{\cslhangindent}{1.5em}
\newenvironment{cslreferences}%
  {}%
  {\par}

\author{}
\date{\vspace{-2.5em}}

\begin{document}

\begin{titlepage} \newgeometry{top=7.5cm}
\ThisCenterWallPaper{1.12}{../cover_page.pdf}
\begin{center} \textbf{\Huge RNA-seq 探究 rTMS 对
SCI 和 NP 的影响} \vspace{4em}
\begin{textblock}{10}(3,5.9) \huge
\textbf{\textcolor{white}{2023-12-12}}
\end{textblock} \begin{textblock}{10}(3,7.3)
\Large \textcolor{black}{LiChuang Huang}
\end{textblock} \begin{textblock}{10}(3,11.3)
\Large \textcolor{black}{@立效研究院}
\end{textblock} \end{center} \end{titlepage}
\restoregeometry

\pagenumbering{roman}

\tableofcontents

\listoffigures

\listoftables

\newpage

\pagenumbering{arabic}

\hypertarget{abstract}{%
\section{摘要}\label{abstract}}

为探究 rTMS 缓解 iSCI 所致 NP 的潜在机理,利用多组 RNA-seq 数据寻找潜在关联基因。结果见 \ref{dis}

\hypertarget{introduction}{%
\section{前言}\label{introduction}}

脊髓损伤 (SCI) 是一种使人衰弱的疾病,经常伴有神经性疼痛。尽管 SCI 后神经性疼痛的发病率很高,但其确切的潜在机制仍不完全清楚\textsuperscript{\protect\hyperlink{ref-TheDualRoleOSunC2023}{1}}。
经颅磁刺激(TMS)是一种无创、无痛的刺激人脑的方法。单脉冲和成对脉冲 TMS 范式是研究神经退行性疾病病理生理机制的有力方法\textsuperscript{\protect\hyperlink{ref-TranscranialMaNiZh2015}{2}}。
已有研究发现,重复经颅磁刺激治疗 (repeat transcranial magnetic stimulation, rTMS) 刺激脑卒中患者病灶侧大脑半球运动区,发现其能减轻患者脑卒中后 神经病理性疼痛 (neuropathic pain, NP)\textsuperscript{\protect\hyperlink{ref-DifferentialEfAmeli2009}{3},\protect\hyperlink{ref-ReductionOfInHiraya2006}{4}}。
然而,目前未有研究阐述 rTMS 缓解 SCI 所致 NP 的机理。
本研究借助分析公共 不完全性脊髓损伤 (Incomplete spinal cord injury, iSCI) 、神经病理性疼痛 (neuropathic pain, NP) 和 重复经颅磁刺激治疗 (repeat transcranial magnetic stimulation, rTMS) RNA-seq 数据集,探究 rTMS 缓解 iSCI 所致 NP 的潜在机理。

\hypertarget{methods}{%
\section{材料和方法}\label{methods}}

\hypertarget{ux6750ux6599}{%
\subsection{材料}\label{ux6750ux6599}}

All used GEO expression data and their design:

\begin{itemize}
\item
  \textbf{GSE126611}: We investigated n=14 samples, no replicates, comparison between two patient groups, and patient group with healthy controls. (NL-1) is with nerve lesion and (NL-0) is without neuropathic pain.
\item
  \textbf{GSE230149}: Gene expression microarray analysis of 48 mouse brain samples, consisting of matched sets of hippocampus (HIPP, N=24) and parietal cortex (PCTX, N=24) from young sham (Y\_Sham, N=4), young iTBS (Y\_i\ldots{}
\item
  \textbf{GSE226238}: RNAsequencing from whole blood taken from participants with SCI within 3 days of injury, at 3 MPI, 6 MPI and 12 MPI. Data was compared to un-injured participants as controls. Inclusion and exclusio\ldots{}
\end{itemize}

\hypertarget{ux65b9ux6cd5}{%
\subsection{方法}\label{ux65b9ux6cd5}}

Mainly used method:

\begin{itemize}
\tightlist
\item
  ClusterProfiler used for gene enrichment analysis.\textsuperscript{\protect\hyperlink{ref-ClusterprofilerWuTi2021}{5}}
\item
  GEO \url{https://www.ncbi.nlm.nih.gov/geo/} used for expression dataset aquisition .
\item
  Limma and edgeR used for differential expression analysis.\textsuperscript{\protect\hyperlink{ref-LimmaPowersDiRitchi2015}{6},\protect\hyperlink{ref-EdgerDifferenChen}{7}}
\item
  Other R packages (eg., \texttt{dplyr} and \texttt{ggplot2}) used for statistic analysis or data visualization.
\end{itemize}

\hypertarget{results}{%
\section{分析结果}\label{results}}

\hypertarget{ux4e0dux5b8cux5168ux6027ux810aux9ad3ux635fux4f24-incomplete-spinal-cord-injury-isci-ux548c-ux795eux7ecfux75c5ux7406ux6027ux75bcux75db-neuropathic-pain-np}{%
\subsection{不完全性脊髓损伤 (Incomplete spinal cord injury, iSCI) 和 神经病理性疼痛 (neuropathic pain, NP)}\label{ux4e0dux5b8cux5168ux6027ux810aux9ad3ux635fux4f24-incomplete-spinal-cord-injury-isci-ux548c-ux795eux7ecfux75c5ux7406ux6027ux75bcux75db-neuropathic-pain-np}}

筛选 GEO 数据库,获取 SCI (GSE226238) 和 NP (GSE126611) 数据集 (来源 Homo sapiens,即,人类的 RNA-seq 数据集),进行差异表达分析。SCI 数据集 (SCI vs control, Fig. \ref{fig:MAIN-fig-1}a) 和 NP 数据集 (NP vs control, Fig. \ref{fig:MAIN-fig-1}b) 的差异基因 (DEGs, p-value 或 adjust p-value \textless{} 0.05, \textbar log\textsubscript{2}(FC)\textbar{} \textgreater{} 0.3) 交集如图 Fig. \ref{fig:MAIN-fig-1}c 所示 (具体基因见 Fig. \ref{fig:SCI-NP-coDEGs})。

Figure \ref{fig:MAIN-fig-1} (下方图) 为图MAIN fig 1概览。

\textbf{(对应文件为 \texttt{./Figure+Table/fig1.pdf})}

\def\@captype{figure}
\begin{center}
\includegraphics[width = 0.9\linewidth]{./Figure+Table/fig1.pdf}
\caption{MAIN fig 1}\label{fig:MAIN-fig-1}
\end{center}

\hypertarget{sci-ux548c-np-ux5173ux8054ux5206ux6790}{%
\subsection{SCI 和 NP 关联分析}\label{sci-ux548c-np-ux5173ux8054ux5206ux6790}}

将上述交集基因 (记为 coDEGs) 在 SCI 和 NP 各自的数据集进行关联分析 (Fig. \ref{fig:MAIN-fig-2}a、b) ,筛选显著关联基因 (p-value \textless{} 0.05) ,并将之合并 (sig-coDegs, 见 \ref{sigCoDEGs}) 。我们发现,所有的 coDEGs 都属于 sig-coDegs,即,在 SCI 和 NP 数据集中,这些基因彼此都至少有一个显著关联的基因。

Figure \ref{fig:MAIN-fig-2} (下方图) 为图MAIN fig 2概览。

\textbf{(对应文件为 \texttt{./Figure+Table/fig2.pdf})}

\def\@captype{figure}
\begin{center}
\includegraphics[width = 0.9\linewidth]{./Figure+Table/fig2.pdf}
\caption{MAIN fig 2}\label{fig:MAIN-fig-2}
\end{center}

\hypertarget{res-rtms}{%
\subsection{重复经颅磁刺激治疗 (rTMS)}\label{res-rtms}}

为了探究 rTMS 刺激所致的大脑转录变化,本研究选用了 GEO 的 rTMS 数据集 (GSE230149) (来源 Rattus norvegicus)。
该数据集包含多种因素分组:

\begin{itemize}
\tightlist
\item
  Young, Y (age)
\item
  Aged, A (age)
\item
  Impaired, I (cognitive status)
\item
  Unimpaired, U (cognitive status)
\item
  Hippocampus, H (tissue region)
\item
  Parietal cortex, P (tissue, region)
\end{itemize}

控制变量以差异分析 (Fig. \ref{fig:MAIN-fig-3}) 。可以发现,rTMS 对大脑带来的主要转录变化发生于 Hippocampus,而对 Parietal cortex 几乎不造成转录影响。取 Fig. \ref{fig:MAIN-fig-3} 的合集 (记为 rTMS-DEGs)。

Figure \ref{fig:MAIN-fig-3} (下方图) 为图MAIN fig 3概览。

\textbf{(对应文件为 \texttt{./Figure+Table/fig3.pdf})}

\def\@captype{figure}
\begin{center}
\includegraphics[width = 0.9\linewidth]{./Figure+Table/fig3.pdf}
\caption{MAIN fig 3}\label{fig:MAIN-fig-3}
\end{center}

\hypertarget{rtms-ux548c-scinp-ux7684ux5173ux8054}{%
\subsection{rTMS 和 SCI、NP 的关联}\label{rtms-ux548c-scinp-ux7684ux5173ux8054}}

假设 rTMS 对人类和对大鼠大脑的影响是相似的,则我们可以把 \ref{res-rtms} 的差异基因 (rTMS-DEGs) 对应到 SCI 和 NP 的数据集中 (见 \ref{mapping}) 。随后,分别以 SCI 和 NP 数据集,对 sig-coDegs 基因集和 rTMS-DEGs 基因集进行关联分析 (Fig. \ref{fig:MAIN-fig-4}a、b)。取显著关联的基因 (见 Tab. \ref{tab:NP-sigCoDEGs-with-rTMS-DEGs-significant-correlation} 和 Tab. \ref{tab:SCI-sigCoDEGs-with-rTMS-DEGs-significant-correlation})。为了挖掘这些关联基因对于 rTMS 于 SCI 和 NP 的潜在功能,我们对这些基因进行了 GO 富集分析。NP 中的关联基因没有富集到通路,而 SCI 的富集结果如图 Fig. \ref{fig:MAIN-fig-4}c 所示。可以发现,这些基因主要与免疫细胞 (lymphocyte、leukocyte、B cell、T cell) 的行为相关,涉及免疫反应 。可以推测,rTMS 对于 SCI 所致的 NP 的缓解作用可能与调节免疫反应相关。

注:The Cellular Component (CC), the Molecular Function (MF) and the Biological Process (BP).

Figure \ref{fig:MAIN-fig-4} (下方图) 为图MAIN fig 4概览。

\textbf{(对应文件为 \texttt{./Figure+Table/fig4.pdf})}

\def\@captype{figure}
\begin{center}
\includegraphics[width = 0.9\linewidth]{./Figure+Table/fig4.pdf}
\caption{MAIN fig 4}\label{fig:MAIN-fig-4}
\end{center}

\hypertarget{dis}{%
\section{结论}\label{dis}}

\begin{itemize}
\tightlist
\item
  本研究发现了一组 SCI 和 NP 中共同存在的 DEGs,并且这些基因相互显著关联。
\item
  rTMS 对大脑刺激会带来 Hippocampus 的转录变化,影响上述基因的转录;通路富集表明,影响的基因主要涉及免疫细胞行为和免疫反应 (Fig. \ref{fig:MAIN-fig-4}c)。
\end{itemize}

\hypertarget{workflow}{%
\section{附:分析流程}\label{workflow}}

\hypertarget{ux4e0dux5b8cux5168ux6027ux810aux9ad3ux635fux4f24-incomplete-spinal-cord-injury-isci-human}{%
\subsection{不完全性脊髓损伤 (Incomplete spinal cord injury, iSCI) (Human)}\label{ux4e0dux5b8cux5168ux6027ux810aux9ad3ux635fux4f24-incomplete-spinal-cord-injury-isci-human}}

\hypertarget{ux5143ux6570ux636e}{%
\subsubsection{元数据}\label{ux5143ux6570ux636e}}

\begin{itemize}
\tightlist
\item
  GSE226238
\end{itemize}

根据文献提供的数据整理信息\textsuperscript{\protect\hyperlink{ref-ProfilingImmunMorris2023}{8}}:

Complete: AIS A-B
Incomplete: AIS C-D

使用的样本的信息:

Table \ref{tab:SCI-used-sample-metadata} (下方表格) 为表格SCI used sample metadata概览。

\textbf{(对应文件为 \texttt{Figure+Table/SCI-used-sample-metadata.xlsx})}

\begin{center}\begin{tcolorbox}[colback=gray!10, colframe=gray!50, width=0.9\linewidth, arc=1mm, boxrule=0.5pt]注:表格共有19行12列,以下预览的表格可能省略部分数据;表格含有19个唯一`sample'。
\end{tcolorbox}
\end{center}
\begin{center}\begin{tcolorbox}[colback=gray!10, colframe=gray!50, width=0.9\linewidth, arc=1mm, boxrule=0.5pt]\begin{enumerate}\tightlist
\item sample: 样品名称
\item group: 分组名称
\end{enumerate}\end{tcolorbox}
\end{center}

\begin{longtable}[]{@{}llllllllll@{}}
\caption{\label{tab:SCI-used-sample-metadata}SCI used sample metadata}\tabularnewline
\toprule
sample & rownames & title & group.ch1 & tissue\ldots{} & treatm\ldots{} & group & id & status & AIS\tabularnewline
\midrule
\endfirsthead
\toprule
sample & rownames & title & group.ch1 & tissue\ldots{} & treatm\ldots{} & group & id & status & AIS\tabularnewline
\midrule
\endhead
ID13 & GSM706\ldots{} & ID13, \ldots{} & CTL & Whole \ldots{} & NA & control & NA & NA & NA\tabularnewline
ID16 & GSM706\ldots{} & ID16, \ldots{} & CTL & Whole \ldots{} & NA & control & NA & NA & NA\tabularnewline
ID14 & GSM706\ldots{} & ID14, \ldots{} & CTL & Whole \ldots{} & NA & control & NA & NA & NA\tabularnewline
ID15 & GSM706\ldots{} & ID15, \ldots{} & CTL & Whole \ldots{} & NA & control & NA & NA & NA\tabularnewline
ID17 & GSM706\ldots{} & ID17, \ldots{} & CTL & Whole \ldots{} & NA & control & NA & NA & NA\tabularnewline
ID1V0 & GSM706\ldots{} & ID1v0,\ldots{} & SCI & Whole \ldots{} & Acute & sci & 1 & 0 & D\tabularnewline
ID18 & GSM706\ldots{} & ID18, \ldots{} & CTL & Whole \ldots{} & NA & control & NA & NA & NA\tabularnewline
ID19 & GSM706\ldots{} & ID19, \ldots{} & CTL & Whole \ldots{} & NA & control & NA & NA & NA\tabularnewline
ID1V12 & GSM706\ldots{} & ID1v12\ldots{} & SCI & Whole \ldots{} & 12mpi & sci & 1 & 12 & D\tabularnewline
ID20 & GSM706\ldots{} & ID20, \ldots{} & CTL & Whole \ldots{} & NA & control & NA & NA & NA\tabularnewline
ID1V3 & GSM706\ldots{} & ID1v3,\ldots{} & SCI & Whole \ldots{} & 3mpi & sci & 1 & 3 & D\tabularnewline
ID1V6 & GSM706\ldots{} & ID1v6,\ldots{} & SCI & Whole \ldots{} & 6mpi & sci & 1 & 6 & D\tabularnewline
ID21 & GSM706\ldots{} & ID21, \ldots{} & CTL & Whole \ldots{} & NA & control & NA & NA & NA\tabularnewline
ID2V3 & GSM706\ldots{} & ID2v3,\ldots{} & SCI & Whole \ldots{} & 3mpi & sci & 2 & 3 & D\tabularnewline
ID2V0 & GSM706\ldots{} & ID2v0,\ldots{} & SCI & Whole \ldots{} & Acute & sci & 2 & 0 & D\tabularnewline
\ldots{} & \ldots{} & \ldots{} & \ldots{} & \ldots{} & \ldots{} & \ldots{} & \ldots{} & \ldots{} & \ldots{}\tabularnewline
\bottomrule
\end{longtable}

\hypertarget{ux5deeux5f02ux5206ux6790}{%
\subsubsection{差异分析}\label{ux5deeux5f02ux5206ux6790}}

Table \ref{tab:SCI-data-DEGs} (下方表格) 为表格SCI data DEGs概览。

\textbf{(对应文件为 \texttt{Figure+Table/SCI-data-DEGs.csv})}

\begin{center}\begin{tcolorbox}[colback=gray!10, colframe=gray!50, width=0.9\linewidth, arc=1mm, boxrule=0.5pt]注:表格共有3508行7列,以下预览的表格可能省略部分数据;表格含有3508个唯一`rownames'。
\end{tcolorbox}
\end{center}
\begin{center}\begin{tcolorbox}[colback=gray!10, colframe=gray!50, width=0.9\linewidth, arc=1mm, boxrule=0.5pt]\begin{enumerate}\tightlist
\item logFC: estimate of the log2-fold-changescorresponding to the effect or contrasts(for ‘topTableF’ there may be severalscolumns of log-fold-changes)
\item AveExpr: average log2-expression for the probesover all arrays and channels, same ass‘Amean’ in the ‘MarrayLM’ object
\item t: moderated t-statistic (omitted fors‘topTableF’)
\item P.Value: raw p-value
\item B: log-odds that the gene is differentiallysexpressed (omitted for ‘topTreat’)
\end{enumerate}\end{tcolorbox}
\end{center}

\begin{longtable}[]{@{}lllllll@{}}
\caption{\label{tab:SCI-data-DEGs}SCI data DEGs}\tabularnewline
\toprule
\begin{minipage}[b]{0.08\columnwidth}\raggedright
rownames\strut
\end{minipage} & \begin{minipage}[b]{0.12\columnwidth}\raggedright
logFC\strut
\end{minipage} & \begin{minipage}[b]{0.12\columnwidth}\raggedright
AveExpr\strut
\end{minipage} & \begin{minipage}[b]{0.12\columnwidth}\raggedright
t\strut
\end{minipage} & \begin{minipage}[b]{0.12\columnwidth}\raggedright
P.Value\strut
\end{minipage} & \begin{minipage}[b]{0.12\columnwidth}\raggedright
adj.P.Val\strut
\end{minipage} & \begin{minipage}[b]{0.12\columnwidth}\raggedright
B\strut
\end{minipage}\tabularnewline
\midrule
\endfirsthead
\toprule
\begin{minipage}[b]{0.08\columnwidth}\raggedright
rownames\strut
\end{minipage} & \begin{minipage}[b]{0.12\columnwidth}\raggedright
logFC\strut
\end{minipage} & \begin{minipage}[b]{0.12\columnwidth}\raggedright
AveExpr\strut
\end{minipage} & \begin{minipage}[b]{0.12\columnwidth}\raggedright
t\strut
\end{minipage} & \begin{minipage}[b]{0.12\columnwidth}\raggedright
P.Value\strut
\end{minipage} & \begin{minipage}[b]{0.12\columnwidth}\raggedright
adj.P.Val\strut
\end{minipage} & \begin{minipage}[b]{0.12\columnwidth}\raggedright
B\strut
\end{minipage}\tabularnewline
\midrule
\endhead
\begin{minipage}[t]{0.08\columnwidth}\raggedright
SLC66A1\strut
\end{minipage} & \begin{minipage}[t]{0.12\columnwidth}\raggedright
-5.3985318\ldots{}\strut
\end{minipage} & \begin{minipage}[t]{0.12\columnwidth}\raggedright
-2.8574726\ldots{}\strut
\end{minipage} & \begin{minipage}[t]{0.12\columnwidth}\raggedright
-22.211843\ldots{}\strut
\end{minipage} & \begin{minipage}[t]{0.12\columnwidth}\raggedright
2.05919519\ldots{}\strut
\end{minipage} & \begin{minipage}[t]{0.12\columnwidth}\raggedright
1.47232456\ldots{}\strut
\end{minipage} & \begin{minipage}[t]{0.12\columnwidth}\raggedright
24.6914363\ldots{}\strut
\end{minipage}\tabularnewline
\begin{minipage}[t]{0.08\columnwidth}\raggedright
CDKN1C\strut
\end{minipage} & \begin{minipage}[t]{0.12\columnwidth}\raggedright
-6.4116063\ldots{}\strut
\end{minipage} & \begin{minipage}[t]{0.12\columnwidth}\raggedright
-2.3451973\ldots{}\strut
\end{minipage} & \begin{minipage}[t]{0.12\columnwidth}\raggedright
-18.884709\ldots{}\strut
\end{minipage} & \begin{minipage}[t]{0.12\columnwidth}\raggedright
4.38737275\ldots{}\strut
\end{minipage} & \begin{minipage}[t]{0.12\columnwidth}\raggedright
1.14693897\ldots{}\strut
\end{minipage} & \begin{minipage}[t]{0.12\columnwidth}\raggedright
22.0223243\ldots{}\strut
\end{minipage}\tabularnewline
\begin{minipage}[t]{0.08\columnwidth}\raggedright
PGAM1\strut
\end{minipage} & \begin{minipage}[t]{0.12\columnwidth}\raggedright
3.931229\strut
\end{minipage} & \begin{minipage}[t]{0.12\columnwidth}\raggedright
4.39159789\ldots{}\strut
\end{minipage} & \begin{minipage}[t]{0.12\columnwidth}\raggedright
18.7916191\ldots{}\strut
\end{minipage} & \begin{minipage}[t]{0.12\columnwidth}\raggedright
4.81233135\ldots{}\strut
\end{minipage} & \begin{minipage}[t]{0.12\columnwidth}\raggedright
1.14693897\ldots{}\strut
\end{minipage} & \begin{minipage}[t]{0.12\columnwidth}\raggedright
21.9396078\ldots{}\strut
\end{minipage}\tabularnewline
\begin{minipage}[t]{0.08\columnwidth}\raggedright
KLHL26\strut
\end{minipage} & \begin{minipage}[t]{0.12\columnwidth}\raggedright
-6.3060655\ldots{}\strut
\end{minipage} & \begin{minipage}[t]{0.12\columnwidth}\raggedright
-2.3125193\ldots{}\strut
\end{minipage} & \begin{minipage}[t]{0.12\columnwidth}\raggedright
-17.087291\ldots{}\strut
\end{minipage} & \begin{minipage}[t]{0.12\columnwidth}\raggedright
2.82330652\ldots{}\strut
\end{minipage} & \begin{minipage}[t]{0.12\columnwidth}\raggedright
5.04666040\ldots{}\strut
\end{minipage} & \begin{minipage}[t]{0.12\columnwidth}\raggedright
20.3361194\ldots{}\strut
\end{minipage}\tabularnewline
\begin{minipage}[t]{0.08\columnwidth}\raggedright
48\strut
\end{minipage} & \begin{minipage}[t]{0.12\columnwidth}\raggedright
-5.4354855\ldots{}\strut
\end{minipage} & \begin{minipage}[t]{0.12\columnwidth}\raggedright
-2.3299163\ldots{}\strut
\end{minipage} & \begin{minipage}[t]{0.12\columnwidth}\raggedright
-16.327301\ldots{}\strut
\end{minipage} & \begin{minipage}[t]{0.12\columnwidth}\raggedright
6.53810955\ldots{}\strut
\end{minipage} & \begin{minipage}[t]{0.12\columnwidth}\raggedright
9.34949665\ldots{}\strut
\end{minipage} & \begin{minipage}[t]{0.12\columnwidth}\raggedright
19.5624017\ldots{}\strut
\end{minipage}\tabularnewline
\begin{minipage}[t]{0.08\columnwidth}\raggedright
120\strut
\end{minipage} & \begin{minipage}[t]{0.12\columnwidth}\raggedright
-2.9235652\ldots{}\strut
\end{minipage} & \begin{minipage}[t]{0.12\columnwidth}\raggedright
1.65380258\ldots{}\strut
\end{minipage} & \begin{minipage}[t]{0.12\columnwidth}\raggedright
-15.788203\ldots{}\strut
\end{minipage} & \begin{minipage}[t]{0.12\columnwidth}\raggedright
1.21118622\ldots{}\strut
\end{minipage} & \begin{minipage}[t]{0.12\columnwidth}\raggedright
1.31385918\ldots{}\strut
\end{minipage} & \begin{minipage}[t]{0.12\columnwidth}\raggedright
18.9895854\ldots{}\strut
\end{minipage}\tabularnewline
\begin{minipage}[t]{0.08\columnwidth}\raggedright
DUSP23\strut
\end{minipage} & \begin{minipage}[t]{0.12\columnwidth}\raggedright
-6.9434252\ldots{}\strut
\end{minipage} & \begin{minipage}[t]{0.12\columnwidth}\raggedright
-1.6984921\ldots{}\strut
\end{minipage} & \begin{minipage}[t]{0.12\columnwidth}\raggedright
-15.736427\ldots{}\strut
\end{minipage} & \begin{minipage}[t]{0.12\columnwidth}\raggedright
1.28629570\ldots{}\strut
\end{minipage} & \begin{minipage}[t]{0.12\columnwidth}\raggedright
1.31385918\ldots{}\strut
\end{minipage} & \begin{minipage}[t]{0.12\columnwidth}\raggedright
18.9334793\ldots{}\strut
\end{minipage}\tabularnewline
\begin{minipage}[t]{0.08\columnwidth}\raggedright
HIC1\strut
\end{minipage} & \begin{minipage}[t]{0.12\columnwidth}\raggedright
-5.417897\strut
\end{minipage} & \begin{minipage}[t]{0.12\columnwidth}\raggedright
-2.4574677\ldots{}\strut
\end{minipage} & \begin{minipage}[t]{0.12\columnwidth}\raggedright
-15.114200\ldots{}\strut
\end{minipage} & \begin{minipage}[t]{0.12\columnwidth}\raggedright
2.68695492\ldots{}\strut
\end{minipage} & \begin{minipage}[t]{0.12\columnwidth}\raggedright
2.40146596\ldots{}\strut
\end{minipage} & \begin{minipage}[t]{0.12\columnwidth}\raggedright
18.2436942\ldots{}\strut
\end{minipage}\tabularnewline
\begin{minipage}[t]{0.08\columnwidth}\raggedright
FAM229A\strut
\end{minipage} & \begin{minipage}[t]{0.12\columnwidth}\raggedright
-5.5729105\ldots{}\strut
\end{minipage} & \begin{minipage}[t]{0.12\columnwidth}\raggedright
-2.1258822\ldots{}\strut
\end{minipage} & \begin{minipage}[t]{0.12\columnwidth}\raggedright
-14.626022\ldots{}\strut
\end{minipage} & \begin{minipage}[t]{0.12\columnwidth}\raggedright
4.87741832\ldots{}\strut
\end{minipage} & \begin{minipage}[t]{0.12\columnwidth}\raggedright
3.87483788\ldots{}\strut
\end{minipage} & \begin{minipage}[t]{0.12\columnwidth}\raggedright
17.6817026\ldots{}\strut
\end{minipage}\tabularnewline
\begin{minipage}[t]{0.08\columnwidth}\raggedright
METTL26\strut
\end{minipage} & \begin{minipage}[t]{0.12\columnwidth}\raggedright
-5.5332784\ldots{}\strut
\end{minipage} & \begin{minipage}[t]{0.12\columnwidth}\raggedright
-2.1951593\ldots{}\strut
\end{minipage} & \begin{minipage}[t]{0.12\columnwidth}\raggedright
-13.639833\ldots{}\strut
\end{minipage} & \begin{minipage}[t]{0.12\columnwidth}\raggedright
1.71439162\ldots{}\strut
\end{minipage} & \begin{minipage}[t]{0.12\columnwidth}\raggedright
1.22579001\ldots{}\strut
\end{minipage} & \begin{minipage}[t]{0.12\columnwidth}\raggedright
16.4867721\ldots{}\strut
\end{minipage}\tabularnewline
\begin{minipage}[t]{0.08\columnwidth}\raggedright
CLTB\strut
\end{minipage} & \begin{minipage}[t]{0.12\columnwidth}\raggedright
5.10489364\ldots{}\strut
\end{minipage} & \begin{minipage}[t]{0.12\columnwidth}\raggedright
-0.3518483\ldots{}\strut
\end{minipage} & \begin{minipage}[t]{0.12\columnwidth}\raggedright
12.8635600\ldots{}\strut
\end{minipage} & \begin{minipage}[t]{0.12\columnwidth}\raggedright
4.86472129\ldots{}\strut
\end{minipage} & \begin{minipage}[t]{0.12\columnwidth}\raggedright
3.16206884\ldots{}\strut
\end{minipage} & \begin{minipage}[t]{0.12\columnwidth}\raggedright
15.4859145\ldots{}\strut
\end{minipage}\tabularnewline
\begin{minipage}[t]{0.08\columnwidth}\raggedright
LSP1\strut
\end{minipage} & \begin{minipage}[t]{0.12\columnwidth}\raggedright
5.033352\strut
\end{minipage} & \begin{minipage}[t]{0.12\columnwidth}\raggedright
3.10275063\ldots{}\strut
\end{minipage} & \begin{minipage}[t]{0.12\columnwidth}\raggedright
12.3121841\ldots{}\strut
\end{minipage} & \begin{minipage}[t]{0.12\columnwidth}\raggedright
1.05255843\ldots{}\strut
\end{minipage} & \begin{minipage}[t]{0.12\columnwidth}\raggedright
6.27149402\ldots{}\strut
\end{minipage} & \begin{minipage}[t]{0.12\columnwidth}\raggedright
14.7403014\ldots{}\strut
\end{minipage}\tabularnewline
\begin{minipage}[t]{0.08\columnwidth}\raggedright
HADH\strut
\end{minipage} & \begin{minipage}[t]{0.12\columnwidth}\raggedright
-4.9657345\ldots{}\strut
\end{minipage} & \begin{minipage}[t]{0.12\columnwidth}\raggedright
-2.8300059\ldots{}\strut
\end{minipage} & \begin{minipage}[t]{0.12\columnwidth}\raggedright
-12.247863\ldots{}\strut
\end{minipage} & \begin{minipage}[t]{0.12\columnwidth}\raggedright
1.15375141\ldots{}\strut
\end{minipage} & \begin{minipage}[t]{0.12\columnwidth}\raggedright
6.34563279\ldots{}\strut
\end{minipage} & \begin{minipage}[t]{0.12\columnwidth}\raggedright
14.6513584\ldots{}\strut
\end{minipage}\tabularnewline
\begin{minipage}[t]{0.08\columnwidth}\raggedright
METTL21A\strut
\end{minipage} & \begin{minipage}[t]{0.12\columnwidth}\raggedright
-4.6597681\ldots{}\strut
\end{minipage} & \begin{minipage}[t]{0.12\columnwidth}\raggedright
-2.8250994\ldots{}\strut
\end{minipage} & \begin{minipage}[t]{0.12\columnwidth}\raggedright
-12.164725\ldots{}\strut
\end{minipage} & \begin{minipage}[t]{0.12\columnwidth}\raggedright
1.29982020\ldots{}\strut
\end{minipage} & \begin{minipage}[t]{0.12\columnwidth}\raggedright
6.63836746\ldots{}\strut
\end{minipage} & \begin{minipage}[t]{0.12\columnwidth}\raggedright
14.5357743\ldots{}\strut
\end{minipage}\tabularnewline
\begin{minipage}[t]{0.08\columnwidth}\raggedright
LRRC24\strut
\end{minipage} & \begin{minipage}[t]{0.12\columnwidth}\raggedright
-4.2782732\ldots{}\strut
\end{minipage} & \begin{minipage}[t]{0.12\columnwidth}\raggedright
-2.4806355\ldots{}\strut
\end{minipage} & \begin{minipage}[t]{0.12\columnwidth}\raggedright
-12.046955\ldots{}\strut
\end{minipage} & \begin{minipage}[t]{0.12\columnwidth}\raggedright
1.54061013\ldots{}\strut
\end{minipage} & \begin{minipage}[t]{0.12\columnwidth}\raggedright
7.12568004\ldots{}\strut
\end{minipage} & \begin{minipage}[t]{0.12\columnwidth}\raggedright
14.3708323\ldots{}\strut
\end{minipage}\tabularnewline
\begin{minipage}[t]{0.08\columnwidth}\raggedright
\ldots{}\strut
\end{minipage} & \begin{minipage}[t]{0.12\columnwidth}\raggedright
\ldots{}\strut
\end{minipage} & \begin{minipage}[t]{0.12\columnwidth}\raggedright
\ldots{}\strut
\end{minipage} & \begin{minipage}[t]{0.12\columnwidth}\raggedright
\ldots{}\strut
\end{minipage} & \begin{minipage}[t]{0.12\columnwidth}\raggedright
\ldots{}\strut
\end{minipage} & \begin{minipage}[t]{0.12\columnwidth}\raggedright
\ldots{}\strut
\end{minipage} & \begin{minipage}[t]{0.12\columnwidth}\raggedright
\ldots{}\strut
\end{minipage}\tabularnewline
\bottomrule
\end{longtable}

Figure \ref{fig:SCI-sci-vs-control-DEGs} (下方图) 为图SCI sci vs control DEGs概览。

\textbf{(对应文件为 \texttt{Figure+Table/SCI-sci-vs-control-DEGs.pdf})}

\def\@captype{figure}
\begin{center}
\includegraphics[width = 0.9\linewidth]{Figure+Table/SCI-sci-vs-control-DEGs.pdf}
\caption{SCI sci vs control DEGs}\label{fig:SCI-sci-vs-control-DEGs}
\end{center}

\hypertarget{ux795eux7ecfux75c5ux7406ux6027ux75bcux75db-neuropathic-pain-np-human}{%
\subsection{神经病理性疼痛 (neuropathic pain, NP) (Human)}\label{ux795eux7ecfux75c5ux7406ux6027ux75bcux75db-neuropathic-pain-np-human}}

\hypertarget{ux5143ux6570ux636e-1}{%
\subsubsection{元数据}\label{ux5143ux6570ux636e-1}}

Table \ref{tab:NP-metadata} (下方表格) 为表格NP metadata概览。

\textbf{(对应文件为 \texttt{Figure+Table/NP-metadata.csv})}

\begin{center}\begin{tcolorbox}[colback=gray!10, colframe=gray!50, width=0.9\linewidth, arc=1mm, boxrule=0.5pt]注:表格共有14行7列,以下预览的表格可能省略部分数据;表格含有14个唯一`rownames'。
\end{tcolorbox}
\end{center}
\begin{center}\begin{tcolorbox}[colback=gray!10, colframe=gray!50, width=0.9\linewidth, arc=1mm, boxrule=0.5pt]\begin{enumerate}\tightlist
\item sample: 样品名称
\item group: 分组名称
\end{enumerate}\end{tcolorbox}
\end{center}

\begin{longtable}[]{@{}lllllll@{}}
\caption{\label{tab:NP-metadata}NP metadata}\tabularnewline
\toprule
\begin{minipage}[b]{0.12\columnwidth}\raggedright
rownames\strut
\end{minipage} & \begin{minipage}[b]{0.07\columnwidth}\raggedright
group\strut
\end{minipage} & \begin{minipage}[b]{0.13\columnwidth}\raggedright
lib.size\strut
\end{minipage} & \begin{minipage}[b]{0.13\columnwidth}\raggedright
norm.factors\strut
\end{minipage} & \begin{minipage}[b]{0.12\columnwidth}\raggedright
sample\strut
\end{minipage} & \begin{minipage}[b]{0.12\columnwidth}\raggedright
title\strut
\end{minipage} & \begin{minipage}[b]{0.13\columnwidth}\raggedright
tissue.ch1\strut
\end{minipage}\tabularnewline
\midrule
\endfirsthead
\toprule
\begin{minipage}[b]{0.12\columnwidth}\raggedright
rownames\strut
\end{minipage} & \begin{minipage}[b]{0.07\columnwidth}\raggedright
group\strut
\end{minipage} & \begin{minipage}[b]{0.13\columnwidth}\raggedright
lib.size\strut
\end{minipage} & \begin{minipage}[b]{0.13\columnwidth}\raggedright
norm.factors\strut
\end{minipage} & \begin{minipage}[b]{0.12\columnwidth}\raggedright
sample\strut
\end{minipage} & \begin{minipage}[b]{0.12\columnwidth}\raggedright
title\strut
\end{minipage} & \begin{minipage}[b]{0.13\columnwidth}\raggedright
tissue.ch1\strut
\end{minipage}\tabularnewline
\midrule
\endhead
\begin{minipage}[t]{0.12\columnwidth}\raggedright
Control\_Rep1\strut
\end{minipage} & \begin{minipage}[t]{0.07\columnwidth}\raggedright
Control\strut
\end{minipage} & \begin{minipage}[t]{0.13\columnwidth}\raggedright
37694075.5\ldots{}\strut
\end{minipage} & \begin{minipage}[t]{0.13\columnwidth}\raggedright
0.99341849\ldots{}\strut
\end{minipage} & \begin{minipage}[t]{0.12\columnwidth}\raggedright
Control\_Rep1\strut
\end{minipage} & \begin{minipage}[t]{0.12\columnwidth}\raggedright
Control\_Rep1\strut
\end{minipage} & \begin{minipage}[t]{0.13\columnwidth}\raggedright
white bloo\ldots{}\strut
\end{minipage}\tabularnewline
\begin{minipage}[t]{0.12\columnwidth}\raggedright
Control\_Rep2\strut
\end{minipage} & \begin{minipage}[t]{0.07\columnwidth}\raggedright
Control\strut
\end{minipage} & \begin{minipage}[t]{0.13\columnwidth}\raggedright
35123400.4\ldots{}\strut
\end{minipage} & \begin{minipage}[t]{0.13\columnwidth}\raggedright
0.98433148\ldots{}\strut
\end{minipage} & \begin{minipage}[t]{0.12\columnwidth}\raggedright
Control\_Rep2\strut
\end{minipage} & \begin{minipage}[t]{0.12\columnwidth}\raggedright
Control\_Rep2\strut
\end{minipage} & \begin{minipage}[t]{0.13\columnwidth}\raggedright
white bloo\ldots{}\strut
\end{minipage}\tabularnewline
\begin{minipage}[t]{0.12\columnwidth}\raggedright
Control\_Rep3\strut
\end{minipage} & \begin{minipage}[t]{0.07\columnwidth}\raggedright
Control\strut
\end{minipage} & \begin{minipage}[t]{0.13\columnwidth}\raggedright
40623787.2\ldots{}\strut
\end{minipage} & \begin{minipage}[t]{0.13\columnwidth}\raggedright
1.03883062\ldots{}\strut
\end{minipage} & \begin{minipage}[t]{0.12\columnwidth}\raggedright
Control\_Rep3\strut
\end{minipage} & \begin{minipage}[t]{0.12\columnwidth}\raggedright
Control\_Rep3\strut
\end{minipage} & \begin{minipage}[t]{0.13\columnwidth}\raggedright
white bloo\ldots{}\strut
\end{minipage}\tabularnewline
\begin{minipage}[t]{0.12\columnwidth}\raggedright
Control\_Rep4\strut
\end{minipage} & \begin{minipage}[t]{0.07\columnwidth}\raggedright
Control\strut
\end{minipage} & \begin{minipage}[t]{0.13\columnwidth}\raggedright
31254712.0\ldots{}\strut
\end{minipage} & \begin{minipage}[t]{0.13\columnwidth}\raggedright
0.90039899\ldots{}\strut
\end{minipage} & \begin{minipage}[t]{0.12\columnwidth}\raggedright
Control\_Rep4\strut
\end{minipage} & \begin{minipage}[t]{0.12\columnwidth}\raggedright
Control\_Rep4\strut
\end{minipage} & \begin{minipage}[t]{0.13\columnwidth}\raggedright
white bloo\ldots{}\strut
\end{minipage}\tabularnewline
\begin{minipage}[t]{0.12\columnwidth}\raggedright
Control\_Rep5\strut
\end{minipage} & \begin{minipage}[t]{0.07\columnwidth}\raggedright
Control\strut
\end{minipage} & \begin{minipage}[t]{0.13\columnwidth}\raggedright
36785783.3\ldots{}\strut
\end{minipage} & \begin{minipage}[t]{0.13\columnwidth}\raggedright
1.09239376\ldots{}\strut
\end{minipage} & \begin{minipage}[t]{0.12\columnwidth}\raggedright
Control\_Rep5\strut
\end{minipage} & \begin{minipage}[t]{0.12\columnwidth}\raggedright
Control\_Rep5\strut
\end{minipage} & \begin{minipage}[t]{0.13\columnwidth}\raggedright
white bloo\ldots{}\strut
\end{minipage}\tabularnewline
\begin{minipage}[t]{0.12\columnwidth}\raggedright
NL.0\_Rep1\strut
\end{minipage} & \begin{minipage}[t]{0.07\columnwidth}\raggedright
NL.0\strut
\end{minipage} & \begin{minipage}[t]{0.13\columnwidth}\raggedright
39063868.2\ldots{}\strut
\end{minipage} & \begin{minipage}[t]{0.13\columnwidth}\raggedright
0.99869084\ldots{}\strut
\end{minipage} & \begin{minipage}[t]{0.12\columnwidth}\raggedright
NL.0\_Rep1\strut
\end{minipage} & \begin{minipage}[t]{0.12\columnwidth}\raggedright
NL-0\_Rep1\strut
\end{minipage} & \begin{minipage}[t]{0.13\columnwidth}\raggedright
white bloo\ldots{}\strut
\end{minipage}\tabularnewline
\begin{minipage}[t]{0.12\columnwidth}\raggedright
NL.0\_Rep2\strut
\end{minipage} & \begin{minipage}[t]{0.07\columnwidth}\raggedright
NL.0\strut
\end{minipage} & \begin{minipage}[t]{0.13\columnwidth}\raggedright
30209440.6\ldots{}\strut
\end{minipage} & \begin{minipage}[t]{0.13\columnwidth}\raggedright
0.83816556\ldots{}\strut
\end{minipage} & \begin{minipage}[t]{0.12\columnwidth}\raggedright
NL.0\_Rep2\strut
\end{minipage} & \begin{minipage}[t]{0.12\columnwidth}\raggedright
NL-0\_Rep2\strut
\end{minipage} & \begin{minipage}[t]{0.13\columnwidth}\raggedright
white bloo\ldots{}\strut
\end{minipage}\tabularnewline
\begin{minipage}[t]{0.12\columnwidth}\raggedright
NL.0\_Rep3\strut
\end{minipage} & \begin{minipage}[t]{0.07\columnwidth}\raggedright
NL.0\strut
\end{minipage} & \begin{minipage}[t]{0.13\columnwidth}\raggedright
33541489.3\ldots{}\strut
\end{minipage} & \begin{minipage}[t]{0.13\columnwidth}\raggedright
0.87392250\ldots{}\strut
\end{minipage} & \begin{minipage}[t]{0.12\columnwidth}\raggedright
NL.0\_Rep3\strut
\end{minipage} & \begin{minipage}[t]{0.12\columnwidth}\raggedright
NL-0\_Rep3\strut
\end{minipage} & \begin{minipage}[t]{0.13\columnwidth}\raggedright
white bloo\ldots{}\strut
\end{minipage}\tabularnewline
\begin{minipage}[t]{0.12\columnwidth}\raggedright
NL.0\_Rep4\strut
\end{minipage} & \begin{minipage}[t]{0.07\columnwidth}\raggedright
NL.0\strut
\end{minipage} & \begin{minipage}[t]{0.13\columnwidth}\raggedright
43997974.4\ldots{}\strut
\end{minipage} & \begin{minipage}[t]{0.13\columnwidth}\raggedright
1.10515311\ldots{}\strut
\end{minipage} & \begin{minipage}[t]{0.12\columnwidth}\raggedright
NL.0\_Rep4\strut
\end{minipage} & \begin{minipage}[t]{0.12\columnwidth}\raggedright
NL-0\_Rep4\strut
\end{minipage} & \begin{minipage}[t]{0.13\columnwidth}\raggedright
white bloo\ldots{}\strut
\end{minipage}\tabularnewline
\begin{minipage}[t]{0.12\columnwidth}\raggedright
NL.1\_Rep1\strut
\end{minipage} & \begin{minipage}[t]{0.07\columnwidth}\raggedright
NL.1\strut
\end{minipage} & \begin{minipage}[t]{0.13\columnwidth}\raggedright
35856170.7\ldots{}\strut
\end{minipage} & \begin{minipage}[t]{0.13\columnwidth}\raggedright
0.94729121\ldots{}\strut
\end{minipage} & \begin{minipage}[t]{0.12\columnwidth}\raggedright
NL.1\_Rep1\strut
\end{minipage} & \begin{minipage}[t]{0.12\columnwidth}\raggedright
NL-1\_Rep1\strut
\end{minipage} & \begin{minipage}[t]{0.13\columnwidth}\raggedright
white bloo\ldots{}\strut
\end{minipage}\tabularnewline
\begin{minipage}[t]{0.12\columnwidth}\raggedright
NL.1\_Rep2\strut
\end{minipage} & \begin{minipage}[t]{0.07\columnwidth}\raggedright
NL.1\strut
\end{minipage} & \begin{minipage}[t]{0.13\columnwidth}\raggedright
47176469.2\ldots{}\strut
\end{minipage} & \begin{minipage}[t]{0.13\columnwidth}\raggedright
1.21729671\ldots{}\strut
\end{minipage} & \begin{minipage}[t]{0.12\columnwidth}\raggedright
NL.1\_Rep2\strut
\end{minipage} & \begin{minipage}[t]{0.12\columnwidth}\raggedright
NL-1\_Rep2\strut
\end{minipage} & \begin{minipage}[t]{0.13\columnwidth}\raggedright
white bloo\ldots{}\strut
\end{minipage}\tabularnewline
\begin{minipage}[t]{0.12\columnwidth}\raggedright
NL.1\_Rep3\strut
\end{minipage} & \begin{minipage}[t]{0.07\columnwidth}\raggedright
NL.1\strut
\end{minipage} & \begin{minipage}[t]{0.13\columnwidth}\raggedright
33852268.1\ldots{}\strut
\end{minipage} & \begin{minipage}[t]{0.13\columnwidth}\raggedright
0.94708847\ldots{}\strut
\end{minipage} & \begin{minipage}[t]{0.12\columnwidth}\raggedright
NL.1\_Rep3\strut
\end{minipage} & \begin{minipage}[t]{0.12\columnwidth}\raggedright
NL-1\_Rep3\strut
\end{minipage} & \begin{minipage}[t]{0.13\columnwidth}\raggedright
white bloo\ldots{}\strut
\end{minipage}\tabularnewline
\begin{minipage}[t]{0.12\columnwidth}\raggedright
NL.1\_Rep4\strut
\end{minipage} & \begin{minipage}[t]{0.07\columnwidth}\raggedright
NL.1\strut
\end{minipage} & \begin{minipage}[t]{0.13\columnwidth}\raggedright
37527290.3\ldots{}\strut
\end{minipage} & \begin{minipage}[t]{0.13\columnwidth}\raggedright
0.99176588\ldots{}\strut
\end{minipage} & \begin{minipage}[t]{0.12\columnwidth}\raggedright
NL.1\_Rep4\strut
\end{minipage} & \begin{minipage}[t]{0.12\columnwidth}\raggedright
NL-1\_Rep4\strut
\end{minipage} & \begin{minipage}[t]{0.13\columnwidth}\raggedright
white bloo\ldots{}\strut
\end{minipage}\tabularnewline
\begin{minipage}[t]{0.12\columnwidth}\raggedright
NL.1\_Rep5\strut
\end{minipage} & \begin{minipage}[t]{0.07\columnwidth}\raggedright
NL.1\strut
\end{minipage} & \begin{minipage}[t]{0.13\columnwidth}\raggedright
45702943.0\ldots{}\strut
\end{minipage} & \begin{minipage}[t]{0.13\columnwidth}\raggedright
1.14296121\ldots{}\strut
\end{minipage} & \begin{minipage}[t]{0.12\columnwidth}\raggedright
NL.1\_Rep5\strut
\end{minipage} & \begin{minipage}[t]{0.12\columnwidth}\raggedright
NL-1\_Rep5\strut
\end{minipage} & \begin{minipage}[t]{0.13\columnwidth}\raggedright
white bloo\ldots{}\strut
\end{minipage}\tabularnewline
\bottomrule
\end{longtable}

\hypertarget{ux5deeux5f02ux5206ux6790-1}{%
\subsubsection{差异分析}\label{ux5deeux5f02ux5206ux6790-1}}

Table \ref{tab:NP-data-NL-1-vs-Control-DEGs} (下方表格) 为表格NP data NL 1 vs Control DEGs概览。

\textbf{(对应文件为 \texttt{Figure+Table/NP-data-NL-1-vs-Control-DEGs.xlsx})}

\begin{center}\begin{tcolorbox}[colback=gray!10, colframe=gray!50, width=0.9\linewidth, arc=1mm, boxrule=0.5pt]注:表格共有263行17列,以下预览的表格可能省略部分数据;表格含有263个唯一`rownames'。
\end{tcolorbox}
\end{center}
\begin{center}\begin{tcolorbox}[colback=gray!10, colframe=gray!50, width=0.9\linewidth, arc=1mm, boxrule=0.5pt]\begin{enumerate}\tightlist
\item Chr: 染色体
\item Start: 起始点
\item End: 结束点
\item symbol: 基因或蛋白符号。
\item logFC: estimate of the log2-fold-changescorresponding to the effect or contrasts(for ‘topTableF’ there may be severalscolumns of log-fold-changes)
\item AveExpr: average log2-expression for the probesover all arrays and channels, same ass‘Amean’ in the ‘MarrayLM’ object
\item t: moderated t-statistic (omitted fors‘topTableF’)
\item P.Value: raw p-value
\item B: log-odds that the gene is differentiallysexpressed (omitted for ‘topTreat’)
\end{enumerate}\end{tcolorbox}
\end{center}

\begin{longtable}[]{@{}llllllllll@{}}
\caption{\label{tab:NP-data-NL-1-vs-Control-DEGs}NP data NL 1 vs Control DEGs}\tabularnewline
\toprule
rownames & symbol & Chr & Source & Feature & Start & End & Frame & Strand & V8\tabularnewline
\midrule
\endfirsthead
\toprule
rownames & symbol & Chr & Source & Feature & Start & End & Frame & Strand & V8\tabularnewline
\midrule
\endhead
1992 & ID2 & chr2 & HAVANA & gene & 8678845 & 8684453 & 0 & + & 0\tabularnewline
9282 & DDIT4 & chr10 & HAVANA & gene & 72273920 & 72276036 & 0 & + & 0\tabularnewline
17418 & MAP3K7CL & chr21 & HAVANA & gene & 29077471 & 29175889 & 0 & + & 0\tabularnewline
5167 & ENC1 & chr5 & HAVANA & gene & 74627406 & 74641424 & 0 & - & 0\tabularnewline
14324 & PER1 & chr17 & HAVANA & gene & 8140472 & 8156506 & 0 & - & 0\tabularnewline
6603 & TNFAIP3 & chr6 & HAVANA & gene & 137867188 & 137883312 & 0 & + & 0\tabularnewline
15786 & PDE4A & chr19 & HAVANA & gene & 10416773 & 10469631 & 0 & + & 0\tabularnewline
14577 & CCL4 & chr17 & HAVANA & gene & 36103590 & 36105621 & 0 & + & 0\tabularnewline
4463 & HOPX & chr4 & HAVANA & gene & 56647988 & 56681899 & 0 & - & 0\tabularnewline
10202 & MS4A14 & chr11 & HAVANA & gene & 60378530 & 60417756 & 0 & + & 0\tabularnewline
12419 & PTGDR & chr14 & HAVANA & gene & 52267713 & 52276724 & 0 & + & 0\tabularnewline
6372 & SH3BGRL2 & chr6 & HAVANA & gene & 79631283 & 79703659 & 0 & + & 0\tabularnewline
5266 & PAM & chr5 & HAVANA & gene & 102753981 & 103031105 & 0 & + & 0\tabularnewline
4533 & PPBP & chr4 & HAVANA & gene & 73987038 & 73988197 & 0 & - & 0\tabularnewline
6909 & MTURN & chr7 & HAVANA & gene & 30134810 & 30162762 & 0 & + & 0\tabularnewline
\ldots{} & \ldots{} & \ldots{} & \ldots{} & \ldots{} & \ldots{} & \ldots{} & \ldots{} & \ldots{} & \ldots{}\tabularnewline
\bottomrule
\end{longtable}

Figure \ref{fig:NP-NL-1-vs-Control-DEGs} (下方图) 为图NP NL 1 vs Control DEGs概览。

\textbf{(对应文件为 \texttt{Figure+Table/NP-NL-1-vs-Control-DEGs.pdf})}

\def\@captype{figure}
\begin{center}
\includegraphics[width = 0.9\linewidth]{Figure+Table/NP-NL-1-vs-Control-DEGs.pdf}
\caption{NP NL 1 vs Control DEGs}\label{fig:NP-NL-1-vs-Control-DEGs}
\end{center}

\hypertarget{sci-ux548c-np-ux5173ux8054ux5206ux6790-1}{%
\subsection{SCI 和 NP 关联分析}\label{sci-ux548c-np-ux5173ux8054ux5206ux6790-1}}

\hypertarget{ux5171ux540cux5deeux5f02ux57faux56e0-codegs}{%
\subsubsection{共同差异基因 coDEGs}\label{ux5171ux540cux5deeux5f02ux57faux56e0-codegs}}

Figure \ref{fig:SCI-NP-coDEGs} (下方图) 为图SCI NP coDEGs概览。

\textbf{(对应文件为 \texttt{Figure+Table/SCI-NP-coDEGs.pdf})}

\def\@captype{figure}
\begin{center}
\includegraphics[width = 0.9\linewidth]{Figure+Table/SCI-NP-coDEGs.pdf}
\caption{SCI NP coDEGs}\label{fig:SCI-NP-coDEGs}
\end{center}
\begin{center}\begin{tcolorbox}[colback=gray!10, colframe=gray!50, width=0.9\linewidth, arc=1mm, boxrule=0.5pt]
\textbf{
Intersection
:}

\vspace{0.5em}

    PDE4A, ADA, LIMS1, FLYWCH2, LDLR, GOLGA8N, NUDT2,
CNTLN, CTSS, KNOP1, PRF1, MAP3K7CL, CLDN5, SHISA8, CMC1,
SLAMF8, ELOVL7, AIFM3, GNGT2, PPT1, DAPP1, AOC1, PSMC1,
PF4, CBLN3, LPAR6, CPM, EGF, SH3PXD2A, SH3YL1, ATF3, PTCRA,
SMIM1, EVI2A

\vspace{2em}
\end{tcolorbox}
\end{center}

\textbf{(上述信息框内容已保存至 \texttt{Figure+Table/SCI-NP-coDEGs-content})}

\hypertarget{sci-ux7684-codegs-ux7684ux5173ux8054ux6027ux5206ux6790}{%
\subsubsection{SCI 的 coDEGs 的关联性分析}\label{sci-ux7684-codegs-ux7684ux5173ux8054ux6027ux5206ux6790}}

Figure \ref{fig:SCI-genes-correlation-heatmap} (下方图) 为图SCI genes correlation heatmap概览。

\textbf{(对应文件为 \texttt{Figure+Table/SCI-genes-correlation-heatmap.pdf})}

\def\@captype{figure}
\begin{center}
\includegraphics[width = 0.9\linewidth]{Figure+Table/SCI-genes-correlation-heatmap.pdf}
\caption{SCI genes correlation heatmap}\label{fig:SCI-genes-correlation-heatmap}
\end{center}

Table \ref{tab:SCI-data-significant-genes-of-correlation} (下方表格) 为表格SCI data significant genes of correlation概览。

\textbf{(对应文件为 \texttt{Figure+Table/SCI-data-significant-genes-of-correlation.csv})}

\begin{center}\begin{tcolorbox}[colback=gray!10, colframe=gray!50, width=0.9\linewidth, arc=1mm, boxrule=0.5pt]注:表格共有796行7列,以下预览的表格可能省略部分数据;表格含有34个唯一`From'。
\end{tcolorbox}
\end{center}
\begin{center}\begin{tcolorbox}[colback=gray!10, colframe=gray!50, width=0.9\linewidth, arc=1mm, boxrule=0.5pt]\begin{enumerate}\tightlist
\item cor: 皮尔逊关联系数,正关联或负关联。
\item pvalue: 显著性 P。
\item -log2(P.value): P 的对数转化。
\item significant: 显著性。
\item sign: 人为赋予的符号,参考 significant。
\end{enumerate}\end{tcolorbox}
\end{center}

\begin{longtable}[]{@{}lllllll@{}}
\caption{\label{tab:SCI-data-significant-genes-of-correlation}SCI data significant genes of correlation}\tabularnewline
\toprule
From & To & cor & pvalue & -log2(P.va\ldots{} & significant & sign\tabularnewline
\midrule
\endfirsthead
\toprule
From & To & cor & pvalue & -log2(P.va\ldots{} & significant & sign\tabularnewline
\midrule
\endhead
ADA & ADA & 1 & 0 & 16.6096404\ldots{} & \textless{} 0.001 & **\tabularnewline
PPT1 & ADA & -0.66 & 0.0022 & 8.82828076\ldots{} & \textless{} 0.05 & *\tabularnewline
LDLR & ADA & 0.86 & 0 & 16.6096404\ldots{} & \textless{} 0.001 & **\tabularnewline
EVI2A & ADA & -0.63 & 0.0036 & 8.11778737\ldots{} & \textless{} 0.05 & *\tabularnewline
CPM & ADA & -0.46 & 0.0452 & 4.46753341\ldots{} & \textless{} 0.05 & *\tabularnewline
KNOP1 & ADA & 0.69 & 0.0012 & 9.70274987\ldots{} & \textless{} 0.05 & *\tabularnewline
AIFM3 & ADA & 0.82 & 0 & 16.6096404\ldots{} & \textless{} 0.001 & **\tabularnewline
CBLN3 & ADA & 0.59 & 0.0077 & 7.02092583\ldots{} & \textless{} 0.05 & *\tabularnewline
PRF1 & ADA & 0.77 & 1e-04 & 13.2877123\ldots{} & \textless{} 0.001 & **\tabularnewline
AOC1 & ADA & 0.66 & 0.0019 & 9.03978486\ldots{} & \textless{} 0.05 & *\tabularnewline
PDE4A & ADA & 0.91 & 0 & 16.6096404\ldots{} & \textless{} 0.001 & **\tabularnewline
CNTLN & ADA & -0.65 & 0.0027 & 8.53282487\ldots{} & \textless{} 0.05 & *\tabularnewline
FLYWCH2 & ADA & -0.7 & 8e-04 & 10.2877123\ldots{} & \textless{} 0.001 & **\tabularnewline
SH3YL1 & ADA & 0.5 & 0.0301 & 5.05409270\ldots{} & \textless{} 0.05 & *\tabularnewline
NUDT2 & ADA & 0.69 & 0.0011 & 9.82828076\ldots{} & \textless{} 0.05 & *\tabularnewline
\ldots{} & \ldots{} & \ldots{} & \ldots{} & \ldots{} & \ldots{} & \ldots{}\tabularnewline
\bottomrule
\end{longtable}

\hypertarget{np-ux7684-codegs-ux7684ux5173ux8054ux6027ux5206ux6790}{%
\subsubsection{NP 的 coDEGs 的关联性分析}\label{np-ux7684-codegs-ux7684ux5173ux8054ux6027ux5206ux6790}}

Figure \ref{fig:NP-genes-correlation-heatmap} (下方图) 为图NP genes correlation heatmap概览。

\textbf{(对应文件为 \texttt{Figure+Table/NP-genes-correlation-heatmap.pdf})}

\def\@captype{figure}
\begin{center}
\includegraphics[width = 0.9\linewidth]{Figure+Table/NP-genes-correlation-heatmap.pdf}
\caption{NP genes correlation heatmap}\label{fig:NP-genes-correlation-heatmap}
\end{center}

Table \ref{tab:NP-data-significant-genes-of-correlation} (下方表格) 为表格NP data significant genes of correlation概览。

\textbf{(对应文件为 \texttt{Figure+Table/NP-data-significant-genes-of-correlation.csv})}

\begin{center}\begin{tcolorbox}[colback=gray!10, colframe=gray!50, width=0.9\linewidth, arc=1mm, boxrule=0.5pt]注:表格共有262行7列,以下预览的表格可能省略部分数据;表格含有34个唯一`From'。
\end{tcolorbox}
\end{center}
\begin{center}\begin{tcolorbox}[colback=gray!10, colframe=gray!50, width=0.9\linewidth, arc=1mm, boxrule=0.5pt]\begin{enumerate}\tightlist
\item cor: 皮尔逊关联系数,正关联或负关联。
\item pvalue: 显著性 P。
\item -log2(P.value): P 的对数转化。
\item significant: 显著性。
\item sign: 人为赋予的符号,参考 significant。
\end{enumerate}\end{tcolorbox}
\end{center}

\begin{longtable}[]{@{}lllllll@{}}
\caption{\label{tab:NP-data-significant-genes-of-correlation}NP data significant genes of correlation}\tabularnewline
\toprule
From & To & cor & pvalue & -log2(P.va\ldots{} & significant & sign\tabularnewline
\midrule
\endfirsthead
\toprule
From & To & cor & pvalue & -log2(P.va\ldots{} & significant & sign\tabularnewline
\midrule
\endhead
SMIM1 & SMIM1 & 1 & 0 & 16.6096404\ldots{} & \textless{} 0.001 & **\tabularnewline
CMC1 & SMIM1 & -0.56 & 0.0377 & 4.72929166\ldots{} & \textless{} 0.05 & *\tabularnewline
PF4 & SMIM1 & 0.58 & 0.0302 & 5.04930764\ldots{} & \textless{} 0.05 & *\tabularnewline
NUDT2 & SMIM1 & 0.59 & 0.0264 & 5.24331826\ldots{} & \textless{} 0.05 & *\tabularnewline
PPT1 & PPT1 & 1 & 0 & 16.6096404\ldots{} & \textless{} 0.001 & **\tabularnewline
CTSS & PPT1 & 0.66 & 0.01 & 6.64385618\ldots{} & \textless{} 0.05 & *\tabularnewline
DAPP1 & PPT1 & 0.58 & 0.028 & 5.15842936\ldots{} & \textless{} 0.05 & *\tabularnewline
SH3PXD2A & PPT1 & 0.62 & 0.0173 & 5.85308415\ldots{} & \textless{} 0.05 & *\tabularnewline
CBLN3 & PPT1 & -0.68 & 0.0069 & 7.17918792\ldots{} & \textless{} 0.05 & *\tabularnewline
PSMC1 & PPT1 & 0.64 & 0.0135 & 6.21089678\ldots{} & \textless{} 0.05 & *\tabularnewline
FLYWCH2 & PPT1 & -0.78 & 9e-04 & 10.1177873\ldots{} & \textless{} 0.001 & **\tabularnewline
GNGT2 & PPT1 & -0.74 & 0.0026 & 8.58727266\ldots{} & \textless{} 0.05 & *\tabularnewline
PPT1 & CTSS & 0.66 & 0.01 & 6.64385618\ldots{} & \textless{} 0.05 & *\tabularnewline
CTSS & CTSS & 1 & 0 & 16.6096404\ldots{} & \textless{} 0.001 & **\tabularnewline
DAPP1 & CTSS & 0.92 & 0 & 16.6096404\ldots{} & \textless{} 0.001 & **\tabularnewline
\ldots{} & \ldots{} & \ldots{} & \ldots{} & \ldots{} & \ldots{} & \ldots{}\tabularnewline
\bottomrule
\end{longtable}

\hypertarget{sigCoDEGs}{%
\subsubsection{SCI 和 NP 数据集共同显著关联的基因集 sigCoDEGs}\label{sigCoDEGs}}

\begin{center}\begin{tcolorbox}[colback=gray!10, colframe=gray!50, width=0.9\linewidth, arc=1mm, boxrule=0.5pt]
\textbf{
sig-coDEGs
:}

\vspace{0.5em}

    ADA, AIFM3, AOC1, ATF3, CBLN3, CLDN5, CMC1, CNTLN, CPM,
CTSS, DAPP1, EGF, ELOVL7, EVI2A, FLYWCH2, GNGT2, GOLGA8N,
KNOP1, LDLR, LIMS1, LPAR6, MAP3K7CL, NUDT2, PDE4A, PF4,
PPT1, PRF1, PSMC1, PTCRA, SH3PXD2A, SH3YL1, SHISA8, SLAMF8,
SMIM1

\vspace{2em}
\end{tcolorbox}
\end{center}

\hypertarget{ux91cdux590dux7ecfux9885ux78c1ux523aux6fc0ux6cbbux7597-repeat-transcranial-magnetic-stimulation-rtms-rat}{%
\subsection{重复经颅磁刺激治疗 (repeat transcranial magnetic stimulation, rTMS) (Rat)}\label{ux91cdux590dux7ecfux9885ux78c1ux523aux6fc0ux6cbbux7597-repeat-transcranial-magnetic-stimulation-rtms-rat}}

\hypertarget{ux5143ux6570ux636e-2}{%
\subsubsection{元数据}\label{ux5143ux6570ux636e-2}}

Table \ref{tab:rTMS-used-sample-metadata} (下方表格) 为表格rTMS used sample metadata概览。

\textbf{(对应文件为 \texttt{Figure+Table/rTMS-used-sample-metadata.csv})}

\begin{center}\begin{tcolorbox}[colback=gray!10, colframe=gray!50, width=0.9\linewidth, arc=1mm, boxrule=0.5pt]注:表格共有56行11列,以下预览的表格可能省略部分数据;表格含有12个唯一`group'。
\end{tcolorbox}
\end{center}
\begin{center}\begin{tcolorbox}[colback=gray!10, colframe=gray!50, width=0.9\linewidth, arc=1mm, boxrule=0.5pt]\begin{enumerate}\tightlist
\item sample: 样品名称
\item group: 分组名称
\end{enumerate}\end{tcolorbox}
\end{center}

\begin{longtable}[]{@{}lllllllllll@{}}
\caption{\label{tab:rTMS-used-sample-metadata}RTMS used sample metadata}\tabularnewline
\toprule
sample & group & title & age.ch1 & cognit\ldots{} & post.t\ldots{} & Sex.ch1 & strain\ldots{} & tissue\ldots\ldots9 & tissue\ldots\ldots10 & \ldots{}\tabularnewline
\midrule
\endfirsthead
\toprule
sample & group & title & age.ch1 & cognit\ldots{} & post.t\ldots{} & Sex.ch1 & strain\ldots{} & tissue\ldots\ldots9 & tissue\ldots\ldots10 & \ldots{}\tabularnewline
\midrule
\endhead
GSM718\ldots{} & Y.H.Sham & Y\_Sham\ldots{} & Young;\ldots{} & unimpa\ldots{} & 48 & Male & Long-E\ldots{} & Hippoc\ldots{} & Brain & \ldots{}\tabularnewline
GSM718\ldots{} & Y.H.Sham & Y\_Sham\ldots{} & Young;\ldots{} & unimpa\ldots{} & 48 & Male & Long-E\ldots{} & Hippoc\ldots{} & Brain & \ldots{}\tabularnewline
GSM718\ldots{} & Y.H.Sham & Y\_Sham\ldots{} & Young;\ldots{} & unimpa\ldots{} & 48 & Male & Long-E\ldots{} & Hippoc\ldots{} & Brain & \ldots{}\tabularnewline
GSM718\ldots{} & Y.H.Sham & Y\_Sham\ldots{} & Young;\ldots{} & unimpa\ldots{} & 48 & Male & Long-E\ldots{} & Hippoc\ldots{} & Brain & \ldots{}\tabularnewline
GSM718\ldots{} & Y.H.iTBS & Y\_iTBS\ldots{} & Young;\ldots{} & unimpa\ldots{} & 48 & Male & Long-E\ldots{} & Hippoc\ldots{} & Brain & \ldots{}\tabularnewline
GSM718\ldots{} & Y.H.iTBS & Y\_iTBS\ldots{} & Young;\ldots{} & unimpa\ldots{} & 48 & Male & Long-E\ldots{} & Hippoc\ldots{} & Brain & \ldots{}\tabularnewline
GSM718\ldots{} & Y.H.iTBS & Y\_iTBS\ldots{} & Young;\ldots{} & unimpa\ldots{} & 48 & Male & Long-E\ldots{} & Hippoc\ldots{} & Brain & \ldots{}\tabularnewline
GSM718\ldots{} & Y.H.iTBS & Y\_iTBS\ldots{} & Young;\ldots{} & unimpa\ldots{} & 48 & Male & Long-E\ldots{} & Hippoc\ldots{} & Brain & \ldots{}\tabularnewline
GSM718\ldots{} & AU.H.Sham & AU\_Sha\ldots{} & Aged;2\ldots{} & unimpa\ldots{} & 48 & Male & Long-E\ldots{} & Hippoc\ldots{} & Brain & \ldots{}\tabularnewline
GSM718\ldots{} & AU.H.Sham & AU\_Sha\ldots{} & Aged;2\ldots{} & unimpa\ldots{} & 48 & Male & Long-E\ldots{} & Hippoc\ldots{} & Brain & \ldots{}\tabularnewline
GSM718\ldots{} & AU.H.Sham & AU\_Sha\ldots{} & Aged;2\ldots{} & unimpa\ldots{} & 48 & Male & Long-E\ldots{} & Hippoc\ldots{} & Brain & \ldots{}\tabularnewline
GSM718\ldots{} & AU.H.Sham & AU\_Sha\ldots{} & Aged;2\ldots{} & unimpa\ldots{} & 48 & Male & Long-E\ldots{} & Hippoc\ldots{} & Brain & \ldots{}\tabularnewline
GSM718\ldots{} & AU.H.iTBS & AU\_iTB\ldots{} & Aged;2\ldots{} & unimpa\ldots{} & 48 & Male & Long-E\ldots{} & Hippoc\ldots{} & Brain & \ldots{}\tabularnewline
GSM718\ldots{} & AU.H.iTBS & AU\_iTB\ldots{} & Aged;2\ldots{} & unimpa\ldots{} & 48 & Male & Long-E\ldots{} & Hippoc\ldots{} & Brain & \ldots{}\tabularnewline
GSM718\ldots{} & AU.H.iTBS & AU\_iTB\ldots{} & Aged;2\ldots{} & unimpa\ldots{} & 48 & Male & Long-E\ldots{} & Hippoc\ldots{} & Brain & \ldots{}\tabularnewline
\ldots{} & \ldots{} & \ldots{} & \ldots{} & \ldots{} & \ldots{} & \ldots{} & \ldots{} & \ldots{} & \ldots{} & \ldots{}\tabularnewline
\bottomrule
\end{longtable}

\hypertarget{ux5deeux5f02ux5206ux6790-2}{%
\subsubsection{差异分析}\label{ux5deeux5f02ux5206ux6790-2}}

Figure \ref{fig:rTMS-All-DEGs-of-contrasts} (下方图) 为图rTMS All DEGs of contrasts概览。

\textbf{(对应文件为 \texttt{Figure+Table/rTMS-All-DEGs-of-contrasts.pdf})}

\def\@captype{figure}
\begin{center}
\includegraphics[width = 0.9\linewidth]{Figure+Table/rTMS-All-DEGs-of-contrasts.pdf}
\caption{RTMS All DEGs of contrasts}\label{fig:rTMS-All-DEGs-of-contrasts}
\end{center}

`RTMS data DEGs' 数据已全部提供。

\textbf{(对应文件为 \texttt{Figure+Table/rTMS-data-DEGs})}

\begin{center}\begin{tcolorbox}[colback=gray!10, colframe=gray!50, width=0.9\linewidth, arc=1mm, boxrule=0.5pt]注:文件夹Figure+Table/rTMS-data-DEGs共包含6个文件。

\begin{enumerate}\tightlist
\item 1\_AU.H.iTBS - AU.H.Sham.csv
\item 2\_AU.P.iTBS - AU.P.iTBS.csv
\item 3\_AI.H.iTBS - AI.H.Sham.csv
\item 4\_AI.P.iTBS - AI.P.iTBS.csv
\item 5\_Y.H.iTBS - Y.H.Sham.csv
\item ...
\end{enumerate}\end{tcolorbox}
\end{center}

\hypertarget{rtms-ux548c-scinp-ux7684ux5173ux8054-1}{%
\subsection{rTMS 和 SCI、NP 的关联}\label{rtms-ux548c-scinp-ux7684ux5173ux8054-1}}

\hypertarget{mapping}{%
\subsubsection{rTMS 数据的差异基因与 sig.coDegs 的关联性}\label{mapping}}

这里,首先将 rTMS 的差异基因对应到 SCI、NP 中的基因 (假设 rTMS 对大鼠和对人的基因转录的影响是相同的,改变相同的基因) :

`Mapping rTMS DEGs to SCI and NP dataset' 数据已全部提供。

\textbf{(对应文件为 \texttt{Figure+Table/Mapping-rTMS-DEGs-to-SCI-and-NP-dataset})}

\begin{center}\begin{tcolorbox}[colback=gray!10, colframe=gray!50, width=0.9\linewidth, arc=1mm, boxrule=0.5pt]注:文件夹Figure+Table/Mapping-rTMS-DEGs-to-SCI-and-NP-dataset共包含2个文件。

\begin{enumerate}\tightlist
\item 1\_gene.np.csv
\item 2\_gene.sci.csv
\end{enumerate}\end{tcolorbox}
\end{center}

随后,进行关联分析:

Figure \ref{fig:NP-sigCoDEGs-with-rTMS-DEGs-correlation-heatmap} (下方图) 为图NP sigCoDEGs with rTMS DEGs correlation heatmap概览。

\textbf{(对应文件为 \texttt{Figure+Table/NP-sigCoDEGs-with-rTMS-DEGs-correlation-heatmap.pdf})}

\def\@captype{figure}
\begin{center}
\includegraphics[width = 0.9\linewidth]{Figure+Table/NP-sigCoDEGs-with-rTMS-DEGs-correlation-heatmap.pdf}
\caption{NP sigCoDEGs with rTMS DEGs correlation heatmap}\label{fig:NP-sigCoDEGs-with-rTMS-DEGs-correlation-heatmap}
\end{center}

Table \ref{tab:NP-sigCoDEGs-with-rTMS-DEGs-significant-correlation} (下方表格) 为表格NP sigCoDEGs with rTMS DEGs significant correlation概览。

\textbf{(对应文件为 \texttt{Figure+Table/NP-sigCoDEGs-with-rTMS-DEGs-significant-correlation.csv})}

\begin{center}\begin{tcolorbox}[colback=gray!10, colframe=gray!50, width=0.9\linewidth, arc=1mm, boxrule=0.5pt]注:表格共有297行7列,以下预览的表格可能省略部分数据;表格含有34个唯一`From'。
\end{tcolorbox}
\end{center}
\begin{center}\begin{tcolorbox}[colback=gray!10, colframe=gray!50, width=0.9\linewidth, arc=1mm, boxrule=0.5pt]\begin{enumerate}\tightlist
\item cor: 皮尔逊关联系数,正关联或负关联。
\item pvalue: 显著性 P。
\item -log2(P.value): P 的对数转化。
\item significant: 显著性。
\item sign: 人为赋予的符号,参考 significant。
\end{enumerate}\end{tcolorbox}
\end{center}

\begin{longtable}[]{@{}lllllll@{}}
\caption{\label{tab:NP-sigCoDEGs-with-rTMS-DEGs-significant-correlation}NP sigCoDEGs with rTMS DEGs significant correlation}\tabularnewline
\toprule
From & To & cor & pvalue & -log2(P.va\ldots{} & significant & sign\tabularnewline
\midrule
\endfirsthead
\toprule
From & To & cor & pvalue & -log2(P.va\ldots{} & significant & sign\tabularnewline
\midrule
\endhead
CMC1 & UBXN10 & 0.58 & 0.0311 & 5.00694160\ldots{} & \textless{} 0.05 & *\tabularnewline
SH3PXD2A & UBXN10 & -0.64 & 0.0142 & 6.13796526\ldots{} & \textless{} 0.05 & *\tabularnewline
CTSS & LDLRAP1 & -0.61 & 0.0202 & 5.62950089\ldots{} & \textless{} 0.05 & *\tabularnewline
SHISA8 & LDLRAP1 & 0.54 & 0.047 & 4.41119543\ldots{} & \textless{} 0.05 & *\tabularnewline
SH3YL1 & ZFP69 & 0.55 & 0.0417 & 4.58380880\ldots{} & \textless{} 0.05 & *\tabularnewline
CPM & ZFP69 & 0.56 & 0.0381 & 4.71406519\ldots{} & \textless{} 0.05 & *\tabularnewline
LPAR6 & ZFP69 & 0.54 & 0.0451 & 4.47072875\ldots{} & \textless{} 0.05 & *\tabularnewline
CLDN5 & ZFP69 & -0.57 & 0.0332 & 4.91267294\ldots{} & \textless{} 0.05 & *\tabularnewline
CMC1 & FOXD2 & 0.58 & 0.0305 & 5.03504694\ldots{} & \textless{} 0.05 & *\tabularnewline
PRF1 & FOXD2 & 0.6 & 0.0222 & 5.49329651\ldots{} & \textless{} 0.05 & *\tabularnewline
GNGT2 & FOXD2 & 0.62 & 0.0186 & 5.74855356\ldots{} & \textless{} 0.05 & *\tabularnewline
PDE4A & FOXD2 & 0.58 & 0.0299 & 5.06371070\ldots{} & \textless{} 0.05 & *\tabularnewline
ADA & FOXD2 & 0.61 & 0.02 & 5.64385618\ldots{} & \textless{} 0.05 & *\tabularnewline
PPT1 & PTGFR & 0.59 & 0.0254 & 5.29902769\ldots{} & \textless{} 0.05 & *\tabularnewline
CTSS & PTGFR & 0.79 & 7e-04 & 10.4803574\ldots{} & \textless{} 0.001 & **\tabularnewline
\ldots{} & \ldots{} & \ldots{} & \ldots{} & \ldots{} & \ldots{} & \ldots{}\tabularnewline
\bottomrule
\end{longtable}

Figure \ref{fig:SCI-sigCoDEGs-with-rTMS-DEGs-correlation-heatmap} (下方图) 为图SCI sigCoDEGs with rTMS DEGs correlation heatmap概览。

\textbf{(对应文件为 \texttt{Figure+Table/SCI-sigCoDEGs-with-rTMS-DEGs-correlation-heatmap.pdf})}

\def\@captype{figure}
\begin{center}
\includegraphics[width = 0.9\linewidth]{Figure+Table/SCI-sigCoDEGs-with-rTMS-DEGs-correlation-heatmap.pdf}
\caption{SCI sigCoDEGs with rTMS DEGs correlation heatmap}\label{fig:SCI-sigCoDEGs-with-rTMS-DEGs-correlation-heatmap}
\end{center}

Table \ref{tab:SCI-sigCoDEGs-with-rTMS-DEGs-significant-correlation} (下方表格) 为表格SCI sigCoDEGs with rTMS DEGs significant correlation概览。

\textbf{(对应文件为 \texttt{Figure+Table/SCI-sigCoDEGs-with-rTMS-DEGs-significant-correlation.csv})}

\begin{center}\begin{tcolorbox}[colback=gray!10, colframe=gray!50, width=0.9\linewidth, arc=1mm, boxrule=0.5pt]注:表格共有406行7列,以下预览的表格可能省略部分数据;表格含有34个唯一`From'。
\end{tcolorbox}
\end{center}
\begin{center}\begin{tcolorbox}[colback=gray!10, colframe=gray!50, width=0.9\linewidth, arc=1mm, boxrule=0.5pt]\begin{enumerate}\tightlist
\item cor: 皮尔逊关联系数,正关联或负关联。
\item pvalue: 显著性 P。
\item -log2(P.value): P 的对数转化。
\item significant: 显著性。
\item sign: 人为赋予的符号,参考 significant。
\end{enumerate}\end{tcolorbox}
\end{center}

\begin{longtable}[]{@{}lllllll@{}}
\caption{\label{tab:SCI-sigCoDEGs-with-rTMS-DEGs-significant-correlation}SCI sigCoDEGs with rTMS DEGs significant correlation}\tabularnewline
\toprule
From & To & cor & pvalue & -log2(P.va\ldots{} & significant & sign\tabularnewline
\midrule
\endfirsthead
\toprule
From & To & cor & pvalue & -log2(P.va\ldots{} & significant & sign\tabularnewline
\midrule
\endhead
PPT1 & CD40LG & -0.54 & 0.0173 & 5.85308415\ldots{} & \textless{} 0.05 & *\tabularnewline
CPM & CD40LG & -0.52 & 0.0214 & 5.54624539\ldots{} & \textless{} 0.05 & *\tabularnewline
KNOP1 & CD40LG & 0.54 & 0.017 & 5.87832144\ldots{} & \textless{} 0.05 & *\tabularnewline
CNTLN & CD40LG & -0.49 & 0.0331 & 4.91702497\ldots{} & \textless{} 0.05 & *\tabularnewline
SH3YL1 & CD40LG & 0.62 & 0.0046 & 7.76415042\ldots{} & \textless{} 0.05 & *\tabularnewline
LIMS1 & CD40LG & -0.5 & 0.028 & 5.15842936\ldots{} & \textless{} 0.05 & *\tabularnewline
MAP3K7CL & CD40LG & -0.52 & 0.0231 & 5.43596333\ldots{} & \textless{} 0.05 & *\tabularnewline
SH3PXD2A & CD40LG & 0.61 & 0.0052 & 7.58727266\ldots{} & \textless{} 0.05 & *\tabularnewline
ADA & F5 & -0.47 & 0.0425 & 4.55639334\ldots{} & \textless{} 0.05 & *\tabularnewline
LDLR & F5 & -0.52 & 0.0228 & 5.45482236\ldots{} & \textless{} 0.05 & *\tabularnewline
CPM & F5 & 0.52 & 0.021 & 5.57346686\ldots{} & \textless{} 0.05 & *\tabularnewline
KNOP1 & F5 & -0.7 & 8e-04 & 10.2877123\ldots{} & \textless{} 0.001 & **\tabularnewline
PDE4A & F5 & -0.5 & 0.0285 & 5.13289427\ldots{} & \textless{} 0.05 & *\tabularnewline
CNTLN & F5 & 0.63 & 0.0036 & 8.11778737\ldots{} & \textless{} 0.05 & *\tabularnewline
FLYWCH2 & F5 & 0.52 & 0.0237 & 5.39896913\ldots{} & \textless{} 0.05 & *\tabularnewline
\ldots{} & \ldots{} & \ldots{} & \ldots{} & \ldots{} & \ldots{} & \ldots{}\tabularnewline
\bottomrule
\end{longtable}

\hypertarget{ux5bccux96c6ux5206ux6790}{%
\subsubsection{富集分析}\label{ux5bccux96c6ux5206ux6790}}

Figure \ref{fig:RTMS-SCI-NP-correlated-sci-GO-enrichment} (下方图) 为图RTMS SCI NP correlated sci GO enrichment概览。

\textbf{(对应文件为 \texttt{Figure+Table/RTMS-SCI-NP-correlated-sci-GO-enrichment.pdf})}

\def\@captype{figure}
\begin{center}
\includegraphics[width = 0.9\linewidth]{Figure+Table/RTMS-SCI-NP-correlated-sci-GO-enrichment.pdf}
\caption{RTMS SCI NP correlated sci GO enrichment}\label{fig:RTMS-SCI-NP-correlated-sci-GO-enrichment}
\end{center}

\hypertarget{bibliography}{%
\section*{Reference}\label{bibliography}}
\addcontentsline{toc}{section}{Reference}

\hypertarget{refs}{}
\begin{cslreferences}
\leavevmode\hypertarget{ref-TheDualRoleOSunC2023}{}%
1. Sun, C. \emph{et al.} The dual role of microglia in neuropathic pain after spinal cord injury: Detrimental and protective effects. \emph{Experimental neurology} \textbf{370}, (2023).

\leavevmode\hypertarget{ref-TranscranialMaNiZh2015}{}%
2. Ni, Z. \& Chen, R. Transcranial magnetic stimulation to understand pathophysiology and as potential treatment for neurodegenerative diseases. \emph{Translational neurodegeneration} \textbf{4}, (2015).

\leavevmode\hypertarget{ref-DifferentialEfAmeli2009}{}%
3. Ameli, M. \emph{et al.} Differential effects of high-frequency repetitive transcranial magnetic stimulation over ipsilesional primary motor cortex in cortical and subcortical middle cerebral artery stroke. \emph{Annals of neurology} \textbf{66}, 298--309 (2009).

\leavevmode\hypertarget{ref-ReductionOfInHiraya2006}{}%
4. Hirayama, A. \emph{et al.} Reduction of intractable deafferentation pain by navigation-guided repetitive transcranial magnetic stimulation of the primary motor cortex. \emph{Pain} \textbf{122}, 22--27 (2006).

\leavevmode\hypertarget{ref-ClusterprofilerWuTi2021}{}%
5. Wu, T. \emph{et al.} ClusterProfiler 4.0: A universal enrichment tool for interpreting omics data. \emph{The Innovation} \textbf{2}, (2021).

\leavevmode\hypertarget{ref-LimmaPowersDiRitchi2015}{}%
6. Ritchie, M. E. \emph{et al.} Limma powers differential expression analyses for rna-sequencing and microarray studies. \emph{Nucleic Acids Research} \textbf{43}, e47 (2015).

\leavevmode\hypertarget{ref-EdgerDifferenChen}{}%
7. Chen, Y., McCarthy, D., Ritchie, M., Robinson, M. \& Smyth, G. EdgeR: Differential analysis of sequence read count data user's guide. 119.

\leavevmode\hypertarget{ref-ProfilingImmunMorris2023}{}%
8. Morrison, D. \emph{et al.} Profiling immunological phenotypes in individuals during the first year after traumatic spinal cord injury: A longitudinal analysis. \emph{Journal of Neurotrauma} \textbf{40}, 2621--2637 (2023).
\end{cslreferences}

\end{document}
