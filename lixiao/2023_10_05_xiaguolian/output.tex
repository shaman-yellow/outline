% Options for packages loaded elsewhere
\PassOptionsToPackage{unicode}{hyperref}
\PassOptionsToPackage{hyphens}{url}
%
\documentclass[
]{article}
\usepackage{lmodern}
\usepackage{amssymb,amsmath}
\usepackage{ifxetex,ifluatex}
\ifnum 0\ifxetex 1\fi\ifluatex 1\fi=0 % if pdftex
  \usepackage[T1]{fontenc}
  \usepackage[utf8]{inputenc}
  \usepackage{textcomp} % provide euro and other symbols
\else % if luatex or xetex
  \usepackage{unicode-math}
  \defaultfontfeatures{Scale=MatchLowercase}
  \defaultfontfeatures[\rmfamily]{Ligatures=TeX,Scale=1}
\fi
% Use upquote if available, for straight quotes in verbatim environments
\IfFileExists{upquote.sty}{\usepackage{upquote}}{}
\IfFileExists{microtype.sty}{% use microtype if available
  \usepackage[]{microtype}
  \UseMicrotypeSet[protrusion]{basicmath} % disable protrusion for tt fonts
}{}
\makeatletter
\@ifundefined{KOMAClassName}{% if non-KOMA class
  \IfFileExists{parskip.sty}{%
    \usepackage{parskip}
  }{% else
    \setlength{\parindent}{0pt}
    \setlength{\parskip}{6pt plus 2pt minus 1pt}}
}{% if KOMA class
  \KOMAoptions{parskip=half}}
\makeatother
\usepackage{xcolor}
\IfFileExists{xurl.sty}{\usepackage{xurl}}{} % add URL line breaks if available
\IfFileExists{bookmark.sty}{\usepackage{bookmark}}{\usepackage{hyperref}}
\hypersetup{
  pdftitle={Analysis},
  pdfauthor={Huang LiChuang of Wie-Biotech},
  hidelinks,
  pdfcreator={LaTeX via pandoc}}
\urlstyle{same} % disable monospaced font for URLs
\usepackage[margin=1in]{geometry}
\usepackage{longtable,booktabs}
% Correct order of tables after \paragraph or \subparagraph
\usepackage{etoolbox}
\makeatletter
\patchcmd\longtable{\par}{\if@noskipsec\mbox{}\fi\par}{}{}
\makeatother
% Allow footnotes in longtable head/foot
\IfFileExists{footnotehyper.sty}{\usepackage{footnotehyper}}{\usepackage{footnote}}
\makesavenoteenv{longtable}
\usepackage{graphicx}
\makeatletter
\def\maxwidth{\ifdim\Gin@nat@width>\linewidth\linewidth\else\Gin@nat@width\fi}
\def\maxheight{\ifdim\Gin@nat@height>\textheight\textheight\else\Gin@nat@height\fi}
\makeatother
% Scale images if necessary, so that they will not overflow the page
% margins by default, and it is still possible to overwrite the defaults
% using explicit options in \includegraphics[width, height, ...]{}
\setkeys{Gin}{width=\maxwidth,height=\maxheight,keepaspectratio}
% Set default figure placement to htbp
\makeatletter
\def\fps@figure{htbp}
\makeatother
\setlength{\emergencystretch}{3em} % prevent overfull lines
\providecommand{\tightlist}{%
  \setlength{\itemsep}{0pt}\setlength{\parskip}{0pt}}
\setcounter{secnumdepth}{5}
\usepackage{caption} \captionsetup{font={footnotesize},width=6in} \renewcommand{\dblfloatpagefraction}{.9} \makeatletter \renewenvironment{figure} {\def\@captype{figure}} \makeatother \definecolor{shadecolor}{RGB}{242,242,242} \usepackage{xeCJK} \usepackage{setspace} \setstretch{1.3} \usepackage{tcolorbox}

\title{Analysis}
\author{Huang LiChuang of Wie-Biotech}
\date{}

\begin{document}
\maketitle

{
\setcounter{tocdepth}{3}
\tableofcontents
}
\listoffigures

\listoftables

\hypertarget{abstract}{%
\section{摘要}\label{abstract}}

筛选丹参酮治疗脓毒症(sepsis)的关键差异表达基因及相关信号通路。

\hypertarget{methods}{%
\section{材料和方法}\label{methods}}

测序数据:Caco-2细胞系,对照组con,脂多糖组LPS,丹参酮组TNA(LPS+TNA)。

GEO 数据:GSE237861

\hypertarget{route}{%
\section{研究设计流程图}\label{route}}

\includegraphics[width=\linewidth]{output_files/figure-latex/unnamed-chunk-4-1}

\hypertarget{results}{%
\section{分析结果}\label{results}}

单以测序数据集筛选到 1797 个靶点,富集分析聚焦到 Hippo 通路(Fig. \ref{fig:genes-enriched-in-hippo-signiling-pathway})。

以 GEO 数据 GSE237861 分析发现,6 种不同组织的 sepsis 病例存在 51 个共同的差异表达基因(Disease vs control)。进一步分析发现:无同时存在于 6 或 5 种组织的Hippo 通路基因(同时也是 Tanshinone IIA 的作用靶点);\emph{BIRC3}、\emph{ID1} 在 4 种组织中差异表达;\emph{DLG4} 在 3 种组织中差异表达(Fig. \ref{fig:Target-genes-of-TNA-in-mutiple-tissue-of-sepsis-of-Hippo-pathway})。分子对接显示,SMAD7, SOX2, TGFBR2, DLG4, DLG2 具有良好亲和度(Fig. \ref{fig:docking-affinity})。综上,\emph{DLG4} 在 3 种sepsis 组织中差异表达,且 DLG4 可与 Tanshinone IIA 结合,因此,\emph{DLG4} 可能是 TNA 治疗 sepsis 的关键靶点之一,对应信号通路为 Hippo。

\hypertarget{dis}{%
\section{结论}\label{dis}}

\emph{DLG4} 可能是 TNA 治疗 sepsis 的关键靶点,相关信号通路为 Hippo。

\hypertarget{ux9644ux5206ux6790ux6d41ux7a0b}{%
\section{附:分析流程}\label{ux9644ux5206ux6790ux6d41ux7a0b}}

\hypertarget{ux6d4bux5e8fux6570ux636e}{%
\subsection{测序数据}\label{ux6d4bux5e8fux6570ux636e}}

\hypertarget{ux5deeux5f02ux5206ux6790}{%
\subsubsection{差异分析}\label{ux5deeux5f02ux5206ux6790}}

Figure \ref{fig:Low-expression-filtering}为图Low expression filtering概览。

\textbf{(对应文件为 \texttt{Figure+Table/Low-expression-filtering.pdf})}

\def\@captype{figure}
\begin{center}
\includegraphics[width = 0.9\linewidth]{Figure+Table/Low-expression-filtering.pdf}
\caption{Low expression filtering}\label{fig:Low-expression-filtering}
\end{center}

Figure \ref{fig:expression-normalization}为图expression normalization概览。

\textbf{(对应文件为 \texttt{Figure+Table/expression-normalization.pdf})}

\def\@captype{figure}
\begin{center}
\includegraphics[width = 0.9\linewidth]{Figure+Table/expression-normalization.pdf}
\caption{Expression normalization}\label{fig:expression-normalization}
\end{center}

Figure \ref{fig:DEGs-of-model-versus-control}为图DEGs of model versus control概览。

\textbf{(对应文件为 \texttt{Figure+Table/DEGs-of-model-versus-control.pdf})}

\def\@captype{figure}
\begin{center}
\includegraphics[width = 0.9\linewidth]{Figure+Table/DEGs-of-model-versus-control.pdf}
\caption{DEGs of model versus control}\label{fig:DEGs-of-model-versus-control}
\end{center}

Figure \ref{fig:DEGs-of-treatment-versus-model}为图DEGs of treatment versus model概览。

\textbf{(对应文件为 \texttt{Figure+Table/DEGs-of-treatment-versus-model.pdf})}

\def\@captype{figure}
\begin{center}
\includegraphics[width = 0.9\linewidth]{Figure+Table/DEGs-of-treatment-versus-model.pdf}
\caption{DEGs of treatment versus model}\label{fig:DEGs-of-treatment-versus-model}
\end{center}

丹参酮的疗效有两种情况:

\begin{itemize}
\tightlist
\item
  模型组相比对照组,基因上调;而以丹参酮处理后,基因下调(相比于模型组)。
\item
  模型组相比对照组,基因下调;而以丹参酮处理后,基因上调(相比于模型组)。
\end{itemize}

Figure \ref{fig:intersection-of-disease-genes-expression-and-treatment-effect-of-TNA}为图intersection of disease genes expression and treatment effect of TNA概览。

\textbf{(对应文件为 \texttt{Figure+Table/intersection-of-disease-genes-expression-and-treatment-effect-of-TNA.pdf})}

\def\@captype{figure}
\begin{center}
\includegraphics[width = 0.9\linewidth]{Figure+Table/intersection-of-disease-genes-expression-and-treatment-effect-of-TNA.pdf}
\caption{Intersection of disease genes expression and treatment effect of TNA}\label{fig:intersection-of-disease-genes-expression-and-treatment-effect-of-TNA}
\end{center}

取 Fig. \ref{fig:intersection-of-disease-genes-expression-and-treatment-effect-of-TNA} 的两组交集的合集(989 + 808),。

\hypertarget{ux5bccux96c6ux5206ux6790}{%
\subsubsection{富集分析}\label{ux5bccux96c6ux5206ux6790}}

以上述合集做富集分析。

Figure \ref{fig:KEGG-enrichment}为图KEGG enrichment概览。

\textbf{(对应文件为 \texttt{Figure+Table/KEGG-enrichment.pdf})}

\def\@captype{figure}
\begin{center}
\includegraphics[width = 0.9\linewidth]{Figure+Table/KEGG-enrichment.pdf}
\caption{KEGG enrichment}\label{fig:KEGG-enrichment}
\end{center}

Hippo 通路为显著富集通路。

Figure \ref{fig:genes-enriched-in-hippo-signiling-pathway}为图genes enriched in hippo signiling pathway概览。

\textbf{(对应文件为 \texttt{Figure+Table/hsa04390.pathview.png})}

\def\@captype{figure}
\begin{center}
\includegraphics[width = 0.9\linewidth]{pathview2023-10-06_15_12_57.141235/hsa04390.pathview.png}
\caption{Genes enriched in hippo signiling pathway}\label{fig:genes-enriched-in-hippo-signiling-pathway}
\end{center}

\hypertarget{geo-sepsis}{%
\subsection{GEO sepsis}\label{geo-sepsis}}

\hypertarget{gse237861-transcriptome-analysis-of-six-tissues-obtained-post-mortem-from-sepsis-patients}{%
\subsubsection{GSE237861: Transcriptome analysis of six tissues obtained post mortem from sepsis patients}\label{gse237861-transcriptome-analysis-of-six-tissues-obtained-post-mortem-from-sepsis-patients}}

\begin{center}\begin{tcolorbox}[colback=gray!10, colframe=gray!50, width=0.9\linewidth, arc=1mm, boxrule=0.5pt]
\textbf{
data\_processing
:}

\vspace{0.5em}

    The libraries were quantified by Qubit dsDNA High
Sensitivity Assay Kit (Life Technologies Corporation,
Carlsbad, CA, United States) and the median sizes were
determined by TapeStation 4200 (Agilent Technologies, USA),
using the High Sensitivity D1000 ScreenTape assay, to form
an equimolar pool.

\vspace{2em}


\textbf{
data\_processing.1
:}

\vspace{0.5em}

    Sequencing was performed as a 75-bp single-read,
single-index run on a NextSeq 500 next-generation sequencer
(Illumina, San Diego, CA, United States) with High Output
kit.

\vspace{2em}


\textbf{
data\_processing.2
:}

\vspace{0.5em}

    Quality control analysis was performed using FastQC
software, showing a Phred value superior to 30

\vspace{2em}


\textbf{
data\_processing.3
:}

\vspace{0.5em}

    Trimmomatic software was used to trim low-quality reads
and adapters. Raw reads were aligned to the hg38 reference
through HISAT2 software. Quantification of the gene
expression data was performed through the function
featureCounts of the R package Rsubread and the counts were
normalized according to log2CPM.

\vspace{2em}


\textbf{
data\_processing.4
:}

\vspace{0.5em}

    Differential expression analysis was performed by the R
package edgeR (FDR < 0.1 was considered significant),
comparing each male patient with sepsis with all male
uninfected controls and the female patients with sepsis
with all female uninfected controls.

\vspace{2em}


\textbf{
data\_processing.5
:}

\vspace{0.5em}

    Assembly: hg38

\vspace{2em}


\textbf{
data\_processing.6
:}

\vspace{0.5em}

    Supplementary files format and content: tab-delimited
text file contains results of differential expression
analysis in edgeR

\vspace{2em}


\textbf{
data\_processing.7
:}

\vspace{0.5em}

    Supplementary files format and content: columns
indicate gene ID, logFc, p-value and FDR

\vspace{2em}
\end{tcolorbox}
\end{center}

Table \ref{tab:metadata-of-GSE237861}为表格metadata of GSE237861概览。

\textbf{(对应文件为 \texttt{Figure+Table/metadata-of-GSE237861.csv})}

\begin{center}\begin{tcolorbox}[colback=gray!10, colframe=gray!50, width=0.9\linewidth, arc=1mm, boxrule=0.5pt]注:表格共有82行2列,以下预览的表格可能省略部分数据;表格含有82个唯一`title'。
\end{tcolorbox}
\end{center}

\begin{longtable}[]{@{}ll@{}}
\caption{\label{tab:metadata-of-GSE237861}Metadata of GSE237861}\tabularnewline
\toprule
title & tissu\ldots{}\tabularnewline
\midrule
\endfirsthead
\toprule
title & tissu\ldots{}\tabularnewline
\midrule
\endhead
sepsi\ldots{} & prefr\ldots{}\tabularnewline
sepsi\ldots{} & hippo\ldots{}\tabularnewline
sepsi\ldots{} & heart\tabularnewline
sepsi\ldots{} & lung\tabularnewline
sepsi\ldots{} & kidney\tabularnewline
sepsi\ldots{} & colon\tabularnewline
sepsi\ldots{} & brain\ldots{}\tabularnewline
sepsi\ldots{} & hippo\ldots{}\tabularnewline
sepsi\ldots{} & heart\tabularnewline
sepsi\ldots{} & lung\tabularnewline
sepsi\ldots{} & kidney\tabularnewline
sepsi\ldots{} & brain\ldots{}\tabularnewline
sepsi\ldots{} & hippo\ldots{}\tabularnewline
sepsi\ldots{} & heart\tabularnewline
sepsi\ldots{} & lung\tabularnewline
\ldots{} & \ldots{}\tabularnewline
\bottomrule
\end{longtable}

Figure \ref{fig:DEGs-number-in-sepsis-of-mutiple-tissue-of-GEO-dataset}为图DEGs number in sepsis of mutiple tissue of GEO dataset概览。

\textbf{(对应文件为 \texttt{Figure+Table/DEGs-number-in-sepsis-of-mutiple-tissue-of-GEO-dataset.pdf})}

\def\@captype{figure}
\begin{center}
\includegraphics[width = 0.9\linewidth]{Figure+Table/DEGs-number-in-sepsis-of-mutiple-tissue-of-GEO-dataset.pdf}
\caption{DEGs number in sepsis of mutiple tissue of GEO dataset}\label{fig:DEGs-number-in-sepsis-of-mutiple-tissue-of-GEO-dataset}
\end{center}

在六种不同的 sepsis 组织中,共有 51 个共同的交集基因(Fig. \ref{fig:intersection-of-DEGs-of-mutiple-tissue-of-sepsis})。

Figure \ref{fig:intersection-of-DEGs-of-mutiple-tissue-of-sepsis}为图intersection of DEGs of mutiple tissue of sepsis概览。

\textbf{(对应文件为 \texttt{Figure+Table/intersection-of-DEGs-of-mutiple-tissue-of-sepsis.pdf})}

\def\@captype{figure}
\begin{center}
\includegraphics[width = 0.9\linewidth]{Figure+Table/intersection-of-DEGs-of-mutiple-tissue-of-sepsis.pdf}
\caption{Intersection of DEGs of mutiple tissue of sepsis}\label{fig:intersection-of-DEGs-of-mutiple-tissue-of-sepsis}
\end{center}

\hypertarget{ux6574ux5408ux6d4bux5e8fux6570ux636eux548c-geo-ux6570ux636e}{%
\subsection{整合:测序数据和 GEO 数据}\label{ux6574ux5408ux6d4bux5e8fux6570ux636eux548c-geo-ux6570ux636e}}

\hypertarget{ux5173ux8054ux57faux56e0}{%
\subsubsection{关联基因}\label{ux5173ux8054ux57faux56e0}}

以 GSE237861 验证 TNA 作用的 Hippo 通路基因,属于 sepsis 哪些组织的差异表达基因,以确认 TNA 是否对其具有疗效。

\begin{itemize}
\tightlist
\item
  BIRC3、ID1 在 4 种组织中差异表达
\item
  DLG4 在 3 种组织中差异表达
\item
  \ldots{}
\end{itemize}

Figure \ref{fig:Target-genes-of-TNA-in-mutiple-tissue-of-sepsis-of-Hippo-pathway}为图Target genes of TNA in mutiple tissue of sepsis of Hippo pathway概览。

\textbf{(对应文件为 \texttt{Figure+Table/Target-genes-of-TNA-in-mutiple-tissue-of-sepsis-of-Hippo-pathway.pdf})}

\def\@captype{figure}
\begin{center}
\includegraphics[width = 0.9\linewidth]{Figure+Table/Target-genes-of-TNA-in-mutiple-tissue-of-sepsis-of-Hippo-pathway.pdf}
\caption{Target genes of TNA in mutiple tissue of sepsis of Hippo pathway}\label{fig:Target-genes-of-TNA-in-mutiple-tissue-of-sepsis-of-Hippo-pathway}
\end{center}

\hypertarget{ux5206ux5b50ux5bf9ux63a5}{%
\subsubsection{分子对接}\label{ux5206ux5b50ux5bf9ux63a5}}

丹参酮 I(Tanshinone IIA, \url{CID:164676})

以 AutoDock Vina 对 Fig. \ref{fig:Target-genes-of-TNA-in-mutiple-tissue-of-sepsis-of-Hippo-pathway} 所示基因的蛋白以 Tanshinone IIA分子对接。

结果显示,SMAD7, SOX2, TGFBR2, DLG4, DLG2 具有良好亲和度。

结合 Fig. \ref{fig:Target-genes-of-TNA-in-mutiple-tissue-of-sepsis-of-Hippo-pathway} 所示的多组织差异表达,DLG4 同时在 3 种组织 sepsis 差异表达,且为 TNA 作用靶点,表现良好对接亲和度,可能是 TNA 治疗的关键靶点之一。

Figure \ref{fig:docking-affinity}为图docking affinity概览。

\textbf{(对应文件为 \texttt{Figure+Table/docking-affinity.pdf})}

\def\@captype{figure}
\begin{center}
\includegraphics[width = 0.9\linewidth]{Figure+Table/docking-affinity.pdf}
\caption{Docking affinity}\label{fig:docking-affinity}
\end{center}

Figure \ref{fig:Tanshinone-IIA-binding-with-protein-DLG4}为图Tanshinone IIA binding with protein DLG4概览。

\textbf{(对应文件为 \texttt{Figure+Table/164676\_into\_1kef.png})}

\def\@captype{figure}
\begin{center}
\includegraphics[width = 0.9\linewidth]{./figs/164676_into_1kef.png}
\caption{Tanshinone IIA binding with protein DLG4}\label{fig:Tanshinone-IIA-binding-with-protein-DLG4}
\end{center}

\end{document}
