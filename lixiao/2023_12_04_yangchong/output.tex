% Options for packages loaded elsewhere
\PassOptionsToPackage{unicode}{hyperref}
\PassOptionsToPackage{hyphens}{url}
%
\documentclass[
]{article}
\usepackage{lmodern}
\usepackage{amssymb,amsmath}
\usepackage{ifxetex,ifluatex}
\ifnum 0\ifxetex 1\fi\ifluatex 1\fi=0 % if pdftex
  \usepackage[T1]{fontenc}
  \usepackage[utf8]{inputenc}
  \usepackage{textcomp} % provide euro and other symbols
\else % if luatex or xetex
  \usepackage{unicode-math}
  \defaultfontfeatures{Scale=MatchLowercase}
  \defaultfontfeatures[\rmfamily]{Ligatures=TeX,Scale=1}
\fi
% Use upquote if available, for straight quotes in verbatim environments
\IfFileExists{upquote.sty}{\usepackage{upquote}}{}
\IfFileExists{microtype.sty}{% use microtype if available
  \usepackage[]{microtype}
  \UseMicrotypeSet[protrusion]{basicmath} % disable protrusion for tt fonts
}{}
\makeatletter
\@ifundefined{KOMAClassName}{% if non-KOMA class
  \IfFileExists{parskip.sty}{%
    \usepackage{parskip}
  }{% else
    \setlength{\parindent}{0pt}
    \setlength{\parskip}{6pt plus 2pt minus 1pt}}
}{% if KOMA class
  \KOMAoptions{parskip=half}}
\makeatother
\usepackage{xcolor}
\IfFileExists{xurl.sty}{\usepackage{xurl}}{} % add URL line breaks if available
\IfFileExists{bookmark.sty}{\usepackage{bookmark}}{\usepackage{hyperref}}
\hypersetup{
  hidelinks,
  pdfcreator={LaTeX via pandoc}}
\urlstyle{same} % disable monospaced font for URLs
\usepackage[margin=1in]{geometry}
\usepackage{longtable,booktabs}
% Correct order of tables after \paragraph or \subparagraph
\usepackage{etoolbox}
\makeatletter
\patchcmd\longtable{\par}{\if@noskipsec\mbox{}\fi\par}{}{}
\makeatother
% Allow footnotes in longtable head/foot
\IfFileExists{footnotehyper.sty}{\usepackage{footnotehyper}}{\usepackage{footnote}}
\makesavenoteenv{longtable}
\usepackage{graphicx}
\makeatletter
\def\maxwidth{\ifdim\Gin@nat@width>\linewidth\linewidth\else\Gin@nat@width\fi}
\def\maxheight{\ifdim\Gin@nat@height>\textheight\textheight\else\Gin@nat@height\fi}
\makeatother
% Scale images if necessary, so that they will not overflow the page
% margins by default, and it is still possible to overwrite the defaults
% using explicit options in \includegraphics[width, height, ...]{}
\setkeys{Gin}{width=\maxwidth,height=\maxheight,keepaspectratio}
% Set default figure placement to htbp
\makeatletter
\def\fps@figure{htbp}
\makeatother
\setlength{\emergencystretch}{3em} % prevent overfull lines
\providecommand{\tightlist}{%
  \setlength{\itemsep}{0pt}\setlength{\parskip}{0pt}}
\setcounter{secnumdepth}{5}
\usepackage{caption} \captionsetup{font={footnotesize},width=6in} \renewcommand{\dblfloatpagefraction}{.9} \makeatletter \renewenvironment{figure} {\def\@captype{figure}} \makeatother \definecolor{shadecolor}{RGB}{242,242,242} \usepackage{xeCJK} \usepackage{setspace} \setstretch{1.3} \usepackage{tcolorbox} \setcounter{secnumdepth}{4} \setcounter{tocdepth}{4} \usepackage{wallpaper} \usepackage[absolute]{textpos}
\newlength{\cslhangindent}
\setlength{\cslhangindent}{1.5em}
\newenvironment{cslreferences}%
  {}%
  {\par}

\author{}
\date{\vspace{-2.5em}}

\begin{document}

\begin{titlepage} \newgeometry{top=7.5cm}
\ThisCenterWallPaper{1.12}{../cover_page.pdf}
\begin{center} \textbf{\Huge ccRCC 单细胞数据的
Treg 细胞差异表达基因} \vspace{4em}
\begin{textblock}{10}(3,5.9) \huge
\textbf{\textcolor{white}{2023-12-06}}
\end{textblock} \begin{textblock}{10}(3,7.3)
\Large \textcolor{black}{LiChuang Huang}
\end{textblock} \begin{textblock}{10}(3,11.3)
\Large \textcolor{black}{@立效研究院}
\end{textblock} \end{center} \end{titlepage}
\restoregeometry

\pagenumbering{roman}

\tableofcontents

\listoffigures

\listoftables

\newpage

\pagenumbering{arabic}

\hypertarget{abstract}{%
\section{摘要}\label{abstract}}

\begin{itemize}
\item
  ccRCC Treg 细胞的差异表达基因见 Tab. \ref{tab:Treg-DEGs}
\item
  Treg 的差异基因和下游靶点的筛选可参考 Tab. \ref{tab:Interaction-of-pathways}、Tab. \ref{tab:Interaction-of-ligand-and-receptor} (细胞通讯角度)
\item
  或者参考通路富集筛选差异基因和下游靶点 \ref{enrich}
\item
  \textbf{Normal Treg vs RCC Treg 差异表达基因见} Tab. \ref{tab:RCC-and-Normal-DEGs-of-the-contrasts}
\end{itemize}

\hypertarget{methods}{%
\section{材料和方法}\label{methods}}

\hypertarget{ux6750ux6599}{%
\subsection{材料}\label{ux6750ux6599}}

All used GEO expression data and their design:

\begin{itemize}
\tightlist
\item
  \textbf{GSE210038}: Seven tumoral and two normal adjacent tissue samples from patients presenting clear-cell Renal Cell Carcinoma were analyzed by single-cell RNA sequencing.
\end{itemize}

\hypertarget{ux65b9ux6cd5}{%
\subsection{方法}\label{ux65b9ux6cd5}}

Mainly used method:

\begin{itemize}
\tightlist
\item
  CellChat used for cell communication analysis.\textsuperscript{\protect\hyperlink{ref-InferenceAndAJinS2021}{1}}
\item
  ClusterProfiler used for GSEA enrichment.\textsuperscript{\protect\hyperlink{ref-ClusterprofilerWuTi2021}{2}}
\item
  GEO \url{https://www.ncbi.nlm.nih.gov/geo/} used for expression dataset aquisition .
\item
  Seurat used for scRNA-seq processing; SCSA used for cell type annotation.\textsuperscript{\protect\hyperlink{ref-IntegratedAnalHaoY2021}{3}--\protect\hyperlink{ref-ScsaACellTyCaoY2020}{5}}
\item
  Other R packages (eg., \texttt{dplyr} and \texttt{ggplot2}) used for statistic analysis or data visualization.
\end{itemize}

\hypertarget{workflow}{%
\section{附:分析流程}\label{workflow}}

\hypertarget{ccrcc-ux5355ux7ec6ux80deux6570ux636e}{%
\subsection{ccRCC 单细胞数据}\label{ccrcc-ux5355ux7ec6ux80deux6570ux636e}}

Figure \ref{fig:UMAP-Clustering} (下方图) 为图UMAP Clustering概览。

\textbf{(对应文件为 \texttt{Figure+Table/UMAP-Clustering.pdf})}

\def\@captype{figure}
\begin{center}
\includegraphics[width = 0.9\linewidth]{Figure+Table/UMAP-Clustering.pdf}
\caption{UMAP Clustering}\label{fig:UMAP-Clustering}
\end{center}

\hypertarget{t1-treg}{%
\subsubsection{鉴定 Treg 细胞}\label{t1-treg}}

根据文献\textsuperscript{\protect\hyperlink{ref-SingleCellSeqKrishn2021}{6}},使用 FOXP3, BATF, CTLA4, TIGIT" 作为 marker 鉴定 Treg 细胞。

Figure \ref{fig:Heatmap-show-the-reference-genes} (下方图) 为图Heatmap show the reference genes概览。

\textbf{(对应文件为 \texttt{Figure+Table/Heatmap-show-the-reference-genes.pdf})}

\def\@captype{figure}
\begin{center}
\includegraphics[width = 0.9\linewidth]{Figure+Table/Heatmap-show-the-reference-genes.pdf}
\caption{Heatmap show the reference genes}\label{fig:Heatmap-show-the-reference-genes}
\end{center}

显然,Cluster 6 为 Treg 细胞。

其余细胞以 SCSA 注释。

Figure \ref{fig:The-cell-type} (下方图) 为图The cell type概览。

\textbf{(对应文件为 \texttt{Figure+Table/The-cell-type.pdf})}

\def\@captype{figure}
\begin{center}
\includegraphics[width = 0.9\linewidth]{Figure+Table/The-cell-type.pdf}
\caption{The cell type}\label{fig:The-cell-type}
\end{center}

\hypertarget{treg-ux7ec6ux80deux7684ux5deeux5f02ux8868ux8fbeux57faux56e0-ux5bf9ux6bd4ux5176ux5b83ux7ec6ux80de}{%
\subsection{Treg 细胞的差异表达基因 (对比其它细胞)}\label{treg-ux7ec6ux80deux7684ux5deeux5f02ux8868ux8fbeux57faux56e0-ux5bf9ux6bd4ux5176ux5b83ux7ec6ux80de}}

Table \ref{tab:Treg-DEGs} (下方表格) 为表格Treg DEGs概览。

\textbf{(对应文件为 \texttt{Figure+Table/Treg-DEGs.csv})}

\begin{center}\begin{tcolorbox}[colback=gray!10, colframe=gray!50, width=0.9\linewidth, arc=1mm, boxrule=0.5pt]注:表格共有652行8列,以下预览的表格可能省略部分数据;表格含有652个唯一`rownames'。
\end{tcolorbox}
\end{center}

\begin{longtable}[]{@{}llllllll@{}}
\caption{\label{tab:Treg-DEGs}Treg DEGs}\tabularnewline
\toprule
rownames & p\_val & avg\_l\ldots{} & pct.1 & pct.2 & p\_val\ldots{} & cluster & gene\tabularnewline
\midrule
\endfirsthead
\toprule
rownames & p\_val & avg\_l\ldots{} & pct.1 & pct.2 & p\_val\ldots{} & cluster & gene\tabularnewline
\midrule
\endhead
CTLA4 & 0 & 3.608\ldots{} & 0.688 & 0.046 & 0 & 6 & CTLA4\tabularnewline
TBC1D4 & 0 & 3.051\ldots{} & 0.749 & 0.086 & 0 & 6 & TBC1D4\tabularnewline
FOXP3 & 0 & 2.156\ldots{} & 0.315 & 0.006 & 0 & 6 & FOXP3\tabularnewline
ICOS & 8.465\ldots{} & 2.946\ldots{} & 0.599 & 0.053 & 1.520\ldots{} & 6 & ICOS\tabularnewline
TIGIT1 & 1.433\ldots{} & 3.012\ldots{} & 0.733 & 0.101 & 2.575\ldots{} & 6 & TIGIT\tabularnewline
RTKN2 & 2.467\ldots{} & 2.752\ldots{} & 0.354 & 0.015 & 4.433\ldots{} & 6 & RTKN2\tabularnewline
RP11-\ldots{} & 2.973\ldots{} & 2.298\ldots{} & 0.298 & 0.011 & 5.342\ldots{} & 6 & RP11-\ldots{}\tabularnewline
LTB1 & 3.956\ldots{} & 2.917\ldots{} & 0.766 & 0.125 & 7.108\ldots{} & 6 & LTB\tabularnewline
BATF3 & 3.925\ldots{} & 3.204\ldots{} & 0.76 & 0.139 & 7.052\ldots{} & 6 & BATF\tabularnewline
TNFRSF18 & 3.680\ldots{} & 3.126\ldots{} & 0.507 & 0.051 & 6.612\ldots{} & 6 & TNFRSF18\tabularnewline
SLAMF1 & 6.452\ldots{} & 2.463\ldots{} & 0.432 & 0.036 & 1.159\ldots{} & 6 & SLAMF1\tabularnewline
CD271 & 3.806\ldots{} & 2.380\ldots{} & 0.702 & 0.118 & 6.838\ldots{} & 6 & CD27\tabularnewline
IL2RA & 2.908\ldots{} & 2.084\ldots{} & 0.256 & 0.012 & 5.225\ldots{} & 6 & IL2RA\tabularnewline
STAM & 2.076\ldots{} & 2.388\ldots{} & 0.529 & 0.077 & 3.730\ldots{} & 6 & STAM\tabularnewline
IKZF2 & 8.030\ldots{} & 2.235\ldots{} & 0.451 & 0.053 & 1.442\ldots{} & 6 & IKZF2\tabularnewline
\ldots{} & \ldots{} & \ldots{} & \ldots{} & \ldots{} & \ldots{} & \ldots{} & \ldots{}\tabularnewline
\bottomrule
\end{longtable}

\hypertarget{enrich}{%
\subsection{Treg 差异基因通路富集}\label{enrich}}

Figure \ref{fig:KEGG-enrichment} (下方图) 为图KEGG enrichment概览。

\textbf{(对应文件为 \texttt{Figure+Table/KEGG-enrichment.pdf})}

\def\@captype{figure}
\begin{center}
\includegraphics[width = 0.9\linewidth]{Figure+Table/KEGG-enrichment.pdf}
\caption{KEGG enrichment}\label{fig:KEGG-enrichment}
\end{center}

Figure \ref{fig:GSEA-plot-of-the-pathways} (下方图) 为图GSEA plot of the pathways概览。

\textbf{(对应文件为 \texttt{Figure+Table/GSEA-plot-of-the-pathways.pdf})}

\def\@captype{figure}
\begin{center}
\includegraphics[width = 0.9\linewidth]{Figure+Table/GSEA-plot-of-the-pathways.pdf}
\caption{GSEA plot of the pathways}\label{fig:GSEA-plot-of-the-pathways}
\end{center}

Figure \ref{fig:view-pathway-of-hsa04060} (下方图) 为图view pathway of hsa04060概览。

\textbf{(对应文件为 \texttt{Figure+Table/hsa04060.pathview.png})}

\def\@captype{figure}
\begin{center}
\includegraphics[width = 0.9\linewidth]{pathview2023-12-06_14_25_28.762807/hsa04060.pathview.png}
\caption{View pathway of hsa04060}\label{fig:view-pathway-of-hsa04060}
\end{center}

\hypertarget{treg-ux7ec6ux80deux901aux8bafux4fe1ux606f}{%
\subsection{Treg 细胞通讯信息}\label{treg-ux7ec6ux80deux901aux8bafux4fe1ux606f}}

Figure \ref{fig:Overall-communication-count} (下方图) 为图Overall communication count概览。

\textbf{(对应文件为 \texttt{Figure+Table/Overall-communication-count.pdf})}

\def\@captype{figure}
\begin{center}
\includegraphics[width = 0.9\linewidth]{Figure+Table/Overall-communication-count.pdf}
\caption{Overall communication count}\label{fig:Overall-communication-count}
\end{center}

Table \ref{tab:Interaction-of-pathways} (下方表格) 为表格Interaction of pathways概览。

\textbf{(对应文件为 \texttt{Figure+Table/Interaction-of-pathways.csv})}

\begin{center}\begin{tcolorbox}[colback=gray!10, colframe=gray!50, width=0.9\linewidth, arc=1mm, boxrule=0.5pt]注:表格共有57行5列,以下预览的表格可能省略部分数据;表格含有10个唯一`source'。
\end{tcolorbox}
\end{center}

\begin{longtable}[]{@{}lllll@{}}
\caption{\label{tab:Interaction-of-pathways}Interaction of pathways}\tabularnewline
\toprule
source & target & pathw\ldots{} & prob & pval\tabularnewline
\midrule
\endfirsthead
\toprule
source & target & pathw\ldots{} & prob & pval\tabularnewline
\midrule
\endhead
B cell & Treg \ldots{} & MHC-II & 0.008\ldots{} & 0\tabularnewline
B cell & Treg \ldots{} & MIF & 0.022\ldots{} & 0\tabularnewline
CD8+ \ldots{} & Treg \ldots{} & CD137 & 0.007\ldots{} & 0\tabularnewline
CD8+ \ldots{} & Treg \ldots{} & MHC-II & 0.029\ldots{} & 0\tabularnewline
CD8+ \ldots{} & Treg \ldots{} & TNF & 0.002\ldots{} & 0\tabularnewline
Endot\ldots{} & Treg \ldots{} & APP & 0.074\ldots{} & 0\tabularnewline
Endot\ldots{} & Treg \ldots{} & COLLAGEN & 0.128\ldots{} & 0\tabularnewline
Endot\ldots{} & Treg \ldots{} & CXCL & 0.024\ldots{} & 0\tabularnewline
Endot\ldots{} & Treg \ldots{} & FN1 & 0.049\ldots{} & 0\tabularnewline
Endot\ldots{} & Treg \ldots{} & LAMININ & 0.038\ldots{} & 0\tabularnewline
Endot\ldots{} & Treg \ldots{} & MHC-II & 0.016\ldots{} & 0\tabularnewline
Macro\ldots{} & Treg \ldots{} & CD86 & 0.010\ldots{} & 0\tabularnewline
Macro\ldots{} & Treg \ldots{} & CXCL & 0.006\ldots{} & 0\tabularnewline
Macro\ldots{} & Treg \ldots{} & GALECTIN & 0.025\ldots{} & 0\tabularnewline
Macro\ldots{} & Treg \ldots{} & MHC-II & 0.265\ldots{} & 0\tabularnewline
\ldots{} & \ldots{} & \ldots{} & \ldots{} & \ldots{}\tabularnewline
\bottomrule
\end{longtable}

Table \ref{tab:Interaction-of-ligand-and-receptor} (下方表格) 为表格Interaction of ligand and receptor概览。

\textbf{(对应文件为 \texttt{Figure+Table/Interaction-of-ligand-and-receptor.csv})}

\begin{center}\begin{tcolorbox}[colback=gray!10, colframe=gray!50, width=0.9\linewidth, arc=1mm, boxrule=0.5pt]注:表格共有136行11列,以下预览的表格可能省略部分数据;表格含有10个唯一`source'。
\end{tcolorbox}
\end{center}

\begin{longtable}[]{@{}llllllllll@{}}
\caption{\label{tab:Interaction-of-ligand-and-receptor}Interaction of ligand and receptor}\tabularnewline
\toprule
source & target & ligand & receptor & prob & pval & inter\ldots\ldots7 & inter\ldots\ldots8 & pathw\ldots{} & annot\ldots{}\tabularnewline
\midrule
\endfirsthead
\toprule
source & target & ligand & receptor & prob & pval & inter\ldots\ldots7 & inter\ldots\ldots8 & pathw\ldots{} & annot\ldots{}\tabularnewline
\midrule
\endhead
Endot\ldots{} & Treg \ldots{} & CXCL12 & CXCR4 & 0.024\ldots{} & 0 & CXCL1\ldots{} & CXCL1\ldots{} & CXCL & Secre\ldots{}\tabularnewline
Macro\ldots{} & Treg \ldots{} & CXCL16 & CXCR6 & 0.006\ldots{} & 0 & CXCL1\ldots{} & CXCL1\ldots{} & CXCL & Secre\ldots{}\tabularnewline
Treg \ldots{} & B cell & MIF & CD74\_\ldots{} & 0.012\ldots{} & 0 & MIF\_C\ldots{} & MIF -\ldots{} & MIF & Secre\ldots{}\tabularnewline
Treg \ldots{} & CD8+ \ldots{} & MIF & CD74\_\ldots{} & 0.015\ldots{} & 0 & MIF\_C\ldots{} & MIF -\ldots{} & MIF & Secre\ldots{}\tabularnewline
Treg \ldots{} & Endot\ldots{} & MIF & CD74\_\ldots{} & 0.006\ldots{} & 0.03 & MIF\_C\ldots{} & MIF -\ldots{} & MIF & Secre\ldots{}\tabularnewline
Treg \ldots{} & Macro\ldots{} & MIF & CD74\_\ldots{} & 0.028\ldots{} & 0 & MIF\_C\ldots{} & MIF -\ldots{} & MIF & Secre\ldots{}\tabularnewline
Treg \ldots{} & Mast \ldots{} & MIF & CD74\_\ldots{} & 0.013\ldots{} & 0 & MIF\_C\ldots{} & MIF -\ldots{} & MIF & Secre\ldots{}\tabularnewline
Treg \ldots{} & Natur\ldots{} & MIF & CD74\_\ldots{} & 0.005\ldots{} & 0.03 & MIF\_C\ldots{} & MIF -\ldots{} & MIF & Secre\ldots{}\tabularnewline
Treg \ldots{} & Neutr\ldots{} & MIF & CD74\_\ldots{} & 0.011\ldots{} & 0.01 & MIF\_C\ldots{} & MIF -\ldots{} & MIF & Secre\ldots{}\tabularnewline
B cell & Treg \ldots{} & MIF & CD74\_\ldots{} & 0.014\ldots{} & 0 & MIF\_C\ldots{} & MIF -\ldots{} & MIF & Secre\ldots{}\tabularnewline
Proxi\ldots{} & Treg \ldots{} & MIF & CD74\_\ldots{} & 0.105\ldots{} & 0 & MIF\_C\ldots{} & MIF -\ldots{} & MIF & Secre\ldots{}\tabularnewline
Treg \ldots{} & Treg \ldots{} & MIF & CD74\_\ldots{} & 0.014\ldots{} & 0 & MIF\_C\ldots{} & MIF -\ldots{} & MIF & Secre\ldots{}\tabularnewline
Treg \ldots{} & B cell & MIF & CD74\_\ldots{} & 0.004\ldots{} & 0 & MIF\_C\ldots{} & MIF -\ldots{} & MIF & Secre\ldots{}\tabularnewline
Treg \ldots{} & CD8+ \ldots{} & MIF & CD74\_\ldots{} & 0.007\ldots{} & 0 & MIF\_C\ldots{} & MIF -\ldots{} & MIF & Secre\ldots{}\tabularnewline
Treg \ldots{} & Macro\ldots{} & MIF & CD74\_\ldots{} & 0.016\ldots{} & 0 & MIF\_C\ldots{} & MIF -\ldots{} & MIF & Secre\ldots{}\tabularnewline
\ldots{} & \ldots{} & \ldots{} & \ldots{} & \ldots{} & \ldots{} & \ldots{} & \ldots{} & \ldots{} & \ldots{}\tabularnewline
\bottomrule
\end{longtable}

\hypertarget{ux8865ux5145normal-ux7ec4ux7ec7ux548c-rcc-ux7ec4ux7ec7ux7684-treg-ux7ec6ux80deux6bd4ux5bf9}{%
\section{补充:Normal 组织和 RCC 组织的 Treg 细胞比对}\label{ux8865ux5145normal-ux7ec4ux7ec7ux548c-rcc-ux7ec4ux7ec7ux7684-treg-ux7ec6ux80deux6bd4ux5bf9}}

\hypertarget{ux4ee5-seurat-ux96c6ux6210ux5904ux7406-normal-ux548c-rcc-ux7ec4ux7ec7-ux5355ux7ec6ux80deux6570ux636e}{%
\subsection{以 Seurat 集成处理 Normal 和 RCC 组织 单细胞数据}\label{ux4ee5-seurat-ux96c6ux6210ux5904ux7406-normal-ux548c-rcc-ux7ec4ux7ec7-ux5355ux7ec6ux80deux6570ux636e}}

使用的数据集来自于样本 GSM6415686、GSM6415694。

\hypertarget{marker-ux57faux56e0}{%
\subsubsection{Marker 基因}\label{marker-ux57faux56e0}}

根据文献\textsuperscript{\protect\hyperlink{ref-SingleCellSeqKrishn2021}{6}},使用 FOXP3, BATF, CTLA4, TIGIT" 作为 marker 鉴定 Treg 细胞。

以下,仅图示基因为高变基因 (Variable feautre) 。鉴定 cluster 6 为 Treg 细胞群 (\ref{t1-treg} 的 Treg 细胞同属于这个细胞聚类团)。

Figure \ref{fig:S-Heatmap-show-the-reference-genes} (下方图) 为图S Heatmap show the reference genes概览。

\textbf{(对应文件为 \texttt{Figure+Table/S-Heatmap-show-the-reference-genes.pdf})}

\def\@captype{figure}
\begin{center}
\includegraphics[width = 0.9\linewidth]{Figure+Table/S-Heatmap-show-the-reference-genes.pdf}
\caption{S Heatmap show the reference genes}\label{fig:S-Heatmap-show-the-reference-genes}
\end{center}

随后,细胞类型注释为:

Figure \ref{fig:S-The-Treg-cells} (下方图) 为图S The Treg cells概览。

\textbf{(对应文件为 \texttt{Figure+Table/S-The-Treg-cells.pdf})}

\def\@captype{figure}
\begin{center}
\includegraphics[width = 0.9\linewidth]{Figure+Table/S-The-Treg-cells.pdf}
\caption{S The Treg cells}\label{fig:S-The-Treg-cells}
\end{center}

\hypertarget{ux5deeux5f02ux8868ux8fbe}{%
\subsubsection{差异表达}\label{ux5deeux5f02ux8868ux8fbe}}

Table \ref{tab:RCC-and-Normal-DEGs-of-the-contrasts} (下方表格) 为表格RCC and Normal DEGs of the contrasts概览。

\textbf{(对应文件为 \texttt{Figure+Table/RCC-and-Normal-DEGs-of-the-contrasts.csv})}

\begin{center}\begin{tcolorbox}[colback=gray!10, colframe=gray!50, width=0.9\linewidth, arc=1mm, boxrule=0.5pt]注:表格共有704行7列,以下预览的表格可能省略部分数据;表格含有1个唯一`contrast'。
\end{tcolorbox}
\end{center}

\begin{longtable}[]{@{}lllllll@{}}
\caption{\label{tab:RCC-and-Normal-DEGs-of-the-contrasts}RCC and Normal DEGs of the contrasts}\tabularnewline
\toprule
contrast & p\_val & avg\_l\ldots{} & pct.1 & pct.2 & p\_val\ldots{} & gene\tabularnewline
\midrule
\endfirsthead
\toprule
contrast & p\_val & avg\_l\ldots{} & pct.1 & pct.2 & p\_val\ldots{} & gene\tabularnewline
\midrule
\endhead
Treg \ldots{} & 9.236\ldots{} & -1.12\ldots{} & 0.512 & 0.027 & 2.770\ldots{} & ITGAX\tabularnewline
Treg \ldots{} & 3.286\ldots{} & -4.47\ldots{} & 0.234 & 0.036 & 9.860\ldots{} & CHML\tabularnewline
Treg \ldots{} & 8.063\ldots{} & -2.89\ldots{} & 0.867 & 0.062 & 2.418\ldots{} & AURKA\tabularnewline
Treg \ldots{} & 1.414\ldots{} & 7.436\ldots{} & 0.178 & 0.027 & 4.244\ldots{} & ZBTB37\tabularnewline
Treg \ldots{} & 1.000\ldots{} & -2.33\ldots{} & 0.739 & 0.089 & 3.001\ldots{} & PDK1\tabularnewline
Treg \ldots{} & 4.158\ldots{} & -4.69\ldots{} & 0.596 & 0.062 & 1.247\ldots{} & ARHGAP10\tabularnewline
Treg \ldots{} & 1.966\ldots{} & 3.428\ldots{} & 0.112 & 0.062 & 5.898\ldots{} & CCDC14\tabularnewline
Treg \ldots{} & 7.855\ldots{} & -1.47\ldots{} & 0.82 & 0.027 & 2.356\ldots{} & MKI67\tabularnewline
Treg \ldots{} & 1.038\ldots{} & -4.16\ldots{} & 0.258 & 0.054 & 3.116\ldots{} & ENC1\tabularnewline
Treg \ldots{} & 4.089\ldots{} & -6.47\ldots{} & 0.519 & 0.098 & 1.226\ldots{} & AOAH\tabularnewline
Treg \ldots{} & 3.070\ldots{} & 0.532\ldots{} & 0.157 & 0.027 & 9.212\ldots{} & TMIGD2\tabularnewline
Treg \ldots{} & 4.034\ldots{} & -8.29\ldots{} & 0.829 & 0.134 & 1.210\ldots{} & ERCC5\tabularnewline
Treg \ldots{} & 1.316\ldots{} & 17.34\ldots{} & 0.144 & 0 & 3.949\ldots{} & HBB\tabularnewline
Treg \ldots{} & 5.742\ldots{} & -2.05\ldots{} & 0.872 & 0.089 & 1.722\ldots{} & GPR18\tabularnewline
Treg \ldots{} & 1.017\ldots{} & 4.533\ldots{} & 0.034 & 0.188 & 3.053\ldots{} & CAPS\tabularnewline
\ldots{} & \ldots{} & \ldots{} & \ldots{} & \ldots{} & \ldots{} & \ldots{}\tabularnewline
\bottomrule
\end{longtable}

\hypertarget{bibliography}{%
\section*{Reference}\label{bibliography}}
\addcontentsline{toc}{section}{Reference}

\hypertarget{refs}{}
\begin{cslreferences}
\leavevmode\hypertarget{ref-InferenceAndAJinS2021}{}%
1. Jin, S. \emph{et al.} Inference and analysis of cell-cell communication using cellchat. \emph{Nature Communications} \textbf{12}, (2021).

\leavevmode\hypertarget{ref-ClusterprofilerWuTi2021}{}%
2. Wu, T. \emph{et al.} ClusterProfiler 4.0: A universal enrichment tool for interpreting omics data. \emph{The Innovation} \textbf{2}, (2021).

\leavevmode\hypertarget{ref-IntegratedAnalHaoY2021}{}%
3. Hao, Y. \emph{et al.} Integrated analysis of multimodal single-cell data. \emph{Cell} \textbf{184}, (2021).

\leavevmode\hypertarget{ref-ComprehensiveIStuart2019}{}%
4. Stuart, T. \emph{et al.} Comprehensive integration of single-cell data. \emph{Cell} \textbf{177}, (2019).

\leavevmode\hypertarget{ref-ScsaACellTyCaoY2020}{}%
5. Cao, Y., Wang, X. \& Peng, G. SCSA: A cell type annotation tool for single-cell rna-seq data. \emph{Frontiers in genetics} \textbf{11}, (2020).

\leavevmode\hypertarget{ref-SingleCellSeqKrishn2021}{}%
6. Krishna, C. \emph{et al.} Single-cell sequencing links multiregional immune landscapes and tissue-resident t~cells in ccRCC to tumor topology and therapy efficacy. \emph{Cancer cell} \textbf{39}, 662--677.e6 (2021).
\end{cslreferences}

\end{document}
