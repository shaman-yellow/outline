% Options for packages loaded elsewhere
\PassOptionsToPackage{unicode}{hyperref}
\PassOptionsToPackage{hyphens}{url}
%
\documentclass[
]{article}
\usepackage{lmodern}
\usepackage{amssymb,amsmath}
\usepackage{ifxetex,ifluatex}
\ifnum 0\ifxetex 1\fi\ifluatex 1\fi=0 % if pdftex
  \usepackage[T1]{fontenc}
  \usepackage[utf8]{inputenc}
  \usepackage{textcomp} % provide euro and other symbols
\else % if luatex or xetex
  \usepackage{unicode-math}
  \defaultfontfeatures{Scale=MatchLowercase}
  \defaultfontfeatures[\rmfamily]{Ligatures=TeX,Scale=1}
\fi
% Use upquote if available, for straight quotes in verbatim environments
\IfFileExists{upquote.sty}{\usepackage{upquote}}{}
\IfFileExists{microtype.sty}{% use microtype if available
  \usepackage[]{microtype}
  \UseMicrotypeSet[protrusion]{basicmath} % disable protrusion for tt fonts
}{}
\makeatletter
\@ifundefined{KOMAClassName}{% if non-KOMA class
  \IfFileExists{parskip.sty}{%
    \usepackage{parskip}
  }{% else
    \setlength{\parindent}{0pt}
    \setlength{\parskip}{6pt plus 2pt minus 1pt}}
}{% if KOMA class
  \KOMAoptions{parskip=half}}
\makeatother
\usepackage{xcolor}
\IfFileExists{xurl.sty}{\usepackage{xurl}}{} % add URL line breaks if available
\IfFileExists{bookmark.sty}{\usepackage{bookmark}}{\usepackage{hyperref}}
\hypersetup{
  hidelinks,
  pdfcreator={LaTeX via pandoc}}
\urlstyle{same} % disable monospaced font for URLs
\usepackage[margin=1in]{geometry}
\usepackage{longtable,booktabs}
% Correct order of tables after \paragraph or \subparagraph
\usepackage{etoolbox}
\makeatletter
\patchcmd\longtable{\par}{\if@noskipsec\mbox{}\fi\par}{}{}
\makeatother
% Allow footnotes in longtable head/foot
\IfFileExists{footnotehyper.sty}{\usepackage{footnotehyper}}{\usepackage{footnote}}
\makesavenoteenv{longtable}
\usepackage{graphicx}
\makeatletter
\def\maxwidth{\ifdim\Gin@nat@width>\linewidth\linewidth\else\Gin@nat@width\fi}
\def\maxheight{\ifdim\Gin@nat@height>\textheight\textheight\else\Gin@nat@height\fi}
\makeatother
% Scale images if necessary, so that they will not overflow the page
% margins by default, and it is still possible to overwrite the defaults
% using explicit options in \includegraphics[width, height, ...]{}
\setkeys{Gin}{width=\maxwidth,height=\maxheight,keepaspectratio}
% Set default figure placement to htbp
\makeatletter
\def\fps@figure{htbp}
\makeatother
\setlength{\emergencystretch}{3em} % prevent overfull lines
\providecommand{\tightlist}{%
  \setlength{\itemsep}{0pt}\setlength{\parskip}{0pt}}
\setcounter{secnumdepth}{5}
\usepackage{caption} \captionsetup{font={footnotesize},width=6in} \renewcommand{\dblfloatpagefraction}{.9} \makeatletter \renewenvironment{figure} {\def\@captype{figure}} \makeatother \@ifundefined{Shaded}{\newenvironment{Shaded}} \@ifundefined{snugshade}{\newenvironment{snugshade}} \renewenvironment{Shaded}{\begin{snugshade}}{\end{snugshade}} \definecolor{shadecolor}{RGB}{230,230,230} \usepackage{xeCJK} \usepackage{setspace} \setstretch{1.3} \usepackage{tcolorbox} \setcounter{secnumdepth}{4} \setcounter{tocdepth}{4} \usepackage{wallpaper} \usepackage[absolute]{textpos} \tcbuselibrary{breakable} \renewenvironment{Shaded} {\begin{tcolorbox}[colback = gray!10, colframe = gray!40, width = 16cm, arc = 1mm, auto outer arc, title = {R input}]} {\end{tcolorbox}} \usepackage{titlesec} \titleformat{\paragraph} {\fontsize{10pt}{0pt}\bfseries} {\arabic{section}.\arabic{subsection}.\arabic{subsubsection}.\arabic{paragraph}} {1em} {} []
\newlength{\cslhangindent}
\setlength{\cslhangindent}{1.5em}
\newenvironment{cslreferences}%
  {}%
  {\par}

\author{}
\date{\vspace{-2.5em}}

\begin{document}

\begin{titlepage} \newgeometry{top=7.5cm}
\ThisCenterWallPaper{1.12}{~/outline/lixiao//cover_page.pdf}
\begin{center} \textbf{\Huge
调控小胶质细胞代谢的关键基因XXX} \vspace{4em}
\begin{textblock}{10}(3,5.9) \huge
\textbf{\textcolor{white}{2024-05-17}}
\end{textblock} \begin{textblock}{10}(3,7.3)
\Large \textcolor{black}{LiChuang Huang}
\end{textblock} \begin{textblock}{10}(3,11.3)
\Large \textcolor{black}{@立效研究院}
\end{textblock} \end{center} \end{titlepage}
\restoregeometry

\pagenumbering{roman}

\tableofcontents

\listoffigures

\listoftables

\newpage

\pagenumbering{arabic}

\hypertarget{abstract}{%
\section{摘要}\label{abstract}}

\hypertarget{ux9700ux6c42}{%
\subsection{需求}\label{ux9700ux6c42}}

\begin{itemize}
\tightlist
\item
  自发性脑出血(ICH)(人或动物)后调控 小胶质细胞代谢 的关键基因XXX,并且该基因可能也调控 星形细胞胶质瘢痕生成。
\item
  客户前期发表的文章有做过盐诱导激酶 2(SIK-2)(\url{PMID:29018127}),
  XXX是否可以是SIK-2,或者XXX能否富集在SIK-2相关通路或其他分子机制上
\end{itemize}

\hypertarget{ux7ed3ux679c}{%
\subsection{结果}\label{ux7ed3ux679c}}

\begin{itemize}
\tightlist
\item
  以小鼠丘脑出血模型 (GSE227033) 的单细胞数据集分析 Microglial cell, 见 Fig. \ref{fig:The-cellType-group}。
  (该模型大体上应该是合适的, 以 collagenase IV 造模 (像 \url{PMID:38433011} 也是这种 ICH 造模)。
  ICH 的单细胞数据很少,基本没有其他合适的数据了)
\item
  以 Microglial Cell 预测代谢通量,并差异分析, 见 Fig. \ref{fig:SCF-Model-vs-Control}
\item
  与差异代谢相关的基因,在拟时轨迹 (Control -\textgreater{} Model) 中的表达见 Fig. \ref{fig:MI-Pseudotime-heatmap-of-genes}
\item
  将这些基因映射到人类的基因 (hgnc symbol) 后,获取上游的转录因子。在这些基因和转录因子中,尝试寻找 SIK-2。无结果。
  此外,SIK-2 为非差异表达基因 (Model vs Control)。
\item
  为筛选 Astrocyte 胶质瘢痕相关基因,首先获取了 Astrocyte 的差异基因 (Model vs Control) ,见
  Tab. \ref{tab:DEGs-of-the-contrasts-Astrocyte}。
  随后,获取 GeneCards 的胶质瘢痕相关基因,见 Tab. \ref{tab:GLIALSCAR-disease-related-targets-from-GeneCards}。
  尝试取交集,见 Fig. \ref{fig:Intersection-of-DB-GlialScar-with-Astrocyte-DEGs},有 11 个基因。
\item
  最后,调控小胶质细胞代谢且与星形细胞胶质瘢痕相关基因,Fig. \ref{fig:UpSet-plot-of-Genes-sources},
  由于 Fig. \ref{fig:Intersection-of-DB-GlialScar-with-Astrocyte-DEGs} 的基因与 Microglial
  代谢相关基因及上游转录因子无交集,因此,筛选时直接用了 Tab. \ref{tab:GLIALSCAR-disease-related-targets-from-GeneCards}
  的基因。获得结果:XYLT1 (即, Xylt1)。
\item
  Xylt1 的表达见 Fig. \ref{fig:Dimension-plot-of-expression-level-of-the-genes}。
  Xylt1 主要集中表达于模型组的 Microglial 中,符合条件。
  Xylt1 相关代谢通路见 Tab. \ref{tab:Xylt1-related-metabolic-flux}。
  Xylt1 的其余信息可参考 Fig. \ref{fig:MI-Pseudotime-heatmap-of-genes}。
\end{itemize}

\hypertarget{introduction}{%
\section{前言}\label{introduction}}

\hypertarget{methods}{%
\section{材料和方法}\label{methods}}

\hypertarget{ux6750ux6599}{%
\subsection{材料}\label{ux6750ux6599}}

All used GEO expression data and their design:

\begin{itemize}
\tightlist
\item
  \textbf{GSE227033}: we sequenced the transcriptomes of 32332 single brain cells, revealing a total of four major cell types within the four thalamus sample from mice.
\end{itemize}

\hypertarget{ux65b9ux6cd5}{%
\subsection{方法}\label{ux65b9ux6cd5}}

Mainly used method:

\begin{itemize}
\tightlist
\item
  R package \texttt{biomaRt} used for gene annotation\textsuperscript{\protect\hyperlink{ref-MappingIdentifDurinc2009}{1}}.
\item
  The \texttt{biomart} was used for mapping genes between organism (e.g., mgi\_symbol to hgnc\_symbol)\textsuperscript{\protect\hyperlink{ref-MappingIdentifDurinc2009}{1}}.
\item
  The \texttt{scFEA} (python) was used to estimate cell-wise metabolic via single cell RNA-seq data\textsuperscript{\protect\hyperlink{ref-AGraphNeuralAlgham2021}{2}}.
\item
  R package \texttt{ClusterProfiler} used for gene enrichment analysis\textsuperscript{\protect\hyperlink{ref-ClusterprofilerWuTi2021}{3}}.
\item
  GEO \url{https://www.ncbi.nlm.nih.gov/geo/} used for expression dataset aquisition.
\item
  The Human Gene Database \texttt{GeneCards} used for disease related genes prediction\textsuperscript{\protect\hyperlink{ref-TheGenecardsSStelze2016}{4}}.
\item
  R package \texttt{Limma} and \texttt{edgeR} used for differential expression analysis\textsuperscript{\protect\hyperlink{ref-LimmaPowersDiRitchi2015}{5},\protect\hyperlink{ref-EdgerDifferenChen}{6}}.
\item
  R package \texttt{Monocle3} used for cell pseudotime analysis\textsuperscript{\protect\hyperlink{ref-ReversedGraphQiuX2017}{7},\protect\hyperlink{ref-TheDynamicsAnTrapne2014}{8}}.
\item
  R package \texttt{STEINGdb} used for PPI network construction\textsuperscript{\protect\hyperlink{ref-TheStringDataSzklar2021}{9},\protect\hyperlink{ref-CytohubbaIdenChin2014}{10}}.
\item
  The R package \texttt{Seurat} used for scRNA-seq processing\textsuperscript{\protect\hyperlink{ref-IntegratedAnalHaoY2021}{11},\protect\hyperlink{ref-ComprehensiveIStuart2019}{12}}.
\item
  The \texttt{Transcription\ Factor\ Target\ Gene\ Database} (\url{https://tfbsdb.systemsbiology.net/}) was used for discovering relationship between transcription factors and genes..\textsuperscript{\protect\hyperlink{ref-CausalMechanisPlaisi2016}{13}}
\item
  \texttt{SCSA} (python) used for cell type annotation\textsuperscript{\protect\hyperlink{ref-ScsaACellTyCaoY2020}{14}}.
\item
  The MCC score was calculated referring to algorithm of \texttt{CytoHubba}\textsuperscript{\protect\hyperlink{ref-CytohubbaIdenChin2014}{10}}.
\item
  R version 4.4.0 (2024-04-24); Other R packages (eg., \texttt{dplyr} and \texttt{ggplot2}) used for statistic analysis or data visualization.
\end{itemize}

\hypertarget{results}{%
\section{分析结果}\label{results}}

\hypertarget{dis}{%
\section{结论}\label{dis}}

\hypertarget{workflow}{%
\section{附:分析流程}\label{workflow}}

\hypertarget{ux5355ux7ec6ux80deux6570ux636eux5206ux6790}{%
\subsection{单细胞数据分析}\label{ux5355ux7ec6ux80deux6570ux636eux5206ux6790}}

\hypertarget{ux6570ux636eux6765ux6e90}{%
\subsubsection{数据来源}\label{ux6570ux636eux6765ux6e90}}

\begin{center}\begin{tcolorbox}[colback=gray!10, colframe=gray!50, width=0.9\linewidth, arc=1mm, boxrule=0.5pt]
\textbf{
Data Source ID
:}

\vspace{0.5em}

    GSE227033

\vspace{2em}


\textbf{
data\_processing
:}

\vspace{0.5em}

    Postprocessing and quality control were performed using
a 10× Cell Ranger package (v1.2.0; 10 × Genomics). Reads
were aligned to the mm10 reference assembly (v1.2.0; 10 ×
Genomics). sn-RNA seq data (Cellranger\_result) contains 4
samples (C1, C2, M1, M2).

\vspace{2em}


\textbf{
data\_processing.1
:}

\vspace{0.5em}

    Assembly: mm10

\vspace{2em}


\textbf{
data\_processing.2
:}

\vspace{0.5em}

    Supplementary files format and content: Tab-separated
values files and matrix files

\vspace{2em}
\end{tcolorbox}
\end{center}

\textbf{(上述信息框内容已保存至 \texttt{Figure+Table/GSE227033-content})}

\hypertarget{ux7ec6ux80deux805aux7c7bux4e0eux9274ux5b9a}{%
\subsubsection{细胞聚类与鉴定}\label{ux7ec6ux80deux805aux7c7bux4e0eux9274ux5b9a}}

Figure \ref{fig:UMAP-Clustering} (下方图) 为图UMAP Clustering概览。

\textbf{(对应文件为 \texttt{Figure+Table/UMAP-Clustering.pdf})}

\def\@captype{figure}
\begin{center}
\includegraphics[width = 0.9\linewidth]{Figure+Table/UMAP-Clustering.pdf}
\caption{UMAP Clustering}\label{fig:UMAP-Clustering}
\end{center}

Figure \ref{fig:The-cellType-group} (下方图) 为图The cellType group概览。

\textbf{(对应文件为 \texttt{Figure+Table/The-cellType-group.pdf})}

\def\@captype{figure}
\begin{center}
\includegraphics[width = 0.9\linewidth]{Figure+Table/The-cellType-group.pdf}
\caption{The cellType group}\label{fig:The-cellType-group}
\end{center}

\hypertarget{ux5c0fux80f6ux8d28ux7ec6ux80deux5206ux6790}{%
\subsubsection{小胶质细胞分析}\label{ux5c0fux80f6ux8d28ux7ec6ux80deux5206ux6790}}

\hypertarget{ux5deeux5f02ux5206ux6790}{%
\paragraph{差异分析}\label{ux5deeux5f02ux5206ux6790}}

Table \ref{tab:DEGs-of-the-contrasts-Microglial} (下方表格) 为表格DEGs of the contrasts Microglial概览。

\textbf{(对应文件为 \texttt{Figure+Table/DEGs-of-the-contrasts-Microglial.csv})}

\begin{center}\begin{tcolorbox}[colback=gray!10, colframe=gray!50, width=0.9\linewidth, arc=1mm, boxrule=0.5pt]注:表格共有1546行7列,以下预览的表格可能省略部分数据;含有1个唯一`contrast'。
\end{tcolorbox}
\end{center}

\begin{longtable}[]{@{}lllllll@{}}
\caption{\label{tab:DEGs-of-the-contrasts-Microglial}DEGs of the contrasts Microglial}\tabularnewline
\toprule
contrast & p\_val & avg\_log2FC & pct.1 & pct.2 & p\_val\_adj & gene\tabularnewline
\midrule
\endfirsthead
\toprule
contrast & p\_val & avg\_log2FC & pct.1 & pct.2 & p\_val\_adj & gene\tabularnewline
\midrule
\endhead
Microglial\ldots{} & 1.83518861\ldots{} & 2.92401237\ldots{} & 0.09 & 0.36 & 5.50556584\ldots{} & Cenpe\tabularnewline
Microglial\ldots{} & 4.48315517\ldots{} & 9.81893833\ldots{} & 0.111 & 0.374 & 1.34494655\ldots{} & Cdkn1a\tabularnewline
Microglial\ldots{} & 1.73323788\ldots{} & -4.0813517\ldots{} & 0.013 & 0.103 & 5.19971364\ldots{} & Qrfpr\tabularnewline
Microglial\ldots{} & 9.12205066\ldots{} & -2.4590397\ldots{} & 0.103 & 0.217 & 2.73661519\ldots{} & Dock6\tabularnewline
Microglial\ldots{} & 2.54749154\ldots{} & 0.28706360\ldots{} & 0.077 & 0.288 & 7.64247464\ldots{} & Gm26870\tabularnewline
Microglial\ldots{} & 2.93243498\ldots{} & 5.68612318\ldots{} & 0.269 & 0.059 & 8.79730495\ldots{} & Pantr2\tabularnewline
Microglial\ldots{} & 8.30783340\ldots{} & -6.9973674\ldots{} & 0.91 & 0.207 & 2.49235002\ldots{} & Agt\tabularnewline
Microglial\ldots{} & 4.69847504\ldots{} & 2.60879746\ldots{} & 0.612 & 0.092 & 1.40954251\ldots{} & Penk\tabularnewline
Microglial\ldots{} & 1.07440758\ldots{} & -1.7746631\ldots{} & 0.104 & 0.087 & 3.22322274\ldots{} & D7Ertd443e\tabularnewline
Microglial\ldots{} & 1.67179273\ldots{} & 8.88979035\ldots{} & 0.136 & 0.499 & 5.01537820\ldots{} & Ndc80\tabularnewline
Microglial\ldots{} & 6.93761245\ldots{} & 5.36162953\ldots{} & 0.125 & 0.34 & 2.08128373\ldots{} & Tmem123\tabularnewline
Microglial\ldots{} & 1.41432503\ldots{} & 12.6809064\ldots{} & 0.124 & 0.184 & 4.24297509\ldots{} & Egln3\tabularnewline
Microglial\ldots{} & 1.97677667\ldots{} & 2.11642957\ldots{} & 0.334 & 0.072 & 5.93033003\ldots{} & Car14\tabularnewline
Microglial\ldots{} & 1.05458751\ldots{} & 9.33084014\ldots{} & 0.089 & 0.178 & 3.16376254\ldots{} & Gadd45b\tabularnewline
Microglial\ldots{} & 3.04118215\ldots{} & 1.46565262\ldots{} & 0.126 & 0.277 & 9.12354645\ldots{} & Parp3\tabularnewline
\ldots{} & \ldots{} & \ldots{} & \ldots{} & \ldots{} & \ldots{} & \ldots{}\tabularnewline
\bottomrule
\end{longtable}

\hypertarget{ux62dfux65f6ux5206ux6790}{%
\paragraph{拟时分析}\label{ux62dfux65f6ux5206ux6790}}

选择 Control 集中区域作为拟时起点。

Figure \ref{fig:MI-principal-points} (下方图) 为图MI principal points概览。

\textbf{(对应文件为 \texttt{Figure+Table/MI-principal-points.pdf})}

\def\@captype{figure}
\begin{center}
\includegraphics[width = 0.9\linewidth]{Figure+Table/MI-principal-points.pdf}
\caption{MI principal points}\label{fig:MI-principal-points}
\end{center}

Figure \ref{fig:MI-pseudotime} (下方图) 为图MI pseudotime概览。

\textbf{(对应文件为 \texttt{Figure+Table/MI-pseudotime.pdf})}

\def\@captype{figure}
\begin{center}
\includegraphics[width = 0.9\linewidth]{Figure+Table/MI-pseudotime.pdf}
\caption{MI pseudotime}\label{fig:MI-pseudotime}
\end{center}

\hypertarget{ux4ee3ux8c22ux901aux91cfux9884ux6d4b}{%
\paragraph{代谢通量预测}\label{ux4ee3ux8c22ux901aux91cfux9884ux6d4b}}

使用 scFEA 预测 Microglial cell 代谢通量。

Figure \ref{fig:Convergency-of-the-loss-terms-during-training} (下方图) 为图Convergency of the loss terms during training概览。

\textbf{(对应文件为 \texttt{Figure+Table/Convergency-of-the-loss-terms-during-training.png})}

\def\@captype{figure}
\begin{center}
\includegraphics[width = 0.9\linewidth]{scfea/loss_20240516-014752.png}
\caption{Convergency of the loss terms during training}\label{fig:Convergency-of-the-loss-terms-during-training}
\end{center}

\hypertarget{ux4ee3ux8c22ux901aux91cfux5deeux5f02ux5206ux6790}{%
\paragraph{代谢通量差异分析}\label{ux4ee3ux8c22ux901aux91cfux5deeux5f02ux5206ux6790}}

Table \ref{tab:SCF-data-Model-vs-Control} (下方表格) 为表格SCF data Model vs Control概览。

\textbf{(对应文件为 \texttt{Figure+Table/SCF-data-Model-vs-Control.csv})}

\begin{center}\begin{tcolorbox}[colback=gray!10, colframe=gray!50, width=0.9\linewidth, arc=1mm, boxrule=0.5pt]注:表格共有13行8列,以下预览的表格可能省略部分数据;含有13个唯一`rownames'。
\end{tcolorbox}
\end{center}
\begin{center}\begin{tcolorbox}[colback=gray!10, colframe=gray!50, width=0.9\linewidth, arc=1mm, boxrule=0.5pt]\begin{enumerate}\tightlist
\item logFC:  estimate of the log2-fold-change corresponding to the effect or contrast (for ‘topTableF’ there may be several columns of log-fold-changes)
\item AveExpr:  average log2-expression for the probe over all arrays and channels, same as ‘Amean’ in the ‘MarrayLM’ object
\item t:  moderated t-statistic (omitted for ‘topTableF’)
\item P.Value:  raw p-value
\item B:  log-odds that the gene is differentially expressed (omitted for ‘topTreat’)
\end{enumerate}\end{tcolorbox}
\end{center}

\begin{longtable}[]{@{}llllllll@{}}
\caption{\label{tab:SCF-data-Model-vs-Control}SCF data Model vs Control}\tabularnewline
\toprule
rownames & logFC & AveExpr & t & P.Value & adj.P.Val & B & name\tabularnewline
\midrule
\endfirsthead
\toprule
rownames & logFC & AveExpr & t & P.Value & adj.P.Val & B & name\tabularnewline
\midrule
\endhead
M\_121 & 0.6117\ldots{} & -3.117\ldots{} & 14.506\ldots{} & 1.1338\ldots{} & 1.9048\ldots{} & 97.444\ldots{} & (Glc)3\ldots{}\tabularnewline
M\_122 & 0.5732\ldots{} & 4.8626\ldots{} & 13.592\ldots{} & 4.5332\ldots{} & 3.8079\ldots{} & 84.670\ldots{} & (GlcNA\ldots{}\tabularnewline
M\_132 & 0.5265\ldots{} & -7.310\ldots{} & 12.485\ldots{} & 9.0708\ldots{} & 5.0796\ldots{} & 70.314\ldots{} & (Gal)2\ldots{}\tabularnewline
M\_89 & 0.4437\ldots{} & -1.982\ldots{} & 10.521\ldots{} & 6.9549\ldots{} & 2.9210\ldots{} & 47.832\ldots{} & B-Alan\ldots{}\tabularnewline
M\_129 & 0.4401\ldots{} & 1.6913\ldots{} & 10.436\ldots{} & 1.7032\ldots{} & 5.7227\ldots{} & 46.948\ldots{} & Protei\ldots{}\tabularnewline
M\_119 & 0.4297\ldots{} & 1.0094\ldots{} & 10.190\ldots{} & 2.2013\ldots{} & 6.1638\ldots{} & 44.425\ldots{} & Dolich\ldots{}\tabularnewline
M\_80 & 0.4095\ldots{} & -5.647\ldots{} & 9.7110\ldots{} & 2.7185\ldots{} & 6.5245\ldots{} & 39.680\ldots{} & Cystei\ldots{}\tabularnewline
M\_125 & 0.3755\ldots{} & -3.824\ldots{} & 8.9058\ldots{} & 5.3173\ldots{} & 1.1166\ldots{} & 32.223\ldots{} & Dolich\ldots{}\tabularnewline
M\_95 & 0.3329\ldots{} & -2.699\ldots{} & 7.8944\ldots{} & 2.9235\ldots{} & 5.4572\ldots{} & 23.771\ldots{} & phenyl\ldots{}\tabularnewline
M\_120 & 0.3083\ldots{} & -8.293\ldots{} & 7.3119\ldots{} & 2.6374\ldots{} & 4.4308\ldots{} & 19.365\ldots{} & (Glc)3\ldots{}\tabularnewline
M\_109 & 0.3038\ldots{} & -1.520\ldots{} & 7.2056\ldots{} & 5.7847\ldots{} & 8.8348\ldots{} & 18.598\ldots{} & Glucos\ldots{}\tabularnewline
M\_94 & 0.3028\ldots{} & 1.1379\ldots{} & 7.1816\ldots{} & 6.8961\ldots{} & 9.6546\ldots{} & 18.426\ldots{} & Tyrosi\ldots{}\tabularnewline
M\_146 & 0.3015\ldots{} & 1.8703\ldots{} & 7.1498\ldots{} & 8.6992\ldots{} & 1.1242\ldots{} & 18.199\ldots{} & Xanthi\ldots{}\tabularnewline
\bottomrule
\end{longtable}

{[}1{]} ``Figure+Table/test1.csv''

Figure \ref{fig:SCF-Model-vs-Control} (下方图) 为图SCF Model vs Control概览。

\textbf{(对应文件为 \texttt{Figure+Table/SCF-Model-vs-Control.pdf})}

\def\@captype{figure}
\begin{center}
\includegraphics[width = 0.9\linewidth]{Figure+Table/SCF-Model-vs-Control.pdf}
\caption{SCF Model vs Control}\label{fig:SCF-Model-vs-Control}
\end{center}
\begin{center}\begin{tcolorbox}[colback=gray!10, colframe=gray!50, width=0.9\linewidth, arc=1mm, boxrule=0.5pt]
\textbf{
adj.P.Val cut-off
:}

\vspace{0.5em}

    0.05

\vspace{2em}


\textbf{
Log2(FC) cut-off
:}

\vspace{0.5em}

    0.3

\vspace{2em}
\end{tcolorbox}
\end{center}

\textbf{(上述信息框内容已保存至 \texttt{Figure+Table/SCF-Model-vs-Control-content})}

\hypertarget{ux5deeux5f02ux4ee3ux8c22ux76f8ux5173ux7684ux57faux56e0}{%
\paragraph{差异代谢相关的基因}\label{ux5deeux5f02ux4ee3ux8c22ux76f8ux5173ux7684ux57faux56e0}}

Figure \ref{fig:TOPFLUX-GO-enrichment} (下方图) 为图TOPFLUX GO enrichment概览。

\textbf{(对应文件为 \texttt{Figure+Table/TOPFLUX-GO-enrichment.pdf})}

\def\@captype{figure}
\begin{center}
\includegraphics[width = 0.9\linewidth]{Figure+Table/TOPFLUX-GO-enrichment.pdf}
\caption{TOPFLUX GO enrichment}\label{fig:TOPFLUX-GO-enrichment}
\end{center}

Figure \ref{fig:MI-Pseudotime-heatmap-of-genes} (下方图) 为图MI Pseudotime heatmap of genes概览。

\textbf{(对应文件为 \texttt{Figure+Table/MI-Pseudotime-heatmap-of-genes.pdf})}

\def\@captype{figure}
\begin{center}
\includegraphics[width = 0.9\linewidth]{Figure+Table/MI-Pseudotime-heatmap-of-genes.pdf}
\caption{MI Pseudotime heatmap of genes}\label{fig:MI-Pseudotime-heatmap-of-genes}
\end{center}

\hypertarget{ppi-ux7f51ux7edc}{%
\paragraph{PPI 网络}\label{ppi-ux7f51ux7edc}}

根据 Fig. \ref{fig:MI-Pseudotime-heatmap-of-genes} 中的基因,
将小鼠的基因 mgi symbol 映射为人类的基因 hgnc symbol,
构建 PPI 网络。

Figure \ref{fig:TOPFLUX-MCC-score} (下方图) 为图TOPFLUX MCC score概览。

\textbf{(对应文件为 \texttt{Figure+Table/TOPFLUX-MCC-score.pdf})}

\def\@captype{figure}
\begin{center}
\includegraphics[width = 0.9\linewidth]{Figure+Table/TOPFLUX-MCC-score.pdf}
\caption{TOPFLUX MCC score}\label{fig:TOPFLUX-MCC-score}
\end{center}

关注高变基因 (variable features, 差异水平更高) 与其它基因对应的蛋白的互作。

Figure \ref{fig:TOPFLUX-Top-MCC-score} (下方图) 为图TOPFLUX Top MCC score概览。

\textbf{(对应文件为 \texttt{Figure+Table/TOPFLUX-Top-MCC-score.pdf})}

\def\@captype{figure}
\begin{center}
\includegraphics[width = 0.9\linewidth]{Figure+Table/TOPFLUX-Top-MCC-score.pdf}
\caption{TOPFLUX Top MCC score}\label{fig:TOPFLUX-Top-MCC-score}
\end{center}

\hypertarget{ux4e0aux6e38ux7684ux8f6cux5f55ux56e0ux5b50}{%
\paragraph{上游的转录因子}\label{ux4e0aux6e38ux7684ux8f6cux5f55ux56e0ux5b50}}

寻找 Fig. \ref{fig:MI-Pseudotime-heatmap-of-genes} 中的基因的上游转录因子。

Table \ref{tab:Transcription-Factor-binding-sites} (下方表格) 为表格Transcription Factor binding sites概览。

\textbf{(对应文件为 \texttt{Figure+Table/Transcription-Factor-binding-sites.csv})}

\begin{center}\begin{tcolorbox}[colback=gray!10, colframe=gray!50, width=0.9\linewidth, arc=1mm, boxrule=0.5pt]注:表格共有15113行10列,以下预览的表格可能省略部分数据;含有56个唯一`target'。
\end{tcolorbox}
\end{center}
\begin{center}\begin{tcolorbox}[colback=gray!10, colframe=gray!50, width=0.9\linewidth, arc=1mm, boxrule=0.5pt]\begin{enumerate}\tightlist
\item Start:  起始点
\end{enumerate}\end{tcolorbox}
\end{center}

\begin{longtable}[]{@{}llllllllll@{}}
\caption{\label{tab:Transcription-Factor-binding-sites}Transcription Factor binding sites}\tabularnewline
\toprule
target & TF\_symbol & Motif & Source & Strand & Start & Stop & PValue & MatchS\ldots{} & Overla\ldots{}\tabularnewline
\midrule
\endfirsthead
\toprule
target & TF\_symbol & Motif & Source & Strand & Start & Stop & PValue & MatchS\ldots{} & Overla\ldots{}\tabularnewline
\midrule
\endhead
XDH & FOXB1 & FOXB1\_\ldots{} & SELEX & + & 31634777 & 31634794 & 2.0E-06 & TTGATA\ldots{} & 18\tabularnewline
XDH & FOXB1 & FOXB1\_\ldots{} & SELEX & - & 31634777 & 31634794 & 4.0E-06 & ATAGTC\ldots{} & 18\tabularnewline
XDH & POU4F2 & POU4F2\ldots{} & SELEX & + & 31635992 & 31636007 & 4.0E-06 & TTTAAT\ldots{} & 16\tabularnewline
XDH & HOXD12 & HOXD12\ldots{} & SELEX & + & 31636000 & 31636008 & 5.0E-06 & ATAATAAAA & 9\tabularnewline
XDH & HOXD12 & HOXD12\ldots{} & SELEX & + & 31636072 & 31636080 & 2.0E-06 & CTAATAAAA & 9\tabularnewline
XDH & FOXJ2 & FOXJ2\_\ldots{} & SELEX & + & 31635996 & 31636008 & 2.0E-06 & ATAAAT\ldots{} & 13\tabularnewline
XDH & FOXJ2 & FOXJ2\_\ldots{} & SELEX & + & 31637732 & 31637744 & 9.0E-06 & GCAAAC\ldots{} & 13\tabularnewline
XDH & HOXC10 & Hoxc10\ldots{} & SELEX & + & 31636000 & 31636009 & 6.0E-06 & ATAATA\ldots{} & 10\tabularnewline
XDH & HOXC10 & Hoxc10\ldots{} & SELEX & + & 31636072 & 31636081 & 3.0E-06 & CTAATA\ldots{} & 10\tabularnewline
XDH & HOXC10 & Hoxc10\ldots{} & SELEX & - & 31639392 & 31639401 & 7.0E-06 & ACAATA\ldots{} & 10\tabularnewline
XDH & SOX21 & SOX21\_\ldots{} & SELEX & - & 31636149 & 31636163 & 0.0E+00 & AGCAAT\ldots{} & 15\tabularnewline
XDH & SOX4 & SOX4\_H\ldots{} & SELEX & - & 31636149 & 31636164 & 1.0E-06 & CAGCAA\ldots{} & 16\tabularnewline
XDH & HOXD9 & Hoxd9\_\ldots{} & SELEX & - & 31639392 & 31639401 & 3.0E-06 & ACAATA\ldots{} & 10\tabularnewline
XDH & POU2F1 & POU2F1\ldots{} & SELEX & + & 31635995 & 31636008 & 5.0E-06 & AATAAA\ldots{} & 14\tabularnewline
XDH & POU2F1 & POU2F1\ldots{} & SELEX & + & 31637713 & 31637726 & 6.0E-06 & TTTACA\ldots{} & 14\tabularnewline
\ldots{} & \ldots{} & \ldots{} & \ldots{} & \ldots{} & \ldots{} & \ldots{} & \ldots{} & \ldots{} & \ldots{}\tabularnewline
\bottomrule
\end{longtable}

\hypertarget{ux661fux5f62ux7ec6ux80deux80f6ux8d28ux7622ux75d5-ux76f8ux5173ux57faux56e0}{%
\subsubsection{星形细胞胶质瘢痕 相关基因}\label{ux661fux5f62ux7ec6ux80deux80f6ux8d28ux7622ux75d5-ux76f8ux5173ux57faux56e0}}

\hypertarget{ux5deeux5f02ux5206ux6790-1}{%
\paragraph{差异分析}\label{ux5deeux5f02ux5206ux6790-1}}

Table \ref{tab:DEGs-of-the-contrasts-Astrocyte} (下方表格) 为表格DEGs of the contrasts Astrocyte概览。

\textbf{(对应文件为 \texttt{Figure+Table/DEGs-of-the-contrasts-Astrocyte.csv})}

\begin{center}\begin{tcolorbox}[colback=gray!10, colframe=gray!50, width=0.9\linewidth, arc=1mm, boxrule=0.5pt]注:表格共有1058行7列,以下预览的表格可能省略部分数据;含有1个唯一`contrast'。
\end{tcolorbox}
\end{center}

\begin{longtable}[]{@{}lllllll@{}}
\caption{\label{tab:DEGs-of-the-contrasts-Astrocyte}DEGs of the contrasts Astrocyte}\tabularnewline
\toprule
contrast & p\_val & avg\_log2FC & pct.1 & pct.2 & p\_val\_adj & gene\tabularnewline
\midrule
\endfirsthead
\toprule
contrast & p\_val & avg\_log2FC & pct.1 & pct.2 & p\_val\_adj & gene\tabularnewline
\midrule
\endhead
Astrocyte\_\ldots{} & 4.37264108\ldots{} & 8.78114541\ldots{} & 0.024 & 0.107 & 1.31179232\ldots{} & Dnah11\tabularnewline
Astrocyte\_\ldots{} & 3.64754326\ldots{} & 2.65537976\ldots{} & 0.093 & 0.243 & 1.09426297\ldots{} & H2-Aa\tabularnewline
Astrocyte\_\ldots{} & 2.70284056\ldots{} & 2.40818618\ldots{} & 0.056 & 0.148 & 8.10852170\ldots{} & Cd53\tabularnewline
Astrocyte\_\ldots{} & 6.14268687\ldots{} & 8.30800266\ldots{} & 0.08 & 0.147 & 1.84280606\ldots{} & Thbs1\tabularnewline
Astrocyte\_\ldots{} & 9.59253467\ldots{} & 2.25479722\ldots{} & 0.062 & 0.198 & 2.87776040\ldots{} & Mpeg1\tabularnewline
Astrocyte\_\ldots{} & 1.90263565\ldots{} & 14.3762318\ldots{} & 0.073 & 0.274 & 5.70790697\ldots{} & Top2a\tabularnewline
Astrocyte\_\ldots{} & 3.92239217\ldots{} & 4.31264893\ldots{} & 0.071 & 0.192 & 1.17671765\ldots{} & Cebpb\tabularnewline
Astrocyte\_\ldots{} & 7.06233131\ldots{} & 7.35551440\ldots{} & 0.058 & 0.21 & 2.11869939\ldots{} & Grap2\tabularnewline
Astrocyte\_\ldots{} & 4.42147374\ldots{} & 2.20926332\ldots{} & 0.069 & 0.205 & 1.32644212\ldots{} & A630001G21Rik\tabularnewline
Astrocyte\_\ldots{} & 1.77494299\ldots{} & 0.93300639\ldots{} & 0.106 & 0.063 & 5.32482897\ldots{} & Stab1\tabularnewline
Astrocyte\_\ldots{} & 6.62357870\ldots{} & -1.2610776\ldots{} & 0.075 & 0.165 & 1.98707361\ldots{} & Cd247\tabularnewline
Astrocyte\_\ldots{} & 1.03473035\ldots{} & -1.9719507\ldots{} & 0.136 & 0.196 & 3.10419105\ldots{} & Gpnmb\tabularnewline
Astrocyte\_\ldots{} & 3.73638612\ldots{} & 12.9129020\ldots{} & 0.14 & 0.225 & 1.12091583\ldots{} & Bcl3\tabularnewline
Astrocyte\_\ldots{} & 3.00567566\ldots{} & 4.02484698\ldots{} & 0.14 & 0.248 & 9.01702698\ldots{} & Ifitm3\tabularnewline
Astrocyte\_\ldots{} & 1.91228260\ldots{} & 5.32650753\ldots{} & 0.047 & 0.244 & 5.73684782\ldots{} & Rnf17\tabularnewline
\ldots{} & \ldots{} & \ldots{} & \ldots{} & \ldots{} & \ldots{} & \ldots{}\tabularnewline
\bottomrule
\end{longtable}

将这些差异基因映射到人类的基因 hgnc\_symbol

Table \ref{tab:mapped-genes-Astrocyte-DEGs} (下方表格) 为表格mapped genes Astrocyte DEGs概览。

\textbf{(对应文件为 \texttt{Figure+Table/mapped-genes-Astrocyte-DEGs.csv})}

\begin{center}\begin{tcolorbox}[colback=gray!10, colframe=gray!50, width=0.9\linewidth, arc=1mm, boxrule=0.5pt]注:表格共有921行2列,以下预览的表格可能省略部分数据;含有892个唯一`mgi\_symbol;含有914个唯一`hgnc\_symbol'。
\end{tcolorbox}
\end{center}
\begin{center}\begin{tcolorbox}[colback=gray!10, colframe=gray!50, width=0.9\linewidth, arc=1mm, boxrule=0.5pt]\begin{enumerate}\tightlist
\item hgnc\_symbol:  基因名 (Human)
\item mgi\_symbol:  基因名 (Mice)
\end{enumerate}\end{tcolorbox}
\end{center}

\begin{longtable}[]{@{}ll@{}}
\caption{\label{tab:mapped-genes-Astrocyte-DEGs}Mapped genes Astrocyte DEGs}\tabularnewline
\toprule
mgi\_symbol & hgnc\_symbol\tabularnewline
\midrule
\endfirsthead
\toprule
mgi\_symbol & hgnc\_symbol\tabularnewline
\midrule
\endhead
Man1a & MAN1A1\tabularnewline
Slc7a7 & SLC7A7\tabularnewline
B4galt1 & B4GALT1\tabularnewline
Xdh & XDH\tabularnewline
St6gal1 & ST6GAL1\tabularnewline
mt-Atp8 & MT-ATP8\tabularnewline
Egln3 & EGLN3\tabularnewline
Mis18bp1 & MIS18BP1\tabularnewline
Cdc42ep1 & CDC42EP1\tabularnewline
Sdc4 & SDC4\tabularnewline
Cd93 & CD93\tabularnewline
Shroom2 & SHROOM2\tabularnewline
Jag1 & JAG1\tabularnewline
Fbxo7 & FBXO7\tabularnewline
Parp4 & PARP4\tabularnewline
\ldots{} & \ldots{}\tabularnewline
\bottomrule
\end{longtable}

\hypertarget{ux80f6ux8d28ux7622ux75d5}{%
\paragraph{胶质瘢痕}\label{ux80f6ux8d28ux7622ux75d5}}

从 genecards 获取 胶质瘢痕 相关基因。

Table \ref{tab:GLIALSCAR-disease-related-targets-from-GeneCards} (下方表格) 为表格GLIALSCAR disease related targets from GeneCards概览。

\textbf{(对应文件为 \texttt{Figure+Table/GLIALSCAR-disease-related-targets-from-GeneCards.csv})}

\begin{center}\begin{tcolorbox}[colback=gray!10, colframe=gray!50, width=0.9\linewidth, arc=1mm, boxrule=0.5pt]注:表格共有90行7列,以下预览的表格可能省略部分数据;含有90个唯一`Symbol'。
\end{tcolorbox}
\end{center}\begin{center}\begin{tcolorbox}[colback=gray!10, colframe=gray!50, width=0.9\linewidth, arc=1mm, boxrule=0.5pt]
\textbf{
The GeneCards data was obtained by querying
:}

\vspace{0.5em}

    Glial scar

\vspace{2em}


\textbf{
Restrict (with quotes)
:}

\vspace{0.5em}

    TRUE

\vspace{2em}


\textbf{
Filtering by Score:
:}

\vspace{0.5em}

    Score > 0

\vspace{2em}
\end{tcolorbox}
\end{center}

\begin{longtable}[]{@{}lllllll@{}}
\caption{\label{tab:GLIALSCAR-disease-related-targets-from-GeneCards}GLIALSCAR disease related targets from GeneCards}\tabularnewline
\toprule
Symbol & Description & Category & UniProt\_ID & GIFtS & GC\_id & Score\tabularnewline
\midrule
\endfirsthead
\toprule
Symbol & Description & Category & UniProt\_ID & GIFtS & GC\_id & Score\tabularnewline
\midrule
\endhead
BDNF-AS & BDNF Antis\ldots{} & RNA Gene (\ldots{} & & 29 & GC11P027466 & 4.85\tabularnewline
TRA-TGC7-1 & TRNA-Ala (\ldots{} & RNA Gene (\ldots{} & & 14 & GC06M093612 & 2.82\tabularnewline
CSPG4 & Chondroiti\ldots{} & Protein Co\ldots{} & Q6UVK1 & 54 & GC15M075674 & 2.4\tabularnewline
GFAP & Glial Fibr\ldots{} & Protein Co\ldots{} & P14136 & 57 & GC17M077883 & 1.91\tabularnewline
MAG & Myelin Ass\ldots{} & Protein Co\ldots{} & P20916 & 55 & GC19P035292 & 1.78\tabularnewline
TNR & Tenascin R & Protein Co\ldots{} & Q92752 & 51 & GC01M175291 & 1.78\tabularnewline
MYOC & Myocilin & Protein Co\ldots{} & Q99972 & 50 & GC01M171604 & 1.78\tabularnewline
MMP9 & Matrix Met\ldots{} & Protein Co\ldots{} & P14780 & 62 & GC20P046008 & 1.7\tabularnewline
S100B & S100 Calci\ldots{} & Protein Co\ldots{} & P04271 & 53 & GC21M053599 & 1.7\tabularnewline
PDGFRB & Platelet D\ldots{} & Protein Co\ldots{} & P09619 & 62 & GC05M150113 & 1.59\tabularnewline
CST3 & Cystatin C & Protein Co\ldots{} & P01034 & 53 & GC20M023930 & 1.59\tabularnewline
NES & Nestin & Protein Co\ldots{} & P48681 & 50 & GC01M156668 & 1.59\tabularnewline
FGFR4 & Fibroblast\ldots{} & Protein Co\ldots{} & P22455 & 60 & GC05P177086 & 1.32\tabularnewline
FGF2 & Fibroblast\ldots{} & Protein Co\ldots{} & P09038 & 54 & GC04P122826 & 1.32\tabularnewline
GALNS & Galactosam\ldots{} & Protein Co\ldots{} & P34059 & 56 & GC16M088813 & 0.7\tabularnewline
\ldots{} & \ldots{} & \ldots{} & \ldots{} & \ldots{} & \ldots{} & \ldots{}\tabularnewline
\bottomrule
\end{longtable}

Figure \ref{fig:Intersection-of-DB-GlialScar-with-Astrocyte-DEGs} (下方图) 为图Intersection of DB GlialScar with Astrocyte DEGs概览。

\textbf{(对应文件为 \texttt{Figure+Table/Intersection-of-DB-GlialScar-with-Astrocyte-DEGs.pdf})}

\def\@captype{figure}
\begin{center}
\includegraphics[width = 0.9\linewidth]{Figure+Table/Intersection-of-DB-GlialScar-with-Astrocyte-DEGs.pdf}
\caption{Intersection of DB GlialScar with Astrocyte DEGs}\label{fig:Intersection-of-DB-GlialScar-with-Astrocyte-DEGs}
\end{center}
\begin{center}\begin{tcolorbox}[colback=gray!10, colframe=gray!50, width=0.9\linewidth, arc=1mm, boxrule=0.5pt]
\textbf{
Intersection
:}

\vspace{0.5em}

    GFAP, PDGFRB, CST3, VIM, BDNF, DCN, TGM2, VCAN, TNC,
MRC1, CCL15-CCL14

\vspace{2em}
\end{tcolorbox}
\end{center}

\textbf{(上述信息框内容已保存至 \texttt{Figure+Table/Intersection-of-DB-GlialScar-with-Astrocyte-DEGs-content})}

\hypertarget{ux8c03ux63a7ux5c0fux80f6ux8d28ux7ec6ux80deux4ee3ux8c22ux4e14ux4e0eux661fux5f62ux7ec6ux80deux80f6ux8d28ux7622ux75d5ux76f8ux5173ux57faux56e0}{%
\subsubsection{调控小胶质细胞代谢且与星形细胞胶质瘢痕相关基因}\label{ux8c03ux63a7ux5c0fux80f6ux8d28ux7ec6ux80deux4ee3ux8c22ux4e14ux4e0eux661fux5f62ux7ec6ux80deux80f6ux8d28ux7622ux75d5ux76f8ux5173ux57faux56e0}}

\hypertarget{ux4ea4ux96c6ux57faux56e0}{%
\paragraph{交集基因}\label{ux4ea4ux96c6ux57faux56e0}}

Figure \ref{fig:UpSet-plot-of-Genes-sources} (下方图) 为图UpSet plot of Genes sources概览。

\textbf{(对应文件为 \texttt{Figure+Table/UpSet-plot-of-Genes-sources.pdf})}

\def\@captype{figure}
\begin{center}
\includegraphics[width = 0.9\linewidth]{Figure+Table/UpSet-plot-of-Genes-sources.pdf}
\caption{UpSet plot of Genes sources}\label{fig:UpSet-plot-of-Genes-sources}
\end{center}
\begin{center}\begin{tcolorbox}[colback=gray!10, colframe=gray!50, width=0.9\linewidth, arc=1mm, boxrule=0.5pt]
\textbf{
All\_intersection
:}

\vspace{0.5em}



\vspace{2em}


\textbf{
Other intersection
:}

\vspace{0.5em}

    "FluxRelated\_DEGs" WITH "DB\_GlialScarRelated": XYLT1
\newline "FluxRelated\_TFs\_DEGs" WITH "DB\_GlialScarRelated":
STAT3

\vspace{2em}
\end{tcolorbox}
\end{center}

\textbf{(上述信息框内容已保存至 \texttt{Figure+Table/UpSet-plot-of-Genes-sources-content})}

\hypertarget{ux5728ux5c0fux80f6ux8d28ux7ec6ux80deux4e2dux7684ux8868ux8fbe}{%
\paragraph{在小胶质细胞中的表达}\label{ux5728ux5c0fux80f6ux8d28ux7ec6ux80deux4e2dux7684ux8868ux8fbe}}

可以发现,`Xylt1' 主要集中表达于模型组的 Microglial 中。

Figure \ref{fig:Dimension-plot-of-expression-level-of-the-genes} (下方图) 为图Dimension plot of expression level of the genes概览。

\textbf{(对应文件为 \texttt{Figure+Table/Dimension-plot-of-expression-level-of-the-genes.pdf})}

\def\@captype{figure}
\begin{center}
\includegraphics[width = 0.9\linewidth]{Figure+Table/Dimension-plot-of-expression-level-of-the-genes.pdf}
\caption{Dimension plot of expression level of the genes}\label{fig:Dimension-plot-of-expression-level-of-the-genes}
\end{center}

Table \ref{tab:Xylt1-related-metabolic-flux} (下方表格) 为表格Xylt1 related metabolic flux概览。

\textbf{(对应文件为 \texttt{Figure+Table/Xylt1-related-metabolic-flux.csv})}

\begin{center}\begin{tcolorbox}[colback=gray!10, colframe=gray!50, width=0.9\linewidth, arc=1mm, boxrule=0.5pt]注:表格共有1行2列,以下预览的表格可能省略部分数据;含有1个唯一`gene'。
\end{tcolorbox}
\end{center}

\begin{longtable}[]{@{}ll@{}}
\caption{\label{tab:Xylt1-related-metabolic-flux}Xylt1 related metabolic flux}\tabularnewline
\toprule
gene & Metabolic\_flux\tabularnewline
\midrule
\endfirsthead
\toprule
gene & Metabolic\_flux\tabularnewline
\midrule
\endhead
Xylt1 & Protein serine -\textgreater{} (Gal)2 (G\ldots{}\tabularnewline
\bottomrule
\end{longtable}

\hypertarget{bibliography}{%
\section*{Reference}\label{bibliography}}
\addcontentsline{toc}{section}{Reference}

\hypertarget{refs}{}
\begin{cslreferences}
\leavevmode\hypertarget{ref-MappingIdentifDurinc2009}{}%
1. Durinck, S., Spellman, P. T., Birney, E. \& Huber, W. Mapping identifiers for the integration of genomic datasets with the r/bioconductor package biomaRt. \emph{Nature protocols} \textbf{4}, 1184--1191 (2009).

\leavevmode\hypertarget{ref-AGraphNeuralAlgham2021}{}%
2. Alghamdi, N. \emph{et al.} A graph neural network model to estimate cell-wise metabolic flux using single-cell rna-seq data. \emph{Genome research} \textbf{31}, 1867--1884 (2021).

\leavevmode\hypertarget{ref-ClusterprofilerWuTi2021}{}%
3. Wu, T. \emph{et al.} ClusterProfiler 4.0: A universal enrichment tool for interpreting omics data. \emph{The Innovation} \textbf{2}, (2021).

\leavevmode\hypertarget{ref-TheGenecardsSStelze2016}{}%
4. Stelzer, G. \emph{et al.} The genecards suite: From gene data mining to disease genome sequence analyses. \emph{Current protocols in bioinformatics} \textbf{54}, 1.30.1--1.30.33 (2016).

\leavevmode\hypertarget{ref-LimmaPowersDiRitchi2015}{}%
5. Ritchie, M. E. \emph{et al.} Limma powers differential expression analyses for rna-sequencing and microarray studies. \emph{Nucleic Acids Research} \textbf{43}, e47 (2015).

\leavevmode\hypertarget{ref-EdgerDifferenChen}{}%
6. Chen, Y., McCarthy, D., Ritchie, M., Robinson, M. \& Smyth, G. EdgeR: Differential analysis of sequence read count data user's guide. 119.

\leavevmode\hypertarget{ref-ReversedGraphQiuX2017}{}%
7. Qiu, X. \emph{et al.} Reversed graph embedding resolves complex single-cell trajectories. \emph{Nature Methods} \textbf{14}, (2017).

\leavevmode\hypertarget{ref-TheDynamicsAnTrapne2014}{}%
8. Trapnell, C. \emph{et al.} The dynamics and regulators of cell fate decisions are revealed by pseudotemporal ordering of single cells. \emph{Nature Biotechnology} \textbf{32}, (2014).

\leavevmode\hypertarget{ref-TheStringDataSzklar2021}{}%
9. Szklarczyk, D. \emph{et al.} The string database in 2021: Customizable proteinprotein networks, and functional characterization of user-uploaded gene/measurement sets. \emph{Nucleic Acids Research} \textbf{49}, D605--D612 (2021).

\leavevmode\hypertarget{ref-CytohubbaIdenChin2014}{}%
10. Chin, C.-H. \emph{et al.} CytoHubba: Identifying hub objects and sub-networks from complex interactome. \emph{BMC Systems Biology} \textbf{8}, S11 (2014).

\leavevmode\hypertarget{ref-IntegratedAnalHaoY2021}{}%
11. Hao, Y. \emph{et al.} Integrated analysis of multimodal single-cell data. \emph{Cell} \textbf{184}, (2021).

\leavevmode\hypertarget{ref-ComprehensiveIStuart2019}{}%
12. Stuart, T. \emph{et al.} Comprehensive integration of single-cell data. \emph{Cell} \textbf{177}, (2019).

\leavevmode\hypertarget{ref-CausalMechanisPlaisi2016}{}%
13. Plaisier, C. L. \emph{et al.} Causal mechanistic regulatory network for glioblastoma deciphered using systems genetics network analysis. \emph{Cell systems} \textbf{3}, 172--186 (2016).

\leavevmode\hypertarget{ref-ScsaACellTyCaoY2020}{}%
14. Cao, Y., Wang, X. \& Peng, G. SCSA: A cell type annotation tool for single-cell rna-seq data. \emph{Frontiers in genetics} \textbf{11}, (2020).
\end{cslreferences}

\end{document}
