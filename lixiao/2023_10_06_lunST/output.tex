% Options for packages loaded elsewhere
\PassOptionsToPackage{unicode}{hyperref}
\PassOptionsToPackage{hyphens}{url}
%
\documentclass[
]{article}
\usepackage{lmodern}
\usepackage{amssymb,amsmath}
\usepackage{ifxetex,ifluatex}
\ifnum 0\ifxetex 1\fi\ifluatex 1\fi=0 % if pdftex
  \usepackage[T1]{fontenc}
  \usepackage[utf8]{inputenc}
  \usepackage{textcomp} % provide euro and other symbols
\else % if luatex or xetex
  \usepackage{unicode-math}
  \defaultfontfeatures{Scale=MatchLowercase}
  \defaultfontfeatures[\rmfamily]{Ligatures=TeX,Scale=1}
\fi
% Use upquote if available, for straight quotes in verbatim environments
\IfFileExists{upquote.sty}{\usepackage{upquote}}{}
\IfFileExists{microtype.sty}{% use microtype if available
  \usepackage[]{microtype}
  \UseMicrotypeSet[protrusion]{basicmath} % disable protrusion for tt fonts
}{}
\makeatletter
\@ifundefined{KOMAClassName}{% if non-KOMA class
  \IfFileExists{parskip.sty}{%
    \usepackage{parskip}
  }{% else
    \setlength{\parindent}{0pt}
    \setlength{\parskip}{6pt plus 2pt minus 1pt}}
}{% if KOMA class
  \KOMAoptions{parskip=half}}
\makeatother
\usepackage{xcolor}
\IfFileExists{xurl.sty}{\usepackage{xurl}}{} % add URL line breaks if available
\IfFileExists{bookmark.sty}{\usepackage{bookmark}}{\usepackage{hyperref}}
\hypersetup{
  pdftitle={Analysis},
  pdfauthor={Huang LiChuang of Wie-Biotech},
  hidelinks,
  pdfcreator={LaTeX via pandoc}}
\urlstyle{same} % disable monospaced font for URLs
\usepackage[margin=1in]{geometry}
\usepackage{longtable,booktabs}
% Correct order of tables after \paragraph or \subparagraph
\usepackage{etoolbox}
\makeatletter
\patchcmd\longtable{\par}{\if@noskipsec\mbox{}\fi\par}{}{}
\makeatother
% Allow footnotes in longtable head/foot
\IfFileExists{footnotehyper.sty}{\usepackage{footnotehyper}}{\usepackage{footnote}}
\makesavenoteenv{longtable}
\usepackage{graphicx}
\makeatletter
\def\maxwidth{\ifdim\Gin@nat@width>\linewidth\linewidth\else\Gin@nat@width\fi}
\def\maxheight{\ifdim\Gin@nat@height>\textheight\textheight\else\Gin@nat@height\fi}
\makeatother
% Scale images if necessary, so that they will not overflow the page
% margins by default, and it is still possible to overwrite the defaults
% using explicit options in \includegraphics[width, height, ...]{}
\setkeys{Gin}{width=\maxwidth,height=\maxheight,keepaspectratio}
% Set default figure placement to htbp
\makeatletter
\def\fps@figure{htbp}
\makeatother
\setlength{\emergencystretch}{3em} % prevent overfull lines
\providecommand{\tightlist}{%
  \setlength{\itemsep}{0pt}\setlength{\parskip}{0pt}}
\setcounter{secnumdepth}{5}
\usepackage{caption} \captionsetup{font={footnotesize},width=6in} \renewcommand{\dblfloatpagefraction}{.9} \makeatletter \renewenvironment{figure} {\def\@captype{figure}} \makeatother \definecolor{shadecolor}{RGB}{242,242,242} \usepackage{xeCJK} \usepackage{setspace} \setstretch{1.3} \usepackage{tcolorbox}
\newlength{\cslhangindent}
\setlength{\cslhangindent}{1.5em}
\newenvironment{cslreferences}%
  {}%
  {\par}

\title{Analysis}
\author{Huang LiChuang of Wie-Biotech}
\date{}

\begin{document}
\maketitle

{
\setcounter{tocdepth}{3}
\tableofcontents
}
\listoffigures

\listoftables

\hypertarget{abstract}{%
\section{摘要}\label{abstract}}

空间转录组分析:

\begin{itemize}
\tightlist
\item
  鉴定肿瘤细胞
\item
  肿瘤细胞亚群分析
\item
  肿瘤细胞和正常细胞之间的差异分析
\item
  细胞通讯分析
\item
  肿瘤细胞和正常细胞(巨噬细胞)之间的通讯分析。
\end{itemize}

\hypertarget{route}{%
\section{研究设计流程图}\label{route}}

\hypertarget{methods}{%
\section{材料和方法}\label{methods}}

\begin{itemize}
\tightlist
\item
  Seurat
\item
  copyKAT\textsuperscript{\protect\hyperlink{ref-DelineatingCopGaoR2021}{1}}
\item
  monocle3
\item
  cellchat
\item
  clusterProfiler
\item
  \ldots{}
\end{itemize}

\hypertarget{results}{%
\section{分析结果}\label{results}}

\begin{itemize}
\tightlist
\item
  以 copykat 鉴定了肿瘤细胞,主要为上皮细胞 (epithelial cells, EC) 或 基底细胞(basal cells, BC)。
\item
  以拟时分析将肿瘤细胞分为三个亚型(cancer 1,cancer 2,cancer 3),表现出时间(拟时)递进变化。
\item
  肿瘤细胞亚型之间的差异分析,肿瘤细胞和正常细胞(EC, BC)之间的差异分析,主要聚焦的通路有 `Phagosome'、'Antigen processing and presentation'、`Focal adhesion' 等。
\item
  细胞通讯分析的整体情况见 Fig. \ref{fig:all-cells-communication-significance}。
\item
  巨噬细胞和肿瘤细胞的通讯,突出表现为 ITGA 受体配体相关通路,共 27 个基因(其中多数为近似的亚型),首要富集于 PI3K-Akt 相关通路,并且和 `Focal adhesion' 等上述分析的结果相一致。
\end{itemize}

\hypertarget{dis}{%
\section{结论}\label{dis}}

见 \ref{results}。

\hypertarget{workflow1}{%
\section{附:分析流程(癌组织切片)}\label{workflow1}}

\hypertarget{st-ux6570ux636eux9884ux5904ux7406}{%
\subsection{ST 数据预处理}\label{st-ux6570ux636eux9884ux5904ux7406}}

Figure \ref{fig:spatial-sample-QC}为图spatial sample QC概览。
选择基因数 2500 - 9000 作为过滤指标。

\textbf{(对应文件为 \texttt{Figure+Table/spatial-sample-QC.pdf})}

\def\@captype{figure}
\begin{center}
\includegraphics[width = 0.9\linewidth]{Figure+Table/spatial-sample-QC.pdf}
\caption{Spatial sample QC}\label{fig:spatial-sample-QC}
\end{center}

Figure \ref{fig:PCA-ranking}为图PCA ranking概览。
选择主成份 1-15 用于后续聚类。

\textbf{(对应文件为 \texttt{Figure+Table/PCA-ranking.pdf})}

\def\@captype{figure}
\begin{center}
\includegraphics[width = 0.9\linewidth]{Figure+Table/PCA-ranking.pdf}
\caption{PCA ranking}\label{fig:PCA-ranking}
\end{center}

\hypertarget{ux7ec6ux80deux6ce8ux91ca}{%
\subsection{细胞注释}\label{ux7ec6ux80deux6ce8ux91ca}}

\hypertarget{ux6240ux6709ux7ec6ux80deux7c7bseurat-clustersux7684-marker}{%
\subsubsection{所有细胞类(Seurat clusters)的 marker}\label{ux6240ux6709ux7ec6ux80deux7c7bseurat-clustersux7684-marker}}

Table \ref{tab:all-markers-of-Seurat-clusters}为表格all markers of Seurat clusters概览。

\textbf{(对应文件为 \texttt{Figure+Table/all-markers-of-Seurat-clusters.csv})}

\begin{center}\begin{tcolorbox}[colback=gray!10, colframe=gray!50, width=0.9\linewidth, arc=1mm, boxrule=0.5pt]注:表格共有4370行8列,以下预览的表格可能省略部分数据;表格含有15个唯一`cluster'。
\end{tcolorbox}
\end{center}

\begin{longtable}[]{@{}llllllll@{}}
\caption{\label{tab:all-markers-of-Seurat-clusters}All markers of Seurat clusters}\tabularnewline
\toprule
rownames & p\_val & avg\_l\ldots{} & pct.1 & pct.2 & p\_val\ldots{} & cluster & gene\tabularnewline
\midrule
\endfirsthead
\toprule
rownames & p\_val & avg\_l\ldots{} & pct.1 & pct.2 & p\_val\ldots{} & cluster & gene\tabularnewline
\midrule
\endhead
ISLR & 1.244\ldots{} & 2.772\ldots{} & 0.97 & 0.385 & 2.437\ldots{} & 0 & ISLR\tabularnewline
FN1 & 2.313\ldots{} & 2.726\ldots{} & 1 & 0.971 & 4.533\ldots{} & 0 & FN1\tabularnewline
COMP & 4.009\ldots{} & 3.497\ldots{} & 0.827 & 0.192 & 7.857\ldots{} & 0 & COMP\tabularnewline
COL1A1 & 1.812\ldots{} & 2.682\ldots{} & 1 & 0.978 & 3.551\ldots{} & 0 & COL1A1\tabularnewline
TAGLN & 4.759\ldots{} & 2.643\ldots{} & 1 & 0.877 & 9.326\ldots{} & 0 & TAGLN\tabularnewline
BGN & 5.201\ldots{} & 2.540\ldots{} & 1 & 0.76 & 1.019\ldots{} & 0 & BGN\tabularnewline
COL1A2 & 1.085\ldots{} & 2.444\ldots{} & 1 & 0.988 & 2.126\ldots{} & 0 & COL1A2\tabularnewline
AEBP1 & 3.097\ldots{} & 2.326\ldots{} & 1 & 0.893 & 6.069\ldots{} & 0 & AEBP1\tabularnewline
RARRES2 & 1.383\ldots{} & 3.100\ldots{} & 1 & 0.716 & 2.711\ldots{} & 0 & RARRES2\tabularnewline
COL5A1 & 1.498\ldots{} & 2.494\ldots{} & 0.992 & 0.671 & 2.937\ldots{} & 0 & COL5A1\tabularnewline
SPARC & 7.275\ldots{} & 2.190\ldots{} & 1 & 0.951 & 1.425\ldots{} & 0 & SPARC\tabularnewline
LUM & 1.446\ldots{} & 2.464\ldots{} & 1 & 0.886 & 2.834\ldots{} & 0 & LUM\tabularnewline
MYL9 & 3.618\ldots{} & 2.054\ldots{} & 1 & 0.915 & 7.089\ldots{} & 0 & MYL9\tabularnewline
NNMT & 3.777\ldots{} & 2.486\ldots{} & 1 & 0.675 & 7.401\ldots{} & 0 & NNMT\tabularnewline
POSTN & 4.428\ldots{} & 2.680\ldots{} & 1 & 0.765 & 8.676\ldots{} & 0 & POSTN\tabularnewline
\ldots{} & \ldots{} & \ldots{} & \ldots{} & \ldots{} & \ldots{} & \ldots{} & \ldots{}\tabularnewline
\bottomrule
\end{longtable}

\hypertarget{scsa-ux6ce8ux91ca}{%
\subsubsection{SCSA 注释}\label{scsa-ux6ce8ux91ca}}

以肺脏组织的数据集注释。

Figure \ref{fig:SCSA-annotation}为图SCSA annotation概览。

\textbf{(对应文件为 \texttt{Figure+Table/SCSA-annotation.pdf})}

\def\@captype{figure}
\begin{center}
\includegraphics[width = 0.9\linewidth]{Figure+Table/SCSA-annotation.pdf}
\caption{SCSA annotation}\label{fig:SCSA-annotation}
\end{center}

\hypertarget{ux4f9dux636eux53d8ux5f02ux62f7ux8d1dux6570ux9274ux5b9aux80bfux7624ux7ec6ux80de}{%
\subsection{依据变异拷贝数鉴定肿瘤细胞}\label{ux4f9dux636eux53d8ux5f02ux62f7ux8d1dux6570ux9274ux5b9aux80bfux7624ux7ec6ux80de}}

\hypertarget{copykat-ux89e3ux6790ux80bfux7624ux7ec6ux80de}{%
\subsubsection{copyKAT 解析肿瘤细胞}\label{copykat-ux89e3ux6790ux80bfux7624ux7ec6ux80de}}

copyKAT\textsuperscript{\protect\hyperlink{ref-DelineatingCopGaoR2021}{1}}

非整倍体是人类肿瘤细胞中最普遍的特征,约 90\% 的肿瘤的基因组是非整倍体,而正常细胞是二倍体\textsuperscript{\protect\hyperlink{ref-CausesAndConsGordon2012}{2}}

Figure \ref{fig:copyKAT-prediction-of-aneuploidy}为图copyKAT prediction of aneuploidy概览。

\textbf{(对应文件为 \texttt{Figure+Table/copykat\_heatmap.png})}

\def\@captype{figure}
\begin{center}
\includegraphics[width = 0.9\linewidth]{./Figure+Table/copykat_heatmap.png}
\caption{CopyKAT prediction of aneuploidy}\label{fig:copyKAT-prediction-of-aneuploidy}
\end{center}

Figure \ref{fig:cell-mapped-of-copyKAT-prediction}为图cell mapped of copyKAT prediction概览。

\textbf{(对应文件为 \texttt{Figure+Table/cell-mapped-of-copyKAT-prediction.pdf})}

\def\@captype{figure}
\begin{center}
\includegraphics[width = 0.9\linewidth]{Figure+Table/cell-mapped-of-copyKAT-prediction.pdf}
\caption{Cell mapped of copyKAT prediction}\label{fig:cell-mapped-of-copyKAT-prediction}
\end{center}

对比 Fig. \ref{fig:cell-mapped-of-copyKAT-prediction} 和 Fig. \ref{fig:SCSA-annotation} 可知,癌细胞主要为上皮细胞或基底细胞。

\hypertarget{ux80bfux7624ux7ec6ux80deux91cdux805aux7c7b}{%
\subsubsection{肿瘤细胞重聚类}\label{ux80bfux7624ux7ec6ux80deux91cdux805aux7c7b}}

为了区分肿瘤细胞之间的亚型,这里首先将肿瘤细胞重新聚类。

Figure \ref{fig:re-classify-of-cancer-cells}为图re classify of cancer cells概览。

\textbf{(对应文件为 \texttt{Figure+Table/re-classify-of-cancer-cells.pdf})}

\def\@captype{figure}
\begin{center}
\includegraphics[width = 0.9\linewidth]{Figure+Table/re-classify-of-cancer-cells.pdf}
\caption{Re classify of cancer cells}\label{fig:re-classify-of-cancer-cells}
\end{center}

\hypertarget{ux62dfux65f6ux5206ux6790ux80bfux7624ux7ec6ux80de}{%
\subsection{拟时分析肿瘤细胞}\label{ux62dfux65f6ux5206ux6790ux80bfux7624ux7ec6ux80de}}

\hypertarget{ux6784ux5efaux62dfux65f6ux8f68ux8ff9}{%
\subsubsection{构建拟时轨迹}\label{ux6784ux5efaux62dfux65f6ux8f68ux8ff9}}

Figure \ref{fig:pseudotime-visualization-of-cancer-cells}为图pseudotime visualization of cancer cells概览。

\textbf{(对应文件为 \texttt{Figure+Table/pseudotime-visualization-of-cancer-cells.pdf})}

\def\@captype{figure}
\begin{center}
\includegraphics[width = 0.9\linewidth]{Figure+Table/pseudotime-visualization-of-cancer-cells.pdf}
\caption{Pseudotime visualization of cancer cells}\label{fig:pseudotime-visualization-of-cancer-cells}
\end{center}

\hypertarget{ux6839ux636eux62dfux65f6ux5206ux6790ux533aux5206ux80bfux7624ux7ec6ux80deux4e9aux7c7b}{%
\subsubsection{根据拟时分析区分肿瘤细胞亚类}\label{ux6839ux636eux62dfux65f6ux5206ux6790ux533aux5206ux80bfux7624ux7ec6ux80deux4e9aux7c7b}}

Figure \ref{fig:gene-module-of-co-expression-analysis}为图gene module of co expression analysis概览。

\textbf{(对应文件为 \texttt{Figure+Table/gene-module-of-co-expression-analysis.pdf})}

\def\@captype{figure}
\begin{center}
\includegraphics[width = 0.9\linewidth]{Figure+Table/gene-module-of-co-expression-analysis.pdf}
\caption{Gene module of co expression analysis}\label{fig:gene-module-of-co-expression-analysis}
\end{center}

Fig. \ref{fig:gene-module-of-co-expression-analysis},根据细胞类的聚类树,可以将细胞类分为三个大类。

为了选择拟时起点,这里,将 Fig. \ref{fig:copyKAT-prediction-of-aneuploidy} 所示的细胞聚类树切分为 15 个大类的细胞,然后映射到
Fig. \ref{fig:pseudotime-visualization-of-cancer-cells} 对应的聚类图中。可以看到(Fig. \ref{fig:test-for-selecting-pseudotime-start-point}),细胞类类 11 首要分布在左侧区域(UMAP 图)(1 - 15,数目越小,代表越趋近于非整倍体,即肿瘤细胞),因此,左侧区域的细胞更接近于正常细胞。

Figure \ref{fig:test-for-selecting-pseudotime-start-point}为图test for selecting pseudotime start point概览。

\textbf{(对应文件为 \texttt{Figure+Table/test-for-selecting-pseudotime-start-point.pdf})}

\def\@captype{figure}
\begin{center}
\includegraphics[width = 0.9\linewidth]{Figure+Table/test-for-selecting-pseudotime-start-point.pdf}
\caption{Test for selecting pseudotime start point}\label{fig:test-for-selecting-pseudotime-start-point}
\end{center}

因此,选择左侧区域的细胞作为拟时起点。

Figure \ref{fig:cancer-cells-subtypes}为图cancer cells subtypes概览。

\textbf{(对应文件为 \texttt{Figure+Table/cancer-cells-subtypes.pdf})}

\def\@captype{figure}
\begin{center}
\includegraphics[width = 0.9\linewidth]{Figure+Table/cancer-cells-subtypes.pdf}
\caption{Cancer cells subtypes}\label{fig:cancer-cells-subtypes}
\end{center}

\hypertarget{en-cancer-sub}{%
\subsubsection{肿瘤细胞亚类差异分析和富集分析}\label{en-cancer-sub}}

以下分析,可关注的通路有:

\begin{itemize}
\tightlist
\item
  Phagosome
\item
  Antigen processing and presentati\ldots{}
\item
  Focal adhesion
\item
  \ldots{}
\end{itemize}

cancer 1 细胞的 marker 的富集分析有结果,但矫正 p 值均不显著。

Table \ref{tab:tables-of-enrichment-of-markers-of-cancer-1-cells}为表格tables of enrichment of markers of cancer 1 cells概览。

\textbf{(对应文件为 \texttt{Figure+Table/tables-of-enrichment-of-markers-of-cancer-1-cells.csv})}

\begin{center}\begin{tcolorbox}[colback=gray!10, colframe=gray!50, width=0.9\linewidth, arc=1mm, boxrule=0.5pt]注:表格共有66行9列,以下预览的表格可能省略部分数据;表格含有66个唯一`ID'。
\end{tcolorbox}
\end{center}

\begin{longtable}[]{@{}lllllllll@{}}
\caption{\label{tab:tables-of-enrichment-of-markers-of-cancer-1-cells}Tables of enrichment of markers of cancer 1 cells}\tabularnewline
\toprule
ID & Descr\ldots{} & GeneR\ldots{} & BgRatio & pvalue & p.adjust & qvalue & geneID & Count\tabularnewline
\midrule
\endfirsthead
\toprule
ID & Descr\ldots{} & GeneR\ldots{} & BgRatio & pvalue & p.adjust & qvalue & geneID & Count\tabularnewline
\midrule
\endhead
hsa04114 & Oocyt\ldots{} & 3/31 & 131/8622 & 0.011\ldots{} & 0.379\ldots{} & 0.369\ldots{} & 9133/\ldots{} & 3\tabularnewline
hsa04218 & Cellu\ldots{} & 3/31 & 156/8622 & 0.018\ldots{} & 0.379\ldots{} & 0.369\ldots{} & 9133/\ldots{} & 3\tabularnewline
hsa04110 & Cell \ldots{} & 3/31 & 157/8622 & 0.018\ldots{} & 0.379\ldots{} & 0.369\ldots{} & 9133/\ldots{} & 3\tabularnewline
hsa01524 & Plati\ldots{} & 2/31 & 73/8622 & 0.028\ldots{} & 0.379\ldots{} & 0.369\ldots{} & 4257/\ldots{} & 2\tabularnewline
hsa04115 & p53 s\ldots{} & 2/31 & 74/8622 & 0.028\ldots{} & 0.379\ldots{} & 0.369\ldots{} & 9133/983 & 2\tabularnewline
hsa04914 & Proge\ldots{} & 2/31 & 102/8622 & 0.051\ldots{} & 0.552\ldots{} & 0.537\ldots{} & 9133/983 & 2\tabularnewline
hsa00100 & Stero\ldots{} & 1/31 & 20/8622 & 0.069\ldots{} & 0.552\ldots{} & 0.537\ldots{} & 3930 & 1\tabularnewline
hsa03060 & Prote\ldots{} & 1/31 & 23/8622 & 0.079\ldots{} & 0.552\ldots{} & 0.537\ldots{} & 6726 & 1\tabularnewline
hsa04514 & Cell \ldots{} & 2/31 & 158/8622 & 0.109\ldots{} & 0.552\ldots{} & 0.537\ldots{} & 214/9076 & 2\tabularnewline
hsa04216 & Ferro\ldots{} & 1/31 & 41/8622 & 0.137\ldots{} & 0.552\ldots{} & 0.537\ldots{} & 7037 & 1\tabularnewline
hsa05014 & Amyot\ldots{} & 3/31 & 364/8622 & 0.141\ldots{} & 0.552\ldots{} & 0.537\ldots{} & 1345/\ldots{} & 3\tabularnewline
hsa02010 & ABC t\ldots{} & 1/31 & 45/8622 & 0.149\ldots{} & 0.552\ldots{} & 0.537\ldots{} & 154664 & 1\tabularnewline
hsa05130 & Patho\ldots{} & 2/31 & 198/8622 & 0.158\ldots{} & 0.552\ldots{} & 0.537\ldots{} & 9076/\ldots{} & 2\tabularnewline
hsa00270 & Cyste\ldots{} & 1/31 & 52/8622 & 0.171\ldots{} & 0.552\ldots{} & 0.537\ldots{} & 10768 & 1\tabularnewline
hsa05170 & Human\ldots{} & 2/31 & 212/8622 & 0.176\ldots{} & 0.552\ldots{} & 0.537\ldots{} & 9133/983 & 2\tabularnewline
\ldots{} & \ldots{} & \ldots{} & \ldots{} & \ldots{} & \ldots{} & \ldots{} & \ldots{} & \ldots{}\tabularnewline
\bottomrule
\end{longtable}

Figure \ref{fig:enrichment-of-markers-of-cancer-2-cells}为图enrichment of markers of cancer 2 cells概览。

\textbf{(对应文件为 \texttt{Figure+Table/enrichment-of-markers-of-cancer-2-cells.pdf})}

\def\@captype{figure}
\begin{center}
\includegraphics[width = 0.9\linewidth]{Figure+Table/enrichment-of-markers-of-cancer-2-cells.pdf}
\caption{Enrichment of markers of cancer 2 cells}\label{fig:enrichment-of-markers-of-cancer-2-cells}
\end{center}

Table \ref{tab:tables-of-enrichment-of-markers-of-cancer-2-cells}为表格tables of enrichment of markers of cancer 2 cells概览。

\textbf{(对应文件为 \texttt{Figure+Table/tables-of-enrichment-of-markers-of-cancer-2-cells.csv})}

\begin{center}\begin{tcolorbox}[colback=gray!10, colframe=gray!50, width=0.9\linewidth, arc=1mm, boxrule=0.5pt]注:表格共有162行9列,以下预览的表格可能省略部分数据;表格含有162个唯一`ID'。
\end{tcolorbox}
\end{center}

\begin{longtable}[]{@{}lllllllll@{}}
\caption{\label{tab:tables-of-enrichment-of-markers-of-cancer-2-cells}Tables of enrichment of markers of cancer 2 cells}\tabularnewline
\toprule
ID & Descr\ldots{} & GeneR\ldots{} & BgRatio & pvalue & p.adjust & qvalue & geneID & Count\tabularnewline
\midrule
\endfirsthead
\toprule
ID & Descr\ldots{} & GeneR\ldots{} & BgRatio & pvalue & p.adjust & qvalue & geneID & Count\tabularnewline
\midrule
\endhead
hsa04612 & Antig\ldots{} & 17/101 & 78/8622 & 1.279\ldots{} & 2.072\ldots{} & 1.561\ldots{} & 567/9\ldots{} & 17\tabularnewline
hsa04145 & Phago\ldots{} & 17/101 & 152/8622 & 1.400\ldots{} & 1.134\ldots{} & 8.550\ldots{} & 929/1\ldots{} & 17\tabularnewline
hsa05330 & Allog\ldots{} & 10/101 & 38/8622 & 1.113\ldots{} & 6.013\ldots{} & 4.532\ldots{} & 3002/\ldots{} & 10\tabularnewline
hsa05332 & Graft\ldots{} & 10/101 & 42/8622 & 3.334\ldots{} & 1.350\ldots{} & 1.018\ldots{} & 3002/\ldots{} & 10\tabularnewline
hsa04940 & Type \ldots{} & 10/101 & 43/8622 & 4.303\ldots{} & 1.394\ldots{} & 1.051\ldots{} & 3002/\ldots{} & 10\tabularnewline
hsa04512 & ECM-r\ldots{} & 12/101 & 89/8622 & 3.862\ldots{} & 9.196\ldots{} & 6.931\ldots{} & 961/1\ldots{} & 12\tabularnewline
hsa05320 & Autoi\ldots{} & 10/101 & 53/8622 & 3.973\ldots{} & 9.196\ldots{} & 6.931\ldots{} & 3002/\ldots{} & 10\tabularnewline
hsa05169 & Epste\ldots{} & 16/101 & 202/8622 & 1.262\ldots{} & 2.556\ldots{} & 1.927\ldots{} & 567/9\ldots{} & 16\tabularnewline
hsa05416 & Viral\ldots{} & 9/101 & 60/8622 & 2.610\ldots{} & 4.340\ldots{} & 3.271\ldots{} & 3105/\ldots{} & 9\tabularnewline
hsa04974 & Prote\ldots{} & 11/101 & 103/8622 & 2.679\ldots{} & 4.340\ldots{} & 3.271\ldots{} & 1306/\ldots{} & 11\tabularnewline
hsa05165 & Human\ldots{} & 16/101 & 331/8622 & 1.273\ldots{} & 1.744\ldots{} & 1.315\ldots{} & 1277/\ldots{} & 16\tabularnewline
hsa05310 & Asthma & 6/101 & 31/8622 & 1.292\ldots{} & 1.744\ldots{} & 1.315\ldots{} & 2207/\ldots{} & 6\tabularnewline
hsa04640 & Hemat\ldots{} & 9/101 & 99/8622 & 2.097\ldots{} & 2.455\ldots{} & 1.850\ldots{} & 929/9\ldots{} & 9\tabularnewline
hsa04514 & Cell \ldots{} & 11/101 & 158/8622 & 2.122\ldots{} & 2.455\ldots{} & 1.850\ldots{} & 914/3\ldots{} & 11\tabularnewline
hsa04510 & Focal\ldots{} & 12/101 & 203/8622 & 3.977\ldots{} & 4.296\ldots{} & 3.238\ldots{} & 1277/\ldots{} & 12\tabularnewline
\ldots{} & \ldots{} & \ldots{} & \ldots{} & \ldots{} & \ldots{} & \ldots{} & \ldots{} & \ldots{}\tabularnewline
\bottomrule
\end{longtable}

Figure \ref{fig:enrichment-of-markers-of-cancer-3-cells}为图enrichment of markers of cancer 3 cells概览。

\textbf{(对应文件为 \texttt{Figure+Table/enrichment-of-markers-of-cancer-3-cells.pdf})}

\def\@captype{figure}
\begin{center}
\includegraphics[width = 0.9\linewidth]{Figure+Table/enrichment-of-markers-of-cancer-3-cells.pdf}
\caption{Enrichment of markers of cancer 3 cells}\label{fig:enrichment-of-markers-of-cancer-3-cells}
\end{center}

Table \ref{tab:tables-of-enrichment-of-markers-of-cancer-3-cells}为表格tables of enrichment of markers of cancer 3 cells概览。

\textbf{(对应文件为 \texttt{Figure+Table/tables-of-enrichment-of-markers-of-cancer-3-cells.xlsx})}

\begin{center}\begin{tcolorbox}[colback=gray!10, colframe=gray!50, width=0.9\linewidth, arc=1mm, boxrule=0.5pt]注:表格共有241行9列,以下预览的表格可能省略部分数据;表格含有241个唯一`ID'。
\end{tcolorbox}
\end{center}

\begin{longtable}[]{@{}lllllllll@{}}
\caption{\label{tab:tables-of-enrichment-of-markers-of-cancer-3-cells}Tables of enrichment of markers of cancer 3 cells}\tabularnewline
\toprule
ID & Descr\ldots{} & GeneR\ldots{} & BgRatio & pvalue & p.adjust & qvalue & geneID & Count\tabularnewline
\midrule
\endfirsthead
\toprule
ID & Descr\ldots{} & GeneR\ldots{} & BgRatio & pvalue & p.adjust & qvalue & geneID & Count\tabularnewline
\midrule
\endhead
hsa04510 & Focal\ldots{} & 18/151 & 203/8622 & 1.346\ldots{} & 3.245\ldots{} & 2.707\ldots{} & 71/33\ldots{} & 18\tabularnewline
hsa00010 & Glyco\ldots{} & 10/151 & 67/8622 & 2.124\ldots{} & 2.559\ldots{} & 2.135\ldots{} & 226/2\ldots{} & 10\tabularnewline
hsa05412 & Arrhy\ldots{} & 10/151 & 77/8622 & 8.090\ldots{} & 6.499\ldots{} & 5.422\ldots{} & 71/18\ldots{} & 10\tabularnewline
hsa01230 & Biosy\ldots{} & 9/151 & 75/8622 & 5.726\ldots{} & 0.000\ldots{} & 0.000\ldots{} & 226/4\ldots{} & 9\tabularnewline
hsa04066 & HIF-1\ldots{} & 10/151 & 109/8622 & 1.948\ldots{} & 0.000\ldots{} & 0.000\ldots{} & 226/1\ldots{} & 10\tabularnewline
hsa05418 & Fluid\ldots{} & 11/151 & 139/8622 & 3.015\ldots{} & 0.001\ldots{} & 0.000\ldots{} & 71/44\ldots{} & 11\tabularnewline
hsa01200 & Carbo\ldots{} & 10/151 & 115/8622 & 3.114\ldots{} & 0.001\ldots{} & 0.000\ldots{} & 226/2\ldots{} & 10\tabularnewline
hsa05132 & Salmo\ldots{} & 14/151 & 249/8622 & 0.000\ldots{} & 0.003\ldots{} & 0.002\ldots{} & 10006\ldots{} & 14\tabularnewline
hsa04512 & ECM-r\ldots{} & 8/151 & 89/8622 & 0.000\ldots{} & 0.004\ldots{} & 0.003\ldots{} & 3339/\ldots{} & 8\tabularnewline
hsa05130 & Patho\ldots{} & 12/151 & 198/8622 & 0.000\ldots{} & 0.004\ldots{} & 0.003\ldots{} & 10006\ldots{} & 12\tabularnewline
hsa05222 & Small\ldots{} & 8/151 & 92/8622 & 0.000\ldots{} & 0.004\ldots{} & 0.003\ldots{} & 330/1\ldots{} & 8\tabularnewline
hsa04210 & Apopt\ldots{} & 9/151 & 136/8622 & 0.000\ldots{} & 0.012\ldots{} & 0.010\ldots{} & 71/33\ldots{} & 9\tabularnewline
hsa05205 & Prote\ldots{} & 11/151 & 205/8622 & 0.000\ldots{} & 0.017\ldots{} & 0.014\ldots{} & 71/85\ldots{} & 11\tabularnewline
hsa05165 & Human\ldots{} & 14/151 & 331/8622 & 0.001\ldots{} & 0.033\ldots{} & 0.027\ldots{} & 3133/\ldots{} & 14\tabularnewline
hsa05100 & Bacte\ldots{} & 6/151 & 77/8622 & 0.002\ldots{} & 0.035\ldots{} & 0.029\ldots{} & 71/10\ldots{} & 6\tabularnewline
\ldots{} & \ldots{} & \ldots{} & \ldots{} & \ldots{} & \ldots{} & \ldots{} & \ldots{} & \ldots{}\tabularnewline
\bottomrule
\end{longtable}

\hypertarget{ux80bfux7624ux7ec6ux80deux6765ux6e90ux5206ux6790}{%
\subsection{肿瘤细胞来源分析}\label{ux80bfux7624ux7ec6ux80deux6765ux6e90ux5206ux6790}}

\hypertarget{ux80bfux7624ux4e0eux4e0aux76aeux7ec6ux80deux6216ux57faux5e95ux7ec6ux80deux5deeux5f02ux5206ux6790}{%
\subsubsection{肿瘤与上皮细胞或基底细胞差异分析}\label{ux80bfux7624ux4e0eux4e0aux76aeux7ec6ux80deux6216ux57faux5e95ux7ec6ux80deux5deeux5f02ux5206ux6790}}

接下来的分析回到 Fig. \ref{fig:SCSA-annotation} 图中,取出上皮细胞或基底细胞对应的细胞类(肿瘤细胞主要分布在这两类细胞中),并且将 Fig. \ref{fig:cancer-cells-subtypes} 对应的细胞亚型映射。

Figure \ref{fig:cancer-cells-in-epithelial--or-basal-cells}为图cancer cells in epithelial or basal cells概览。

\textbf{(对应文件为 \texttt{Figure+Table/cancer-cells-in-epithelial-\/-or-basal-cells.pdf})}

\def\@captype{figure}
\begin{center}
\includegraphics[width = 0.9\linewidth]{Figure+Table/cancer-cells-in-epithelial--or-basal-cells.pdf}
\caption{Cancer cells in epithelial  or basal cells}\label{fig:cancer-cells-in-epithelial--or-basal-cells}
\end{center}

\hypertarget{en-diff}{%
\subsubsection{肿瘤与上皮细胞或基底细胞差异基因的富集分析}\label{en-diff}}

以下富集与 \ref{en-cancer-sub} 相对应,``Phagosome''、``Antigen processing and presentati\ldots{}'' 等为差异基因的主要富集通路。

Figure \ref{fig:enrichment-of-DEGs-of-Cancer-1-cells-vs-Basal-cells}为图enrichment of DEGs of Cancer 1 cells vs Basal cells概览。

\textbf{(对应文件为 \texttt{Figure+Table/enrichment-of-DEGs-of-Cancer-1-cells-vs-Basal-cells.pdf})}

\def\@captype{figure}
\begin{center}
\includegraphics[width = 0.9\linewidth]{Figure+Table/enrichment-of-DEGs-of-Cancer-1-cells-vs-Basal-cells.pdf}
\caption{Enrichment of DEGs of Cancer 1 cells vs Basal cells}\label{fig:enrichment-of-DEGs-of-Cancer-1-cells-vs-Basal-cells}
\end{center}

Figure \ref{fig:enrichment-of-DEGs-of-Cancer-1-cells-vs-Epithelial-cells}为图enrichment of DEGs of Cancer 1 cells vs Epithelial cells概览。

\textbf{(对应文件为 \texttt{Figure+Table/enrichment-of-DEGs-of-Cancer-1-cells-vs-Epithelial-cells.pdf})}

\def\@captype{figure}
\begin{center}
\includegraphics[width = 0.9\linewidth]{Figure+Table/enrichment-of-DEGs-of-Cancer-1-cells-vs-Epithelial-cells.pdf}
\caption{Enrichment of DEGs of Cancer 1 cells vs Epithelial cells}\label{fig:enrichment-of-DEGs-of-Cancer-1-cells-vs-Epithelial-cells}
\end{center}

`All enrichments of DEGs of cancer vs epithelial or basal cells' 数据已全部提供。

\textbf{(对应文件为 \texttt{all-enrichments-of-DEGs-of-cancer-vs-epithelial-or-basal-cells})}

\begin{center}\begin{tcolorbox}[colback=gray!10, colframe=gray!50, width=0.9\linewidth, arc=1mm, boxrule=0.5pt]注:文件夹all-enrichments-of-DEGs-of-cancer-vs-epithelial-or-basal-cells共包含6个文件。

\begin{enumerate}\tightlist
\item 1\_cancer\_1\_vs\_Basal cell.pdf
\item 2\_cancer\_1\_vs\_Epithelial cell.pdf
\item 3\_cancer\_2\_vs\_Basal cell.pdf
\item 4\_cancer\_2\_vs\_Epithelial cell.pdf
\item 5\_cancer\_3\_vs\_Basal cell.pdf
\item ...
\end{enumerate}\end{tcolorbox}
\end{center}

`Tables of all enrichments of DEGs of cancer vs epithelial or basal cells' 数据已全部提供。

\textbf{(对应文件为 \texttt{tables-of-all-enrichments-of-DEGs-of-cancer-vs-epithelial-or-basal-cells})}

\begin{center}\begin{tcolorbox}[colback=gray!10, colframe=gray!50, width=0.9\linewidth, arc=1mm, boxrule=0.5pt]注:文件夹tables-of-all-enrichments-of-DEGs-of-cancer-vs-epithelial-or-basal-cells共包含6个文件。

\begin{enumerate}\tightlist
\item 1\_cancer\_1\_vs\_Basal cell.csv
\item 2\_cancer\_1\_vs\_Epithelial cell.csv
\item 3\_cancer\_2\_vs\_Basal cell.csv
\item 4\_cancer\_2\_vs\_Epithelial cell.csv
\item 5\_cancer\_3\_vs\_Basal cell.csv
\item ...
\end{enumerate}\end{tcolorbox}
\end{center}

Table \ref{tab:tables-of-all-DEGs-of-cancer-vs-epithelial-or-basal-cells}为表格tables of all DEGs of cancer vs epithelial or basal cells概览。

\textbf{(对应文件为 \texttt{Figure+Table/tables-of-all-DEGs-of-cancer-vs-epithelial-or-basal-cells.csv})}

\begin{center}\begin{tcolorbox}[colback=gray!10, colframe=gray!50, width=0.9\linewidth, arc=1mm, boxrule=0.5pt]注:表格共有1502行7列,以下预览的表格可能省略部分数据;表格含有6个唯一`contrast'。
\end{tcolorbox}
\end{center}

\begin{longtable}[]{@{}lllllll@{}}
\caption{\label{tab:tables-of-all-DEGs-of-cancer-vs-epithelial-or-basal-cells}Tables of all DEGs of cancer vs epithelial or basal cells}\tabularnewline
\toprule
contrast & p\_val & avg\_l\ldots{} & pct.1 & pct.2 & p\_val\ldots{} & gene\tabularnewline
\midrule
\endfirsthead
\toprule
contrast & p\_val & avg\_l\ldots{} & pct.1 & pct.2 & p\_val\ldots{} & gene\tabularnewline
\midrule
\endhead
cance\ldots{} & 4.341\ldots{} & 0.369\ldots{} & 1 & 1 & 8.506\ldots{} & RPL9\tabularnewline
cance\ldots{} & 1.782\ldots{} & 0.404\ldots{} & 1 & 1 & 3.492\ldots{} & RPL7\tabularnewline
cance\ldots{} & 2.722\ldots{} & -0.97\ldots{} & 1 & 1 & 5.335\ldots{} & MT-ATP6\tabularnewline
cance\ldots{} & 5.164\ldots{} & -1.17\ldots{} & 1 & 1 & 1.011\ldots{} & MT-ND2\tabularnewline
cance\ldots{} & 5.763\ldots{} & -0.84\ldots{} & 1 & 1 & 1.129\ldots{} & MT-CO3\tabularnewline
cance\ldots{} & 7.030\ldots{} & -0.98\ldots{} & 1 & 1 & 1.377\ldots{} & MT-ND1\tabularnewline
cance\ldots{} & 7.923\ldots{} & -0.85\ldots{} & 1 & 1 & 1.552\ldots{} & MT-ND3\tabularnewline
cance\ldots{} & 6.299\ldots{} & -0.71\ldots{} & 1 & 1 & 1.234\ldots{} & MT-CO2\tabularnewline
cance\ldots{} & 6.455\ldots{} & -0.89\ldots{} & 1 & 1 & 1.264\ldots{} & MT-ND4\tabularnewline
cance\ldots{} & 1.075\ldots{} & -0.73\ldots{} & 1 & 1 & 2.106\ldots{} & MT-CO1\tabularnewline
cance\ldots{} & 2.747\ldots{} & -0.42\ldots{} & 1 & 1 & 5.383\ldots{} & HLA-B\tabularnewline
cance\ldots{} & 3.005\ldots{} & 0.327\ldots{} & 1 & 1 & 5.888\ldots{} & RPS15A\tabularnewline
cance\ldots{} & 5.993\ldots{} & 0.258\ldots{} & 1 & 1 & 1.174\ldots{} & RPL37A\tabularnewline
cance\ldots{} & 7.982\ldots{} & 0.289\ldots{} & 1 & 1 & 1.564\ldots{} & RPL17\tabularnewline
cance\ldots{} & 2.149\ldots{} & -0.78\ldots{} & 1 & 1 & 4.211\ldots{} & MT-CYB\tabularnewline
\ldots{} & \ldots{} & \ldots{} & \ldots{} & \ldots{} & \ldots{} & \ldots{}\tabularnewline
\bottomrule
\end{longtable}

\hypertarget{ux7ec6ux80deux901aux8baf}{%
\subsection{细胞通讯}\label{ux7ec6ux80deux901aux8baf}}

以下分析使用的为 Tab. \ref{tab:all-markers-of-Seurat-clusters} 中的基因。

\hypertarget{ux6240ux6709ux7ec6ux80deux4e4bux95f4ux7684ux901aux8baf}{%
\subsubsection{所有细胞之间的通讯}\label{ux6240ux6709ux7ec6ux80deux4e4bux95f4ux7684ux901aux8baf}}

将肿瘤细胞亚型映射到 Fig. \ref{fig:SCSA-annotation} 中,得到 Fig. \ref{fig:cancer-subtypes-in-all-cells}。

Figure \ref{fig:cancer-subtypes-in-all-cells}为图cancer subtypes in all cells概览。

\textbf{(对应文件为 \texttt{Figure+Table/cancer-subtypes-in-all-cells.pdf})}

\def\@captype{figure}
\begin{center}
\includegraphics[width = 0.9\linewidth]{Figure+Table/cancer-subtypes-in-all-cells.pdf}
\caption{Cancer subtypes in all cells}\label{fig:cancer-subtypes-in-all-cells}
\end{center}

以 cellchat 计算所有这些细胞之间的通讯关系\textsuperscript{\protect\hyperlink{ref-InferenceAndAJinS2021}{3}}。

Figure \ref{fig:overview-of-cells-communication}为图overview of cells communication概览。

\textbf{(对应文件为 \texttt{Figure+Table/overview-of-cells-communication.pdf})}

\def\@captype{figure}
\begin{center}
\includegraphics[width = 0.9\linewidth]{Figure+Table/overview-of-cells-communication.pdf}
\caption{Overview of cells communication}\label{fig:overview-of-cells-communication}
\end{center}

Figure \ref{fig:all-cells-communication-significance}为图all cells communication significance概览。

\textbf{(对应文件为 \texttt{Figure+Table/all-cells-communication-significance.pdf})}

\def\@captype{figure}
\begin{center}
\includegraphics[width = 0.9\linewidth]{Figure+Table/all-cells-communication-significance.pdf}
\caption{All cells communication significance}\label{fig:all-cells-communication-significance}
\end{center}

Figure \ref{fig:all-cells-communication-roles}为图all cells communication roles概览。

\textbf{(对应文件为 \texttt{Figure+Table/all-cells-communication-roles.pdf})}

\def\@captype{figure}
\begin{center}
\includegraphics[width = 0.9\linewidth]{Figure+Table/all-cells-communication-roles.pdf}
\caption{All cells communication roles}\label{fig:all-cells-communication-roles}
\end{center}

\hypertarget{ux5de8ux566cux7ec6ux80deux548cux80bfux7624ux7ec6ux80deux4e4bux95f4ux7684ux4e92ux4f5c}{%
\subsubsection{巨噬细胞和肿瘤细胞之间的互作}\label{ux5de8ux566cux7ec6ux80deux548cux80bfux7624ux7ec6ux80deux4e4bux95f4ux7684ux4e92ux4f5c}}

以下,我们主要聚焦于巨噬细胞和癌细胞之间的通讯关系。

Table \ref{tab:table-of-communication-between-macrophage-and-cancer-cells}为表格table of communication between macrophage and cancer cells概览。

\textbf{(对应文件为 \texttt{Figure+Table/table-of-communication-between-macrophage-and-cancer-cells.csv})}

\begin{center}\begin{tcolorbox}[colback=gray!10, colframe=gray!50, width=0.9\linewidth, arc=1mm, boxrule=0.5pt]注:表格共有409行11列,以下预览的表格可能省略部分数据;表格含有4个唯一`source'。
\end{tcolorbox}
\end{center}

\begin{longtable}[]{@{}llllllllllll@{}}
\caption{\label{tab:table-of-communication-between-macrophage-and-cancer-cells}Table of communication between macrophage and cancer cells}\tabularnewline
\toprule
source & target & ligand & receptor & prob & pval & inter\ldots\ldots7 & inter\ldots\ldots8 & pathw\ldots{} & annot\ldots{} & evidence & \ldots{}\tabularnewline
\midrule
\endfirsthead
\toprule
source & target & ligand & receptor & prob & pval & inter\ldots\ldots7 & inter\ldots\ldots8 & pathw\ldots{} & annot\ldots{} & evidence & \ldots{}\tabularnewline
\midrule
\endhead
Macro\ldots{} & cancer\_2 & TGFB1 & ACVR1\ldots{} & 0.000\ldots{} & 0.01 & TGFB1\ldots{} & TGFB1\ldots{} & TGFb & Secre\ldots{} & PMID:\ldots{} & \ldots{}\tabularnewline
Macro\ldots{} & cancer\_3 & TGFB1 & ACVR1\ldots{} & 0.000\ldots{} & 0.03 & TGFB1\ldots{} & TGFB1\ldots{} & TGFb & Secre\ldots{} & PMID:\ldots{} & \ldots{}\tabularnewline
cancer\_2 & Macro\ldots{} & TGFB1 & ACVR1\ldots{} & 0.001\ldots{} & 0 & TGFB1\ldots{} & TGFB1\ldots{} & TGFb & Secre\ldots{} & PMID:\ldots{} & \ldots{}\tabularnewline
cancer\_3 & Macro\ldots{} & TGFB1 & ACVR1\ldots{} & 0.000\ldots{} & 0 & TGFB1\ldots{} & TGFB1\ldots{} & TGFb & Secre\ldots{} & PMID:\ldots{} & \ldots{}\tabularnewline
Macro\ldots{} & cancer\_1 & TGFB1 & ACVR1\ldots{} & 0.000\ldots{} & 0.04 & TGFB1\ldots{} & TGFB1\ldots{} & TGFb & Secre\ldots{} & PMID:\ldots{} & \ldots{}\tabularnewline
Macro\ldots{} & cancer\_2 & TGFB1 & ACVR1\ldots{} & 0.000\ldots{} & 0 & TGFB1\ldots{} & TGFB1\ldots{} & TGFb & Secre\ldots{} & PMID:\ldots{} & \ldots{}\tabularnewline
Macro\ldots{} & cancer\_1 & WNT5A & FZD10 & 0.001\ldots{} & 0 & WNT5A\ldots{} & WNT5A\ldots{} & ncWNT & Secre\ldots{} & KEGG:\ldots{} & \ldots{}\tabularnewline
Macro\ldots{} & cancer\_2 & WNT5A & FZD10 & 0.001\ldots{} & 0.02 & WNT5A\ldots{} & WNT5A\ldots{} & ncWNT & Secre\ldots{} & KEGG:\ldots{} & \ldots{}\tabularnewline
Macro\ldots{} & cancer\_3 & WNT5A & FZD10 & 0.001\ldots{} & 0.02 & WNT5A\ldots{} & WNT5A\ldots{} & ncWNT & Secre\ldots{} & KEGG:\ldots{} & \ldots{}\tabularnewline
Macro\ldots{} & cancer\_1 & WNT5A & FZD6 & 0.011\ldots{} & 0.04 & WNT5A\ldots{} & WNT5A\ldots{} & ncWNT & Secre\ldots{} & KEGG:\ldots{} & \ldots{}\tabularnewline
Macro\ldots{} & cancer\_2 & WNT5A & FZD6 & 0.011\ldots{} & 0.04 & WNT5A\ldots{} & WNT5A\ldots{} & ncWNT & Secre\ldots{} & KEGG:\ldots{} & \ldots{}\tabularnewline
Macro\ldots{} & cancer\_2 & WNT5A & MCAM & 0.001\ldots{} & 0.04 & WNT5A\ldots{} & WNT5A\ldots{} & ncWNT & Secre\ldots{} & PMID:\ldots{} & \ldots{}\tabularnewline
cancer\_2 & Macro\ldots{} & WNT5A & MCAM & 0.005\ldots{} & 0 & WNT5A\ldots{} & WNT5A\ldots{} & ncWNT & Secre\ldots{} & PMID:\ldots{} & \ldots{}\tabularnewline
cancer\_3 & Macro\ldots{} & WNT5A & MCAM & 0.004\ldots{} & 0.04 & WNT5A\ldots{} & WNT5A\ldots{} & ncWNT & Secre\ldots{} & PMID:\ldots{} & \ldots{}\tabularnewline
cancer\_1 & Macro\ldots{} & PDGFB & PDGFRA & 0.000\ldots{} & 0 & PDGFB\ldots{} & PDGFB\ldots{} & PDGF & Secre\ldots{} & PMID:\ldots{} & \ldots{}\tabularnewline
\ldots{} & \ldots{} & \ldots{} & \ldots{} & \ldots{} & \ldots{} & \ldots{} & \ldots{} & \ldots{} & \ldots{} & \ldots{} & \ldots{}\tabularnewline
\bottomrule
\end{longtable}

Figure \ref{fig:visualization-of-communication-between-macrophage-and-cancer-cells}为图visualization of communication between macrophage and cancer cells概览。

\textbf{(对应文件为 \texttt{Figure+Table/visualization-of-communication-between-macrophage-and-cancer-cells.pdf})}

\def\@captype{figure}
\begin{center}
\includegraphics[width = 0.9\linewidth]{Figure+Table/visualization-of-communication-between-macrophage-and-cancer-cells.pdf}
\caption{Visualization of communication between macrophage and cancer cells}\label{fig:visualization-of-communication-between-macrophage-and-cancer-cells}
\end{center}

根据 Fig. \ref{fig:visualization-of-communication-between-macrophage-and-cancer-cells},
可以关注到互作网络图中中心度 (centrality degree) 较高的 ITGA 相关受体或配体。

\hypertarget{itga-ux76f8ux5173ux914dux4f53ux53d7ux4f53ux5bccux96c6ux5206ux6790}{%
\subsubsection{ITGA 相关配体受体富集分析}\label{itga-ux76f8ux5173ux914dux4f53ux53d7ux4f53ux5bccux96c6ux5206ux6790}}

根据 Fig. \ref{fig:visualization-of-communication-between-macrophage-and-cancer-cells},以下将 ITGA 相关基因做富集分析。

`Enrichment of ITGA related genes' 数据已全部提供。

\textbf{(对应文件为 \texttt{enrichment-of-ITGA-related-genes})}

\begin{center}\begin{tcolorbox}[colback=gray!10, colframe=gray!50, width=0.9\linewidth, arc=1mm, boxrule=0.5pt]注:文件夹enrichment-of-ITGA-related-genes共包含1个文件。

\begin{enumerate}\tightlist
\item 1\_ids.pdf
\end{enumerate}\end{tcolorbox}
\end{center}

Fig. \ref{fig:enrichment-of-ITGA-related-genes} 首要富集到 PI3K-AKT 通路。还可以发现,``Small cell lung cancer'' 也是显著富集结果之一。

Table \ref{tab:tables-of-enrichment-of-ITGA-related-genes}为表格tables of enrichment of ITGA related genes概览。

\textbf{(对应文件为 \texttt{Figure+Table/tables-of-enrichment-of-ITGA-related-genes.csv})}

\begin{center}\begin{tcolorbox}[colback=gray!10, colframe=gray!50, width=0.9\linewidth, arc=1mm, boxrule=0.5pt]注:表格共有65行9列,以下预览的表格可能省略部分数据;表格含有65个唯一`ID'。
\end{tcolorbox}
\end{center}

\begin{longtable}[]{@{}lllllllll@{}}
\caption{\label{tab:tables-of-enrichment-of-ITGA-related-genes}Tables of enrichment of ITGA related genes}\tabularnewline
\toprule
ID & Descr\ldots{} & GeneR\ldots{} & BgRatio & pvalue & p.adjust & qvalue & geneID & Count\tabularnewline
\midrule
\endfirsthead
\toprule
ID & Descr\ldots{} & GeneR\ldots{} & BgRatio & pvalue & p.adjust & qvalue & geneID & Count\tabularnewline
\midrule
\endhead
hsa04512 & ECM-r\ldots{} & 27/36 & 89/8622 & 2.645\ldots{} & 1.719\ldots{} & 8.633\ldots{} & 1277/\ldots{} & 27\tabularnewline
hsa04510 & Focal\ldots{} & 27/36 & 203/8622 & 1.474\ldots{} & 4.791\ldots{} & 2.405\ldots{} & 1277/\ldots{} & 27\tabularnewline
hsa05165 & Human\ldots{} & 27/36 & 331/8622 & 1.433\ldots{} & 3.106\ldots{} & 1.559\ldots{} & 1277/\ldots{} & 27\tabularnewline
hsa04151 & PI3K-\ldots{} & 27/36 & 359/8622 & 1.360\ldots{} & 2.211\ldots{} & 1.110\ldots{} & 1277/\ldots{} & 27\tabularnewline
hsa05146 & Amoeb\ldots{} & 18/36 & 102/8622 & 3.269\ldots{} & 4.250\ldots{} & 2.133\ldots{} & 1277/\ldots{} & 18\tabularnewline
hsa05222 & Small\ldots{} & 17/36 & 92/8622 & 4.653\ldots{} & 5.041\ldots{} & 2.530\ldots{} & 1282/\ldots{} & 17\tabularnewline
hsa04974 & Prote\ldots{} & 11/36 & 103/8622 & 1.922\ldots{} & 1.785\ldots{} & 8.962\ldots{} & 1277/\ldots{} & 11\tabularnewline
hsa05145 & Toxop\ldots{} & 10/36 & 111/8622 & 1.592\ldots{} & 1.294\ldots{} & 6.497\ldots{} & 3688/\ldots{} & 10\tabularnewline
hsa04933 & AGE-R\ldots{} & 9/36 & 100/8622 & 1.914\ldots{} & 1.382\ldots{} & 6.942\ldots{} & 1277/\ldots{} & 9\tabularnewline
hsa04810 & Regul\ldots{} & 9/36 & 229/8622 & 2.833\ldots{} & 1.841\ldots{} & 9.245\ldots{} & 2335/\ldots{} & 9\tabularnewline
hsa05412 & Arrhy\ldots{} & 6/36 & 77/8622 & 6.555\ldots{} & 3.873\ldots{} & 1.944\ldots{} & 3672/\ldots{} & 6\tabularnewline
hsa04926 & Relax\ldots{} & 7/36 & 129/8622 & 8.294\ldots{} & 4.492\ldots{} & 2.255\ldots{} & 1277/\ldots{} & 7\tabularnewline
hsa05410 & Hyper\ldots{} & 6/36 & 90/8622 & 1.655\ldots{} & 8.278\ldots{} & 4.156\ldots{} & 3672/\ldots{} & 6\tabularnewline
hsa05414 & Dilat\ldots{} & 6/36 & 96/8622 & 2.421\ldots{} & 1.124\ldots{} & 5.643\ldots{} & 3672/\ldots{} & 6\tabularnewline
hsa04670 & Leuko\ldots{} & 6/36 & 115/8622 & 6.942\ldots{} & 3.008\ldots{} & 1.510\ldots{} & 50848\ldots{} & 6\tabularnewline
\ldots{} & \ldots{} & \ldots{} & \ldots{} & \ldots{} & \ldots{} & \ldots{} & \ldots{} & \ldots{}\tabularnewline
\bottomrule
\end{longtable}

\hypertarget{ux9996ux8981ux5bccux96c6ux7684-pi3k-ux901aux8def}{%
\subsubsection{首要富集的 PI3K 通路}\label{ux9996ux8981ux5bccux96c6ux7684-pi3k-ux901aux8def}}

以下结果可以和 \ref{en-cancer-sub} 和 \ref{en-diff} 相对应。

Figure \ref{fig:view-of-enriched-genes-in-PI3K-pathway}为图view of enriched genes in PI3K pathway概览。

\textbf{(对应文件为 \texttt{Figure+Table/hsa04151.pathview.png})}

\def\@captype{figure}
\begin{center}
\includegraphics[width = 0.9\linewidth]{pathview2023-10-17_12_31_52.147711/hsa04151.pathview.png}
\caption{View of enriched genes in PI3K pathway}\label{fig:view-of-enriched-genes-in-PI3K-pathway}
\end{center}

\hypertarget{pi3k-ux901aux8defux548c-itga-ux76f8ux5173ux53d7ux4f53ux914dux4f53ux7684ux4ea4ux96c6}{%
\subsubsection{PI3K 通路和 ITGA 相关受体配体的交集}\label{pi3k-ux901aux8defux548c-itga-ux76f8ux5173ux53d7ux4f53ux914dux4f53ux7684ux4ea4ux96c6}}

\begin{center}\begin{tcolorbox}[colback=gray!10, colframe=gray!50, width=0.9\linewidth, arc=1mm, boxrule=0.5pt]
\textbf{
ITGA\_with\_PI3K
:}

\vspace{0.5em}

    COL1A1, COL1A2, COL2A1, COL4A1, COL4A2, COL4A4, COL4A5,
COL4A6, COL6A1, COL6A2, COL6A3, FN1, ITGA1, ITGA11, ITGA2,
ITGA5, ITGB1, LAMA1, LAMA3, LAMA4, LAMA5, LAMB1, LAMB2,
LAMB3, LAMC1, LAMC2, SPP1

\vspace{2em}
\end{tcolorbox}
\end{center}

\hypertarget{ux901aux8bafux57faux56e0ux7684ux8868ux8fbeux5728ux80bfux7624ux7ec6ux80deux4e2dux7684ux62dfux65f6ux53d8ux5316}{%
\subsubsection{通讯基因的表达在肿瘤细胞中的拟时变化}\label{ux901aux8bafux57faux56e0ux7684ux8868ux8fbeux5728ux80bfux7624ux7ec6ux80deux4e2dux7684ux62dfux65f6ux53d8ux5316}}

以下分析表现了 ITGA 和 PI3K 通路相关的 27 个基因在肿瘤细胞亚型之间的转化(拟时过程)过程中的表达变化。

Figure \ref{fig:group-1-communication-related-genes-in-pseudotime}为图group 1 communication related genes in pseudotime概览。

\textbf{(对应文件为 \texttt{Figure+Table/group-1-communication-related-genes-in-pseudotime.pdf})}

\def\@captype{figure}
\begin{center}
\includegraphics[width = 0.9\linewidth]{Figure+Table/group-1-communication-related-genes-in-pseudotime.pdf}
\caption{Group 1 communication related genes in pseudotime}\label{fig:group-1-communication-related-genes-in-pseudotime}
\end{center}

Figure \ref{fig:group-2-communication-related-genes-in-pseudotime}为图group 2 communication related genes in pseudotime概览。

\textbf{(对应文件为 \texttt{Figure+Table/group-2-communication-related-genes-in-pseudotime.pdf})}

\def\@captype{figure}
\begin{center}
\includegraphics[width = 0.9\linewidth]{Figure+Table/group-2-communication-related-genes-in-pseudotime.pdf}
\caption{Group 2 communication related genes in pseudotime}\label{fig:group-2-communication-related-genes-in-pseudotime}
\end{center}

Figure \ref{fig:group-3-communication-related-genes-in-pseudotime}为图group 3 communication related genes in pseudotime概览。

\textbf{(对应文件为 \texttt{Figure+Table/group-3-communication-related-genes-in-pseudotime.pdf})}

\def\@captype{figure}
\begin{center}
\includegraphics[width = 0.9\linewidth]{Figure+Table/group-3-communication-related-genes-in-pseudotime.pdf}
\caption{Group 3 communication related genes in pseudotime}\label{fig:group-3-communication-related-genes-in-pseudotime}
\end{center}

\hypertarget{workflow2}{%
\section{附:分析流程(癌旁组织切片)}\label{workflow2}}

\hypertarget{clustering-and-annotation}{%
\subsection{Clustering and annotation}\label{clustering-and-annotation}}

Figure \ref{fig:extra-QC}为图extra QC概览。

\textbf{(对应文件为 \texttt{Figure+Table/extra-QC.pdf})}

\def\@captype{figure}
\begin{center}
\includegraphics[width = 0.9\linewidth]{Figure+Table/extra-QC.pdf}
\caption{Extra QC}\label{fig:extra-QC}
\end{center}

Figure \ref{fig:extra-SCSA-annotation}为图extra SCSA annotation概览。

\textbf{(对应文件为 \texttt{Figure+Table/extra-SCSA-annotation.pdf})}

\def\@captype{figure}
\begin{center}
\includegraphics[width = 0.9\linewidth]{Figure+Table/extra-SCSA-annotation.pdf}
\caption{Extra SCSA annotation}\label{fig:extra-SCSA-annotation}
\end{center}

\hypertarget{copykat-prediction}{%
\subsection{Copykat prediction}\label{copykat-prediction}}

可以观察到,copyKAT 的细胞类型预测,癌旁组织 Fig. \ref{fig:extra-copyKAT-prediction} 的聚类不及 癌组织切片的 Fig. \ref{fig:copyKAT-prediction-of-aneuploidy} 明显。

Figure \ref{fig:extra-copyKAT-prediction}为图extra copyKAT prediction概览。

\textbf{(对应文件为 \texttt{Figure+Table/copykat\_para\_cancer.png})}

\def\@captype{figure}
\begin{center}
\includegraphics[width = 0.9\linewidth]{figs/copykat_para_cancer.png}
\caption{Extra copyKAT prediction}\label{fig:extra-copyKAT-prediction}
\end{center}

\hypertarget{bibliography}{%
\section*{Reference}\label{bibliography}}
\addcontentsline{toc}{section}{Reference}

\hypertarget{refs}{}
\begin{cslreferences}
\leavevmode\hypertarget{ref-DelineatingCopGaoR2021}{}%
1. Gao, R. \emph{et al.} Delineating copy number and clonal substructure in human tumors from single-cell transcriptomes. \emph{Nature Biotechnology} \textbf{39}, 599--608 (2021).

\leavevmode\hypertarget{ref-CausesAndConsGordon2012}{}%
2. Gordon, D. J., Resio, B. \& Pellman, D. Causes and consequences of aneuploidy in cancer. \emph{Nature Reviews Genetics} \textbf{13}, 189--203 (2012).

\leavevmode\hypertarget{ref-InferenceAndAJinS2021}{}%
3. Jin, S. \emph{et al.} Inference and analysis of cell-cell communication using cellchat. \emph{Nature Communications} \textbf{12}, (2021).
\end{cslreferences}

\end{document}
