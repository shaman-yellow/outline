% Options for packages loaded elsewhere
\PassOptionsToPackage{unicode}{hyperref}
\PassOptionsToPackage{hyphens}{url}
%
\documentclass[
]{article}
\usepackage{lmodern}
\usepackage{amssymb,amsmath}
\usepackage{ifxetex,ifluatex}
\ifnum 0\ifxetex 1\fi\ifluatex 1\fi=0 % if pdftex
  \usepackage[T1]{fontenc}
  \usepackage[utf8]{inputenc}
  \usepackage{textcomp} % provide euro and other symbols
\else % if luatex or xetex
  \usepackage{unicode-math}
  \defaultfontfeatures{Scale=MatchLowercase}
  \defaultfontfeatures[\rmfamily]{Ligatures=TeX,Scale=1}
\fi
% Use upquote if available, for straight quotes in verbatim environments
\IfFileExists{upquote.sty}{\usepackage{upquote}}{}
\IfFileExists{microtype.sty}{% use microtype if available
  \usepackage[]{microtype}
  \UseMicrotypeSet[protrusion]{basicmath} % disable protrusion for tt fonts
}{}
\makeatletter
\@ifundefined{KOMAClassName}{% if non-KOMA class
  \IfFileExists{parskip.sty}{%
    \usepackage{parskip}
  }{% else
    \setlength{\parindent}{0pt}
    \setlength{\parskip}{6pt plus 2pt minus 1pt}}
}{% if KOMA class
  \KOMAoptions{parskip=half}}
\makeatother
\usepackage{xcolor}
\IfFileExists{xurl.sty}{\usepackage{xurl}}{} % add URL line breaks if available
\IfFileExists{bookmark.sty}{\usepackage{bookmark}}{\usepackage{hyperref}}
\hypersetup{
  hidelinks,
  pdfcreator={LaTeX via pandoc}}
\urlstyle{same} % disable monospaced font for URLs
\usepackage[margin=1in]{geometry}
\usepackage{longtable,booktabs}
% Correct order of tables after \paragraph or \subparagraph
\usepackage{etoolbox}
\makeatletter
\patchcmd\longtable{\par}{\if@noskipsec\mbox{}\fi\par}{}{}
\makeatother
% Allow footnotes in longtable head/foot
\IfFileExists{footnotehyper.sty}{\usepackage{footnotehyper}}{\usepackage{footnote}}
\makesavenoteenv{longtable}
\usepackage{graphicx}
\makeatletter
\def\maxwidth{\ifdim\Gin@nat@width>\linewidth\linewidth\else\Gin@nat@width\fi}
\def\maxheight{\ifdim\Gin@nat@height>\textheight\textheight\else\Gin@nat@height\fi}
\makeatother
% Scale images if necessary, so that they will not overflow the page
% margins by default, and it is still possible to overwrite the defaults
% using explicit options in \includegraphics[width, height, ...]{}
\setkeys{Gin}{width=\maxwidth,height=\maxheight,keepaspectratio}
% Set default figure placement to htbp
\makeatletter
\def\fps@figure{htbp}
\makeatother
\setlength{\emergencystretch}{3em} % prevent overfull lines
\providecommand{\tightlist}{%
  \setlength{\itemsep}{0pt}\setlength{\parskip}{0pt}}
\setcounter{secnumdepth}{5}
\usepackage{caption} \captionsetup{font={footnotesize},width=6in} \renewcommand{\dblfloatpagefraction}{.9} \makeatletter \renewenvironment{figure} {\def\@captype{figure}} \makeatother \@ifundefined{Shaded}{\newenvironment{Shaded}} \@ifundefined{snugshade}{\newenvironment{snugshade}} \renewenvironment{Shaded}{\begin{snugshade}}{\end{snugshade}} \definecolor{shadecolor}{RGB}{230,230,230} \usepackage{xeCJK} \usepackage{setspace} \setstretch{1.3} \usepackage{tcolorbox} \setcounter{secnumdepth}{4} \setcounter{tocdepth}{4} \usepackage{wallpaper} \usepackage[absolute]{textpos} \tcbuselibrary{breakable} \renewenvironment{Shaded} {\begin{tcolorbox}[colback = gray!10, colframe = gray!40, width = 16cm, arc = 1mm, auto outer arc, title = {R input}]} {\end{tcolorbox}} \usepackage{titlesec} \titleformat{\paragraph} {\fontsize{10pt}{0pt}\bfseries} {\arabic{section}.\arabic{subsection}.\arabic{subsubsection}.\arabic{paragraph}} {1em} {} []
\newlength{\cslhangindent}
\setlength{\cslhangindent}{1.5em}
\newenvironment{cslreferences}%
  {}%
  {\par}

\author{}
\date{\vspace{-2.5em}}

\begin{document}

\begin{titlepage} \newgeometry{top=7.5cm}
\ThisCenterWallPaper{1.12}{~/outline/lixiao//cover_page.pdf}
\begin{center} \textbf{\Huge
中药-有效成分-乳腺癌相关靶点的网药分析}
\vspace{4em} \begin{textblock}{10}(3,5.9) \huge
\textbf{\textcolor{white}{2024-04-17}}
\end{textblock} \begin{textblock}{10}(3,7.3)
\Large \textcolor{black}{LiChuang Huang}
\end{textblock} \begin{textblock}{10}(3,11.3)
\Large \textcolor{black}{@立效研究院}
\end{textblock} \end{center} \end{titlepage}
\restoregeometry

\pagenumbering{roman}

\tableofcontents

\listoffigures

\listoftables

\newpage

\pagenumbering{arabic}

\hypertarget{abstract}{%
\section{摘要}\label{abstract}}

\hypertarget{ux9700ux6c42}{%
\subsection{需求}\label{ux9700ux6c42}}

网络药理学分析

\begin{itemize}
\tightlist
\item
  药对:白花蛇舌草,半枝莲,浙贝母
\item
  疾病:乳腺癌
\item
  目标:提供中药-有效成分-乳腺癌相关靶点的网药分析
\end{itemize}

\hypertarget{ux7ed3ux679c}{%
\subsection{结果}\label{ux7ed3ux679c}}

\begin{itemize}
\tightlist
\item
  数据来源于 TCMSP,以 OB、DL 筛选过化合物 Tab. \ref{tab:Compounds-filtered-by-OB-and-DL}。
\item
  疾病靶点来源于 GeneCards, Tab. \ref{tab:Disease-related-targets-from-GeneCards}
\item
  疾病成分靶点网络图:Fig. \ref{fig:Targets-intersect-with-targets-of-diseases}
\item
  包含通路:Fig. \ref{fig:Network-pharmacology-with-disease-and-pathway},
  Tab. \ref{tab:Network-pharmacology-with-disease-and-pathway-data}
\end{itemize}

\hypertarget{introduction}{%
\section{前言}\label{introduction}}

\hypertarget{methods}{%
\section{材料和方法}\label{methods}}

\hypertarget{ux6750ux6599}{%
\subsection{材料}\label{ux6750ux6599}}

\hypertarget{ux65b9ux6cd5}{%
\subsection{方法}\label{ux65b9ux6cd5}}

Mainly used method:

\begin{itemize}
\tightlist
\item
  R package \texttt{ClusterProfiler} used for gene enrichment analysis\textsuperscript{\protect\hyperlink{ref-ClusterprofilerWuTi2021}{1}}.
\item
  The Human Gene Database \texttt{GeneCards} used for disease related genes prediction\textsuperscript{\protect\hyperlink{ref-TheGenecardsSStelze2016}{2}}.
\item
  Website \texttt{TCMSP} \url{https://tcmsp-e.com/} used for data source\textsuperscript{\protect\hyperlink{ref-TcmspADatabaRuJi2014}{3}}.
\item
  The API of \texttt{UniProtKB} (\url{https://www.uniprot.org/help/api_queries}) used for mapping of names or IDs of proteins.
\item
  R version 4.3.2 (2023-10-31); Other R packages (eg., \texttt{dplyr} and \texttt{ggplot2}) used for statistic analysis or data visualization.
\end{itemize}

\hypertarget{results}{%
\section{分析结果}\label{results}}

\hypertarget{dis}{%
\section{结论}\label{dis}}

\hypertarget{workflow}{%
\section{附:分析流程}\label{workflow}}

\hypertarget{ux7f51ux7edcux836fux7406ux5b66}{%
\subsection{网络药理学}\label{ux7f51ux7edcux836fux7406ux5b66}}

\hypertarget{ux6210ux5206}{%
\subsubsection{成分}\label{ux6210ux5206}}

Table \ref{tab:Herbs-information} (下方表格) 为表格Herbs information概览。

\textbf{(对应文件为 \texttt{Figure+Table/Herbs-information.csv})}

\begin{center}\begin{tcolorbox}[colback=gray!10, colframe=gray!50, width=0.9\linewidth, arc=1mm, boxrule=0.5pt]注:表格共有3行2列,以下预览的表格可能省略部分数据;含有3个唯一`Herb\_pinyin\_name'。
\end{tcolorbox}
\end{center}

\begin{longtable}[]{@{}ll@{}}
\caption{\label{tab:Herbs-information}Herbs information}\tabularnewline
\toprule
Herb\_pinyin\_name & Herb\_cn\_name\tabularnewline
\midrule
\endfirsthead
\toprule
Herb\_pinyin\_name & Herb\_cn\_name\tabularnewline
\midrule
\endhead
Baihuasheshecao & 白花蛇舌草\tabularnewline
Banzhilian & 半枝莲\tabularnewline
Zhebeimu & 浙贝母\tabularnewline
\bottomrule
\end{longtable}

Table \ref{tab:Compounds-filtered-by-OB-and-DL} (下方表格) 为表格Compounds filtered by OB and DL概览。

\textbf{(对应文件为 \texttt{Figure+Table/Compounds-filtered-by-OB-and-DL.xlsx})}

\begin{center}\begin{tcolorbox}[colback=gray!10, colframe=gray!50, width=0.9\linewidth, arc=1mm, boxrule=0.5pt]注:表格共有43行15列,以下预览的表格可能省略部分数据;含有39个唯一`Mol ID;含有3个唯一`Herb\_pinyin\_name'。
\end{tcolorbox}
\end{center}\begin{center}\begin{tcolorbox}[colback=gray!10, colframe=gray!50, width=0.9\linewidth, arc=1mm, boxrule=0.5pt]
\textbf{
OB (\%) and DL cut-off
:}

\vspace{0.5em}

    OB >= 30\%; DL >= 0.18

\vspace{2em}
\end{tcolorbox}
\end{center}

\begin{longtable}[]{@{}llllllllll@{}}
\caption{\label{tab:Compounds-filtered-by-OB-and-DL}Compounds filtered by OB and DL}\tabularnewline
\toprule
Mol ID & Molecu\ldots{} & MW & AlogP & Hdon & Hacc & OB (\%) & Caco-2 & BBB & DL\tabularnewline
\midrule
\endfirsthead
\toprule
Mol ID & Molecu\ldots{} & MW & AlogP & Hdon & Hacc & OB (\%) & Caco-2 & BBB & DL\tabularnewline
\midrule
\endhead
MOL001646 & 2,3-di\ldots{} & 282.310 & 3.262 & 0 & 4 & 34.858\ldots{} & 0.75128 & 0.17357 & 0.26255\tabularnewline
MOL001659 & Porife\ldots{} & 412.770 & 7.640 & 1 & 1 & 43.829\ldots{} & 1.43659 & 1.03472 & 0.75596\tabularnewline
MOL001663 & (4aS,6\ldots{} & 456.780 & 6.422 & 2 & 3 & 32.028\ldots{} & 0.60932 & 0.39268 & 0.75713\tabularnewline
MOL001670 & 2-meth\ldots{} & 252.280 & 3.278 & 0 & 3 & 37.827\ldots{} & 0.72896 & -0.12795 & 0.20517\tabularnewline
MOL000449 & Stigma\ldots{} & 412.770 & 7.640 & 1 & 1 & 43.829\ldots{} & 1.44458 & 1.00045 & 0.75665\tabularnewline
MOL000358 & beta-s\ldots{} & 414.790 & 8.084 & 1 & 1 & 36.913\ldots{} & 1.32463 & 0.98588 & 0.75123\tabularnewline
MOL000098 & quercetin & 302.250 & 1.504 & 5 & 7 & 46.433\ldots{} & 0.04842 & -0.76890 & 0.27525\tabularnewline
MOL001040 & (2R)-5\ldots{} & 272.270 & 2.298 & 3 & 5 & 42.363\ldots{} & 0.37818 & -0.47578 & 0.21141\tabularnewline
MOL012245 & 5,7,4'\ldots{} & 302.300 & 2.281 & 3 & 6 & 36.626\ldots{} & 0.43274 & -0.31890 & 0.26833\tabularnewline
MOL012246 & 5,7,4'\ldots{} & 302.300 & 2.281 & 3 & 6 & 74.235\ldots{} & 0.37328 & -0.43273 & 0.26479\tabularnewline
MOL012248 & 5-hydr\ldots{} & 328.340 & 2.820 & 1 & 6 & 65.818\ldots{} & 0.84750 & 0.07437 & 0.32874\tabularnewline
MOL012250 & 7-hydr\ldots{} & 298.310 & 2.836 & 1 & 5 & 43.716\ldots{} & 0.95759 & 0.22129 & 0.25376\tabularnewline
MOL012251 & Chrysi\ldots{} & 268.280 & 2.853 & 1 & 4 & 37.268\ldots{} & 0.90922 & 0.15556 & 0.20317\tabularnewline
MOL012252 & 9,19-c\ldots{} & 426.800 & 7.554 & 1 & 1 & 38.685\ldots{} & 1.44891 & 1.16360 & 0.78074\tabularnewline
MOL002776 & Baicalin & 446.390 & 0.639 & 6 & 11 & 40.123\ldots{} & -0.84777 & -1.74426 & 0.75264\tabularnewline
\ldots{} & \ldots{} & \ldots{} & \ldots{} & \ldots{} & \ldots{} & \ldots{} & \ldots{} & \ldots{} & \ldots{}\tabularnewline
\bottomrule
\end{longtable}

Figure \ref{fig:intersection-of-all-compounds} (下方图) 为图intersection of all compounds概览。

\textbf{(对应文件为 \texttt{Figure+Table/intersection-of-all-compounds.pdf})}

\def\@captype{figure}
\begin{center}
\includegraphics[width = 0.9\linewidth]{Figure+Table/intersection-of-all-compounds.pdf}
\caption{Intersection of all compounds}\label{fig:intersection-of-all-compounds}
\end{center}
\begin{center}\begin{tcolorbox}[colback=gray!10, colframe=gray!50, width=0.9\linewidth, arc=1mm, boxrule=0.5pt]
\textbf{
All\_intersection
:}

\vspace{0.5em}

    MOL000358

\vspace{2em}
\end{tcolorbox}
\end{center}

\textbf{(上述信息框内容已保存至 \texttt{Figure+Table/intersection-of-all-compounds-content})}

\hypertarget{ux6210ux5206ux9776ux70b9}{%
\subsubsection{成分靶点}\label{ux6210ux5206ux9776ux70b9}}

Table \ref{tab:tables-of-Herbs-compounds-and-targets} (下方表格) 为表格tables of Herbs compounds and targets概览。

\textbf{(对应文件为 \texttt{Figure+Table/tables-of-Herbs-compounds-and-targets.xlsx})}

\begin{center}\begin{tcolorbox}[colback=gray!10, colframe=gray!50, width=0.9\linewidth, arc=1mm, boxrule=0.5pt]注:表格共有1846行4列,以下预览的表格可能省略部分数据;含有3个唯一`Herb\_pinyin\_name'。
\end{tcolorbox}
\end{center}

\begin{longtable}[]{@{}llll@{}}
\caption{\label{tab:tables-of-Herbs-compounds-and-targets}Tables of Herbs compounds and targets}\tabularnewline
\toprule
Herb\_pinyin\_name & Molecule name & symbols & protein.names\tabularnewline
\midrule
\endfirsthead
\toprule
Herb\_pinyin\_name & Molecule name & symbols & protein.names\tabularnewline
\midrule
\endhead
Banzhilian & luteolin & NA & NA\tabularnewline
Banzhilian & luteolin & MMP2 & 72 kDa type IV co\ldots{}\tabularnewline
Banzhilian & luteolin & CLG4A & 72 kDa type IV co\ldots{}\tabularnewline
Banzhilian & luteolin & ADCY2 & Adenylate cyclase\ldots{}\tabularnewline
Banzhilian & luteolin & KIAA1060 & Adenylate cyclase\ldots{}\tabularnewline
Banzhilian & luteolin & APP & Amyloid-beta prec\ldots{}\tabularnewline
Banzhilian & luteolin & A4 & Amyloid-beta prec\ldots{}\tabularnewline
Banzhilian & luteolin & AD1 & Amyloid-beta prec\ldots{}\tabularnewline
Banzhilian & luteolin & AR & Androgen receptor\ldots{}\tabularnewline
Banzhilian & luteolin & DHTR & Androgen receptor\ldots{}\tabularnewline
Banzhilian & luteolin & NR3C4 & Androgen receptor\ldots{}\tabularnewline
Banzhilian & luteolin & XIAP & E3 ubiquitin-prot\ldots{}\tabularnewline
Banzhilian & luteolin & API3 & E3 ubiquitin-prot\ldots{}\tabularnewline
Banzhilian & luteolin & BIRC4 & E3 ubiquitin-prot\ldots{}\tabularnewline
Banzhilian & luteolin & IAP3 & E3 ubiquitin-prot\ldots{}\tabularnewline
\ldots{} & \ldots{} & \ldots{} & \ldots{}\tabularnewline
\bottomrule
\end{longtable}

\hypertarget{ux75beux75c5ux9776ux70b9}{%
\subsubsection{疾病靶点}\label{ux75beux75c5ux9776ux70b9}}

Table \ref{tab:Disease-related-targets-from-GeneCards} (下方表格) 为表格Disease related targets from GeneCards概览。

\textbf{(对应文件为 \texttt{Figure+Table/Disease-related-targets-from-GeneCards.xlsx})}

\begin{center}\begin{tcolorbox}[colback=gray!10, colframe=gray!50, width=0.9\linewidth, arc=1mm, boxrule=0.5pt]注:表格共有1746行7列,以下预览的表格可能省略部分数据;含有1746个唯一`Symbol'。
\end{tcolorbox}
\end{center}\begin{center}\begin{tcolorbox}[colback=gray!10, colframe=gray!50, width=0.9\linewidth, arc=1mm, boxrule=0.5pt]
\textbf{
The GeneCards data was obtained by querying
:}

\vspace{0.5em}

    breast cancer

\vspace{2em}


\textbf{
Restrict (with quotes)
:}

\vspace{0.5em}

    FALSE

\vspace{2em}


\textbf{
Filtering by Score:
:}

\vspace{0.5em}

    Score > 15

\vspace{2em}
\end{tcolorbox}
\end{center}

\begin{longtable}[]{@{}lllllll@{}}
\caption{\label{tab:Disease-related-targets-from-GeneCards}Disease related targets from GeneCards}\tabularnewline
\toprule
Symbol & Description & Category & UniProt\_ID & GIFtS & GC\_id & Score\tabularnewline
\midrule
\endfirsthead
\toprule
Symbol & Description & Category & UniProt\_ID & GIFtS & GC\_id & Score\tabularnewline
\midrule
\endhead
BRCA2 & BRCA2 DNA \ldots{} & Protein Co\ldots{} & P51587 & 56 & GC13P032315 & 584.27\tabularnewline
BRCA1 & BRCA1 DNA \ldots{} & Protein Co\ldots{} & P38398 & 59 & GC17M043044 & 565.02\tabularnewline
PALB2 & Partner An\ldots{} & Protein Co\ldots{} & Q86YC2 & 53 & GC16M023603 & 366.84\tabularnewline
ATM & ATM Serine\ldots{} & Protein Co\ldots{} & Q13315 & 61 & GC11P108223 & 340.7\tabularnewline
CHEK2 & Checkpoint\ldots{} & Protein Co\ldots{} & O96017 & 63 & GC22M028687 & 336.43\tabularnewline
BRIP1 & BRCA1 Inte\ldots{} & Protein Co\ldots{} & Q9BX63 & 57 & GC17M061679 & 325.07\tabularnewline
CDH1 & Cadherin 1 & Protein Co\ldots{} & P12830 & 58 & GC16P068737 & 306.68\tabularnewline
BARD1 & BRCA1 Asso\ldots{} & Protein Co\ldots{} & Q99728 & 55 & GC02M214725 & 291.41\tabularnewline
TP53 & Tumor Prot\ldots{} & Protein Co\ldots{} & P04637 & 62 & GC17M007661 & 287.34\tabularnewline
MSH6 & MutS Homol\ldots{} & Protein Co\ldots{} & P52701 & 58 & GC02P047695 & 239.29\tabularnewline
MSH2 & MutS Homol\ldots{} & Protein Co\ldots{} & P43246 & 57 & GC02P047403 & 231.87\tabularnewline
MLH1 & MutL Homol\ldots{} & Protein Co\ldots{} & P40692 & 58 & GC03P036993 & 223.25\tabularnewline
C11orf65 & Chromosome\ldots{} & Protein Co\ldots{} & Q8NCR3 & 40 & GC11M108308 & 218.43\tabularnewline
LOC126862571 & BRD4-Indep\ldots{} & Functional\ldots{} & & 10 & GC17P114574 & 215.91\tabularnewline
APC & APC Regula\ldots{} & Protein Co\ldots{} & P25054 & 58 & GC05P112707 & 199.23\tabularnewline
\ldots{} & \ldots{} & \ldots{} & \ldots{} & \ldots{} & \ldots{} & \ldots{}\tabularnewline
\bottomrule
\end{longtable}

\hypertarget{ux75beux75c5-ux6210ux5206-ux9776ux70b9ux7f51ux7edcux56fe}{%
\subsubsection{疾病-成分-靶点网络图}\label{ux75beux75c5-ux6210ux5206-ux9776ux70b9ux7f51ux7edcux56fe}}

Figure \ref{fig:Network-pharmacology-with-disease} (下方图) 为图Network pharmacology with disease概览。

\textbf{(对应文件为 \texttt{Figure+Table/Network-pharmacology-with-disease.pdf})}

\def\@captype{figure}
\begin{center}
\includegraphics[width = 0.9\linewidth]{Figure+Table/Network-pharmacology-with-disease.pdf}
\caption{Network pharmacology with disease}\label{fig:Network-pharmacology-with-disease}
\end{center}

Figure \ref{fig:Targets-intersect-with-targets-of-diseases} (下方图) 为图Targets intersect with targets of diseases概览。

\textbf{(对应文件为 \texttt{Figure+Table/Targets-intersect-with-targets-of-diseases.pdf})}

\def\@captype{figure}
\begin{center}
\includegraphics[width = 0.9\linewidth]{Figure+Table/Targets-intersect-with-targets-of-diseases.pdf}
\caption{Targets intersect with targets of diseases}\label{fig:Targets-intersect-with-targets-of-diseases}
\end{center}
\begin{center}\begin{tcolorbox}[colback=gray!10, colframe=gray!50, width=0.9\linewidth, arc=1mm, boxrule=0.5pt]
\textbf{
Intersection
:}

\vspace{0.5em}

    CHEK2, TP53, PTEN, ERBB2, CDKN2A, AKT1, AR, CASP8,
ERBB3, JUN, MYC, IL2, MDM2, CDK2, IL1B, FGFR4, BCL2, BAX,
TGFB1, ESR2, IGF2, NFE2L2, PPARG, EGF, PTGS2, TNF, MMP2,
MMP9, RAF1, CASP3, CYP1A1, NFKB1, CTSD, PCNA, PLAU, TOP2A,
CDK1, MMP1, E2F1, VEGFC, IFNG, CYP1B1, CHEK1, PIK3CG, IL10,
CASP9, CAV1,...

\vspace{2em}
\end{tcolorbox}
\end{center}

\textbf{(上述信息框内容已保存至 \texttt{Figure+Table/Targets-intersect-with-targets-of-diseases-content})}

\hypertarget{ux5bccux96c6ux5206ux6790}{%
\subsubsection{富集分析}\label{ux5bccux96c6ux5206ux6790}}

Figure \ref{fig:KEGG-enrichment} (下方图) 为图KEGG enrichment概览。

\textbf{(对应文件为 \texttt{Figure+Table/KEGG-enrichment.pdf})}

\def\@captype{figure}
\begin{center}
\includegraphics[width = 0.9\linewidth]{Figure+Table/KEGG-enrichment.pdf}
\caption{KEGG enrichment}\label{fig:KEGG-enrichment}
\end{center}

Figure \ref{fig:GO-enrichment} (下方图) 为图GO enrichment概览。

\textbf{(对应文件为 \texttt{Figure+Table/GO-enrichment.pdf})}

\def\@captype{figure}
\begin{center}
\includegraphics[width = 0.9\linewidth]{Figure+Table/GO-enrichment.pdf}
\caption{GO enrichment}\label{fig:GO-enrichment}
\end{center}

\hypertarget{ux75beux75c5-ux6210ux5206-ux9776ux70b9-ux901aux8defux7f51ux7edcux56fe}{%
\subsubsection{疾病-成分-靶点-通路网络图}\label{ux75beux75c5-ux6210ux5206-ux9776ux70b9-ux901aux8defux7f51ux7edcux56fe}}

Figure \ref{fig:Network-pharmacology-with-disease-and-pathway} (下方图) 为图Network pharmacology with disease and pathway概览。

\textbf{(对应文件为 \texttt{Figure+Table/Network-pharmacology-with-disease-and-pathway.pdf})}

\def\@captype{figure}
\begin{center}
\includegraphics[width = 0.9\linewidth]{Figure+Table/Network-pharmacology-with-disease-and-pathway.pdf}
\caption{Network pharmacology with disease and pathway}\label{fig:Network-pharmacology-with-disease-and-pathway}
\end{center}

Table \ref{tab:Network-pharmacology-with-disease-and-pathway-data} (下方表格) 为表格Network pharmacology with disease and pathway data概览。

\textbf{(对应文件为 \texttt{Figure+Table/Network-pharmacology-with-disease-and-pathway-data.xlsx})}

\begin{center}\begin{tcolorbox}[colback=gray!10, colframe=gray!50, width=0.9\linewidth, arc=1mm, boxrule=0.5pt]注:表格共有431行5列,以下预览的表格可能省略部分数据;含有3个唯一`Herb\_pinyin\_name;含有24个唯一`Ingredient.name;含有101个唯一`Target.name'。
\end{tcolorbox}
\end{center}

\begin{longtable}[]{@{}lllll@{}}
\caption{\label{tab:Network-pharmacology-with-disease-and-pathway-data}Network pharmacology with disease and pathway data}\tabularnewline
\toprule
\begin{minipage}[b]{0.17\columnwidth}\raggedright
Herb\_pinyin\_name\strut
\end{minipage} & \begin{minipage}[b]{0.16\columnwidth}\raggedright
Ingredient.name\strut
\end{minipage} & \begin{minipage}[b]{0.12\columnwidth}\raggedright
Target.name\strut
\end{minipage} & \begin{minipage}[b]{0.19\columnwidth}\raggedright
Hit\_pathway\_number\strut
\end{minipage} & \begin{minipage}[b]{0.21\columnwidth}\raggedright
Enriched\_pathways\strut
\end{minipage}\tabularnewline
\midrule
\endfirsthead
\toprule
\begin{minipage}[b]{0.17\columnwidth}\raggedright
Herb\_pinyin\_name\strut
\end{minipage} & \begin{minipage}[b]{0.16\columnwidth}\raggedright
Ingredient.name\strut
\end{minipage} & \begin{minipage}[b]{0.12\columnwidth}\raggedright
Target.name\strut
\end{minipage} & \begin{minipage}[b]{0.19\columnwidth}\raggedright
Hit\_pathway\_number\strut
\end{minipage} & \begin{minipage}[b]{0.21\columnwidth}\raggedright
Enriched\_pathways\strut
\end{minipage}\tabularnewline
\midrule
\endhead
\begin{minipage}[t]{0.17\columnwidth}\raggedright
Baihuasheshecao\strut
\end{minipage} & \begin{minipage}[t]{0.16\columnwidth}\raggedright
quercetin\strut
\end{minipage} & \begin{minipage}[t]{0.12\columnwidth}\raggedright
NFKB1\strut
\end{minipage} & \begin{minipage}[t]{0.19\columnwidth}\raggedright
18\strut
\end{minipage} & \begin{minipage}[t]{0.21\columnwidth}\raggedright
AGE-RAGE signalin\ldots{}\strut
\end{minipage}\tabularnewline
\begin{minipage}[t]{0.17\columnwidth}\raggedright
Baihuasheshecao\strut
\end{minipage} & \begin{minipage}[t]{0.16\columnwidth}\raggedright
quercetin\strut
\end{minipage} & \begin{minipage}[t]{0.12\columnwidth}\raggedright
RELA\strut
\end{minipage} & \begin{minipage}[t]{0.19\columnwidth}\raggedright
18\strut
\end{minipage} & \begin{minipage}[t]{0.21\columnwidth}\raggedright
AGE-RAGE signalin\ldots{}\strut
\end{minipage}\tabularnewline
\begin{minipage}[t]{0.17\columnwidth}\raggedright
Banzhilian\strut
\end{minipage} & \begin{minipage}[t]{0.16\columnwidth}\raggedright
baicalein\strut
\end{minipage} & \begin{minipage}[t]{0.12\columnwidth}\raggedright
RELA\strut
\end{minipage} & \begin{minipage}[t]{0.19\columnwidth}\raggedright
18\strut
\end{minipage} & \begin{minipage}[t]{0.21\columnwidth}\raggedright
AGE-RAGE signalin\ldots{}\strut
\end{minipage}\tabularnewline
\begin{minipage}[t]{0.17\columnwidth}\raggedright
Banzhilian\strut
\end{minipage} & \begin{minipage}[t]{0.16\columnwidth}\raggedright
luteolin\strut
\end{minipage} & \begin{minipage}[t]{0.12\columnwidth}\raggedright
RELA\strut
\end{minipage} & \begin{minipage}[t]{0.19\columnwidth}\raggedright
18\strut
\end{minipage} & \begin{minipage}[t]{0.21\columnwidth}\raggedright
AGE-RAGE signalin\ldots{}\strut
\end{minipage}\tabularnewline
\begin{minipage}[t]{0.17\columnwidth}\raggedright
Banzhilian\strut
\end{minipage} & \begin{minipage}[t]{0.16\columnwidth}\raggedright
quercetin\strut
\end{minipage} & \begin{minipage}[t]{0.12\columnwidth}\raggedright
NFKB1\strut
\end{minipage} & \begin{minipage}[t]{0.19\columnwidth}\raggedright
18\strut
\end{minipage} & \begin{minipage}[t]{0.21\columnwidth}\raggedright
AGE-RAGE signalin\ldots{}\strut
\end{minipage}\tabularnewline
\begin{minipage}[t]{0.17\columnwidth}\raggedright
Banzhilian\strut
\end{minipage} & \begin{minipage}[t]{0.16\columnwidth}\raggedright
quercetin\strut
\end{minipage} & \begin{minipage}[t]{0.12\columnwidth}\raggedright
RELA\strut
\end{minipage} & \begin{minipage}[t]{0.19\columnwidth}\raggedright
18\strut
\end{minipage} & \begin{minipage}[t]{0.21\columnwidth}\raggedright
AGE-RAGE signalin\ldots{}\strut
\end{minipage}\tabularnewline
\begin{minipage}[t]{0.17\columnwidth}\raggedright
Banzhilian\strut
\end{minipage} & \begin{minipage}[t]{0.16\columnwidth}\raggedright
wogonin\strut
\end{minipage} & \begin{minipage}[t]{0.12\columnwidth}\raggedright
RELA\strut
\end{minipage} & \begin{minipage}[t]{0.19\columnwidth}\raggedright
18\strut
\end{minipage} & \begin{minipage}[t]{0.21\columnwidth}\raggedright
AGE-RAGE signalin\ldots{}\strut
\end{minipage}\tabularnewline
\begin{minipage}[t]{0.17\columnwidth}\raggedright
Baihuasheshecao\strut
\end{minipage} & \begin{minipage}[t]{0.16\columnwidth}\raggedright
quercetin\strut
\end{minipage} & \begin{minipage}[t]{0.12\columnwidth}\raggedright
AKT1\strut
\end{minipage} & \begin{minipage}[t]{0.19\columnwidth}\raggedright
17\strut
\end{minipage} & \begin{minipage}[t]{0.21\columnwidth}\raggedright
AGE-RAGE signalin\ldots{}\strut
\end{minipage}\tabularnewline
\begin{minipage}[t]{0.17\columnwidth}\raggedright
Banzhilian\strut
\end{minipage} & \begin{minipage}[t]{0.16\columnwidth}\raggedright
baicalein\strut
\end{minipage} & \begin{minipage}[t]{0.12\columnwidth}\raggedright
AKT1\strut
\end{minipage} & \begin{minipage}[t]{0.19\columnwidth}\raggedright
17\strut
\end{minipage} & \begin{minipage}[t]{0.21\columnwidth}\raggedright
AGE-RAGE signalin\ldots{}\strut
\end{minipage}\tabularnewline
\begin{minipage}[t]{0.17\columnwidth}\raggedright
Banzhilian\strut
\end{minipage} & \begin{minipage}[t]{0.16\columnwidth}\raggedright
luteolin\strut
\end{minipage} & \begin{minipage}[t]{0.12\columnwidth}\raggedright
AKT1\strut
\end{minipage} & \begin{minipage}[t]{0.19\columnwidth}\raggedright
17\strut
\end{minipage} & \begin{minipage}[t]{0.21\columnwidth}\raggedright
AGE-RAGE signalin\ldots{}\strut
\end{minipage}\tabularnewline
\begin{minipage}[t]{0.17\columnwidth}\raggedright
Banzhilian\strut
\end{minipage} & \begin{minipage}[t]{0.16\columnwidth}\raggedright
quercetin\strut
\end{minipage} & \begin{minipage}[t]{0.12\columnwidth}\raggedright
AKT1\strut
\end{minipage} & \begin{minipage}[t]{0.19\columnwidth}\raggedright
17\strut
\end{minipage} & \begin{minipage}[t]{0.21\columnwidth}\raggedright
AGE-RAGE signalin\ldots{}\strut
\end{minipage}\tabularnewline
\begin{minipage}[t]{0.17\columnwidth}\raggedright
Banzhilian\strut
\end{minipage} & \begin{minipage}[t]{0.16\columnwidth}\raggedright
wogonin\strut
\end{minipage} & \begin{minipage}[t]{0.12\columnwidth}\raggedright
AKT1\strut
\end{minipage} & \begin{minipage}[t]{0.19\columnwidth}\raggedright
17\strut
\end{minipage} & \begin{minipage}[t]{0.21\columnwidth}\raggedright
AGE-RAGE signalin\ldots{}\strut
\end{minipage}\tabularnewline
\begin{minipage}[t]{0.17\columnwidth}\raggedright
Baihuasheshecao\strut
\end{minipage} & \begin{minipage}[t]{0.16\columnwidth}\raggedright
quercetin\strut
\end{minipage} & \begin{minipage}[t]{0.12\columnwidth}\raggedright
TP53\strut
\end{minipage} & \begin{minipage}[t]{0.19\columnwidth}\raggedright
14\strut
\end{minipage} & \begin{minipage}[t]{0.21\columnwidth}\raggedright
Cellular senescen\ldots{}\strut
\end{minipage}\tabularnewline
\begin{minipage}[t]{0.17\columnwidth}\raggedright
Banzhilian\strut
\end{minipage} & \begin{minipage}[t]{0.16\columnwidth}\raggedright
baicalein\strut
\end{minipage} & \begin{minipage}[t]{0.12\columnwidth}\raggedright
TP53\strut
\end{minipage} & \begin{minipage}[t]{0.19\columnwidth}\raggedright
14\strut
\end{minipage} & \begin{minipage}[t]{0.21\columnwidth}\raggedright
Cellular senescen\ldots{}\strut
\end{minipage}\tabularnewline
\begin{minipage}[t]{0.17\columnwidth}\raggedright
Banzhilian\strut
\end{minipage} & \begin{minipage}[t]{0.16\columnwidth}\raggedright
luteolin\strut
\end{minipage} & \begin{minipage}[t]{0.12\columnwidth}\raggedright
TP53\strut
\end{minipage} & \begin{minipage}[t]{0.19\columnwidth}\raggedright
14\strut
\end{minipage} & \begin{minipage}[t]{0.21\columnwidth}\raggedright
Cellular senescen\ldots{}\strut
\end{minipage}\tabularnewline
\begin{minipage}[t]{0.17\columnwidth}\raggedright
\ldots{}\strut
\end{minipage} & \begin{minipage}[t]{0.16\columnwidth}\raggedright
\ldots{}\strut
\end{minipage} & \begin{minipage}[t]{0.12\columnwidth}\raggedright
\ldots{}\strut
\end{minipage} & \begin{minipage}[t]{0.19\columnwidth}\raggedright
\ldots{}\strut
\end{minipage} & \begin{minipage}[t]{0.21\columnwidth}\raggedright
\ldots{}\strut
\end{minipage}\tabularnewline
\bottomrule
\end{longtable}

\hypertarget{bibliography}{%
\section*{Reference}\label{bibliography}}
\addcontentsline{toc}{section}{Reference}

\hypertarget{refs}{}
\begin{cslreferences}
\leavevmode\hypertarget{ref-ClusterprofilerWuTi2021}{}%
1. Wu, T. \emph{et al.} ClusterProfiler 4.0: A universal enrichment tool for interpreting omics data. \emph{The Innovation} \textbf{2}, (2021).

\leavevmode\hypertarget{ref-TheGenecardsSStelze2016}{}%
2. Stelzer, G. \emph{et al.} The genecards suite: From gene data mining to disease genome sequence analyses. \emph{Current protocols in bioinformatics} \textbf{54}, 1.30.1--1.30.33 (2016).

\leavevmode\hypertarget{ref-TcmspADatabaRuJi2014}{}%
3. Ru, J. \emph{et al.} TCMSP: A database of systems pharmacology for drug discovery from herbal medicines. \emph{Journal of cheminformatics} \textbf{6}, (2014).
\end{cslreferences}

\end{document}
