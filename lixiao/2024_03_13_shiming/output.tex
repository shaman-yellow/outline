% Options for packages loaded elsewhere
\PassOptionsToPackage{unicode}{hyperref}
\PassOptionsToPackage{hyphens}{url}
%
\documentclass[
]{article}
\usepackage{lmodern}
\usepackage{amssymb,amsmath}
\usepackage{ifxetex,ifluatex}
\ifnum 0\ifxetex 1\fi\ifluatex 1\fi=0 % if pdftex
  \usepackage[T1]{fontenc}
  \usepackage[utf8]{inputenc}
  \usepackage{textcomp} % provide euro and other symbols
\else % if luatex or xetex
  \usepackage{unicode-math}
  \defaultfontfeatures{Scale=MatchLowercase}
  \defaultfontfeatures[\rmfamily]{Ligatures=TeX,Scale=1}
\fi
% Use upquote if available, for straight quotes in verbatim environments
\IfFileExists{upquote.sty}{\usepackage{upquote}}{}
\IfFileExists{microtype.sty}{% use microtype if available
  \usepackage[]{microtype}
  \UseMicrotypeSet[protrusion]{basicmath} % disable protrusion for tt fonts
}{}
\makeatletter
\@ifundefined{KOMAClassName}{% if non-KOMA class
  \IfFileExists{parskip.sty}{%
    \usepackage{parskip}
  }{% else
    \setlength{\parindent}{0pt}
    \setlength{\parskip}{6pt plus 2pt minus 1pt}}
}{% if KOMA class
  \KOMAoptions{parskip=half}}
\makeatother
\usepackage{xcolor}
\IfFileExists{xurl.sty}{\usepackage{xurl}}{} % add URL line breaks if available
\IfFileExists{bookmark.sty}{\usepackage{bookmark}}{\usepackage{hyperref}}
\hypersetup{
  hidelinks,
  pdfcreator={LaTeX via pandoc}}
\urlstyle{same} % disable monospaced font for URLs
\usepackage[margin=1in]{geometry}
\usepackage{longtable,booktabs}
% Correct order of tables after \paragraph or \subparagraph
\usepackage{etoolbox}
\makeatletter
\patchcmd\longtable{\par}{\if@noskipsec\mbox{}\fi\par}{}{}
\makeatother
% Allow footnotes in longtable head/foot
\IfFileExists{footnotehyper.sty}{\usepackage{footnotehyper}}{\usepackage{footnote}}
\makesavenoteenv{longtable}
\usepackage{graphicx}
\makeatletter
\def\maxwidth{\ifdim\Gin@nat@width>\linewidth\linewidth\else\Gin@nat@width\fi}
\def\maxheight{\ifdim\Gin@nat@height>\textheight\textheight\else\Gin@nat@height\fi}
\makeatother
% Scale images if necessary, so that they will not overflow the page
% margins by default, and it is still possible to overwrite the defaults
% using explicit options in \includegraphics[width, height, ...]{}
\setkeys{Gin}{width=\maxwidth,height=\maxheight,keepaspectratio}
% Set default figure placement to htbp
\makeatletter
\def\fps@figure{htbp}
\makeatother
\setlength{\emergencystretch}{3em} % prevent overfull lines
\providecommand{\tightlist}{%
  \setlength{\itemsep}{0pt}\setlength{\parskip}{0pt}}
\setcounter{secnumdepth}{5}
\usepackage{caption} \captionsetup{font={footnotesize},width=6in} \renewcommand{\dblfloatpagefraction}{.9} \makeatletter \renewenvironment{figure} {\def\@captype{figure}} \makeatother \@ifundefined{Shaded}{\newenvironment{Shaded}} \@ifundefined{snugshade}{\newenvironment{snugshade}} \renewenvironment{Shaded}{\begin{snugshade}}{\end{snugshade}} \definecolor{shadecolor}{RGB}{230,230,230} \usepackage{xeCJK} \usepackage{setspace} \setstretch{1.3} \usepackage{tcolorbox} \setcounter{secnumdepth}{4} \setcounter{tocdepth}{4} \usepackage{wallpaper} \usepackage[absolute]{textpos} \tcbuselibrary{breakable} \renewenvironment{Shaded} {\begin{tcolorbox}[colback = gray!10, colframe = gray!40, width = 16cm, arc = 1mm, auto outer arc, title = {R input}]} {\end{tcolorbox}} \usepackage{titlesec} \titleformat{\paragraph} {\fontsize{10pt}{0pt}\bfseries} {\arabic{section}.\arabic{subsection}.\arabic{subsubsection}.\arabic{paragraph}} {1em} {} []
\newlength{\cslhangindent}
\setlength{\cslhangindent}{1.5em}
\newenvironment{cslreferences}%
  {}%
  {\par}

\author{}
\date{\vspace{-2.5em}}

\begin{document}

\begin{titlepage} \newgeometry{top=7.5cm}
\ThisCenterWallPaper{1.12}{~/outline/lixiao//cover_page.pdf}
\begin{center} \textbf{\Huge
养阴通脑颗粒中关键成分对脑缺血再灌注的影响}
\vspace{4em} \begin{textblock}{10}(3,5.9) \huge
\textbf{\textcolor{white}{2024-03-15}}
\end{textblock} \begin{textblock}{10}(3,7.3)
\Large \textcolor{black}{LiChuang Huang}
\end{textblock} \begin{textblock}{10}(3,11.3)
\Large \textcolor{black}{@立效研究院}
\end{textblock} \end{center} \end{titlepage}
\restoregeometry

\pagenumbering{roman}

\tableofcontents

\listoffigures

\listoftables

\newpage

\pagenumbering{arabic}

\hypertarget{abstract}{%
\section{摘要}\label{abstract}}

\hypertarget{ux9700ux6c42}{%
\subsection{需求}\label{ux9700ux6c42}}

\begin{itemize}
\tightlist
\item
  养阴通脑颗粒中治疗脑缺血再灌注的关键成分及相应信号通路(信号通路需要创新性的),1-3条
\item
  同时重点分析水蛭素对应的治疗脑缺血再灌注的信号通路
\end{itemize}

养阴通脑颗粒:地黄15g、黄芪15g、葛根18g、石斛15g、水蛭3g、川芎9g

\hypertarget{ux7ed3ux679c}{%
\subsection{结果}\label{ux7ed3ux679c}}

\hypertarget{ux6574ux4f53ux590dux65b9}{%
\subsubsection{整体复方}\label{ux6574ux4f53ux590dux65b9}}

\begin{itemize}
\tightlist
\item
  常规网络药理学,见 Fig. \ref{fig:Network-pharmacology-with-disease}, 富集结果见 Fig. \ref{fig:HERBS-KEGG-enrichment}
\item
  额外对 CIR 的 GEO 数据差异分析,富集结果 Fig. \ref{fig:MAP-KEGG-enrichment}
\item
  综合以上富集,发现 MARK 通路 (Fig. \ref{fig:HERBS-hsa04010-visualization}) 可能是治疗的关键通路之一,其靶向成分见 Tab. \ref{tab:Network-pharmacology-target-MARK-data}
\end{itemize}

\hypertarget{ux6743ux8861-hirudin-ux7684ux4f5cux7528}{%
\subsubsection{权衡 Hirudin 的作用}\label{ux6743ux8861-hirudin-ux7684ux4f5cux7528}}

HERBs 数据库 (其他数据库也是如此) 包含的 Hirudin 靶点较少。
这里,额外从 GeneCards 获取了 Hirudin 的靶点 (Tab. \ref{tab:Hirudin-targets-from-GeneCards})。

为了缩小可选通路范围,这里尝试将以下的富集结果取共同的交集 (已在上述部分完成) :

\begin{itemize}
\tightlist
\item
  复方靶向 CIR (靶点来源见 Fig. \ref{fig:Overall-targets-number-of-datasets}) 的通路
  (富集见 Fig. \ref{fig:HERBS-KEGG-enrichment})
\item
  GEO 数据集 (GSE163614) CIR DEGs 的富集结果的通路 (富集见 Fig. \ref{fig:MAP-KEGG-enrichment})
\item
  获取了更多靶点信息 (因为 HERBS 数据库或其他数据库包含的靶点信息太少,不利于分析) 的 Hirudin 靶向 CIR (GEO DEGs) 的基因的富集分析 (Fig. \ref{fig:HIRUDIN-CIR-KEGG-enrichment})
\end{itemize}

得到 (去除了名称包含其他疾病的通路):Tab. \ref{tab:All-pathways-intersection}

\begin{itemize}
\tightlist
\item
  HIF-1 signaling pathway
\item
  Apelin signaling pathway
\end{itemize}

更多信息见 \ref{he-t} 和 \ref{hi-t}

\hypertarget{introduction}{%
\section{前言}\label{introduction}}

\hypertarget{methods}{%
\section{材料和方法}\label{methods}}

\hypertarget{ux6750ux6599}{%
\subsection{材料}\label{ux6750ux6599}}

All used GEO expression data and their design:

\begin{itemize}
\tightlist
\item
  \textbf{GSE163614}: Examination of MCAO/R and Sham rat brain samples(n=3)
\end{itemize}

\hypertarget{ux65b9ux6cd5}{%
\subsection{方法}\label{ux65b9ux6cd5}}

Mainly used method:

\begin{itemize}
\tightlist
\item
  The \texttt{BindingDB} database was used for discovering association between Ligands and Receptors\textsuperscript{\protect\hyperlink{ref-BindingdbIn20Gilson2016}{1}}.
\item
  The \texttt{biomart} was used for mapping genes between organism (e.g., mgi\_symbol to hgnc\_symbol)\textsuperscript{\protect\hyperlink{ref-MappingIdentifDurinc2009}{2}}.
\item
  R package \texttt{ClusterProfiler} used for gene enrichment analysis\textsuperscript{\protect\hyperlink{ref-ClusterprofilerWuTi2021}{3}}.
\item
  GEO \url{https://www.ncbi.nlm.nih.gov/geo/} used for expression dataset aquisition.
\item
  Databses of \texttt{DisGeNet}, \texttt{GeneCards}, \texttt{PharmGKB} used for collating disease related targets\textsuperscript{\protect\hyperlink{ref-TheDisgenetKnPinero2019}{4}--\protect\hyperlink{ref-PharmgkbAWorBarbar2018}{6}}.
\item
  Website \texttt{HERB} \url{http://herb.ac.cn/} used for data source\textsuperscript{\protect\hyperlink{ref-HerbAHighThFang2021}{7}}.
\item
  R package \texttt{Limma} and \texttt{edgeR} used for differential expression analysis\textsuperscript{\protect\hyperlink{ref-LimmaPowersDiRitchi2015}{8},\protect\hyperlink{ref-EdgerDifferenChen}{9}}.
\item
  R package \texttt{PubChemR} used for querying compounds information.
\item
  R package \texttt{STEINGdb} used for PPI network construction\textsuperscript{\protect\hyperlink{ref-TheStringDataSzklar2021}{10},\protect\hyperlink{ref-CytohubbaIdenChin2014}{11}}.
\item
  Web tool of \texttt{Super-PRED} used for drug-targets relationship prediction\textsuperscript{\protect\hyperlink{ref-SuperpredUpdaNickel2014}{12}}.
\item
  The MCC score was calculated referring to algorithm of \texttt{CytoHubba}\textsuperscript{\protect\hyperlink{ref-CytohubbaIdenChin2014}{11}}.
\item
  R package \texttt{UniProt.ws} used for querying Gene or Protein information.
\item
  R version 4.3.2 (2023-10-31); Other R packages (eg., \texttt{dplyr} and \texttt{ggplot2}) used for statistic analysis or data visualization.
\end{itemize}

\hypertarget{results}{%
\section{分析结果}\label{results}}

\hypertarget{dis}{%
\section{结论}\label{dis}}

\hypertarget{workflow}{%
\section{附:分析流程}\label{workflow}}

\hypertarget{ux517bux9634ux901aux8111ux9897ux7c92}{%
\subsection{养阴通脑颗粒}\label{ux517bux9634ux901aux8111ux9897ux7c92}}

\hypertarget{ux6210ux5206}{%
\subsubsection{成分}\label{ux6210ux5206}}

Table \ref{tab:Herbs-information} (下方表格) 为表格Herbs information概览。

\textbf{(对应文件为 \texttt{Figure+Table/Herbs-information.xlsx})}

\begin{center}\begin{tcolorbox}[colback=gray!10, colframe=gray!50, width=0.9\linewidth, arc=1mm, boxrule=0.5pt]注:表格共有6行18列,以下预览的表格可能省略部分数据;表格含有6个唯一`Herb\_'。
\end{tcolorbox}
\end{center}

\begin{longtable}[]{@{}lllllllllll@{}}
\caption{\label{tab:Herbs-information}Herbs information}\tabularnewline
\toprule
Herb\_ & Herb\_p\ldots{} & Herb\_c\ldots{} & Herb\_e\ldots{} & Herb\_l\ldots{} & Proper\ldots{} & Meridians & UsePart & Function & Indica\ldots{} & \ldots{}\tabularnewline
\midrule
\endfirsthead
\toprule
Herb\_ & Herb\_p\ldots{} & Herb\_c\ldots{} & Herb\_e\ldots{} & Herb\_l\ldots{} & Proper\ldots{} & Meridians & UsePart & Function & Indica\ldots{} & \ldots{}\tabularnewline
\midrule
\endhead
HERB00\ldots{} & CHUAN \ldots{} & 川芎 & Chuanx\ldots{} & Radix \ldots{} & Warm; \ldots{} & Liver;\ldots{} & rhizome & 1. To \ldots{} & Cerebr\ldots{} & \ldots{}\tabularnewline
HERB00\ldots{} & DI HUANG & 地黄 & Radix \ldots{} & NA & NA & NA & NA & NA & NA & \ldots{}\tabularnewline
HERB00\ldots{} & GE GEN & 葛根 & root o\ldots{} & Radix \ldots{} & Cool; \ldots{} & Spleen\ldots{} & tuberoid & To rel\ldots{} & Angina\ldots{} & \ldots{}\tabularnewline
HERB00\ldots{} & HUANG QI & 黄芪 & root o\ldots{} & Radix \ldots{} & Warm; \ldots{} & Lung; \ldots{} & root & To rei\ldots{} & Common\ldots{} & \ldots{}\tabularnewline
HERB00\ldots{} & SHI HU & 石斛 & Dendro\ldots{} & Herba \ldots{} & Minor \ldots{} & Stomac\ldots{} & Dendro\ldots{} & Treatm\ldots{} & 1. Den\ldots{} & \ldots{}\tabularnewline
HERB00\ldots{} & SHUI ZHI & 水蛭 & Bigflo\ldots{} & Garden\ldots{} & Mild; \ldots{} & Liver & fruit & To cle\ldots{} & Heat t\ldots{} & \ldots{}\tabularnewline
\bottomrule
\end{longtable}

Table \ref{tab:Components-of-Herbs} (下方表格) 为表格Components of Herbs概览。

\textbf{(对应文件为 \texttt{Figure+Table/Components-of-Herbs.xlsx})}

\begin{center}\begin{tcolorbox}[colback=gray!10, colframe=gray!50, width=0.9\linewidth, arc=1mm, boxrule=0.5pt]注:表格共有725行4列,以下预览的表格可能省略部分数据;表格含有696个唯一`Ingredient.name'。
\end{tcolorbox}
\end{center}

\begin{longtable}[]{@{}llll@{}}
\caption{\label{tab:Components-of-Herbs}Components of Herbs}\tabularnewline
\toprule
herb\_id & Ingredient.id & Ingredient.name & Ingredient.alias\tabularnewline
\midrule
\endfirsthead
\toprule
herb\_id & Ingredient.id & Ingredient.name & Ingredient.alias\tabularnewline
\midrule
\endhead
HERB002560 & HBIN001244 & 13-hydroxy-9,11-o\ldots{} & NA\tabularnewline
HERB002560 & HBIN002016 & 1,7-Dihydroxy-3,9\ldots{} & 1,7-dihydroxy-3,9\ldots{}\tabularnewline
HERB002560 & HBIN003405 & 20-Hexadecanoylin\ldots{} & 20-hexadecanoylin\ldots{}\tabularnewline
HERB002560 & HBIN003436 & 20(r)-21,24-cyclo\ldots{} & 20(r)-21,24-cyclo\ldots{}\tabularnewline
HERB002560 & HBIN004319 & 2',4' -\ldots{} & 2', 4'-\ldots{}\tabularnewline
HERB002560 & HBIN005731 & 2'-hydroxy-3 & NA\tabularnewline
HERB002560 & HBIN005735 & 2'-hydroxy-3\ldots{} & NA\tabularnewline
HERB002560 & HBIN005744 & 2-hydroxy-3-metho\ldots{} & NA\tabularnewline
HERB002560 & HBIN006143 & 2-Nonyl acetate & ANW-21203; SCHEMB\ldots{}\tabularnewline
HERB002560 & HBIN006743 & (2S)-4-methoxy-7-\ldots{} & (2S)-4-methoxy-7-\ldots{}\tabularnewline
HERB002560 & HBIN007657 & 3,5-dimethoxystil\ldots{} & 78916-49-1; TR-03\ldots{}\tabularnewline
HERB002560 & HBIN007848 & 3,9-di-O-methylni\ldots{} & NA\tabularnewline
HERB002560 & HBIN008647 & 3-Hydroxy-2-picoline & BTB 09012; 3-Hydr\ldots{}\tabularnewline
HERB002560 & HBIN008667 & 3'-hydroxy-4\ldots{} & NA\tabularnewline
HERB002560 & HBIN008668 & 3'-Hydroxy-4\ldots{} & 3-(3-hydroxy-4-me\ldots{}\tabularnewline
\ldots{} & \ldots{} & \ldots{} & \ldots{}\tabularnewline
\bottomrule
\end{longtable}

Figure \ref{fig:intersection-of-all-compounds} (下方图) 为图intersection of all compounds概览。

\textbf{(对应文件为 \texttt{Figure+Table/intersection-of-all-compounds.pdf})}

\def\@captype{figure}
\begin{center}
\includegraphics[width = 0.9\linewidth]{Figure+Table/intersection-of-all-compounds.pdf}
\caption{Intersection of all compounds}\label{fig:intersection-of-all-compounds}
\end{center}
\begin{center}\begin{tcolorbox}[colback=gray!10, colframe=gray!50, width=0.9\linewidth, arc=1mm, boxrule=0.5pt]
\textbf{
All\_intersection
:}

\vspace{0.5em}



\vspace{2em}
\end{tcolorbox}
\end{center}

\textbf{(上述信息框内容已保存至 \texttt{Figure+Table/intersection-of-all-compounds-content})}

\hypertarget{ux6210ux5206ux9776ux70b9}{%
\subsubsection{成分靶点}\label{ux6210ux5206ux9776ux70b9}}

Table \ref{tab:tables-of-Herbs-compounds-and-targets} (下方表格) 为表格tables of Herbs compounds and targets概览。

\textbf{(对应文件为 \texttt{Figure+Table/tables-of-Herbs-compounds-and-targets.xlsx})}

\begin{center}\begin{tcolorbox}[colback=gray!10, colframe=gray!50, width=0.9\linewidth, arc=1mm, boxrule=0.5pt]注:表格共有13356行9列,以下预览的表格可能省略部分数据;表格含有696个唯一`Ingredient.id'。
\end{tcolorbox}
\end{center}

\begin{longtable}[]{@{}lllllllll@{}}
\caption{\label{tab:tables-of-Herbs-compounds-and-targets}Tables of Herbs compounds and targets}\tabularnewline
\toprule
Ingred\ldots\ldots1 & Herb\_p\ldots{} & Ingred\ldots\ldots3 & Ingred\ldots\ldots4 & Target.id & Target\ldots{} & Databa\ldots{} & Paper.id & \ldots{}\tabularnewline
\midrule
\endfirsthead
\toprule
Ingred\ldots\ldots1 & Herb\_p\ldots{} & Ingred\ldots\ldots3 & Ingred\ldots\ldots4 & Target.id & Target\ldots{} & Databa\ldots{} & Paper.id & \ldots{}\tabularnewline
\midrule
\endhead
HBIN00\ldots{} & SHI HU & 10,12-\ldots{} & NA & HBTAR0\ldots{} & ATIC & NA & NA & \ldots{}\tabularnewline
HBIN00\ldots{} & SHI HU & 10,12-\ldots{} & NA & HBTAR0\ldots{} & FPGS & NA & NA & \ldots{}\tabularnewline
HBIN00\ldots{} & SHI HU & 10,12-\ldots{} & NA & HBTAR0\ldots{} & GART & NA & NA & \ldots{}\tabularnewline
HBIN00\ldots{} & SHI HU & 10,12-\ldots{} & NA & HBTAR0\ldots{} & MTHFD1 & NA & NA & \ldots{}\tabularnewline
HBIN00\ldots{} & SHI HU & 10,12-\ldots{} & NA & HBTAR0\ldots{} & MTHFD2 & NA & NA & \ldots{}\tabularnewline
HBIN00\ldots{} & SHI HU & 10,12-\ldots{} & NA & HBTAR0\ldots{} & ALDH1L1 & NA & NA & \ldots{}\tabularnewline
HBIN00\ldots{} & SHI HU & 10,12-\ldots{} & NA & HBTAR0\ldots{} & MTHFD1L & NA & NA & \ldots{}\tabularnewline
HBIN00\ldots{} & SHI HU & 10,12-\ldots{} & NA & HBTAR0\ldots{} & MTFMT & NA & NA & \ldots{}\tabularnewline
HBIN00\ldots{} & SHI HU & 10,12-\ldots{} & NA & HBTAR0\ldots{} & ALDH1L2 & NA & NA & \ldots{}\tabularnewline
HBIN00\ldots{} & SHI HU & 10,12-\ldots{} & NA & HBTAR0\ldots{} & MTHFD2L & NA & NA & \ldots{}\tabularnewline
HBIN00\ldots{} & SHI HU & 10β,13\ldots{} & NA & NA & NA & NA & NA & \ldots{}\tabularnewline
HBIN00\ldots{} & CHUAN \ldots{} & 10-(be\ldots{} & 10-(β-\ldots{} & NA & NA & NA & NA & \ldots{}\tabularnewline
HBIN00\ldots{} & CHUAN \ldots{} & 1,1-Di\ldots{} & 3658-9\ldots{} & NA & NA & NA & NA & \ldots{}\tabularnewline
HBIN00\ldots{} & CHUAN \ldots{} & 1,2,3,\ldots{} & NA & NA & NA & NA & NA & \ldots{}\tabularnewline
HBIN00\ldots{} & CHUAN \ldots{} & 1,3,8-\ldots{} & 1,3,8-\ldots{} & HBTAR0\ldots{} & ACHE & NA & NA & \ldots{}\tabularnewline
\ldots{} & \ldots{} & \ldots{} & \ldots{} & \ldots{} & \ldots{} & \ldots{} & \ldots{} & \ldots{}\tabularnewline
\bottomrule
\end{longtable}

\hypertarget{ux8111ux7f3aux8840ux518dux704cux6ce8-cerebral-ischemia-reperfusion-cir-ux9776ux70b9}{%
\subsubsection{脑缺血再灌注 cerebral ischemia reperfusion (CIR) 靶点}\label{ux8111ux7f3aux8840ux518dux704cux6ce8-cerebral-ischemia-reperfusion-cir-ux9776ux70b9}}

Figure \ref{fig:Overall-targets-number-of-datasets} (下方图) 为图Overall targets number of datasets概览。

\textbf{(对应文件为 \texttt{Figure+Table/Overall-targets-number-of-datasets.pdf})}

\def\@captype{figure}
\begin{center}
\includegraphics[width = 0.9\linewidth]{Figure+Table/Overall-targets-number-of-datasets.pdf}
\caption{Overall targets number of datasets}\label{fig:Overall-targets-number-of-datasets}
\end{center}

\begin{center}\begin{tcolorbox}[colback=gray!10, colframe=gray!50, width=0.9\linewidth, arc=1mm, boxrule=0.5pt]
\textbf{
The GeneCards data was obtained by querying
:}

\vspace{0.5em}

    cerebral ischemia reperfusion

\vspace{2em}


\textbf{
Restrict (with quotes)
:}

\vspace{0.5em}

    TRUE

\vspace{2em}


\textbf{
Filtering by Score:
:}

\vspace{0.5em}

    Score > 1

\vspace{2em}
\end{tcolorbox}
\end{center}

Table \ref{tab:CIR-GeneCards-used-data} (下方表格) 为表格CIR GeneCards used data概览。

\textbf{(对应文件为 \texttt{Figure+Table/CIR-GeneCards-used-data.xlsx})}

\begin{center}\begin{tcolorbox}[colback=gray!10, colframe=gray!50, width=0.9\linewidth, arc=1mm, boxrule=0.5pt]注:表格共有139行7列,以下预览的表格可能省略部分数据;表格含有139个唯一`Symbol'。
\end{tcolorbox}
\end{center}

\begin{longtable}[]{@{}lllllll@{}}
\caption{\label{tab:CIR-GeneCards-used-data}CIR GeneCards used data}\tabularnewline
\toprule
Symbol & Description & Category & UniProt\_ID & GIFtS & GC\_id & Score\tabularnewline
\midrule
\endfirsthead
\toprule
Symbol & Description & Category & UniProt\_ID & GIFtS & GC\_id & Score\tabularnewline
\midrule
\endhead
BDNF-AS & BDNF Antis\ldots{} & RNA Gene & & 28 & GC11P027466 & 11.94\tabularnewline
CERNA3 & Competing \ldots{} & RNA Gene & & 19 & GC08P056101 & 6.64\tabularnewline
MEG3 & Maternally\ldots{} & RNA Gene & & 34 & GC14P115583 & 6.13\tabularnewline
SNHG12 & Small Nucl\ldots{} & RNA Gene & Q9BXW3 & 29 & GC01M030655 & 6.06\tabularnewline
MIR211 & MicroRNA 211 & RNA Gene & & 28 & GC15M031065 & 5.85\tabularnewline
SNHG14 & Small Nucl\ldots{} & RNA Gene & & 24 & GC15P147532 & 5.69\tabularnewline
SOD2-OT1 & SOD2 Overl\ldots{} & RNA Gene & & 18 & GC06M159772 & 5.41\tabularnewline
H19 & H19 Imprin\ldots{} & RNA Gene & & 34 & GC11M001995 & 4.64\tabularnewline
GAS5 & Growth Arr\ldots{} & RNA Gene & & 30 & GC01M173947 & 4.56\tabularnewline
TUG1 & Taurine Up\ldots{} & Protein Co\ldots{} & A0A6I8PU40 & 32 & GC22P030969 & 4.15\tabularnewline
MIR496 & MicroRNA 496 & RNA Gene & & 16 & GC14P115621 & 4.07\tabularnewline
BCL2 & BCL2 Apopt\ldots{} & Protein Co\ldots{} & P10415 & 59 & GC18M063123 & 3.7\tabularnewline
MIR532 & MicroRNA 532 & RNA Gene & & 23 & GC0XP056752 & 3.7\tabularnewline
SCARNA5 & Small Caja\ldots{} & RNA Gene & & 23 & GC02P233275 & 3.7\tabularnewline
NFE2L2 & NFE2 Like \ldots{} & Protein Co\ldots{} & Q16236 & 60 & GC02M177227 & 3.64\tabularnewline
\ldots{} & \ldots{} & \ldots{} & \ldots{} & \ldots{} & \ldots{} & \ldots{}\tabularnewline
\bottomrule
\end{longtable}

\hypertarget{ux7f51ux7edcux836fux7406-ux75beux75c5}{%
\subsubsection{网络药理-疾病}\label{ux7f51ux7edcux836fux7406-ux75beux75c5}}

Figure \ref{fig:Network-pharmacology-with-disease} (下方图) 为图Network pharmacology with disease概览。

\textbf{(对应文件为 \texttt{Figure+Table/Network-pharmacology-with-disease.pdf})}

\def\@captype{figure}
\begin{center}
\includegraphics[width = 0.9\linewidth]{Figure+Table/Network-pharmacology-with-disease.pdf}
\caption{Network pharmacology with disease}\label{fig:Network-pharmacology-with-disease}
\end{center}

Figure \ref{fig:Targets-intersect-with-targets-of-diseases} (下方图) 为图Targets intersect with targets of diseases概览。

\textbf{(对应文件为 \texttt{Figure+Table/Targets-intersect-with-targets-of-diseases.pdf})}

\def\@captype{figure}
\begin{center}
\includegraphics[width = 0.9\linewidth]{Figure+Table/Targets-intersect-with-targets-of-diseases.pdf}
\caption{Targets intersect with targets of diseases}\label{fig:Targets-intersect-with-targets-of-diseases}
\end{center}
\begin{center}\begin{tcolorbox}[colback=gray!10, colframe=gray!50, width=0.9\linewidth, arc=1mm, boxrule=0.5pt]
\textbf{
Intersection
:}

\vspace{0.5em}

    IL10, HMOX1, MMP9, PTGS2, SOD2, MPO, NOS2, IL6, CAT,
CXCL2, TLR4, ALOX5, RELA, CCL2, CASP3, SELE, XDH, FOS,
EDN1, TLR2, PLAT, PTEN, MAPK8, PPARA, CDKN1A, KDR, ADORA2A,
CXCL1, PLAU, BCL2, SOD1, PPARG, NOS3, TNF, IL1B, MAPK9,
ICAM1, TERT, JUN, ADORA2B, EFNB2, HGF, CD36, IRAK3, SLPI,
IL12A, CXCL8, C...

\vspace{2em}
\end{tcolorbox}
\end{center}

\textbf{(上述信息框内容已保存至 \texttt{Figure+Table/Targets-intersect-with-targets-of-diseases-content})}

\hypertarget{ppi-ux7f51ux7edc}{%
\subsubsection{PPI 网络}\label{ppi-ux7f51ux7edc}}

Figure \ref{fig:HERBS-raw-PPI-network} (下方图) 为图HERBS raw PPI network概览。

\textbf{(对应文件为 \texttt{Figure+Table/HERBS-raw-PPI-network.pdf})}

\def\@captype{figure}
\begin{center}
\includegraphics[width = 0.9\linewidth]{Figure+Table/HERBS-raw-PPI-network.pdf}
\caption{HERBS raw PPI network}\label{fig:HERBS-raw-PPI-network}
\end{center}

Figure \ref{fig:HERBS-Top30-MCC-score} (下方图) 为图HERBS Top30 MCC score概览。

\textbf{(对应文件为 \texttt{Figure+Table/HERBS-Top30-MCC-score.pdf})}

\def\@captype{figure}
\begin{center}
\includegraphics[width = 0.9\linewidth]{Figure+Table/HERBS-Top30-MCC-score.pdf}
\caption{HERBS Top30 MCC score}\label{fig:HERBS-Top30-MCC-score}
\end{center}

\hypertarget{ux5bccux96c6ux5206ux6790-top30}{%
\subsubsection{富集分析 (Top30)}\label{ux5bccux96c6ux5206ux6790-top30}}

Figure \ref{fig:HERBS-KEGG-enrichment} (下方图) 为图HERBS KEGG enrichment概览。

\textbf{(对应文件为 \texttt{Figure+Table/HERBS-KEGG-enrichment.pdf})}

\def\@captype{figure}
\begin{center}
\includegraphics[width = 0.9\linewidth]{Figure+Table/HERBS-KEGG-enrichment.pdf}
\caption{HERBS KEGG enrichment}\label{fig:HERBS-KEGG-enrichment}
\end{center}

Table \ref{tab:HERBS-KEGG-enrichment-data} (下方表格) 为表格HERBS KEGG enrichment data概览。

\textbf{(对应文件为 \texttt{Figure+Table/HERBS-KEGG-enrichment-data.xlsx})}

\begin{center}\begin{tcolorbox}[colback=gray!10, colframe=gray!50, width=0.9\linewidth, arc=1mm, boxrule=0.5pt]注:表格共有181行9列,以下预览的表格可能省略部分数据;表格含有181个唯一`ID'。
\end{tcolorbox}
\end{center}
\begin{center}\begin{tcolorbox}[colback=gray!10, colframe=gray!50, width=0.9\linewidth, arc=1mm, boxrule=0.5pt]\begin{enumerate}\tightlist
\item pvalue:  显著性 P。
\end{enumerate}\end{tcolorbox}
\end{center}

\begin{longtable}[]{@{}lllllllll@{}}
\caption{\label{tab:HERBS-KEGG-enrichment-data}HERBS KEGG enrichment data}\tabularnewline
\toprule
ID & Descri\ldots{} & GeneRatio & BgRatio & pvalue & p.adjust & qvalue & geneID & Count\tabularnewline
\midrule
\endfirsthead
\toprule
ID & Descri\ldots{} & GeneRatio & BgRatio & pvalue & p.adjust & qvalue & geneID & Count\tabularnewline
\midrule
\endhead
hsa05167 & Kaposi\ldots{} & 19/29 & 194/8661 & 3.0530\ldots{} & 5.5259\ldots{} & 1.0605\ldots{} & 207/59\ldots{} & 19\tabularnewline
hsa05212 & Pancre\ldots{} & 14/29 & 76/8661 & 3.1769\ldots{} & 2.8751\ldots{} & 5.5178\ldots{} & 207/59\ldots{} & 14\tabularnewline
hsa05205 & Proteo\ldots{} & 17/29 & 205/8661 & 4.7725\ldots{} & 2.8794\ldots{} & 5.5261\ldots{} & 207/85\ldots{} & 17\tabularnewline
hsa05418 & Fluid \ldots{} & 15/29 & 139/8661 & 3.5822\ldots{} & 1.6209\ldots{} & 3.1109\ldots{} & 207/85\ldots{} & 15\tabularnewline
hsa05161 & Hepati\ldots{} & 15/29 & 162/8661 & 3.8556\ldots{} & 1.3957\ldots{} & 2.6786\ldots{} & 207/13\ldots{} & 15\tabularnewline
hsa05208 & Chemic\ldots{} & 16/29 & 223/8661 & 1.1000\ldots{} & 3.3185\ldots{} & 6.3688\ldots{} & 207/19\ldots{} & 16\tabularnewline
hsa04917 & Prolac\ldots{} & 12/29 & 70/8661 & 1.3458\ldots{} & 3.4799\ldots{} & 6.6785\ldots{} & 207/59\ldots{} & 12\tabularnewline
hsa04933 & AGE-RA\ldots{} & 13/29 & 100/8661 & 1.6908\ldots{} & 3.8255\ldots{} & 7.3418\ldots{} & 207/59\ldots{} & 13\tabularnewline
hsa04010 & MAPK s\ldots{} & 17/29 & 301/8661 & 3.5870\ldots{} & 7.2139\ldots{} & 1.3844\ldots{} & 207/13\ldots{} & 17\tabularnewline
hsa05210 & Colore\ldots{} & 12/29 & 86/8661 & 1.8784\ldots{} & 3.4000\ldots{} & 6.5252\ldots{} & 207/59\ldots{} & 12\tabularnewline
hsa05235 & PD-L1 \ldots{} & 12/29 & 89/8661 & 2.9003\ldots{} & 4.4268\ldots{} & 8.4958\ldots{} & 207/19\ldots{} & 12\tabularnewline
hsa05417 & Lipid \ldots{} & 15/29 & 215/8661 & 2.9349\ldots{} & 4.4268\ldots{} & 8.4958\ldots{} & 207/23\ldots{} & 15\tabularnewline
hsa04151 & PI3K-A\ldots{} & 17/29 & 359/8661 & 7.1506\ldots{} & 9.9559\ldots{} & 1.9106\ldots{} & 207/13\ldots{} & 17\tabularnewline
hsa01522 & Endocr\ldots{} & 12/29 & 98/8661 & 9.7631\ldots{} & 1.2622\ldots{} & 2.4224\ldots{} & 207/59\ldots{} & 12\tabularnewline
hsa01521 & EGFR t\ldots{} & 11/29 & 79/8661 & 5.3600\ldots{} & 6.2132\ldots{} & 1.1924\ldots{} & 207/19\ldots{} & 11\tabularnewline
\ldots{} & \ldots{} & \ldots{} & \ldots{} & \ldots{} & \ldots{} & \ldots{} & \ldots{} & \ldots{}\tabularnewline
\bottomrule
\end{longtable}

\hypertarget{cir-ux7684-geo-ux6570ux636eux5deeux5f02ux5206ux6790}{%
\subsubsection{CIR 的 GEO 数据差异分析}\label{cir-ux7684-geo-ux6570ux636eux5deeux5f02ux5206ux6790}}

\begin{center}\begin{tcolorbox}[colback=gray!10, colframe=gray!50, width=0.9\linewidth, arc=1mm, boxrule=0.5pt]
\textbf{
Data Source ID
:}

\vspace{0.5em}

    GSE163614

\vspace{2em}


\textbf{
data\_processing
:}

\vspace{0.5em}

    paired-end reads were harvested from Illumina NovaSeq
6000 sequencer, and were quality controlled by Q30.

\vspace{2em}


\textbf{
data\_processing.1
:}

\vspace{0.5em}

    After 3’ adaptor-trimming and low quality reads
removing by cutadapt software (v1.9.3), the high quality
trimmed reads were aligned to the rat reference genome
(UCSC RN5).

\vspace{2em}


\textbf{
data\_processing.2
:}

\vspace{0.5em}

    Then, guided by the Ensembl gtf gene annotation file
with hisat2 software (v2.0.4), cuffdiff software (v2.2.1,
part of cufflinks) was used to get the gene level FPKM as
the expression profiles of mRNA, and fold change and
p-value were calculated based on FPKM, differentially
expressed mRNA were i...

\vspace{2em}


\textbf{
data\_processing.3
:}

\vspace{0.5em}

    Genome\_build: UCSC RN5

\vspace{2em}


\textbf{
(Others)
:}

\vspace{0.5em}

    ...

\vspace{2em}
\end{tcolorbox}
\end{center}

Table \ref{tab:RAT-metadata} (下方表格) 为表格RAT metadata概览。

\textbf{(对应文件为 \texttt{Figure+Table/RAT-metadata.csv})}

\begin{center}\begin{tcolorbox}[colback=gray!10, colframe=gray!50, width=0.9\linewidth, arc=1mm, boxrule=0.5pt]注:表格共有6行9列,以下预览的表格可能省略部分数据;表格含有6个唯一`sample'。
\end{tcolorbox}
\end{center}
\begin{center}\begin{tcolorbox}[colback=gray!10, colframe=gray!50, width=0.9\linewidth, arc=1mm, boxrule=0.5pt]\begin{enumerate}\tightlist
\item sample:  样品名称
\item group:  分组名称
\end{enumerate}\end{tcolorbox}
\end{center}

\begin{longtable}[]{@{}lllllllll@{}}
\caption{\label{tab:RAT-metadata}RAT metadata}\tabularnewline
\toprule
sample & group & lib.size & norm.f\ldots{} & rownames & title & strain\ldots{} & time.p\ldots{} & tissue\ldots{}\tabularnewline
\midrule
\endfirsthead
\toprule
sample & group & lib.size & norm.f\ldots{} & rownames & title & strain\ldots{} & time.p\ldots{} & tissue\ldots{}\tabularnewline
\midrule
\endhead
MCAO1 & Model & 523780\ldots{} & 1 & GSM498\ldots{} & MCAO/R-1 & Spragu\ldots{} & 24 h & brain\tabularnewline
MCAO2 & Model & 531002\ldots{} & 1 & GSM498\ldots{} & MCAO/R-2 & Spragu\ldots{} & 24 h & brain\tabularnewline
MCAO3 & Model & 582734\ldots{} & 1 & GSM498\ldots{} & MCAO/R-3 & Spragu\ldots{} & 24 h & brain\tabularnewline
Sham1 & Control & 599207\ldots{} & 1 & GSM498\ldots{} & Sham-1 & Spragu\ldots{} & 24 h & brain\tabularnewline
Sham2 & Control & 585317\ldots{} & 1 & GSM498\ldots{} & Sham-2 & Spragu\ldots{} & 24 h & brain\tabularnewline
Sham3 & Control & 588288\ldots{} & 1 & GSM498\ldots{} & Sham-3 & Spragu\ldots{} & 24 h & brain\tabularnewline
\bottomrule
\end{longtable}

\hypertarget{ux5deeux5f02ux5206ux6790}{%
\paragraph{差异分析}\label{ux5deeux5f02ux5206ux6790}}

Figure \ref{fig:RAT-Model-vs-Control-DEGs} (下方图) 为图RAT Model vs Control DEGs概览。

\textbf{(对应文件为 \texttt{Figure+Table/RAT-Model-vs-Control-DEGs.pdf})}

\def\@captype{figure}
\begin{center}
\includegraphics[width = 0.9\linewidth]{Figure+Table/RAT-Model-vs-Control-DEGs.pdf}
\caption{RAT Model vs Control DEGs}\label{fig:RAT-Model-vs-Control-DEGs}
\end{center}

\hypertarget{ux7531ux5927ux9f20ux57faux56e0ux6620ux5c04ux5230ux4ebaux7c7bux57faux56e0}{%
\paragraph{由大鼠基因映射到人类基因}\label{ux7531ux5927ux9f20ux57faux56e0ux6620ux5c04ux5230ux4ebaux7c7bux57faux56e0}}

使用 biomart 将基因映射。

Table \ref{tab:RAT-Mapped-DEGs} (下方表格) 为表格RAT Mapped DEGs概览。

\textbf{(对应文件为 \texttt{Figure+Table/RAT-Mapped-DEGs.tsv})}

\begin{center}\begin{tcolorbox}[colback=gray!10, colframe=gray!50, width=0.9\linewidth, arc=1mm, boxrule=0.5pt]注:表格共有2921行23列,以下预览的表格可能省略部分数据;表格含有2921个唯一`hgnc\_symbol'。
\end{tcolorbox}
\end{center}
\begin{center}\begin{tcolorbox}[colback=gray!10, colframe=gray!50, width=0.9\linewidth, arc=1mm, boxrule=0.5pt]\begin{enumerate}\tightlist
\item hgnc\_symbol:  基因名 (Human)
\item pathway:  相关通路。
\item logFC:  estimate of the log2-fold-change corresponding to the effect or contrast (for ‘topTableF’ there may be several columns of log-fold-changes)
\item AveExpr:  average log2-expression for the probe over all arrays and channels, same as ‘Amean’ in the ‘MarrayLM’ object
\item t:  moderated t-statistic (omitted for ‘topTableF’)
\item P.Value:  raw p-value
\item B:  log-odds that the gene is differentially expressed (omitted for ‘topTreat’)
\item gene\_id:  GENCODE/Ensembl gene ID
\item strand:  genomic strand
\end{enumerate}\end{tcolorbox}
\end{center}

\begin{longtable}[]{@{}llllllllll@{}}
\caption{\label{tab:RAT-Mapped-DEGs}RAT Mapped DEGs}\tabularnewline
\toprule
hgnc\_s\ldots{} & rgd\_sy\ldots{} & rownames & gene\_id & gene\_s\ldots{} & biotype & strand & locus & Synonyms & dbXrefs\tabularnewline
\midrule
\endfirsthead
\toprule
hgnc\_s\ldots{} & rgd\_sy\ldots{} & rownames & gene\_id & gene\_s\ldots{} & biotype & strand & locus & Synonyms & dbXrefs\tabularnewline
\midrule
\endhead
GPNMB & Gpnmb & 4927 & ENSRNO\ldots{} & Gpnmb & protei\ldots{} & + & chr4:1\ldots{} & - & RGD:71\ldots{}\tabularnewline
PDPN & Pdpn & 8530 & ENSRNO\ldots{} & Pdpn & protei\ldots{} & - & chr5:1\ldots{} & E11\textbar Gp\ldots{} & RGD:61\ldots{}\tabularnewline
STAT3 & Stat3 & 11467 & ENSRNO\ldots{} & Stat3 & protei\ldots{} & - & chr10:\ldots{} & - & RGD:37\ldots{}\tabularnewline
CNN3 & Cnn3 & 6554 & ENSRNO\ldots{} & Cnn3 & protei\ldots{} & + & chr2:2\ldots{} & - & RGD:71\ldots{}\tabularnewline
DDX21 & Ddx21 & 18611 & ENSRNO\ldots{} & Ddx21 & protei\ldots{} & - & chr20:\ldots{} & Ddx21a\ldots{} & RGD:13\ldots{}\tabularnewline
FLNC & Flnc & 4001 & ENSRNO\ldots{} & Flnc & protei\ldots{} & + & chr4:5\ldots{} & ABP-L\textbar\ldots{} & RGD:13\ldots{}\tabularnewline
IGFBP3 & Igfbp3 & 4835 & ENSRNO\ldots{} & Igfbp3 & protei\ldots{} & - & chr14:\ldots{} & IGF-BP3 & RGD:28\ldots{}\tabularnewline
MMP9 & Mmp9 & 10085 & ENSRNO\ldots{} & Mmp9 & protei\ldots{} & + & chr3:1\ldots{} & - & RGD:62\ldots{}\tabularnewline
PDE10A & Pde10a & 6404 & ENSRNO\ldots{} & Pde10a & protei\ldots{} & - & chr1:5\ldots{} & Pde10a3 & RGD:68\ldots{}\tabularnewline
SBNO2 & Sbno2 & 7959 & ENSRNO\ldots{} & Sbno2 & protei\ldots{} & + & chr7:1\ldots{} & RGD130\ldots{} & RGD:13\ldots{}\tabularnewline
SERPINA3 & Serpina3n & 5928 & ENSRNO\ldots{} & Serpina3n & protei\ldots{} & + & chr6:1\ldots{} & CPi-26\ldots{} & RGD:37\ldots{}\tabularnewline
CSF2RB & Csf2rb & 83 & ENSRNO\ldots{} & Csf2rb & protei\ldots{} & + & chr7:1\ldots{} & Csf2rb1 & RGD:62\ldots{}\tabularnewline
FLNA & Flna & 17331 & ENSRNO\ldots{} & Flna & protei\ldots{} & + & chr1:1\ldots{} & RGD156\ldots{} & RGD:15\ldots{}\tabularnewline
LCP1 & Lcp1 & 5808 & ENSRNO\ldots{} & Lcp1 & protei\ldots{} & + & chr15:\ldots{} & - & RGD:13\ldots{}\tabularnewline
MAST3 & Mast3 & 12964 & ENSRNO\ldots{} & Mast3 & protei\ldots{} & + & chr16:\ldots{} & - & RGD:15\ldots{}\tabularnewline
\ldots{} & \ldots{} & \ldots{} & \ldots{} & \ldots{} & \ldots{} & \ldots{} & \ldots{} & \ldots{} & \ldots{}\tabularnewline
\bottomrule
\end{longtable}

\hypertarget{ux5bccux96c6ux5206ux6790}{%
\paragraph{富集分析}\label{ux5bccux96c6ux5206ux6790}}

Figure \ref{fig:MAP-KEGG-enrichment} (下方图) 为图MAP KEGG enrichment概览。

\textbf{(对应文件为 \texttt{Figure+Table/MAP-KEGG-enrichment.pdf})}

\def\@captype{figure}
\begin{center}
\includegraphics[width = 0.9\linewidth]{Figure+Table/MAP-KEGG-enrichment.pdf}
\caption{MAP KEGG enrichment}\label{fig:MAP-KEGG-enrichment}
\end{center}

Table \ref{tab:MAP-KEGG-enrichment-data} (下方表格) 为表格MAP KEGG enrichment data概览。

\textbf{(对应文件为 \texttt{Figure+Table/MAP-KEGG-enrichment-data.xlsx})}

\begin{center}\begin{tcolorbox}[colback=gray!10, colframe=gray!50, width=0.9\linewidth, arc=1mm, boxrule=0.5pt]注:表格共有337行9列,以下预览的表格可能省略部分数据;表格含有337个唯一`ID'。
\end{tcolorbox}
\end{center}
\begin{center}\begin{tcolorbox}[colback=gray!10, colframe=gray!50, width=0.9\linewidth, arc=1mm, boxrule=0.5pt]\begin{enumerate}\tightlist
\item pvalue:  显著性 P。
\end{enumerate}\end{tcolorbox}
\end{center}

\begin{longtable}[]{@{}lllllllll@{}}
\caption{\label{tab:MAP-KEGG-enrichment-data}MAP KEGG enrichment data}\tabularnewline
\toprule
ID & Descri\ldots{} & GeneRatio & BgRatio & pvalue & p.adjust & qvalue & geneID & Count\tabularnewline
\midrule
\endfirsthead
\toprule
ID & Descri\ldots{} & GeneRatio & BgRatio & pvalue & p.adjust & qvalue & geneID & Count\tabularnewline
\midrule
\endhead
hsa04724 & Glutam\ldots{} & 57/1486 & 115/8661 & 9.1917\ldots{} & 3.0976\ldots{} & 1.6641\ldots{} & 107/19\ldots{} & 57\tabularnewline
hsa05033 & Nicoti\ldots{} & 27/1486 & 40/8661 & 2.0977\ldots{} & 3.5346\ldots{} & 1.8989\ldots{} & 773/77\ldots{} & 27\tabularnewline
hsa04010 & MAPK s\ldots{} & 98/1486 & 301/8661 & 2.5161\ldots{} & 2.8264\ldots{} & 1.5185\ldots{} & 10000/\ldots{} & 98\tabularnewline
hsa04727 & GABAer\ldots{} & 42/1486 & 89/8661 & 4.4346\ldots{} & 3.7361\ldots{} & 2.0072\ldots{} & 18/107\ldots{} & 42\tabularnewline
hsa05205 & Proteo\ldots{} & 73/1486 & 205/8661 & 9.9744\ldots{} & 6.7227\ldots{} & 3.6117\ldots{} & 60/71/\ldots{} & 73\tabularnewline
hsa05163 & Human \ldots{} & 76/1486 & 225/8661 & 7.0361\ldots{} & 3.9519\ldots{} & 2.1231\ldots{} & 107/19\ldots{} & 76\tabularnewline
hsa04971 & Gastri\ldots{} & 36/1486 & 76/8661 & 9.5518\ldots{} & 4.5985\ldots{} & 2.4705\ldots{} & 60/71/\ldots{} & 36\tabularnewline
hsa04925 & Aldost\ldots{} & 42/1486 & 98/8661 & 1.8482\ldots{} & 7.7857\ldots{} & 4.1828\ldots{} & 107/19\ldots{} & 42\tabularnewline
hsa04510 & Focal \ldots{} & 69/1486 & 203/8661 & 3.2209\ldots{} & 1.2060\ldots{} & 6.4794\ldots{} & 60/71/\ldots{} & 69\tabularnewline
hsa04933 & AGE-RA\ldots{} & 42/1486 & 100/8661 & 3.8738\ldots{} & 1.3054\ldots{} & 7.0136\ldots{} & 183/10\ldots{} & 42\tabularnewline
hsa04015 & Rap1 s\ldots{} & 70/1486 & 210/8661 & 6.3009\ldots{} & 1.8176\ldots{} & 9.7653\ldots{} & 60/71/\ldots{} & 70\tabularnewline
hsa05032 & Morphi\ldots{} & 39/1486 & 91/8661 & 7.0065\ldots{} & 1.8176\ldots{} & 9.7653\ldots{} & 107/19\ldots{} & 39\tabularnewline
hsa04360 & Axon g\ldots{} & 63/1486 & 182/8661 & 7.0117\ldots{} & 1.8176\ldots{} & 9.7653\ldots{} & 655/65\ldots{} & 63\tabularnewline
hsa04611 & Platel\ldots{} & 48/1486 & 124/8661 & 7.7305\ldots{} & 1.8608\ldots{} & 9.9974\ldots{} & 60/71/\ldots{} & 48\tabularnewline
hsa04670 & Leukoc\ldots{} & 45/1486 & 115/8661 & 1.5365\ldots{} & 3.4520\ldots{} & 1.8546\ldots{} & 60/71/\ldots{} & 45\tabularnewline
\ldots{} & \ldots{} & \ldots{} & \ldots{} & \ldots{} & \ldots{} & \ldots{} & \ldots{} & \ldots{}\tabularnewline
\bottomrule
\end{longtable}

可以发现,`MARK' 通路居于首位。以下展示 Fig. \ref{fig:HERBS-KEGG-enrichment} 富集结果的 `MARK' 通路:

Figure \ref{fig:HERBS-hsa04010-visualization} (下方图) 为图HERBS hsa04010 visualization概览。

\textbf{(对应文件为 \texttt{Figure+Table/hsa04010.pathview.png})}

\def\@captype{figure}
\begin{center}
\includegraphics[width = 0.9\linewidth]{pathview2024-03-14_14_08_35.864097/hsa04010.pathview.png}
\caption{HERBS hsa04010 visualization}\label{fig:HERBS-hsa04010-visualization}
\end{center}
\begin{center}\begin{tcolorbox}[colback=gray!10, colframe=gray!50, width=0.9\linewidth, arc=1mm, boxrule=0.5pt]
\textbf{
Interactive figure
:}

\vspace{0.5em}

    \url{https://www.genome.jp/pathway/hsa04010}

\vspace{2em}
\end{tcolorbox}
\end{center}

\hypertarget{ux590dux65b9ux9776ux70b9ux901aux8defux4e0e-cir-degs-ux5bccux96c6ux7ed3ux679cux7684ux5171ux540cux5bccux96c6ux901aux8def}{%
\subsubsection{复方靶点通路与 CIR DEGs 富集结果的共同富集通路}\label{ux590dux65b9ux9776ux70b9ux901aux8defux4e0e-cir-degs-ux5bccux96c6ux7ed3ux679cux7684ux5171ux540cux5bccux96c6ux901aux8def}}

Table \ref{tab:HERBS-pathways-intersection} (下方表格) 为表格HERBS pathways intersection概览。

\textbf{(对应文件为 \texttt{Figure+Table/HERBS-pathways-intersection.xlsx})}

\begin{center}\begin{tcolorbox}[colback=gray!10, colframe=gray!50, width=0.9\linewidth, arc=1mm, boxrule=0.5pt]注:表格共有99行9列,以下预览的表格可能省略部分数据;表格含有99个唯一`ID'。
\end{tcolorbox}
\end{center}
\begin{center}\begin{tcolorbox}[colback=gray!10, colframe=gray!50, width=0.9\linewidth, arc=1mm, boxrule=0.5pt]\begin{enumerate}\tightlist
\item pvalue:  显著性 P。
\end{enumerate}\end{tcolorbox}
\end{center}

\begin{longtable}[]{@{}lllllllll@{}}
\caption{\label{tab:HERBS-pathways-intersection}HERBS pathways intersection}\tabularnewline
\toprule
ID & Descri\ldots{} & GeneRatio & BgRatio & pvalue & p.adjust & qvalue & geneID & Count\tabularnewline
\midrule
\endfirsthead
\toprule
ID & Descri\ldots{} & GeneRatio & BgRatio & pvalue & p.adjust & qvalue & geneID & Count\tabularnewline
\midrule
\endhead
hsa05167 & Kaposi\ldots{} & 19/29 & 194/8661 & 3.0530\ldots{} & 5.5259\ldots{} & 1.0605\ldots{} & 207/59\ldots{} & 19\tabularnewline
hsa05212 & Pancre\ldots{} & 14/29 & 76/8661 & 3.1769\ldots{} & 2.8751\ldots{} & 5.5178\ldots{} & 207/59\ldots{} & 14\tabularnewline
hsa05205 & Proteo\ldots{} & 17/29 & 205/8661 & 4.7725\ldots{} & 2.8794\ldots{} & 5.5261\ldots{} & 207/85\ldots{} & 17\tabularnewline
hsa05418 & Fluid \ldots{} & 15/29 & 139/8661 & 3.5822\ldots{} & 1.6209\ldots{} & 3.1109\ldots{} & 207/85\ldots{} & 15\tabularnewline
hsa05161 & Hepati\ldots{} & 15/29 & 162/8661 & 3.8556\ldots{} & 1.3957\ldots{} & 2.6786\ldots{} & 207/13\ldots{} & 15\tabularnewline
hsa04933 & AGE-RA\ldots{} & 13/29 & 100/8661 & 1.6908\ldots{} & 3.8255\ldots{} & 7.3418\ldots{} & 207/59\ldots{} & 13\tabularnewline
hsa04010 & MAPK s\ldots{} & 17/29 & 301/8661 & 3.5870\ldots{} & 7.2139\ldots{} & 1.3844\ldots{} & 207/13\ldots{} & 17\tabularnewline
hsa05210 & Colore\ldots{} & 12/29 & 86/8661 & 1.8784\ldots{} & 3.4000\ldots{} & 6.5252\ldots{} & 207/59\ldots{} & 12\tabularnewline
hsa05417 & Lipid \ldots{} & 15/29 & 215/8661 & 2.9349\ldots{} & 4.4268\ldots{} & 8.4958\ldots{} & 207/23\ldots{} & 15\tabularnewline
hsa04151 & PI3K-A\ldots{} & 17/29 & 359/8661 & 7.1506\ldots{} & 9.9559\ldots{} & 1.9106\ldots{} & 207/13\ldots{} & 17\tabularnewline
hsa01522 & Endocr\ldots{} & 12/29 & 98/8661 & 9.7631\ldots{} & 1.2622\ldots{} & 2.4224\ldots{} & 207/59\ldots{} & 12\tabularnewline
hsa04510 & Focal \ldots{} & 14/29 & 203/8661 & 5.4923\ldots{} & 6.2132\ldots{} & 1.1924\ldots{} & 207/85\ldots{} & 14\tabularnewline
hsa05207 & Chemic\ldots{} & 14/29 & 212/8661 & 1.0132\ldots{} & 1.0787\ldots{} & 2.0703\ldots{} & 207/13\ldots{} & 14\tabularnewline
hsa05163 & Human \ldots{} & 14/29 & 225/8661 & 2.3412\ldots{} & 2.3542\ldots{} & 4.5181\ldots{} & 207/13\ldots{} & 14\tabularnewline
hsa04926 & Relaxi\ldots{} & 12/29 & 129/8661 & 2.9702\ldots{} & 2.8295\ldots{} & 5.4304\ldots{} & 207/13\ldots{} & 12\tabularnewline
\ldots{} & \ldots{} & \ldots{} & \ldots{} & \ldots{} & \ldots{} & \ldots{} & \ldots{} & \ldots{}\tabularnewline
\bottomrule
\end{longtable}

\hypertarget{ux590dux65b9ux5bf9-mark-ux901aux8def}{%
\subsubsection{复方对 MARK 通路}\label{ux590dux65b9ux5bf9-mark-ux901aux8def}}

Figure \ref{fig:Network-pharmacology-target-MARK} (下方图) 为图Network pharmacology target MARK概览。

\textbf{(对应文件为 \texttt{Figure+Table/Network-pharmacology-target-MARK.pdf})}

\def\@captype{figure}
\begin{center}
\includegraphics[width = 0.9\linewidth]{Figure+Table/Network-pharmacology-target-MARK.pdf}
\caption{Network pharmacology target MARK}\label{fig:Network-pharmacology-target-MARK}
\end{center}

\hypertarget{ux590dux65b9ux4f5cux7528ux4e8e-mark-ux901aux8defux7684ux6210ux5206}{%
\paragraph{复方作用于 MARK 通路的成分}\label{ux590dux65b9ux4f5cux7528ux4e8e-mark-ux901aux8defux7684ux6210ux5206}}

Table \ref{tab:Network-pharmacology-target-MARK-data} (下方表格) 为表格Network pharmacology target MARK data概览。

\textbf{(对应文件为 \texttt{Figure+Table/Network-pharmacology-target-MARK-data.xlsx})}

\begin{center}\begin{tcolorbox}[colback=gray!10, colframe=gray!50, width=0.9\linewidth, arc=1mm, boxrule=0.5pt]注:表格共有297行3列,以下预览的表格可能省略部分数据;表格含有6个唯一`Herb\_pinyin\_name'。
\end{tcolorbox}
\end{center}

\begin{longtable}[]{@{}lll@{}}
\caption{\label{tab:Network-pharmacology-target-MARK-data}Network pharmacology target MARK data}\tabularnewline
\toprule
Herb\_pinyin\_name & Ingredient.name & Target.name\tabularnewline
\midrule
\endfirsthead
\toprule
Herb\_pinyin\_name & Ingredient.name & Target.name\tabularnewline
\midrule
\endhead
HUANG QI & 1,7-Dihydroxy-3,9-dimethoxy\ldots{} & MAPK14\tabularnewline
CHUAN XIONG & 1-Acetyl-beta-carboline & MAPK14\tabularnewline
CHUAN XIONG & 1-beta-ethylacrylate-7-alde\ldots{} & MAPK14\tabularnewline
HUANG QI & 3,9-di-O-methylnissolin & MAPK14\tabularnewline
CHUAN XIONG & 3-butylidene-phalide & TP53\tabularnewline
GE GEN & 3'-Methoxydaidzein & MAPK14\tabularnewline
HUANG QI & 5'-hydroxyiso-muronula\ldots{} & RELA\tabularnewline
HUANG QI & (6aR,11aR)-9,10-dimethoxy-6\ldots{} & MAPK14\tabularnewline
GE GEN & 7,8,4'-Trihydroxyisofl\ldots{} & MAPK14\tabularnewline
HUANG QI & 7-O-methylisomucronulatol & MAPK14\tabularnewline
HUANG QI & acetic acid & FOS\tabularnewline
HUANG QI & acetic acid & RELA\tabularnewline
HUANG QI & acetic acid & FOS\tabularnewline
HUANG QI & acetic acid & RELA\tabularnewline
HUANG QI & adeninenucleoside & FOS\tabularnewline
\ldots{} & \ldots{} & \ldots{}\tabularnewline
\bottomrule
\end{longtable}

\hypertarget{ux6c34ux86edux7d20-hirudin}{%
\subsection{水蛭素 Hirudin}\label{ux6c34ux86edux7d20-hirudin}}

\hypertarget{hirudin-ux9776ux70b9-ux83b7ux53d6ux66f4ux591aux9776ux70b9}{%
\subsubsection{Hirudin 靶点 (获取更多靶点)}\label{hirudin-ux9776ux70b9-ux83b7ux53d6ux66f4ux591aux9776ux70b9}}

HERBs 数据库包含的 Hirudin 靶点较少:

Table \ref{tab:Hirudin-targets-in-HERB-database} (下方表格) 为表格Hirudin targets in HERB database概览。

\textbf{(对应文件为 \texttt{Figure+Table/Hirudin-targets-in-HERB-database.csv})}

\begin{center}\begin{tcolorbox}[colback=gray!10, colframe=gray!50, width=0.9\linewidth, arc=1mm, boxrule=0.5pt]注:表格共有4行3列,以下预览的表格可能省略部分数据;表格含有1个唯一`Herb\_pinyin\_name'。
\end{tcolorbox}
\end{center}

\begin{longtable}[]{@{}lll@{}}
\caption{\label{tab:Hirudin-targets-in-HERB-database}Hirudin targets in HERB database}\tabularnewline
\toprule
Herb\_pinyin\_name & Ingredient.name & Target.name\tabularnewline
\midrule
\endfirsthead
\toprule
Herb\_pinyin\_name & Ingredient.name & Target.name\tabularnewline
\midrule
\endhead
SHUI ZHI & hirudin & F2\tabularnewline
SHUI ZHI & hirudin & F3\tabularnewline
SHUI ZHI & hirudin & F5\tabularnewline
SHUI ZHI & hirudin & MIF\tabularnewline
\bottomrule
\end{longtable}

\hypertarget{genecards-ux83b7ux53d6ux5316ux5408ux7269ux9776ux70b9}{%
\paragraph{GeneCards 获取化合物靶点}\label{genecards-ux83b7ux53d6ux5316ux5408ux7269ux9776ux70b9}}

bindingdb, drugbank, 以及预测工具 Super-Pred 等都难以获取更多关于 hirudin 靶点信息。
因此,这里使用 \texttt{GeneCards} 搜索。

\begin{center}\begin{tcolorbox}[colback=gray!10, colframe=gray!50, width=0.9\linewidth, arc=1mm, boxrule=0.5pt]
\textbf{
The GeneCards data was obtained by querying
:}

\vspace{0.5em}

    hirudin

\vspace{2em}


\textbf{
Restrict (with quotes)
:}

\vspace{0.5em}

    FALSE

\vspace{2em}


\textbf{
Filtering by Score:
:}

\vspace{0.5em}

    Score > 0

\vspace{2em}


\textbf{
Advance search:
:}

\vspace{0.5em}

    [compounds] ( hirudin )

\vspace{2em}
\end{tcolorbox}
\end{center}

Table \ref{tab:Hirudin-targets-from-GeneCards} (下方表格) 为表格Hirudin targets from GeneCards概览。

\textbf{(对应文件为 \texttt{Figure+Table/Hirudin-targets-from-GeneCards.xlsx})}

\begin{center}\begin{tcolorbox}[colback=gray!10, colframe=gray!50, width=0.9\linewidth, arc=1mm, boxrule=0.5pt]注:表格共有45行7列,以下预览的表格可能省略部分数据;表格含有45个唯一`Symbol'。
\end{tcolorbox}
\end{center}

\begin{longtable}[]{@{}lllllll@{}}
\caption{\label{tab:Hirudin-targets-from-GeneCards}Hirudin targets from GeneCards}\tabularnewline
\toprule
Symbol & Description & Category & UniProt\_ID & GIFtS & GC\_id & Score\tabularnewline
\midrule
\endfirsthead
\toprule
Symbol & Description & Category & UniProt\_ID & GIFtS & GC\_id & Score\tabularnewline
\midrule
\endhead
F2 & Coagulatio\ldots{} & Protein Co\ldots{} & P00734 & 58 & GC11P047386 & 2.58\tabularnewline
F2R & Coagulatio\ldots{} & Protein Co\ldots{} & P25116 & 55 & GC05P076716 & 2.23\tabularnewline
F10 & Coagulatio\ldots{} & Protein Co\ldots{} & P00742 & 58 & GC13P113122 & 1.76\tabularnewline
FGA & Fibrinogen\ldots{} & Protein Co\ldots{} & P02671 & 58 & GC04M154583 & 1.76\tabularnewline
PLAT & Plasminoge\ldots{} & Protein Co\ldots{} & P00750 & 57 & GC08M042174 & 1.76\tabularnewline
F3 & Coagulatio\ldots{} & Protein Co\ldots{} & P13726 & 54 & GC01M094825 & 1.76\tabularnewline
PLG & Plasminogen & Protein Co\ldots{} & P00747 & 58 & GC06P160702 & 1.59\tabularnewline
CPA1 & Carboxypep\ldots{} & Protein Co\ldots{} & P15085 & 51 & GC07P130380 & 1.12\tabularnewline
PLAU & Plasminoge\ldots{} & Protein Co\ldots{} & P00749 & 60 & GC10P073909 & 0.64\tabularnewline
SERPINE1 & Serpin Fam\ldots{} & Protein Co\ldots{} & P05121 & 59 & GC07P101127 & 0.64\tabularnewline
CCL2 & C-C Motif \ldots{} & Protein Co\ldots{} & P13500 & 58 & GC17P034255 & 0.64\tabularnewline
CD40LG & CD40 Ligand & Protein Co\ldots{} & P29965 & 58 & GC0XP136649 & 0.64\tabularnewline
CD55 & CD55 Molec\ldots{} & Protein Co\ldots{} & P08174 & 58 & GC01P207321 & 0.64\tabularnewline
SERPINC1 & Serpin Fam\ldots{} & Protein Co\ldots{} & P01008 & 58 & GC01M174899 & 0.64\tabularnewline
TBXA2R & Thromboxan\ldots{} & Protein Co\ldots{} & P21731 & 58 & GC19M003594 & 0.64\tabularnewline
\ldots{} & \ldots{} & \ldots{} & \ldots{} & \ldots{} & \ldots{} & \ldots{}\tabularnewline
\bottomrule
\end{longtable}

\hypertarget{hirudin-ux9776ux70b9ux4e0e-cir-degs-ux4ea4ux96c6}{%
\subsubsection{Hirudin 靶点与 CIR DEGs 交集}\label{hirudin-ux9776ux70b9ux4e0e-cir-degs-ux4ea4ux96c6}}

Figure \ref{fig:Intersection-of-Hirudin-Targets-with-CIR-DEGs} (下方图) 为图Intersection of Hirudin Targets with CIR DEGs概览。

\textbf{(对应文件为 \texttt{Figure+Table/Intersection-of-Hirudin-Targets-with-CIR-DEGs.pdf})}

\def\@captype{figure}
\begin{center}
\includegraphics[width = 0.9\linewidth]{Figure+Table/Intersection-of-Hirudin-Targets-with-CIR-DEGs.pdf}
\caption{Intersection of Hirudin Targets with CIR DEGs}\label{fig:Intersection-of-Hirudin-Targets-with-CIR-DEGs}
\end{center}
\begin{center}\begin{tcolorbox}[colback=gray!10, colframe=gray!50, width=0.9\linewidth, arc=1mm, boxrule=0.5pt]
\textbf{
Intersection
:}

\vspace{0.5em}

    PLAT, PLAU, SERPINE1, VWF, THBD, SELP, THBS1, TIMP1,
PLAUR, F2RL1, SELE, PROCR, FGL2, SCG5

\vspace{2em}
\end{tcolorbox}
\end{center}

\textbf{(上述信息框内容已保存至 \texttt{Figure+Table/Intersection-of-Hirudin-Targets-with-CIR-DEGs-content})}

\hypertarget{ux4ea4ux96c6ux57faux56e0ux7684ux5bccux96c6ux5206ux6790}{%
\paragraph{交集基因的富集分析}\label{ux4ea4ux96c6ux57faux56e0ux7684ux5bccux96c6ux5206ux6790}}

Figure \ref{fig:HIRUDIN-CIR-KEGG-enrichment} (下方图) 为图HIRUDIN CIR KEGG enrichment概览。

\textbf{(对应文件为 \texttt{Figure+Table/HIRUDIN-CIR-KEGG-enrichment.pdf})}

\def\@captype{figure}
\begin{center}
\includegraphics[width = 0.9\linewidth]{Figure+Table/HIRUDIN-CIR-KEGG-enrichment.pdf}
\caption{HIRUDIN CIR KEGG enrichment}\label{fig:HIRUDIN-CIR-KEGG-enrichment}
\end{center}

\hypertarget{ux4e0eux590dux65b9ux5171ux540cux4f5cux7528ux7684ux4fe1ux53f7ux901aux8def}{%
\paragraph{与复方共同作用的信号通路}\label{ux4e0eux590dux65b9ux5171ux540cux4f5cux7528ux7684ux4fe1ux53f7ux901aux8def}}

因为在 Hirudin 的富集分析前,额外从 GeneCards 获取了 Hirudin 的靶点,这一部分在复方分析中是不包含的;
因此,这里尝试寻找它们共同的靶向通路 (复方与获取了额外靶点的 Hirudin 的共同富集通路)。

Table \ref{tab:HIRUDIN-Herbs-pathways-intersection} (下方表格) 为表格HIRUDIN Herbs pathways intersection概览。

\textbf{(对应文件为 \texttt{Figure+Table/HIRUDIN-Herbs-pathways-intersection.csv})}

\begin{center}\begin{tcolorbox}[colback=gray!10, colframe=gray!50, width=0.9\linewidth, arc=1mm, boxrule=0.5pt]注:表格共有7行9列,以下预览的表格可能省略部分数据;表格含有7个唯一`ID'。
\end{tcolorbox}
\end{center}
\begin{center}\begin{tcolorbox}[colback=gray!10, colframe=gray!50, width=0.9\linewidth, arc=1mm, boxrule=0.5pt]\begin{enumerate}\tightlist
\item pvalue:  显著性 P。
\end{enumerate}\end{tcolorbox}
\end{center}

\begin{longtable}[]{@{}lllllllll@{}}
\caption{\label{tab:HIRUDIN-Herbs-pathways-intersection}HIRUDIN Herbs pathways intersection}\tabularnewline
\toprule
ID & Descri\ldots{} & GeneRatio & BgRatio & pvalue & p.adjust & qvalue & geneID & Count\tabularnewline
\midrule
\endfirsthead
\toprule
ID & Descri\ldots{} & GeneRatio & BgRatio & pvalue & p.adjust & qvalue & geneID & Count\tabularnewline
\midrule
\endhead
hsa04933 & AGE-RA\ldots{} & 3/12 & 100/8661 & 0.0003\ldots{} & 0.0034\ldots{} & 0.0018\ldots{} & 6401/5\ldots{} & 3\tabularnewline
hsa05418 & Fluid \ldots{} & 3/12 & 139/8661 & 0.0008\ldots{} & 0.0068\ldots{} & 0.0035\ldots{} & 5327/6\ldots{} & 3\tabularnewline
hsa05205 & Proteo\ldots{} & 3/12 & 205/8661 & 0.0024\ldots{} & 0.0139\ldots{} & 0.0073\ldots{} & 5328/5\ldots{} & 3\tabularnewline
hsa04115 & p53 si\ldots{} & 2/12 & 75/8661 & 0.0046\ldots{} & 0.0224\ldots{} & 0.0118\ldots{} & 5054/7057 & 2\tabularnewline
hsa05215 & Prosta\ldots{} & 2/12 & 97/8661 & 0.0076\ldots{} & 0.0287\ldots{} & 0.0151\ldots{} & 5327/5328 & 2\tabularnewline
hsa04066 & HIF-1 \ldots{} & 2/12 & 109/8661 & 0.0095\ldots{} & 0.0324\ldots{} & 0.0170\ldots{} & 5054/7076 & 2\tabularnewline
hsa04371 & Apelin\ldots{} & 2/12 & 139/8661 & 0.0151\ldots{} & 0.0469\ldots{} & 0.0247\ldots{} & 5327/5054 & 2\tabularnewline
\bottomrule
\end{longtable}

\hypertarget{ux6700ux7ec8ux7b5bux9009-ux7740ux91cdux8003ux8651-hirudin}{%
\subsection{最终筛选 (着重考虑 Hirudin)}\label{ux6700ux7ec8ux7b5bux9009-ux7740ux91cdux8003ux8651-hirudin}}

为了缩小可选通路范围,这里尝试将以下的富集结果取共同的交集 (已在上述部分完成) :

\begin{itemize}
\tightlist
\item
  复方靶向 CIR (靶点来源见 Fig. \ref{fig:Overall-targets-number-of-datasets}) 的通路
  (富集见 Fig. \ref{fig:HERBS-KEGG-enrichment})
\item
  GEO 数据集 (GSE163614) CIR DEGs 的富集结果的通路 (富集见 Fig. \ref{fig:MAP-KEGG-enrichment})
\item
  获取了更多靶点信息 (因为 HERBS 数据库或其他数据库包含的靶点信息太少,不利于分析) 的 Hirudin 靶向 CIR (GEO DEGs) 的基因的富集分析 (Fig. \ref{fig:HIRUDIN-CIR-KEGG-enrichment})
\end{itemize}

得到 (去除了名称包含其他疾病的通路):

Table \ref{tab:All-pathways-intersection} (下方表格) 为表格All pathways intersection概览。

\textbf{(对应文件为 \texttt{Figure+Table/All-pathways-intersection.csv})}

\begin{center}\begin{tcolorbox}[colback=gray!10, colframe=gray!50, width=0.9\linewidth, arc=1mm, boxrule=0.5pt]注:表格共有2行9列,以下预览的表格可能省略部分数据;表格含有2个唯一`ID'。
\end{tcolorbox}
\end{center}
\begin{center}\begin{tcolorbox}[colback=gray!10, colframe=gray!50, width=0.9\linewidth, arc=1mm, boxrule=0.5pt]\begin{enumerate}\tightlist
\item pvalue:  显著性 P。
\end{enumerate}\end{tcolorbox}
\end{center}

\begin{longtable}[]{@{}lllllllll@{}}
\caption{\label{tab:All-pathways-intersection}All pathways intersection}\tabularnewline
\toprule
ID & Descri\ldots{} & GeneRatio & BgRatio & pvalue & p.adjust & qvalue & geneID & Count\tabularnewline
\midrule
\endfirsthead
\toprule
ID & Descri\ldots{} & GeneRatio & BgRatio & pvalue & p.adjust & qvalue & geneID & Count\tabularnewline
\midrule
\endhead
hsa04066 & HIF-1 \ldots{} & 11/29 & 109/8661 & 2.1482\ldots{} & 1.4400\ldots{} & 2.7637\ldots{} & 207/10\ldots{} & 11\tabularnewline
hsa04371 & Apelin\ldots{} & 3/29 & 139/8661 & 0.0108\ldots{} & 0.0141\ldots{} & 0.0027\ldots{} & 207/59\ldots{} & 3\tabularnewline
\bottomrule
\end{longtable}

\hypertarget{he-t}{%
\subsubsection{复方对筛选通路的靶向}\label{he-t}}

Figure \ref{fig:HERBS-hsa04066-visualization} (下方图) 为图HERBS hsa04066 visualization概览。

\textbf{(对应文件为 \texttt{Figure+Table/hsa04066.pathview.png})}

\def\@captype{figure}
\begin{center}
\includegraphics[width = 0.9\linewidth]{pathview2024-03-14_14_08_35.864097/hsa04066.pathview.png}
\caption{HERBS hsa04066 visualization}\label{fig:HERBS-hsa04066-visualization}
\end{center}
\begin{center}\begin{tcolorbox}[colback=gray!10, colframe=gray!50, width=0.9\linewidth, arc=1mm, boxrule=0.5pt]
\textbf{
Interactive figure
:}

\vspace{0.5em}

    \url{https://www.genome.jp/pathway/hsa04066}

\vspace{2em}
\end{tcolorbox}
\end{center}

Figure \ref{fig:HERBS-hsa04371-visualization} (下方图) 为图HERBS hsa04371 visualization概览。

\textbf{(对应文件为 \texttt{Figure+Table/hsa04371.pathview.png})}

\def\@captype{figure}
\begin{center}
\includegraphics[width = 0.9\linewidth]{pathview2024-03-14_14_08_35.864097/hsa04371.pathview.png}
\caption{HERBS hsa04371 visualization}\label{fig:HERBS-hsa04371-visualization}
\end{center}
\begin{center}\begin{tcolorbox}[colback=gray!10, colframe=gray!50, width=0.9\linewidth, arc=1mm, boxrule=0.5pt]
\textbf{
Interactive figure
:}

\vspace{0.5em}

    \url{https://www.genome.jp/pathway/hsa04371}

\vspace{2em}
\end{tcolorbox}
\end{center}

\hypertarget{ux76f8ux5173ux6210ux5206}{%
\paragraph{相关成分}\label{ux76f8ux5173ux6210ux5206}}

Table \ref{tab:Compounds-target-HIF-1-signaling-pathway} (下方表格) 为表格Compounds target HIF 1 signaling pathway概览。

\textbf{(对应文件为 \texttt{Figure+Table/Compounds-target-HIF-1-signaling-pathway.xlsx})}

\begin{center}\begin{tcolorbox}[colback=gray!10, colframe=gray!50, width=0.9\linewidth, arc=1mm, boxrule=0.5pt]注:表格共有137行9列,以下预览的表格可能省略部分数据;表格含有38个唯一`Ingredient.id'。
\end{tcolorbox}
\end{center}

\begin{longtable}[]{@{}lllllllll@{}}
\caption{\label{tab:Compounds-target-HIF-1-signaling-pathway}Compounds target HIF 1 signaling pathway}\tabularnewline
\toprule
Ingred\ldots\ldots1 & Herb\_p\ldots{} & Ingred\ldots\ldots3 & Ingred\ldots\ldots4 & Target.id & Target\ldots{} & Databa\ldots{} & Paper.id & \ldots{}\tabularnewline
\midrule
\endfirsthead
\toprule
Ingred\ldots\ldots1 & Herb\_p\ldots{} & Ingred\ldots\ldots3 & Ingred\ldots\ldots4 & Target.id & Target\ldots{} & Databa\ldots{} & Paper.id & \ldots{}\tabularnewline
\midrule
\endhead
HBIN00\ldots{} & CHUAN \ldots{} & 3-buty\ldots{} & NA & HBTAR0\ldots{} & CDKN1A & NA & NA & \ldots{}\tabularnewline
HBIN01\ldots{} & HUANG QI & 5\&apos\ldots{} & NA & HBTAR0\ldots{} & RELA & NA & NA & \ldots{}\tabularnewline
HBIN01\ldots{} & HUANG QI & acetic\ldots{} & AI3-02\ldots{} & HBTAR0\ldots{} & RELA & NA & NA & \ldots{}\tabularnewline
HBIN01\ldots{} & HUANG QI & acetic\ldots{} & AI3-02\ldots{} & HBTAR0\ldots{} & RELA & NA & NA & \ldots{}\tabularnewline
HBIN01\ldots{} & HUANG QI & adenin\ldots{} & NA & HBTAR0\ldots{} & HIF1A & NA & NA & \ldots{}\tabularnewline
HBIN01\ldots{} & HUANG QI & adenin\ldots{} & NA & HBTAR0\ldots{} & VEGFA & NA & NA & \ldots{}\tabularnewline
HBIN01\ldots{} & HUANG QI & adenin\ldots{} & NA & HBTAR0\ldots{} & HIF1A & NA & NA & \ldots{}\tabularnewline
HBIN01\ldots{} & HUANG QI & adenin\ldots{} & NA & HBTAR0\ldots{} & VEGFA & NA & NA & \ldots{}\tabularnewline
HBIN01\ldots{} & HUANG QI & adenin\ldots{} & NA & HBTAR0\ldots{} & HIF1A & NA & NA & \ldots{}\tabularnewline
HBIN01\ldots{} & HUANG QI & adenin\ldots{} & NA & HBTAR0\ldots{} & VEGFA & NA & NA & \ldots{}\tabularnewline
HBIN01\ldots{} & HUANG QI & adenin\ldots{} & NA & HBTAR0\ldots{} & HIF1A & NA & NA & \ldots{}\tabularnewline
HBIN01\ldots{} & HUANG QI & adenin\ldots{} & NA & HBTAR0\ldots{} & VEGFA & NA & NA & \ldots{}\tabularnewline
HBIN01\ldots{} & HUANG QI & astram\ldots{} & AC1L3V\ldots{} & HBTAR0\ldots{} & AKT1 & NA & NA & \ldots{}\tabularnewline
HBIN01\ldots{} & HUANG QI & astram\ldots{} & AC1L3V\ldots{} & HBTAR0\ldots{} & TEK & NA & NA & \ldots{}\tabularnewline
HBIN01\ldots{} & HUANG QI & beta c\ldots{} & Spectr\ldots{} & HBTAR0\ldots{} & AKT1 & NA & NA & \ldots{}\tabularnewline
\ldots{} & \ldots{} & \ldots{} & \ldots{} & \ldots{} & \ldots{} & \ldots{} & \ldots{} & \ldots{}\tabularnewline
\bottomrule
\end{longtable}

Table \ref{tab:Compounds-target-Apelin-signaling-pathway} (下方表格) 为表格Compounds target Apelin signaling pathway概览。

\textbf{(对应文件为 \texttt{Figure+Table/Compounds-target-Apelin-signaling-pathway.xlsx})}

\begin{center}\begin{tcolorbox}[colback=gray!10, colframe=gray!50, width=0.9\linewidth, arc=1mm, boxrule=0.5pt]注:表格共有61行9列,以下预览的表格可能省略部分数据;表格含有17个唯一`Ingredient.id'。
\end{tcolorbox}
\end{center}

\begin{longtable}[]{@{}lllllllll@{}}
\caption{\label{tab:Compounds-target-Apelin-signaling-pathway}Compounds target Apelin signaling pathway}\tabularnewline
\toprule
Ingred\ldots\ldots1 & Herb\_p\ldots{} & Ingred\ldots\ldots3 & Ingred\ldots\ldots4 & Target.id & Target\ldots{} & Databa\ldots{} & Paper.id & \ldots{}\tabularnewline
\midrule
\endfirsthead
\toprule
Ingred\ldots\ldots1 & Herb\_p\ldots{} & Ingred\ldots\ldots3 & Ingred\ldots\ldots4 & Target.id & Target\ldots{} & Databa\ldots{} & Paper.id & \ldots{}\tabularnewline
\midrule
\endhead
HBIN00\ldots{} & CHUAN \ldots{} & 3-buty\ldots{} & NA & HBTAR0\ldots{} & CCND1 & NA & NA & \ldots{}\tabularnewline
HBIN01\ldots{} & HUANG QI & 5\&apos\ldots{} & NA & HBTAR0\ldots{} & CCND1 & NA & NA & \ldots{}\tabularnewline
HBIN01\ldots{} & HUANG QI & astram\ldots{} & AC1L3V\ldots{} & HBTAR0\ldots{} & AKT1 & NA & NA & \ldots{}\tabularnewline
HBIN01\ldots{} & HUANG QI & beta c\ldots{} & Spectr\ldots{} & HBTAR0\ldots{} & AKT1 & NA & NA & \ldots{}\tabularnewline
HBIN01\ldots{} & HUANG QI & beta c\ldots{} & Spectr\ldots{} & HBTAR0\ldots{} & AKT1 & NA & NA & \ldots{}\tabularnewline
HBIN01\ldots{} & HUANG QI & calycosin & HSDB 8\ldots{} & HBTAR0\ldots{} & AKT1 & NA & HBREF0\ldots{} & \ldots{}\tabularnewline
HBIN01\ldots{} & HUANG QI & calycosin & HSDB 8\ldots{} & HBTAR0\ldots{} & MAPK1 & NA & HBREF0\ldots{} & \ldots{}\tabularnewline
HBIN01\ldots{} & HUANG QI & calycosin & HSDB 8\ldots{} & HBTAR0\ldots{} & AKT1 & NA & HBREF0\ldots{} & \ldots{}\tabularnewline
HBIN01\ldots{} & HUANG QI & calycosin & HSDB 8\ldots{} & HBTAR0\ldots{} & MAPK1 & NA & HBREF0\ldots{} & \ldots{}\tabularnewline
HBIN01\ldots{} & HUANG QI & calycosin & HSDB 8\ldots{} & HBTAR0\ldots{} & AKT1 & NA & HBREF0\ldots{} & \ldots{}\tabularnewline
HBIN01\ldots{} & HUANG QI & calycosin & HSDB 8\ldots{} & HBTAR0\ldots{} & MAPK1 & NA & HBREF0\ldots{} & \ldots{}\tabularnewline
HBIN01\ldots{} & HUANG QI & calycosin & HSDB 8\ldots{} & HBTAR0\ldots{} & AKT1 & NA & HBREF0\ldots{} & \ldots{}\tabularnewline
HBIN01\ldots{} & HUANG QI & calycosin & HSDB 8\ldots{} & HBTAR0\ldots{} & MAPK1 & NA & HBREF0\ldots{} & \ldots{}\tabularnewline
HBIN01\ldots{} & HUANG QI & calycosin & HSDB 8\ldots{} & HBTAR0\ldots{} & AKT1 & NA & HBREF0\ldots{} & \ldots{}\tabularnewline
HBIN01\ldots{} & HUANG QI & calycosin & HSDB 8\ldots{} & HBTAR0\ldots{} & MAPK1 & NA & HBREF0\ldots{} & \ldots{}\tabularnewline
\ldots{} & \ldots{} & \ldots{} & \ldots{} & \ldots{} & \ldots{} & \ldots{} & \ldots{} & \ldots{}\tabularnewline
\bottomrule
\end{longtable}

\hypertarget{hi-t}{%
\subsubsection{Hirudin 对筛选通路的靶向}\label{hi-t}}

Figure \ref{fig:HIRUDIN-hsa04066-visualization} (下方图) 为图HIRUDIN hsa04066 visualization概览。

\textbf{(对应文件为 \texttt{Figure+Table/hsa04066.pathview.png})}

\def\@captype{figure}
\begin{center}
\includegraphics[width = 0.9\linewidth]{pathview2024-03-14_16_15_48.263803/hsa04066.pathview.png}
\caption{HIRUDIN hsa04066 visualization}\label{fig:HIRUDIN-hsa04066-visualization}
\end{center}
\begin{center}\begin{tcolorbox}[colback=gray!10, colframe=gray!50, width=0.9\linewidth, arc=1mm, boxrule=0.5pt]
\textbf{
Interactive figure
:}

\vspace{0.5em}

    \url{https://www.genome.jp/pathway/hsa04066}

\vspace{2em}
\end{tcolorbox}
\end{center}

Figure \ref{fig:HIRUDIN-hsa04371-visualization} (下方图) 为图HIRUDIN hsa04371 visualization概览。

\textbf{(对应文件为 \texttt{Figure+Table/hsa04371.pathview.png})}

\def\@captype{figure}
\begin{center}
\includegraphics[width = 0.9\linewidth]{pathview2024-03-14_16_15_48.263803/hsa04371.pathview.png}
\caption{HIRUDIN hsa04371 visualization}\label{fig:HIRUDIN-hsa04371-visualization}
\end{center}
\begin{center}\begin{tcolorbox}[colback=gray!10, colframe=gray!50, width=0.9\linewidth, arc=1mm, boxrule=0.5pt]
\textbf{
Interactive figure
:}

\vspace{0.5em}

    \url{https://www.genome.jp/pathway/hsa04371}

\vspace{2em}
\end{tcolorbox}
\end{center}

\hypertarget{bibliography}{%
\section*{Reference}\label{bibliography}}
\addcontentsline{toc}{section}{Reference}

\hypertarget{refs}{}
\begin{cslreferences}
\leavevmode\hypertarget{ref-BindingdbIn20Gilson2016}{}%
1. Gilson, M. K. \emph{et al.} BindingDB in 2015: A public database for medicinal chemistry, computational chemistry and systems pharmacology. \emph{Nucleic acids research} \textbf{44}, D1045--D1053 (2016).

\leavevmode\hypertarget{ref-MappingIdentifDurinc2009}{}%
2. Durinck, S., Spellman, P. T., Birney, E. \& Huber, W. Mapping identifiers for the integration of genomic datasets with the r/bioconductor package biomaRt. \emph{Nature protocols} \textbf{4}, 1184--1191 (2009).

\leavevmode\hypertarget{ref-ClusterprofilerWuTi2021}{}%
3. Wu, T. \emph{et al.} ClusterProfiler 4.0: A universal enrichment tool for interpreting omics data. \emph{The Innovation} \textbf{2}, (2021).

\leavevmode\hypertarget{ref-TheDisgenetKnPinero2019}{}%
4. Piñero, J. \emph{et al.} The disgenet knowledge platform for disease genomics: 2019 update. \emph{Nucleic Acids Research} (2019) doi:\href{https://doi.org/10.1093/nar/gkz1021}{10.1093/nar/gkz1021}.

\leavevmode\hypertarget{ref-TheGenecardsSStelze2016}{}%
5. Stelzer, G. \emph{et al.} The genecards suite: From gene data mining to disease genome sequence analyses. \emph{Current protocols in bioinformatics} \textbf{54}, 1.30.1--1.30.33 (2016).

\leavevmode\hypertarget{ref-PharmgkbAWorBarbar2018}{}%
6. Barbarino, J. M., Whirl-Carrillo, M., Altman, R. B. \& Klein, T. E. PharmGKB: A worldwide resource for pharmacogenomic information. \emph{Wiley interdisciplinary reviews. Systems biology and medicine} \textbf{10}, (2018).

\leavevmode\hypertarget{ref-HerbAHighThFang2021}{}%
7. Fang, S. \emph{et al.} HERB: A high-throughput experiment- and reference-guided database of traditional chinese medicine. \emph{Nucleic Acids Research} \textbf{49}, D1197--D1206 (2021).

\leavevmode\hypertarget{ref-LimmaPowersDiRitchi2015}{}%
8. Ritchie, M. E. \emph{et al.} Limma powers differential expression analyses for rna-sequencing and microarray studies. \emph{Nucleic Acids Research} \textbf{43}, e47 (2015).

\leavevmode\hypertarget{ref-EdgerDifferenChen}{}%
9. Chen, Y., McCarthy, D., Ritchie, M., Robinson, M. \& Smyth, G. EdgeR: Differential analysis of sequence read count data user's guide. 119.

\leavevmode\hypertarget{ref-TheStringDataSzklar2021}{}%
10. Szklarczyk, D. \emph{et al.} The string database in 2021: Customizable proteinprotein networks, and functional characterization of user-uploaded gene/measurement sets. \emph{Nucleic Acids Research} \textbf{49}, D605--D612 (2021).

\leavevmode\hypertarget{ref-CytohubbaIdenChin2014}{}%
11. Chin, C.-H. \emph{et al.} CytoHubba: Identifying hub objects and sub-networks from complex interactome. \emph{BMC Systems Biology} \textbf{8}, S11 (2014).

\leavevmode\hypertarget{ref-SuperpredUpdaNickel2014}{}%
12. Nickel, J. \emph{et al.} SuperPred: Update on drug classification and target prediction. \emph{Nucleic acids research} \textbf{42}, W26--W31 (2014).
\end{cslreferences}

\end{document}
