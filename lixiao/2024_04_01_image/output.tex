% Options for packages loaded elsewhere
\PassOptionsToPackage{unicode}{hyperref}
\PassOptionsToPackage{hyphens}{url}
%
\documentclass[
]{article}
\usepackage{lmodern}
\usepackage{amssymb,amsmath}
\usepackage{ifxetex,ifluatex}
\ifnum 0\ifxetex 1\fi\ifluatex 1\fi=0 % if pdftex
  \usepackage[T1]{fontenc}
  \usepackage[utf8]{inputenc}
  \usepackage{textcomp} % provide euro and other symbols
\else % if luatex or xetex
  \usepackage{unicode-math}
  \defaultfontfeatures{Scale=MatchLowercase}
  \defaultfontfeatures[\rmfamily]{Ligatures=TeX,Scale=1}
\fi
% Use upquote if available, for straight quotes in verbatim environments
\IfFileExists{upquote.sty}{\usepackage{upquote}}{}
\IfFileExists{microtype.sty}{% use microtype if available
  \usepackage[]{microtype}
  \UseMicrotypeSet[protrusion]{basicmath} % disable protrusion for tt fonts
}{}
\makeatletter
\@ifundefined{KOMAClassName}{% if non-KOMA class
  \IfFileExists{parskip.sty}{%
    \usepackage{parskip}
  }{% else
    \setlength{\parindent}{0pt}
    \setlength{\parskip}{6pt plus 2pt minus 1pt}}
}{% if KOMA class
  \KOMAoptions{parskip=half}}
\makeatother
\usepackage{xcolor}
\IfFileExists{xurl.sty}{\usepackage{xurl}}{} % add URL line breaks if available
\IfFileExists{bookmark.sty}{\usepackage{bookmark}}{\usepackage{hyperref}}
\hypersetup{
  hidelinks,
  pdfcreator={LaTeX via pandoc}}
\urlstyle{same} % disable monospaced font for URLs
\usepackage[margin=1in]{geometry}
\usepackage{longtable,booktabs}
% Correct order of tables after \paragraph or \subparagraph
\usepackage{etoolbox}
\makeatletter
\patchcmd\longtable{\par}{\if@noskipsec\mbox{}\fi\par}{}{}
\makeatother
% Allow footnotes in longtable head/foot
\IfFileExists{footnotehyper.sty}{\usepackage{footnotehyper}}{\usepackage{footnote}}
\makesavenoteenv{longtable}
\usepackage{graphicx}
\makeatletter
\def\maxwidth{\ifdim\Gin@nat@width>\linewidth\linewidth\else\Gin@nat@width\fi}
\def\maxheight{\ifdim\Gin@nat@height>\textheight\textheight\else\Gin@nat@height\fi}
\makeatother
% Scale images if necessary, so that they will not overflow the page
% margins by default, and it is still possible to overwrite the defaults
% using explicit options in \includegraphics[width, height, ...]{}
\setkeys{Gin}{width=\maxwidth,height=\maxheight,keepaspectratio}
% Set default figure placement to htbp
\makeatletter
\def\fps@figure{htbp}
\makeatother
\setlength{\emergencystretch}{3em} % prevent overfull lines
\providecommand{\tightlist}{%
  \setlength{\itemsep}{0pt}\setlength{\parskip}{0pt}}
\setcounter{secnumdepth}{5}
\usepackage{tikz} \usepackage{auto-pst-pdf} \usepackage{pgfornament} \usepackage{pstricks-add} \usepackage{caption} \captionsetup{font={footnotesize},width=6in} \renewcommand{\dblfloatpagefraction}{.9} \makeatletter \renewenvironment{figure} {\def\@captype{figure}} \makeatother \@ifundefined{Shaded}{\newenvironment{Shaded}} \@ifundefined{snugshade}{\newenvironment{snugshade}} \renewenvironment{Shaded}{\begin{snugshade}}{\end{snugshade}} \definecolor{shadecolor}{RGB}{230,230,230} \usepackage{xeCJK} \usepackage{setspace} \setstretch{1.3} \usepackage{tcolorbox} \setcounter{secnumdepth}{4} \setcounter{tocdepth}{4} \usepackage{wallpaper} \usepackage[absolute]{textpos} \tcbuselibrary{breakable} \renewenvironment{Shaded} {\begin{tcolorbox}[colback = gray!10, colframe = gray!40, width = 16cm, arc = 1mm, auto outer arc, title = {R input}]} {\end{tcolorbox}} \usepackage{titlesec} \titleformat{\paragraph} {\fontsize{10pt}{0pt}\bfseries} {\arabic{section}.\arabic{subsection}.\arabic{subsubsection}.\arabic{paragraph}} {1em} {} []

\author{}
\date{\vspace{-2.5em}}

\begin{document}

\begin{titlepage} \newgeometry{top=7.5cm}
\ThisCenterWallPaper{1.12}{~/outline/lixiao//cover_page.pdf}
\begin{center} \textbf{\Huge 图片查重程序}
\vspace{4em} \begin{textblock}{10}(3,5.9) \huge
\textbf{\textcolor{white}{2024-06-27}}
\end{textblock} \begin{textblock}{10}(3,7.3)
\Large \textcolor{black}{LiChuang Huang}
\end{textblock} \begin{textblock}{10}(3,11.3)
\Large \textcolor{black}{@立效研究院}
\end{textblock} \end{center} \end{titlepage}
\restoregeometry

\pagenumbering{roman}

\begin{center}\vspace{1.5cm}\pgfornament[anchor=center,ydelta=0pt,width=8cm]{84}\end{center}\tableofcontents

\begin{center}\vspace{1.5cm}\pgfornament[anchor=center,ydelta=0pt,width=8cm]{88}\end{center}\listoffigures

\begin{center}\vspace{1.5cm}\pgfornament[anchor=center,ydelta=0pt,width=8cm]{89}\end{center}\listoftables

\newpage

\pagenumbering{arabic}

\hypertarget{abstract}{%
\section{摘要}\label{abstract}}

Findsimilar 可递归搜索文件夹下所有图片 (没有明确的数量上限, 但最少 10 张图片),根据设定的阈值 (threshold 参数),寻找相似图片,
最后将结果以网页报告的形式输出。

步骤 1:选择搜索目录 (需要搜索的路径)。
步骤 2:选择输出目录 (生成的报告文件和其他分析数据存放)。
步骤 3:点击 ``Run''。
步骤 4:点击 ``Similarity Gallery'',在网页浏览器中显示报告。

注意:
支持的格式:.png, .jpg, .jpeg, .gif, .giff, .tif, .tiff, .heic, .heif, .bmp, .webp, .jfif

注:以下面板示例为 Linux 系统的界面,Windows 下略有不同。

\begin{center}\vspace{1.5cm}\pgfornament[anchor=center,ydelta=0pt,width=9cm]{88}\end{center}

Figure \ref{fig:Panel} (下方图) 为图Panel概览。

\textbf{(对应文件为 \texttt{Figure+Table/Panel.png})}

\def\@captype{figure}
\begin{center}
\includegraphics[width = 0.9\linewidth]{/home/echo/Pictures/Screenshots/Screenshot from 2024-06-27 16-14-44.png}
\caption{Panel}\label{fig:Panel}
\end{center}

\begin{center}\pgfornament[anchor=center,ydelta=0pt,width=9cm]{88}\vspace{1.5cm}\end{center}

\hypertarget{ux5b89ux88c5}{%
\section{安装}\label{ux5b89ux88c5}}

\hypertarget{python-3.9.0}{%
\subsection{Python 3.9.0}\label{python-3.9.0}}

请下载并安装 Python 3.9.0:

\url{https://www.python.org/ftp/python/3.9.0/python-3.9.0-amd64.exe}

\hypertarget{findsimilar}{%
\subsection{findsimilar}\label{findsimilar}}

\hypertarget{option-1-ux8fd0ux884c-bat-ux6587ux4ef6ux5b89ux88c5-ux672aux6d4bux8bd5}{%
\subsubsection{(Option 1) 运行 bat 文件安装 (未测试)}\label{option-1-ux8fd0ux884c-bat-ux6587ux4ef6ux5b89ux88c5-ux672aux6d4bux8bd5}}

请确认已安装完成 Python, 然后打开文件夹

\begin{center}\pgfornament[anchor=center,ydelta=0pt,width=9cm]{85}\vspace{1.5cm}\end{center}

`All files' 数据已全部提供。

\textbf{(对应文件为 \texttt{./use\_for\_install/})}

\begin{center}\begin{tcolorbox}[colback=gray!10, colframe=gray!50, width=0.9\linewidth, arc=1mm, boxrule=0.5pt]注:文件夹./use\_for\_install/共包含3个文件。

\begin{enumerate}\tightlist
\item findsimilar
\item get\_shortcut.bat
\item install.bat
\end{enumerate}\end{tcolorbox}
\end{center}

\begin{center}\pgfornament[anchor=center,ydelta=0pt,width=9cm]{85}\vspace{1.5cm}\end{center}

\begin{enumerate}
\def\labelenumi{\arabic{enumi}.}
\tightlist
\item
  双击 \texttt{install.bat}, 等待安装完成。
\item
  双击 \texttt{get\_shortcut.bat}, 这会在桌面生成 \texttt{findsimilar.bat}。
\item
  完成。
\end{enumerate}

\hypertarget{option-2-ux901aux8fc7ux547dux4ee4ux5b89ux88c5}{%
\subsubsection{(Option 2) 通过命令安装}\label{option-2-ux901aux8fc7ux547dux4ee4ux5b89ux88c5}}

按 \texttt{Win\ +\ R}, 输入 \texttt{cmd}, 确认打开 \texttt{cmd} 界面,输入以下安装。

\begin{tcolorbox}[colback = gray!10, colframe = red!50, width = 16cm, arc = 1mm, auto outer arc, title = {cmd input}]
\begin{verbatim}

# 请确认 cmd 已经切换到 findsimilar 安装包的路径
pip install findsimilar -i https://pypi.tuna.tsinghua.edu.cn/simple

\end{verbatim}
\end{tcolorbox}

\begin{tcolorbox}[colback = gray!10, colframe = red!50, width = 16cm, arc = 1mm, auto outer arc, title = {cmd input}]
\begin{verbatim}

echo $(which findsimilar) > "%USERPROFILE%\Desktop\findsimilar.bat"

\end{verbatim}
\end{tcolorbox}

\hypertarget{ux4f7fux7528ux793aux4f8b}{%
\section{使用示例}\label{ux4f7fux7528ux793aux4f8b}}

可以使用如下文件夹作为测试。

\begin{center}\pgfornament[anchor=center,ydelta=0pt,width=9cm]{85}\vspace{1.5cm}\end{center}

`Test files' 数据已全部提供。

\textbf{(对应文件为 \texttt{./test/})}

\begin{center}\begin{tcolorbox}[colback=gray!10, colframe=gray!50, width=0.9\linewidth, arc=1mm, boxrule=0.5pt]注:文件夹./test/共包含10个文件。

\begin{enumerate}\tightlist
\item Control-1.tif
\item Control.tif
\item DAPIPKH67.tif
\item Figure 4 revise.tif
\item Figure 5 revise.tif
\item ...
\end{enumerate}\end{tcolorbox}
\end{center}

\begin{center}\pgfornament[anchor=center,ydelta=0pt,width=9cm]{85}\vspace{1.5cm}\end{center}

新建一个空的文件夹作为输出目录。

\begin{center}\vspace{1.5cm}\pgfornament[anchor=center,ydelta=0pt,width=9cm]{88}\end{center}

Figure \ref{fig:unnamed-chunk-11} (下方图) 为图unnamed chunk 11概览。

\textbf{(对应文件为 \texttt{Figure+Table/unnamed-chunk-11.png})}

\def\@captype{figure}
\begin{center}
\includegraphics[width = 0.9\linewidth]{/home/echo/Pictures/Screenshots/Screenshot from 2024-06-27 16-43-27.png}
\caption{Unnamed chunk 11}\label{fig:unnamed-chunk-11}
\end{center}

\begin{center}\pgfornament[anchor=center,ydelta=0pt,width=9cm]{88}\vspace{1.5cm}\end{center}

如上运行完成后,点击 ``Similarity Gallery'' ,得到 HTML 报告。

\begin{center}\vspace{1.5cm}\pgfornament[anchor=center,ydelta=0pt,width=9cm]{88}\end{center}

Figure \ref{fig:unnamed-chunk-12} (下方图) 为图unnamed chunk 12概览。

\textbf{(对应文件为 \texttt{Figure+Table/unnamed-chunk-12.png})}

\def\@captype{figure}
\begin{center}
\includegraphics[width = 0.9\linewidth]{/home/echo/Pictures/Screenshots/Screenshot from 2024-06-27 16-48-30.png}
\caption{Unnamed chunk 12}\label{fig:unnamed-chunk-12}
\end{center}

\begin{center}\pgfornament[anchor=center,ydelta=0pt,width=9cm]{88}\vspace{1.5cm}\end{center}

\end{document}
