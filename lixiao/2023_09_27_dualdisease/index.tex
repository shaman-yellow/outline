% Options for packages loaded elsewhere
\PassOptionsToPackage{unicode}{hyperref}
\PassOptionsToPackage{hyphens}{url}
%
\documentclass[
  ignorenonframetext,
]{beamer}
\usepackage{pgfpages}
\setbeamertemplate{caption}[numbered]
\setbeamertemplate{caption label separator}{: }
\setbeamercolor{caption name}{fg=normal text.fg}
\beamertemplatenavigationsymbolsempty
% Prevent slide breaks in the middle of a paragraph
\widowpenalties 1 10000
\raggedbottom
\setbeamertemplate{part page}{
  \centering
  \begin{beamercolorbox}[sep=16pt,center]{part title}
    \usebeamerfont{part title}\insertpart\par
  \end{beamercolorbox}
}
\setbeamertemplate{section page}{
  \centering
  \begin{beamercolorbox}[sep=12pt,center]{part title}
    \usebeamerfont{section title}\insertsection\par
  \end{beamercolorbox}
}
\setbeamertemplate{subsection page}{
  \centering
  \begin{beamercolorbox}[sep=8pt,center]{part title}
    \usebeamerfont{subsection title}\insertsubsection\par
  \end{beamercolorbox}
}
\AtBeginPart{
  \frame{\partpage}
}
\AtBeginSection{
  \ifbibliography
  \else
    \frame{\sectionpage}
  \fi
}
\AtBeginSubsection{
  \frame{\subsectionpage}
}
\usepackage{lmodern}
\usepackage{amssymb,amsmath}
\usepackage{ifxetex,ifluatex}
\ifnum 0\ifxetex 1\fi\ifluatex 1\fi=0 % if pdftex
  \usepackage[T1]{fontenc}
  \usepackage[utf8]{inputenc}
  \usepackage{textcomp} % provide euro and other symbols
\else % if luatex or xetex
  \usepackage{unicode-math}
  \defaultfontfeatures{Scale=MatchLowercase}
  \defaultfontfeatures[\rmfamily]{Ligatures=TeX,Scale=1}
\fi
% Use upquote if available, for straight quotes in verbatim environments
\IfFileExists{upquote.sty}{\usepackage{upquote}}{}
\IfFileExists{microtype.sty}{% use microtype if available
  \usepackage[]{microtype}
  \UseMicrotypeSet[protrusion]{basicmath} % disable protrusion for tt fonts
}{}
\makeatletter
\@ifundefined{KOMAClassName}{% if non-KOMA class
  \IfFileExists{parskip.sty}{%
    \usepackage{parskip}
  }{% else
    \setlength{\parindent}{0pt}
    \setlength{\parskip}{6pt plus 2pt minus 1pt}}
}{% if KOMA class
  \KOMAoptions{parskip=half}}
\makeatother
\usepackage{xcolor}
\IfFileExists{xurl.sty}{\usepackage{xurl}}{} % add URL line breaks if available
\IfFileExists{bookmark.sty}{\usepackage{bookmark}}{\usepackage{hyperref}}
\hypersetup{
  hidelinks,
  pdfcreator={LaTeX via pandoc}}
\urlstyle{same} % disable monospaced font for URLs
\newif\ifbibliography
\setlength{\emergencystretch}{3em} % prevent overfull lines
\providecommand{\tightlist}{%
  \setlength{\itemsep}{0pt}\setlength{\parskip}{0pt}}
\setcounter{secnumdepth}{-\maxdimen} % remove section numbering

\author{}
\date{\vspace{-2.5em}}

\begin{document}

\begin{frame}
\listoffigures

\listoftables
\end{frame}

\begin{frame}{题目}
\protect\hypertarget{ux9898ux76ee}{}
\end{frame}

\begin{frame}{摘要}
\protect\hypertarget{abstract}{}
\end{frame}

\begin{frame}{前言}
\protect\hypertarget{introduction}{}
CKD 与患癌症风险之间的联系尚未明确。尽管多项研究观察到需要透析或肾移植的
ESRD
患者患癌症的风险较高,但不太严重的肾脏疾病是否与癌症相关仍知之甚少{[}@OnconephrologyRosner2021;
@CkdAndTheRisLowran2014; @CancerRiskAndKitchl2022{]}。

肾癌流行病学{[}@EpidemiologyOfBukavi2022{]}

CKD 和肾癌{[}@RenalCellCancSaly2021{]}

Cancer Risk and Mortality in Patients With Kidney Disease: Cancer risk
was increased in mild to moderate CKD and among transplant
recipients{[}@CancerRiskAndKitchl2022{]}
\end{frame}

\begin{frame}{研究设计流程图}
\protect\hypertarget{route}{}
TCGA 突变数据 TCGA 转录数据 基因共表达 免疫浸润水平 生存分析 timeROC
单细胞数据验证

多种慢性肾病的 FC 癌症的 FC 用基因集的 FC 来关联分析 (需要多个 GEO
数据集)

\begin{itemize}
\tightlist
\item
  Kidney status markers
\item
  cancer markers
\item
  micro-envir\ldots{}
\item
  ferroptosis
\item
  \ldots{}
\end{itemize}
\end{frame}

\begin{frame}{材料和方法}
\protect\hypertarget{methods}{}
\end{frame}

\begin{frame}{分析结果}
\protect\hypertarget{results}{}
\end{frame}

\begin{frame}{结论}
\protect\hypertarget{dis}{}
\end{frame}

\begin{frame}[fragile]{附:分析流程}
\protect\hypertarget{ux9644ux5206ux6790ux6d41ux7a0b}{}
\begin{block}{肾癌 (GEO)}
\protect\hypertarget{ux80beux764c-geo}{}
\begin{block}{GSE171306}
\protect\hypertarget{gse171306}{}
\begin{verbatim}
- Single-cell RNA sequencing (scRNA-seq) was performed on bilateral clear
  cell RCC (ccRCC). Primary kidney samples from 3 patients were used for
  single cell RNA sequencing by 10X Genomics
\end{verbatim}

\begin{block}{细胞聚类和注释}
\protect\hypertarget{ux7ec6ux80deux805aux7c7bux548cux6ce8ux91ca}{}
\end{block}

\begin{block}{癌细胞识别}
\protect\hypertarget{ux764cux7ec6ux80deux8bc6ux522b}{}
\end{block}

\begin{block}{癌细胞拟时分析}
\protect\hypertarget{ux764cux7ec6ux80deux62dfux65f6ux5206ux6790}{}
\end{block}
\end{block}
\end{block}

\begin{block}{肾癌 (TCGA-KIRC)}
\protect\hypertarget{ux80beux764c-tcga-kirc}{}
\begin{block}{TCGA 数据}
\protect\hypertarget{tcga-ux6570ux636e}{}
\end{block}
\end{block}

\begin{block}{血管性疾患}
\protect\hypertarget{ux8840ux7ba1ux6027ux75beux60a3}{}
\begin{block}{高血压性肾炎: GSE210898 (hypertensive nephropathy)}
\protect\hypertarget{ux9ad8ux8840ux538bux6027ux80beux708e-gse210898-hypertensive-nephropathy}{}
\begin{itemize}
\tightlist
\item
  Single-cell RNA transcriptomics of hypertensive nephropathy patients.
  We analyzed kidney samples from 3 patients with HTN using single-cell
  RNA sequencing, compared with previous data of controls
\end{itemize}

\begin{block}{细胞聚类和注释}
\protect\hypertarget{ux7ec6ux80deux805aux7c7bux548cux6ce8ux91ca-1}{}
\end{block}
\end{block}
\end{block}

\begin{block}{(原发性)肾小球性疾患}
\protect\hypertarget{ux539fux53d1ux6027ux80beux5c0fux7403ux6027ux75beux60a3}{}
\begin{block}{IgA 肾病: GSE171314 (IgA Nephropathy)}
\protect\hypertarget{iga-ux80beux75c5-gse171314-iga-nephropathy}{}
\begin{itemize}
\item
  Single-cell RNA sequencing (scRNA-seq) was applied to kidney biopsies
  from 4 IgAN and 1 control subjects to define the transcriptomic
  landscape at the single-cell resolution.
\item
  GSE145652, RNA-seq
\item
  GSE175759, RNA-seq
\end{itemize}

\begin{block}{细胞聚类和注释}
\protect\hypertarget{ux7ec6ux80deux805aux7c7bux548cux6ce8ux91ca-2}{}
\end{block}
\end{block}

\begin{block}{膜性肾病: GSE241302 (idiopathic membranous nephropathy)}
\protect\hypertarget{ux819cux6027ux80beux75c5-gse241302-idiopathic-membranous-nephropathy}{}
\begin{itemize}
\item
  In order to explore the molecular mechanism of IMN, we collected renal
  tissue samples from 3 IMN patients and 1 healthy controls and
  performed analysis by single-cell RNA sequencing.
\item
  GSE241302, scRNA
\item
  GSE216841, RNA-seq
\item
  GSE175759, RNA-seq
\end{itemize}

\begin{block}{细胞聚类和注释}
\protect\hypertarget{ux7ec6ux80deux805aux7c7bux548cux6ce8ux91ca-3}{}
\end{block}
\end{block}

\begin{block}{微小病变肾病 (Minimal change disease)}
\protect\hypertarget{ux5faeux5c0fux75c5ux53d8ux80beux75c5-minimal-change-disease}{}
\begin{itemize}
\tightlist
\item
  GSE176465, scRNA (only but not fit)
\item
  GSE216841, RNA-seq
\item
  GSE175759, RNA-seq
\end{itemize}
\end{block}
\end{block}

\begin{block}{(继发性)肾小球性疾患}
\protect\hypertarget{ux7ee7ux53d1ux6027ux80beux5c0fux7403ux6027ux75beux60a3}{}
\begin{block}{糖尿病肾病 (Diabetic kidney disease)}
\protect\hypertarget{ux7cd6ux5c3fux75c5ux80beux75c5-diabetic-kidney-disease}{}
\begin{itemize}
\tightlist
\item
  GSE204880, scRNA
\item
  GSE175759, RNA-seq
\item
  GSE199838, RNA-seq
\item
  GSE217709, RNA-seq
\end{itemize}
\end{block}
\end{block}
\end{frame}

\end{document}
