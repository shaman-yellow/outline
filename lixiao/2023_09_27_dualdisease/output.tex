% Options for packages loaded elsewhere
\PassOptionsToPackage{unicode}{hyperref}
\PassOptionsToPackage{hyphens}{url}
%
\documentclass[
]{article}
\usepackage{lmodern}
\usepackage{amssymb,amsmath}
\usepackage{ifxetex,ifluatex}
\ifnum 0\ifxetex 1\fi\ifluatex 1\fi=0 % if pdftex
  \usepackage[T1]{fontenc}
  \usepackage[utf8]{inputenc}
  \usepackage{textcomp} % provide euro and other symbols
\else % if luatex or xetex
  \usepackage{unicode-math}
  \defaultfontfeatures{Scale=MatchLowercase}
  \defaultfontfeatures[\rmfamily]{Ligatures=TeX,Scale=1}
\fi
% Use upquote if available, for straight quotes in verbatim environments
\IfFileExists{upquote.sty}{\usepackage{upquote}}{}
\IfFileExists{microtype.sty}{% use microtype if available
  \usepackage[]{microtype}
  \UseMicrotypeSet[protrusion]{basicmath} % disable protrusion for tt fonts
}{}
\makeatletter
\@ifundefined{KOMAClassName}{% if non-KOMA class
  \IfFileExists{parskip.sty}{%
    \usepackage{parskip}
  }{% else
    \setlength{\parindent}{0pt}
    \setlength{\parskip}{6pt plus 2pt minus 1pt}}
}{% if KOMA class
  \KOMAoptions{parskip=half}}
\makeatother
\usepackage{xcolor}
\IfFileExists{xurl.sty}{\usepackage{xurl}}{} % add URL line breaks if available
\IfFileExists{bookmark.sty}{\usepackage{bookmark}}{\usepackage{hyperref}}
\hypersetup{
  pdftitle={Analysis},
  pdfauthor={Huang LiChuang of Wie-Biotech},
  hidelinks,
  pdfcreator={LaTeX via pandoc}}
\urlstyle{same} % disable monospaced font for URLs
\usepackage[margin=1in]{geometry}
\usepackage{longtable,booktabs}
% Correct order of tables after \paragraph or \subparagraph
\usepackage{etoolbox}
\makeatletter
\patchcmd\longtable{\par}{\if@noskipsec\mbox{}\fi\par}{}{}
\makeatother
% Allow footnotes in longtable head/foot
\IfFileExists{footnotehyper.sty}{\usepackage{footnotehyper}}{\usepackage{footnote}}
\makesavenoteenv{longtable}
\usepackage{graphicx}
\makeatletter
\def\maxwidth{\ifdim\Gin@nat@width>\linewidth\linewidth\else\Gin@nat@width\fi}
\def\maxheight{\ifdim\Gin@nat@height>\textheight\textheight\else\Gin@nat@height\fi}
\makeatother
% Scale images if necessary, so that they will not overflow the page
% margins by default, and it is still possible to overwrite the defaults
% using explicit options in \includegraphics[width, height, ...]{}
\setkeys{Gin}{width=\maxwidth,height=\maxheight,keepaspectratio}
% Set default figure placement to htbp
\makeatletter
\def\fps@figure{htbp}
\makeatother
\setlength{\emergencystretch}{3em} % prevent overfull lines
\providecommand{\tightlist}{%
  \setlength{\itemsep}{0pt}\setlength{\parskip}{0pt}}
\setcounter{secnumdepth}{5}
\usepackage{caption} \captionsetup{font={footnotesize},width=6in} \renewcommand{\dblfloatpagefraction}{.9} \makeatletter \renewenvironment{figure} {\def\@captype{figure}} \makeatother \definecolor{shadecolor}{RGB}{242,242,242} \usepackage{xeCJK} \usepackage{setspace} \setstretch{1.3} \usepackage{tcolorbox}
\newlength{\cslhangindent}
\setlength{\cslhangindent}{1.5em}
\newenvironment{cslreferences}%
  {}%
  {\par}

\title{Analysis}
\author{Huang LiChuang of Wie-Biotech}
\date{}

\begin{document}
\maketitle

{
\setcounter{tocdepth}{4}
\tableofcontents
}
\listoffigures

\listoftables

\hypertarget{ux9898ux76ee}{%
\section{题目}\label{ux9898ux76ee}}

\hypertarget{abstract}{%
\section{摘要}\label{abstract}}

\hypertarget{introduction}{%
\section{前言}\label{introduction}}

慢性肾病(CKD)与患癌症风险之间的联系尚未明确。尽管多项研究观察到需要透析或肾移植的 ESRD 患者患癌症的风险较高,但相对不严重的肾脏疾病是否与癌症相关仍知之甚少\textsuperscript{\protect\hyperlink{ref-OnconephrologyRosner2021}{1}--\protect\hyperlink{ref-CancerRiskAndKitchl2022}{3}}。已有研究论及 CKD 和肾癌之间的关联性和转化的风险\textsuperscript{\protect\hyperlink{ref-CkdAndTheRisLowran2014}{2}--\protect\hyperlink{ref-RenalCellCancSaly2021}{4}},轻度至中度 CKD 和移植受者的癌症风险增加\textsuperscript{\protect\hyperlink{ref-CancerRiskAndKitchl2022}{3}}。

在本研究中,为了探究 CKD 与肾癌(RCC)之间的相关性和转化风险,重新分析了多组公共数据库的单细胞数据集。考虑到 CKD 的复杂性,这里并不采用单一的 CKD 病型研究,而是搜集了不同类型的 CKD(hypertensive nephropathy,HN;IgA;idiopathic membranous nephropathy,IMN),并采用了强大的单细胞数据集成算法 RISC 消除不同来源(不同实验室、不同批次、不同类型患者等不相关因素)批次效应,试探索 CKD 与肾癌之间的相关性。并在此之后,以一批新的 CKD 和 RCC 单细胞数据集验证。

\hypertarget{methods}{%
\section{材料和方法}\label{methods}}

涉及的 6 个 GEO 数据集可参考 \ref{flow} 的次级标题。

涉及的方法:

\begin{itemize}
\tightlist
\item
  Seurat\textsuperscript{\protect\hyperlink{ref-IntegratedAnalHaoY2021}{5},\protect\hyperlink{ref-ComprehensiveIStuart2019}{6}}
\item
  Monocle\textsuperscript{\protect\hyperlink{ref-ReversedGraphQiuX2017}{7},\protect\hyperlink{ref-TheDynamicsAnTrapne2014}{8}}
\item
  copyKAT\textsuperscript{\protect\hyperlink{ref-DelineatingCopGaoR2021}{9}}
\item
  RISC\textsuperscript{\protect\hyperlink{ref-RobustIntegratLiuY2021}{10}}
\item
  ClusterProfiler\textsuperscript{\protect\hyperlink{ref-ClusterprofilerWuTi2021}{11}}
\item
  \ldots{}
\end{itemize}

\hypertarget{results}{%
\section{分析结果}\label{results}}

\hypertarget{rcc-ux6570ux636eux96c6ux5206ux6790ux9274ux5b9a}{%
\subsection{RCC 数据集分析鉴定}\label{rcc-ux6570ux636eux96c6ux5206ux6790ux9274ux5b9a}}

Figure \ref{fig:Main-figure-1}为图Main figure 1概览。

\textbf{(对应文件为 \texttt{./Figure+Table/fig1.pdf})}

\def\@captype{figure}
\begin{center}
\includegraphics[width = 0.9\linewidth]{./Figure+Table/fig1.pdf}
\caption{Main figure 1}\label{fig:Main-figure-1}
\end{center}

\hypertarget{ux4e0dux540cux7c7bux578b-ckd-ux6570ux636eux5171ux8868ux8fbeux5206ux6790ux548cux91cdux805aux7c7b}{%
\subsection{不同类型 CKD 数据共表达分析和重聚类}\label{ux4e0dux540cux7c7bux578b-ckd-ux6570ux636eux5171ux8868ux8fbeux5206ux6790ux548cux91cdux805aux7c7b}}

Figure \ref{fig:Main-figure-2}为图Main figure 2概览。

\textbf{(对应文件为 \texttt{./Figure+Table/fig2.pdf})}

\def\@captype{figure}
\begin{center}
\includegraphics[width = 0.9\linewidth]{./Figure+Table/fig2.pdf}
\caption{Main figure 2}\label{fig:Main-figure-2}
\end{center}

\hypertarget{rcc-ux548c-ckd-ux6570ux636eux96c6ux6210}{%
\subsection{RCC 和 CKD 数据集成}\label{rcc-ux548c-ckd-ux6570ux636eux96c6ux6210}}

Figure \ref{fig:Main-figure-3}为图Main figure 3概览。

\textbf{(对应文件为 \texttt{./Figure+Table/fig3.pdf})}

\def\@captype{figure}
\begin{center}
\includegraphics[width = 0.9\linewidth]{./Figure+Table/fig3.pdf}
\caption{Main figure 3}\label{fig:Main-figure-3}
\end{center}

Sets 1-5 对应 RCC, HM, IgA, IMN, HC 单细胞数据集。

\hypertarget{ux5deeux5f02ux5206ux6790ux548cux5bccux96c6ux5206ux6790}{%
\subsection{差异分析和富集分析}\label{ux5deeux5f02ux5206ux6790ux548cux5bccux96c6ux5206ux6790}}

Figure \ref{fig:Main-figure-4}为图Main figure 4概览。

\textbf{(对应文件为 \texttt{./Figure+Table/fig4.pdf})}

\def\@captype{figure}
\begin{center}
\includegraphics[width = 0.9\linewidth]{./Figure+Table/fig4.pdf}
\caption{Main figure 4}\label{fig:Main-figure-4}
\end{center}

\hypertarget{ux4ee5ux65b0ux7684-iga-ux548c-rcc-ux6570ux636eux96c6ux9a8cux8bc1}{%
\subsection{以新的 IgA 和 RCC 数据集验证}\label{ux4ee5ux65b0ux7684-iga-ux548c-rcc-ux6570ux636eux96c6ux9a8cux8bc1}}

Figure \ref{fig:Main-figure-5}为图Main figure 5概览。

\textbf{(对应文件为 \texttt{./Figure+Table/fig5.pdf})}

\def\@captype{figure}
\begin{center}
\includegraphics[width = 0.9\linewidth]{./Figure+Table/fig5.pdf}
\caption{Main figure 5}\label{fig:Main-figure-5}
\end{center}

\hypertarget{dis}{%
\section{结论}\label{dis}}

\hypertarget{flow}{%
\section{附:分析流程}\label{flow}}

\hypertarget{rcc1}{%
\subsection{肾癌 (GEO)}\label{rcc1}}

\hypertarget{gse171306}{%
\subsubsection{GSE171306}\label{gse171306}}

\begin{verbatim}
- Single-cell RNA sequencing (scRNA-seq) was performed on bilateral clear
  cell RCC (ccRCC). Primary kidney samples from 3 patients were used for
  single cell RNA sequencing by 10X Genomics
\end{verbatim}

\hypertarget{ux7ec6ux80deux805aux7c7bux548cux6ce8ux91ca}{%
\paragraph{细胞聚类和注释}\label{ux7ec6ux80deux805aux7c7bux548cux6ce8ux91ca}}

Figure \ref{fig:RCC-data1-QC}为图RCC data1 QC概览。

\textbf{(对应文件为 \texttt{Figure+Table/RCC-data1-QC.pdf})}

\def\@captype{figure}
\begin{center}
\includegraphics[width = 0.9\linewidth]{Figure+Table/RCC-data1-QC.pdf}
\caption{RCC data1 QC}\label{fig:RCC-data1-QC}
\end{center}

使用 SCSA 对细胞类型注释(以下注释同)\textsuperscript{\protect\hyperlink{ref-ScsaACellTyCaoY2020}{12}}。

Figure \ref{fig:RCC-data1-SCSA-mapping}为图RCC data1 SCSA mapping概览。

\textbf{(对应文件为 \texttt{Figure+Table/RCC-data1-SCSA-mapping.pdf})}

\def\@captype{figure}
\begin{center}
\includegraphics[width = 0.9\linewidth]{Figure+Table/RCC-data1-SCSA-mapping.pdf}
\caption{RCC data1 SCSA mapping}\label{fig:RCC-data1-SCSA-mapping}
\end{center}

\hypertarget{ux764cux7ec6ux80deux8bc6ux522b}{%
\paragraph{癌细胞识别}\label{ux764cux7ec6ux80deux8bc6ux522b}}

使用 copyKAT 预测癌细胞\textsuperscript{\protect\hyperlink{ref-DelineatingCopGaoR2021}{9}}。

Figure \ref{fig:RCC-data1-copykat-prediction}为图RCC data1 copykat prediction概览。

\textbf{(对应文件为 \texttt{Figure+Table/copykat\_heatmap.png})}

\def\@captype{figure}
\begin{center}
\includegraphics[width = 0.9\linewidth]{./figs/copykat_heatmap.png}
\caption{RCC data1 copykat prediction}\label{fig:RCC-data1-copykat-prediction}
\end{center}

Figure \ref{fig:RCC-data1-SCSA-mapping-with-copykat-prediction}为图RCC data1 SCSA mapping with copykat prediction概览。

\textbf{(对应文件为 \texttt{Figure+Table/RCC-data1-SCSA-mapping-with-copykat-prediction.pdf})}

\def\@captype{figure}
\begin{center}
\includegraphics[width = 0.9\linewidth]{Figure+Table/RCC-data1-SCSA-mapping-with-copykat-prediction.pdf}
\caption{RCC data1 SCSA mapping with copykat prediction}\label{fig:RCC-data1-SCSA-mapping-with-copykat-prediction}
\end{center}

\hypertarget{ux764cux7ec6ux80deux62dfux65f6ux5206ux6790ux548cux5171ux8868ux8fbeux5206ux6790}{%
\paragraph{癌细胞拟时分析和共表达分析}\label{ux764cux7ec6ux80deux62dfux65f6ux5206ux6790ux548cux5171ux8868ux8fbeux5206ux6790}}

使用 monocle3 拟时分析\textsuperscript{\protect\hyperlink{ref-ReversedGraphQiuX2017}{7},\protect\hyperlink{ref-TheDynamicsAnTrapne2014}{8}}。

Figure \ref{fig:RCC-data1-pseudotime}为图RCC data1 pseudotime概览。

\textbf{(对应文件为 \texttt{Figure+Table/RCC-data1-pseudotime.pdf})}

\def\@captype{figure}
\begin{center}
\includegraphics[width = 0.9\linewidth]{Figure+Table/RCC-data1-pseudotime.pdf}
\caption{RCC data1 pseudotime}\label{fig:RCC-data1-pseudotime}
\end{center}

结合拟时分析和基因共表达模块,将肿瘤细胞类型切分为 5 个类型。

Figure \ref{fig:RCC-data1-co-expression-modules}为图RCC data1 co expression modules概览。

\textbf{(对应文件为 \texttt{Figure+Table/RCC-data1-co-expression-modules.pdf})}

\def\@captype{figure}
\begin{center}
\includegraphics[width = 0.9\linewidth]{Figure+Table/RCC-data1-co-expression-modules.pdf}
\caption{RCC data1 co expression modules}\label{fig:RCC-data1-co-expression-modules}
\end{center}

Figure \ref{fig:RCC-data1-cancer-subtype}为图RCC data1 cancer subtype概览。

\textbf{(对应文件为 \texttt{Figure+Table/RCC-data1-cancer-subtype.pdf})}

\def\@captype{figure}
\begin{center}
\includegraphics[width = 0.9\linewidth]{Figure+Table/RCC-data1-cancer-subtype.pdf}
\caption{RCC data1 cancer subtype}\label{fig:RCC-data1-cancer-subtype}
\end{center}

\hypertarget{ux8840ux7ba1ux6027ux75beux60a3}{%
\subsection{血管性疾患}\label{ux8840ux7ba1ux6027ux75beux60a3}}

\hypertarget{ux9ad8ux8840ux538bux6027ux80beux708e-gse210898-hypertensive-nephropathy}{%
\subsubsection{高血压性肾炎: GSE210898 (hypertensive nephropathy)}\label{ux9ad8ux8840ux538bux6027ux80beux708e-gse210898-hypertensive-nephropathy}}

\begin{itemize}
\tightlist
\item
  Single-cell RNA transcriptomics of hypertensive nephropathy patients. We
  analyzed kidney samples from 3 patients with HTN using single-cell RNA
  sequencing, compared with previous data of controls
\end{itemize}

\hypertarget{ux7ec6ux80deux805aux7c7bux548cux6ce8ux91ca-1}{%
\paragraph{细胞聚类和注释}\label{ux7ec6ux80deux805aux7c7bux548cux6ce8ux91ca-1}}

Figure \ref{fig:HN-data-QC}为图HN data QC概览。

\textbf{(对应文件为 \texttt{Figure+Table/HN-data-QC.pdf})}

\def\@captype{figure}
\begin{center}
\includegraphics[width = 0.9\linewidth]{Figure+Table/HN-data-QC.pdf}
\caption{HN data QC}\label{fig:HN-data-QC}
\end{center}

Figure \ref{fig:HN-data-SCSA-mapping}为图HN data SCSA mapping概览。

\textbf{(对应文件为 \texttt{Figure+Table/HN-data-SCSA-mapping.pdf})}

\def\@captype{figure}
\begin{center}
\includegraphics[width = 0.9\linewidth]{Figure+Table/HN-data-SCSA-mapping.pdf}
\caption{HN data SCSA mapping}\label{fig:HN-data-SCSA-mapping}
\end{center}

\hypertarget{ux5171ux8868ux8fbeux5206ux6790}{%
\paragraph{共表达分析}\label{ux5171ux8868ux8fbeux5206ux6790}}

Figure \ref{fig:HN-data-pseudotime}为图HN data pseudotime概览。

\textbf{(对应文件为 \texttt{Figure+Table/HN-data-pseudotime.pdf})}

\def\@captype{figure}
\begin{center}
\includegraphics[width = 0.9\linewidth]{Figure+Table/HN-data-pseudotime.pdf}
\caption{HN data pseudotime}\label{fig:HN-data-pseudotime}
\end{center}

Figure \ref{fig:HN-data-co-expression-modules}为图HN data co expression modules概览。

\textbf{(对应文件为 \texttt{Figure+Table/HN-data-co-expression-modules.pdf})}

\def\@captype{figure}
\begin{center}
\includegraphics[width = 0.9\linewidth]{Figure+Table/HN-data-co-expression-modules.pdf}
\caption{HN data co expression modules}\label{fig:HN-data-co-expression-modules}
\end{center}

Figure \ref{fig:HN-data-Proximal-tubule-cell-subtype}为图HN data Proximal tubule cell subtype概览。

\textbf{(对应文件为 \texttt{Figure+Table/HN-data-Proximal-tubule-cell-subtype.pdf})}

\def\@captype{figure}
\begin{center}
\includegraphics[width = 0.9\linewidth]{Figure+Table/HN-data-Proximal-tubule-cell-subtype.pdf}
\caption{HN data Proximal tubule cell subtype}\label{fig:HN-data-Proximal-tubule-cell-subtype}
\end{center}

\hypertarget{ux539fux53d1ux6027ux80beux5c0fux7403ux6027ux75beux60a3}{%
\subsection{(原发性)肾小球性疾患}\label{ux539fux53d1ux6027ux80beux5c0fux7403ux6027ux75beux60a3}}

\hypertarget{iga-ux80beux75c5-gse171314-iga-nephropathy}{%
\subsubsection{IgA 肾病: GSE171314 (IgA Nephropathy)}\label{iga-ux80beux75c5-gse171314-iga-nephropathy}}

\begin{itemize}
\tightlist
\item
  Single-cell RNA sequencing (scRNA-seq) was applied to kidney biopsies from 4
  IgAN and 1 control subjects to define the transcriptomic landscape at the
  single-cell resolution.
\end{itemize}

\hypertarget{ux7ec6ux80deux805aux7c7bux548cux6ce8ux91ca-2}{%
\paragraph{细胞聚类和注释}\label{ux7ec6ux80deux805aux7c7bux548cux6ce8ux91ca-2}}

Figure \ref{fig:IgA-data1-QC}为图IgA data1 QC概览。

\textbf{(对应文件为 \texttt{Figure+Table/IgA-data1-QC.pdf})}

\def\@captype{figure}
\begin{center}
\includegraphics[width = 0.9\linewidth]{Figure+Table/IgA-data1-QC.pdf}
\caption{IgA data1 QC}\label{fig:IgA-data1-QC}
\end{center}

Figure \ref{fig:IgA-data1-SCSA-mapping}为图IgA data1 SCSA mapping概览。

\textbf{(对应文件为 \texttt{Figure+Table/IgA-data1-SCSA-mapping.pdf})}

\def\@captype{figure}
\begin{center}
\includegraphics[width = 0.9\linewidth]{Figure+Table/IgA-data1-SCSA-mapping.pdf}
\caption{IgA data1 SCSA mapping}\label{fig:IgA-data1-SCSA-mapping}
\end{center}

\hypertarget{ux5171ux8868ux8fbeux5206ux6790-1}{%
\paragraph{共表达分析}\label{ux5171ux8868ux8fbeux5206ux6790-1}}

Figure \ref{fig:IgA-data1-pseudotime}为图IgA data1 pseudotime概览。

\textbf{(对应文件为 \texttt{Figure+Table/IgA-data1-pseudotime.pdf})}

\def\@captype{figure}
\begin{center}
\includegraphics[width = 0.9\linewidth]{Figure+Table/IgA-data1-pseudotime.pdf}
\caption{IgA data1 pseudotime}\label{fig:IgA-data1-pseudotime}
\end{center}

Figure \ref{fig:IgA-data1-co-expression-modules}为图IgA data1 co expression modules概览。

\textbf{(对应文件为 \texttt{Figure+Table/IgA-data1-co-expression-modules.pdf})}

\def\@captype{figure}
\begin{center}
\includegraphics[width = 0.9\linewidth]{Figure+Table/IgA-data1-co-expression-modules.pdf}
\caption{IgA data1 co expression modules}\label{fig:IgA-data1-co-expression-modules}
\end{center}

Figure \ref{fig:IgA-data1-Proximal-tubule-cell-subtype}为图IgA data1 Proximal tubule cell subtype概览。

\textbf{(对应文件为 \texttt{Figure+Table/IgA-data1-Proximal-tubule-cell-subtype.pdf})}

\def\@captype{figure}
\begin{center}
\includegraphics[width = 0.9\linewidth]{Figure+Table/IgA-data1-Proximal-tubule-cell-subtype.pdf}
\caption{IgA data1 Proximal tubule cell subtype}\label{fig:IgA-data1-Proximal-tubule-cell-subtype}
\end{center}

\hypertarget{ux819cux6027ux80beux75c5-gse241302-idiopathic-membranous-nephropathy}{%
\subsubsection{膜性肾病: GSE241302 (idiopathic membranous nephropathy)}\label{ux819cux6027ux80beux75c5-gse241302-idiopathic-membranous-nephropathy}}

\begin{itemize}
\item
  In order to explore the molecular mechanism of IMN, we collected renal tissue
  samples from 3 IMN patients and 1 healthy controls and performed analysis by
  single-cell RNA sequencing.
\item
  GSE241302, scRNA
\item
  GSE216841, RNA-seq
\item
  GSE175759, RNA-seq
\end{itemize}

\hypertarget{ux7ec6ux80deux805aux7c7bux548cux6ce8ux91ca-3}{%
\paragraph{细胞聚类和注释}\label{ux7ec6ux80deux805aux7c7bux548cux6ce8ux91ca-3}}

Figure \ref{fig:IMN-data-QC}为图IMN data QC概览。

\textbf{(对应文件为 \texttt{Figure+Table/IMN-data-QC.pdf})}

\def\@captype{figure}
\begin{center}
\includegraphics[width = 0.9\linewidth]{Figure+Table/IMN-data-QC.pdf}
\caption{IMN data QC}\label{fig:IMN-data-QC}
\end{center}

Figure \ref{fig:IMN-data-SCSA-mapping}为图IMN data SCSA mapping概览。

\textbf{(对应文件为 \texttt{Figure+Table/IMN-data-SCSA-mapping.pdf})}

\def\@captype{figure}
\begin{center}
\includegraphics[width = 0.9\linewidth]{Figure+Table/IMN-data-SCSA-mapping.pdf}
\caption{IMN data SCSA mapping}\label{fig:IMN-data-SCSA-mapping}
\end{center}

\hypertarget{ux5171ux8868ux8fbeux5206ux6790-2}{%
\paragraph{共表达分析}\label{ux5171ux8868ux8fbeux5206ux6790-2}}

Figure \ref{fig:IMN-data-pseudotime}为图IMN data pseudotime概览。

\textbf{(对应文件为 \texttt{Figure+Table/IMN-data-pseudotime.pdf})}

\def\@captype{figure}
\begin{center}
\includegraphics[width = 0.9\linewidth]{Figure+Table/IMN-data-pseudotime.pdf}
\caption{IMN data pseudotime}\label{fig:IMN-data-pseudotime}
\end{center}

Figure \ref{fig:IMN-data-co-expression-modules}为图IMN data co expression modules概览。

\textbf{(对应文件为 \texttt{Figure+Table/IMN-data-co-expression-modules.pdf})}

\def\@captype{figure}
\begin{center}
\includegraphics[width = 0.9\linewidth]{Figure+Table/IMN-data-co-expression-modules.pdf}
\caption{IMN data co expression modules}\label{fig:IMN-data-co-expression-modules}
\end{center}

Figure \ref{fig:IMN-data-Proximal-tubule-cell-subtype}为图IMN data Proximal tubule cell subtype概览。

\textbf{(对应文件为 \texttt{Figure+Table/IMN-data-Proximal-tubule-cell-subtype.pdf})}

\def\@captype{figure}
\begin{center}
\includegraphics[width = 0.9\linewidth]{Figure+Table/IMN-data-Proximal-tubule-cell-subtype.pdf}
\caption{IMN data Proximal tubule cell subtype}\label{fig:IMN-data-Proximal-tubule-cell-subtype}
\end{center}

\hypertarget{ux6b63ux5e38ux5bf9ux7167healthy-controlgse171314}{%
\subsection{正常对照(Healthy control)(GSE171314)}\label{ux6b63ux5e38ux5bf9ux7167healthy-controlgse171314}}

\hypertarget{ux7ec6ux80deux805aux7c7bux548cux6ce8ux91ca-4}{%
\paragraph{细胞聚类和注释}\label{ux7ec6ux80deux805aux7c7bux548cux6ce8ux91ca-4}}

Figure \ref{fig:HC-data-QC}为图HC data QC概览。

\textbf{(对应文件为 \texttt{Figure+Table/HC-data-QC.pdf})}

\def\@captype{figure}
\begin{center}
\includegraphics[width = 0.9\linewidth]{Figure+Table/HC-data-QC.pdf}
\caption{HC data QC}\label{fig:HC-data-QC}
\end{center}

Figure \ref{fig:HC-data-SCSA-mapping}为图HC data SCSA mapping概览。

\textbf{(对应文件为 \texttt{Figure+Table/HC-data-SCSA-mapping.pdf})}

\def\@captype{figure}
\begin{center}
\includegraphics[width = 0.9\linewidth]{Figure+Table/HC-data-SCSA-mapping.pdf}
\caption{HC data SCSA mapping}\label{fig:HC-data-SCSA-mapping}
\end{center}

\hypertarget{ux591aux6570ux636eux96c6ux6210ccrcc-ux548c-ckd-scrna-seq}{%
\subsection{多数据集成:ccRCC 和 CKD scRNA-seq}\label{ux591aux6570ux636eux96c6ux6210ccrcc-ux548c-ckd-scrna-seq}}

\hypertarget{ux4f7fux7528-risc-ux5bf9ux4e0dux540cux6765ux6e90ux7684ux6570ux636eux96c6ux6210}{%
\subsubsection{使用 RISC 对不同来源的数据集成}\label{ux4f7fux7528-risc-ux5bf9ux4e0dux540cux6765ux6e90ux7684ux6570ux636eux96c6ux6210}}

RISC\textsuperscript{\protect\hyperlink{ref-RobustIntegratLiuY2021}{10}}

Figure \ref{fig:SETS1-select-reference-dataset-for-integration}为图SETS1 select reference dataset for integration概览。

\textbf{(对应文件为 \texttt{Figure+Table/SETS1-select-reference-dataset-for-integration.pdf})}

\def\@captype{figure}
\begin{center}
\includegraphics[width = 0.9\linewidth]{Figure+Table/SETS1-select-reference-dataset-for-integration.pdf}
\caption{SETS1 select reference dataset for integration}\label{fig:SETS1-select-reference-dataset-for-integration}
\end{center}

Figure \ref{fig:SETS1-umap-mapping}为图SETS1 umap mapping概览。

\textbf{(对应文件为 \texttt{Figure+Table/SETS1-umap-mapping.pdf})}

\def\@captype{figure}
\begin{center}
\includegraphics[width = 0.9\linewidth]{Figure+Table/SETS1-umap-mapping.pdf}
\caption{SETS1 umap mapping}\label{fig:SETS1-umap-mapping}
\end{center}

\hypertarget{ux62dfux65f6ux5206ux6790}{%
\subsubsection{拟时分析}\label{ux62dfux65f6ux5206ux6790}}

在不同的肾病聚类中,IgA 与 RCC 最为接近。IgA 肾病可能发展成 RCC。

Figure \ref{fig:SETS1-trajectory}为图SETS1 trajectory概览。

\textbf{(对应文件为 \texttt{Figure+Table/SETS1-trajectory.pdf})}

\def\@captype{figure}
\begin{center}
\includegraphics[width = 0.9\linewidth]{Figure+Table/SETS1-trajectory.pdf}
\caption{SETS1 trajectory}\label{fig:SETS1-trajectory}
\end{center}

Figure \ref{fig:SETS1-pseudotime}为图SETS1 pseudotime概览。

\textbf{(对应文件为 \texttt{Figure+Table/SETS1-pseudotime.pdf})}

\def\@captype{figure}
\begin{center}
\includegraphics[width = 0.9\linewidth]{Figure+Table/SETS1-pseudotime.pdf}
\caption{SETS1 pseudotime}\label{fig:SETS1-pseudotime}
\end{center}

\hypertarget{igacancerhc-ux5deeux5f02ux5206ux6790}{%
\subsubsection{IgA、Cancer、HC 差异分析}\label{igacancerhc-ux5deeux5f02ux5206ux6790}}

Figure \ref{fig:SETS1-intersection-of-contrasts-DEGs}为图SETS1 intersection of contrasts DEGs概览。

\textbf{(对应文件为 \texttt{Figure+Table/SETS1-intersection-of-contrasts-DEGs.pdf})}

\def\@captype{figure}
\begin{center}
\includegraphics[width = 0.9\linewidth]{Figure+Table/SETS1-intersection-of-contrasts-DEGs.pdf}
\caption{SETS1 intersection of contrasts DEGs}\label{fig:SETS1-intersection-of-contrasts-DEGs}
\end{center}

Table \ref{tab:SETS1-DEGs}为表格SETS1 DEGs概览。

\textbf{(对应文件为 \texttt{Figure+Table/SETS1-DEGs.csv})}

\begin{center}\begin{tcolorbox}[colback=gray!10, colframe=gray!50, width=0.9\linewidth, arc=1mm, boxrule=0.5pt]注:表格共有14359行7列,以下预览的表格可能省略部分数据;表格含有3个唯一`contrast'。
\end{tcolorbox}
\end{center}

\begin{longtable}[]{@{}lllllll@{}}
\caption{\label{tab:SETS1-DEGs}SETS1 DEGs}\tabularnewline
\toprule
contrast & p\_val & avg\_l\ldots{} & pct.1 & pct.2 & p\_val\ldots{} & gene\tabularnewline
\midrule
\endfirsthead
\toprule
contrast & p\_val & avg\_l\ldots{} & pct.1 & pct.2 & p\_val\ldots{} & gene\tabularnewline
\midrule
\endhead
Cance\ldots{} & 0 & -11.8\ldots{} & 0 & 1 & 0 & MT-RNR1\tabularnewline
Cance\ldots{} & 0 & -8.47\ldots{} & 0 & 0.809 & 0 & FABP1\tabularnewline
Cance\ldots{} & 0 & -11.9\ldots{} & 0 & 0.806 & 0 & MTATP6P1\tabularnewline
Cance\ldots{} & 0 & -12.0\ldots{} & 0 & 0.988 & 0 & ALDOB\tabularnewline
Cance\ldots{} & 0 & -3.24\ldots{} & 0.97 & 0.056 & 0 & ANXA2\tabularnewline
Cance\ldots{} & 0 & -9.49\ldots{} & 0 & 0.839 & 0 & SLC5A12\tabularnewline
Cance\ldots{} & 0 & -4.24\ldots{} & 1 & 0.075 & 0 & VIM\tabularnewline
Cance\ldots{} & 0 & -1.19\ldots{} & 0.928 & 0.015 & 0 & PFKP\tabularnewline
Cance\ldots{} & 0 & -0.97\ldots{} & 0.91 & 0.014 & 0 & PERP\tabularnewline
Cance\ldots{} & 0 & -2.67\ldots{} & 0.933 & 0.03 & 0 & LGALS1\tabularnewline
Cance\ldots{} & 0 & -1.90\ldots{} & 0.887 & 0.029 & 0 & BST2\tabularnewline
Cance\ldots{} & 0 & -1.09\ldots{} & 0.848 & 0.017 & 0 & CYTOR\tabularnewline
Cance\ldots{} & 0 & -2.57\ldots{} & 0.981 & 0.023 & 0 & TIMP1\tabularnewline
Cance\ldots{} & 0 & 0.906\ldots{} & 0.784 & 0.007 & 0 & MT3\tabularnewline
Cance\ldots{} & 0 & -1.71\ldots{} & 0.859 & 0.022 & 0 & AHNAK\tabularnewline
\ldots{} & \ldots{} & \ldots{} & \ldots{} & \ldots{} & \ldots{} & \ldots{}\tabularnewline
\bottomrule
\end{longtable}

\hypertarget{ux5bccux96c6ux5206ux6790gsea}{%
\subsubsection{富集分析(GSEA)}\label{ux5bccux96c6ux5206ux6790gsea}}

\hypertarget{cancer-vs-iga}{%
\paragraph{Cancer vs IgA}\label{cancer-vs-iga}}

Figure \ref{fig:SETS1-GSEA-enrichment-of-KEGG-CancerVsIgA}为图SETS1 GSEA enrichment of KEGG CancerVsIgA概览。

\textbf{(对应文件为 \texttt{Figure+Table/SETS1-GSEA-enrichment-of-KEGG-CancerVsIgA.pdf})}

\def\@captype{figure}
\begin{center}
\includegraphics[width = 0.9\linewidth]{Figure+Table/SETS1-GSEA-enrichment-of-KEGG-CancerVsIgA.pdf}
\caption{SETS1 GSEA enrichment of KEGG CancerVsIgA}\label{fig:SETS1-GSEA-enrichment-of-KEGG-CancerVsIgA}
\end{center}

Figure \ref{fig:SETS1-GSEA-enrichment-of-GO-CancerVsIgA}为图SETS1 GSEA enrichment of GO CancerVsIgA概览。

\textbf{(对应文件为 \texttt{Figure+Table/SETS1-GSEA-enrichment-of-GO-CancerVsIgA.pdf})}

\def\@captype{figure}
\begin{center}
\includegraphics[width = 0.9\linewidth]{Figure+Table/SETS1-GSEA-enrichment-of-GO-CancerVsIgA.pdf}
\caption{SETS1 GSEA enrichment of GO CancerVsIgA}\label{fig:SETS1-GSEA-enrichment-of-GO-CancerVsIgA}
\end{center}

Fig. \ref{fig:SETS1-GSEA-show-OXPHOS-pathway} 富集于 OXPHOS 通路的基因表达量整体下降。

Figure \ref{fig:SETS1-GSEA-show-OXPHOS-pathway-CancerVsIgA}为图SETS1 GSEA show OXPHOS pathway CancerVsIgA概览。

\textbf{(对应文件为 \texttt{Figure+Table/SETS1-GSEA-show-OXPHOS-pathway-CancerVsIgA.pdf})}

\def\@captype{figure}
\begin{center}
\includegraphics[width = 0.9\linewidth]{Figure+Table/SETS1-GSEA-show-OXPHOS-pathway-CancerVsIgA.pdf}
\caption{SETS1 GSEA show OXPHOS pathway CancerVsIgA}\label{fig:SETS1-GSEA-show-OXPHOS-pathway-CancerVsIgA}
\end{center}

Figure \ref{fig:SETS1-GSEA-show-OXPHOS-pathway-by-pathview-CancerVsIgA}为图SETS1 GSEA show OXPHOS pathway by pathview CancerVsIgA概览。

\textbf{(对应文件为 \texttt{Figure+Table/hsa00190.pathview.png})}

\def\@captype{figure}
\begin{center}
\includegraphics[width = 0.9\linewidth]{pathview2023-10-31_16_52_42.995662/hsa00190.pathview.png}
\caption{SETS1 GSEA show OXPHOS pathway by pathview CancerVsIgA}\label{fig:SETS1-GSEA-show-OXPHOS-pathway-by-pathview-CancerVsIgA}
\end{center}

\hypertarget{cancer-vs-control}{%
\paragraph{Cancer vs Control}\label{cancer-vs-control}}

Fig. \ref{fig:SETS1-GSEA-enrichment-of-KEGG-CancerVsControl} 与 Fig. \ref{fig:SETS1-GSEA-enrichment-of-KEGG-CancerVsIgA} 相比,
并不富集于 OXPHOS 通路。

Figure \ref{fig:SETS1-GSEA-enrichment-of-KEGG-CancerVsControl}为图SETS1 GSEA enrichment of KEGG CancerVsControl概览。

\textbf{(对应文件为 \texttt{Figure+Table/SETS1-GSEA-enrichment-of-KEGG-CancerVsControl.pdf})}

\def\@captype{figure}
\begin{center}
\includegraphics[width = 0.9\linewidth]{Figure+Table/SETS1-GSEA-enrichment-of-KEGG-CancerVsControl.pdf}
\caption{SETS1 GSEA enrichment of KEGG CancerVsControl}\label{fig:SETS1-GSEA-enrichment-of-KEGG-CancerVsControl}
\end{center}

Figure \ref{fig:SETS1-GSEA-enrichment-of-GO-CancerVsControl}为图SETS1 GSEA enrichment of GO CancerVsControl概览。

\textbf{(对应文件为 \texttt{Figure+Table/SETS1-GSEA-enrichment-of-GO-CancerVsControl.pdf})}

\def\@captype{figure}
\begin{center}
\includegraphics[width = 0.9\linewidth]{Figure+Table/SETS1-GSEA-enrichment-of-GO-CancerVsControl.pdf}
\caption{SETS1 GSEA enrichment of GO CancerVsControl}\label{fig:SETS1-GSEA-enrichment-of-GO-CancerVsControl}
\end{center}

\hypertarget{ux9a8cux8bc1}{%
\subsection{验证}\label{ux9a8cux8bc1}}

\hypertarget{ux53e6ux4e00ux7ec4-iga-ux80beux75c5-gse127136}{%
\subsubsection{另一组 IgA 肾病: GSE127136}\label{ux53e6ux4e00ux7ec4-iga-ux80beux75c5-gse127136}}

\hypertarget{ux7ec6ux80deux805aux7c7bux548cux6ce8ux91ca-5}{%
\paragraph{细胞聚类和注释}\label{ux7ec6ux80deux805aux7c7bux548cux6ce8ux91ca-5}}

\hypertarget{ux53e6ux4e00ux7ec4-rcc-gse202374}{%
\subsubsection{另一组 RCC: GSE202374}\label{ux53e6ux4e00ux7ec4-rcc-gse202374}}

\hypertarget{ux7ec6ux80deux805aux7c7bux548cux6ce8ux91ca-6}{%
\paragraph{细胞聚类和注释}\label{ux7ec6ux80deux805aux7c7bux548cux6ce8ux91ca-6}}

同 \ref{rcc1}, 使用 SCSA 注释后,以 copyKAT 预测癌细胞,随后将肿瘤细胞映射到 UMAP 聚类图中。

Figure \ref{fig:RCC-data2-SCSA-mapping-with-copykat-prediction}为图RCC data2 SCSA mapping with copykat prediction概览。

\textbf{(对应文件为 \texttt{Figure+Table/RCC-data2-SCSA-mapping-with-copykat-prediction.pdf})}

\def\@captype{figure}
\begin{center}
\includegraphics[width = 0.9\linewidth]{Figure+Table/RCC-data2-SCSA-mapping-with-copykat-prediction.pdf}
\caption{RCC data2 SCSA mapping with copykat prediction}\label{fig:RCC-data2-SCSA-mapping-with-copykat-prediction}
\end{center}

\hypertarget{ux96c6ux6210ux9a8cux8bc1canceriga}{%
\subsubsection{集成验证:Cancer、IgA}\label{ux96c6ux6210ux9a8cux8bc1canceriga}}

\hypertarget{risc-ux96c6ux6210}{%
\paragraph{RISC 集成}\label{risc-ux96c6ux6210}}

Figure \ref{fig:SETS2-select-reference-dataset-for-integration}为图SETS2 select reference dataset for integration概览。

\textbf{(对应文件为 \texttt{Figure+Table/SETS2-select-reference-dataset-for-integration.pdf})}

\def\@captype{figure}
\begin{center}
\includegraphics[width = 0.9\linewidth]{Figure+Table/SETS2-select-reference-dataset-for-integration.pdf}
\caption{SETS2 select reference dataset for integration}\label{fig:SETS2-select-reference-dataset-for-integration}
\end{center}

Figure \ref{fig:SETS2-umap-mapping}为图SETS2 umap mapping概览。

\textbf{(对应文件为 \texttt{Figure+Table/SETS2-umap-mapping.pdf})}

\def\@captype{figure}
\begin{center}
\includegraphics[width = 0.9\linewidth]{Figure+Table/SETS2-umap-mapping.pdf}
\caption{SETS2 umap mapping}\label{fig:SETS2-umap-mapping}
\end{center}

\hypertarget{ux62dfux65f6ux5206ux6790-1}{%
\paragraph{拟时分析}\label{ux62dfux65f6ux5206ux6790-1}}

依据拟时轨迹Fig. \ref{fig:SETS1-pseudotime},同 Fig. \ref{fig:SETS1-trajectory}相似,IgA 与 RCC 聚类邻近,可能在疾病发展过程中转化为癌症。

Figure \ref{fig:SETS2-pseudotime}为图SETS2 pseudotime概览。

\textbf{(对应文件为 \texttt{Figure+Table/SETS2-pseudotime.pdf})}

\def\@captype{figure}
\begin{center}
\includegraphics[width = 0.9\linewidth]{Figure+Table/SETS2-pseudotime.pdf}
\caption{SETS2 pseudotime}\label{fig:SETS2-pseudotime}
\end{center}

\hypertarget{ux5bccux96c6ux5206ux6790}{%
\paragraph{富集分析}\label{ux5bccux96c6ux5206ux6790}}

Figure \ref{fig:SETS2-GSEA-enrichment-of-KEGG-CancerVsIgA}为图SETS2 GSEA enrichment of KEGG CancerVsIgA概览。

\textbf{(对应文件为 \texttt{Figure+Table/SETS2-GSEA-enrichment-of-KEGG-CancerVsIgA.pdf})}

\def\@captype{figure}
\begin{center}
\includegraphics[width = 0.9\linewidth]{Figure+Table/SETS2-GSEA-enrichment-of-KEGG-CancerVsIgA.pdf}
\caption{SETS2 GSEA enrichment of KEGG CancerVsIgA}\label{fig:SETS2-GSEA-enrichment-of-KEGG-CancerVsIgA}
\end{center}

Figure \ref{fig:SETS2-GSEA-enrichment-of-GO-CancerVsIgA}为图SETS2 GSEA enrichment of GO CancerVsIgA概览。

\textbf{(对应文件为 \texttt{Figure+Table/SETS2-GSEA-enrichment-of-GO-CancerVsIgA.pdf})}

\def\@captype{figure}
\begin{center}
\includegraphics[width = 0.9\linewidth]{Figure+Table/SETS2-GSEA-enrichment-of-GO-CancerVsIgA.pdf}
\caption{SETS2 GSEA enrichment of GO CancerVsIgA}\label{fig:SETS2-GSEA-enrichment-of-GO-CancerVsIgA}
\end{center}

Fig. \ref{fig:SETS2-GSEA-show-OXPHOS-pathway} 富集于 OXPHOS 通路的基因表达量整体上升(与Fig. \ref{fig:SETS1-GSEA-show-OXPHOS-pathway} 相反)。系肿瘤细胞的异质性所致\textsuperscript{\protect\hyperlink{ref-HallmarksOfCaSancho2016}{13}},RCC 肿瘤可分为 OXPHOS 依赖型和非依赖型代谢增强。

Figure \ref{fig:SETS2-GSEA-show-OXPHOS-pathway-CancerVsIgA}为图SETS2 GSEA show OXPHOS pathway CancerVsIgA概览。

\textbf{(对应文件为 \texttt{Figure+Table/SETS2-GSEA-show-OXPHOS-pathway-CancerVsIgA.pdf})}

\def\@captype{figure}
\begin{center}
\includegraphics[width = 0.9\linewidth]{Figure+Table/SETS2-GSEA-show-OXPHOS-pathway-CancerVsIgA.pdf}
\caption{SETS2 GSEA show OXPHOS pathway CancerVsIgA}\label{fig:SETS2-GSEA-show-OXPHOS-pathway-CancerVsIgA}
\end{center}

Figure \ref{fig:SETS2-GSEA-show-OXPHOS-pathway-by-pathview-CancerVsIgA}为图SETS2 GSEA show OXPHOS pathway by pathview CancerVsIgA概览。

\textbf{(对应文件为 \texttt{Figure+Table/hsa00190.pathview.png})}

\def\@captype{figure}
\begin{center}
\includegraphics[width = 0.9\linewidth]{pathview2023-10-31_15_25_14.115006/hsa00190.pathview.png}
\caption{SETS2 GSEA show OXPHOS pathway by pathview CancerVsIgA}\label{fig:SETS2-GSEA-show-OXPHOS-pathway-by-pathview-CancerVsIgA}
\end{center}

\hypertarget{bibliography}{%
\section*{Reference}\label{bibliography}}
\addcontentsline{toc}{section}{Reference}

\hypertarget{refs}{}
\begin{cslreferences}
\leavevmode\hypertarget{ref-OnconephrologyRosner2021}{}%
1. Rosner, M. H., Jhaveri, K. D., McMahon, B. A. \& Perazella, M. A. Onconephrology: The intersections between the kidney and cancer. \emph{CA: a cancer journal for clinicians} \textbf{71}, 47--77 (2021).

\leavevmode\hypertarget{ref-CkdAndTheRisLowran2014}{}%
2. Lowrance, W. T., Ordoñez, J., Udaltsova, N., Russo, P. \& Go, A. S. CKD and the risk of incident cancer. \emph{Journal of the American Society of Nephrology} \textbf{25}, (2014).

\leavevmode\hypertarget{ref-CancerRiskAndKitchl2022}{}%
3. Kitchlu, A. \emph{et al.} Cancer risk and mortality in patients with kidney disease: A population-based cohort study. \emph{American journal of kidney diseases : the official journal of the National Kidney Foundation} \textbf{80}, 436--448.e1 (2022).

\leavevmode\hypertarget{ref-RenalCellCancSaly2021}{}%
4. Saly, D. L., Eswarappa, M. S., Street, S. E. \& Deshpande, P. Renal cell cancer and chronic kidney disease. \emph{Advances in chronic kidney disease} \textbf{28}, 460--468.e1 (2021).

\leavevmode\hypertarget{ref-IntegratedAnalHaoY2021}{}%
5. Hao, Y. \emph{et al.} Integrated analysis of multimodal single-cell data. \emph{Cell} \textbf{184}, (2021).

\leavevmode\hypertarget{ref-ComprehensiveIStuart2019}{}%
6. Stuart, T. \emph{et al.} Comprehensive integration of single-cell data. \emph{Cell} \textbf{177}, (2019).

\leavevmode\hypertarget{ref-ReversedGraphQiuX2017}{}%
7. Qiu, X. \emph{et al.} Reversed graph embedding resolves complex single-cell trajectories. \emph{Nature Methods} \textbf{14}, (2017).

\leavevmode\hypertarget{ref-TheDynamicsAnTrapne2014}{}%
8. Trapnell, C. \emph{et al.} The dynamics and regulators of cell fate decisions are revealed by pseudotemporal ordering of single cells. \emph{Nature Biotechnology} \textbf{32}, (2014).

\leavevmode\hypertarget{ref-DelineatingCopGaoR2021}{}%
9. Gao, R. \emph{et al.} Delineating copy number and clonal substructure in human tumors from single-cell transcriptomes. \emph{Nature Biotechnology} \textbf{39}, 599--608 (2021).

\leavevmode\hypertarget{ref-RobustIntegratLiuY2021}{}%
10. Liu, Y., Wang, T., Zhou, B. \& Zheng, D. Robust integration of multiple single-cell rna sequencing datasets using a single reference space. \emph{Nature biotechnology} \textbf{39}, 877--884 (2021).

\leavevmode\hypertarget{ref-ClusterprofilerWuTi2021}{}%
11. Wu, T. \emph{et al.} ClusterProfiler 4.0: A universal enrichment tool for interpreting omics data. \emph{The Innovation} \textbf{2}, (2021).

\leavevmode\hypertarget{ref-ScsaACellTyCaoY2020}{}%
12. Cao, Y., Wang, X. \& Peng, G. SCSA: A cell type annotation tool for single-cell rna-seq data. \emph{Frontiers in genetics} \textbf{11}, (2020).

\leavevmode\hypertarget{ref-HallmarksOfCaSancho2016}{}%
13. Sancho, P., Barneda, D. \& Heeschen, C. Hallmarks of cancer stem cell metabolism. \emph{British journal of cancer} \textbf{114}, 1305--1312 (2016).
\end{cslreferences}

\end{document}
