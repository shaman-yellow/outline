% Options for packages loaded elsewhere
\PassOptionsToPackage{unicode}{hyperref}
\PassOptionsToPackage{hyphens}{url}
%
\documentclass[
]{article}
\usepackage{lmodern}
\usepackage{amssymb,amsmath}
\usepackage{ifxetex,ifluatex}
\ifnum 0\ifxetex 1\fi\ifluatex 1\fi=0 % if pdftex
  \usepackage[T1]{fontenc}
  \usepackage[utf8]{inputenc}
  \usepackage{textcomp} % provide euro and other symbols
\else % if luatex or xetex
  \usepackage{unicode-math}
  \defaultfontfeatures{Scale=MatchLowercase}
  \defaultfontfeatures[\rmfamily]{Ligatures=TeX,Scale=1}
\fi
% Use upquote if available, for straight quotes in verbatim environments
\IfFileExists{upquote.sty}{\usepackage{upquote}}{}
\IfFileExists{microtype.sty}{% use microtype if available
  \usepackage[]{microtype}
  \UseMicrotypeSet[protrusion]{basicmath} % disable protrusion for tt fonts
}{}
\makeatletter
\@ifundefined{KOMAClassName}{% if non-KOMA class
  \IfFileExists{parskip.sty}{%
    \usepackage{parskip}
  }{% else
    \setlength{\parindent}{0pt}
    \setlength{\parskip}{6pt plus 2pt minus 1pt}}
}{% if KOMA class
  \KOMAoptions{parskip=half}}
\makeatother
\usepackage{xcolor}
\IfFileExists{xurl.sty}{\usepackage{xurl}}{} % add URL line breaks if available
\IfFileExists{bookmark.sty}{\usepackage{bookmark}}{\usepackage{hyperref}}
\hypersetup{
  pdftitle={Analysis},
  pdfauthor={Huang LiChuang of Wie-Biotech},
  hidelinks,
  pdfcreator={LaTeX via pandoc}}
\urlstyle{same} % disable monospaced font for URLs
\usepackage[margin=1in]{geometry}
\usepackage{longtable,booktabs}
% Correct order of tables after \paragraph or \subparagraph
\usepackage{etoolbox}
\makeatletter
\patchcmd\longtable{\par}{\if@noskipsec\mbox{}\fi\par}{}{}
\makeatother
% Allow footnotes in longtable head/foot
\IfFileExists{footnotehyper.sty}{\usepackage{footnotehyper}}{\usepackage{footnote}}
\makesavenoteenv{longtable}
\usepackage{graphicx}
\makeatletter
\def\maxwidth{\ifdim\Gin@nat@width>\linewidth\linewidth\else\Gin@nat@width\fi}
\def\maxheight{\ifdim\Gin@nat@height>\textheight\textheight\else\Gin@nat@height\fi}
\makeatother
% Scale images if necessary, so that they will not overflow the page
% margins by default, and it is still possible to overwrite the defaults
% using explicit options in \includegraphics[width, height, ...]{}
\setkeys{Gin}{width=\maxwidth,height=\maxheight,keepaspectratio}
% Set default figure placement to htbp
\makeatletter
\def\fps@figure{htbp}
\makeatother
\setlength{\emergencystretch}{3em} % prevent overfull lines
\providecommand{\tightlist}{%
  \setlength{\itemsep}{0pt}\setlength{\parskip}{0pt}}
\setcounter{secnumdepth}{5}
\usepackage{caption} \captionsetup{font={footnotesize},width=6in} \renewcommand{\dblfloatpagefraction}{.9} \makeatletter \renewenvironment{figure} {\def\@captype{figure}} \makeatother \definecolor{shadecolor}{RGB}{242,242,242} \usepackage{xeCJK} \usepackage{setspace} \setstretch{1.3} \usepackage{tcolorbox}
\newlength{\cslhangindent}
\setlength{\cslhangindent}{1.5em}
\newenvironment{cslreferences}%
  {}%
  {\par}

\title{Analysis}
\author{Huang LiChuang of Wie-Biotech}
\date{}

\begin{document}
\maketitle

{
\setcounter{tocdepth}{3}
\tableofcontents
}
\listoffigures

\listoftables

\hypertarget{ux9898ux76ee}{%
\section{题目}\label{ux9898ux76ee}}

待定(根据实际分析)

\hypertarget{abstract}{%
\section{摘要(拟设计)}\label{abstract}}

\hypertarget{ux8981ux6c42}{%
\subsection{要求}\label{ux8981ux6c42}}

\begin{itemize}
\tightlist
\item
  方向:皮肤科,特应性皮炎(Atopic Dermatitis,AD)\textsuperscript{\protect\hyperlink{ref-MolecularMechaSroka2021}{1},\protect\hyperlink{ref-FromEmollientsSalvat2021}{2}}
\item
  分值:5分+
\end{itemize}

\hypertarget{ad-ux7814ux7a76ux73b0ux72b6ux6982ux8ff0}{%
\subsection{AD 研究现状概述}\label{ad-ux7814ux7a76ux73b0ux72b6ux6982ux8ff0}}

AD 涉及的因素较多(epidermal barrier dysfunction, host genetics, environmental factors, and immune perturbations\textsuperscript{\protect\hyperlink{ref-FromEmollientsSalvat2021}{2}})。AD 中调节皮肤屏障功能和免疫反应的潜在机制和关键分子已被揭示\textsuperscript{\protect\hyperlink{ref-SkinBarrierAbYang2020}{3}}。JAK-STAT 通路在 AD 机制中起到关键作用\textsuperscript{\protect\hyperlink{ref-JakStatSignalHuang2022}{4}}。AD 的 Biomarkers 已被综述\textsuperscript{\protect\hyperlink{ref-BiomarkersInABakker2023}{5}}。

AD 被认为不仅是一种皮肤病(Is Atopic Dermatitis Only a Skin Disease?\textsuperscript{\protect\hyperlink{ref-IsAtopicDermaMesjas2023}{6}}),而是与其它疾病在一定环境因素下互关联(cardiovascular, autoimmune, neurological, psychiatric, and metabolic disorders);这些被怀疑通过更复杂的遗传和免疫学机制与 AD 的发病机制相关,但其相关性仍知之甚少(2023 年文献)\textsuperscript{\protect\hyperlink{ref-IsAtopicDermaMesjas2023}{6}}。

关联的疾病\textsuperscript{\protect\hyperlink{ref-IsAtopicDermaMesjas2023}{6}}:

\begin{itemize}
\tightlist
\item
  Cardiovascular Diseases

  \begin{itemize}
  \tightlist
  \item
    coronary artery disease
  \item
    angina pectoris
  \item
    myocardial infarction
  \item
    stroke
  \item
    peripheral vascular disease
  \end{itemize}
\item
  Neurologic and Psychiatric Diseases

  \begin{itemize}
  \tightlist
  \item
    Epilepsy
  \item
    Autism
  \item
    Depression
  \item
    \ldots{}
  \end{itemize}
\item
  Autoimmune Diseases

  \begin{itemize}
  \tightlist
  \item
    Alopecia Areata
  \item
    Vitiligo
  \item
    Rheumatoid Diseases
  \item
    Type I Diabetes
  \end{itemize}
\item
  Obesity
\end{itemize}

\hypertarget{ux62dfux89e3ux51b3ux7684ux95eeux9898}{%
\subsection{拟解决的问题}\label{ux62dfux89e3ux51b3ux7684ux95eeux9898}}

AD 和其他疾病(Cardiovascular Diseases, Neurologic and Psychiatric Diseases, Autoimmune Diseases, Obesity)的关联机制探究。从基因表达数据入手,研究 AD 和多种疾病的互关联基因和机制。

Figure \ref{fig:Comorbidities-frequently-associated-with-AD}为图Comorbidities frequently associated with AD概览。

\textbf{(对应文件为 \texttt{\textasciitilde{}/Pictures/Screenshots/Screenshot\ from\ 2023-10-09\ 10-20-39.png})}

\def\@captype{figure}
\begin{center}
\includegraphics[width = 0.9\linewidth]{~/Pictures/Screenshots/Screenshot from 2023-10-09 10-20-39.png}
\caption{Comorbidities frequently associated with AD}\label{fig:Comorbidities-frequently-associated-with-AD}
\end{center}

\hypertarget{ux9884ux671fux8defux7ebf}{%
\subsection{预期路线}\label{ux9884ux671fux8defux7ebf}}

AD 可能会存在不同类型,可能需要根据 Fig. \ref{fig:Comorbidities-frequently-associated-with-AD} 将不同病例(RNA-seq 数据)聚类,各自和各种疾病关联分析,寻找相关基因。各类型的并发症(并发症 a、b、c)可能和 AD 之间有互关联基因,假设这些为集合 A,集合 B,集合 C。那么,集合 A、B、C 等之间,可能还存在某种关联性(该关联性需要采取适当的方法探讨,联系实际解决问题)。预期的关联性较强,有突出结果,则根据这个思路深入分析;若关联性弱,则调整方向,可能调整研究的并发症(缩减范围),寻找有关联的点深入突破。后期可再以 RNA-seq 或 scRNA-seq 进一步验证。

\hypertarget{ux53efux7528ux7684ux6570ux636eux96c6}{%
\subsection{可用的数据集}\label{ux53efux7528ux7684ux6570ux636eux96c6}}

AD 的数据集:

\begin{itemize}
\tightlist
\item
  GSE224783, RNA-seq
\item
  GSE237920, RNA-seq
\item
  GSE184509, RNA-seq
\item
  GSE213849, scRNA-seq
\item
  GSE197023, ST, scRNA-seq
\item
  GSE222840, scRNA-seq
\item
  \ldots{}
\end{itemize}

除上述 AD 数据集外,还需要其它关联疾病的(GEO)数据集。这些数据将根据实际情况选用。

\hypertarget{ux6570ux636eux5206ux6790ux62dfux91c7ux7528ux7684ux65b9ux6cd5}{%
\subsection{数据分析拟采用的方法:}\label{ux6570ux636eux5206ux6790ux62dfux91c7ux7528ux7684ux65b9ux6cd5}}

\begin{itemize}
\tightlist
\item
  RNA-seq 数据分析

  \begin{itemize}
  \tightlist
  \item
    差异分析
  \item
    富集分析\textsuperscript{\protect\hyperlink{ref-ClusterprofilerWuTi2021}{7}}
  \item
    免疫浸润预测\textsuperscript{\protect\hyperlink{ref-EstimatingTheBecht2016}{8}}
  \item
    基因共表达(WGCNA)\textsuperscript{\protect\hyperlink{ref-WgcnaAnRPacLangfe2008}{9}}
  \item
    \ldots{}
  \end{itemize}
\item
  scRNA-seq 数据分析(或验证)

  \begin{itemize}
  \tightlist
  \item
    细胞鉴定\textsuperscript{\protect\hyperlink{ref-IntegratedAnalHaoY2021}{10}}(The pathology of AD is accompanied by an imbalance in immunity involving Th1, Th2, and Treg cells, culminating in alterations in Th1- and Th2-mediated immune responses and IgE-mediated hypersensitivity\textsuperscript{\protect\hyperlink{ref-SkinBarrierAbYang2020}{3}})
  \item
    拟时分析\textsuperscript{\protect\hyperlink{ref-TheDynamicsAnTrapne2014}{11}}
  \item
    细胞通讯\textsuperscript{\protect\hyperlink{ref-InferenceAndAJinS2021}{12}}
  \item
    \ldots{}
  \end{itemize}
\item
  关联因素探讨

  \begin{itemize}
  \tightlist
  \item
    共表达分析
  \item
    蛋白互作
  \item
    文献调研
  \item
    \ldots{}
  \end{itemize}
\end{itemize}

\hypertarget{ux5de5ux4f5cux91cf}{%
\subsection{工作量}\label{ux5de5ux4f5cux91cf}}

预期工作量较大,需要1-2周时间。

\hypertarget{introduction}{%
\section{前言}\label{introduction}}

\hypertarget{route}{%
\section{研究设计流程图}\label{route}}

\hypertarget{methods}{%
\section{材料和方法}\label{methods}}

\hypertarget{results}{%
\section{分析结果}\label{results}}

\hypertarget{dis}{%
\section{结论}\label{dis}}

\hypertarget{bibliography}{%
\section*{Reference}\label{bibliography}}
\addcontentsline{toc}{section}{Reference}

\hypertarget{refs}{}
\begin{cslreferences}
\leavevmode\hypertarget{ref-MolecularMechaSroka2021}{}%
1. Sroka-Tomaszewska, J. \& Trzeciak, M. Molecular mechanisms of atopic dermatitis pathogenesis. \emph{International journal of molecular sciences} \textbf{22}, (2021).

\leavevmode\hypertarget{ref-FromEmollientsSalvat2021}{}%
2. Salvati, L., Cosmi, L. \& Annunziato, F. From emollients to biologicals: Targeting atopic dermatitis. \emph{International journal of molecular sciences} \textbf{22}, (2021).

\leavevmode\hypertarget{ref-SkinBarrierAbYang2020}{}%
3. Yang, G. \emph{et al.} Skin barrier abnormalities and immune dysfunction in atopic dermatitis. \emph{International journal of molecular sciences} \textbf{21}, (2020).

\leavevmode\hypertarget{ref-JakStatSignalHuang2022}{}%
4. Huang, I.-H., Chung, W.-H., Wu, P.-C. \& Chen, C.-B. JAK-stat signaling pathway in the pathogenesis of atopic dermatitis: An updated review. \emph{Frontiers in immunology} \textbf{13}, (2022).

\leavevmode\hypertarget{ref-BiomarkersInABakker2023}{}%
5. Bakker, D., Bruin-Weller, M. de, Drylewicz, J., Wijk, F. van \& Thijs, J. Biomarkers in atopic dermatitis. \emph{The Journal of allergy and clinical immunology} \textbf{151}, 1163--1168 (2023).

\leavevmode\hypertarget{ref-IsAtopicDermaMesjas2023}{}%
6. Mesjasz, A., Zawadzka, M., Chałubiński, M. \& Trzeciak, M. Is atopic dermatitis only a skin disease? \emph{International journal of molecular sciences} \textbf{24}, (2023).

\leavevmode\hypertarget{ref-ClusterprofilerWuTi2021}{}%
7. Wu, T. \emph{et al.} ClusterProfiler 4.0: A universal enrichment tool for interpreting omics data. \emph{The Innovation} \textbf{2}, (2021).

\leavevmode\hypertarget{ref-EstimatingTheBecht2016}{}%
8. Becht, E. \emph{et al.} Estimating the~population abundance of tissue-infiltrating immune and stromal cell populations using gene expression. \emph{Genome Biology} \textbf{17}, (2016).

\leavevmode\hypertarget{ref-WgcnaAnRPacLangfe2008}{}%
9. Langfelder, P. \& Horvath, S. WGCNA: An r package for weighted correlation network analysis. \emph{BMC Bioinformatics} \textbf{9}, (2008).

\leavevmode\hypertarget{ref-IntegratedAnalHaoY2021}{}%
10. Hao, Y. \emph{et al.} Integrated analysis of multimodal single-cell data. \emph{Cell} \textbf{184}, (2021).

\leavevmode\hypertarget{ref-TheDynamicsAnTrapne2014}{}%
11. Trapnell, C. \emph{et al.} The dynamics and regulators of cell fate decisions are revealed by pseudotemporal ordering of single cells. \emph{Nature Biotechnology} \textbf{32}, (2014).

\leavevmode\hypertarget{ref-InferenceAndAJinS2021}{}%
12. Jin, S. \emph{et al.} Inference and analysis of cell-cell communication using cellchat. \emph{Nature Communications} \textbf{12}, (2021).
\end{cslreferences}

\end{document}
