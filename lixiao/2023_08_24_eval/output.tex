% Options for packages loaded elsewhere
\PassOptionsToPackage{unicode}{hyperref}
\PassOptionsToPackage{hyphens}{url}
%
\documentclass[
]{article}
\usepackage{lmodern}
\usepackage{amssymb,amsmath}
\usepackage{ifxetex,ifluatex}
\ifnum 0\ifxetex 1\fi\ifluatex 1\fi=0 % if pdftex
  \usepackage[T1]{fontenc}
  \usepackage[utf8]{inputenc}
  \usepackage{textcomp} % provide euro and other symbols
\else % if luatex or xetex
  \usepackage{unicode-math}
  \defaultfontfeatures{Scale=MatchLowercase}
  \defaultfontfeatures[\rmfamily]{Ligatures=TeX,Scale=1}
\fi
% Use upquote if available, for straight quotes in verbatim environments
\IfFileExists{upquote.sty}{\usepackage{upquote}}{}
\IfFileExists{microtype.sty}{% use microtype if available
  \usepackage[]{microtype}
  \UseMicrotypeSet[protrusion]{basicmath} % disable protrusion for tt fonts
}{}
\makeatletter
\@ifundefined{KOMAClassName}{% if non-KOMA class
  \IfFileExists{parskip.sty}{%
    \usepackage{parskip}
  }{% else
    \setlength{\parindent}{0pt}
    \setlength{\parskip}{6pt plus 2pt minus 1pt}}
}{% if KOMA class
  \KOMAoptions{parskip=half}}
\makeatother
\usepackage{xcolor}
\IfFileExists{xurl.sty}{\usepackage{xurl}}{} % add URL line breaks if available
\IfFileExists{bookmark.sty}{\usepackage{bookmark}}{\usepackage{hyperref}}
\hypersetup{
  pdftitle={Evaluation},
  pdfauthor={Huang LiChuang of Wie-Biotech},
  hidelinks,
  pdfcreator={LaTeX via pandoc}}
\urlstyle{same} % disable monospaced font for URLs
\usepackage[margin=1in]{geometry}
\usepackage{longtable,booktabs}
% Correct order of tables after \paragraph or \subparagraph
\usepackage{etoolbox}
\makeatletter
\patchcmd\longtable{\par}{\if@noskipsec\mbox{}\fi\par}{}{}
\makeatother
% Allow footnotes in longtable head/foot
\IfFileExists{footnotehyper.sty}{\usepackage{footnotehyper}}{\usepackage{footnote}}
\makesavenoteenv{longtable}
\usepackage{graphicx}
\makeatletter
\def\maxwidth{\ifdim\Gin@nat@width>\linewidth\linewidth\else\Gin@nat@width\fi}
\def\maxheight{\ifdim\Gin@nat@height>\textheight\textheight\else\Gin@nat@height\fi}
\makeatother
% Scale images if necessary, so that they will not overflow the page
% margins by default, and it is still possible to overwrite the defaults
% using explicit options in \includegraphics[width, height, ...]{}
\setkeys{Gin}{width=\maxwidth,height=\maxheight,keepaspectratio}
% Set default figure placement to htbp
\makeatletter
\def\fps@figure{htbp}
\makeatother
\setlength{\emergencystretch}{3em} % prevent overfull lines
\providecommand{\tightlist}{%
  \setlength{\itemsep}{0pt}\setlength{\parskip}{0pt}}
\setcounter{secnumdepth}{5}
\usepackage{caption} \captionsetup{font={footnotesize},width=6in} \renewcommand{\dblfloatpagefraction}{.9} \makeatletter \renewenvironment{figure} {\def\@captype{figure}} \makeatother \definecolor{shadecolor}{RGB}{242,242,242} \usepackage{xeCJK} \usepackage{setspace} \setstretch{1.3} \usepackage{tcolorbox}
\newlength{\cslhangindent}
\setlength{\cslhangindent}{1.5em}
\newenvironment{cslreferences}%
  {}%
  {\par}

\title{Evaluation}
\author{Huang LiChuang of Wie-Biotech}
\date{}

\begin{document}
\maketitle

{
\setcounter{tocdepth}{3}
\tableofcontents
}
\listoffigures

\listoftables

\hypertarget{abstract}{%
\section{摘要}\label{abstract}}

以下为需要评估的内容以及相应的答复:

\begin{enumerate}
\def\labelenumi{\arabic{enumi}.}
\tightlist
\item
  用网络药理学分析中药方的主要活性成分,对比文献研究现状挑选4个左右的活性成分作为候选药物成分

  \begin{itemize}
  \tightlist
  \item
    可分析主要成分,结合目标疾病筛选主要活性成分。
  \item
    可从 HERB (\url{http://herb.ac.cn/Download/}) 获取成分信息(该数据库整合了较多的其他数据库)。
  \item
    目标疾病是否为糖尿病肾病?且有代谢组数据?如果为(中药方成分的)非靶向代谢组,则能根据非靶向代谢组数据鉴定更多的化合物用于网络药理学分析,而不是只通过数据库筛选。
  \item
    可进一步通过 PPI 网络和通路富集分析筛选活性成分\textsuperscript{\protect\hyperlink{ref-NeuronSpecificMurtaz2022}{1}}。
  \item
    可查阅文献(较为主观,可作为辅助手段)
  \end{itemize}
\item
  分析糖尿病肾病肠道差异菌群

  \begin{itemize}
  \tightlist
  \item
    需要 16s RNA数据,或从 GEO 公共数据库获取(\url{https://www.ncbi.nlm.nih.gov/gds/?term=16s})。
  \item
    可用 qiime2 (\url{https://qiime2.org/})筛选肠道差异菌群\textsuperscript{\protect\hyperlink{ref-MicrobiomeDataRaiS2021}{2},\protect\hyperlink{ref-LongitudinalInWang2019}{3}}。
  \end{itemize}
\item
  分析糖尿病肾病代谢组学差异

  \begin{itemize}
  \tightlist
  \item
    需明确,是人的肾脏的代谢组,还是肠道菌的代谢组(因为上述有肠道菌分析,容易混淆)。
  \item
    可分为挖掘公共数据库(GNPS:\url{https://gnps.ucsd.edu/ProteoSAFe/static/gnps-splash.jsp})和客户提供代谢数据的情况。
  \item
    如果是客户提供,请考虑:

    \begin{itemize}
    \tightlist
    \item
      包含对照组和模型组的数据,需要生物学重复,最好为非靶向代谢组数据(这种情况下,能鉴定和找到最多的差异代谢物)
    \item
      如果是非靶向代谢组,需要数据鉴定。可分为谱图匹配性鉴定(常规方法),和预测性鉴定(例如,SIRIUS\textsuperscript{\protect\hyperlink{ref-Sirius4ARapDuhrko2019}{4}})。
    \item
      如果是靶向代谢组,已知目标代谢物,则不需要额外的鉴定,根据分子量比对即可。
    \item
      需确认代谢组数据的采集是否包含 MS\textsuperscript{2}。如果仅包含 MS\textsuperscript{1},则鉴定准确度会相对偏低。如果是靶向代谢组,仅有 MS\textsuperscript{1} 亦可。
    \end{itemize}
  \item
    以标准的方法:PCA 聚类,OPLS-DA 聚类、VIP 、P 值筛选差异代谢物。
  \item
    还可以结合不同 Feature selection 算法进一步筛选,例如LASSO,EFS\textsuperscript{\protect\hyperlink{ref-EfsAnEnsemblNeuman2017}{5}}等。
  \item
    差异代谢物可通路富集分析,结合疾病,进一步筛选。可用方法为 MetaboAnalyst (\url{https://www.metaboanalyst.ca/MetaboAnalyst/ModuleView.xhtml})\textsuperscript{\protect\hyperlink{ref-MetaboanalystrPang2020}{6}}。
  \item
    根据上述情况不同,工作量会大不相同。可做大量分析,也可仅做少量分析。
  \end{itemize}
\item
  分析差异菌群与差异代谢物的相关性(桑基图)

  \begin{itemize}
  \tightlist
  \item
    16s RNA 和代谢组的多组学分析\textsuperscript{\protect\hyperlink{ref-DisruptedSpermZhang2021}{7}}。
  \item
    对差异菌群和差异代谢物关联性分析,桑基图、热图均可(热图可能更直观,见 Fig. \ref{fig:correlation-of-metabolites-with-microbiota})。
  \end{itemize}
\item
  结合药物成分筛选目标菌群和相关代谢物

  \begin{itemize}
  \tightlist
  \item
    首先需要获取药物成分的靶点。可通过 Binding DB 获取 \url{https://www.bindingdb.org/rwd/bind/index.jsp}。
  \item
    通过(人的)代谢物的通路富集结果(步骤 3)得到相应蛋白,可对药物靶点取交集。
  \item
    通过机器预测药物和肠道菌互作\textsuperscript{\protect\hyperlink{ref-PredictingDrugMccoub2022}{8},\protect\hyperlink{ref-MachineLearninMccoub2021}{9}}。目前似乎存在一系列方法,需要探索找到合适的方法。
  \item
    若上述方法简便容易,以下或许可以不用考虑:

    \begin{itemize}
    \tightlist
    \item
      关于药物-肠道菌互作\textsuperscript{\protect\hyperlink{ref-PersonalizedMaJavdan2020}{10}},药物作用于菌,得到代谢物(细菌的代谢物,而非人)。
    \item
      想要从药物中筛选,需要细菌的代谢物信息(这一步骤可能存在一定苦难,需要结合实际筛选的细菌考虑)。
    \item
      上述,肠道菌代谢物的获取是关键步骤之一,可以从 gutMGene(\url{http://bio-annotation.cn/gutmgene/home.dhtml})获取。
    \item
      通过计算药物和肠道菌代谢物的分子相似性,推测是否存在药物-菌作用关系\textsuperscript{\protect\hyperlink{ref-MolecularSimilGandin2022}{11}}(具有不确定性)
    \end{itemize}
  \end{itemize}
\item
  代谢小分子靶点蛋白分析(这个看能不能用分子对接的方式获取)

  \begin{itemize}
  \tightlist
  \item
    只要上述(步骤 5)的药物和靶点都具备,即可以分子对接方式分析。
  \end{itemize}
\item
  广泛的靶点蛋白与糖尿病肾病差异基因取交集,筛选候选基因

  \begin{itemize}
  \tightlist
  \item
    结合疾病的公共数据库筛选进一步筛选,例如 genecards \url{https://www.genecards.org/}。
  \item
    需要明确,这里还能进一步通过分析其他 GEO 数据筛选差异基因,再结合筛选。是否需要?
  \end{itemize}
\end{enumerate}

总体上,工作量较大,视情况可能需要 1-3 周。

\hypertarget{route}{%
\section{研究设计流程图}\label{route}}

\hypertarget{methods}{%
\section{材料和方法}\label{methods}}

\hypertarget{results}{%
\section{分析结果}\label{results}}

以下内容为仅为示例。

\hypertarget{ux7f51ux7edcux836fux7406ux5b66ux548c-ppi}{%
\subsection{网络药理学和 PPI}\label{ux7f51ux7edcux836fux7406ux5b66ux548c-ppi}}

Figure \ref{fig:PPI-network-for-targets}为图PPI network for targets概览。

\textbf{(对应文件为 \texttt{\textasciitilde{}/Pictures/Screenshots/Screenshot\ from\ 2023-08-24\ 11-56-42.png})}

\def\@captype{figure}
\begin{center}
\includegraphics[width = 0.9\linewidth]{~/Pictures/Screenshots/Screenshot from 2023-08-24 11-56-42.png}
\caption{PPI network for targets}\label{fig:PPI-network-for-targets}
\end{center}

Figure \ref{fig:enrichment-analysis}为图enrichment analysis概览。

\textbf{(对应文件为 \texttt{\textasciitilde{}/Pictures/Screenshots/Screenshot\ from\ 2023-08-24\ 13-03-48.png})}

\def\@captype{figure}
\begin{center}
\includegraphics[width = 0.9\linewidth]{~/Pictures/Screenshots/Screenshot from 2023-08-24 13-03-48.png}
\caption{Enrichment analysis}\label{fig:enrichment-analysis}
\end{center}

\hypertarget{ux80a0ux9053ux83ccux5206ux6790}{%
\subsection{肠道菌分析}\label{ux80a0ux9053ux83ccux5206ux6790}}

Figure \ref{fig:flow-chart-of-qiime2-processing}为图flow chart of qiime2 processing概览。

\textbf{(对应文件为 \texttt{\textasciitilde{}/Pictures/Screenshots/Screenshot\ from\ 2023-08-24\ 11-26-30.png})}

\def\@captype{figure}
\begin{center}
\includegraphics[width = 0.9\linewidth]{~/Pictures/Screenshots/Screenshot from 2023-08-24 11-26-30.png}
\caption{Flow chart of qiime2 processing}\label{fig:flow-chart-of-qiime2-processing}
\end{center}

Figure \ref{fig:qiime2-quality-control}为图qiime2 quality control概览。

\textbf{(对应文件为 \texttt{\textasciitilde{}/Pictures/Screenshots/Screenshot\ from\ 2023-08-24\ 11-36-35.png})}

\def\@captype{figure}
\begin{center}
\includegraphics[width = 0.9\linewidth]{~/Pictures/Screenshots/Screenshot from 2023-08-24 11-36-35.png}
\caption{Qiime2 quality control}\label{fig:qiime2-quality-control}
\end{center}

Figure \ref{fig:gut-microbiome-abundance}为图gut microbiome abundance概览。

\textbf{(对应文件为 \texttt{\textasciitilde{}/Pictures/Screenshots/Screenshot\ from\ 2023-08-24\ 11-40-38.png})}

\def\@captype{figure}
\begin{center}
\includegraphics[width = 0.9\linewidth]{~/Pictures/Screenshots/Screenshot from 2023-08-24 11-40-38.png}
\caption{Gut microbiome abundance}\label{fig:gut-microbiome-abundance}
\end{center}

\hypertarget{ux4ee3ux8c22ux7269ux5206ux6790}{%
\subsection{代谢物分析}\label{ux4ee3ux8c22ux7269ux5206ux6790}}

Figure \ref{fig:use-MetaboAnalyst-for-analysis-of-metabolites}为图use MetaboAnalyst for analysis of metabolites概览。

\textbf{(对应文件为 \texttt{\textasciitilde{}/Pictures/Screenshots/Screenshot\ from\ 2023-08-24\ 11-32-20.png})}

\def\@captype{figure}
\begin{center}
\includegraphics[width = 0.9\linewidth]{~/Pictures/Screenshots/Screenshot from 2023-08-24 11-32-20.png}
\caption{Use MetaboAnalyst for analysis of metabolites}\label{fig:use-MetaboAnalyst-for-analysis-of-metabolites}
\end{center}

Figure \ref{fig:identify-compounds-with-SIRIUS-4-OPTIONAL}为图identify compounds with SIRIUS 4 OPTIONAL概览。

\textbf{(对应文件为 \texttt{\textasciitilde{}/Pictures/Screenshots/Screenshot\ from\ 2023-08-24\ 11-28-15.png})}

\def\@captype{figure}
\begin{center}
\includegraphics[width = 0.9\linewidth]{~/Pictures/Screenshots/Screenshot from 2023-08-24 11-28-15.png}
\caption{Identify compounds with SIRIUS 4 OPTIONAL}\label{fig:identify-compounds-with-SIRIUS-4-OPTIONAL}
\end{center}

\hypertarget{ux80a0ux9053ux83ccux548cux4ee3ux8c22ux7269ux6574ux5408}{%
\subsection{肠道菌和代谢物整合}\label{ux80a0ux9053ux83ccux548cux4ee3ux8c22ux7269ux6574ux5408}}

Figure \ref{fig:correlation-of-metabolites-with-microbiota}为图correlation of metabolites with microbiota概览。

\textbf{(对应文件为 \texttt{\textasciitilde{}/Pictures/Screenshots/Screenshot\ from\ 2023-08-24\ 11-24-36.png})}

\def\@captype{figure}
\begin{center}
\includegraphics[width = 0.9\linewidth]{~/Pictures/Screenshots/Screenshot from 2023-08-24 11-24-36.png}
\caption{Correlation of metabolites with microbiota}\label{fig:correlation-of-metabolites-with-microbiota}
\end{center}

\hypertarget{ux836fux7269ux548cux80a0ux9053ux83ccux7684ux76f8ux4e92ux4f5cux7528}{%
\subsection{药物和肠道菌的相互作用}\label{ux836fux7269ux548cux80a0ux9053ux83ccux7684ux76f8ux4e92ux4f5cux7528}}

\hypertarget{ux901aux8fc7ux673aux5668ux5b66ux4e60ux9884ux6d4b}{%
\subsubsection{通过机器学习预测}\label{ux901aux8fc7ux673aux5668ux5b66ux4e60ux9884ux6d4b}}

来自于综述文章,需要找到合适的方法。

Figure \ref{fig:machine-learning-prediction-of-drug-towards-microbiota}为图machine learning prediction of drug towards microbiota概览。

\textbf{(对应文件为 \texttt{\textasciitilde{}/Pictures/Screenshots/Screenshot\ from\ 2023-08-24\ 13-38-39.png})}

\def\@captype{figure}
\begin{center}
\includegraphics[width = 0.9\linewidth]{~/Pictures/Screenshots/Screenshot from 2023-08-24 13-38-39.png}
\caption{Machine learning prediction of drug towards microbiota}\label{fig:machine-learning-prediction-of-drug-towards-microbiota}
\end{center}

\hypertarget{ux836fux7269ux88abux80a0ux9053ux83ccux4ee3ux8c22}{%
\subsubsection{药物被肠道菌代谢}\label{ux836fux7269ux88abux80a0ux9053ux83ccux4ee3ux8c22}}

Figure \ref{fig:drug-metabolized-by-microbiota}为图drug metabolized by microbiota概览。

\textbf{(对应文件为 \texttt{\textasciitilde{}/Pictures/Screenshots/Screenshot\ from\ 2023-08-24\ 13-13-48.png})}

\def\@captype{figure}
\begin{center}
\includegraphics[width = 0.9\linewidth]{~/Pictures/Screenshots/Screenshot from 2023-08-24 13-13-48.png}
\caption{Drug metabolized by microbiota}\label{fig:drug-metabolized-by-microbiota}
\end{center}

Figure \ref{fig:chemical-modification-by-microbiota}为图chemical modification by microbiota概览。

\textbf{(对应文件为 \texttt{\textasciitilde{}/Pictures/Screenshots/Screenshot\ from\ 2023-08-24\ 13-16-48.png})}

\def\@captype{figure}
\begin{center}
\includegraphics[width = 0.9\linewidth]{~/Pictures/Screenshots/Screenshot from 2023-08-24 13-16-48.png}
\caption{Chemical modification by microbiota}\label{fig:chemical-modification-by-microbiota}
\end{center}

\hypertarget{ux5206ux5b50ux76f8ux4f3cux6027}{%
\subsubsection{分子相似性}\label{ux5206ux5b50ux76f8ux4f3cux6027}}

Figure \ref{fig:molecule-similarity}为图molecule similarity概览。

\textbf{(对应文件为 \texttt{\textasciitilde{}/Pictures/Screenshots/Screenshot\ from\ 2023-08-24\ 12-52-35.png})}

\def\@captype{figure}
\begin{center}
\includegraphics[width = 0.9\linewidth]{~/Pictures/Screenshots/Screenshot from 2023-08-24 12-52-35.png}
\caption{Molecule similarity}\label{fig:molecule-similarity}
\end{center}

\hypertarget{ux5206ux5b50ux5bf9ux63a5}{%
\subsection{分子对接}\label{ux5206ux5b50ux5bf9ux63a5}}

Figure \ref{fig:autodock-vina-binding-affinity}为图autodock vina binding affinity概览。

\textbf{(对应文件为 \texttt{../2023\_06\_30\_eval/figs/Docking\_Affinity.pdf})}

\def\@captype{figure}
\begin{center}
\includegraphics[width = 0.9\linewidth]{../2023_06_30_eval/figs/Docking_Affinity.pdf}
\caption{Autodock vina binding affinity}\label{fig:autodock-vina-binding-affinity}
\end{center}

\hypertarget{dis}{%
\section{结论}\label{dis}}

\hypertarget{bibliography}{%
\section*{Reference}\label{bibliography}}
\addcontentsline{toc}{section}{Reference}

\hypertarget{refs}{}
\begin{cslreferences}
\leavevmode\hypertarget{ref-NeuronSpecificMurtaz2022}{}%
1. Murtaza, N. \emph{et al.} Neuron-specific protein network mapping of autism risk genes identifies shared biological mechanisms and disease-relevant pathologies. \emph{Cell Reports} \textbf{41}, (2022).

\leavevmode\hypertarget{ref-MicrobiomeDataRaiS2021}{}%
2. Rai, S. N. \emph{et al.} Microbiome data analysis with applications to pre-clinical studies using qiime2: Statistical considerations. \emph{Genes \textbackslash\&amp; Diseases} \textbf{8}, (2021).

\leavevmode\hypertarget{ref-LongitudinalInWang2019}{}%
3. Wang, X. \emph{et al.} Longitudinal investigation of the swine gut microbiome from birth to market reveals stage and growth performance associated bacteria. \emph{Microbiome} \textbf{7}, (2019).

\leavevmode\hypertarget{ref-Sirius4ARapDuhrko2019}{}%
4. Dührkop, K. \emph{et al.} SIRIUS 4: A rapid tool for turning tandem mass spectra into metabolite structure information. \emph{Nature Methods} \textbf{16}, 299--302 (2019).

\leavevmode\hypertarget{ref-EfsAnEnsemblNeuman2017}{}%
5. Neumann, U., Genze, N. \& Heider, D. EFS: An ensemble feature selection tool implemented as r-package and web-application. \emph{BioData Mining} \textbf{10}, 21 (2017).

\leavevmode\hypertarget{ref-MetaboanalystrPang2020}{}%
6. Pang, Z., Chong, J., Li, S. \& Xia, J. MetaboAnalystR 3.0: Toward an optimized workflow for global metabolomics. \emph{Metabolites} (2020) doi:\href{https://doi.org/10.3390/metabo10050186}{10.3390/metabo10050186}.

\leavevmode\hypertarget{ref-DisruptedSpermZhang2021}{}%
7. Zhang, T. \emph{et al.} Disrupted spermatogenesis in a metabolic syndrome model: The role of vitamin a metabolism in the guttestis axis. \emph{Gut} \textbf{71}, (2021).

\leavevmode\hypertarget{ref-PredictingDrugMccoub2022}{}%
8. McCoubrey, L. E., Gaisford, S., Orlu, M. \& Basit, A. W. Predicting drug-microbiome interactions with machine learning. \emph{Biotechnology Advances} \textbf{54}, (2022).

\leavevmode\hypertarget{ref-MachineLearninMccoub2021}{}%
9. McCoubrey, L. E., Elbadawi, M., Orlu, M., Gaisford, S. \& Basit, A. W. Machine learning uncovers adverse drug effects on intestinal bacteria. \emph{Pharmaceutics} \textbf{13}, (2021).

\leavevmode\hypertarget{ref-PersonalizedMaJavdan2020}{}%
10. Javdan, B. \emph{et al.} Personalized mapping of drug metabolism by the human gut microbiome. \emph{Cell} \textbf{181}, (2020).

\leavevmode\hypertarget{ref-MolecularSimilGandin2022}{}%
11. Gandini, E. \emph{et al.} Molecular similarity perception based on machine-learning models. \emph{International Journal of Molecular Sciences} \textbf{23}, (2022).
\end{cslreferences}

\end{document}
