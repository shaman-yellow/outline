% Options for packages loaded elsewhere
\PassOptionsToPackage{unicode}{hyperref}
\PassOptionsToPackage{hyphens}{url}
%
\documentclass[
]{article}
\usepackage{lmodern}
\usepackage{amssymb,amsmath}
\usepackage{ifxetex,ifluatex}
\ifnum 0\ifxetex 1\fi\ifluatex 1\fi=0 % if pdftex
  \usepackage[T1]{fontenc}
  \usepackage[utf8]{inputenc}
  \usepackage{textcomp} % provide euro and other symbols
\else % if luatex or xetex
  \usepackage{unicode-math}
  \defaultfontfeatures{Scale=MatchLowercase}
  \defaultfontfeatures[\rmfamily]{Ligatures=TeX,Scale=1}
\fi
% Use upquote if available, for straight quotes in verbatim environments
\IfFileExists{upquote.sty}{\usepackage{upquote}}{}
\IfFileExists{microtype.sty}{% use microtype if available
  \usepackage[]{microtype}
  \UseMicrotypeSet[protrusion]{basicmath} % disable protrusion for tt fonts
}{}
\makeatletter
\@ifundefined{KOMAClassName}{% if non-KOMA class
  \IfFileExists{parskip.sty}{%
    \usepackage{parskip}
  }{% else
    \setlength{\parindent}{0pt}
    \setlength{\parskip}{6pt plus 2pt minus 1pt}}
}{% if KOMA class
  \KOMAoptions{parskip=half}}
\makeatother
\usepackage{xcolor}
\IfFileExists{xurl.sty}{\usepackage{xurl}}{} % add URL line breaks if available
\IfFileExists{bookmark.sty}{\usepackage{bookmark}}{\usepackage{hyperref}}
\hypersetup{
  hidelinks,
  pdfcreator={LaTeX via pandoc}}
\urlstyle{same} % disable monospaced font for URLs
\usepackage[margin=1in]{geometry}
\usepackage{longtable,booktabs}
% Correct order of tables after \paragraph or \subparagraph
\usepackage{etoolbox}
\makeatletter
\patchcmd\longtable{\par}{\if@noskipsec\mbox{}\fi\par}{}{}
\makeatother
% Allow footnotes in longtable head/foot
\IfFileExists{footnotehyper.sty}{\usepackage{footnotehyper}}{\usepackage{footnote}}
\makesavenoteenv{longtable}
\usepackage{graphicx}
\makeatletter
\def\maxwidth{\ifdim\Gin@nat@width>\linewidth\linewidth\else\Gin@nat@width\fi}
\def\maxheight{\ifdim\Gin@nat@height>\textheight\textheight\else\Gin@nat@height\fi}
\makeatother
% Scale images if necessary, so that they will not overflow the page
% margins by default, and it is still possible to overwrite the defaults
% using explicit options in \includegraphics[width, height, ...]{}
\setkeys{Gin}{width=\maxwidth,height=\maxheight,keepaspectratio}
% Set default figure placement to htbp
\makeatletter
\def\fps@figure{htbp}
\makeatother
\setlength{\emergencystretch}{3em} % prevent overfull lines
\providecommand{\tightlist}{%
  \setlength{\itemsep}{0pt}\setlength{\parskip}{0pt}}
\setcounter{secnumdepth}{5}
\usepackage{caption} \captionsetup{font={footnotesize},width=6in} \renewcommand{\dblfloatpagefraction}{.9} \makeatletter \renewenvironment{figure} {\def\@captype{figure}} \makeatother \newenvironment{Shaded}{\begin{snugshade}}{\end{snugshade}} \definecolor{shadecolor}{RGB}{230,230,230} \usepackage{xeCJK} \usepackage{setspace} \setstretch{1.3} \usepackage{tcolorbox} \setcounter{secnumdepth}{4} \setcounter{tocdepth}{4} \usepackage{wallpaper} \usepackage[absolute]{textpos} \tcbuselibrary{breakable} \renewenvironment{Shaded} {\begin{tcolorbox}[colback = gray!10, colframe = gray!40, width = 16cm, arc = 1mm, auto outer arc, title = {Input}]} {\end{tcolorbox}} \usepackage{titlesec} \titleformat{\paragraph} {\fontsize{10pt}{0pt}\bfseries} {\arabic{section}.\arabic{subsection}.\arabic{subsubsection}.\arabic{paragraph}} {1em} {} []
\newlength{\cslhangindent}
\setlength{\cslhangindent}{1.5em}
\newenvironment{cslreferences}%
  {}%
  {\par}

\author{}
\date{\vspace{-2.5em}}

\begin{document}

\begin{titlepage} \newgeometry{top=7.5cm}
\ThisCenterWallPaper{1.12}{../cover_page.pdf}
\begin{center} \textbf{\Huge 白芍网络药理学}
\vspace{4em} \begin{textblock}{10}(3,5.9) \huge
\textbf{\textcolor{white}{2024-01-02}}
\end{textblock} \begin{textblock}{10}(3,7.3)
\Large \textcolor{black}{LiChuang Huang}
\end{textblock} \begin{textblock}{10}(3,11.3)
\Large \textcolor{black}{@立效研究院}
\end{textblock} \end{center} \end{titlepage}
\restoregeometry

\pagenumbering{roman}

\tableofcontents

\listoffigures

\listoftables

\newpage

\pagenumbering{arabic}

\hypertarget{abstract}{%
\section{摘要}\label{abstract}}

\hypertarget{ux9700ux6c42ux548cux7ed3ux679c}{%
\subsection{需求和结果}\label{ux9700ux6c42ux548cux7ed3ux679c}}

\begin{itemize}
\tightlist
\item
  白芍总苷 Total glucosides of paeony 中主要化学成分10-20个(TCMSP筛选下口服利用度等)及各个化学成分对应的作用靶点
  (gene与AR过敏性鼻炎相关),最终形成drug-chemical-target gene靶点图

  \begin{itemize}
  \tightlist
  \item
    由于苷类 (Glycosides, G) 成分过少(Tab. \ref{tab:Baishao-glycosides-related-compounds},
    Fig. \ref{fig:Classification-hierarchy}),没有利用 OB 筛选。
    由于靶点过少,这里也没有根据 AR 相关过滤后绘制成分靶点图,而是直接绘制,
    网络药理图见 Fig. \ref{fig:Baishao-Network-pharmacology-visualization}。
    其中与 AR 相关的基因见 Fig. \ref{fig:Baishao-glucosides-targets-intersect-with-AR-related-targets}。
  \end{itemize}
\item
  将获得的靶点进行GO, KEGG富集分析,目标靶点为USP5,关联成分为芍药苷Paeoniflorin

  \begin{itemize}
  \tightlist
  \item
    苷类 (Glycosides, G) 没有富集到 USP5 (TCMSP 的苷类 (Glycosides, G) 靶点信息,不包含 USP5 和 SOX18) ,
    Fig. \ref{fig:Gly-Interect-genes-KEGG-enrichment} 和 Fig. \ref{fig:Gly-Interect-genes-GO-enrichment} 为富集分析结果。
  \end{itemize}
\item
  将芍药苷pae单独拎出,形成pae-targets-pathway网络,此处形成的target genes的GO、KEGG富集图也需要,
  备注USP5参与哪些部分(功能、通路)

  \begin{itemize}
  \tightlist
  \item
    Fig. \ref{fig:Paeoniflorin-Network-pharmacology-visualization} 为 Paeoniflorin 网络药理图
    (由于靶点过少,这里也没有根据 AR 相关过滤后绘制成分靶点图,而是直接绘制) 。
    Paeoniflorin 与 AR 交集基因为 Fig. \ref{fig:Paeoniflorin-targets-intersect-with-AR-related-targets}
  \end{itemize}
\item
  分子对接模拟芍药苷与USP5互作

  \begin{itemize}
  \tightlist
  \item
    见 Fig. \ref{fig:Overall-combining-Affinity} 和 Fig. \ref{fig:Paeoniflorin-combine-USP5}
  \end{itemize}
\item
  转至第2步目标靶点为SOX18,关联成分为芍药苷Paeoniflorin

  \begin{itemize}
  \tightlist
  \item
    Paeoniflorin 不包含 SOX18
  \end{itemize}
\item
  第3步中备注SOX18参与哪些部分(功能、通路)

  \begin{itemize}
  \tightlist
  \item
    不参与
  \end{itemize}
\item
  分子对接模拟芍药苷与SOX18互作

  \begin{itemize}
  \tightlist
  \item
    见 Fig. \ref{fig:Paeoniflorin-combine-SOX18}
  \end{itemize}
\end{itemize}

\hypertarget{introduction}{%
\section{前言}\label{introduction}}

\hypertarget{methods}{%
\section{材料和方法}\label{methods}}

\hypertarget{ux6750ux6599}{%
\subsection{材料}\label{ux6750ux6599}}

\hypertarget{ux65b9ux6cd5}{%
\subsection{方法}\label{ux65b9ux6cd5}}

Mainly used method:

\begin{itemize}
\tightlist
\item
  Database \texttt{PubChem} used for querying information (e.g., InChIKey, CID) of chemical compounds; Tools of \texttt{Classyfire} used for get systematic classification of chemical compounds.\textsuperscript{\protect\hyperlink{ref-PubchemSubstanKimS2015}{1},\protect\hyperlink{ref-ClassyfireAutDjoumb2016}{2}}
\item
  R package \texttt{ClusterProfiler} used for gene enrichment analysis.\textsuperscript{\protect\hyperlink{ref-ClusterprofilerWuTi2021}{3}}
\item
  The API of \texttt{UniProtKB} (\url{https://www.uniprot.org/help/api_queries}) used for mapping of names or IDs of proteins .
\item
  Website \texttt{TCMSP} \url{https://tcmsp-e.com/tcmsp.php} used for data source.\textsuperscript{\protect\hyperlink{ref-TcmspADatabaRuJi2014}{4}}
\item
  \texttt{AutoDock\ vina} used for molecular docking.\textsuperscript{\protect\hyperlink{ref-AutodockVina1Eberha2021}{5}}
\item
  Other R packages (eg., \texttt{dplyr} and \texttt{ggplot2}) used for statistic analysis or data visualization.
\end{itemize}

\hypertarget{results}{%
\section{分析结果}\label{results}}

\hypertarget{dis}{%
\section{结论}\label{dis}}

\hypertarget{workflow}{%
\section{附:分析流程}\label{workflow}}

\hypertarget{tcmsp-ux767dux828dux6210ux5206ux83b7ux53d6}{%
\subsection{TCMSP 白芍成分获取}\label{tcmsp-ux767dux828dux6210ux5206ux83b7ux53d6}}

Table \ref{tab:Baishao-Compounds-and-targets} (下方表格) 为表格Baishao Compounds and targets概览。

\textbf{(对应文件为 \texttt{Figure+Table/Baishao-Compounds-and-targets.xlsx})}

\begin{center}\begin{tcolorbox}[colback=gray!10, colframe=gray!50, width=0.9\linewidth, arc=1mm, boxrule=0.5pt]注:表格共有1036行17列,以下预览的表格可能省略部分数据;表格含有85个唯一`Mol ID'。
\end{tcolorbox}
\end{center}

\begin{longtable}[]{@{}llllllllll@{}}
\caption{\label{tab:Baishao-Compounds-and-targets}Baishao Compounds and targets}\tabularnewline
\toprule
Mol ID & Herb\_p\ldots{} & Molecu\ldots\ldots3 & Molecu\ldots\ldots4 & MW & AlogP & Hdon & Hacc & OB (\%) & Caco-2\tabularnewline
\midrule
\endfirsthead
\toprule
Mol ID & Herb\_p\ldots{} & Molecu\ldots\ldots3 & Molecu\ldots\ldots4 & MW & AlogP & Hdon & Hacc & OB (\%) & Caco-2\tabularnewline
\midrule
\endhead
MOL000106 & Baishao & PYG & https:\ldots{} & 126.12 & 1.03 & 3 & 3 & 22.98 & 0.69\tabularnewline
MOL000211 & Baishao & Mairin & https:\ldots{} & 456.78 & 6.52 & 2 & 3 & 55.38 & 0.73\tabularnewline
MOL000219 & Baishao & BOX & https:\ldots{} & 121.12 & 0.76 & 0 & 2 & 31.55 & 0.54\tabularnewline
MOL000219 & Baishao & BOX & https:\ldots{} & 121.12 & 0.76 & 0 & 2 & 31.55 & 0.54\tabularnewline
MOL000219 & Baishao & BOX & https:\ldots{} & 121.12 & 0.76 & 0 & 2 & 31.55 & 0.54\tabularnewline
MOL000219 & Baishao & BOX & https:\ldots{} & 121.12 & 0.76 & 0 & 2 & 31.55 & 0.54\tabularnewline
MOL000219 & Baishao & BOX & https:\ldots{} & 121.12 & 0.76 & 0 & 2 & 31.55 & 0.54\tabularnewline
MOL000219 & Baishao & BOX & https:\ldots{} & 121.12 & 0.76 & 0 & 2 & 31.55 & 0.54\tabularnewline
MOL000219 & Baishao & BOX & https:\ldots{} & 121.12 & 0.76 & 0 & 2 & 31.55 & 0.54\tabularnewline
MOL000219 & Baishao & BOX & https:\ldots{} & 121.12 & 0.76 & 0 & 2 & 31.55 & 0.54\tabularnewline
MOL000219 & Baishao & BOX & https:\ldots{} & 121.12 & 0.76 & 0 & 2 & 31.55 & 0.54\tabularnewline
MOL000219 & Baishao & BOX & https:\ldots{} & 121.12 & 0.76 & 0 & 2 & 31.55 & 0.54\tabularnewline
MOL000219 & Baishao & BOX & https:\ldots{} & 121.12 & 0.76 & 0 & 2 & 31.55 & 0.54\tabularnewline
MOL000219 & Baishao & BOX & https:\ldots{} & 121.12 & 0.76 & 0 & 2 & 31.55 & 0.54\tabularnewline
MOL000219 & Baishao & BOX & https:\ldots{} & 121.12 & 0.76 & 0 & 2 & 31.55 & 0.54\tabularnewline
\ldots{} & \ldots{} & \ldots{} & \ldots{} & \ldots{} & \ldots{} & \ldots{} & \ldots{} & \ldots{} & \ldots{}\tabularnewline
\bottomrule
\end{longtable}

\hypertarget{ux767dux828dux6240ux6709ux5316ux5408ux7269-tcmsp-ux7684ux5316ux5b66ux7c7b}{%
\subsection{白芍所有化合物 (TCMSP) 的化学类}\label{ux767dux828dux6240ux6709ux5316ux5408ux7269-tcmsp-ux7684ux5316ux5b66ux7c7b}}

\hypertarget{ux767dux828dux7684ux6240ux6709ux6210ux5206}{%
\subsubsection{白芍的所有成分}\label{ux767dux828dux7684ux6240ux6709ux6210ux5206}}

Figure \ref{fig:Classification-hierarchy} (下方图) 为图Classification hierarchy概览。

\textbf{(对应文件为 \texttt{Figure+Table/Classification-hierarchy.pdf})}

\def\@captype{figure}
\begin{center}
\includegraphics[width = 0.9\linewidth]{Figure+Table/Classification-hierarchy.pdf}
\caption{Classification hierarchy}\label{fig:Classification-hierarchy}
\end{center}

\hypertarget{ux767dux828dux7684ux82f7ux7c7b-glycosides-g-ux6210ux5206ux548cux9776ux57faux56e0}{%
\subsubsection{白芍的苷类 (Glycosides, G) 成分和靶基因}\label{ux767dux828dux7684ux82f7ux7c7b-glycosides-g-ux6210ux5206ux548cux9776ux57faux56e0}}

Table \ref{tab:Baishao-glycosides-related-compounds} (下方表格) 为表格Baishao glycosides related compounds概览。

\textbf{(对应文件为 \texttt{Figure+Table/Baishao-glycosides-related-compounds.xlsx})}

\begin{center}\begin{tcolorbox}[colback=gray!10, colframe=gray!50, width=0.9\linewidth, arc=1mm, boxrule=0.5pt]注:表格共有93行20列,以下预览的表格可能省略部分数据;表格含有1个唯一`Herb\_pinyin\_name'。
\end{tcolorbox}
\end{center}

\begin{longtable}[]{@{}llllllllll@{}}
\caption{\label{tab:Baishao-glycosides-related-compounds}Baishao glycosides related compounds}\tabularnewline
\toprule
Herb\_p\ldots{} & compounds & Target\ldots{} & Mol ID & Molecu\ldots{} & MW & AlogP & Hdon & Hacc & OB (\%)\tabularnewline
\midrule
\endfirsthead
\toprule
Herb\_p\ldots{} & compounds & Target\ldots{} & Mol ID & Molecu\ldots{} & MW & AlogP & Hdon & Hacc & OB (\%)\tabularnewline
\midrule
\endhead
Baishao & (Z)-(1\ldots{} & NA & MOL001908 & https:\ldots{} & 446.55 & -1.28 & 6 & 10 & 5.74\tabularnewline
Baishao & albifl\ldots{} & NA & MOL001911 & https:\ldots{} & 480.51 & -1.91 & 5 & 11 & 21.29\tabularnewline
Baishao & albifl\ldots{} & NA & MOL001927 & https:\ldots{} & 480.51 & -1.33 & 5 & 11 & 12.09\tabularnewline
Baishao & galloy\ldots{} & NA & MOL001932 & https:\ldots{} & 632.62 & -0.04 & 7 & 15 & 3.03\tabularnewline
Baishao & oxypae\ldots{} & NA & MOL001933 & https:\ldots{} & 496.51 & -1.55 & 6 & 12 & 21.88\tabularnewline
Baishao & Oxypae\ldots{} & NA & MOL005089 & https:\ldots{} & 496.51 & -1.55 & 6 & 12 & 8.38\tabularnewline
Baishao & sucrose & Aldose\ldots{} & MOL000842 & https:\ldots{} & 342.34 & -4.31 & 8 & 11 & 7.17\tabularnewline
Baishao & sucrose & Aldose\ldots{} & MOL000842 & https:\ldots{} & 342.34 & -4.31 & 8 & 11 & 7.17\tabularnewline
Baishao & sucrose & Aldose\ldots{} & MOL000842 & https:\ldots{} & 342.34 & -4.31 & 8 & 11 & 7.17\tabularnewline
Baishao & sucrose & Alpha-\ldots{} & MOL000842 & https:\ldots{} & 342.34 & -4.31 & 8 & 11 & 7.17\tabularnewline
Baishao & Astrag\ldots{} & Calmod\ldots{} & MOL000561 & https:\ldots{} & 448.41 & -0.32 & 7 & 11 & 14.03\tabularnewline
Baishao & Astrag\ldots{} & Calmod\ldots{} & MOL000561 & https:\ldots{} & 448.41 & -0.32 & 7 & 11 & 14.03\tabularnewline
Baishao & sucrose & Chitin\ldots{} & MOL000842 & https:\ldots{} & 342.34 & -4.31 & 8 & 11 & 7.17\tabularnewline
Baishao & Astrag\ldots{} & Coagul\ldots{} & MOL000561 & https:\ldots{} & 448.41 & -0.32 & 7 & 11 & 14.03\tabularnewline
Baishao & Astrag\ldots{} & Coagul\ldots{} & MOL000561 & https:\ldots{} & 448.41 & -0.32 & 7 & 11 & 14.03\tabularnewline
\ldots{} & \ldots{} & \ldots{} & \ldots{} & \ldots{} & \ldots{} & \ldots{} & \ldots{} & \ldots{} & \ldots{}\tabularnewline
\bottomrule
\end{longtable}

\hypertarget{ux767dux828dux82f7ux7c7b-glycosides-g-ux7684ux7f51ux7edcux836fux7406ux5b66ux5206ux6790}{%
\subsection{白芍苷类 (Glycosides, G) 的网络药理学分析}\label{ux767dux828dux82f7ux7c7b-glycosides-g-ux7684ux7f51ux7edcux836fux7406ux5b66ux5206ux6790}}

\hypertarget{ux767dux828d-ux82f7ux7c7b-glycosides-g--ux9776ux70b9}{%
\subsubsection{白芍-苷类 (Glycosides, G) -靶点}\label{ux767dux828d-ux82f7ux7c7b-glycosides-g--ux9776ux70b9}}

Figure \ref{fig:Baishao-Network-pharmacology-visualization} (下方图) 为图Baishao Network pharmacology visualization概览。

\textbf{(对应文件为 \texttt{Figure+Table/Baishao-Network-pharmacology-visualization.pdf})}

\def\@captype{figure}
\begin{center}
\includegraphics[width = 0.9\linewidth]{Figure+Table/Baishao-Network-pharmacology-visualization.pdf}
\caption{Baishao Network pharmacology visualization}\label{fig:Baishao-Network-pharmacology-visualization}
\end{center}

Figure \ref{fig:Paeoniflorin-Network-pharmacology-visualization} (下方图) 为图Paeoniflorin Network pharmacology visualization概览。

\textbf{(对应文件为 \texttt{Figure+Table/Paeoniflorin-Network-pharmacology-visualization.pdf})}

\def\@captype{figure}
\begin{center}
\includegraphics[width = 0.9\linewidth]{Figure+Table/Paeoniflorin-Network-pharmacology-visualization.pdf}
\caption{Paeoniflorin Network pharmacology visualization}\label{fig:Paeoniflorin-Network-pharmacology-visualization}
\end{center}

\hypertarget{ux8fc7ux654fux6027ux9f3bux708e-allergic-rhinitis-ar-ux76f8ux5173ux57faux56e0}{%
\subsubsection{过敏性鼻炎 (allergic rhinitis, AR) 相关基因}\label{ux8fc7ux654fux6027ux9f3bux708e-allergic-rhinitis-ar-ux76f8ux5173ux57faux56e0}}

AR 相关基因通过 geneCards 获取,并通过 Biomart 注释。

Table \ref{tab:AR-related-genes} (下方表格) 为表格AR related genes概览。

\textbf{(对应文件为 \texttt{Figure+Table/AR-related-genes.xlsx})}

\begin{center}\begin{tcolorbox}[colback=gray!10, colframe=gray!50, width=0.9\linewidth, arc=1mm, boxrule=0.5pt]注:表格共有178行8列,以下预览的表格可能省略部分数据;表格含有178个唯一`hgnc\_symbol'。
\end{tcolorbox}
\end{center}
\begin{center}\begin{tcolorbox}[colback=gray!10, colframe=gray!50, width=0.9\linewidth, arc=1mm, boxrule=0.5pt]\begin{enumerate}\tightlist
\item hgnc\_symbol:  基因名 (Human)
\end{enumerate}\end{tcolorbox}
\end{center}

\begin{longtable}[]{@{}llllllll@{}}
\caption{\label{tab:AR-related-genes}AR related genes}\tabularnewline
\toprule
hgnc\_s\ldots{} & ensemb\ldots{} & entrez\ldots{} & refseq\ldots{} & chromo\ldots{} & start\_\ldots{} & end\_po\ldots{} & descri\ldots{}\tabularnewline
\midrule
\endfirsthead
\toprule
hgnc\_s\ldots{} & ensemb\ldots{} & entrez\ldots{} & refseq\ldots{} & chromo\ldots{} & start\_\ldots{} & end\_po\ldots{} & descri\ldots{}\tabularnewline
\midrule
\endhead
ADRB2 & ENSG00\ldots{} & 154 & NM\_000024 & 5 & 148826611 & 148828623 & adreno\ldots{}\tabularnewline
ALOX5AP & ENSG00\ldots{} & 241 & NM\_001629 & 13 & 30713478 & 30764426 & arachi\ldots{}\tabularnewline
BDNF-AS & ENSG00\ldots{} & 497258 & & 11 & 27506830 & 27698231 & BDNF a\ldots{}\tabularnewline
BGLAP & ENSG00\ldots{} & 632 & NM\_199173 & 1 & 156242184 & 156243317 & bone g\ldots{}\tabularnewline
BPIFA1 & ENSG00\ldots{} & 51297 & NM\_001\ldots{} & 20 & 33235995 & 33243311 & BPI fo\ldots{}\tabularnewline
BTK & ENSG00\ldots{} & 695 & & X & 101349338 & 101390796 & Bruton\ldots{}\tabularnewline
C5AR1 & ENSG00\ldots{} & 728 & & 19 & 47290023 & 47322066 & comple\ldots{}\tabularnewline
CCDC40 & ENSG00\ldots{} & 55036 & NM\_001\ldots{} & HG2118\ldots{} & 59543 & 125319 & coiled\ldots{}\tabularnewline
CCL1 & ENSG00\ldots{} & 6346 & NM\_002981 & 17 & 34360328 & 34363233 & C-C mo\ldots{}\tabularnewline
CCL18 & ENSG00\ldots{} & 6362 & NM\_002988 & HSCHR1\ldots{} & 18377 & 26129 & C-C mo\ldots{}\tabularnewline
CCL20 & ENSG00\ldots{} & 6364 & & 2 & 227805739 & 227817564 & C-C mo\ldots{}\tabularnewline
CCL4 & ENSG00\ldots{} & 6351 & NM\_002984 & HSCHR1\ldots{} & 57924 & 59718 & C-C mo\ldots{}\tabularnewline
CCNO & ENSG00\ldots{} & 10309 & & 5 & 55231152 & 55233608 & cyclin\ldots{}\tabularnewline
CCR5 & ENSG00\ldots{} & 1234 & NM\_000579 & 3 & 46370946 & 46376206 & C-C mo\ldots{}\tabularnewline
CCR8 & ENSG00\ldots{} & 1237 & NM\_005201 & 3 & 39329709 & 39333680 & C-C mo\ldots{}\tabularnewline
\ldots{} & \ldots{} & \ldots{} & \ldots{} & \ldots{} & \ldots{} & \ldots{} & \ldots{}\tabularnewline
\bottomrule
\end{longtable}

\hypertarget{ux82f7ux7c7b-glycosides-g-ux548c-ux8fc7ux654fux6027ux9f3bux708e-allergic-rhinitis-ar-ux9776ux57faux56e0ux7684ux4ea4ux96c6}{%
\subsubsection{苷类 (Glycosides, G) 和 过敏性鼻炎 (allergic rhinitis, AR) 靶基因的交集}\label{ux82f7ux7c7b-glycosides-g-ux548c-ux8fc7ux654fux6027ux9f3bux708e-allergic-rhinitis-ar-ux9776ux57faux56e0ux7684ux4ea4ux96c6}}

Figure \ref{fig:Baishao-glucosides-targets-intersect-with-AR-related-targets} (下方图) 为图Baishao glucosides targets intersect with AR related targets概览。

\textbf{(对应文件为 \texttt{Figure+Table/Baishao-glucosides-targets-intersect-with-AR-related-targets.pdf})}

\def\@captype{figure}
\begin{center}
\includegraphics[width = 0.9\linewidth]{Figure+Table/Baishao-glucosides-targets-intersect-with-AR-related-targets.pdf}
\caption{Baishao glucosides targets intersect with AR related targets}\label{fig:Baishao-glucosides-targets-intersect-with-AR-related-targets}
\end{center}
\begin{center}\begin{tcolorbox}[colback=gray!10, colframe=gray!50, width=0.9\linewidth, arc=1mm, boxrule=0.5pt]
\textbf{
Intersection
:}

\vspace{0.5em}

    NOS2, PIK3CG, PTGS1, PTGS2

\vspace{2em}
\end{tcolorbox}
\end{center}

\textbf{(上述信息框内容已保存至 \texttt{Figure+Table/Baishao-glucosides-targets-intersect-with-AR-related-targets-content})}

\hypertarget{ux828dux836fux82f7-paeoniflorin-p-ux548c-ux8fc7ux654fux6027ux9f3bux708e-allergic-rhinitis-ar-ux9776ux57faux56e0ux7684ux4ea4ux96c6}{%
\subsubsection{芍药苷 (Paeoniflorin, P) 和 过敏性鼻炎 (allergic rhinitis, AR) 靶基因的交集}\label{ux828dux836fux82f7-paeoniflorin-p-ux548c-ux8fc7ux654fux6027ux9f3bux708e-allergic-rhinitis-ar-ux9776ux57faux56e0ux7684ux4ea4ux96c6}}

Figure \ref{fig:Paeoniflorin-targets-intersect-with-AR-related-targets} (下方图) 为图Paeoniflorin targets intersect with AR related targets概览。

\textbf{(对应文件为 \texttt{Figure+Table/Paeoniflorin-targets-intersect-with-AR-related-targets.pdf})}

\def\@captype{figure}
\begin{center}
\includegraphics[width = 0.9\linewidth]{Figure+Table/Paeoniflorin-targets-intersect-with-AR-related-targets.pdf}
\caption{Paeoniflorin targets intersect with AR related targets}\label{fig:Paeoniflorin-targets-intersect-with-AR-related-targets}
\end{center}
\begin{center}\begin{tcolorbox}[colback=gray!10, colframe=gray!50, width=0.9\linewidth, arc=1mm, boxrule=0.5pt]
\textbf{
Intersection
:}

\vspace{0.5em}



\vspace{2em}
\end{tcolorbox}
\end{center}

\textbf{(上述信息框内容已保存至 \texttt{Figure+Table/Paeoniflorin-targets-intersect-with-AR-related-targets-content})}

\hypertarget{ux5bccux96c6ux5206ux6790}{%
\subsection{富集分析}\label{ux5bccux96c6ux5206ux6790}}

\hypertarget{ux767dux828dux82f7ux7c7b-glycosides-g-ux4e0e-ar-ux4ea4ux96c6ux57faux56e0ux7684ux5bccux96c6ux5206ux6790}{%
\subsubsection{白芍苷类 (Glycosides, G) 与 AR 交集基因的富集分析}\label{ux767dux828dux82f7ux7c7b-glycosides-g-ux4e0e-ar-ux4ea4ux96c6ux57faux56e0ux7684ux5bccux96c6ux5206ux6790}}

Figure \ref{fig:Gly-Interect-genes-KEGG-enrichment} (下方图) 为图Gly Interect genes KEGG enrichment概览。

\textbf{(对应文件为 \texttt{Figure+Table/Gly-Interect-genes-KEGG-enrichment.pdf})}

\def\@captype{figure}
\begin{center}
\includegraphics[width = 0.9\linewidth]{Figure+Table/Gly-Interect-genes-KEGG-enrichment.pdf}
\caption{Gly Interect genes KEGG enrichment}\label{fig:Gly-Interect-genes-KEGG-enrichment}
\end{center}

Figure \ref{fig:Gly-Interect-genes-GO-enrichment} (下方图) 为图Gly Interect genes GO enrichment概览。

\textbf{(对应文件为 \texttt{Figure+Table/Gly-Interect-genes-GO-enrichment.pdf})}

\def\@captype{figure}
\begin{center}
\includegraphics[width = 0.9\linewidth]{Figure+Table/Gly-Interect-genes-GO-enrichment.pdf}
\caption{Gly Interect genes GO enrichment}\label{fig:Gly-Interect-genes-GO-enrichment}
\end{center}

\hypertarget{ux5206ux5b50ux5bf9ux63a5}{%
\subsection{分子对接}\label{ux5206ux5b50ux5bf9ux63a5}}

对接的对象为: SOX18, USP5

\hypertarget{ux828dux836fux82f7-paeoniflorin-p}{%
\subsubsection{芍药苷 (Paeoniflorin, P)}\label{ux828dux836fux82f7-paeoniflorin-p}}

Figure \ref{fig:Overall-combining-Affinity} (下方图) 为图Overall combining Affinity概览。

\textbf{(对应文件为 \texttt{Figure+Table/Overall-combining-Affinity.pdf})}

\def\@captype{figure}
\begin{center}
\includegraphics[width = 0.9\linewidth]{Figure+Table/Overall-combining-Affinity.pdf}
\caption{Overall combining Affinity}\label{fig:Overall-combining-Affinity}
\end{center}

Figure \ref{fig:Paeoniflorin-combine-USP5} (下方图) 为图Paeoniflorin combine USP5概览。

\textbf{(对应文件为 \texttt{Figure+Table/442534\_into\_2dag.png})}

\def\@captype{figure}
\begin{center}
\includegraphics[width = 0.9\linewidth]{./figs/442534_into_2dag.png}
\caption{Paeoniflorin combine USP5}\label{fig:Paeoniflorin-combine-USP5}
\end{center}

Figure \ref{fig:Paeoniflorin-combine-SOX18} (下方图) 为图Paeoniflorin combine SOX18概览。

\textbf{(对应文件为 \texttt{Figure+Table/442534\_into\_sox18.png})}

\def\@captype{figure}
\begin{center}
\includegraphics[width = 0.9\linewidth]{./figs/442534_into_sox18.png}
\caption{Paeoniflorin combine SOX18}\label{fig:Paeoniflorin-combine-SOX18}
\end{center}

\hypertarget{bibliography}{%
\section*{Reference}\label{bibliography}}
\addcontentsline{toc}{section}{Reference}

\hypertarget{refs}{}
\begin{cslreferences}
\leavevmode\hypertarget{ref-PubchemSubstanKimS2015}{}%
1. Kim, S. \emph{et al.} PubChem substance and compound databases. \emph{Nucleic Acids Research} (2015).

\leavevmode\hypertarget{ref-ClassyfireAutDjoumb2016}{}%
2. Djoumbou Feunang, Y. \emph{et al.} ClassyFire: Automated chemical classification with a comprehensive, computable taxonomy. \emph{Journal of Cheminformatics} \textbf{8}, 61 (2016).

\leavevmode\hypertarget{ref-ClusterprofilerWuTi2021}{}%
3. Wu, T. \emph{et al.} ClusterProfiler 4.0: A universal enrichment tool for interpreting omics data. \emph{The Innovation} \textbf{2}, (2021).

\leavevmode\hypertarget{ref-TcmspADatabaRuJi2014}{}%
4. Ru, J. \emph{et al.} TCMSP: A database of systems pharmacology for drug discovery from herbal medicines. \emph{Journal of cheminformatics} \textbf{6}, (2014).

\leavevmode\hypertarget{ref-AutodockVina1Eberha2021}{}%
5. Eberhardt, J., Santos-Martins, D., Tillack, A. F. \& Forli, S. AutoDock vina 1.2.0: New docking methods, expanded force field, and python bindings. \emph{Journal of Chemical Information and Modeling} \textbf{61}, 3891--3898 (2021).
\end{cslreferences}

\end{document}
