% Options for packages loaded elsewhere
\PassOptionsToPackage{unicode}{hyperref}
\PassOptionsToPackage{hyphens}{url}
%
\documentclass[
]{article}
\usepackage{lmodern}
\usepackage{amssymb,amsmath}
\usepackage{ifxetex,ifluatex}
\ifnum 0\ifxetex 1\fi\ifluatex 1\fi=0 % if pdftex
  \usepackage[T1]{fontenc}
  \usepackage[utf8]{inputenc}
  \usepackage{textcomp} % provide euro and other symbols
\else % if luatex or xetex
  \usepackage{unicode-math}
  \defaultfontfeatures{Scale=MatchLowercase}
  \defaultfontfeatures[\rmfamily]{Ligatures=TeX,Scale=1}
\fi
% Use upquote if available, for straight quotes in verbatim environments
\IfFileExists{upquote.sty}{\usepackage{upquote}}{}
\IfFileExists{microtype.sty}{% use microtype if available
  \usepackage[]{microtype}
  \UseMicrotypeSet[protrusion]{basicmath} % disable protrusion for tt fonts
}{}
\makeatletter
\@ifundefined{KOMAClassName}{% if non-KOMA class
  \IfFileExists{parskip.sty}{%
    \usepackage{parskip}
  }{% else
    \setlength{\parindent}{0pt}
    \setlength{\parskip}{6pt plus 2pt minus 1pt}}
}{% if KOMA class
  \KOMAoptions{parskip=half}}
\makeatother
\usepackage{xcolor}
\IfFileExists{xurl.sty}{\usepackage{xurl}}{} % add URL line breaks if available
\IfFileExists{bookmark.sty}{\usepackage{bookmark}}{\usepackage{hyperref}}
\hypersetup{
  hidelinks,
  pdfcreator={LaTeX via pandoc}}
\urlstyle{same} % disable monospaced font for URLs
\usepackage[margin=1in]{geometry}
\usepackage{longtable,booktabs}
% Correct order of tables after \paragraph or \subparagraph
\usepackage{etoolbox}
\makeatletter
\patchcmd\longtable{\par}{\if@noskipsec\mbox{}\fi\par}{}{}
\makeatother
% Allow footnotes in longtable head/foot
\IfFileExists{footnotehyper.sty}{\usepackage{footnotehyper}}{\usepackage{footnote}}
\makesavenoteenv{longtable}
\usepackage{graphicx}
\makeatletter
\def\maxwidth{\ifdim\Gin@nat@width>\linewidth\linewidth\else\Gin@nat@width\fi}
\def\maxheight{\ifdim\Gin@nat@height>\textheight\textheight\else\Gin@nat@height\fi}
\makeatother
% Scale images if necessary, so that they will not overflow the page
% margins by default, and it is still possible to overwrite the defaults
% using explicit options in \includegraphics[width, height, ...]{}
\setkeys{Gin}{width=\maxwidth,height=\maxheight,keepaspectratio}
% Set default figure placement to htbp
\makeatletter
\def\fps@figure{htbp}
\makeatother
\setlength{\emergencystretch}{3em} % prevent overfull lines
\providecommand{\tightlist}{%
  \setlength{\itemsep}{0pt}\setlength{\parskip}{0pt}}
\setcounter{secnumdepth}{5}
\usepackage{caption} \captionsetup{font={footnotesize},width=6in} \renewcommand{\dblfloatpagefraction}{.9} \makeatletter \renewenvironment{figure} {\def\@captype{figure}} \makeatother \newenvironment{Shaded}{\begin{snugshade}}{\end{snugshade}} \definecolor{shadecolor}{RGB}{230,230,230} \usepackage{xeCJK} \usepackage{setspace} \setstretch{1.3} \usepackage{tcolorbox} \setcounter{secnumdepth}{4} \setcounter{tocdepth}{4} \usepackage{wallpaper} \usepackage[absolute]{textpos} \tcbuselibrary{breakable} \renewenvironment{Shaded} {\begin{tcolorbox}[colback = gray!10, colframe = gray!40, width = 16cm, arc = 1mm, auto outer arc, title = {Input}]} {\end{tcolorbox}} \usepackage{titlesec} \titleformat{\paragraph} {\fontsize{10pt}{0pt}\bfseries} {\arabic{section}.\arabic{subsection}.\arabic{subsubsection}.\arabic{paragraph}} {1em} {} []
\newlength{\cslhangindent}
\setlength{\cslhangindent}{1.5em}
\newenvironment{cslreferences}%
  {}%
  {\par}

\author{}
\date{\vspace{-2.5em}}

\begin{document}

\begin{titlepage} \newgeometry{top=7.5cm}
\ThisCenterWallPaper{1.12}{../cover_page.pdf}
\begin{center} \textbf{\Huge 白芍网络药理学}
\vspace{4em} \begin{textblock}{10}(3,5.9) \huge
\textbf{\textcolor{white}{2024-01-04}}
\end{textblock} \begin{textblock}{10}(3,7.3)
\Large \textcolor{black}{LiChuang Huang}
\end{textblock} \begin{textblock}{10}(3,11.3)
\Large \textcolor{black}{@立效研究院}
\end{textblock} \end{center} \end{titlepage}
\restoregeometry

\pagenumbering{roman}

\tableofcontents

\listoffigures

\listoftables

\newpage

\pagenumbering{arabic}

\hypertarget{abstract}{%
\section{摘要}\label{abstract}}

\hypertarget{ux9700ux6c42ux548cux7ed3ux679c}{%
\subsection{需求和结果}\label{ux9700ux6c42ux548cux7ed3ux679c}}

\begin{itemize}
\tightlist
\item
  白芍总苷 Total glucosides of paeony 中主要化学成分10-20个(TCMSP筛选下口服利用度等)及各个化学成分对应的作用靶点
  (gene与AR过敏性鼻炎相关),最终形成drug-chemical-target gene靶点图
\item
  将获得的靶点进行GO, KEGG富集分析,目标靶点为USP5,关联成分为芍药苷Paeoniflorin
\item
  将芍药苷pae单独拎出,形成pae-targets-pathway网络,此处形成的target genes的GO、KEGG富集图也需要,
  备注USP5参与哪些部分(功能、通路)
\item
  分子对接模拟芍药苷与USP5互作
\item
  转至第2步目标靶点为SOX18,关联成分为芍药苷Paeoniflorin
\item
  第3步中备注SOX18参与哪些部分(功能、通路)
\item
  分子对接模拟芍药苷与SOX18互作
\end{itemize}

注:USP5 和 SOX18 不参与功能、通路。其它分析结果见 \ref{workflow}

\hypertarget{introduction}{%
\section{前言}\label{introduction}}

\hypertarget{methods}{%
\section{材料和方法}\label{methods}}

\hypertarget{ux6750ux6599}{%
\subsection{材料}\label{ux6750ux6599}}

\hypertarget{ux65b9ux6cd5}{%
\subsection{方法}\label{ux65b9ux6cd5}}

Mainly used method:

\begin{itemize}
\tightlist
\item
  R package \texttt{ClusterProfiler} used for gene enrichment analysis.\textsuperscript{\protect\hyperlink{ref-ClusterprofilerWuTi2021}{1}}
\item
  The API of \texttt{UniProtKB} (\url{https://www.uniprot.org/help/api_queries}) used for mapping of names or IDs of proteins .
\item
  R package \texttt{PubChemR} used for querying compounds information .
\item
  Web tool of \texttt{SwissTargetPrediction} used for drug-targets prediction.\textsuperscript{\protect\hyperlink{ref-SwisstargetpredDaina2019}{2}}
\item
  Website \texttt{TCMSP} \url{https://tcmsp-e.com/tcmsp.php} used for data source.\textsuperscript{\protect\hyperlink{ref-TcmspADatabaRuJi2014}{3}}
\item
  \texttt{AutoDock\ vina} used for molecular docking.\textsuperscript{\protect\hyperlink{ref-AutodockVina1Eberha2021}{4}}
\item
  The Human Gene Database \texttt{GeneCards} used for disease related genes prediction.\textsuperscript{\protect\hyperlink{ref-TheGenecardsSStelze2016}{5}}
\item
  R package \texttt{biomaRt} used for gene annotation.\textsuperscript{\protect\hyperlink{ref-MappingIdentifDurinc2009}{6}}
\item
  Other R packages (eg., \texttt{dplyr} and \texttt{ggplot2}) used for statistic analysis or data visualization.
\end{itemize}

\hypertarget{results}{%
\section{分析结果}\label{results}}

\hypertarget{dis}{%
\section{结论}\label{dis}}

\hypertarget{workflow}{%
\section{附:分析流程}\label{workflow}}

\hypertarget{tcmsp-ux767dux828dux6210ux5206ux83b7ux53d6}{%
\subsection{TCMSP 白芍成分获取}\label{tcmsp-ux767dux828dux6210ux5206ux83b7ux53d6}}

Table \ref{tab:Baishao-Compounds-and-targets} (下方表格) 为表格Baishao Compounds and targets概览。

\textbf{(对应文件为 \texttt{Figure+Table/Baishao-Compounds-and-targets.xlsx})}

\begin{center}\begin{tcolorbox}[colback=gray!10, colframe=gray!50, width=0.9\linewidth, arc=1mm, boxrule=0.5pt]注:表格共有990行3列,以下预览的表格可能省略部分数据;表格含有39个唯一`Mol ID'。
\end{tcolorbox}
\end{center}

\begin{longtable}[]{@{}lll@{}}
\caption{\label{tab:Baishao-Compounds-and-targets}Baishao Compounds and targets}\tabularnewline
\toprule
Mol ID & Molecule Name & Target name\tabularnewline
\midrule
\endfirsthead
\toprule
Mol ID & Molecule Name & Target name\tabularnewline
\midrule
\endhead
MOL001246 & (1R)-()-Nopinone & Gamma-aminobutyric-acid rec\ldots{}\tabularnewline
MOL001246 & (1R)-()-Nopinone & Cytochrome P450-cam\tabularnewline
MOL001246 & (1R)-()-Nopinone & Lysozyme\tabularnewline
MOL001246 & (1R)-()-Nopinone & Alcohol dehydrogenase 1C\tabularnewline
MOL001246 & (1R)-()-Nopinone & Nicotinate-nucleotide--dime\ldots{}\tabularnewline
MOL001393 & myristic acid & Prostaglandin G/H synthase 1\tabularnewline
MOL001393 & myristic acid & Prostaglandin G/H synthase 2\tabularnewline
MOL001393 & myristic acid & Cholinesterase\tabularnewline
MOL001393 & myristic acid & Phospholipase A2\tabularnewline
MOL001393 & myristic acid & Rhinovirus coat protein\tabularnewline
MOL001393 & myristic acid & Ig gamma-1 chain C region\tabularnewline
MOL001393 & myristic acid & Ferrichrome-iron receptor\tabularnewline
MOL001393 & myristic acid & 3-oxoacyl-{[}acyl-carrier-pro\ldots{}\tabularnewline
MOL001393 & myristic acid & Nuclear receptor coactivator 2\tabularnewline
MOL001393 & myristic acid & Nuclear receptor coactivator 1\tabularnewline
\ldots{} & \ldots{} & \ldots{}\tabularnewline
\bottomrule
\end{longtable}

\hypertarget{ux767dux828dux603bux82f7-total-glucosides-of-paeony-tgp-ux6210ux5206}{%
\subsection{白芍总苷 (Total glucosides of paeony, TGP) 成分}\label{ux767dux828dux603bux82f7-total-glucosides-of-paeony-tgp-ux6210ux5206}}

\hypertarget{ux767dux828dux603bux82f7-total-glucosides-of-paeony-tgp-ux6210ux5206ux548cux7b5bux9009}{%
\subsubsection{白芍总苷 (Total glucosides of paeony, TGP) 成分和筛选}\label{ux767dux828dux603bux82f7-total-glucosides-of-paeony-tgp-ux6210ux5206ux548cux7b5bux9009}}

根据提供的文献,搜集其中的白芍总苷 (Total glucosides of paeony, TGP)\textsuperscript{\protect\hyperlink{ref-TotalGlucosideJiang2020}{7}}。

\begin{center}\begin{tcolorbox}[colback=gray!10, colframe=gray!50, width=0.9\linewidth, arc=1mm, boxrule=0.5pt]
\textbf{
TGP
:}

\vspace{0.5em}

    442534, 51346141, 21631105, 21631106, 138113866,
14605198, 50163461, 102000323, 494717, 138108175,
124079396, 101382399, 102516499, 71452334, 137705343

\vspace{2em}
\end{tcolorbox}
\end{center}

以 \texttt{PubChemR} 获取这些化合物的同义名:

\begin{center}\begin{tcolorbox}[colback=gray!10, colframe=gray!50, width=0.9\linewidth, arc=1mm, boxrule=0.5pt]
\textbf{
442534
:}

\vspace{0.5em}

    Paeoniflorin, 23180-57-6, Peoniflorin, Paeonia moutan,
NSC 178886, UNII-21AIQ4EV64, 21AIQ4EV64, CCRIS 6494, EINECS
245-476-2, PAEONIFLORIN (USP-RS), PAEONIFLORIN [USP-RS],
NSC-178886,
((2S,2aR,2a1S,3aR,4R,5aR)-4-Hydroxy-2-methyl-2a-(((2S,3R,4S,5S,6R)-3,4,5-trihydroxy-6-(hydroxymethyl)tetrahydro-2...

\vspace{2em}


\textbf{
51346141
:}

\vspace{0.5em}

    Albiflorin, 39011-90-0, SCHEMBL24008597, AC-34702

\vspace{2em}


\textbf{
21631105
:}

\vspace{0.5em}

    Oxypaeoniflorin, Oxypaeoniflora, 39011-91-1,
UNII-3A7O4NBD5S, 3A7O4NBD5S, OXYPEONIFLORIN, NSC 258310,
NSC-258310, J17.727J, beta-D-GLUCOPYRANOSIDE,
(1AR,2S,3AR,5R,5AR,5BS)-TETRAHYDRO-5-HYDROXY-5B-(((4-HYDROXYBENZOYL)OXY)METHYL)-2-METHYL-2,5-METHANO-1H-3,4-DIOXACYCLOBUTA(CD)PENTALEN-1A(2H)-YL,
bet...

\vspace{2em}


\textbf{
21631106
:}

\vspace{0.5em}

    Benzoylpaeoniflorin, 38642-49-8, CHEMBL4861111,
CHEBI:69583, HMS3886L18, MFCD00869479, s9149,
AKOS037645102, CCG-270143, AC-34005, AS-57134, Q27137925

\vspace{2em}


\textbf{
138113866
:}

\vspace{0.5em}

    A866179,
-D-Glucopyranoside,tetrahydro-5-hydroxy-5b-[[(4-hydroxybenzoyl)oxy]methyl]-2-methyl-2,5-methano-1H-3,4-dioxacyclobuta[cd]pentalen-1a(2H)-yl,6-benzoate,[1aR-(1aa,2b,3aa,5a,5aa,5ba)]-

\vspace{2em}


\textbf{
(Others)
:}

\vspace{0.5em}

    ...

\vspace{2em}
\end{tcolorbox}
\end{center}

根据同义名,在 Tab. \ref{tab:Baishao-Compounds-and-targets} 中搜索这些化合物,得到:

Table \ref{tab:TCMSP-Baishao-the-found-TGP} (下方表格) 为表格TCMSP Baishao the found TGP概览。

\textbf{(对应文件为 \texttt{Figure+Table/TCMSP-Baishao-the-found-TGP.xlsx})}

\begin{center}\begin{tcolorbox}[colback=gray!10, colframe=gray!50, width=0.9\linewidth, arc=1mm, boxrule=0.5pt]注:表格共有85行15列,以下预览的表格可能省略部分数据;表格含有1个唯一`Herb\_pinyin\_name'。
\end{tcolorbox}
\end{center}

\begin{longtable}[]{@{}llllllllll@{}}
\caption{\label{tab:TCMSP-Baishao-the-found-TGP}TCMSP Baishao the found TGP}\tabularnewline
\toprule
Herb\_p\ldots{} & Mol ID & Molecu\ldots\ldots3 & Molecu\ldots\ldots4 & MW & AlogP & Hdon & Hacc & OB (\%) & Caco-2\tabularnewline
\midrule
\endfirsthead
\toprule
Herb\_p\ldots{} & Mol ID & Molecu\ldots\ldots3 & Molecu\ldots\ldots4 & MW & AlogP & Hdon & Hacc & OB (\%) & Caco-2\tabularnewline
\midrule
\endhead
Baishao & MOL000106 & PYG & https:\ldots{} & 126.12 & 1.03 & 3 & 3 & 22.98 & 0.69\tabularnewline
Baishao & MOL001218 & Pisol & https:\ldots{} & 186.38 & 4.62 & 1 & 1 & 18.5 & 1.23\tabularnewline
Baishao & MOL001246 & (1R)-(\ldots{} & https:\ldots{} & 138.23 & 1.52 & 0 & 1 & 57.86 & 1.23\tabularnewline
Baishao & MOL001393 & myrist\ldots{} & https:\ldots{} & 228.42 & 5.46 & 1 & 2 & 21.18 & 1.07\tabularnewline
Baishao & MOL001396 & PENTAD\ldots{} & https:\ldots{} & 242.45 & 5.91 & 1 & 2 & 20.18 & 1.08\tabularnewline
Baishao & MOL001402 & Octaco\ldots{} & https:\ldots{} & 394.86 & 13.15 & 0 & 0 & 8.15 & 1.91\tabularnewline
Baishao & MOL001644 & Dodecanal & https:\ldots{} & 184.36 & 4.59 & 0 & 1 & 21.52 & 1.4\tabularnewline
Baishao & MOL001801 & salicy\ldots{} & https:\ldots{} & 138.13 & 1.17 & 2 & 3 & 32.13 & 0.63\tabularnewline
Baishao & MOL001888 & 2,2-di\ldots{} & https:\ldots{} & 128.24 & 2.09 & 1 & 1 & 82.54 & 1.22\tabularnewline
Baishao & MOL001889 & Methyl\ldots{} & https:\ldots{} & 294.53 & 6.64 & 0 & 2 & 41.93 & 1.46\tabularnewline
Baishao & MOL001890 & octade\ldots{} & https:\ldots{} & 252.54 & 8.14 & 0 & 0 & 19.5 & 1.87\tabularnewline
Baishao & MOL001891 & 9-meth\ldots{} & https:\ldots{} & 178.24 & 3.55 & 0 & 0 & 26.87 & 1.95\tabularnewline
Baishao & MOL001892 & Diprop\ldots{} & https:\ldots{} & 250.32 & 3.29 & 0 & 4 & 66.3 & 0.78\tabularnewline
Baishao & MOL001893 & BU3 & https:\ldots{} & 90.14 & -0.14 & 2 & 2 & 34.87 & 0.19\tabularnewline
Baishao & MOL001894 & Bicetyl & https:\ldots{} & 450.98 & 14.97 & 0 & 0 & 8.03 & 1.96\tabularnewline
\ldots{} & \ldots{} & \ldots{} & \ldots{} & \ldots{} & \ldots{} & \ldots{} & \ldots{} & \ldots{} & \ldots{}\tabularnewline
\bottomrule
\end{longtable}

根据 OB、DL 筛选:

Figure \ref{fig:Filterd-TGP} (下方图) 为图Filterd TGP概览。

\textbf{(对应文件为 \texttt{Figure+Table/Filterd-TGP.pdf})}

\def\@captype{figure}
\begin{center}
\includegraphics[width = 0.9\linewidth]{Figure+Table/Filterd-TGP.pdf}
\caption{Filterd TGP}\label{fig:Filterd-TGP}
\end{center}

\hypertarget{ux767dux828dux603bux82f7-total-glucosides-of-paeony-tgp-ux6210ux5206ux7684ux9776ux70b9ux9884ux6d4b}{%
\subsubsection{白芍总苷 (Total glucosides of paeony, TGP) 成分的靶点预测}\label{ux767dux828dux603bux82f7-total-glucosides-of-paeony-tgp-ux6210ux5206ux7684ux9776ux70b9ux9884ux6d4b}}

通过 \texttt{SwissTargetPrediction} 预测靶点。

Figure \ref{fig:SwissTargetPrediction-results} (下方图) 为图SwissTargetPrediction results概览。

\textbf{(对应文件为 \texttt{Figure+Table/SwissTargetPrediction-results.pdf})}

\def\@captype{figure}
\begin{center}
\includegraphics[width = 0.9\linewidth]{Figure+Table/SwissTargetPrediction-results.pdf}
\caption{SwissTargetPrediction results}\label{fig:SwissTargetPrediction-results}
\end{center}

\hypertarget{ux767dux828dux603bux82f7-total-glucosides-of-paeony-tgp-ux7684ux7f51ux7edcux836fux7406ux5b66ux5206ux6790}{%
\subsection{白芍总苷 (Total glucosides of paeony, TGP) 的网络药理学分析}\label{ux767dux828dux603bux82f7-total-glucosides-of-paeony-tgp-ux7684ux7f51ux7edcux836fux7406ux5b66ux5206ux6790}}

\hypertarget{ux767dux828dux603bux82f7-total-glucosides-of-paeony-tgp-ux6210ux5206-ux9776ux70b9}{%
\subsubsection{白芍总苷 (Total glucosides of paeony, TGP) 成分-靶点}\label{ux767dux828dux603bux82f7-total-glucosides-of-paeony-tgp-ux6210ux5206-ux9776ux70b9}}

Figure \ref{fig:Network-pharmacology-visualization} (下方图) 为图Network pharmacology visualization概览。

\textbf{(对应文件为 \texttt{Figure+Table/Network-pharmacology-visualization.pdf})}

\def\@captype{figure}
\begin{center}
\includegraphics[width = 0.9\linewidth]{Figure+Table/Network-pharmacology-visualization.pdf}
\caption{Network pharmacology visualization}\label{fig:Network-pharmacology-visualization}
\end{center}

\hypertarget{ux767dux828dux603bux82f7-total-glucosides-of-paeony-tgp-ux548c-ux8fc7ux654fux6027ux9f3bux708e-allergic-rhinitis-ar-ux9776ux57faux56e0ux7684ux4ea4ux96c6}{%
\subsubsection{白芍总苷 (Total glucosides of paeony, TGP) 和 过敏性鼻炎 (allergic rhinitis, AR) 靶基因的交集}\label{ux767dux828dux603bux82f7-total-glucosides-of-paeony-tgp-ux548c-ux8fc7ux654fux6027ux9f3bux708e-allergic-rhinitis-ar-ux9776ux57faux56e0ux7684ux4ea4ux96c6}}

Figure \ref{fig:Baishao-TGP-targets-intersect-with-AR-related-targets} (下方图) 为图Baishao TGP targets intersect with AR related targets概览。

\textbf{(对应文件为 \texttt{Figure+Table/Baishao-TGP-targets-intersect-with-AR-related-targets.pdf})}

\def\@captype{figure}
\begin{center}
\includegraphics[width = 0.9\linewidth]{Figure+Table/Baishao-TGP-targets-intersect-with-AR-related-targets.pdf}
\caption{Baishao TGP targets intersect with AR related targets}\label{fig:Baishao-TGP-targets-intersect-with-AR-related-targets}
\end{center}
\begin{center}\begin{tcolorbox}[colback=gray!10, colframe=gray!50, width=0.9\linewidth, arc=1mm, boxrule=0.5pt]
\textbf{
Intersection
:}

\vspace{0.5em}

    LGALS3, EGFR, VEGFA, CYP2D6, SELP, SERPINE1, PIK3CG,
MMP9, ITK, ADRB2, STAT3, PTGS2

\vspace{2em}
\end{tcolorbox}
\end{center}

\textbf{(上述信息框内容已保存至 \texttt{Figure+Table/Baishao-TGP-targets-intersect-with-AR-related-targets-content})}

Figure \ref{fig:Targets-of-compounds-and-related-disease} (下方图) 为图Targets of compounds and related disease概览。

\textbf{(对应文件为 \texttt{Figure+Table/Targets-of-compounds-and-related-disease.pdf})}

\def\@captype{figure}
\begin{center}
\includegraphics[width = 0.9\linewidth]{Figure+Table/Targets-of-compounds-and-related-disease.pdf}
\caption{Targets of compounds and related disease}\label{fig:Targets-of-compounds-and-related-disease}
\end{center}

\hypertarget{ux828dux836fux82f7-paeoniflorin-p-ux548c-ux8fc7ux654fux6027ux9f3bux708e-allergic-rhinitis-ar-ux9776ux57faux56e0ux7684ux4ea4ux96c6}{%
\subsubsection{芍药苷 (Paeoniflorin, P) 和 过敏性鼻炎 (allergic rhinitis, AR) 靶基因的交集}\label{ux828dux836fux82f7-paeoniflorin-p-ux548c-ux8fc7ux654fux6027ux9f3bux708e-allergic-rhinitis-ar-ux9776ux57faux56e0ux7684ux4ea4ux96c6}}

Figure \ref{fig:Paeoniflorin-targets-intersect-with-AR-related-targets} (下方图) 为图Paeoniflorin targets intersect with AR related targets概览。

\textbf{(对应文件为 \texttt{Figure+Table/Paeoniflorin-targets-intersect-with-AR-related-targets.pdf})}

\def\@captype{figure}
\begin{center}
\includegraphics[width = 0.9\linewidth]{Figure+Table/Paeoniflorin-targets-intersect-with-AR-related-targets.pdf}
\caption{Paeoniflorin targets intersect with AR related targets}\label{fig:Paeoniflorin-targets-intersect-with-AR-related-targets}
\end{center}
\begin{center}\begin{tcolorbox}[colback=gray!10, colframe=gray!50, width=0.9\linewidth, arc=1mm, boxrule=0.5pt]
\textbf{
Intersection
:}

\vspace{0.5em}

    LGALS3, VEGFA, SERPINE1, SELP, ADRB2, CYP2D6, STAT3,
PTGS2

\vspace{2em}
\end{tcolorbox}
\end{center}

\textbf{(上述信息框内容已保存至 \texttt{Figure+Table/Paeoniflorin-targets-intersect-with-AR-related-targets-content})}

Figure \ref{fig:Network-pharmacology-visualization-of-Paeoniflorin} (下方图) 为图Network pharmacology visualization of Paeoniflorin概览。

\textbf{(对应文件为 \texttt{Figure+Table/Network-pharmacology-visualization-of-Paeoniflorin.pdf})}

\def\@captype{figure}
\begin{center}
\includegraphics[width = 0.9\linewidth]{Figure+Table/Network-pharmacology-visualization-of-Paeoniflorin.pdf}
\caption{Network pharmacology visualization of Paeoniflorin}\label{fig:Network-pharmacology-visualization-of-Paeoniflorin}
\end{center}

\hypertarget{ux5bccux96c6ux5206ux6790}{%
\subsection{富集分析}\label{ux5bccux96c6ux5206ux6790}}

\hypertarget{ux767dux828dux603bux82f7-total-glucosides-of-paeony-tgp-ux4e0e-ar-ux4ea4ux96c6ux57faux56e0ux7684ux5bccux96c6ux5206ux6790}{%
\subsubsection{白芍总苷 (Total glucosides of paeony, TGP) 与 AR 交集基因的富集分析}\label{ux767dux828dux603bux82f7-total-glucosides-of-paeony-tgp-ux4e0e-ar-ux4ea4ux96c6ux57faux56e0ux7684ux5bccux96c6ux5206ux6790}}

Figure \ref{fig:TGP-Interect-genes-KEGG-enrichment} (下方图) 为图TGP Interect genes KEGG enrichment概览。

\textbf{(对应文件为 \texttt{Figure+Table/TGP-Interect-genes-KEGG-enrichment.pdf})}

\def\@captype{figure}
\begin{center}
\includegraphics[width = 0.9\linewidth]{Figure+Table/TGP-Interect-genes-KEGG-enrichment.pdf}
\caption{TGP Interect genes KEGG enrichment}\label{fig:TGP-Interect-genes-KEGG-enrichment}
\end{center}

Figure \ref{fig:TGP-Interect-genes-GO-enrichment} (下方图) 为图TGP Interect genes GO enrichment概览。

\textbf{(对应文件为 \texttt{Figure+Table/TGP-Interect-genes-GO-enrichment.pdf})}

\def\@captype{figure}
\begin{center}
\includegraphics[width = 0.9\linewidth]{Figure+Table/TGP-Interect-genes-GO-enrichment.pdf}
\caption{TGP Interect genes GO enrichment}\label{fig:TGP-Interect-genes-GO-enrichment}
\end{center}

\hypertarget{ux828dux836fux82f7-paeoniflorin-p-ux4e0e-ar-ux4ea4ux96c6ux57faux56e0ux7684ux5bccux96c6ux5206ux6790}{%
\subsubsection{芍药苷 (Paeoniflorin, P) 与 AR 交集基因的富集分析}\label{ux828dux836fux82f7-paeoniflorin-p-ux4e0e-ar-ux4ea4ux96c6ux57faux56e0ux7684ux5bccux96c6ux5206ux6790}}

Figure \ref{fig:Pae-Interect-genes-KEGG-enrichment} (下方图) 为图Pae Interect genes KEGG enrichment概览。

\textbf{(对应文件为 \texttt{Figure+Table/Pae-Interect-genes-KEGG-enrichment.pdf})}

\def\@captype{figure}
\begin{center}
\includegraphics[width = 0.9\linewidth]{Figure+Table/Pae-Interect-genes-KEGG-enrichment.pdf}
\caption{Pae Interect genes KEGG enrichment}\label{fig:Pae-Interect-genes-KEGG-enrichment}
\end{center}

Figure \ref{fig:Pae-Interect-genes-GO-enrichment} (下方图) 为图Pae Interect genes GO enrichment概览。

\textbf{(对应文件为 \texttt{Figure+Table/Pae-Interect-genes-GO-enrichment.pdf})}

\def\@captype{figure}
\begin{center}
\includegraphics[width = 0.9\linewidth]{Figure+Table/Pae-Interect-genes-GO-enrichment.pdf}
\caption{Pae Interect genes GO enrichment}\label{fig:Pae-Interect-genes-GO-enrichment}
\end{center}

\hypertarget{ux5206ux5b50ux5bf9ux63a5}{%
\subsection{分子对接}\label{ux5206ux5b50ux5bf9ux63a5}}

对接的对象为: SOX18, USP5

\hypertarget{ux828dux836fux82f7-paeoniflorin-p}{%
\subsubsection{芍药苷 (Paeoniflorin, P)}\label{ux828dux836fux82f7-paeoniflorin-p}}

Figure \ref{fig:Overall-combining-Affinity} (下方图) 为图Overall combining Affinity概览。

\textbf{(对应文件为 \texttt{Figure+Table/Overall-combining-Affinity.pdf})}

\def\@captype{figure}
\begin{center}
\includegraphics[width = 0.9\linewidth]{Figure+Table/Overall-combining-Affinity.pdf}
\caption{Overall combining Affinity}\label{fig:Overall-combining-Affinity}
\end{center}

Figure \ref{fig:Paeoniflorin-combine-USP5} (下方图) 为图Paeoniflorin combine USP5概览。

\textbf{(对应文件为 \texttt{Figure+Table/442534\_into\_2dag.png})}

\def\@captype{figure}
\begin{center}
\includegraphics[width = 0.9\linewidth]{./figs/442534_into_2dag.png}
\caption{Paeoniflorin combine USP5}\label{fig:Paeoniflorin-combine-USP5}
\end{center}

Figure \ref{fig:Paeoniflorin-combine-SOX18} (下方图) 为图Paeoniflorin combine SOX18概览。

\textbf{(对应文件为 \texttt{Figure+Table/442534\_into\_sox18.png})}

\def\@captype{figure}
\begin{center}
\includegraphics[width = 0.9\linewidth]{./figs/442534_into_sox18.png}
\caption{Paeoniflorin combine SOX18}\label{fig:Paeoniflorin-combine-SOX18}
\end{center}

\hypertarget{bibliography}{%
\section*{Reference}\label{bibliography}}
\addcontentsline{toc}{section}{Reference}

\hypertarget{refs}{}
\begin{cslreferences}
\leavevmode\hypertarget{ref-ClusterprofilerWuTi2021}{}%
1. Wu, T. \emph{et al.} ClusterProfiler 4.0: A universal enrichment tool for interpreting omics data. \emph{The Innovation} \textbf{2}, (2021).

\leavevmode\hypertarget{ref-SwisstargetpredDaina2019}{}%
2. Daina, A., Michielin, O. \& Zoete, V. SwissTargetPrediction: Updated data and new features for efficient prediction of protein targets of small molecules. \emph{Nucleic Acids Research} \textbf{47}, W357--W364 (2019).

\leavevmode\hypertarget{ref-TcmspADatabaRuJi2014}{}%
3. Ru, J. \emph{et al.} TCMSP: A database of systems pharmacology for drug discovery from herbal medicines. \emph{Journal of cheminformatics} \textbf{6}, (2014).

\leavevmode\hypertarget{ref-AutodockVina1Eberha2021}{}%
4. Eberhardt, J., Santos-Martins, D., Tillack, A. F. \& Forli, S. AutoDock vina 1.2.0: New docking methods, expanded force field, and python bindings. \emph{Journal of Chemical Information and Modeling} \textbf{61}, 3891--3898 (2021).

\leavevmode\hypertarget{ref-TheGenecardsSStelze2016}{}%
5. Stelzer, G. \emph{et al.} The genecards suite: From gene data mining to disease genome sequence analyses. \emph{Current protocols in bioinformatics} \textbf{54}, 1.30.1--1.30.33 (2016).

\leavevmode\hypertarget{ref-MappingIdentifDurinc2009}{}%
6. Durinck, S., Spellman, P. T., Birney, E. \& Huber, W. Mapping identifiers for the integration of genomic datasets with the r/bioconductor package biomaRt. \emph{Nature protocols} \textbf{4}, 1184--1191 (2009).

\leavevmode\hypertarget{ref-TotalGlucosideJiang2020}{}%
7. Jiang, H. \emph{et al.} Total glucosides of paeony: A review of its phytochemistry, role in autoimmune diseases, and mechanisms of action. \emph{Journal of Ethnopharmacology} \textbf{258}, (2020).
\end{cslreferences}

\end{document}
