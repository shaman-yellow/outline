% Options for packages loaded elsewhere
\PassOptionsToPackage{unicode}{hyperref}
\PassOptionsToPackage{hyphens}{url}
%
\documentclass[
]{article}
\usepackage{lmodern}
\usepackage{amssymb,amsmath}
\usepackage{ifxetex,ifluatex}
\ifnum 0\ifxetex 1\fi\ifluatex 1\fi=0 % if pdftex
  \usepackage[T1]{fontenc}
  \usepackage[utf8]{inputenc}
  \usepackage{textcomp} % provide euro and other symbols
\else % if luatex or xetex
  \usepackage{unicode-math}
  \defaultfontfeatures{Scale=MatchLowercase}
  \defaultfontfeatures[\rmfamily]{Ligatures=TeX,Scale=1}
\fi
% Use upquote if available, for straight quotes in verbatim environments
\IfFileExists{upquote.sty}{\usepackage{upquote}}{}
\IfFileExists{microtype.sty}{% use microtype if available
  \usepackage[]{microtype}
  \UseMicrotypeSet[protrusion]{basicmath} % disable protrusion for tt fonts
}{}
\makeatletter
\@ifundefined{KOMAClassName}{% if non-KOMA class
  \IfFileExists{parskip.sty}{%
    \usepackage{parskip}
  }{% else
    \setlength{\parindent}{0pt}
    \setlength{\parskip}{6pt plus 2pt minus 1pt}}
}{% if KOMA class
  \KOMAoptions{parskip=half}}
\makeatother
\usepackage{xcolor}
\IfFileExists{xurl.sty}{\usepackage{xurl}}{} % add URL line breaks if available
\IfFileExists{bookmark.sty}{\usepackage{bookmark}}{\usepackage{hyperref}}
\hypersetup{
  pdftitle={Analysis},
  pdfauthor={Huang LiChuang of Wie-Biotech},
  hidelinks,
  pdfcreator={LaTeX via pandoc}}
\urlstyle{same} % disable monospaced font for URLs
\usepackage[margin=1in]{geometry}
\usepackage{longtable,booktabs}
% Correct order of tables after \paragraph or \subparagraph
\usepackage{etoolbox}
\makeatletter
\patchcmd\longtable{\par}{\if@noskipsec\mbox{}\fi\par}{}{}
\makeatother
% Allow footnotes in longtable head/foot
\IfFileExists{footnotehyper.sty}{\usepackage{footnotehyper}}{\usepackage{footnote}}
\makesavenoteenv{longtable}
\usepackage{graphicx}
\makeatletter
\def\maxwidth{\ifdim\Gin@nat@width>\linewidth\linewidth\else\Gin@nat@width\fi}
\def\maxheight{\ifdim\Gin@nat@height>\textheight\textheight\else\Gin@nat@height\fi}
\makeatother
% Scale images if necessary, so that they will not overflow the page
% margins by default, and it is still possible to overwrite the defaults
% using explicit options in \includegraphics[width, height, ...]{}
\setkeys{Gin}{width=\maxwidth,height=\maxheight,keepaspectratio}
% Set default figure placement to htbp
\makeatletter
\def\fps@figure{htbp}
\makeatother
\setlength{\emergencystretch}{3em} % prevent overfull lines
\providecommand{\tightlist}{%
  \setlength{\itemsep}{0pt}\setlength{\parskip}{0pt}}
\setcounter{secnumdepth}{5}
\usepackage{caption} \captionsetup{font={footnotesize},width=6in} \renewcommand{\dblfloatpagefraction}{.9} \makeatletter \renewenvironment{figure} {\def\@captype{figure}} \makeatother \definecolor{shadecolor}{RGB}{242,242,242} \usepackage{xeCJK} \usepackage{setspace} \setstretch{1.3} \usepackage{tcolorbox}

\title{Analysis}
\author{Huang LiChuang of Wie-Biotech}
\date{}

\begin{document}
\maketitle

{
\setcounter{tocdepth}{3}
\tableofcontents
}
\listoffigures

\listoftables

\hypertarget{abstract}{%
\section{摘要}\label{abstract}}

建立肺癌和奥沙利铂(Oxaliplatin)疗法相关的预后工具。

大致方案如下:

\begin{itemize}
\tightlist
\item
  使用 CellMiner 数据库的 NCI-60 数据集,筛选与奥沙利铂处理敏感的基因。
\item
  使用 TCGA 数据库,获取肺癌的病患者 mRNA 数据集。
\item
  以随机森林、LASSO 回归等方式筛选 TCGA 数据集的关键基因,并建立回归模型。
\item
  以生存分析等方式验证结果。
\item
  得到用于预测肺癌和奥沙利铂疗法预后的预测模型(用于计算指标)。
\end{itemize}

\hypertarget{route}{%
\section{研究设计流程图}\label{route}}

\includegraphics[width=\linewidth]{output_files/figure-latex/unnamed-chunk-4-1}

\hypertarget{methods}{%
\section{材料和方法}\label{methods}}

\hypertarget{results}{%
\section{分析结果}\label{results}}

Figure \ref{fig:CellMiner-for--Drug-sensitivity-analysis}为图CellMiner for Drug sensitivity analysis概览。

\textbf{(对应文件为 \texttt{../2023\_06\_25\_fix/figs/pearsonTest.pdf})}

\def\@captype{figure}
\begin{center}
\includegraphics[width = 0.9\linewidth]{../2023_06_25_fix/figs/pearsonTest.pdf}
\caption{CellMiner for  Drug sensitivity analysis}\label{fig:CellMiner-for--Drug-sensitivity-analysis}
\end{center}

Figure \ref{fig:EFS-for-select-features}为图EFS for select features概览。

\textbf{(对应文件为 \texttt{../2023\_07\_24\_base/Figure+Table/EFS-top30-genes.pdf})}

\def\@captype{figure}
\begin{center}
\includegraphics[width = 0.9\linewidth]{../2023_07_24_base/Figure+Table/EFS-top30-genes.pdf}
\caption{EFS for select features}\label{fig:EFS-for-select-features}
\end{center}

Figure \ref{fig:LASSO-models}为图LASSO models概览。

\textbf{(对应文件为 \texttt{../2023\_07\_24\_base/Figure+Table/LASSO-model.pdf})}

\def\@captype{figure}
\begin{center}
\includegraphics[width = 0.9\linewidth]{../2023_07_24_base/Figure+Table/LASSO-model.pdf}
\caption{LASSO models}\label{fig:LASSO-models}
\end{center}

Figure \ref{fig:LASSO-ROC}为图LASSO ROC概览。

\textbf{(对应文件为 \texttt{../2023\_07\_24\_base/Figure+Table/LASSO-ROC.pdf})}

\def\@captype{figure}
\begin{center}
\includegraphics[width = 0.9\linewidth]{../2023_07_24_base/Figure+Table/LASSO-ROC.pdf}
\caption{LASSO ROC}\label{fig:LASSO-ROC}
\end{center}

Figure \ref{fig:Survival-analysis}为图Survival analysis概览。

\textbf{(对应文件为 \texttt{../2023\_07\_24\_base/Figure+Table/Survival-analysis-of-PGK1.pdf})}

\def\@captype{figure}
\begin{center}
\includegraphics[width = 0.9\linewidth]{../2023_07_24_base/Figure+Table/Survival-analysis-of-PGK1.pdf}
\caption{Survival analysis}\label{fig:Survival-analysis}
\end{center}

\hypertarget{dis}{%
\section{结论}\label{dis}}

\end{document}
