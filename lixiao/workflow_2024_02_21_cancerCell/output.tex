% Options for packages loaded elsewhere
\PassOptionsToPackage{unicode}{hyperref}
\PassOptionsToPackage{hyphens}{url}
%
\documentclass[
]{article}
\usepackage{lmodern}
\usepackage{amssymb,amsmath}
\usepackage{ifxetex,ifluatex}
\ifnum 0\ifxetex 1\fi\ifluatex 1\fi=0 % if pdftex
  \usepackage[T1]{fontenc}
  \usepackage[utf8]{inputenc}
  \usepackage{textcomp} % provide euro and other symbols
\else % if luatex or xetex
  \usepackage{unicode-math}
  \defaultfontfeatures{Scale=MatchLowercase}
  \defaultfontfeatures[\rmfamily]{Ligatures=TeX,Scale=1}
\fi
% Use upquote if available, for straight quotes in verbatim environments
\IfFileExists{upquote.sty}{\usepackage{upquote}}{}
\IfFileExists{microtype.sty}{% use microtype if available
  \usepackage[]{microtype}
  \UseMicrotypeSet[protrusion]{basicmath} % disable protrusion for tt fonts
}{}
\makeatletter
\@ifundefined{KOMAClassName}{% if non-KOMA class
  \IfFileExists{parskip.sty}{%
    \usepackage{parskip}
  }{% else
    \setlength{\parindent}{0pt}
    \setlength{\parskip}{6pt plus 2pt minus 1pt}}
}{% if KOMA class
  \KOMAoptions{parskip=half}}
\makeatother
\usepackage{xcolor}
\IfFileExists{xurl.sty}{\usepackage{xurl}}{} % add URL line breaks if available
\IfFileExists{bookmark.sty}{\usepackage{bookmark}}{\usepackage{hyperref}}
\hypersetup{
  hidelinks,
  pdfcreator={LaTeX via pandoc}}
\urlstyle{same} % disable monospaced font for URLs
\usepackage[margin=1in]{geometry}
\usepackage{color}
\usepackage{fancyvrb}
\newcommand{\VerbBar}{|}
\newcommand{\VERB}{\Verb[commandchars=\\\{\}]}
\DefineVerbatimEnvironment{Highlighting}{Verbatim}{commandchars=\\\{\}}
% Add ',fontsize=\small' for more characters per line
\usepackage{framed}
\definecolor{shadecolor}{RGB}{248,248,248}
\newenvironment{Shaded}{\begin{snugshade}}{\end{snugshade}}
\newcommand{\AlertTok}[1]{\textcolor[rgb]{0.94,0.16,0.16}{#1}}
\newcommand{\AnnotationTok}[1]{\textcolor[rgb]{0.56,0.35,0.01}{\textbf{\textit{#1}}}}
\newcommand{\AttributeTok}[1]{\textcolor[rgb]{0.77,0.63,0.00}{#1}}
\newcommand{\BaseNTok}[1]{\textcolor[rgb]{0.00,0.00,0.81}{#1}}
\newcommand{\BuiltInTok}[1]{#1}
\newcommand{\CharTok}[1]{\textcolor[rgb]{0.31,0.60,0.02}{#1}}
\newcommand{\CommentTok}[1]{\textcolor[rgb]{0.56,0.35,0.01}{\textit{#1}}}
\newcommand{\CommentVarTok}[1]{\textcolor[rgb]{0.56,0.35,0.01}{\textbf{\textit{#1}}}}
\newcommand{\ConstantTok}[1]{\textcolor[rgb]{0.00,0.00,0.00}{#1}}
\newcommand{\ControlFlowTok}[1]{\textcolor[rgb]{0.13,0.29,0.53}{\textbf{#1}}}
\newcommand{\DataTypeTok}[1]{\textcolor[rgb]{0.13,0.29,0.53}{#1}}
\newcommand{\DecValTok}[1]{\textcolor[rgb]{0.00,0.00,0.81}{#1}}
\newcommand{\DocumentationTok}[1]{\textcolor[rgb]{0.56,0.35,0.01}{\textbf{\textit{#1}}}}
\newcommand{\ErrorTok}[1]{\textcolor[rgb]{0.64,0.00,0.00}{\textbf{#1}}}
\newcommand{\ExtensionTok}[1]{#1}
\newcommand{\FloatTok}[1]{\textcolor[rgb]{0.00,0.00,0.81}{#1}}
\newcommand{\FunctionTok}[1]{\textcolor[rgb]{0.00,0.00,0.00}{#1}}
\newcommand{\ImportTok}[1]{#1}
\newcommand{\InformationTok}[1]{\textcolor[rgb]{0.56,0.35,0.01}{\textbf{\textit{#1}}}}
\newcommand{\KeywordTok}[1]{\textcolor[rgb]{0.13,0.29,0.53}{\textbf{#1}}}
\newcommand{\NormalTok}[1]{#1}
\newcommand{\OperatorTok}[1]{\textcolor[rgb]{0.81,0.36,0.00}{\textbf{#1}}}
\newcommand{\OtherTok}[1]{\textcolor[rgb]{0.56,0.35,0.01}{#1}}
\newcommand{\PreprocessorTok}[1]{\textcolor[rgb]{0.56,0.35,0.01}{\textit{#1}}}
\newcommand{\RegionMarkerTok}[1]{#1}
\newcommand{\SpecialCharTok}[1]{\textcolor[rgb]{0.00,0.00,0.00}{#1}}
\newcommand{\SpecialStringTok}[1]{\textcolor[rgb]{0.31,0.60,0.02}{#1}}
\newcommand{\StringTok}[1]{\textcolor[rgb]{0.31,0.60,0.02}{#1}}
\newcommand{\VariableTok}[1]{\textcolor[rgb]{0.00,0.00,0.00}{#1}}
\newcommand{\VerbatimStringTok}[1]{\textcolor[rgb]{0.31,0.60,0.02}{#1}}
\newcommand{\WarningTok}[1]{\textcolor[rgb]{0.56,0.35,0.01}{\textbf{\textit{#1}}}}
\usepackage{longtable,booktabs}
% Correct order of tables after \paragraph or \subparagraph
\usepackage{etoolbox}
\makeatletter
\patchcmd\longtable{\par}{\if@noskipsec\mbox{}\fi\par}{}{}
\makeatother
% Allow footnotes in longtable head/foot
\IfFileExists{footnotehyper.sty}{\usepackage{footnotehyper}}{\usepackage{footnote}}
\makesavenoteenv{longtable}
\usepackage{graphicx}
\makeatletter
\def\maxwidth{\ifdim\Gin@nat@width>\linewidth\linewidth\else\Gin@nat@width\fi}
\def\maxheight{\ifdim\Gin@nat@height>\textheight\textheight\else\Gin@nat@height\fi}
\makeatother
% Scale images if necessary, so that they will not overflow the page
% margins by default, and it is still possible to overwrite the defaults
% using explicit options in \includegraphics[width, height, ...]{}
\setkeys{Gin}{width=\maxwidth,height=\maxheight,keepaspectratio}
% Set default figure placement to htbp
\makeatletter
\def\fps@figure{htbp}
\makeatother
\setlength{\emergencystretch}{3em} % prevent overfull lines
\providecommand{\tightlist}{%
  \setlength{\itemsep}{0pt}\setlength{\parskip}{0pt}}
\setcounter{secnumdepth}{5}
\usepackage{caption} \captionsetup{font={footnotesize},width=6in} \renewcommand{\dblfloatpagefraction}{.9} \makeatletter \renewenvironment{figure} {\def\@captype{figure}} \makeatother \@ifundefined{Shaded}{\newenvironment{Shaded}} \@ifundefined{snugshade}{\newenvironment{snugshade}} \renewenvironment{Shaded}{\begin{snugshade}}{\end{snugshade}} \definecolor{shadecolor}{RGB}{230,230,230} \usepackage{xeCJK} \usepackage{setspace} \setstretch{1.3} \usepackage{tcolorbox} \setcounter{secnumdepth}{4} \setcounter{tocdepth}{4} \usepackage{wallpaper} \usepackage[absolute]{textpos} \tcbuselibrary{breakable} \renewenvironment{Shaded} {\begin{tcolorbox}[colback = gray!10, colframe = gray!40, width = 16cm, arc = 1mm, auto outer arc, title = {R input}]} {\end{tcolorbox}} \usepackage{titlesec} \titleformat{\paragraph} {\fontsize{10pt}{0pt}\bfseries} {\arabic{section}.\arabic{subsection}.\arabic{subsubsection}.\arabic{paragraph}} {1em} {} []
\newlength{\cslhangindent}
\setlength{\cslhangindent}{1.5em}
\newenvironment{cslreferences}%
  {}%
  {\par}

\author{}
\date{\vspace{-2.5em}}

\begin{document}

\begin{titlepage} \newgeometry{top=7.5cm}
\ThisCenterWallPaper{1.12}{~/outline/lixiao//cover_page.pdf}
\begin{center} \textbf{\Huge Step 系列:scRNA-seq
癌细胞鉴定} \vspace{4em}
\begin{textblock}{10}(3,5.9) \huge
\textbf{\textcolor{white}{2024-02-21}}
\end{textblock} \begin{textblock}{10}(3,7.3)
\Large \textcolor{black}{LiChuang Huang}
\end{textblock} \begin{textblock}{10}(3,11.3)
\Large \textcolor{black}{@立效研究院}
\end{textblock} \end{center} \end{titlepage}
\restoregeometry

\pagenumbering{roman}

\tableofcontents

\listoffigures

\listoftables

\newpage

\pagenumbering{arabic}

\hypertarget{abstract}{%
\section{摘要}\label{abstract}}

\hypertarget{ux76eeux7684}{%
\subsection{目的}\label{ux76eeux7684}}

解决癌组织单细胞数据集中,癌细胞鉴定的难题,并延续一般性的单细胞数据分析流程。

\hypertarget{ux89e3ux51b3ux7684ux95eeux9898-ux6280ux672fux6027ux7684}{%
\subsection{解决的问题 (技术性的)}\label{ux89e3ux51b3ux7684ux95eeux9898-ux6280ux672fux6027ux7684}}

不同的 R 包或其他工具之间的数据转换和衔接。

\hypertarget{ux524dux60c5ux8d44ux6599}{%
\section{前情资料}\label{ux524dux60c5ux8d44ux6599}}

这份文档是以下的补充资料 (以下的配置和使用是前提条件) :

\begin{itemize}
\tightlist
\item
  《Step 系列:scRNA-seq 基本分析》
\end{itemize}

如果你还不知道 Step 系列的基本特性以及一些泛用的提取和存储方法,请先阅读:

\begin{itemize}
\tightlist
\item
  《Step 系列:Prologue and Get-start》
\end{itemize}

\hypertarget{ux9002ux914dux6027}{%
\section{适配性}\label{ux9002ux914dux6027}}

同《Step 系列:scRNA-seq 基本分析》。

\hypertarget{ux65b9ux6cd5}{%
\section{方法}\label{ux65b9ux6cd5}}

以下是我在这个工作流中涉及的方法和程序:

Mainly used method:

\begin{itemize}
\tightlist
\item
  R package \texttt{copyKAT} used for aneuploid cell or cancer cell prediction\textsuperscript{\protect\hyperlink{ref-DelineatingCopGaoR2021}{1}}.
\item
  R package \texttt{Monocle3} used for cell pseudotime analysis\textsuperscript{\protect\hyperlink{ref-ReversedGraphQiuX2017}{2},\protect\hyperlink{ref-TheDynamicsAnTrapne2014}{3}}.
\item
  The R package \texttt{Seurat} used for scRNA-seq processing; \texttt{SCSA} (python) used for cell type annotation\textsuperscript{\protect\hyperlink{ref-IntegratedAnalHaoY2021}{4}--\protect\hyperlink{ref-ScsaACellTyCaoY2020}{6}}.
\item
  Other R packages (eg., \texttt{dplyr} and \texttt{ggplot2}) used for statistic analysis or data visualization.
\end{itemize}

\hypertarget{ux5b89ux88c5-ux9996ux6b21ux4f7fux7528}{%
\section{安装 (首次使用)}\label{ux5b89ux88c5-ux9996ux6b21ux4f7fux7528}}

\hypertarget{ux5b89ux88c5ux4f9dux8d56}{%
\subsection{安装依赖}\label{ux5b89ux88c5ux4f9dux8d56}}

《Step 系列:scRNA-seq 基本分析》已经例举了大部分程序的安装方法,
以下,仅展示 R 包 \texttt{copykat} 的安装。

\begin{Shaded}
\begin{Highlighting}[]
\NormalTok{devtools}\OperatorTok{::}\KeywordTok{install\_github}\NormalTok{(}\StringTok{"navinlabcode/copykat"}\NormalTok{)}
\end{Highlighting}
\end{Shaded}

\hypertarget{ux793aux4f8bux5206ux6790}{%
\section{示例分析}\label{ux793aux4f8bux5206ux6790}}

在《Step 系列:scRNA-seq 基本分析》中,我以 GSE171306 做了 scRNA-seq 一般性的示例。
但其实 GSE171306 是一批癌组织数据集,理应鉴定癌细胞,而 SCSA 和多数其他自动注释工具,
都无法鉴定出癌细胞。这样,就有必要专门鉴定癌细胞了。

以下针对癌细胞鉴定的问题展开示例分析,并依然使用 GSE171306 这批数据。

注:只要是 《Step 系列:scRNA-seq 基本分析》 中提及的,以下就不再赘述;只细述
新的内容。

\hypertarget{ux6570ux636eux51c6ux5907}{%
\subsection{数据准备}\label{ux6570ux636eux51c6ux5907}}

\hypertarget{obtain}{%
\subsubsection{快速获取示例数据}\label{obtain}}

运行以下代码获取数据 (和《Step 系列:scRNA-seq 基本分析》中的是一样的):

\begin{Shaded}
\begin{Highlighting}[]
\NormalTok{geo \textless{}{-}}\StringTok{ }\KeywordTok{job\_geo}\NormalTok{(}\StringTok{"GSE171306"}\NormalTok{)}
\NormalTok{geo \textless{}{-}}\StringTok{ }\KeywordTok{step1}\NormalTok{(geo)}
\NormalTok{geo \textless{}{-}}\StringTok{ }\KeywordTok{step2}\NormalTok{(geo)}
\KeywordTok{untar}\NormalTok{(}\StringTok{"./GSE171306/GSE171306\_RAW.tar"}\NormalTok{, }\DataTypeTok{exdir =} \StringTok{"./GSE171306"}\NormalTok{)}
\KeywordTok{prepare\_10x}\NormalTok{(}\StringTok{"./GSE171306/"}\NormalTok{, }\StringTok{"ccRCC1"}\NormalTok{, }\DataTypeTok{single =}\NormalTok{ F)}
\NormalTok{sr \textless{}{-}}\StringTok{ }\KeywordTok{job\_seurat}\NormalTok{(}\StringTok{"./GSE171306/GSM5222644\_ccRCC1\_barcodes"}\NormalTok{)}
\end{Highlighting}
\end{Shaded}

\hypertarget{ux5206ux6790ux6d41ux7a0b}{%
\subsection{分析流程}\label{ux5206ux6790ux6d41ux7a0b}}

\hypertarget{ux5355ux7ec6ux80deux6570ux636eux7684ux8d28ux63a7ux805aux7c7bmarker-ux9274ux5b9aux7ec6ux80deux6ce8ux91caux7b49}{%
\subsubsection{单细胞数据的质控、聚类、Marker 鉴定、细胞注释等}\label{ux5355ux7ec6ux80deux6570ux636eux7684ux8d28ux63a7ux805aux7c7bmarker-ux9274ux5b9aux7ec6ux80deux6ce8ux91caux7b49}}

以下直接给出代码 (和《Step 系列:scRNA-seq 基本分析》相同):

\begin{Shaded}
\begin{Highlighting}[]
\NormalTok{sr \textless{}{-}}\StringTok{ }\KeywordTok{step1}\NormalTok{(sr)}
\NormalTok{sr \textless{}{-}}\StringTok{ }\KeywordTok{step2}\NormalTok{(sr, }\DecValTok{0}\NormalTok{, }\DecValTok{7500}\NormalTok{, }\DecValTok{35}\NormalTok{)}
\NormalTok{sr \textless{}{-}}\StringTok{ }\KeywordTok{step3}\NormalTok{(sr, }\DecValTok{1}\OperatorTok{:}\DecValTok{15}\NormalTok{, }\FloatTok{1.2}\NormalTok{)}
\NormalTok{sr \textless{}{-}}\StringTok{ }\KeywordTok{step4}\NormalTok{(sr, }\StringTok{""}\NormalTok{)}
\NormalTok{sr \textless{}{-}}\StringTok{ }\KeywordTok{step5}\NormalTok{(sr, }\DecValTok{5}\NormalTok{)}
\NormalTok{sr \textless{}{-}}\StringTok{ }\KeywordTok{step6}\NormalTok{(sr, }\StringTok{"Kidney"}\NormalTok{)}
\end{Highlighting}
\end{Shaded}

\hypertarget{ux764cux7ec6ux80deux9274ux5b9a}{%
\subsubsection{癌细胞鉴定}\label{ux764cux7ec6ux80deux9274ux5b9a}}

\hypertarget{as-job-kat-ux5c06ux524dux5904ux7406ux5b8cux6bd5ux7684-job_seurat-ux6570ux636eux5bf9ux8c61ux8f6cux5316}{%
\paragraph{\texorpdfstring{As-job-kat 将前处理完毕的 \texttt{job\_seurat} 数据对象转化}{As-job-kat 将前处理完毕的 job\_seurat 数据对象转化}}\label{as-job-kat-ux5c06ux524dux5904ux7406ux5b8cux6bd5ux7684-job_seurat-ux6570ux636eux5bf9ux8c61ux8f6cux5316}}

\begin{Shaded}
\begin{Highlighting}[]
\NormalTok{kat \textless{}{-}}\StringTok{ }\KeywordTok{asjob\_kat}\NormalTok{(sr)}
\end{Highlighting}
\end{Shaded}

注意,这一步默认将 \texttt{sr@object} (也就是 \texttt{seurat} 数据对象) 中的 \texttt{assays} 数据槽中的第一个数据集
用于鉴定癌细胞。转化完成后,你会在 \texttt{kat@object} 看到这个矩阵数据集。
一般情况下,使用的都是第一个数据集。

你可以手动指定数据集,例如:

\begin{Shaded}
\begin{Highlighting}[]
\CommentTok{\# 不要运行}
\NormalTok{kat \textless{}{-}}\StringTok{ }\KeywordTok{asjob\_kat}\NormalTok{(sr, }\DataTypeTok{use =} \KeywordTok{names}\NormalTok{(x}\OperatorTok{@}\NormalTok{object}\OperatorTok{@}\NormalTok{assays)[[}\DecValTok{1}\NormalTok{]])}
\end{Highlighting}
\end{Shaded}

\hypertarget{step1-ux6839ux636eux53d8ux5f02ux62f7ux8d1dux6570ux9274ux5b9aux764cux7ec6ux80de}{%
\paragraph{Step1 根据变异拷贝数鉴定癌细胞}\label{step1-ux6839ux636eux53d8ux5f02ux62f7ux8d1dux6570ux9274ux5b9aux764cux7ec6ux80de}}

\texttt{copykat} 有一个优点 (相对于 \texttt{inferCNV}) ,不需要手动指定参考细胞,程序会自主在数据集中选择参考细胞。
所以,将所有的细胞表达数据输入就可以了。

(这一步会比较耗时,1-5 小时)

\begin{Shaded}
\begin{Highlighting}[]
\CommentTok{\# 后一个参数指定线程数}
\NormalTok{kat \textless{}{-}}\StringTok{ }\KeywordTok{step1}\NormalTok{(kat, }\DecValTok{8}\NormalTok{)}
\CommentTok{\# 你还可以通过添加 \textasciigrave{}path\textasciigrave{} 参数,指定其他文件夹用以存放中间数据}
\end{Highlighting}
\end{Shaded}

这里,我有必要做一些说明。
\texttt{copykat} 内置一些参考数据集,用以参考和计算。对于人类,它内置的是 `hg20' 参考集。
目前遇到更多的可能是 `hg38'。
\texttt{copykat} 内部并不会对基因符号 (symbol) 后缀的版本信息进行替换和匹配,因此,如果输入的基因即使是
同一种,然而因为版本不同,也会无法匹配上 (例如,你输入的是 `AHR.5',内置的参考集包含的是 `AHR.1') 。

因此,我在 \texttt{job\_kat} 的 \texttt{step1} 中补充了这一部分工作,能够让这一步能够顺利进行下去。
但我无法预料是否还会有其他特殊情况,所以以上事项需知悉。

如果你同样对以上事项保持警惕,可以用以下确认我做了哪些补充工作:

\begin{Shaded}
\begin{Highlighting}[]
\KeywordTok{selectMethod}\NormalTok{(step1, }\StringTok{"job\_kat"}\NormalTok{)}
\end{Highlighting}
\end{Shaded}

我还需要备注的是,由于我只做到过人类的癌细胞鉴定,所以小鼠的数据集目前还不支持。

\hypertarget{step2-ux53efux89c6ux5316ux9274ux5b9aux7ed3ux679c}{%
\paragraph{Step2 可视化鉴定结果}\label{step2-ux53efux89c6ux5316ux9274ux5b9aux7ed3ux679c}}

\texttt{copykat} 自带可视化的函数;然而,由于其中的图例绘制的太糟糕,因此,这里
我去除了 \texttt{copykat} 绘制的热图的图例,重新添加了一个新的图例。

这一步也会比较耗时 (热图很大) 。

\begin{Shaded}
\begin{Highlighting}[]
\NormalTok{kat \textless{}{-}}\StringTok{ }\KeywordTok{step2}\NormalTok{(kat)}
\end{Highlighting}
\end{Shaded}

该热图可以直接提取查看;但最好不要这么做,因为加载这个热图太耗时:

\begin{itemize}
\tightlist
\item
  \texttt{kat@plots\$step2\$p.copykat}
\end{itemize}

推荐的做法是 (见 \ref{clear} ),运行下一部分 (\texttt{clear}) 之后,再将输出的 png 图片查看。

这里,我们也可以直接获取鉴定结果表格:

\begin{Shaded}
\begin{Highlighting}[]
\NormalTok{kat}\OperatorTok{@}\NormalTok{tables}\OperatorTok{$}\NormalTok{step2}\OperatorTok{$}\NormalTok{res\_copykat}
\end{Highlighting}
\end{Shaded}

\begin{longtable}[]{@{}lll@{}}
\caption{\label{tab:copyKAT-results}CopyKAT results}\tabularnewline
\toprule
cell.names & copykat.pred & copykat\_cell\tabularnewline
\midrule
\endfirsthead
\toprule
cell.names & copykat.pred & copykat\_cell\tabularnewline
\midrule
\endhead
AAACCCAAGCTGTGCC-1 & diploid & Normal cell\tabularnewline
AAACCCACAGCCGGTT-1 & diploid & Normal cell\tabularnewline
AAACCCATCATGAGGG-1 & diploid & Normal cell\tabularnewline
AAACGAAAGTCACAGG-1 & diploid & Normal cell\tabularnewline
AAACGAACACACAGAG-1 & diploid & Normal cell\tabularnewline
AAACGAACACCTCTAC-1 & diploid & Normal cell\tabularnewline
AAACGAATCAACACGT-1 & diploid & Normal cell\tabularnewline
AAACGAATCACTACTT-1 & diploid & Normal cell\tabularnewline
AAACGAATCGGCATAT-1 & aneuploid & Cancer cell\tabularnewline
AAACGCTAGACAGTCG-1 & diploid & Normal cell\tabularnewline
AAACGCTAGTCCCAAT-1 & aneuploid & Cancer cell\tabularnewline
AAACGCTCATCAGTCA-1 & diploid & Normal cell\tabularnewline
AAACGCTCATGACTCA-1 & diploid & Normal cell\tabularnewline
AAACGCTGTAGGACCA-1 & diploid & Normal cell\tabularnewline
AAACGCTGTGAGCAGT-1 & diploid & Normal cell\tabularnewline
\ldots{} & \ldots{} & \ldots{}\tabularnewline
\bottomrule
\end{longtable}

\hypertarget{clear}{%
\paragraph{\texorpdfstring{(额外的) 保存 \texttt{job\_kat} 并输出结果}{(额外的) 保存 job\_kat 并输出结果}}\label{clear}}

\begin{Shaded}
\begin{Highlighting}[]
\NormalTok{kat \textless{}{-}}\StringTok{ }\KeywordTok{clear}\NormalTok{(kat)}
\end{Highlighting}
\end{Shaded}

这样,就能取得想要的结果了:

\def\@captype{figure}
\begin{center}
\includegraphics[width = 0.9\linewidth]{figs/copykat2024-02-21_15_28_39.png}
\caption{Copykat prediction}\label{fig:copykat-prediction}
\end{center}

Fig. \ref{fig:copykat-prediction},图中的 `aneuploidy' 即为癌细胞。

\hypertarget{map}{%
\paragraph{Map 将结果映射回 Seurat}\label{map}}

\begin{Shaded}
\begin{Highlighting}[]
\NormalTok{sr \textless{}{-}}\StringTok{ }\KeywordTok{map}\NormalTok{(sr, kat)}
\NormalTok{p.sr\_vis \textless{}{-}}\StringTok{ }\KeywordTok{vis}\NormalTok{(sr, }\StringTok{"scsa\_copykat"}\NormalTok{)}
\NormalTok{p.sr\_vis}
\end{Highlighting}
\end{Shaded}

这样我们就能在 Fig. \ref{fig:The-scsa-copykat} 中看到,被注释为 `Cancer cell' 的细胞群体。

\def\@captype{figure}
\begin{center}
\includegraphics[width = 0.9\linewidth]{Figure+Table/The-scsa-copykat.pdf}
\caption{The scsa copykat}\label{fig:The-scsa-copykat}
\end{center}

我们不妨对比一下 SCSA 的注释结果 (Fig. \ref{fig:SCSA-Cell-type-annotation}),看看 `Cancer cell' 可能的来源细胞。

\def\@captype{figure}
\begin{center}
\includegraphics[width = 0.9\linewidth]{Figure+Table/SCSA-Cell-type-annotation.pdf}
\caption{SCSA Cell type annotation}\label{fig:SCSA-Cell-type-annotation}
\end{center}

现在可以推断,`Cancer cell' 主要来源于 `Proximal tubular cell'。

\hypertarget{ux62dfux65f6ux5206ux6790}{%
\subsubsection{拟时分析}\label{ux62dfux65f6ux5206ux6790}}

其实到 \ref{map} 为止,本文档的主要内容,即鉴定癌细胞,已经结束了。

然而如果我们进一步思考,如果将 \texttt{copykat} 依据变异拷贝数鉴定癌细胞的原理推进,结合拟时分析,
也许就能分析出,癌细胞或正常细胞是如何沿着 `拟时轨迹',转变成癌细胞了。

这会是一种可以泛用于肿瘤组织单细胞数据的分析方法。

\hypertarget{do-monocle-ux5bf9ux764cux7ec6ux80deux8fdbux884cux62dfux65f6ux5206ux6790}{%
\paragraph{do-monocle 对癌细胞进行拟时分析}\label{do-monocle-ux5bf9ux764cux7ec6ux80deux8fdbux884cux62dfux65f6ux5206ux6790}}

按照上述思路,这里需要将癌细胞单独取出,用以拟时分析。

我提供了一个便捷的方法,以快速达成这一目的:

\begin{Shaded}
\begin{Highlighting}[]
\NormalTok{mn \textless{}{-}}\StringTok{ }\KeywordTok{do\_monocle}\NormalTok{(sr, kat)}
\end{Highlighting}
\end{Shaded}

在整个 Step 系列方法中,\texttt{do\_*} 形式的目前还很少;这是一种根据传入的前两个参数的类来决定调用的函数
的系列方法。而 \texttt{asjob\_*} 形式的,只根据第一种参数来决定。

其实,\texttt{do\_monocle} 内部重新对传入的 \texttt{sr} 数据运行了 \texttt{step3},即对分离的癌细胞重新聚类,区分出更多的群体 (可能会是亚型)

\hypertarget{step1-ux6784ux5efaux62dfux65f6ux8f68ux8ff9}{%
\paragraph{Step1 构建拟时轨迹}\label{step1-ux6784ux5efaux62dfux65f6ux8f68ux8ff9}}

\begin{Shaded}
\begin{Highlighting}[]
\NormalTok{mn \textless{}{-}}\StringTok{ }\KeywordTok{step1}\NormalTok{(mn)}
\end{Highlighting}
\end{Shaded}

Fig. \ref{fig:Cancer-cell-prin} 是用来确认选择拟时起点的。

\begin{Shaded}
\begin{Highlighting}[]
\CommentTok{\# 以 wrap 调整了长宽比例}
\KeywordTok{wrap}\NormalTok{(mn}\OperatorTok{@}\NormalTok{plots}\OperatorTok{$}\NormalTok{step1}\OperatorTok{$}\NormalTok{p.prin, }\DecValTok{6}\NormalTok{, }\DecValTok{4}\NormalTok{)}
\end{Highlighting}
\end{Shaded}

\def\@captype{figure}
\begin{center}
\includegraphics[width = 0.9\linewidth]{Figure+Table/Cancer-cell-prin.pdf}
\caption{Cancer cell prin}\label{fig:Cancer-cell-prin}
\end{center}

\hypertarget{step2-ux9009ux62e9ux62dfux65f6ux8d77ux70b9}{%
\paragraph{Step2 选择拟时起点}\label{step2-ux9009ux62e9ux62dfux65f6ux8d77ux70b9}}

选择合适的拟时起点依然会是一个难题。这里提供了一种可借鉴的,用以选择癌细胞拟时起点的思路。
见 Fig. \ref{fig:copykat-prediction}, 对于 `aneuploid',即癌细胞,`gain'、`loss' 水平更接近
`diploid' 的细胞,或许更适合作为拟时起点。
具体而言,根据 Fig. \ref{fig:copykat-prediction} 的侧边聚类树,我们或许可以试着将肿瘤细胞切分为多个小群体,然后根据它们相较于
正常细胞的近似程度,选择拟时起点。那么切分之后,哪一群体更加接近正常细胞呢?

见 Fig. \ref{fig:copykat-prediction},如果切分为两个群体,下方高度 (Height) 更低的群体更近似正常细胞。
这样,我们就能大致决定:Height 更低的群体作为拟时起点。

我们可以把 \texttt{copyKAT} 的聚类结果隐射到 UMAP 聚类上:
\texttt{mn} 来自于 \texttt{do\_monocle(sr,\ kat)},相较于《Step 系列:scRNA-seq 基本分析》中所述的,有一点特殊之处,即,
额外生成了图片,也就是我们需要的映射图,只不过,它一共做了 1-30 种切分
(注意,该切分是针对 Fig. \ref{fig:copykat-prediction} 中的所有细胞进行的切分,而不是单单癌细胞) 。

\begin{Shaded}
\begin{Highlighting}[]
\NormalTok{mn}\OperatorTok{@}\NormalTok{plots}\OperatorTok{$}\NormalTok{step1}\OperatorTok{$}\NormalTok{p.cancer\_position}
\end{Highlighting}
\end{Shaded}

\def\@captype{figure}
\begin{center}
\includegraphics[width = 0.9\linewidth]{Figure+Table/Cut-tree.pdf}
\caption{Cut tree}\label{fig:Cut-tree}
\end{center}

见 Fig. \ref{fig:Cut-tree},图中的数值越小,代表 Height 越低。
如果我们观察最后一副子图,会发现 `9' 代表的群体,主要集中在 UMAP 的上半部分。
因此,这里我们试着将更上半部分的细胞作为拟时起点。
那么,也就是 Fig. \ref{fig:Cancer-cell-prin} 中的 `Y\_14'。

\hypertarget{step3-ux62dfux65f6ux5206ux6790ux57faux7840ux4e0aux7684ux5deeux5f02ux5206ux6790ux548cux57faux56e0ux8868ux8fbeux6a21ux5757}{%
\paragraph{Step3 拟时分析基础上的差异分析和基因表达模块}\label{step3-ux62dfux65f6ux5206ux6790ux57faux7840ux4e0aux7684ux5deeux5f02ux5206ux6790ux548cux57faux56e0ux8868ux8fbeux6a21ux5757}}

\hypertarget{ux8fdbux9636-ux6839ux636eux62dfux65f6ux5206ux6790ux7ed3ux679cux91cdux65b0ux5212ux5206ux7ec6ux80deux7fa4ux4f53}{%
\paragraph{(进阶) 根据拟时分析结果重新划分细胞群体}\label{ux8fdbux9636-ux6839ux636eux62dfux65f6ux5206ux6790ux7ed3ux679cux91cdux65b0ux5212ux5206ux7ec6ux80deux7fa4ux4f53}}

\hypertarget{ux5b8cux6574ux793aux4f8bux4ee3ux7801}{%
\subsection{完整示例代码}\label{ux5b8cux6574ux793aux4f8bux4ee3ux7801}}

\hypertarget{bibliography}{%
\section*{Reference}\label{bibliography}}
\addcontentsline{toc}{section}{Reference}

\hypertarget{refs}{}
\begin{cslreferences}
\leavevmode\hypertarget{ref-DelineatingCopGaoR2021}{}%
1. Gao, R. \emph{et al.} Delineating copy number and clonal substructure in human tumors from single-cell transcriptomes. \emph{Nature Biotechnology} \textbf{39}, 599--608 (2021).

\leavevmode\hypertarget{ref-ReversedGraphQiuX2017}{}%
2. Qiu, X. \emph{et al.} Reversed graph embedding resolves complex single-cell trajectories. \emph{Nature Methods} \textbf{14}, (2017).

\leavevmode\hypertarget{ref-TheDynamicsAnTrapne2014}{}%
3. Trapnell, C. \emph{et al.} The dynamics and regulators of cell fate decisions are revealed by pseudotemporal ordering of single cells. \emph{Nature Biotechnology} \textbf{32}, (2014).

\leavevmode\hypertarget{ref-IntegratedAnalHaoY2021}{}%
4. Hao, Y. \emph{et al.} Integrated analysis of multimodal single-cell data. \emph{Cell} \textbf{184}, (2021).

\leavevmode\hypertarget{ref-ComprehensiveIStuart2019}{}%
5. Stuart, T. \emph{et al.} Comprehensive integration of single-cell data. \emph{Cell} \textbf{177}, (2019).

\leavevmode\hypertarget{ref-ScsaACellTyCaoY2020}{}%
6. Cao, Y., Wang, X. \& Peng, G. SCSA: A cell type annotation tool for single-cell rna-seq data. \emph{Frontiers in genetics} \textbf{11}, (2020).
\end{cslreferences}

\end{document}
