% Options for packages loaded elsewhere
\PassOptionsToPackage{unicode}{hyperref}
\PassOptionsToPackage{hyphens}{url}
%
\documentclass[
]{article}
\usepackage{lmodern}
\usepackage{amssymb,amsmath}
\usepackage{ifxetex,ifluatex}
\ifnum 0\ifxetex 1\fi\ifluatex 1\fi=0 % if pdftex
  \usepackage[T1]{fontenc}
  \usepackage[utf8]{inputenc}
  \usepackage{textcomp} % provide euro and other symbols
\else % if luatex or xetex
  \usepackage{unicode-math}
  \defaultfontfeatures{Scale=MatchLowercase}
  \defaultfontfeatures[\rmfamily]{Ligatures=TeX,Scale=1}
\fi
% Use upquote if available, for straight quotes in verbatim environments
\IfFileExists{upquote.sty}{\usepackage{upquote}}{}
\IfFileExists{microtype.sty}{% use microtype if available
  \usepackage[]{microtype}
  \UseMicrotypeSet[protrusion]{basicmath} % disable protrusion for tt fonts
}{}
\makeatletter
\@ifundefined{KOMAClassName}{% if non-KOMA class
  \IfFileExists{parskip.sty}{%
    \usepackage{parskip}
  }{% else
    \setlength{\parindent}{0pt}
    \setlength{\parskip}{6pt plus 2pt minus 1pt}}
}{% if KOMA class
  \KOMAoptions{parskip=half}}
\makeatother
\usepackage{xcolor}
\IfFileExists{xurl.sty}{\usepackage{xurl}}{} % add URL line breaks if available
\IfFileExists{bookmark.sty}{\usepackage{bookmark}}{\usepackage{hyperref}}
\hypersetup{
  pdftitle={Analysis},
  pdfauthor={Huang LiChuang of Wie-Biotech},
  hidelinks,
  pdfcreator={LaTeX via pandoc}}
\urlstyle{same} % disable monospaced font for URLs
\usepackage[margin=1in]{geometry}
\usepackage{color}
\usepackage{fancyvrb}
\newcommand{\VerbBar}{|}
\newcommand{\VERB}{\Verb[commandchars=\\\{\}]}
\DefineVerbatimEnvironment{Highlighting}{Verbatim}{commandchars=\\\{\}}
% Add ',fontsize=\small' for more characters per line
\usepackage{framed}
\definecolor{shadecolor}{RGB}{248,248,248}
\newenvironment{Shaded}{\begin{snugshade}}{\end{snugshade}}
\newcommand{\AlertTok}[1]{\textcolor[rgb]{0.94,0.16,0.16}{#1}}
\newcommand{\AnnotationTok}[1]{\textcolor[rgb]{0.56,0.35,0.01}{\textbf{\textit{#1}}}}
\newcommand{\AttributeTok}[1]{\textcolor[rgb]{0.77,0.63,0.00}{#1}}
\newcommand{\BaseNTok}[1]{\textcolor[rgb]{0.00,0.00,0.81}{#1}}
\newcommand{\BuiltInTok}[1]{#1}
\newcommand{\CharTok}[1]{\textcolor[rgb]{0.31,0.60,0.02}{#1}}
\newcommand{\CommentTok}[1]{\textcolor[rgb]{0.56,0.35,0.01}{\textit{#1}}}
\newcommand{\CommentVarTok}[1]{\textcolor[rgb]{0.56,0.35,0.01}{\textbf{\textit{#1}}}}
\newcommand{\ConstantTok}[1]{\textcolor[rgb]{0.00,0.00,0.00}{#1}}
\newcommand{\ControlFlowTok}[1]{\textcolor[rgb]{0.13,0.29,0.53}{\textbf{#1}}}
\newcommand{\DataTypeTok}[1]{\textcolor[rgb]{0.13,0.29,0.53}{#1}}
\newcommand{\DecValTok}[1]{\textcolor[rgb]{0.00,0.00,0.81}{#1}}
\newcommand{\DocumentationTok}[1]{\textcolor[rgb]{0.56,0.35,0.01}{\textbf{\textit{#1}}}}
\newcommand{\ErrorTok}[1]{\textcolor[rgb]{0.64,0.00,0.00}{\textbf{#1}}}
\newcommand{\ExtensionTok}[1]{#1}
\newcommand{\FloatTok}[1]{\textcolor[rgb]{0.00,0.00,0.81}{#1}}
\newcommand{\FunctionTok}[1]{\textcolor[rgb]{0.00,0.00,0.00}{#1}}
\newcommand{\ImportTok}[1]{#1}
\newcommand{\InformationTok}[1]{\textcolor[rgb]{0.56,0.35,0.01}{\textbf{\textit{#1}}}}
\newcommand{\KeywordTok}[1]{\textcolor[rgb]{0.13,0.29,0.53}{\textbf{#1}}}
\newcommand{\NormalTok}[1]{#1}
\newcommand{\OperatorTok}[1]{\textcolor[rgb]{0.81,0.36,0.00}{\textbf{#1}}}
\newcommand{\OtherTok}[1]{\textcolor[rgb]{0.56,0.35,0.01}{#1}}
\newcommand{\PreprocessorTok}[1]{\textcolor[rgb]{0.56,0.35,0.01}{\textit{#1}}}
\newcommand{\RegionMarkerTok}[1]{#1}
\newcommand{\SpecialCharTok}[1]{\textcolor[rgb]{0.00,0.00,0.00}{#1}}
\newcommand{\SpecialStringTok}[1]{\textcolor[rgb]{0.31,0.60,0.02}{#1}}
\newcommand{\StringTok}[1]{\textcolor[rgb]{0.31,0.60,0.02}{#1}}
\newcommand{\VariableTok}[1]{\textcolor[rgb]{0.00,0.00,0.00}{#1}}
\newcommand{\VerbatimStringTok}[1]{\textcolor[rgb]{0.31,0.60,0.02}{#1}}
\newcommand{\WarningTok}[1]{\textcolor[rgb]{0.56,0.35,0.01}{\textbf{\textit{#1}}}}
\usepackage{longtable,booktabs}
% Correct order of tables after \paragraph or \subparagraph
\usepackage{etoolbox}
\makeatletter
\patchcmd\longtable{\par}{\if@noskipsec\mbox{}\fi\par}{}{}
\makeatother
% Allow footnotes in longtable head/foot
\IfFileExists{footnotehyper.sty}{\usepackage{footnotehyper}}{\usepackage{footnote}}
\makesavenoteenv{longtable}
\usepackage{graphicx}
\makeatletter
\def\maxwidth{\ifdim\Gin@nat@width>\linewidth\linewidth\else\Gin@nat@width\fi}
\def\maxheight{\ifdim\Gin@nat@height>\textheight\textheight\else\Gin@nat@height\fi}
\makeatother
% Scale images if necessary, so that they will not overflow the page
% margins by default, and it is still possible to overwrite the defaults
% using explicit options in \includegraphics[width, height, ...]{}
\setkeys{Gin}{width=\maxwidth,height=\maxheight,keepaspectratio}
% Set default figure placement to htbp
\makeatletter
\def\fps@figure{htbp}
\makeatother
\setlength{\emergencystretch}{3em} % prevent overfull lines
\providecommand{\tightlist}{%
  \setlength{\itemsep}{0pt}\setlength{\parskip}{0pt}}
\setcounter{secnumdepth}{5}
\usepackage{caption} \captionsetup{font={footnotesize},width=6in} \renewcommand{\dblfloatpagefraction}{.9} \makeatletter \renewenvironment{figure} {\def\@captype{figure}} \makeatother \definecolor{shadecolor}{RGB}{242,242,242} \usepackage{xeCJK} \usepackage{setspace} \setstretch{1.3} \usepackage{tcolorbox}

\title{Analysis}
\author{Huang LiChuang of Wie-Biotech}
\date{}

\begin{document}
\maketitle

{
\setcounter{tocdepth}{3}
\tableofcontents
}
\listoffigures

\listoftables

\hypertarget{abstract}{%
\section{摘要}\label{abstract}}

创建了可呈现蛋白质基本信息(分子量、活性位点、结合位点)和结构信息(3D 结构)的 R 包。用户可通过该 R 包,获取存储任意的蛋白质信息。运行指令后,可获得 html(网页)文件,以浏览器打开(已附示例的 html 文件)。

注: \texttt{touchPDB\_0.0.0.9000.tar.gz} 为 R 包,\texttt{annotation.html} 为示例的结果。

\hypertarget{results}{%
\section{使用说明}\label{results}}

\hypertarget{ux5b89ux88c5}{%
\subsection{安装}\label{ux5b89ux88c5}}

\hypertarget{ux5b89ux88c5ux4f9dux8d56}{%
\subsubsection{安装依赖}\label{ux5b89ux88c5ux4f9dux8d56}}

以下安装已经过 \texttt{conda} 新建独立环境完成测试(R 4.2)。

\begin{Shaded}
\begin{Highlighting}[]
\KeywordTok{lapply}\NormalTok{(}\KeywordTok{c}\NormalTok{(}\StringTok{"cli"}\NormalTok{, }\StringTok{"pbapply"}\NormalTok{, }\StringTok{"R.utils"}\NormalTok{, }\StringTok{"r3dmol"}\NormalTok{, }\StringTok{"RCurl"}\NormalTok{,}
    \StringTok{"rmarkdown"}\NormalTok{, }\StringTok{"utils"}\NormalTok{, }\StringTok{"XML"}\NormalTok{, }\StringTok{"shiny"}\NormalTok{, }\StringTok{"BiocManager"}\NormalTok{),}
  \ControlFlowTok{function}\NormalTok{(pkg) \{}
    \ControlFlowTok{if}\NormalTok{ (}\OperatorTok{!}\KeywordTok{requireNamespace}\NormalTok{(pkg))}
      \KeywordTok{install.packages}\NormalTok{(pkg)}
\NormalTok{  \})}

\NormalTok{BiocManager}\OperatorTok{::}\KeywordTok{install}\NormalTok{(}\KeywordTok{c}\NormalTok{(}\StringTok{"BiocStyle"}\NormalTok{, }\StringTok{"UniProt.ws"}\NormalTok{))}
\end{Highlighting}
\end{Shaded}

注:如果是在 Linux 下安装,可能面临需要安装 libcurl 的问题。可以使用 \texttt{sudo\ apt\ install\ libcurl4-openssl-dev} 解决;
如果是 \texttt{conda} 环境下,请使用 \texttt{conda\ install\ -c\ conda-forge\ r-curl} 安装 \texttt{libcurl}。

\hypertarget{ux5b89ux88c5-1}{%
\subsubsection{安装}\label{ux5b89ux88c5-1}}

请确保安装包在当前目录下,如果不在,请输入正确的路径。

\begin{Shaded}
\begin{Highlighting}[]
\KeywordTok{install.packages}\NormalTok{(}\StringTok{"touchPDB\_0.0.0.9000.tar.gz"}\NormalTok{)}
\end{Highlighting}
\end{Shaded}

\hypertarget{ux4f7fux7528ux793aux4f8b}{%
\subsection{使用示例}\label{ux4f7fux7528ux793aux4f8b}}

在 R 命令行中:

\hypertarget{ins}{%
\subsubsection{示例 1}\label{ins}}

\begin{Shaded}
\begin{Highlighting}[]
\CommentTok{\#\# 加载包}
\KeywordTok{require}\NormalTok{(touchPDB)}

\CommentTok{\#\# 需要查询的蛋白质的 Symbol(任意数量)}
\NormalTok{syms \textless{}{-}}\StringTok{ }\KeywordTok{c}\NormalTok{(}\StringTok{"ERBB4"}\NormalTok{, }\StringTok{"Pik3r1"}\NormalTok{, }\StringTok{"AHR"}\NormalTok{, }\StringTok{"TP53"}\NormalTok{)}
\CommentTok{\#\# 新建项目}
\NormalTok{pd \textless{}{-}}\StringTok{ }\KeywordTok{new\_pdb}\NormalTok{()}
\CommentTok{\#\# 获取文件}
\NormalTok{pd \textless{}{-}}\StringTok{ }\KeywordTok{via\_symbol}\NormalTok{(pd, syms)}
\CommentTok{\#\# 生成注释网页(本地 html 文件)}
\KeywordTok{anno}\NormalTok{(pd, syms)}
\end{Highlighting}
\end{Shaded}

\hypertarget{ux793aux4f8b-2}{%
\subsubsection{示例 2}\label{ux793aux4f8b-2}}

\begin{Shaded}
\begin{Highlighting}[]
\KeywordTok{require}\NormalTok{(touchPDB)}

\NormalTok{pd \textless{}{-}}\StringTok{ }\KeywordTok{new\_pdb}\NormalTok{()}
\NormalTok{pd \textless{}{-}}\StringTok{ }\KeywordTok{via\_symbol}\NormalTok{(pd, }\KeywordTok{c}\NormalTok{(}\StringTok{"ERBB4"}\NormalTok{))}
\KeywordTok{anno}\NormalTok{(pd, syms)}
\end{Highlighting}
\end{Shaded}

\hypertarget{ux7ed3ux679cux5448ux73b0}{%
\subsection{结果呈现}\label{ux7ed3ux679cux5448ux73b0}}

以下展示示例 1(\ref{ins} )运行结果的截图。

Figure \ref{fig:index}为图index概览。

\textbf{(对应文件为 \texttt{\textasciitilde{}/Pictures/Screenshots/Screenshot\ from\ 2023-09-19\ 10-12-28.png})}

\def\@captype{figure}
\begin{center}
\includegraphics[width = 0.9\linewidth]{~/Pictures/Screenshots/Screenshot from 2023-09-19 10-12-28.png}
\caption{Index}\label{fig:index}
\end{center}

Figure \ref{fig:Infomation}为图Infomation概览。

\textbf{(对应文件为 \texttt{\textasciitilde{}/Pictures/Screenshots/Screenshot\ from\ 2023-09-19\ 10-13-28.png})}

\def\@captype{figure}
\begin{center}
\includegraphics[width = 0.9\linewidth]{~/Pictures/Screenshots/Screenshot from 2023-09-19 10-13-28.png}
\caption{Infomation}\label{fig:Infomation}
\end{center}

Figure \ref{fig:protein-structure-eg-1}为图protein structure eg 1概览。

\textbf{(对应文件为 \texttt{\textasciitilde{}/Pictures/Screenshots/Screenshot\ from\ 2023-09-19\ 10-14-19.png})}

\def\@captype{figure}
\begin{center}
\includegraphics[width = 0.9\linewidth]{~/Pictures/Screenshots/Screenshot from 2023-09-19 10-14-19.png}
\caption{Protein structure eg 1}\label{fig:protein-structure-eg-1}
\end{center}

\end{document}
