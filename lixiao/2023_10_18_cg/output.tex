% Options for packages loaded elsewhere
\PassOptionsToPackage{unicode}{hyperref}
\PassOptionsToPackage{hyphens}{url}
%
\documentclass[
]{article}
\usepackage{lmodern}
\usepackage{amssymb,amsmath}
\usepackage{ifxetex,ifluatex}
\ifnum 0\ifxetex 1\fi\ifluatex 1\fi=0 % if pdftex
  \usepackage[T1]{fontenc}
  \usepackage[utf8]{inputenc}
  \usepackage{textcomp} % provide euro and other symbols
\else % if luatex or xetex
  \usepackage{unicode-math}
  \defaultfontfeatures{Scale=MatchLowercase}
  \defaultfontfeatures[\rmfamily]{Ligatures=TeX,Scale=1}
\fi
% Use upquote if available, for straight quotes in verbatim environments
\IfFileExists{upquote.sty}{\usepackage{upquote}}{}
\IfFileExists{microtype.sty}{% use microtype if available
  \usepackage[]{microtype}
  \UseMicrotypeSet[protrusion]{basicmath} % disable protrusion for tt fonts
}{}
\makeatletter
\@ifundefined{KOMAClassName}{% if non-KOMA class
  \IfFileExists{parskip.sty}{%
    \usepackage{parskip}
  }{% else
    \setlength{\parindent}{0pt}
    \setlength{\parskip}{6pt plus 2pt minus 1pt}}
}{% if KOMA class
  \KOMAoptions{parskip=half}}
\makeatother
\usepackage{xcolor}
\IfFileExists{xurl.sty}{\usepackage{xurl}}{} % add URL line breaks if available
\IfFileExists{bookmark.sty}{\usepackage{bookmark}}{\usepackage{hyperref}}
\hypersetup{
  pdftitle={Analysis},
  pdfauthor={Huang LiChuang of Wie-Biotech},
  hidelinks,
  pdfcreator={LaTeX via pandoc}}
\urlstyle{same} % disable monospaced font for URLs
\usepackage[margin=1in]{geometry}
\usepackage{longtable,booktabs}
% Correct order of tables after \paragraph or \subparagraph
\usepackage{etoolbox}
\makeatletter
\patchcmd\longtable{\par}{\if@noskipsec\mbox{}\fi\par}{}{}
\makeatother
% Allow footnotes in longtable head/foot
\IfFileExists{footnotehyper.sty}{\usepackage{footnotehyper}}{\usepackage{footnote}}
\makesavenoteenv{longtable}
\usepackage{graphicx}
\makeatletter
\def\maxwidth{\ifdim\Gin@nat@width>\linewidth\linewidth\else\Gin@nat@width\fi}
\def\maxheight{\ifdim\Gin@nat@height>\textheight\textheight\else\Gin@nat@height\fi}
\makeatother
% Scale images if necessary, so that they will not overflow the page
% margins by default, and it is still possible to overwrite the defaults
% using explicit options in \includegraphics[width, height, ...]{}
\setkeys{Gin}{width=\maxwidth,height=\maxheight,keepaspectratio}
% Set default figure placement to htbp
\makeatletter
\def\fps@figure{htbp}
\makeatother
\setlength{\emergencystretch}{3em} % prevent overfull lines
\providecommand{\tightlist}{%
  \setlength{\itemsep}{0pt}\setlength{\parskip}{0pt}}
\setcounter{secnumdepth}{5}
\usepackage{caption} \captionsetup{font={footnotesize},width=6in} \renewcommand{\dblfloatpagefraction}{.9} \makeatletter \renewenvironment{figure} {\def\@captype{figure}} \makeatother \definecolor{shadecolor}{RGB}{242,242,242} \usepackage{xeCJK} \usepackage{setspace} \setstretch{1.3} \usepackage{tcolorbox}

\title{Analysis}
\author{Huang LiChuang of Wie-Biotech}
\date{}

\begin{document}
\maketitle

{
\setcounter{tocdepth}{3}
\tableofcontents
}
\listoffigures

\listoftables

\hypertarget{abstract}{%
\section{摘要}\label{abstract}}

\hypertarget{introduction}{%
\section{前言}\label{introduction}}

\hypertarget{route}{%
\section{研究设计流程图}\label{route}}

\hypertarget{methods}{%
\section{材料和方法}\label{methods}}

\hypertarget{results}{%
\section{分析结果}\label{results}}

\hypertarget{dis}{%
\section{结论}\label{dis}}

\hypertarget{workflow}{%
\section{附:分析流程}\label{workflow}}

\hypertarget{gse171904-gsm5237106}{%
\subsection{GSE171904: GSM5237106}\label{gse171904-gsm5237106}}

\hypertarget{qc}{%
\subsubsection{QC}\label{qc}}

Figure \ref{fig:QC}为图QC概览。

\textbf{(对应文件为 \texttt{Figure+Table/QC.pdf})}

\def\@captype{figure}
\begin{center}
\includegraphics[width = 0.9\linewidth]{Figure+Table/QC.pdf}
\caption{QC}\label{fig:QC}
\end{center}

Figure \ref{fig:PCA-rank}为图PCA rank概览。

\textbf{(对应文件为 \texttt{Figure+Table/PCA-rank.pdf})}

\def\@captype{figure}
\begin{center}
\includegraphics[width = 0.9\linewidth]{Figure+Table/PCA-rank.pdf}
\caption{PCA rank}\label{fig:PCA-rank}
\end{center}

\hypertarget{clustering}{%
\subsubsection{Clustering}\label{clustering}}

Figure \ref{fig:UMAP-clustering}为图UMAP clustering概览。

\textbf{(对应文件为 \texttt{Figure+Table/UMAP-clustering.pdf})}

\def\@captype{figure}
\begin{center}
\includegraphics[width = 0.9\linewidth]{Figure+Table/UMAP-clustering.pdf}
\caption{UMAP clustering}\label{fig:UMAP-clustering}
\end{center}

\hypertarget{top-markers}{%
\subsubsection{Top markers}\label{top-markers}}

Table \ref{tab:All-markers-tables-Seurat-clusters}为表格All markers tables Seurat clusters概览。

\textbf{(对应文件为 \texttt{Figure+Table/All-markers-tables-Seurat-clusters.csv})}

\begin{center}\begin{tcolorbox}[colback=gray!10, colframe=gray!50, width=0.9\linewidth, arc=1mm, boxrule=0.5pt]注:表格共有36327行8列,以下预览的表格可能省略部分数据;表格含有36327个唯一`rownames'。
\end{tcolorbox}
\end{center}

\begin{longtable}[]{@{}llllllll@{}}
\caption{\label{tab:All-markers-tables-Seurat-clusters}All markers tables Seurat clusters}\tabularnewline
\toprule
rownames & p\_val & avg\_l\ldots{} & pct.1 & pct.2 & p\_val\ldots{} & cluster & gene\tabularnewline
\midrule
\endfirsthead
\toprule
rownames & p\_val & avg\_l\ldots{} & pct.1 & pct.2 & p\_val\ldots{} & cluster & gene\tabularnewline
\midrule
\endhead
Cthrc1 & 0 & 2.604\ldots{} & 0.751 & 0.161 & 0 & 0 & Cthrc1\tabularnewline
Crlf1 & 0 & 2.320\ldots{} & 0.513 & 0.083 & 0 & 0 & Crlf1\tabularnewline
Timp1 & 0 & 1.991\ldots{} & 0.995 & 0.815 & 0 & 0 & Timp1\tabularnewline
Col15a1 & 0 & 1.962\ldots{} & 0.947 & 0.509 & 0 & 0 & Col15a1\tabularnewline
Ch25h & 0 & 1.750\ldots{} & 0.48 & 0.139 & 0 & 0 & Ch25h\tabularnewline
Dclk1 & 0 & 1.676\ldots{} & 0.78 & 0.257 & 0 & 0 & Dclk1\tabularnewline
Acta2 & 0 & 1.639\ldots{} & 0.997 & 0.876 & 0 & 0 & Acta2\tabularnewline
Vegfd & 0 & 1.614\ldots{} & 0.371 & 0.095 & 0 & 0 & Vegfd\tabularnewline
Lox & 0 & 1.604\ldots{} & 0.908 & 0.474 & 0 & 0 & Lox\tabularnewline
Vcan & 0 & 1.591\ldots{} & 0.706 & 0.22 & 0 & 0 & Vcan\tabularnewline
Tnc & 0 & 1.558\ldots{} & 0.969 & 0.632 & 0 & 0 & Tnc\tabularnewline
Kdelr3 & 0 & 1.490\ldots{} & 0.913 & 0.47 & 0 & 0 & Kdelr3\tabularnewline
Cgref1 & 0 & 1.476\ldots{} & 0.255 & 0.038 & 0 & 0 & Cgref1\tabularnewline
Col1a1 & 0 & 1.467\ldots{} & 1 & 0.953 & 0 & 0 & Col1a1\tabularnewline
Zfp469 & 0 & 1.462\ldots{} & 0.583 & 0.176 & 0 & 0 & Zfp469\tabularnewline
\ldots{} & \ldots{} & \ldots{} & \ldots{} & \ldots{} & \ldots{} & \ldots{} & \ldots{}\tabularnewline
\bottomrule
\end{longtable}

\hypertarget{cell-annotation}{%
\subsubsection{Cell annotation}\label{cell-annotation}}

Figure \ref{fig:Cell-annotation}为图Cell annotation概览。

\textbf{(对应文件为 \texttt{Figure+Table/Cell-annotation.pdf})}

\def\@captype{figure}
\begin{center}
\includegraphics[width = 0.9\linewidth]{Figure+Table/Cell-annotation.pdf}
\caption{Cell annotation}\label{fig:Cell-annotation}
\end{center}

\hypertarget{ndrg1-in-scsa-annotation-cells}{%
\subsubsection{Ndrg1 in SCSA annotation cells}\label{ndrg1-in-scsa-annotation-cells}}

Figure \ref{fig:Ndrg1-expression}为图Ndrg1 expression概览。

\textbf{(对应文件为 \texttt{Figure+Table/Ndrg1-expression.pdf})}

\def\@captype{figure}
\begin{center}
\includegraphics[width = 0.9\linewidth]{Figure+Table/Ndrg1-expression.pdf}
\caption{Ndrg1 expression}\label{fig:Ndrg1-expression}
\end{center}

Figure \ref{fig:Ndrg1-expression-violin}为图Ndrg1 expression violin概览。

\textbf{(对应文件为 \texttt{Figure+Table/Ndrg1-expression-violin.pdf})}

\def\@captype{figure}
\begin{center}
\includegraphics[width = 0.9\linewidth]{Figure+Table/Ndrg1-expression-violin.pdf}
\caption{Ndrg1 expression violin}\label{fig:Ndrg1-expression-violin}
\end{center}

\hypertarget{ndrg1-in-seurat-clusters}{%
\subsubsection{Ndrg1 in Seurat clusters}\label{ndrg1-in-seurat-clusters}}

Figure \ref{fig:Ndrg1-expression-violin-Seurat-clusters}为图Ndrg1 expression violin Seurat clusters概览。

\textbf{(对应文件为 \texttt{Figure+Table/Ndrg1-expression-violin-Seurat-clusters.pdf})}

\def\@captype{figure}
\begin{center}
\includegraphics[width = 0.9\linewidth]{Figure+Table/Ndrg1-expression-violin-Seurat-clusters.pdf}
\caption{Ndrg1 expression violin Seurat clusters}\label{fig:Ndrg1-expression-violin-Seurat-clusters}
\end{center}

\end{document}
