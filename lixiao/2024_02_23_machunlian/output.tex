% Options for packages loaded elsewhere
\PassOptionsToPackage{unicode}{hyperref}
\PassOptionsToPackage{hyphens}{url}
%
\documentclass[
]{article}
\usepackage{lmodern}
\usepackage{amssymb,amsmath}
\usepackage{ifxetex,ifluatex}
\ifnum 0\ifxetex 1\fi\ifluatex 1\fi=0 % if pdftex
  \usepackage[T1]{fontenc}
  \usepackage[utf8]{inputenc}
  \usepackage{textcomp} % provide euro and other symbols
\else % if luatex or xetex
  \usepackage{unicode-math}
  \defaultfontfeatures{Scale=MatchLowercase}
  \defaultfontfeatures[\rmfamily]{Ligatures=TeX,Scale=1}
\fi
% Use upquote if available, for straight quotes in verbatim environments
\IfFileExists{upquote.sty}{\usepackage{upquote}}{}
\IfFileExists{microtype.sty}{% use microtype if available
  \usepackage[]{microtype}
  \UseMicrotypeSet[protrusion]{basicmath} % disable protrusion for tt fonts
}{}
\makeatletter
\@ifundefined{KOMAClassName}{% if non-KOMA class
  \IfFileExists{parskip.sty}{%
    \usepackage{parskip}
  }{% else
    \setlength{\parindent}{0pt}
    \setlength{\parskip}{6pt plus 2pt minus 1pt}}
}{% if KOMA class
  \KOMAoptions{parskip=half}}
\makeatother
\usepackage{xcolor}
\IfFileExists{xurl.sty}{\usepackage{xurl}}{} % add URL line breaks if available
\IfFileExists{bookmark.sty}{\usepackage{bookmark}}{\usepackage{hyperref}}
\hypersetup{
  hidelinks,
  pdfcreator={LaTeX via pandoc}}
\urlstyle{same} % disable monospaced font for URLs
\usepackage[margin=1in]{geometry}
\usepackage{longtable,booktabs}
% Correct order of tables after \paragraph or \subparagraph
\usepackage{etoolbox}
\makeatletter
\patchcmd\longtable{\par}{\if@noskipsec\mbox{}\fi\par}{}{}
\makeatother
% Allow footnotes in longtable head/foot
\IfFileExists{footnotehyper.sty}{\usepackage{footnotehyper}}{\usepackage{footnote}}
\makesavenoteenv{longtable}
\usepackage{graphicx}
\makeatletter
\def\maxwidth{\ifdim\Gin@nat@width>\linewidth\linewidth\else\Gin@nat@width\fi}
\def\maxheight{\ifdim\Gin@nat@height>\textheight\textheight\else\Gin@nat@height\fi}
\makeatother
% Scale images if necessary, so that they will not overflow the page
% margins by default, and it is still possible to overwrite the defaults
% using explicit options in \includegraphics[width, height, ...]{}
\setkeys{Gin}{width=\maxwidth,height=\maxheight,keepaspectratio}
% Set default figure placement to htbp
\makeatletter
\def\fps@figure{htbp}
\makeatother
\setlength{\emergencystretch}{3em} % prevent overfull lines
\providecommand{\tightlist}{%
  \setlength{\itemsep}{0pt}\setlength{\parskip}{0pt}}
\setcounter{secnumdepth}{5}
\usepackage{caption} \captionsetup{font={footnotesize},width=6in} \renewcommand{\dblfloatpagefraction}{.9} \makeatletter \renewenvironment{figure} {\def\@captype{figure}} \makeatother \@ifundefined{Shaded}{\newenvironment{Shaded}} \@ifundefined{snugshade}{\newenvironment{snugshade}} \renewenvironment{Shaded}{\begin{snugshade}}{\end{snugshade}} \definecolor{shadecolor}{RGB}{230,230,230} \usepackage{xeCJK} \usepackage{setspace} \setstretch{1.3} \usepackage{tcolorbox} \setcounter{secnumdepth}{4} \setcounter{tocdepth}{4} \usepackage{wallpaper} \usepackage[absolute]{textpos} \tcbuselibrary{breakable} \renewenvironment{Shaded} {\begin{tcolorbox}[colback = gray!10, colframe = gray!40, width = 16cm, arc = 1mm, auto outer arc, title = {R input}]} {\end{tcolorbox}} \usepackage{titlesec} \titleformat{\paragraph} {\fontsize{10pt}{0pt}\bfseries} {\arabic{section}.\arabic{subsection}.\arabic{subsubsection}.\arabic{paragraph}} {1em} {} []
\newlength{\cslhangindent}
\setlength{\cslhangindent}{1.5em}
\newenvironment{cslreferences}%
  {}%
  {\par}

\author{}
\date{\vspace{-2.5em}}

\begin{document}

\begin{titlepage} \newgeometry{top=7.5cm}
\ThisCenterWallPaper{1.12}{~/outline/lixiao//cover_page.pdf}
\begin{center} \textbf{\Huge
乙肝病毒HBx利用泛素化系统降解XXX上调YYY诱导肝癌线粒体自噬}
\vspace{4em} \begin{textblock}{10}(3,5.9) \huge
\textbf{\textcolor{white}{2024-02-27}}
\end{textblock} \begin{textblock}{10}(3,7.3)
\Large \textcolor{black}{LiChuang Huang}
\end{textblock} \begin{textblock}{10}(3,11.3)
\Large \textcolor{black}{@立效研究院}
\end{textblock} \end{center} \end{titlepage}
\restoregeometry

\pagenumbering{roman}

\tableofcontents

\listoffigures

\listoftables

\newpage

\pagenumbering{arabic}

\hypertarget{abstract}{%
\section{摘要}\label{abstract}}

\hypertarget{ux9700ux6c42}{%
\subsection{需求}\label{ux9700ux6c42}}

乙肝病毒HBx利用泛素化系统降解XXX上调YYY诱导肝癌线粒体自噬

筛选建议:

1、筛选乙肝病毒HBx (乙型肝炎病毒的外壳蛋白) 处理诱导肝癌细胞差异表达基因集A;

2、基因集A与线粒体自噬相关基因B的相关性(PPI);

3、筛选最佳相关性组合XXX和YYY。

\hypertarget{ux7ed3ux679c}{%
\subsection{结果}\label{ux7ed3ux679c}}

注:与上述建议有不同之处,考虑了与泛素化相关基因的关联。

\begin{itemize}
\tightlist
\item
  以 GSE186862 数据集差异分析获得基因集 DEGs (Fig. \ref{fig:L02-Model-vs-Control-DEGs},
  Tab. \ref{tab:L02-data-Model-vs-Control-DEGs})
\item
  获取自噬相关基因集 Mitophagy (Tab. \ref{tab:MIT-related-targets-from-GeneCards})
\item
  分析 DEGs 中的上调、下调组与 Mitophagy 的交集 (Fig. \ref{fig:UpSet-Intersection-DEGs-with-Mitophagy-related})。
\item
  DEGs 构建 PPI 网络 (Fig. \ref{fig:Raw-PPI-network}):

  \begin{itemize}
  \tightlist
  \item
    预计泛素化会导致基因的表达量下降\textsuperscript{\protect\hyperlink{ref-UbiquitinationPopovi2014}{1}},因此这里推断,受泛素化的 XXX 基因主要存在于 DEGs-down;
    随后,挖掘 DEGs-up-with-Mitophagy (DEGs-up 与 Mitophagy 交集) 与 DEGs-down 的关联
    (Fig. \ref{fig:Filtered-and-formated-PPI-network})。
  \item
    根据 DEGs-down 的 MCC score 筛选 Top 10 (Fig. \ref{fig:Top-MCC-score}) 。
  \end{itemize}
\item
  获取泛素化相关基因集 (Tab. \ref{tab:UBI-related-targets-from-GeneCards})
\item
  泛素化相关的筛选:

  \begin{itemize}
  \tightlist
  \item
    将 Tab. \ref{tab:UBI-related-targets-from-GeneCards} 和 Fig. \ref{fig:Top-MCC-score} 中的 Top 10 DEGs-down 关联分析 (GSE186862 数据集),获得关联热图 (Fig. \ref{fig:L02-correlation-heatmap}) 。
  \item
    以 P-value \textless{} 0.001 筛选 Fig. \ref{fig:L02-correlation-heatmap},得到 Fig. \ref{fig:Correlation-filtered}。
  \end{itemize}
\item
  整合上述过程的数据:泛素化 -\textgreater{} DEGs-down -\textgreater{} DEGs-up-Mitophagy,Fig. \ref{fig:integrated-relationship}
\item
  将整合后的所有基因富集分析,Fig. \ref{fig:INTE-KEGG-enrichment},Fig. \ref{fig:INTE-GO-enrichment}:

  \begin{itemize}
  \tightlist
  \item
    主要关注两条通路 (分别与泛素化和自噬相关) :Fig. \ref{fig:INTE-hsa04120-visualization},Fig. \ref{fig:INTE-hsa04137-visualization}
  \item
    两条通路有交错的基因 (\ref{path-intersect}): HUWE1, RPS27A
  \item
    根据交错基因重新整理 Fig. \ref{fig:integrated-relationship},
    得到 Fig. \ref{fig:co-Exists-in-integrated-relationship},Tab. \ref{tab:co-Exists-in-integrated-relationship-data}
  \end{itemize}
\item
  最终筛选:

  \begin{itemize}
  \tightlist
  \item
    建议:结合通路 Mitophagy (Fig. \ref{fig:INTE-hsa04137-visualization}),
    和 Fig. \ref{fig:co-Exists-in-integrated-relationship},
    可发现: HUWE1 -\textgreater{} RPS27A (UB) -\textgreater{} ULK1 之间存在关联。
  \item
    额外:可根据 Tab. \ref{tab:co-Exists-in-integrated-relationship} 筛选其他可能。
  \end{itemize}
\end{itemize}

\hypertarget{introduction}{%
\section{前言}\label{introduction}}

\hypertarget{methods}{%
\section{材料和方法}\label{methods}}

\hypertarget{ux6750ux6599}{%
\subsection{材料}\label{ux6750ux6599}}

All used GEO expression data and their design:

\begin{itemize}
\tightlist
\item
  \textbf{GSE186862}: mRNA profiles of L02-vector and L02-HBx cells
\end{itemize}

\hypertarget{ux65b9ux6cd5}{%
\subsection{方法}\label{ux65b9ux6cd5}}

Mainly used method:

\begin{itemize}
\tightlist
\item
  R package \texttt{ClusterProfiler} used for gene enrichment analysis\textsuperscript{\protect\hyperlink{ref-ClusterprofilerWuTi2021}{2}}.
\item
  The Human Gene Database \texttt{GeneCards} used for disease related genes prediction\textsuperscript{\protect\hyperlink{ref-TheGenecardsSStelze2016}{3}}.
\item
  GEO \url{https://www.ncbi.nlm.nih.gov/geo/} used for expression dataset aquisition.
\item
  R package \texttt{ClusterProfiler} used for GSEA enrichment\textsuperscript{\protect\hyperlink{ref-ClusterprofilerWuTi2021}{2}}.
\item
  R package \texttt{Limma} and \texttt{edgeR} used for differential expression analysis\textsuperscript{\protect\hyperlink{ref-LimmaPowersDiRitchi2015}{4},\protect\hyperlink{ref-EdgerDifferenChen}{5}}.
\item
  R package \texttt{STEINGdb} used for PPI network construction\textsuperscript{\protect\hyperlink{ref-TheStringDataSzklar2021}{6},\protect\hyperlink{ref-CytohubbaIdenChin2014}{7}}.
\item
  The MCC score was calculated referring to algorithm of \texttt{CytoHubba}\textsuperscript{\protect\hyperlink{ref-CytohubbaIdenChin2014}{7}}.
\item
  Other R packages (eg., \texttt{dplyr} and \texttt{ggplot2}) used for statistic analysis or data visualization.
\end{itemize}

\hypertarget{results}{%
\section{分析结果}\label{results}}

\hypertarget{dis}{%
\section{结论}\label{dis}}

\hypertarget{workflow}{%
\section{附:分析流程}\label{workflow}}

\hypertarget{ux4e59ux809dux75c5ux6bd2-hbx-ux5904ux7406-degs}{%
\subsection{乙肝病毒 HBx 处理 DEGs}\label{ux4e59ux809dux75c5ux6bd2-hbx-ux5904ux7406-degs}}

\hypertarget{ux6570ux636eux6765ux6e90}{%
\subsubsection{数据来源}\label{ux6570ux636eux6765ux6e90}}

\begin{center}\begin{tcolorbox}[colback=gray!10, colframe=gray!50, width=0.9\linewidth, arc=1mm, boxrule=0.5pt]
\textbf{
Data Source ID
:}

\vspace{0.5em}

    GSE186862

\vspace{2em}


\textbf{
data\_processing
:}

\vspace{0.5em}

    Illumina Casava1.7 software used for basecalling.

\vspace{2em}


\textbf{
data\_processing.1
:}

\vspace{0.5em}

    Sequenced reads were trimmed for adaptor sequence, and
masked for low-complexity or low-quality sequence, then
mapped to mm8 whole genome using bowtie v0.12.2 with
parameters -q -p 4 -e 100 -y -a -m 10 --best --strata

\vspace{2em}


\textbf{
data\_processing.2
:}

\vspace{0.5em}

    Reads Per Kilobase of exon per Megabase of library size
(RPKM) were calculated using a protocol from Chepelev et
al., Nucleic Acids Research, 2009. In short, exons from all
isoforms of a gene were merged to create one
meta-transcript. The number of reads falling in the exons
of this meta-transcri...

\vspace{2em}


\textbf{
data\_processing.3
:}

\vspace{0.5em}

    Genome\_build: HG19

\vspace{2em}


\textbf{
(Others)
:}

\vspace{0.5em}

    ...

\vspace{2em}
\end{tcolorbox}
\end{center}

Table \ref{tab:L02-metadata} (下方表格) 为表格L02 metadata概览。

\textbf{(对应文件为 \texttt{Figure+Table/L02-metadata.csv})}

\begin{center}\begin{tcolorbox}[colback=gray!10, colframe=gray!50, width=0.9\linewidth, arc=1mm, boxrule=0.5pt]注:表格共有6行9列,以下预览的表格可能省略部分数据;表格含有6个唯一`rownames'。
\end{tcolorbox}
\end{center}
\begin{center}\begin{tcolorbox}[colback=gray!10, colframe=gray!50, width=0.9\linewidth, arc=1mm, boxrule=0.5pt]\begin{enumerate}\tightlist
\item sample:  样品名称
\item group:  分组名称
\end{enumerate}\end{tcolorbox}
\end{center}

\begin{longtable}[]{@{}lllllllll@{}}
\caption{\label{tab:L02-metadata}L02 metadata}\tabularnewline
\toprule
rownames & sample & group & lib.size & norm.f\ldots{} & title & cell.l\ldots{} & cell.t\ldots{} & genoty\ldots{}\tabularnewline
\midrule
\endfirsthead
\toprule
rownames & sample & group & lib.size & norm.f\ldots{} & title & cell.l\ldots{} & cell.t\ldots{} & genoty\ldots{}\tabularnewline
\midrule
\endhead
A & A & Control & 16890967 & 1 & L02-ve\ldots{} & L02 & liver \ldots{} & control\tabularnewline
C & C & Control & 13575225 & 1 & L02-ve\ldots{} & L02 & liver \ldots{} & control\tabularnewline
E & E & Control & 14827232 & 1 & L02-ve\ldots{} & L02 & liver \ldots{} & control\tabularnewline
B & B & Model & 21985666 & 1 & L02-HBx1 & L02 & liver \ldots{} & HBx ex\ldots{}\tabularnewline
D & D & Model & 16595110 & 1 & L02-HBx2 & L02 & liver \ldots{} & HBx ex\ldots{}\tabularnewline
F & F & Model & 19786946 & 1 & L02-HBx3 & L02 & liver \ldots{} & HBx ex\ldots{}\tabularnewline
\bottomrule
\end{longtable}

注:该 GSE 数据集的补充材料没有注明样品分组 (即,A、B、C\ldots\ldots 等样品是属于哪个组别) ,
元数据表格中的分组信息,是我根据原文 Figure 和 LogFC 数值推断的\textsuperscript{\protect\hyperlink{ref-HbxIncreasesCZheng2022}{8}}。

\hypertarget{degs}{%
\subsubsection{DEGs}\label{degs}}

Figure \ref{fig:L02-Model-vs-Control-DEGs} (下方图) 为图L02 Model vs Control DEGs概览。

\textbf{(对应文件为 \texttt{Figure+Table/L02-Model-vs-Control-DEGs.pdf})}

\def\@captype{figure}
\begin{center}
\includegraphics[width = 0.9\linewidth]{Figure+Table/L02-Model-vs-Control-DEGs.pdf}
\caption{L02 Model vs Control DEGs}\label{fig:L02-Model-vs-Control-DEGs}
\end{center}

Table \ref{tab:L02-data-Model-vs-Control-DEGs} (下方表格) 为表格L02 data Model vs Control DEGs概览。

\textbf{(对应文件为 \texttt{Figure+Table/L02-data-Model-vs-Control-DEGs.xlsx})}

\begin{center}\begin{tcolorbox}[colback=gray!10, colframe=gray!50, width=0.9\linewidth, arc=1mm, boxrule=0.5pt]注:表格共有2352行12列,以下预览的表格可能省略部分数据;表格含有2205个唯一`Symbol'。
\end{tcolorbox}
\end{center}
\begin{center}\begin{tcolorbox}[colback=gray!10, colframe=gray!50, width=0.9\linewidth, arc=1mm, boxrule=0.5pt]\begin{enumerate}\tightlist
\item logFC:  estimate of the log2-fold-change corresponding to the effect or contrast (for ‘topTableF’ there may be several columns of log-fold-changes)
\item AveExpr:  average log2-expression for the probe over all arrays and channels, same as ‘Amean’ in the ‘MarrayLM’ object
\item t:  moderated t-statistic (omitted for ‘topTableF’)
\item P.Value:  raw p-value
\item B:  log-odds that the gene is differentially expressed (omitted for ‘topTreat’)
\end{enumerate}\end{tcolorbox}
\end{center}

\begin{longtable}[]{@{}llllllllll@{}}
\caption{\label{tab:L02-data-Model-vs-Control-DEGs}L02 data Model vs Control DEGs}\tabularnewline
\toprule
rownames & AccID & AccID.1 & Symbol & Strand & KeggID & logFC & AveExpr & t & P.Value\tabularnewline
\midrule
\endfirsthead
\toprule
rownames & AccID & AccID.1 & Symbol & Strand & KeggID & logFC & AveExpr & t & P.Value\tabularnewline
\midrule
\endhead
38934 & ENSG00\ldots{} & NM\_019\ldots{} & UGT1A7 & hsa:54577 & UDP gl\ldots{} & -3.200\ldots{} & 5.1819\ldots{} & -24.69\ldots{} & 2.2577\ldots{}\tabularnewline
12949 & ENSG00\ldots{} & NM\_153\ldots{} & HSPA8 & hsa:3312 & heat s\ldots{} & -2.694\ldots{} & 10.198\ldots{} & -23.85\ldots{} & 3.1834\ldots{}\tabularnewline
15559 & ENSG00\ldots{} & NM\_006472 & TXNIP & hsa:10628 & thiore\ldots{} & 3.6547\ldots{} & 5.2722\ldots{} & 24.094\ldots{} & 2.8880\ldots{}\tabularnewline
22453 & ENSG00\ldots{} & NM\_003\ldots{} & STC2 & hsa:8614 & stanni\ldots{} & 2.6030\ldots{} & 7.9409\ldots{} & 22.909\ldots{} & 4.7679\ldots{}\tabularnewline
56468 & ENSG00\ldots{} & NM\_052\ldots{} & FAM129A & hsa:11\ldots{} & family\ldots{} & 2.2312\ldots{} & 7.2870\ldots{} & 18.291\ldots{} & 4.4197\ldots{}\tabularnewline
52927 & ENSG00\ldots{} & NM\_144\ldots{} & IL20RB & hsa:53833 & interl\ldots{} & 2.3520\ldots{} & 4.4217\ldots{} & 16.820\ldots{} & 1.0066\ldots{}\tabularnewline
22482 & ENSG00\ldots{} & NM\_004\ldots{} & ATF3 & hsa:467 & activa\ldots{} & 2.9807\ldots{} & 4.1438\ldots{} & 16.708\ldots{} & 1.0745\ldots{}\tabularnewline
15033 & ENSG00\ldots{} & NM\_005\ldots{} & SGK1 & hsa:6446 & serum/\ldots{} & -1.942\ldots{} & 6.7155\ldots{} & -16.43\ldots{} & 1.2627\ldots{}\tabularnewline
49697 & ENSG00\ldots{} & NM\_001\ldots{} & GADD45A & hsa:1647 & growth\ldots{} & 2.1993\ldots{} & 5.0430\ldots{} & 16.435\ldots{} & 1.2622\ldots{}\tabularnewline
23878 & ENSG00\ldots{} & NR\_120\ldots{} & LINC01468 & hsa:10\ldots{} & long i\ldots{} & -2.572\ldots{} & 3.8734\ldots{} & -16.50\ldots{} & 1.2116\ldots{}\tabularnewline
39631 & ENSG00\ldots{} & NM\_014\ldots{} & PPP1R15A & hsa:23645 & protei\ldots{} & 1.8632\ldots{} & 6.8100\ldots{} & 15.752\ldots{} & 1.9102\ldots{}\tabularnewline
5042 & ENSG00\ldots{} & NM\_004\ldots{} & F2RL2 & hsa:2151 & coagul\ldots{} & 2.3576\ldots{} & 3.9660\ldots{} & 15.718\ldots{} & 1.9511\ldots{}\tabularnewline
7150 & ENSG00\ldots{} & NM\_002\ldots{} & PTX3 & hsa:5806 & pentra\ldots{} & 3.0542\ldots{} & 3.7001\ldots{} & 15.552\ldots{} & 2.1645\ldots{}\tabularnewline
3587 & ENSG00\ldots{} & NM\_002\ldots{} & CXCL2 & hsa:2920 & chemok\ldots{} & 2.0506\ldots{} & 4.7692\ldots{} & 15.311\ldots{} & 2.5193\ldots{}\tabularnewline
58151 & ENSG00\ldots{} & NM\_005\ldots{} & HSPA1A & hsa:3303 & heat s\ldots{} & -2.338\ldots{} & 6.8642\ldots{} & -15.26\ldots{} & 2.5935\ldots{}\tabularnewline
\ldots{} & \ldots{} & \ldots{} & \ldots{} & \ldots{} & \ldots{} & \ldots{} & \ldots{} & \ldots{} & \ldots{}\tabularnewline
\bottomrule
\end{longtable}

\hypertarget{ux5bccux96c6ux5206ux6790-ux5c1dux8bd5}{%
\subsubsection{富集分析 (尝试)}\label{ux5bccux96c6ux5206ux6790-ux5c1dux8bd5}}

Figure \ref{fig:L02-KEGG-enrichment-with-enriched-genes} (下方图) 为图L02 KEGG enrichment with enriched genes概览。

\textbf{(对应文件为 \texttt{Figure+Table/L02-KEGG-enrichment-with-enriched-genes.pdf})}

\def\@captype{figure}
\begin{center}
\includegraphics[width = 0.9\linewidth]{Figure+Table/L02-KEGG-enrichment-with-enriched-genes.pdf}
\caption{L02 KEGG enrichment with enriched genes}\label{fig:L02-KEGG-enrichment-with-enriched-genes}
\end{center}

Figure \ref{fig:L02-GSEA-plot-of-the-pathways} (下方图) 为图L02 GSEA plot of the pathways概览。

\textbf{(对应文件为 \texttt{Figure+Table/L02-GSEA-plot-of-the-pathways.pdf})}

\def\@captype{figure}
\begin{center}
\includegraphics[width = 0.9\linewidth]{Figure+Table/L02-GSEA-plot-of-the-pathways.pdf}
\caption{L02 GSEA plot of the pathways}\label{fig:L02-GSEA-plot-of-the-pathways}
\end{center}

\hypertarget{ux7ebfux7c92ux4f53ux81eaux566c}{%
\subsection{线粒体自噬}\label{ux7ebfux7c92ux4f53ux81eaux566c}}

\hypertarget{genecards}{%
\subsubsection{GeneCards}\label{genecards}}

\begin{center}\begin{tcolorbox}[colback=gray!10, colframe=gray!50, width=0.9\linewidth, arc=1mm, boxrule=0.5pt]
\textbf{
The GeneCards data was obtained by filtering:
:}

\vspace{0.5em}

    Score > 1

\vspace{2em}
\end{tcolorbox}
\end{center}

Table \ref{tab:MIT-related-targets-from-GeneCards} (下方表格) 为表格MIT related targets from GeneCards概览。

\textbf{(对应文件为 \texttt{Figure+Table/MIT-related-targets-from-GeneCards.xlsx})}

\begin{center}\begin{tcolorbox}[colback=gray!10, colframe=gray!50, width=0.9\linewidth, arc=1mm, boxrule=0.5pt]注:表格共有1686行7列,以下预览的表格可能省略部分数据;表格含有1686个唯一`Symbol'。
\end{tcolorbox}
\end{center}

\begin{longtable}[]{@{}lllllll@{}}
\caption{\label{tab:MIT-related-targets-from-GeneCards}MIT related targets from GeneCards}\tabularnewline
\toprule
Symbol & Description & Category & UniProt\_ID & GIFtS & GC\_id & Score\tabularnewline
\midrule
\endfirsthead
\toprule
Symbol & Description & Category & UniProt\_ID & GIFtS & GC\_id & Score\tabularnewline
\midrule
\endhead
PRKN & Parkin RBR\ldots{} & Protein Co\ldots{} & O60260 & 57 & GC06M161348 & 19.14\tabularnewline
PINK1 & PTEN Induc\ldots{} & Protein Co\ldots{} & Q9BXM7 & 55 & GC01P020634 & 18.01\tabularnewline
MAP1LC3B & Microtubul\ldots{} & Protein Co\ldots{} & Q9GZQ8 & 50 & GC16P087413 & 11.07\tabularnewline
VDAC1 & Voltage De\ldots{} & Protein Co\ldots{} & P21796 & 54 & GC05M133975 & 10.11\tabularnewline
FUNDC1 & FUN14 Doma\ldots{} & Protein Co\ldots{} & Q8IVP5 & 37 & GC0XM044523 & 9.21\tabularnewline
MFN2 & Mitofusin 2 & Protein Co\ldots{} & O95140 & 57 & GC01P011980 & 9.05\tabularnewline
SQSTM1 & Sequestoso\ldots{} & Protein Co\ldots{} & Q13501 & 58 & GC05P179806 & 8.40\tabularnewline
MAP1LC3A & Microtubul\ldots{} & Protein Co\ldots{} & Q9H492 & 48 & GC20P034546 & 7.92\tabularnewline
ULK1 & Unc-51 Lik\ldots{} & Protein Co\ldots{} & O75385 & 54 & GC12P131894 & 7.22\tabularnewline
UBC & Ubiquitin C & Protein Co\ldots{} & P0CG48 & 51 & GC12M124911 & 6.81\tabularnewline
PHB2 & Prohibitin 2 & Protein Co\ldots{} & Q99623 & 49 & GC12M006965 & 6.75\tabularnewline
ATG13 & Autophagy \ldots{} & Protein Co\ldots{} & O75143 & 49 & GC11P047383 & 6.54\tabularnewline
SOD2-OT1 & SOD2 Overl\ldots{} & RNA Gene & & 18 & GC06M159772 & 6.47\tabularnewline
TOMM20 & Translocas\ldots{} & Protein Co\ldots{} & Q15388 & 48 & GC01M235109 & 6.45\tabularnewline
AMBRA1 & Autophagy \ldots{} & Protein Co\ldots{} & Q9C0C7 & 46 & GC11M120823 & 6.26\tabularnewline
\ldots{} & \ldots{} & \ldots{} & \ldots{} & \ldots{} & \ldots{} & \ldots{}\tabularnewline
\bottomrule
\end{longtable}

\hypertarget{degs-ux4e0eux7ebfux7c92ux4f53ux81eaux566c}{%
\subsection{DEGs 与线粒体自噬}\label{degs-ux4e0eux7ebfux7c92ux4f53ux81eaux566c}}

\hypertarget{ux4ea4ux96c6-inter-degs-mito}{%
\subsubsection{交集 (Inter-DEGs-Mito)}\label{ux4ea4ux96c6-inter-degs-mito}}

Figure \ref{fig:Venn-Intersection-DEGs-with-Mitophagy-related} (下方图) 为图Venn Intersection DEGs with Mitophagy related概览。

\textbf{(对应文件为 \texttt{Figure+Table/Venn-Intersection-DEGs-with-Mitophagy-related.pdf})}

\def\@captype{figure}
\begin{center}
\includegraphics[width = 0.9\linewidth]{Figure+Table/Venn-Intersection-DEGs-with-Mitophagy-related.pdf}
\caption{Venn Intersection DEGs with Mitophagy related}\label{fig:Venn-Intersection-DEGs-with-Mitophagy-related}
\end{center}
\begin{center}\begin{tcolorbox}[colback=gray!10, colframe=gray!50, width=0.9\linewidth, arc=1mm, boxrule=0.5pt]
\textbf{
Intersection
:}

\vspace{0.5em}

    HSPA8, HSPA1A, SESN2, HMGCS1, RNF41, PSAT1, NDRG1,
DNAJA1, PCK2, ZC3HAV1, SLFN11, RCAN1, KPNA2, BNIP3, UGP2,
SHMT2, LDHA, VDAC1, PLOD2, ANLN, BNIP3L, NSDHL, PCYOX1,
PHGDH, TRIM25, PDP1, SQSTM1, HSPH1, PLSCR1, SLC3A2,
GABARAPL1, HK2, HSP90AA1, APAF1, LMO7, ARHGEF2, GPCPD1,
NFKB1, CUL3, SMAD3, NFKB...

\vspace{2em}
\end{tcolorbox}
\end{center}

\textbf{(上述信息框内容已保存至 \texttt{Figure+Table/Venn-Intersection-DEGs-with-Mitophagy-related-content})}

Figure \ref{fig:UpSet-Intersection-DEGs-with-Mitophagy-related} (下方图) 为图UpSet Intersection DEGs with Mitophagy related概览。

\textbf{(对应文件为 \texttt{Figure+Table/UpSet-Intersection-DEGs-with-Mitophagy-related.pdf})}

\def\@captype{figure}
\begin{center}
\includegraphics[width = 0.9\linewidth]{Figure+Table/UpSet-Intersection-DEGs-with-Mitophagy-related.pdf}
\caption{UpSet Intersection DEGs with Mitophagy related}\label{fig:UpSet-Intersection-DEGs-with-Mitophagy-related}
\end{center}
\begin{center}\begin{tcolorbox}[colback=gray!10, colframe=gray!50, width=0.9\linewidth, arc=1mm, boxrule=0.5pt]
\textbf{
All\_intersection
:}

\vspace{0.5em}



\vspace{2em}
\end{tcolorbox}
\end{center}

\textbf{(上述信息框内容已保存至 \texttt{Figure+Table/UpSet-Intersection-DEGs-with-Mitophagy-related-content})}

\hypertarget{ppi}{%
\subsubsection{PPI}\label{ppi}}

构建 DEGs 的 PPI 网络。

Figure \ref{fig:Raw-PPI-network} (下方图) 为图Raw PPI network概览。

\textbf{(对应文件为 \texttt{Figure+Table/Raw-PPI-network.pdf})}

\def\@captype{figure}
\begin{center}
\includegraphics[width = 0.9\linewidth]{Figure+Table/Raw-PPI-network.pdf}
\caption{Raw PPI network}\label{fig:Raw-PPI-network}
\end{center}

预计泛素化会导致基因的表达量下降\textsuperscript{\protect\hyperlink{ref-UbiquitinationPopovi2014}{1}},因此这里可以推断,受泛素化的 XXX 基因主要存在于 DEGs-down。

挖掘 DEGs-up-with-Mitophagy (DEGs-up 与 Mitophagy 交集) 与 DEGs-down 的关联。

Figure \ref{fig:Filtered-and-formated-PPI-network} (下方图) 为图Filtered and formated PPI network概览。

\textbf{(对应文件为 \texttt{Figure+Table/Filtered-and-formated-PPI-network.pdf})}

\def\@captype{figure}
\begin{center}
\includegraphics[width = 0.9\linewidth]{Figure+Table/Filtered-and-formated-PPI-network.pdf}
\caption{Filtered and formated PPI network}\label{fig:Filtered-and-formated-PPI-network}
\end{center}

根据 DEGs-down 的 MCC score 筛选 Top 10。

Figure \ref{fig:Top-MCC-score} (下方图) 为图Top MCC score概览。

\textbf{(对应文件为 \texttt{Figure+Table/Top-MCC-score.pdf})}

\def\@captype{figure}
\begin{center}
\includegraphics[width = 0.9\linewidth]{Figure+Table/Top-MCC-score.pdf}
\caption{Top MCC score}\label{fig:Top-MCC-score}
\end{center}

\hypertarget{ux6cdbux7d20ux5316}{%
\subsection{泛素化}\label{ux6cdbux7d20ux5316}}

\hypertarget{genecards-1}{%
\subsubsection{GeneCards}\label{genecards-1}}

\begin{center}\begin{tcolorbox}[colback=gray!10, colframe=gray!50, width=0.9\linewidth, arc=1mm, boxrule=0.5pt]
\textbf{
The GeneCards data was obtained by filtering:
:}

\vspace{0.5em}

    Score > 15

\vspace{2em}
\end{tcolorbox}
\end{center}

Table \ref{tab:UBI-related-targets-from-GeneCards} (下方表格) 为表格UBI related targets from GeneCards概览。

\textbf{(对应文件为 \texttt{Figure+Table/UBI-related-targets-from-GeneCards.xlsx})}

\begin{center}\begin{tcolorbox}[colback=gray!10, colframe=gray!50, width=0.9\linewidth, arc=1mm, boxrule=0.5pt]注:表格共有161行7列,以下预览的表格可能省略部分数据;表格含有161个唯一`Symbol'。
\end{tcolorbox}
\end{center}

\begin{longtable}[]{@{}lllllll@{}}
\caption{\label{tab:UBI-related-targets-from-GeneCards}UBI related targets from GeneCards}\tabularnewline
\toprule
Symbol & Description & Category & UniProt\_ID & GIFtS & GC\_id & Score\tabularnewline
\midrule
\endfirsthead
\toprule
Symbol & Description & Category & UniProt\_ID & GIFtS & GC\_id & Score\tabularnewline
\midrule
\endhead
RPS27A & Ribosomal \ldots{} & Protein Co\ldots{} & P62979 & 51 & GC02P055231 & 41.57\tabularnewline
PRKN & Parkin RBR\ldots{} & Protein Co\ldots{} & O60260 & 57 & GC06M161348 & 40.64\tabularnewline
UBC & Ubiquitin C & Protein Co\ldots{} & P0CG48 & 51 & GC12M124911 & 37.54\tabularnewline
UBE2D1 & Ubiquitin \ldots{} & Protein Co\ldots{} & P51668 & 52 & GC10P058334 & 37.22\tabularnewline
UBE2D3 & Ubiquitin \ldots{} & Protein Co\ldots{} & P61077 & 52 & GC04M102794 & 35.79\tabularnewline
UBE2D2 & Ubiquitin \ldots{} & Protein Co\ldots{} & P62837 & 51 & GC05P139526 & 35.33\tabularnewline
UBE2L3 & Ubiquitin \ldots{} & Protein Co\ldots{} & P68036 & 54 & GC22P021549 & 33.3\tabularnewline
UBE2N & Ubiquitin \ldots{} & Protein Co\ldots{} & P61088 & 55 & GC12M093406 & 32.84\tabularnewline
RBX1 & Ring-Box 1 & Protein Co\ldots{} & P62877 & 51 & GC22P040951 & 30.81\tabularnewline
USP7 & Ubiquitin \ldots{} & Protein Co\ldots{} & Q93009 & 57 & GC16M008892 & 30.55\tabularnewline
VCP & Valosin Co\ldots{} & Protein Co\ldots{} & P55072 & 58 & GC09M035056 & 30.55\tabularnewline
UBE3A & Ubiquitin \ldots{} & Protein Co\ldots{} & Q05086 & 56 & GC15M025333 & 30.26\tabularnewline
MDM2 & MDM2 Proto\ldots{} & Protein Co\ldots{} & Q00987 & 62 & GC12P068808 & 30.23\tabularnewline
STUB1 & STIP1 Homo\ldots{} & Protein Co\ldots{} & Q9UNE7 & 54 & GC16P064961 & 30.01\tabularnewline
UBE4B & Ubiquitina\ldots{} & Protein Co\ldots{} & O95155 & 49 & GC01P010032 & 29.68\tabularnewline
\ldots{} & \ldots{} & \ldots{} & \ldots{} & \ldots{} & \ldots{} & \ldots{}\tabularnewline
\bottomrule
\end{longtable}

\hypertarget{ux6cdbux7d20ux5316ux57faux56e0ux96c6ux4e0eux7b5bux9009ux57faux56e0ux96c6-degs-down-ux7684ux76f8ux5173ux6027}{%
\subsection{泛素化基因集与筛选基因集 (DEGs-down) 的相关性}\label{ux6cdbux7d20ux5316ux57faux56e0ux96c6ux4e0eux7b5bux9009ux57faux56e0ux96c6-degs-down-ux7684ux76f8ux5173ux6027}}

\hypertarget{ux5173ux8054ux70edux56fe}{%
\subsubsection{关联热图}\label{ux5173ux8054ux70edux56fe}}

将 Tab. \ref{tab:UBI-related-targets-from-GeneCards} 和 Fig. \ref{fig:Top-MCC-score} 中的 Top 10 DEGs-down
关联分析 (GSE186862 数据集)。

Figure \ref{fig:L02-correlation-heatmap} (下方图) 为图L02 correlation heatmap概览。

\textbf{(对应文件为 \texttt{Figure+Table/L02-correlation-heatmap.pdf})}

\def\@captype{figure}
\begin{center}
\includegraphics[width = 0.9\linewidth]{Figure+Table/L02-correlation-heatmap.pdf}
\caption{L02 correlation heatmap}\label{fig:L02-correlation-heatmap}
\end{center}

\hypertarget{ux6784ux5efaux7f51ux7edc}{%
\subsubsection{构建网络}\label{ux6784ux5efaux7f51ux7edc}}

以 P-value \textless{} 0.001 筛选 Fig. \ref{fig:L02-correlation-heatmap}。

Figure \ref{fig:Correlation-filtered} (下方图) 为图Correlation filtered概览。

\textbf{(对应文件为 \texttt{Figure+Table/Correlation-filtered.pdf})}

\def\@captype{figure}
\begin{center}
\includegraphics[width = 0.9\linewidth]{Figure+Table/Correlation-filtered.pdf}
\caption{Correlation filtered}\label{fig:Correlation-filtered}
\end{center}

\hypertarget{ux6574ux5408ux6cdbux7d20ux5316---degs-down---degs-up-mitophagy}{%
\subsection{整合:泛素化 -\textgreater{} DEGs-down -\textgreater{} DEGs-up-Mitophagy}\label{ux6574ux5408ux6cdbux7d20ux5316---degs-down---degs-up-mitophagy}}

Figure \ref{fig:integrated-relationship} (下方图) 为图integrated relationship概览。

\textbf{(对应文件为 \texttt{Figure+Table/integrated-relationship.pdf})}

\def\@captype{figure}
\begin{center}
\includegraphics[width = 0.9\linewidth]{Figure+Table/integrated-relationship.pdf}
\caption{Integrated relationship}\label{fig:integrated-relationship}
\end{center}

\hypertarget{ux5bccux96c6ux5206ux6790}{%
\subsection{富集分析}\label{ux5bccux96c6ux5206ux6790}}

\hypertarget{kegg}{%
\subsubsection{KEGG}\label{kegg}}

Figure \ref{fig:INTE-KEGG-enrichment} (下方图) 为图INTE KEGG enrichment概览。

\textbf{(对应文件为 \texttt{Figure+Table/INTE-KEGG-enrichment.pdf})}

\def\@captype{figure}
\begin{center}
\includegraphics[width = 0.9\linewidth]{Figure+Table/INTE-KEGG-enrichment.pdf}
\caption{INTE KEGG enrichment}\label{fig:INTE-KEGG-enrichment}
\end{center}

Figure \ref{fig:INTE-GO-enrichment} (下方图) 为图INTE GO enrichment概览。

\textbf{(对应文件为 \texttt{Figure+Table/INTE-GO-enrichment.pdf})}

\def\@captype{figure}
\begin{center}
\includegraphics[width = 0.9\linewidth]{Figure+Table/INTE-GO-enrichment.pdf}
\caption{INTE GO enrichment}\label{fig:INTE-GO-enrichment}
\end{center}

\hypertarget{pathway-visualization}{%
\subsubsection{pathway visualization}\label{pathway-visualization}}

Figure \ref{fig:INTE-hsa04120-visualization} (下方图) 为图INTE hsa04120 visualization概览。

\textbf{(对应文件为 \texttt{Figure+Table/hsa04120.pathview.png})}

\def\@captype{figure}
\begin{center}
\includegraphics[width = 0.9\linewidth]{pathview2024-02-27_14_52_06.625626/hsa04120.pathview.png}
\caption{INTE hsa04120 visualization}\label{fig:INTE-hsa04120-visualization}
\end{center}

Figure \ref{fig:INTE-hsa04137-visualization} (下方图) 为图INTE hsa04137 visualization概览。

\textbf{(对应文件为 \texttt{Figure+Table/hsa04137.pathview.png})}

\def\@captype{figure}
\begin{center}
\includegraphics[width = 0.9\linewidth]{pathview2024-02-27_14_52_06.625626/hsa04137.pathview.png}
\caption{INTE hsa04137 visualization}\label{fig:INTE-hsa04137-visualization}
\end{center}

\hypertarget{path-intersect}{%
\subsubsection{富集于 hsa04120 (Ubiquitination) 与 hsa04137 (Mitophagy) 的基因}\label{path-intersect}}

\begin{center}\begin{tcolorbox}[colback=gray!10, colframe=gray!50, width=0.9\linewidth, arc=1mm, boxrule=0.5pt]
\textbf{
Content
:}

\vspace{0.5em}

    HUWE1, RPS27A

\vspace{2em}
\end{tcolorbox}
\end{center}

Figure \ref{fig:co-Exists-in-integrated-relationship} (下方图) 为图co Exists in integrated relationship概览。

\textbf{(对应文件为 \texttt{Figure+Table/co-Exists-in-integrated-relationship.pdf})}

\def\@captype{figure}
\begin{center}
\includegraphics[width = 0.9\linewidth]{Figure+Table/co-Exists-in-integrated-relationship.pdf}
\caption{Co Exists in integrated relationship}\label{fig:co-Exists-in-integrated-relationship}
\end{center}

Table \ref{tab:co-Exists-in-integrated-relationship-data} (下方表格) 为表格co Exists in integrated relationship data概览。

\textbf{(对应文件为 \texttt{Figure+Table/co-Exists-in-integrated-relationship-data.csv})}

\begin{center}\begin{tcolorbox}[colback=gray!10, colframe=gray!50, width=0.9\linewidth, arc=1mm, boxrule=0.5pt]注:表格共有32行3列,以下预览的表格可能省略部分数据;表格含有2个唯一`Ubiquitination\_related'。
\end{tcolorbox}
\end{center}

\begin{longtable}[]{@{}lll@{}}
\caption{\label{tab:co-Exists-in-integrated-relationship-data}Co Exists in integrated relationship data}\tabularnewline
\toprule
Ubiquitination\_related & DEGs\_down & DEGs\_up\_Mitophagy\tabularnewline
\midrule
\endfirsthead
\toprule
Ubiquitination\_related & DEGs\_down & DEGs\_up\_Mitophagy\tabularnewline
\midrule
\endhead
RPS27A & RPS27A & AIMP2\tabularnewline
RPS27A & RPS27A & NFKB1\tabularnewline
RPS27A & RPS27A & DUSP1\tabularnewline
RPS27A & RPS27A & NR1D1\tabularnewline
RPS27A & RPS27A & DDB2\tabularnewline
RPS27A & RPS27A & EP300\tabularnewline
RPS27A & RPS27A & BAX\tabularnewline
RPS27A & RPS27A & TRIM25\tabularnewline
RPS27A & RPS27A & ULK1\tabularnewline
RPS27A & RPS27A & SMAD3\tabularnewline
RPS27A & RPS27A & RNF41\tabularnewline
RPS27A & RPS27A & UBE2Z\tabularnewline
RPS27A & RPS27A & NFKB2\tabularnewline
RPS27A & RPS27A & SRC\tabularnewline
RPS27A & RPS27A & SQSTM1\tabularnewline
\ldots{} & \ldots{} & \ldots{}\tabularnewline
\bottomrule
\end{longtable}

\hypertarget{bibliography}{%
\section*{Reference}\label{bibliography}}
\addcontentsline{toc}{section}{Reference}

\hypertarget{refs}{}
\begin{cslreferences}
\leavevmode\hypertarget{ref-UbiquitinationPopovi2014}{}%
1. Popovic, D., Vucic, D. \& Dikic, I. Ubiquitination in disease pathogenesis and treatment. \emph{Nature Medicine} \textbf{20}, (2014).

\leavevmode\hypertarget{ref-ClusterprofilerWuTi2021}{}%
2. Wu, T. \emph{et al.} ClusterProfiler 4.0: A universal enrichment tool for interpreting omics data. \emph{The Innovation} \textbf{2}, (2021).

\leavevmode\hypertarget{ref-TheGenecardsSStelze2016}{}%
3. Stelzer, G. \emph{et al.} The genecards suite: From gene data mining to disease genome sequence analyses. \emph{Current protocols in bioinformatics} \textbf{54}, 1.30.1--1.30.33 (2016).

\leavevmode\hypertarget{ref-LimmaPowersDiRitchi2015}{}%
4. Ritchie, M. E. \emph{et al.} Limma powers differential expression analyses for rna-sequencing and microarray studies. \emph{Nucleic Acids Research} \textbf{43}, e47 (2015).

\leavevmode\hypertarget{ref-EdgerDifferenChen}{}%
5. Chen, Y., McCarthy, D., Ritchie, M., Robinson, M. \& Smyth, G. EdgeR: Differential analysis of sequence read count data user's guide. 119.

\leavevmode\hypertarget{ref-TheStringDataSzklar2021}{}%
6. Szklarczyk, D. \emph{et al.} The string database in 2021: Customizable proteinprotein networks, and functional characterization of user-uploaded gene/measurement sets. \emph{Nucleic Acids Research} \textbf{49}, D605--D612 (2021).

\leavevmode\hypertarget{ref-CytohubbaIdenChin2014}{}%
7. Chin, C.-H. \emph{et al.} CytoHubba: Identifying hub objects and sub-networks from complex interactome. \emph{BMC Systems Biology} \textbf{8}, S11 (2014).

\leavevmode\hypertarget{ref-HbxIncreasesCZheng2022}{}%
8. Zheng, C., Liu, M., Ge, Y., Qian, Y. \& Fan, H. HBx increases chromatin accessibility and etv4 expression to regulate dishevelled-2 and promote hcc progression. \emph{Cell death \textbackslash\& disease} \textbf{13}, (2022).
\end{cslreferences}

\end{document}
