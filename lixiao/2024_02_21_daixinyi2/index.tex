% Options for packages loaded elsewhere
\PassOptionsToPackage{unicode}{hyperref}
\PassOptionsToPackage{hyphens}{url}
%
\documentclass[
  ignorenonframetext,
]{beamer}
\usepackage{pgfpages}
\setbeamertemplate{caption}[numbered]
\setbeamertemplate{caption label separator}{: }
\setbeamercolor{caption name}{fg=normal text.fg}
\beamertemplatenavigationsymbolsempty
% Prevent slide breaks in the middle of a paragraph
\widowpenalties 1 10000
\raggedbottom
\setbeamertemplate{part page}{
  \centering
  \begin{beamercolorbox}[sep=16pt,center]{part title}
    \usebeamerfont{part title}\insertpart\par
  \end{beamercolorbox}
}
\setbeamertemplate{section page}{
  \centering
  \begin{beamercolorbox}[sep=12pt,center]{part title}
    \usebeamerfont{section title}\insertsection\par
  \end{beamercolorbox}
}
\setbeamertemplate{subsection page}{
  \centering
  \begin{beamercolorbox}[sep=8pt,center]{part title}
    \usebeamerfont{subsection title}\insertsubsection\par
  \end{beamercolorbox}
}
\AtBeginPart{
  \frame{\partpage}
}
\AtBeginSection{
  \ifbibliography
  \else
    \frame{\sectionpage}
  \fi
}
\AtBeginSubsection{
  \frame{\subsectionpage}
}
\usepackage{lmodern}
\usepackage{amssymb,amsmath}
\usepackage{ifxetex,ifluatex}
\ifnum 0\ifxetex 1\fi\ifluatex 1\fi=0 % if pdftex
  \usepackage[T1]{fontenc}
  \usepackage[utf8]{inputenc}
  \usepackage{textcomp} % provide euro and other symbols
\else % if luatex or xetex
  \usepackage{unicode-math}
  \defaultfontfeatures{Scale=MatchLowercase}
  \defaultfontfeatures[\rmfamily]{Ligatures=TeX,Scale=1}
\fi
% Use upquote if available, for straight quotes in verbatim environments
\IfFileExists{upquote.sty}{\usepackage{upquote}}{}
\IfFileExists{microtype.sty}{% use microtype if available
  \usepackage[]{microtype}
  \UseMicrotypeSet[protrusion]{basicmath} % disable protrusion for tt fonts
}{}
\makeatletter
\@ifundefined{KOMAClassName}{% if non-KOMA class
  \IfFileExists{parskip.sty}{%
    \usepackage{parskip}
  }{% else
    \setlength{\parindent}{0pt}
    \setlength{\parskip}{6pt plus 2pt minus 1pt}}
}{% if KOMA class
  \KOMAoptions{parskip=half}}
\makeatother
\usepackage{xcolor}
\IfFileExists{xurl.sty}{\usepackage{xurl}}{} % add URL line breaks if available
\IfFileExists{bookmark.sty}{\usepackage{bookmark}}{\usepackage{hyperref}}
\hypersetup{
  hidelinks,
  pdfcreator={LaTeX via pandoc}}
\urlstyle{same} % disable monospaced font for URLs
\newif\ifbibliography
\setlength{\emergencystretch}{3em} % prevent overfull lines
\providecommand{\tightlist}{%
  \setlength{\itemsep}{0pt}\setlength{\parskip}{0pt}}
\setcounter{secnumdepth}{-\maxdimen} % remove section numbering

\author{}
\date{\vspace{-2.5em}}

\begin{document}

\begin{frame}
\begin{titlepage} \newgeometry{top=7.5cm}
\ThisCenterWallPaper{1.12}{~/outline/lixiao//cover_page.pdf}
\begin{center} \textbf{\Huge HNRNPH1
与瘢痕增生的关联性挖掘} \vspace{4em}
\begin{textblock}{10}(3,5.9) \huge
\textbf{\textcolor{white}{2024-04-25}}
\end{textblock} \begin{textblock}{10}(3,7.3)
\Large \textcolor{black}{LiChuang Huang}
\end{textblock} \begin{textblock}{10}(3,11.3)
\Large \textcolor{black}{@立效研究院}
\end{textblock} \end{center} \end{titlepage}
\restoregeometry

\pagenumbering{roman}

\tableofcontents

\listoffigures

\listoftables



\pagenumbering{arabic}
\end{frame}

\begin{frame}{摘要}
\protect\hypertarget{abstract}{}
\begin{enumerate}
\tightlist
\item
  客户的 RNA-seq 数据集,以 DEGs 建立 PPI 网络,试分析 HNRNPH1
  的作用,以及 wnt 通路。
\item
  scRNA-seq (可能需要两组数据,瘢痕增生 (SH) 和正常组织), HNRNPH1
  的作用,免疫细胞的行为,免疫细胞的 DEGs。
\end{enumerate}

\begin{itemize}
\tightlist
\item
  拟时分析,HNRNPH1、TCF4 的拟时表达变化等
\item
  细胞通讯,巨噬细胞等的通讯,Wnt 通路相关基因的表达和通讯
\end{itemize}

\begin{enumerate}
\setcounter{enumi}{2}
\tightlist
\item
  姜黄素对 HNRNPH1 的作用
  (直接作用还是间接,是否可以结合,可以尝试分子对接,或者从转录因子角度出发)
\item
  视结果整理,可做一些新的分析,或探究一些新的方法。
\item
  源代码
\end{enumerate}
\end{frame}

\begin{frame}{前言}
\protect\hypertarget{introduction}{}
\end{frame}

\begin{frame}[fragile]{材料和方法}
\protect\hypertarget{methods}{}
\begin{block}{材料}
\protect\hypertarget{ux6750ux6599}{}
All used GEO expression data and their design:

\begin{itemize}
\tightlist
\item
  \textbf{GSE156326}: Single-cell transcriptome of human hypertrophic
  scars and human skin, and 6 and 8 weeks old mouse scars
  \textgreater\textgreater\textgreater{} Raw data are unvailable due to
  patient privacy concerns \textless\textless\textless{}
\end{itemize}
\end{block}

\begin{block}{方法}
\protect\hypertarget{ux65b9ux6cd5}{}
Mainly used method:

\begin{itemize}
\tightlist
\item
  R package \texttt{ClusterProfiler} used for gene enrichment
  analysis{[}@ClusterprofilerWuTi2021{]}.
\item
  R package \texttt{ClusterProfiler} used for GSEA
  enrichment{[}@ClusterprofilerWuTi2021{]}.
\item
  GEO \url{https://www.ncbi.nlm.nih.gov/geo/} used for expression
  dataset aquisition.
\item
  R package \texttt{Limma} and \texttt{edgeR} used for differential
  expression analysis{[}@LimmaPowersDiRitchi2015; @EdgerDifferenChen{]}.
\item
  R package \texttt{STEINGdb} used for PPI network
  construction{[}@TheStringDataSzklar2021; @CytohubbaIdenChin2014{]}.
\item
  The R package \texttt{Seurat} used for scRNA-seq
  processing{[}@IntegratedAnalHaoY2021; @ComprehensiveIStuart2019{]}.
\item
  R package \texttt{pathview} used for KEGG pathways
  visualization{[}@PathviewAnRLuoW2013{]}.
\item
  The MCC score was calculated referring to algorithm of
  \texttt{CytoHubba}{[}@CytohubbaIdenChin2014{]}.
\item
  \texttt{SCSA} (python) used for cell type
  annotation{[}@ScsaACellTyCaoY2020{]}.
\item
  R version 4.3.3 (2024-02-29); Other R packages (eg., \texttt{dplyr}
  and \texttt{ggplot2}) used for statistic analysis or data
  visualization.
\end{itemize}
\end{block}
\end{frame}

\begin{frame}{分析结果}
\protect\hypertarget{results}{}
\end{frame}

\begin{frame}{结论}
\protect\hypertarget{dis}{}
\end{frame}

\begin{frame}{附:分析流程}
\protect\hypertarget{workflow}{}
\begin{block}{HNRNPH1 与 PPI 网络分析}
\protect\hypertarget{hnrnph1-ux4e0e-ppi-ux7f51ux7edcux5206ux6790}{}
\begin{block}{DEGs}
\protect\hypertarget{degs}{}
\end{block}

\begin{block}{wnt 信号通路}
\protect\hypertarget{wnt-ux4fe1ux53f7ux901aux8def}{}
\end{block}

\begin{block}{构建 PPI 网络}
\protect\hypertarget{ux6784ux5efa-ppi-ux7f51ux7edc}{}
\begin{block}{DEGs PPI}
\protect\hypertarget{degs-ppi}{}
\end{block}

\begin{block}{HNRNPH1 与 Wnt 通路}
\protect\hypertarget{hnrnph1-ux4e0e-wnt-ux901aux8def}{}
HNRNPH1 与 TP53 protein-docking
\end{block}
\end{block}
\end{block}

\begin{block}{HNRNPH1 与斑痕的 scRNA-seq 分析}
\protect\hypertarget{hnrnph1-ux4e0eux6591ux75d5ux7684-scrna-seq-ux5206ux6790}{}
\begin{block}{数据来源}
\protect\hypertarget{ux6570ux636eux6765ux6e90}{}
\end{block}

\begin{block}{细胞聚类}
\protect\hypertarget{ux7ec6ux80deux805aux7c7b}{}
\end{block}
\end{block}

\begin{block}{HNRNPH1 与姜黄素}
\protect\hypertarget{hnrnph1-ux4e0eux59dcux9ec4ux7d20}{}
\end{block}
\end{frame}

\end{document}
