% Options for packages loaded elsewhere
\PassOptionsToPackage{unicode}{hyperref}
\PassOptionsToPackage{hyphens}{url}
%
\documentclass[
]{article}
\usepackage{lmodern}
\usepackage{amssymb,amsmath}
\usepackage{ifxetex,ifluatex}
\ifnum 0\ifxetex 1\fi\ifluatex 1\fi=0 % if pdftex
  \usepackage[T1]{fontenc}
  \usepackage[utf8]{inputenc}
  \usepackage{textcomp} % provide euro and other symbols
\else % if luatex or xetex
  \usepackage{unicode-math}
  \defaultfontfeatures{Scale=MatchLowercase}
  \defaultfontfeatures[\rmfamily]{Ligatures=TeX,Scale=1}
\fi
% Use upquote if available, for straight quotes in verbatim environments
\IfFileExists{upquote.sty}{\usepackage{upquote}}{}
\IfFileExists{microtype.sty}{% use microtype if available
  \usepackage[]{microtype}
  \UseMicrotypeSet[protrusion]{basicmath} % disable protrusion for tt fonts
}{}
\makeatletter
\@ifundefined{KOMAClassName}{% if non-KOMA class
  \IfFileExists{parskip.sty}{%
    \usepackage{parskip}
  }{% else
    \setlength{\parindent}{0pt}
    \setlength{\parskip}{6pt plus 2pt minus 1pt}}
}{% if KOMA class
  \KOMAoptions{parskip=half}}
\makeatother
\usepackage{xcolor}
\IfFileExists{xurl.sty}{\usepackage{xurl}}{} % add URL line breaks if available
\IfFileExists{bookmark.sty}{\usepackage{bookmark}}{\usepackage{hyperref}}
\hypersetup{
  hidelinks,
  pdfcreator={LaTeX via pandoc}}
\urlstyle{same} % disable monospaced font for URLs
\usepackage[margin=1in]{geometry}
\usepackage{longtable,booktabs}
% Correct order of tables after \paragraph or \subparagraph
\usepackage{etoolbox}
\makeatletter
\patchcmd\longtable{\par}{\if@noskipsec\mbox{}\fi\par}{}{}
\makeatother
% Allow footnotes in longtable head/foot
\IfFileExists{footnotehyper.sty}{\usepackage{footnotehyper}}{\usepackage{footnote}}
\makesavenoteenv{longtable}
\usepackage{graphicx}
\makeatletter
\def\maxwidth{\ifdim\Gin@nat@width>\linewidth\linewidth\else\Gin@nat@width\fi}
\def\maxheight{\ifdim\Gin@nat@height>\textheight\textheight\else\Gin@nat@height\fi}
\makeatother
% Scale images if necessary, so that they will not overflow the page
% margins by default, and it is still possible to overwrite the defaults
% using explicit options in \includegraphics[width, height, ...]{}
\setkeys{Gin}{width=\maxwidth,height=\maxheight,keepaspectratio}
% Set default figure placement to htbp
\makeatletter
\def\fps@figure{htbp}
\makeatother
\usepackage[normalem]{ulem}
% Avoid problems with \sout in headers with hyperref
\pdfstringdefDisableCommands{\renewcommand{\sout}{}}
\setlength{\emergencystretch}{3em} % prevent overfull lines
\providecommand{\tightlist}{%
  \setlength{\itemsep}{0pt}\setlength{\parskip}{0pt}}
\setcounter{secnumdepth}{5}
\usepackage{caption} \captionsetup{font={footnotesize},width=6in} \renewcommand{\dblfloatpagefraction}{.9} \makeatletter \renewenvironment{figure} {\def\@captype{figure}} \makeatother \@ifundefined{Shaded}{\newenvironment{Shaded}} \@ifundefined{snugshade}{\newenvironment{snugshade}} \renewenvironment{Shaded}{\begin{snugshade}}{\end{snugshade}} \definecolor{shadecolor}{RGB}{230,230,230} \usepackage{xeCJK} \usepackage{setspace} \setstretch{1.3} \usepackage{tcolorbox} \setcounter{secnumdepth}{4} \setcounter{tocdepth}{4} \usepackage{wallpaper} \usepackage[absolute]{textpos} \tcbuselibrary{breakable} \renewenvironment{Shaded} {\begin{tcolorbox}[colback = gray!10, colframe = gray!40, width = 16cm, arc = 1mm, auto outer arc, title = {R input}]} {\end{tcolorbox}} \usepackage{titlesec} \titleformat{\paragraph} {\fontsize{10pt}{0pt}\bfseries} {\arabic{section}.\arabic{subsection}.\arabic{subsubsection}.\arabic{paragraph}} {1em} {} []
\newlength{\cslhangindent}
\setlength{\cslhangindent}{1.5em}
\newenvironment{cslreferences}%
  {}%
  {\par}

\author{}
\date{\vspace{-2.5em}}

\begin{document}

\begin{titlepage} \newgeometry{top=7.5cm}
\ThisCenterWallPaper{1.12}{~/outline/lixiao//cover_page.pdf}
\begin{center} \textbf{\Huge
中药复方乌梅丸网络药理学分析} \vspace{4em}
\begin{textblock}{10}(3,5.9) \huge
\textbf{\textcolor{white}{2024-04-19}}
\end{textblock} \begin{textblock}{10}(3,7.3)
\Large \textcolor{black}{LiChuang Huang}
\end{textblock} \begin{textblock}{10}(3,11.3)
\Large \textcolor{black}{@立效研究院}
\end{textblock} \end{center} \end{titlepage}
\restoregeometry

\pagenumbering{roman}

\tableofcontents

\listoffigures

\listoftables

\newpage

\pagenumbering{arabic}

\hypertarget{abstract}{%
\section{摘要}\label{abstract}}

\hypertarget{ux9700ux6c42}{%
\subsection{需求}\label{ux9700ux6c42}}

中药复方乌梅丸:乌梅,花椒,细辛,黄连,黄柏,干姜,附子,桂枝,人参,当归

疾病:慢性结肠炎,炎性肠病,肠纤维化,可以都包含,也可以单独一种疾病(如果单独疾病可以做出来,优先级按照语序来)

目标:筛出有效成分XX,及其作用靶点蛋白YY,YY需满足:1、与XX能对接 2、富集分析显示YY与疾病相关的机制相关(比如炎症,纤维化,再放宽可以免疫细胞调控)

其它:

1、选取中药复方(乌梅丸)中和调控纤维化相关的单体成分,结合pubchem、chemical book、scifinder等数据库分析排名靠前的化合物的活性信息,并通过中医网络药理学方法(如TCMSP平台和BATMAN-TCM数据库),分析有效成分XXX的对应靶点YYY,功能富集分析显示YYY调控肠道纤维化。

2、、通过PubChem数据库获取中药单体主要活性成分的2D化学结构,在PDB数据库中查找相关核心靶点蛋白3D结构,通过Autodock软件进行分子对接,获取结合能最高的位点,最后通过Pymol软件进行可视化处理。

化合物 3081405

交付:

\begin{enumerate}
\def\labelenumi{\arabic{enumi}.}
\tightlist
\item
  疾病-复方-成分-靶点网络图
\item
  成分XX-靶点网络图
\item
  XX与YY分子对接pymol可视化图,注意细节标注
\item
  成分XX靶点功能富集分析
\item
  总的复方靶点的功能富集分析
\item
  YY可能参与的环节需要标注在4上或者单独注释分
\end{enumerate}

\hypertarget{ux7ed3ux679c}{%
\subsection{结果}\label{ux7ed3ux679c}}

注:以下 ``??'' 引用为修改后暂保留内容。

\begin{enumerate}
\def\labelenumi{\arabic{enumi}.}
\tightlist
\item
  利用 \sout{BATMAN-TCM} 数据库作为成分靶点数据库,并结合 Fig. \ref{fig:Overall-targets-number-of-datasets} 所示的疾病靶点数据
  获得的疾病-复方-成分-靶点网络图见 Fig. \ref{fig:Network-pharmacology-with-disease}
\item
  筛选的成分的靶点关系图见 Fig. \ref{fig:TOP-pharmacology-visualization}。
  \sout{这里的成分是后续的分析和分子对接筛选的 TOP 1,其名称等相关信息 (TOP 1-3) 可参考 Tab. \ref{tab:Metadata-of-visualized-Docking}}
\item
  \sout{Pymol 可视化见 Fig. \ref{fig:Docking-72326-into-9ilb-detail} (局部放大加注释),
  Fig. \ref{fig:Docking-72326-into-9ilb} (全局图)。
  此外,对接 TOP 2 和 TOP 3 的可视化也附在随后。}
\item
  \sout{TOP 1 成分的富集分析见 Fig. \ref{fig:TOP-KEGG-enrichment} 和 Fig. \ref{fig:TOP-GO-enrichment}}
\item
  总的复方靶点的富集分析见 Fig. \ref{fig:HERBS-KEGG-enrichment} 和 Fig. \ref{fig:HERBS-GO-enrichment}
\item
  \sout{YY (TOP 1 对应的结合靶点为 IL1B ) 参与的环节见 Fig. \ref{fig:TOP-hsa05321-visualization}}
\end{enumerate}

补充说明:

\begin{itemize}
\tightlist
\item
  TCMSP 网站最近几日都无法打开,所以草药数据来源只选用 BATMAN (这个数据库比 TCMSP 全面) 。
\item
  关于 ``结合pubchem、chemical book、scifinder等数据库分析排名靠前的化合物的活性信息'',
  chemical book 和 scifinder 为商业工具,预计是无法获取权限的,这里没有使用;
  而 PubChem,我这里的分析中获取了成分的文献记录,即 LiteratureCount,
  具体可见 Tab. \ref{tab:All-compounds-Literature-Count},
  Tab. \ref{tab:hsa05321-related-genes-and-compounds}。
  此外,还根据 CTD 对疾病相关的成分做了筛选,
  Fig. \ref{fig:Intersection-of-CTD-records-with-herbs-of-hsa05321-related}
\item
  其它候选成分靶点 Tab. \ref{tab:Intersection-Herbs-compounds-and-targets}
\item
  分子对接良好的结果汇总表格 Tab. \ref{tab:Combining-Affinity}
\end{itemize}

\hypertarget{ux8865ux5145ux7684ux5185ux5bb9}{%
\subsection{补充的内容}\label{ux8865ux5145ux7684ux5185ux5bb9}}

\begin{itemize}
\tightlist
\item
  分子对接前的网络图 Fig. \ref{fig:CTD-filtered-Compounds-Network-pharmacology-with-disease},
  仅根据 Fig. \ref{fig:Overall-targets-number-of-datasets} 过滤靶点,和
  Fig. \ref{fig:Intersection-of-CTD-records-with-herbs-of-hsa05321-related} 过滤成分。
\item
  随后分子对接已全部重做。
\item
  分子对接后,筛选 Affinity \textless{} -1.2, 网络图 Fig. \ref{fig:Network-pharmacology-Affinity-filtered}
  (唯独 Fig. \ref{fig:Network-pharmacology-Affinity-filtered} 中化合物采用了最简洁的同义名,其他图没有修改;
  此外,Tab. \ref{tab:Combining-Affinity} 有化合物名称和来源药物)
\item
  关于化合物 3081405, 存在于收集的复方成分中,
  可在 Tab. \ref{tab:tables-of-Herbs-compounds-and-targets} 中找到;
  但不在 Tab. \ref{tab:Intersection-Herbs-compounds-and-targets} 中,
  是 CTD 的步骤过滤除外的 (Fig. \ref{fig:Intersection-of-CTD-records-with-herbs-of-hsa05321-related})。
\end{itemize}

\hypertarget{ux518dux6b21ux4feeux6539ux7684ux5185ux5bb9}{%
\subsection{再次修改的内容}\label{ux518dux6b21ux4feeux6539ux7684ux5185ux5bb9}}

\begin{itemize}
\tightlist
\item
  数据库已换成 TCMSP, 根据 DL 和 OB 过滤成分, Tab. \ref{tab:Compounds-filtered-by-OB-and-DL}
  (修改数据库来源后,全部内容已重做)
\item
  疾病靶点与成分靶点做交后,不再与IBD通路做交
\item
  成分-靶点通过批量分子对接滤过能量 \textgreater-1.2kcal/mol 的靶点,
  随后创建网络药理图 (对接能量为负值的结果见 Tab. \ref{tab:Combining-Affinity})
\item
  中药-成分-靶点基因-相关通路, 见 Fig. \ref{fig:Network-pharmacology-Affinity-filtered},对应数据表格见
  Tab. \ref{tab:Network-pharmacology-Affinity-filtered-data}
\end{itemize}

\hypertarget{introduction}{%
\section{前言}\label{introduction}}

\hypertarget{methods}{%
\section{材料和方法}\label{methods}}

\hypertarget{ux6750ux6599}{%
\subsection{材料}\label{ux6750ux6599}}

\hypertarget{ux65b9ux6cd5}{%
\subsection{方法}\label{ux65b9ux6cd5}}

Mainly used method:

\begin{itemize}
\tightlist
\item
  Database \texttt{BATMAN-TCM} was used as source data of TCM ingredients and target proteins\textsuperscript{\protect\hyperlink{ref-BatmanTcm20Kong2024}{1}}.
\item
  The Comparative Toxicogenomics Database (CTD) used for finding relationship between chemicals and disease\textsuperscript{\protect\hyperlink{ref-ComparativeToxDavis2023}{2}}.
\item
  R package \texttt{ClusterProfiler} used for gene enrichment analysis\textsuperscript{\protect\hyperlink{ref-ClusterprofilerWuTi2021}{3}}.
\item
  Databses of \texttt{DisGeNet}, \texttt{GeneCards}, \texttt{PharmGKB} used for collating disease related targets\textsuperscript{\protect\hyperlink{ref-TheDisgenetKnPinero2019}{4}--\protect\hyperlink{ref-PharmgkbAWorBarbar2018}{6}}.
\item
  The API of \texttt{UniProtKB} (\url{https://www.uniprot.org/help/api_queries}) used for mapping of names or IDs of proteins.
\item
  Website \texttt{TCMSP} \url{https://tcmsp-e.com/} used for data source\textsuperscript{\protect\hyperlink{ref-TcmspADatabaRuJi2014}{7}}.
\item
  The CLI tools of \texttt{AutoDock\ vina} and \texttt{ADFR} software used for auto molecular docking\textsuperscript{\protect\hyperlink{ref-AutodockVina1Eberha2021}{8}--\protect\hyperlink{ref-AutodockfrAdvRavind2015}{12}}.
\item
  R package \texttt{pathview} used for KEGG pathways visualization\textsuperscript{\protect\hyperlink{ref-PathviewAnRLuoW2013}{13}}.
\item
  R version 4.3.2 (2023-10-31); Other R packages (eg., \texttt{dplyr} and \texttt{ggplot2}) used for statistic analysis or data visualization.
\end{itemize}

\hypertarget{results}{%
\section{分析结果}\label{results}}

\hypertarget{dis}{%
\section{结论}\label{dis}}

\hypertarget{workflow}{%
\section{附:分析流程}\label{workflow}}

\hypertarget{ux7f51ux7edcux836fux7406ux5b66}{%
\subsection{网络药理学}\label{ux7f51ux7edcux836fux7406ux5b66}}

\hypertarget{ux6210ux5206}{%
\subsubsection{成分}\label{ux6210ux5206}}

Table \ref{tab:Herbs-information} (下方表格) 为表格Herbs information概览。

\textbf{(对应文件为 \texttt{Figure+Table/Herbs-information.csv})}

\begin{center}\begin{tcolorbox}[colback=gray!10, colframe=gray!50, width=0.9\linewidth, arc=1mm, boxrule=0.5pt]注:表格共有10行2列,以下预览的表格可能省略部分数据;含有10个唯一`Herb\_pinyin\_name'。
\end{tcolorbox}
\end{center}

\begin{longtable}[]{@{}ll@{}}
\caption{\label{tab:Herbs-information}Herbs information}\tabularnewline
\toprule
Herb\_pinyin\_name & Herb\_cn\_name\tabularnewline
\midrule
\endfirsthead
\toprule
Herb\_pinyin\_name & Herb\_cn\_name\tabularnewline
\midrule
\endhead
Wumei & 乌梅\tabularnewline
Huajiao & 花椒\tabularnewline
Xixin & 细辛\tabularnewline
Huanglian & 黄连\tabularnewline
Huangbo & 黄柏\tabularnewline
Ganjiang & 干姜\tabularnewline
Fuzi & 附子\tabularnewline
Guizhi & 桂枝\tabularnewline
Renshen & 人参\tabularnewline
Danggui & 当归\tabularnewline
\bottomrule
\end{longtable}

Table \ref{tab:Compounds-filtered-by-OB-and-DL} (下方表格) 为表格Compounds filtered by OB and DL概览。

\textbf{(对应文件为 \texttt{Figure+Table/Compounds-filtered-by-OB-and-DL.xlsx})}

\begin{center}\begin{tcolorbox}[colback=gray!10, colframe=gray!50, width=0.9\linewidth, arc=1mm, boxrule=0.5pt]注:表格共有129行15列,以下预览的表格可能省略部分数据;含有102个唯一`Mol ID;含有10个唯一`Herb\_pinyin\_name'。
\end{tcolorbox}
\end{center}\begin{center}\begin{tcolorbox}[colback=gray!10, colframe=gray!50, width=0.9\linewidth, arc=1mm, boxrule=0.5pt]
\textbf{
OB (%) cut-off
:}

\vspace{0.5em}

    30%

\vspace{2em}


\textbf{
DL cut-off
:}

\vspace{0.5em}

    0.18

\vspace{2em}
\end{tcolorbox}
\end{center}

\begin{longtable}[]{@{}llllllllll@{}}
\caption{\label{tab:Compounds-filtered-by-OB-and-DL}Compounds filtered by OB and DL}\tabularnewline
\toprule
Mol ID & Molecu\ldots{} & MW & AlogP & Hdon & Hacc & OB (\%) & Caco-2 & BBB & DL\tabularnewline
\midrule
\endfirsthead
\toprule
Mol ID & Molecu\ldots{} & MW & AlogP & Hdon & Hacc & OB (\%) & Caco-2 & BBB & DL\tabularnewline
\midrule
\endhead
MOL001040 & (2R)-5\ldots{} & 272.270 & 2.298 & 3 & 5 & 42.363\ldots{} & 0.37818 & -0.47578 & 0.21141\tabularnewline
MOL000358 & beta-s\ldots{} & 414.790 & 8.084 & 1 & 1 & 36.913\ldots{} & 1.32463 & 0.98588 & 0.75123\tabularnewline
MOL000422 & kaempf\ldots{} & 286.250 & 1.771 & 4 & 6 & 41.882\ldots{} & 0.26096 & -0.55335 & 0.24066\tabularnewline
MOL000449 & Stigma\ldots{} & 412.770 & 7.640 & 1 & 1 & 43.829\ldots{} & 1.44458 & 1.00045 & 0.75665\tabularnewline
MOL005043 & campes\ldots{} & 400.760 & 7.628 & 1 & 1 & 37.576\ldots{} & 1.31892 & 0.93697 & 0.71481\tabularnewline
MOL008601 & Methyl\ldots{} & 318.550 & 6.665 & 0 & 2 & 46.899\ldots{} & 1.48280 & 0.92545 & 0.23381\tabularnewline
MOL000953 & CLR & 386.730 & 7.376 & 1 & 1 & 37.873\ldots{} & 1.43101 & 1.12678 & 0.67677\tabularnewline
MOL000098 & quercetin & 302.250 & 1.504 & 5 & 7 & 46.433\ldots{} & 0.04842 & -0.76890 & 0.27525\tabularnewline
MOL013271 & Kokusa\ldots{} & 259.280 & 2.330 & 0 & 5 & 66.676\ldots{} & 0.94967 & 0.66840 & 0.19584\tabularnewline
MOL002663 & Skimmi\ldots{} & 259.280 & 2.330 & 0 & 5 & 40.136\ldots{} & 1.26344 & 1.09995 & 0.19638\tabularnewline
MOL002881 & Diosmetin & 300.280 & 2.318 & 3 & 6 & 31.137\ldots{} & 0.46152 & -0.66187 & 0.27442\tabularnewline
MOL000358 & beta-s\ldots{} & 414.790 & 8.084 & 1 & 1 & 36.913\ldots{} & 1.32463 & 0.98588 & 0.75123\tabularnewline
MOL000098 & quercetin & 302.250 & 1.504 & 5 & 7 & 46.433\ldots{} & 0.04842 & -0.76890 & 0.27525\tabularnewline
MOL012140 & 4,9-di\ldots{} & 254.310 & 3.375 & 0 & 3 & 65.301\ldots{} & 1.21324 & 0.72110 & 0.19237\tabularnewline
MOL012141 & Caribine & 326.430 & 1.220 & 2 & 5 & 37.064\ldots{} & 0.33508 & -0.14706 & 0.82656\tabularnewline
\ldots{} & \ldots{} & \ldots{} & \ldots{} & \ldots{} & \ldots{} & \ldots{} & \ldots{} & \ldots{} & \ldots{}\tabularnewline
\bottomrule
\end{longtable}

Figure \ref{fig:intersection-of-all-compounds} (下方图) 为图intersection of all compounds概览。

\textbf{(对应文件为 \texttt{Figure+Table/intersection-of-all-compounds.pdf})}

\def\@captype{figure}
\begin{center}
\includegraphics[width = 0.9\linewidth]{Figure+Table/intersection-of-all-compounds.pdf}
\caption{Intersection of all compounds}\label{fig:intersection-of-all-compounds}
\end{center}
\begin{center}\begin{tcolorbox}[colback=gray!10, colframe=gray!50, width=0.9\linewidth, arc=1mm, boxrule=0.5pt]
\textbf{
All\_intersection
:}

\vspace{0.5em}



\vspace{2em}
\end{tcolorbox}
\end{center}

\textbf{(上述信息框内容已保存至 \texttt{Figure+Table/intersection-of-all-compounds-content})}

\hypertarget{ux6210ux5206ux9776ux70b9}{%
\subsubsection{成分靶点}\label{ux6210ux5206ux9776ux70b9}}

Table \ref{tab:tables-of-Herbs-compounds-and-targets} (下方表格) 为表格tables of Herbs compounds and targets概览。

\textbf{(对应文件为 \texttt{Figure+Table/tables-of-Herbs-compounds-and-targets.xlsx})}

\begin{center}\begin{tcolorbox}[colback=gray!10, colframe=gray!50, width=0.9\linewidth, arc=1mm, boxrule=0.5pt]注:表格共有3763行4列,以下预览的表格可能省略部分数据;含有10个唯一`Herb\_pinyin\_name'。
\end{tcolorbox}
\end{center}

\begin{longtable}[]{@{}llll@{}}
\caption{\label{tab:tables-of-Herbs-compounds-and-targets}Tables of Herbs compounds and targets}\tabularnewline
\toprule
Herb\_pinyin\_name & Molecule name & symbols & protein.names\tabularnewline
\midrule
\endfirsthead
\toprule
Herb\_pinyin\_name & Molecule name & symbols & protein.names\tabularnewline
\midrule
\endhead
Guizhi & ent-Epicatechin & MBLAC1 & Metallo-beta-lact\ldots{}\tabularnewline
Guizhi & ent-Epicatechin & NCOA7 & Nuclear receptor \ldots{}\tabularnewline
Guizhi & ent-Epicatechin & ERAP140 & Nuclear receptor \ldots{}\tabularnewline
Guizhi & ent-Epicatechin & ESNA1 & Nuclear receptor \ldots{}\tabularnewline
Guizhi & ent-Epicatechin & Nbla00052 & Nuclear receptor \ldots{}\tabularnewline
Guizhi & ent-Epicatechin & Nbla10993 & Nuclear receptor \ldots{}\tabularnewline
Guizhi & ent-Epicatechin & HSP90AA2P & Heat shock protei\ldots{}\tabularnewline
Guizhi & ent-Epicatechin & HSP90AA2 & Heat shock protei\ldots{}\tabularnewline
Guizhi & ent-Epicatechin & HSPCAL3 & Heat shock protei\ldots{}\tabularnewline
Guizhi & ent-Epicatechin & PTGS1 & Prostaglandin G/H\ldots{}\tabularnewline
Guizhi & ent-Epicatechin & COX1 & Prostaglandin G/H\ldots{}\tabularnewline
Guizhi & ent-Epicatechin & PTGS2 & Prostaglandin G/H\ldots{}\tabularnewline
Guizhi & ent-Epicatechin & COX2 & Prostaglandin G/H\ldots{}\tabularnewline
Guizhi & ent-Epicatechin & PRKACA & cAMP-dependent pr\ldots{}\tabularnewline
Guizhi & ent-Epicatechin & PKACA & cAMP-dependent pr\ldots{}\tabularnewline
\ldots{} & \ldots{} & \ldots{} & \ldots{}\tabularnewline
\bottomrule
\end{longtable}

\hypertarget{ux75beux75c5ux9776ux70b9}{%
\subsubsection{疾病靶点}\label{ux75beux75c5ux9776ux70b9}}

Figure \ref{fig:Overall-targets-number-of-datasets} (下方图) 为图Overall targets number of datasets概览。

\textbf{(对应文件为 \texttt{Figure+Table/Overall-targets-number-of-datasets.pdf})}

\def\@captype{figure}
\begin{center}
\includegraphics[width = 0.9\linewidth]{Figure+Table/Overall-targets-number-of-datasets.pdf}
\caption{Overall targets number of datasets}\label{fig:Overall-targets-number-of-datasets}
\end{center}

Table \ref{tab:GeneCards-used-data} (下方表格) 为表格GeneCards used data概览。

\textbf{(对应文件为 \texttt{Figure+Table/GeneCards-used-data.xlsx})}

\begin{center}\begin{tcolorbox}[colback=gray!10, colframe=gray!50, width=0.9\linewidth, arc=1mm, boxrule=0.5pt]注:表格共有172行7列,以下预览的表格可能省略部分数据;含有172个唯一`Symbol'。
\end{tcolorbox}
\end{center}\begin{center}\begin{tcolorbox}[colback=gray!10, colframe=gray!50, width=0.9\linewidth, arc=1mm, boxrule=0.5pt]
\textbf{
The GeneCards data was obtained by querying
:}

\vspace{0.5em}

    chronic colitis

\vspace{2em}


\textbf{
Restrict (with quotes)
:}

\vspace{0.5em}

    TRUE

\vspace{2em}


\textbf{
Filtering by Score:
:}

\vspace{0.5em}

    Score > 0

\vspace{2em}
\end{tcolorbox}
\end{center}

\begin{longtable}[]{@{}lllllll@{}}
\caption{\label{tab:GeneCards-used-data}GeneCards used data}\tabularnewline
\toprule
Symbol & Description & Category & UniProt\_ID & GIFtS & GC\_id & Score\tabularnewline
\midrule
\endfirsthead
\toprule
Symbol & Description & Category & UniProt\_ID & GIFtS & GC\_id & Score\tabularnewline
\midrule
\endhead
CARMIL2 & Capping Pr\ldots{} & Protein Co\ldots{} & Q6F5E8 & 43 & GC16P067644 & 3\tabularnewline
WAS & WASP Actin\ldots{} & Protein Co\ldots{} & P42768 & 56 & GC0XP048676 & 2.13\tabularnewline
IL37 & Interleuki\ldots{} & Protein Co\ldots{} & Q9NZH6 & 48 & GC02P141239 & 2.04\tabularnewline
LINC02605 & Long Inter\ldots{} & RNA Gene & & 19 & GC08P078838 & 2.04\tabularnewline
TNFSF15 & TNF Superf\ldots{} & Protein Co\ldots{} & O95150 & 52 & GC09M114784 & 1.96\tabularnewline
LINC01672 & Long Inter\ldots{} & RNA Gene & & 18 & GC01P011469 & 1.96\tabularnewline
STAT3 & Signal Tra\ldots{} & Protein Co\ldots{} & P40763 & 62 & GC17M042313 & 1.91\tabularnewline
BDNF-AS & BDNF Antis\ldots{} & RNA Gene & & 28 & GC11P027466 & 1.91\tabularnewline
CERNA3 & Competing \ldots{} & RNA Gene & & 19 & GC08P056101 & 1.8\tabularnewline
NOS2 & Nitric Oxi\ldots{} & Protein Co\ldots{} & P35228 & 58 & GC17M027756 & 1.73\tabularnewline
IL17A & Interleuki\ldots{} & Protein Co\ldots{} & Q16552 & 52 & GC06P052186 & 1.69\tabularnewline
IFNG & Interferon\ldots{} & Protein Co\ldots{} & P01579 & 59 & GC12M068154 & 1.49\tabularnewline
IL23A & Interleuki\ldots{} & Protein Co\ldots{} & Q9NPF7 & 48 & GC12P059649 & 1.44\tabularnewline
MPO & Myeloperox\ldots{} & Protein Co\ldots{} & P05164 & 61 & GC17M058269 & 1.38\tabularnewline
IL7R & Interleuki\ldots{} & Protein Co\ldots{} & P16871 & 55 & GC05P035852 & 1.38\tabularnewline
\ldots{} & \ldots{} & \ldots{} & \ldots{} & \ldots{} & \ldots{} & \ldots{}\tabularnewline
\bottomrule
\end{longtable}

\hypertarget{ux75beux75c5-ux6210ux5206-ux9776ux70b9}{%
\subsubsection{疾病-成分-靶点}\label{ux75beux75c5-ux6210ux5206-ux9776ux70b9}}

Figure \ref{fig:Network-pharmacology-with-disease} (下方图) 为图Network pharmacology with disease概览。

\textbf{(对应文件为 \texttt{Figure+Table/Network-pharmacology-with-disease.pdf})}

\def\@captype{figure}
\begin{center}
\includegraphics[width = 0.9\linewidth]{Figure+Table/Network-pharmacology-with-disease.pdf}
\caption{Network pharmacology with disease}\label{fig:Network-pharmacology-with-disease}
\end{center}

Figure \ref{fig:Targets-intersect-with-targets-of-diseases} (下方图) 为图Targets intersect with targets of diseases概览。

\textbf{(对应文件为 \texttt{Figure+Table/Targets-intersect-with-targets-of-diseases.pdf})}

\def\@captype{figure}
\begin{center}
\includegraphics[width = 0.9\linewidth]{Figure+Table/Targets-intersect-with-targets-of-diseases.pdf}
\caption{Targets intersect with targets of diseases}\label{fig:Targets-intersect-with-targets-of-diseases}
\end{center}
\begin{center}\begin{tcolorbox}[colback=gray!10, colframe=gray!50, width=0.9\linewidth, arc=1mm, boxrule=0.5pt]
\textbf{
Intersection
:}

\vspace{0.5em}

    IL10, TNF, IL1B, TP53, ESR2, NFE2L2, MMP9, NOS2, IFNG,
MPO, EGF, VCAM1, MAP3K7, ERBB2, MMP1, PTGS2, RAF1, HMOX1,
CDKN2A, GJA1, CD40LG, ALOX5, NCF2, IL4, VEGFC

\vspace{2em}
\end{tcolorbox}
\end{center}

\textbf{(上述信息框内容已保存至 \texttt{Figure+Table/Targets-intersect-with-targets-of-diseases-content})}

\hypertarget{ux5bccux96c6ux5206ux6790}{%
\subsubsection{富集分析}\label{ux5bccux96c6ux5206ux6790}}

Figure \ref{fig:HERBS-KEGG-enrichment} (下方图) 为图HERBS KEGG enrichment概览。

\textbf{(对应文件为 \texttt{Figure+Table/HERBS-KEGG-enrichment.pdf})}

\def\@captype{figure}
\begin{center}
\includegraphics[width = 0.9\linewidth]{Figure+Table/HERBS-KEGG-enrichment.pdf}
\caption{HERBS KEGG enrichment}\label{fig:HERBS-KEGG-enrichment}
\end{center}

Figure \ref{fig:HERBS-GO-enrichment} (下方图) 为图HERBS GO enrichment概览。

\textbf{(对应文件为 \texttt{Figure+Table/HERBS-GO-enrichment.pdf})}

\def\@captype{figure}
\begin{center}
\includegraphics[width = 0.9\linewidth]{Figure+Table/HERBS-GO-enrichment.pdf}
\caption{HERBS GO enrichment}\label{fig:HERBS-GO-enrichment}
\end{center}

\hypertarget{ux4e0eux75beux75c5ux76f8ux5173ux7684ux6d3bux6027ux6210ux5206ux7b5bux9009}{%
\subsubsection{与疾病相关的活性成分筛选}\label{ux4e0eux75beux75c5ux76f8ux5173ux7684ux6d3bux6027ux6210ux5206ux7b5bux9009}}

\hypertarget{ctd-ux6570ux636eux5e93ux8bb0ux5f55ux4e0eux80a0ux708e-colitis-ux76f8ux5173ux7684ux5316ux5408ux7269}{%
\paragraph{CTD 数据库记录与肠炎 (Colitis) 相关的化合物}\label{ctd-ux6570ux636eux5e93ux8bb0ux5f55ux4e0eux80a0ux708e-colitis-ux76f8ux5173ux7684ux5316ux5408ux7269}}

Figure \ref{fig:Intersection-of-CTD-records-with-herbs-of-hsa05321-related} (下方图) 为图Intersection of CTD records with herbs of hsa05321 related概览。

\textbf{(对应文件为 \texttt{Figure+Table/Intersection-of-CTD-records-with-herbs-of-hsa05321-related.pdf})}

\def\@captype{figure}
\begin{center}
\includegraphics[width = 0.9\linewidth]{Figure+Table/Intersection-of-CTD-records-with-herbs-of-hsa05321-related.pdf}
\caption{Intersection of CTD records with herbs of hsa05321 related}\label{fig:Intersection-of-CTD-records-with-herbs-of-hsa05321-related}
\end{center}
\begin{center}\begin{tcolorbox}[colback=gray!10, colframe=gray!50, width=0.9\linewidth, arc=1mm, boxrule=0.5pt]
\textbf{
Intersection
:}

\vspace{0.5em}

    11066, 444899, 2353, 9064, 193148, 2703, 5281612,
119307, 5280863, 19009, 5280343, 65752, 72307, 5280794,
439533

\vspace{2em}
\end{tcolorbox}
\end{center}

\textbf{(上述信息框内容已保存至 \texttt{Figure+Table/Intersection-of-CTD-records-with-herbs-of-hsa05321-related-content})}

Table \ref{tab:Intersection-Herbs-compounds-and-targets} (下方表格) 为表格Intersection Herbs compounds and targets概览。

\textbf{(对应文件为 \texttt{Figure+Table/Intersection-Herbs-compounds-and-targets.csv})}

\begin{center}\begin{tcolorbox}[colback=gray!10, colframe=gray!50, width=0.9\linewidth, arc=1mm, boxrule=0.5pt]注:表格共有141行4列,以下预览的表格可能省略部分数据;含有8个唯一`Herb\_pinyin\_name;含有15个唯一`Ingredient.name;含有24个唯一`Target.name'。
\end{tcolorbox}
\end{center}

\begin{longtable}[]{@{}llll@{}}
\caption{\label{tab:Intersection-Herbs-compounds-and-targets}Intersection Herbs compounds and targets}\tabularnewline
\toprule
Herb\_pinyin\_name & Ingredient.name & Target.name & CID\tabularnewline
\midrule
\endfirsthead
\toprule
Herb\_pinyin\_name & Ingredient.name & Target.name & CID\tabularnewline
\midrule
\endhead
Wumei & quercetin & ALOX5 & 5280343\tabularnewline
Wumei & quercetin & CD40LG & 5280343\tabularnewline
Wumei & quercetin & TP53 & 5280343\tabularnewline
Wumei & quercetin & CDKN2A & 5280343\tabularnewline
Wumei & quercetin & EGF & 5280343\tabularnewline
Wumei & quercetin & GJA1 & 5280343\tabularnewline
Wumei & quercetin & HMOX1 & 5280343\tabularnewline
Wumei & quercetin & IFNG & 5280343\tabularnewline
Wumei & quercetin & IL1B & 5280343\tabularnewline
Wumei & quercetin & IL10 & 5280343\tabularnewline
Wumei & quercetin & MMP1 & 5280343\tabularnewline
Wumei & quercetin & MMP9 & 5280343\tabularnewline
Wumei & quercetin & MAP3K7 & 5280343\tabularnewline
Wumei & quercetin & MPO & 5280343\tabularnewline
Wumei & quercetin & NCF2 & 5280343\tabularnewline
\ldots{} & \ldots{} & \ldots{} & \ldots{}\tabularnewline
\bottomrule
\end{longtable}

\hypertarget{ux5206ux5b50ux5bf9ux63a5ux524dux7684ux7f51ux7edcux56fe}{%
\subsection{分子对接前的网络图}\label{ux5206ux5b50ux5bf9ux63a5ux524dux7684ux7f51ux7edcux56fe}}

Figure \ref{fig:CTD-filtered-Compounds-Network-pharmacology-with-disease} (下方图) 为图CTD filtered Compounds Network pharmacology with disease概览。

\textbf{(对应文件为 \texttt{Figure+Table/CTD-filtered-Compounds-Network-pharmacology-with-disease.pdf})}

\def\@captype{figure}
\begin{center}
\includegraphics[width = 0.9\linewidth]{Figure+Table/CTD-filtered-Compounds-Network-pharmacology-with-disease.pdf}
\caption{CTD filtered Compounds Network pharmacology with disease}\label{fig:CTD-filtered-Compounds-Network-pharmacology-with-disease}
\end{center}

\hypertarget{ux5206ux5b50ux5bf9ux63a5}{%
\subsection{分子对接}\label{ux5206ux5b50ux5bf9ux63a5}}

\hypertarget{top-docking}{%
\subsubsection{Top docking}\label{top-docking}}

取 Fig. \ref{fig:CTD-filtered-Compounds-Network-pharmacology-with-disease} 成分与靶点,进行批量分子对接。

以下展示了各个靶点结合度 Top 的成分

Figure \ref{fig:Overall-combining-Affinity} (下方图) 为图Overall combining Affinity概览。

\textbf{(对应文件为 \texttt{Figure+Table/Overall-combining-Affinity.pdf})}

\def\@captype{figure}
\begin{center}
\includegraphics[width = 0.9\linewidth]{Figure+Table/Overall-combining-Affinity.pdf}
\caption{Overall combining Affinity}\label{fig:Overall-combining-Affinity}
\end{center}

Table \ref{tab:Combining-Affinity} (下方表格) 为表格Combining Affinity概览。

\textbf{(对应文件为 \texttt{Figure+Table/Combining-Affinity.csv})}

\begin{center}\begin{tcolorbox}[colback=gray!10, colframe=gray!50, width=0.9\linewidth, arc=1mm, boxrule=0.5pt]注:表格共有31行7列,以下预览的表格可能省略部分数据;含有15个唯一`hgnc\_symbol;含有8个唯一`Herb\_pinyin\_name'。
\end{tcolorbox}
\end{center}
\begin{center}\begin{tcolorbox}[colback=gray!10, colframe=gray!50, width=0.9\linewidth, arc=1mm, boxrule=0.5pt]\begin{enumerate}\tightlist
\item hgnc\_symbol:  基因名 (Human)
\end{enumerate}\end{tcolorbox}
\end{center}

\begin{longtable}[]{@{}lllllll@{}}
\caption{\label{tab:Combining-Affinity}Combining Affinity}\tabularnewline
\toprule
hgnc\_symbol & Ingredient\ldots{} & Affinity & PubChem\_id & PDB\_ID & Combn & Herb\_pinyi\ldots{}\tabularnewline
\midrule
\endfirsthead
\toprule
hgnc\_symbol & Ingredient\ldots{} & Affinity & PubChem\_id & PDB\_ID & Combn & Herb\_pinyi\ldots{}\tabularnewline
\midrule
\endhead
TP53 & berberine & -5.465 & 2353 & 8dc8 & 2353\_into\_\ldots{} & Huanglian;\ldots{}\tabularnewline
IL1B & quercetin & -4.52 & 5280343 & 9ilb & 5280343\_in\ldots{} & Wumei; Hua\ldots{}\tabularnewline
IL1B & ginsenosid\ldots{} & -4.4 & 119307 & 9ilb & 119307\_int\ldots{} & Renshen\tabularnewline
TP53 & quercetin & -4.325 & 5280343 & 8dc8 & 5280343\_in\ldots{} & Wumei; Hua\ldots{}\tabularnewline
ESR2 & palmatine & -4.267 & 19009 & 7xwr & 19009\_into\ldots{} & Huanglian;\ldots{}\tabularnewline
NFE2L2 & sesamin & -4.113 & 72307 & 7o7b & 72307\_into\ldots{} & Xixin\tabularnewline
NFE2L2 & Chelerythrine & -4.061 & 2703 & 7o7b & 2703\_into\_\ldots{} & Huangbo\tabularnewline
NFE2L2 & quercetin & -3.555 & 5280343 & 7o7b & 5280343\_in\ldots{} & Wumei; Hua\ldots{}\tabularnewline
EGF & quercetin & -2.956 & 5280343 & 2kv4 & 5280343\_in\ldots{} & Wumei; Hua\ldots{}\tabularnewline
HMOX1 & quercetin & -2.428 & 5280343 & 6eha & 5280343\_in\ldots{} & Wumei; Hua\ldots{}\tabularnewline
HMOX1 & kaempferol & -2.141 & 5280863 & 6eha & 5280863\_in\ldots{} & Wumei; Xix\ldots{}\tabularnewline
IL1B & Chelerythrine & -2.079 & 2703 & 9ilb & 2703\_into\_\ldots{} & Huangbo\tabularnewline
RAF1 & quercetin & -0.797 & 5280343 & 7jhp & 5280343\_in\ldots{} & Wumei; Hua\ldots{}\tabularnewline
NOS2 & palmatine & -0.521 & 19009 & 4nos & 19009\_into\ldots{} & Huanglian;\ldots{}\tabularnewline
NOS2 & berberine & -0.315 & 2353 & 4nos & 2353\_into\_\ldots{} & Huanglian;\ldots{}\tabularnewline
\ldots{} & \ldots{} & \ldots{} & \ldots{} & \ldots{} & \ldots{} & \ldots{}\tabularnewline
\bottomrule
\end{longtable}

\hypertarget{ux5bf9ux63a5ux80fdux91cf--1.2-ux7684ux6210ux5206ux4e0eux9776ux70b9ux5206ux6790}{%
\subsubsection{对接能量 \textless{} -1.2 的成分与靶点分析}\label{ux5bf9ux63a5ux80fdux91cf--1.2-ux7684ux6210ux5206ux4e0eux9776ux70b9ux5206ux6790}}

\hypertarget{ux5bf9ux5e94ux9776ux70b9ux7684ux5bccux96c6ux5206ux6790}{%
\paragraph{对应靶点的富集分析}\label{ux5bf9ux5e94ux9776ux70b9ux7684ux5bccux96c6ux5206ux6790}}

Figure \ref{fig:AFF-KEGG-enrichment} (下方图) 为图AFF KEGG enrichment概览。

\textbf{(对应文件为 \texttt{Figure+Table/AFF-KEGG-enrichment.pdf})}

\def\@captype{figure}
\begin{center}
\includegraphics[width = 0.9\linewidth]{Figure+Table/AFF-KEGG-enrichment.pdf}
\caption{AFF KEGG enrichment}\label{fig:AFF-KEGG-enrichment}
\end{center}

\hypertarget{ux4e2dux836f-ux6210ux5206-ux9776ux70b9-ux901aux8def}{%
\paragraph{中药-成分-靶点-通路}\label{ux4e2dux836f-ux6210ux5206-ux9776ux70b9-ux901aux8def}}

Figure \ref{fig:Network-pharmacology-Affinity-filtered} (下方图) 为图Network pharmacology Affinity filtered概览。

\textbf{(对应文件为 \texttt{Figure+Table/Network-pharmacology-Affinity-filtered.pdf})}

\def\@captype{figure}
\begin{center}
\includegraphics[width = 0.9\linewidth]{Figure+Table/Network-pharmacology-Affinity-filtered.pdf}
\caption{Network pharmacology Affinity filtered}\label{fig:Network-pharmacology-Affinity-filtered}
\end{center}

Table \ref{tab:Network-pharmacology-Affinity-filtered-data} (下方表格) 为表格Network pharmacology Affinity filtered data概览。

\textbf{(对应文件为 \texttt{Figure+Table/Network-pharmacology-Affinity-filtered-data.csv})}

\begin{center}\begin{tcolorbox}[colback=gray!10, colframe=gray!50, width=0.9\linewidth, arc=1mm, boxrule=0.5pt]注:表格共有26行5列,以下预览的表格可能省略部分数据;含有6个唯一`Herb\_pinyin\_name;含有4个唯一`Ingredient.name;含有6个唯一`Target.name'。
\end{tcolorbox}
\end{center}

\begin{longtable}[]{@{}lllll@{}}
\caption{\label{tab:Network-pharmacology-Affinity-filtered-data}Network pharmacology Affinity filtered data}\tabularnewline
\toprule
\begin{minipage}[b]{0.17\columnwidth}\raggedright
Herb\_pinyin\_name\strut
\end{minipage} & \begin{minipage}[b]{0.16\columnwidth}\raggedright
Ingredient.name\strut
\end{minipage} & \begin{minipage}[b]{0.12\columnwidth}\raggedright
Target.name\strut
\end{minipage} & \begin{minipage}[b]{0.19\columnwidth}\raggedright
Hit\_pathway\_number\strut
\end{minipage} & \begin{minipage}[b]{0.21\columnwidth}\raggedright
Enriched\_pathways\strut
\end{minipage}\tabularnewline
\midrule
\endfirsthead
\toprule
\begin{minipage}[b]{0.17\columnwidth}\raggedright
Herb\_pinyin\_name\strut
\end{minipage} & \begin{minipage}[b]{0.16\columnwidth}\raggedright
Ingredient.name\strut
\end{minipage} & \begin{minipage}[b]{0.12\columnwidth}\raggedright
Target.name\strut
\end{minipage} & \begin{minipage}[b]{0.19\columnwidth}\raggedright
Hit\_pathway\_number\strut
\end{minipage} & \begin{minipage}[b]{0.21\columnwidth}\raggedright
Enriched\_pathways\strut
\end{minipage}\tabularnewline
\midrule
\endhead
\begin{minipage}[t]{0.17\columnwidth}\raggedright
Huajiao\strut
\end{minipage} & \begin{minipage}[t]{0.16\columnwidth}\raggedright
quercetin\strut
\end{minipage} & \begin{minipage}[t]{0.12\columnwidth}\raggedright
TP53\strut
\end{minipage} & \begin{minipage}[t]{0.19\columnwidth}\raggedright
9\strut
\end{minipage} & \begin{minipage}[t]{0.21\columnwidth}\raggedright
Bladder cancer; B\ldots{}\strut
\end{minipage}\tabularnewline
\begin{minipage}[t]{0.17\columnwidth}\raggedright
Huangbo\strut
\end{minipage} & \begin{minipage}[t]{0.16\columnwidth}\raggedright
quercetin\strut
\end{minipage} & \begin{minipage}[t]{0.12\columnwidth}\raggedright
TP53\strut
\end{minipage} & \begin{minipage}[t]{0.19\columnwidth}\raggedright
9\strut
\end{minipage} & \begin{minipage}[t]{0.21\columnwidth}\raggedright
Bladder cancer; B\ldots{}\strut
\end{minipage}\tabularnewline
\begin{minipage}[t]{0.17\columnwidth}\raggedright
Huanglian\strut
\end{minipage} & \begin{minipage}[t]{0.16\columnwidth}\raggedright
quercetin\strut
\end{minipage} & \begin{minipage}[t]{0.12\columnwidth}\raggedright
TP53\strut
\end{minipage} & \begin{minipage}[t]{0.19\columnwidth}\raggedright
9\strut
\end{minipage} & \begin{minipage}[t]{0.21\columnwidth}\raggedright
Bladder cancer; B\ldots{}\strut
\end{minipage}\tabularnewline
\begin{minipage}[t]{0.17\columnwidth}\raggedright
Wumei\strut
\end{minipage} & \begin{minipage}[t]{0.16\columnwidth}\raggedright
quercetin\strut
\end{minipage} & \begin{minipage}[t]{0.12\columnwidth}\raggedright
TP53\strut
\end{minipage} & \begin{minipage}[t]{0.19\columnwidth}\raggedright
9\strut
\end{minipage} & \begin{minipage}[t]{0.21\columnwidth}\raggedright
Bladder cancer; B\ldots{}\strut
\end{minipage}\tabularnewline
\begin{minipage}[t]{0.17\columnwidth}\raggedright
Huajiao\strut
\end{minipage} & \begin{minipage}[t]{0.16\columnwidth}\raggedright
quercetin\strut
\end{minipage} & \begin{minipage}[t]{0.12\columnwidth}\raggedright
EGF\strut
\end{minipage} & \begin{minipage}[t]{0.19\columnwidth}\raggedright
6\strut
\end{minipage} & \begin{minipage}[t]{0.21\columnwidth}\raggedright
Bladder cancer; B\ldots{}\strut
\end{minipage}\tabularnewline
\begin{minipage}[t]{0.17\columnwidth}\raggedright
Huangbo\strut
\end{minipage} & \begin{minipage}[t]{0.16\columnwidth}\raggedright
quercetin\strut
\end{minipage} & \begin{minipage}[t]{0.12\columnwidth}\raggedright
EGF\strut
\end{minipage} & \begin{minipage}[t]{0.19\columnwidth}\raggedright
6\strut
\end{minipage} & \begin{minipage}[t]{0.21\columnwidth}\raggedright
Bladder cancer; B\ldots{}\strut
\end{minipage}\tabularnewline
\begin{minipage}[t]{0.17\columnwidth}\raggedright
Huanglian\strut
\end{minipage} & \begin{minipage}[t]{0.16\columnwidth}\raggedright
quercetin\strut
\end{minipage} & \begin{minipage}[t]{0.12\columnwidth}\raggedright
EGF\strut
\end{minipage} & \begin{minipage}[t]{0.19\columnwidth}\raggedright
6\strut
\end{minipage} & \begin{minipage}[t]{0.21\columnwidth}\raggedright
Bladder cancer; B\ldots{}\strut
\end{minipage}\tabularnewline
\begin{minipage}[t]{0.17\columnwidth}\raggedright
Wumei\strut
\end{minipage} & \begin{minipage}[t]{0.16\columnwidth}\raggedright
quercetin\strut
\end{minipage} & \begin{minipage}[t]{0.12\columnwidth}\raggedright
EGF\strut
\end{minipage} & \begin{minipage}[t]{0.19\columnwidth}\raggedright
6\strut
\end{minipage} & \begin{minipage}[t]{0.21\columnwidth}\raggedright
Bladder cancer; B\ldots{}\strut
\end{minipage}\tabularnewline
\begin{minipage}[t]{0.17\columnwidth}\raggedright
Huajiao\strut
\end{minipage} & \begin{minipage}[t]{0.16\columnwidth}\raggedright
quercetin\strut
\end{minipage} & \begin{minipage}[t]{0.12\columnwidth}\raggedright
HMOX1\strut
\end{minipage} & \begin{minipage}[t]{0.19\columnwidth}\raggedright
4\strut
\end{minipage} & \begin{minipage}[t]{0.21\columnwidth}\raggedright
Chemical carcinog\ldots{}\strut
\end{minipage}\tabularnewline
\begin{minipage}[t]{0.17\columnwidth}\raggedright
Huajiao\strut
\end{minipage} & \begin{minipage}[t]{0.16\columnwidth}\raggedright
quercetin\strut
\end{minipage} & \begin{minipage}[t]{0.12\columnwidth}\raggedright
NFE2L2\strut
\end{minipage} & \begin{minipage}[t]{0.19\columnwidth}\raggedright
4\strut
\end{minipage} & \begin{minipage}[t]{0.21\columnwidth}\raggedright
Chemical carcinog\ldots{}\strut
\end{minipage}\tabularnewline
\begin{minipage}[t]{0.17\columnwidth}\raggedright
Huangbo\strut
\end{minipage} & \begin{minipage}[t]{0.16\columnwidth}\raggedright
quercetin\strut
\end{minipage} & \begin{minipage}[t]{0.12\columnwidth}\raggedright
HMOX1\strut
\end{minipage} & \begin{minipage}[t]{0.19\columnwidth}\raggedright
4\strut
\end{minipage} & \begin{minipage}[t]{0.21\columnwidth}\raggedright
Chemical carcinog\ldots{}\strut
\end{minipage}\tabularnewline
\begin{minipage}[t]{0.17\columnwidth}\raggedright
Huangbo\strut
\end{minipage} & \begin{minipage}[t]{0.16\columnwidth}\raggedright
quercetin\strut
\end{minipage} & \begin{minipage}[t]{0.12\columnwidth}\raggedright
NFE2L2\strut
\end{minipage} & \begin{minipage}[t]{0.19\columnwidth}\raggedright
4\strut
\end{minipage} & \begin{minipage}[t]{0.21\columnwidth}\raggedright
Chemical carcinog\ldots{}\strut
\end{minipage}\tabularnewline
\begin{minipage}[t]{0.17\columnwidth}\raggedright
Huanglian\strut
\end{minipage} & \begin{minipage}[t]{0.16\columnwidth}\raggedright
quercetin\strut
\end{minipage} & \begin{minipage}[t]{0.12\columnwidth}\raggedright
HMOX1\strut
\end{minipage} & \begin{minipage}[t]{0.19\columnwidth}\raggedright
4\strut
\end{minipage} & \begin{minipage}[t]{0.21\columnwidth}\raggedright
Chemical carcinog\ldots{}\strut
\end{minipage}\tabularnewline
\begin{minipage}[t]{0.17\columnwidth}\raggedright
Huanglian\strut
\end{minipage} & \begin{minipage}[t]{0.16\columnwidth}\raggedright
quercetin\strut
\end{minipage} & \begin{minipage}[t]{0.12\columnwidth}\raggedright
NFE2L2\strut
\end{minipage} & \begin{minipage}[t]{0.19\columnwidth}\raggedright
4\strut
\end{minipage} & \begin{minipage}[t]{0.21\columnwidth}\raggedright
Chemical carcinog\ldots{}\strut
\end{minipage}\tabularnewline
\begin{minipage}[t]{0.17\columnwidth}\raggedright
Renshen\strut
\end{minipage} & \begin{minipage}[t]{0.16\columnwidth}\raggedright
kaempferol\strut
\end{minipage} & \begin{minipage}[t]{0.12\columnwidth}\raggedright
HMOX1\strut
\end{minipage} & \begin{minipage}[t]{0.19\columnwidth}\raggedright
4\strut
\end{minipage} & \begin{minipage}[t]{0.21\columnwidth}\raggedright
Chemical carcinog\ldots{}\strut
\end{minipage}\tabularnewline
\begin{minipage}[t]{0.17\columnwidth}\raggedright
\ldots{}\strut
\end{minipage} & \begin{minipage}[t]{0.16\columnwidth}\raggedright
\ldots{}\strut
\end{minipage} & \begin{minipage}[t]{0.12\columnwidth}\raggedright
\ldots{}\strut
\end{minipage} & \begin{minipage}[t]{0.19\columnwidth}\raggedright
\ldots{}\strut
\end{minipage} & \begin{minipage}[t]{0.21\columnwidth}\raggedright
\ldots{}\strut
\end{minipage}\tabularnewline
\bottomrule
\end{longtable}

\hypertarget{kaempferol-ux548c-hmox1-ux5bf9ux63a5ux53efux89c6ux5316}{%
\subsubsection{kaempferol 和 HMOX1 对接可视化}\label{kaempferol-ux548c-hmox1-ux5bf9ux63a5ux53efux89c6ux5316}}

Figure \ref{fig:Docking-5280863-into-6eha} (下方图) 为图Docking 5280863 into 6eha概览。

\textbf{(对应文件为 \texttt{Figure+Table/5280863\_into\_6eha.png})}

\def\@captype{figure}
\begin{center}
\includegraphics[width = 0.9\linewidth]{vina_space/5280863_into_6eha/5280863_into_6eha.png}
\caption{Docking 5280863 into 6eha}\label{fig:Docking-5280863-into-6eha}
\end{center}

Figure \ref{fig:Docking-5280863-into-6eha-detail} (下方图) 为图Docking 5280863 into 6eha detail概览。

\textbf{(对应文件为 \texttt{Figure+Table/detail\_5280863\_into\_6eha.png})}

\def\@captype{figure}
\begin{center}
\includegraphics[width = 0.9\linewidth]{vina_space/5280863_into_6eha/detail_5280863_into_6eha.png}
\caption{Docking 5280863 into 6eha detail}\label{fig:Docking-5280863-into-6eha-detail}
\end{center}

\hypertarget{bibliography}{%
\section*{Reference}\label{bibliography}}
\addcontentsline{toc}{section}{Reference}

\hypertarget{refs}{}
\begin{cslreferences}
\leavevmode\hypertarget{ref-BatmanTcm20Kong2024}{}%
1. Kong, X. \emph{et al.} BATMAN-tcm 2.0: An enhanced integrative database for known and predicted interactions between traditional chinese medicine ingredients and target proteins. \emph{Nucleic acids research} \textbf{52}, D1110--D1120 (2024).

\leavevmode\hypertarget{ref-ComparativeToxDavis2023}{}%
2. Davis, A. P. \emph{et al.} Comparative toxicogenomics database (ctd): Update 2023. \emph{Nucleic acids research} \textbf{51}, D1257--D1262 (2023).

\leavevmode\hypertarget{ref-ClusterprofilerWuTi2021}{}%
3. Wu, T. \emph{et al.} ClusterProfiler 4.0: A universal enrichment tool for interpreting omics data. \emph{The Innovation} \textbf{2}, (2021).

\leavevmode\hypertarget{ref-TheDisgenetKnPinero2019}{}%
4. Piñero, J. \emph{et al.} The disgenet knowledge platform for disease genomics: 2019 update. \emph{Nucleic Acids Research} (2019) doi:\href{https://doi.org/10.1093/nar/gkz1021}{10.1093/nar/gkz1021}.

\leavevmode\hypertarget{ref-TheGenecardsSStelze2016}{}%
5. Stelzer, G. \emph{et al.} The genecards suite: From gene data mining to disease genome sequence analyses. \emph{Current protocols in bioinformatics} \textbf{54}, 1.30.1--1.30.33 (2016).

\leavevmode\hypertarget{ref-PharmgkbAWorBarbar2018}{}%
6. Barbarino, J. M., Whirl-Carrillo, M., Altman, R. B. \& Klein, T. E. PharmGKB: A worldwide resource for pharmacogenomic information. \emph{Wiley interdisciplinary reviews. Systems biology and medicine} \textbf{10}, (2018).

\leavevmode\hypertarget{ref-TcmspADatabaRuJi2014}{}%
7. Ru, J. \emph{et al.} TCMSP: A database of systems pharmacology for drug discovery from herbal medicines. \emph{Journal of cheminformatics} \textbf{6}, (2014).

\leavevmode\hypertarget{ref-AutodockVina1Eberha2021}{}%
8. Eberhardt, J., Santos-Martins, D., Tillack, A. F. \& Forli, S. AutoDock vina 1.2.0: New docking methods, expanded force field, and python bindings. \emph{Journal of Chemical Information and Modeling} \textbf{61}, 3891--3898 (2021).

\leavevmode\hypertarget{ref-AutogridfrImpZhang2019}{}%
9. Zhang, Y., Forli, S., Omelchenko, A. \& Sanner, M. F. AutoGridFR: Improvements on autodock affinity maps and associated software tools. \emph{Journal of computational chemistry} \textbf{40}, 2882--2886 (2019).

\leavevmode\hypertarget{ref-AutodockCrankpZhang2019}{}%
10. Zhang, Y. \& Sanner, M. F. AutoDock crankpep: Combining folding and docking to predict protein-peptide complexes. \emph{Bioinformatics (Oxford, England)} \textbf{35}, 5121--5127 (2019).

\leavevmode\hypertarget{ref-AutositeAnAuRavind2016}{}%
11. Ravindranath, P. A. \& Sanner, M. F. AutoSite: An automated approach for pseudo-ligands prediction-from ligand-binding sites identification to predicting key ligand atoms. \emph{Bioinformatics (Oxford, England)} \textbf{32}, 3142--3149 (2016).

\leavevmode\hypertarget{ref-AutodockfrAdvRavind2015}{}%
12. Ravindranath, P. A., Forli, S., Goodsell, D. S., Olson, A. J. \& Sanner, M. F. AutoDockFR: Advances in protein-ligand docking with explicitly specified binding site flexibility. \emph{PLoS computational biology} \textbf{11}, (2015).

\leavevmode\hypertarget{ref-PathviewAnRLuoW2013}{}%
13. Luo, W. \& Brouwer, C. Pathview: An r/bioconductor package for pathway-based data integration and visualization. \emph{Bioinformatics (Oxford, England)} \textbf{29}, 1830--1831 (2013).
\end{cslreferences}

\end{document}
