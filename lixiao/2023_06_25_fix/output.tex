% Options for packages loaded elsewhere
\PassOptionsToPackage{unicode}{hyperref}
\PassOptionsToPackage{hyphens}{url}
%
\documentclass[
]{article}
\usepackage{lmodern}
\usepackage{amssymb,amsmath}
\usepackage{ifxetex,ifluatex}
\ifnum 0\ifxetex 1\fi\ifluatex 1\fi=0 % if pdftex
  \usepackage[T1]{fontenc}
  \usepackage[utf8]{inputenc}
  \usepackage{textcomp} % provide euro and other symbols
\else % if luatex or xetex
  \usepackage{unicode-math}
  \defaultfontfeatures{Scale=MatchLowercase}
  \defaultfontfeatures[\rmfamily]{Ligatures=TeX,Scale=1}
\fi
% Use upquote if available, for straight quotes in verbatim environments
\IfFileExists{upquote.sty}{\usepackage{upquote}}{}
\IfFileExists{microtype.sty}{% use microtype if available
  \usepackage[]{microtype}
  \UseMicrotypeSet[protrusion]{basicmath} % disable protrusion for tt fonts
}{}
\makeatletter
\@ifundefined{KOMAClassName}{% if non-KOMA class
  \IfFileExists{parskip.sty}{%
    \usepackage{parskip}
  }{% else
    \setlength{\parindent}{0pt}
    \setlength{\parskip}{6pt plus 2pt minus 1pt}}
}{% if KOMA class
  \KOMAoptions{parskip=half}}
\makeatother
\usepackage{xcolor}
\IfFileExists{xurl.sty}{\usepackage{xurl}}{} % add URL line breaks if available
\IfFileExists{bookmark.sty}{\usepackage{bookmark}}{\usepackage{hyperref}}
\hypersetup{
  pdftitle={Report of Analysis},
  pdfauthor={Huang LiChuang of Wie-Biotech},
  hidelinks,
  pdfcreator={LaTeX via pandoc}}
\urlstyle{same} % disable monospaced font for URLs
\usepackage[margin=1in]{geometry}
\usepackage{longtable,booktabs}
% Correct order of tables after \paragraph or \subparagraph
\usepackage{etoolbox}
\makeatletter
\patchcmd\longtable{\par}{\if@noskipsec\mbox{}\fi\par}{}{}
\makeatother
% Allow footnotes in longtable head/foot
\IfFileExists{footnotehyper.sty}{\usepackage{footnotehyper}}{\usepackage{footnote}}
\makesavenoteenv{longtable}
\usepackage{graphicx}
\makeatletter
\def\maxwidth{\ifdim\Gin@nat@width>\linewidth\linewidth\else\Gin@nat@width\fi}
\def\maxheight{\ifdim\Gin@nat@height>\textheight\textheight\else\Gin@nat@height\fi}
\makeatother
% Scale images if necessary, so that they will not overflow the page
% margins by default, and it is still possible to overwrite the defaults
% using explicit options in \includegraphics[width, height, ...]{}
\setkeys{Gin}{width=\maxwidth,height=\maxheight,keepaspectratio}
% Set default figure placement to htbp
\makeatletter
\def\fps@figure{htbp}
\makeatother
\setlength{\emergencystretch}{3em} % prevent overfull lines
\providecommand{\tightlist}{%
  \setlength{\itemsep}{0pt}\setlength{\parskip}{0pt}}
\setcounter{secnumdepth}{5}
\usepackage{caption} \captionsetup{font={footnotesize},width=6in} \renewcommand{\dblfloatpagefraction}{.9} \makeatletter \renewenvironment{figure} {\def\@captype{figure}} \makeatletter \definecolor{shadecolor}{RGB}{242,242,242} \usepackage{xeCJK} \usepackage{setspace} \setstretch{1.3}
\newlength{\cslhangindent}
\setlength{\cslhangindent}{1.5em}
\newenvironment{cslreferences}%
  {}%
  {\par}

\title{Report of Analysis}
\author{Huang LiChuang of Wie-Biotech}
\date{}

\begin{document}
\maketitle

{
\setcounter{tocdepth}{3}
\tableofcontents
}
\hypertarget{ux7b2cux4e00ux90e8ux5206}{%
\section{第一部分}\label{ux7b2cux4e00ux90e8ux5206}}

\hypertarget{etcm-ux4e2dux836fux4e39ux53c2ux7684ux5316ux5408ux7269ux4ee5ux53caux9776ux70b9ux57faux56e0}{%
\subsection{ETCM 中药丹参的化合物以及靶点基因}\label{etcm-ux4e2dux836fux4e39ux53c2ux7684ux5316ux5408ux7269ux4ee5ux53caux9776ux70b9ux57faux56e0}}

\hypertarget{ux83b7ux53d6-etcm-ux7f51ux7ad9ux6570ux636e}{%
\subsubsection{获取 ETCM 网站数据}\label{ux83b7ux53d6-etcm-ux7f51ux7ad9ux6570ux636e}}

通过编写R 函数以快速获取 ETCM 网站的中药和对应靶点数据。

丹参的96种化合物和相关靶基因概览。其中,靶点基因(非重复)共216 个。
\textbf{(对应文件为 \texttt{./96\_components.txt}, \texttt{./components\_and\_target\_genes.csv})}

\begin{verbatim}
## # A tibble: 532 x 3
##    components               genes  links                                                   
##    <chr>                    <chr>  <chr>                                                   
##  1 Sitosterol,Î’-Sitosterol AKR1C1 /ETCM/index.php/Home/Index/jyjb_details.html?gene=AKR1C1
##  2 Sitosterol,Î’-Sitosterol AKR1C2 /ETCM/index.php/Home/Index/jyjb_details.html?gene=AKR1C2
##  3 Sitosterol,Î’-Sitosterol AR     /ETCM/index.php/Home/Index/jyjb_details.html?gene=AR    
##  4 Sitosterol,Î’-Sitosterol CLEC4E /ETCM/index.php/Home/Index/jyjb_details.html?gene=CLEC4E
##  5 Sitosterol,Î’-Sitosterol ESR1   /ETCM/index.php/Home/Index/jyjb_details.html?gene=ESR1  
##  6 Sitosterol,Î’-Sitosterol ESR2   /ETCM/index.php/Home/Index/jyjb_details.html?gene=ESR2  
##  7 Sitosterol,Î’-Sitosterol GABRA1 /ETCM/index.php/Home/Index/jyjb_details.html?gene=GABRA1
##  8 Sitosterol,Î’-Sitosterol GABRA2 /ETCM/index.php/Home/Index/jyjb_details.html?gene=GABRA2
##  9 Sitosterol,Î’-Sitosterol GABRA3 /ETCM/index.php/Home/Index/jyjb_details.html?gene=GABRA3
## 10 Sitosterol,Î’-Sitosterol GABRA4 /ETCM/index.php/Home/Index/jyjb_details.html?gene=GABRA4
## # i 522 more rows
\end{verbatim}

\hypertarget{ux83b7ux53d6-herb-ux7f51ux7ad9ux6570ux636e}{%
\subsubsection{获取 HERB 网站数据}\label{ux83b7ux53d6-herb-ux7f51ux7ad9ux6570ux636e}}

由于 ETCM 缺少相当一部分化合物的靶点数据,因而使用 HERB 数据库补充。
(对于丹参,HERB 包含更多的化合物)
\textbf{(对应文件为 \texttt{./HERB\_compounds\_target.tsv})}

\hypertarget{ux83b7ux53d6-pubchem-ux6570ux636eux5e93ux5173ux4e8eux5316ux5408ux7269ux7684ux522bux540d}{%
\subsubsection{获取 PubChem 数据库关于化合物的别名}\label{ux83b7ux53d6-pubchem-ux6570ux636eux5e93ux5173ux4e8eux5316ux5408ux7269ux7684ux522bux540d}}

ETCM 数据库是个封闭的网站,不包含和任何其它数据库相同的ID信息。
为了以 HERB 数据库的靶点数据补充 ETCM 的化合物靶点数据,
根据 PubChem CID (HERB 数据库提供)搜索 PubChem 获得化合物的别名。
\textbf{(对应文件为 \texttt{./synos.tsv})}

\hypertarget{ux4ee5-herb-ux7684ux6570ux636eux8865ux5145-etcm-ux7684ux6570ux636e}{%
\subsubsection{以 HERB 的数据补充 ETCM 的数据}\label{ux4ee5-herb-ux7684ux6570ux636eux8865ux5145-etcm-ux7684ux6570ux636e}}

以下是可以在 HERB 找到靶点基因的化合物,但是在 ETCM 找不到靶点基因的化合物:

\begin{verbatim}
##          cid                syno
##  1: 11425923 Dihydrotanshinone I
##  2: 11600642           Danshensu
##  3:    68081      Isoimperatorin
##  4: 44425165 Neocryptotanshinone
##  5:   160254    Cryptotanshinone
##  6: 11629084       Danshensuan B
##  7:  3082765    Dehydromiltirone
##  8:   626608 Isocryptotanshinone
##  9:   622085     Przewaquinone B
## 10:   126072       Tanshindiol C
## 11:   389885         Salvilenone
## 12:  5321622         Tanshinol A
## 13:   114917        Tanshinone I
\end{verbatim}

尽管如此,由于 ETCM 缺乏其他数据库的索引,还是有一部分的化合物不知道来源,所以无法从其他数据库得到靶点基因的补充:

\begin{verbatim}
##  [1] "Methylene Tanshinquinone"           "Dihydroisotanshinone I"             "Tanshinlactone"                    
##  [4] "Salvianolic Acid B"                 "Danshexinkum D"                     "Danshexinkum A"                    
##  [7] "Methyl Tanshinonate"                "Ethyl Lithospermate"                "Salvinone"                         
## [10] "Isotanshinoneii B"                  "Prgewaquinone A"                    "Danshenol A"                       
## [13] "Danshenspiroketallactone"           "Danshenxinkun A"                    "Danshenxinkun B"                   
## [16] "Danshenxinkun C"                    "Danshenxinkun D"                    "Δ1-Dehydrotanshinone"             
## [19] "Dihydroisotanshinone I"             "1,2-Dihydrotanshiquinone"           "Epidanshenspiroketallactone"       
## [22] "3-Β-Hydroxymethylenetanshiquinone" "3Α-Hydroxytanshinone Iia"          "3Β-Hydroxytanshinone Iia"         
## [25] "Isotanshinone I"                    "Isotanshinone Iia"                  "Lithospermate B"                   
## [28] "Magnesium Lithospermate B"          "Methyl Tanshinonate"                "Monomethyl Lithospermate"          
## [31] "Nortanshinone"                      "Salvianolic Acid B"                 "Salvianolic Acid C"                
## [34] "Salvianolic Acid G"                 "Salvinone"                          "Tanshindiol A"                     
## [37] "Tanshindiol B"                      "Tanshinlactone"                     "Tanshinol B"                       
## [40] "Tanshinone Iia"                     "Tanshinone Iib"                     "Tanshinone Vi"
\end{verbatim}

\hypertarget{ux603bux7ed3}{%
\subsubsection{总结}\label{ux603bux7ed3}}

HERB 数据库包含更多的化合物和靶点信息,所以以下分析以 HERB 数据库为主。
HERB 记录的丹参的化合物有330 个,
能找到靶点基因信息的化合物有187个。
\textbf{(对应文件为 \texttt{./HERB\_compounds\_of\_danshen.xlsx}, \texttt{./HERB\_targets\_of\_compounds.xlsx})}

\hypertarget{ux7b2cux4e8cux90e8ux5206}{%
\section{第二部分}\label{ux7b2cux4e8cux90e8ux5206}}

\hypertarget{ux5728genecardsux7f51ux7ad9ux4e0aux68c0ux7d22ux80c3ux764cux76f8ux5173ux7684ux57faux56e0}{%
\subsection{在genecards网站上检索胃癌相关的基因}\label{ux5728genecardsux7f51ux7ad9ux4e0aux68c0ux7d22ux80c3ux764cux76f8ux5173ux7684ux57faux56e0}}

在网站Genecards检索胃癌,获取相关数据后,根据Relevance score进行筛选(\textgreater{} 5)。

所有胃癌相关基因概览,基因数量为4475
\textbf{(对应文件为 \texttt{./all\_gastric\_Cancer\_related\_genes.csv})}

\begin{verbatim}
## # A tibble: 4,475 x 8
##    Gene.Symbol Description                            Category       Uniprot.ID Gifts GC.Id       Relevance.score GeneCards.Link  
##    <chr>       <chr>                                  <chr>          <chr>      <int> <chr>                 <dbl> <chr>           
##  1 CDH1        Cadherin 1                             Protein Coding P12830        56 GC16P068737            379. https://www.gen~
##  2 BRCA2       BRCA2 DNA Repair Associated            Protein Coding P51587        54 GC13P032315            303. https://www.gen~
##  3 BRCA1       BRCA1 DNA Repair Associated            Protein Coding P38398        57 GC17M043044            291. https://www.gen~
##  4 TP53        Tumor Protein P53                      Protein Coding P04637        60 GC17M007661            233. https://www.gen~
##  5 APC         APC Regulator Of WNT Signaling Pathway Protein Coding P25054        56 GC05P112707            218. https://www.gen~
##  6 CHEK2       Checkpoint Kinase 2                    Protein Coding O96017        61 GC22M028687            204. https://www.gen~
##  7 PALB2       Partner And Localizer Of BRCA2         Protein Coding Q86YC2        51 GC16M023603            204. https://www.gen~
##  8 ATM         ATM Serine/Threonine Kinase            Protein Coding Q13315        60 GC11P108222            191. https://www.gen~
##  9 MLH1        MutL Homolog 1                         Protein Coding P40692        56 GC03P036993            182. https://www.gen~
## 10 BRIP1       BRCA1 Interacting Helicase 1           Protein Coding Q9BX63        56 GC17M061679            180. https://www.gen~
## # i 4,465 more rows
\end{verbatim}

根据丹参靶点基因过滤数据集,即,将筛选过胃癌基因数据和丹参数据根据基因合并。

韦恩图见Figure \ref{fig:fig2},说明:韦恩图分三个区域,左侧加上中间区域对应187个化合物所有靶点基因;
中间区域加上右侧区域对应所有胃癌相关基因。
\textbf{(对应文件为 \texttt{./figs/venn\_plot.pdf})}

\begin{figure}
\centering
\includegraphics{figs/venn_plot.pdf}
\caption{\label{fig:fig2}靶点基因和胃癌相关基因交集韦恩图}
\end{figure}

187个化合物和胃癌相关基因的交集数据概览(包含交集基因以及对应化合物),化合物和靶点基因一一对应,
化合物共 187 个,非重复基因273个,与上述韦恩图一致。
\textbf{(对应文件为 \texttt{./gastric\_Cancer\_related\_genes\_Intersect\_with\_targetGenes\_components.csv})}

\begin{verbatim}
## # A tibble: 2,083 x 9
##    components                               Gene.Symbol Description Category Uniprot.ID Gifts GC.Id Relevance.score GeneCards.Link
##    <chr>                                    <chr>       <chr>       <chr>    <chr>      <int> <chr>           <dbl> <chr>         
##  1 (-)-Epicedrol                            ACHE        Acetylchol~ Protein~ P22303        53 GC07~            8.07 https://www.g~
##  2 (-)-Epicedrol                            DPP4        Dipeptidyl~ Protein~ P27487        56 GC02~           16.6  https://www.g~
##  3 (-)-beta-Phellandrene                    ACHE        Acetylchol~ Protein~ P22303        53 GC07~            8.07 https://www.g~
##  4 (-)-beta-Phellandrene                    DPP4        Dipeptidyl~ Protein~ P27487        56 GC02~           16.6  https://www.g~
##  5 (-)-beta-Phellandrene                    NR3C1       Nuclear Re~ Protein~ P04150        56 GC05~           14.9  https://www.g~
##  6 (-)-beta-Phellandrene                    PRSS1       Serine Pro~ Protein~ P07477        52 GC07~           20.9  https://www.g~
##  7 (1R,4R,5S)-1-isopropyl-4-methyl-4-bicyc~ ACHE        Acetylchol~ Protein~ P22303        53 GC07~            8.07 https://www.g~
##  8 (1R,4R,5S)-1-isopropyl-4-methyl-4-bicyc~ DPP4        Dipeptidyl~ Protein~ P27487        56 GC02~           16.6  https://www.g~
##  9 (1R,4S,4aR,8aR)-4-isopropyl-1,6-dimethy~ ACHE        Acetylchol~ Protein~ P22303        53 GC07~            8.07 https://www.g~
## 10 (1R,4S,4aR,8aR)-4-isopropyl-1,6-dimethy~ AR          Androgen R~ Protein~ P10275        58 GC0X~           64.6  https://www.g~
## # i 2,073 more rows
\end{verbatim}

\hypertarget{ux7b2cux4e09ux90e8ux5206}{%
\section{第三部分}\label{ux7b2cux4e09ux90e8ux5206}}

\hypertarget{ux4f7fux7528cellminerux6570ux636eux5e93ux7684nci-60ux6570ux636eux96c6}{%
\subsection{使用CellMiner数据库的NCI-60数据集}\label{ux4f7fux7528cellminerux6570ux636eux5e93ux7684nci-60ux6570ux636eux96c6}}

下载并预处理 NCI-60 的数据以备药物敏感性分析。

Cisplatin 活性数据,包含60个癌细胞的活性IC50 Z-score :

\begin{verbatim}
## # A tibble: 1 x 61
##   `Drug name` `BR:MCF7` `BR:MDA-MB-231` `BR:HS 578T` `BR:BT-549` `BR:T-47D` `CNS:SF-268` `CNS:SF-295` `CNS:SF-539` `CNS:SNB-19`
##   <chr>       <chr>     <chr>           <chr>        <chr>       <chr>      <chr>        <chr>        <chr>        <chr>       
## 1 Cisplatin   0.26      -1.8            -0.56        -0.29       -1.32      1.42         1.15         0.63         -0.2        
## # i 51 more variables: `CNS:SNB-75` <chr>, `CNS:U251` <chr>, `CO:COLO 205` <chr>, `CO:HCC-2998` <chr>, `CO:HCT-116` <chr>,
## #   `CO:HCT-15` <chr>, `CO:HT29` <chr>, `CO:KM12` <chr>, `CO:SW-620` <chr>, `LE:CCRF-CEM` <chr>, `LE:HL-60(TB)` <chr>,
## #   `LE:K-562` <chr>, `LE:MOLT-4` <chr>, `LE:RPMI-8226` <chr>, `LE:SR` <chr>, `ME:LOX IMVI` <chr>, `ME:MALME-3M` <chr>,
## #   `ME:M14` <chr>, `ME:SK-MEL-2` <chr>, `ME:SK-MEL-28` <chr>, `ME:SK-MEL-5` <chr>, `ME:UACC-257` <chr>, `ME:UACC-62` <chr>,
## #   `ME:MDA-MB-435` <chr>, `ME:MDA-N` <chr>, `LC:A549/ATCC` <chr>, `LC:EKVX` <chr>, `LC:HOP-62` <chr>, `LC:HOP-92` <chr>,
## #   `LC:NCI-H226` <chr>, `LC:NCI-H23` <chr>, `LC:NCI-H322M` <chr>, `LC:NCI-H460` <chr>, `LC:NCI-H522` <chr>, `OV:IGROV1` <chr>,
## #   `OV:OVCAR-3` <chr>, `OV:OVCAR-4` <chr>, `OV:OVCAR-5` <chr>, `OV:OVCAR-8` <chr>, `OV:SK-OV-3` <chr>, ...
\end{verbatim}

NCI-60 表达数据,包含60个癌细胞的基因表达数据(FPKM):

\begin{verbatim}
## # A tibble: 270 x 67
##    `Gene name d` `Entrez gene id e` `Chromosome f` `Start f`   `End f` `Cytoband f` `BR:MCF7` `BR:MDA-MB-231` `BR:HS 578T`
##    <chr>                      <dbl> <chr>              <dbl>     <dbl> <chr>            <dbl>           <dbl>        <dbl>
##  1 PARK7                      11315 1                8021713   8045342 1p36.23          5.42            6.95         5.44 
##  2 PIK3CD                      5293 1                9711789   9789172 1p36.2           0.132           0.507        1.55 
##  3 MTOR                        2475 1               11166587  11322608 1p36.2           2.54            1.76         1.78 
##  4 CTRC                       11330 1               15764937  15773153 1p36.21          0               0            0    
##  5 CASP9                        842 1               15817895  15851285 1p36.21          0.669           0.377        1.05 
##  6 LCK                         3932 1               32716839  32751768 1p34.3           0               0            0.183
##  7 AKR1A1                     10327 1               46016454  46035723 1p33-p32         3.38            1.83         3.50 
##  8 JUN                         3725 1               59246462  59249785 1p32-p31         0.76            3.82         3.38 
##  9 GCLM                        2730 1               94350755  94375154 1p22.1           1.02            4.24         1.62 
## 10 VCAM1                       7412 1              101185195 101204601 1p32-p31         0               0            0.072
## # i 260 more rows
## # i 58 more variables: `BR:BT-549` <dbl>, `BR:T-47D` <dbl>, `CNS:SF-268` <dbl>, `CNS:SF-295` <dbl>, `CNS:SF-539` <dbl>,
## #   `CNS:SNB-19` <dbl>, `CNS:SNB-75` <dbl>, `CNS:U251` <dbl>, `CO:COLO 205` <dbl>, `CO:HCC-2998` <dbl>, `CO:HCT-116` <dbl>,
## #   `CO:HCT-15` <dbl>, `CO:HT29` <dbl>, `CO:KM12` <dbl>, `CO:SW-620` <dbl>, `LE:CCRF-CEM` <dbl>, `LE:HL-60(TB)` <dbl>,
## #   `LE:K-562` <dbl>, `LE:MOLT-4` <dbl>, `LE:RPMI-8226` <dbl>, `LE:SR` <dbl>, `ME:LOX IMVI` <dbl>, `ME:MALME-3M` <dbl>,
## #   `ME:M14` <dbl>, `ME:SK-MEL-2` <dbl>, `ME:SK-MEL-28` <dbl>, `ME:SK-MEL-5` <dbl>, `ME:UACC-257` <dbl>, `ME:UACC-62` <dbl>,
## #   `ME:MDA-MB-435` <dbl>, `ME:MDA-N` <dbl>, `LC:A549/ATCC` <dbl>, `LC:EKVX` <dbl>, `LC:HOP-62` <dbl>, `LC:HOP-92` <dbl>, ...
\end{verbatim}

\hypertarget{ux836fux7269ux654fux611fux6027ux5206ux6790}{%
\subsection{药物敏感性分析}\label{ux836fux7269ux654fux611fux6027ux5206ux6790}}

将药物活性数据和基因表达数据关联分析(Pearson)。

其中有显著性意义的有 39 个基因(p \textless{} 0.05),可视化见Figure \ref{fig:fig3}
\textbf{(对应文件为 \texttt{figs/pearsonTest.pdf})}。
这意味着,与顺铂协作的靶点基因有 39 个。

\begin{figure}
\centering
\includegraphics{figs/pearsonTest.pdf}
\caption{\label{fig:fig3}关联性分析回归曲线图}
\end{figure}

将这 39 个显著基因的分析数据与 187 个化合物及其靶点基因数据合并,得到作用于显著基因的化合物数据。

关联性分析(Pearson)结果概览(已包含基因和对应化合物数据),
\textbf{(对应文件为 \texttt{./pearsonTest\_allResults.csv}, \texttt{./pearsonTest\_results\_with\_components.csv})}
其中,协同顺铂靶基因的化合物共有73 个。

\begin{verbatim}
## # A tibble: 117 x 11
##    name      cor p.value components                Description      Category Uniprot.ID Gifts GC.Id Relevance.score GeneCards.Link
##    <chr>   <dbl>   <dbl> <chr>                     <chr>            <chr>    <chr>      <int> <chr>           <dbl> <chr>         
##  1 AHR    -0.283  0.0287 kaempferol                Aryl Hydrocarbo~ Protein~ P35869        53 GC07~            14.6 https://www.g~
##  2 BAX     0.267  0.0393 aloeemodin                BCL2 Associated~ Protein~ Q07812        57 GC19~            52.9 https://www.g~
##  3 BAX     0.267  0.0393 cryptotanshinone          BCL2 Associated~ Protein~ Q07812        57 GC19~            52.9 https://www.g~
##  4 BAX     0.267  0.0393 kaempferol                BCL2 Associated~ Protein~ Q07812        57 GC19~            52.9 https://www.g~
##  5 BAX     0.267  0.0393 rhein                     BCL2 Associated~ Protein~ Q07812        57 GC19~            52.9 https://www.g~
##  6 BAX     0.267  0.0393 aucubin                   BCL2 Associated~ Protein~ Q07812        57 GC19~            52.9 https://www.g~
##  7 BAX     0.267  0.0393 tanshinone i              BCL2 Associated~ Protein~ Q07812        57 GC19~            52.9 https://www.g~
##  8 BAX     0.267  0.0393 beta-sitosterol           BCL2 Associated~ Protein~ Q07812        57 GC19~            52.9 https://www.g~
##  9 BAX     0.267  0.0393 protocatechuic acid       BCL2 Associated~ Protein~ Q07812        57 GC19~            52.9 https://www.g~
## 10 BCL2L1 -0.256  0.0480 15,16-dihydrotanshinone i BCL2 Like 1      Protein~ Q07817        54 GC20~            31.4 https://www.g~
## # i 107 more rows
\end{verbatim}

\hypertarget{ux7b2cux56dbux90e8ux5206}{%
\section{第四部分}\label{ux7b2cux56dbux90e8ux5206}}

\hypertarget{ux4f7fux7528biomartux6ce8ux91caux9776ux70b9ux57faux56e0}{%
\subsection{使用BiomaRt注释靶点基因}\label{ux4f7fux7528biomartux6ce8ux91caux9776ux70b9ux57faux56e0}}

使用R 包\texttt{biomaRt}获取靶点基因的Entrezgene id 以便后续分析。

\begin{verbatim}
## # A tibble: 39 x 3
##    ensembl_gene_id entrezgene_id hgnc_symbol
##    <chr>                   <int> <chr>      
##  1 ENSG00000106546           196 AHR        
##  2 ENSG00000087088           581 BAX        
##  3 ENSG00000171552           598 BCL2L1     
##  4 ENSG00000129473           599 BCL2L2     
##  5 ENSG00000183625          1232 CCR3       
##  6 ENSG00000175315          1474 CST6       
##  7 ENSG00000162438         11330 CTRC       
##  8 ENSG00000141086          1506 CTRL       
##  9 ENSG00000277571          1991 ELANE      
## 10 ENSG00000012061          2067 ERCC1      
## # i 29 more rows
\end{verbatim}

将注释数据与筛选的化合物的靶点基因数据合并,并按照化合物分组。

各个化合物包含的显著性靶点基因数量信息:

\begin{verbatim}
## $`(2R)-3-(3,4-dihydroxyphenyl)-2-[(Z)-3-(3,4-dihydroxyphenyl)acryloyl]oxy-propionic acid`
## [1] 1
## 
## $`(2S,3S)-2-(3,4-dihydroxyphenyl)-7-hydroxy-4-[(E)-3-hydroxy-3-oxoprop-1-enyl]-2,3-dihydrobenzofuran-3-carboxylic acid`
## [1] 1
## 
## $`(6S)-6-(hydroxymethyl)-1,6-dimethyl-8,9-dihydro-7H-naphtho[8,7-g]benzofuran-10,11-dione`
## [1] 1
## 
## $`(E)-3-(3-hydroxy-4,5-dimethoxy-phenyl)acrylic acid`
## [1] 1
## 
## $`(E)-3-[2-(3,4-dihydroxyphenyl)-7-hydroxy-benzofuran-4-yl]acrylic acid`
## [1] 1
## 
## $`(R)-p-Menth-1-en-4-ol`
## [1] 2
## 
## $`1-methyl-8,9-dihydro-7H-naphtho[5,6-g]benzofuran-6,10,11-trione`
## [1] 1
## 
## $`1,2-DT-Quinone`
## [1] 1
## 
## $`1,2,5,6-tetrahydrotanshinone`
## [1] 1
## 
## $`15,16-dihydrotanshinone i`
## [1] 1
## 
## $`2-isopropyl-8-methylphenanthrene-3,4-dione`
## [1] 1
## 
## $`3-beta-Hydroxymethyllenetanshiquinone`
## [1] 1
## 
## $`3beta-Hydroxytanshinone IIA`
## [1] 1
## 
## $`3α-hydroxytanshinoneⅡa`
## [1] 1
## 
## $`4-methylenemiltirone`
## [1] 1
## 
## $`7-oxoroyleanone2`
## [1] 1
## 
## $aloeemodin
## [1] 3
## 
## $`alpha-amyrin`
## [1] 2
## 
## $aucubin
## [1] 1
## 
## $`beta-sitosterol`
## [1] 4
## 
## $`caffeic acid`
## [1] 2
## 
## $carnosol
## [1] 2
## 
## $`chlorogenic acid`
## [1] 1
## 
## $cryptotanshinone
## [1] 3
## 
## $cyanidol
## [1] 1
## 
## $`dan-shexinkum b`
## [1] 1
## 
## $`dan-shexinkum d`
## [1] 1
## 
## $`Danshenol A`
## [1] 1
## 
## $danshensu
## [1] 1
## 
## $`Dehydrotanshinone II A`
## [1] 1
## 
## $dihydroisotanshinoneⅠ
## [1] 1
## 
## $dihydrotanshinlactone
## [1] 1
## 
## $`dihydrotanshinone i`
## [1] 2
## 
## $dihydrotanshinoneⅠ
## [1] 1
## 
## $dimethyllithospermate
## [1] 1
## 
## $DTY
## [1] 1
## 
## $EIC
## [1] 1
## 
## $`ferulic acid`
## [1] 1
## 
## $formyltanshinone
## [1] 1
## 
## $GLY
## [1] 2
## 
## $`isoferulic acid`
## [1] 2
## 
## $`isotanshinone i`
## [1] 1
## 
## $kaempferol
## [1] 5
## 
## $labiatenicacid
## [1] 1
## 
## $Methylenetanshinquinone
## [1] 1
## 
## $methylrosmarinate
## [1] 1
## 
## $methyltanshinonate
## [1] 1
## 
## $`Mono-O-methylwightin`
## [1] 1
## 
## $Nortrachelogenin
## [1] 1
## 
## $`oleanolic acid`
## [1] 2
## 
## $`palmitic acid`
## [1] 3
## 
## $PHA
## [1] 1
## 
## $Poriferasterol
## [1] 1
## 
## $`prolithospermic acid`
## [1] 1
## 
## $`protocatechuic acid`
## [1] 2
## 
## $`przewalskin a`
## [1] 1
## 
## $`przewalskin b`
## [1] 1
## 
## $`Przewaquinone B`
## [1] 1
## 
## $`przewaquinone f`
## [1] 1
## 
## $rhein
## [1] 1
## 
## $`Rosemary acid`
## [1] 1
## 
## $rutin
## [1] 2
## 
## $`Sal A`
## [1] 2
## 
## $`salvianic acid c`
## [1] 1
## 
## $`salvianolic acid a`
## [1] 4
## 
## $Salvigenin
## [1] 1
## 
## $`salvilenone Ⅰ`
## [1] 1
## 
## $Tanshilactone
## [1] 1
## 
## $tanshinaldehyde
## [1] 1
## 
## $`Tanshinol A`
## [1] 1
## 
## $`tanshinone i`
## [1] 18
## 
## $`tanshinone Ⅵ`
## [1] 1
## 
## $`Z-8-Hexadecen-1-ol acetate`
## [1] 1
\end{verbatim}

除了tanshinone i,其他化合物都不超过 5 个靶点基因。

\hypertarget{ux86cbux767dux4e92ux4f5cux548c-hubgenes-ux7b5bux9009}{%
\subsection{蛋白互作和 Hubgenes 筛选}\label{ux86cbux767dux4e92ux4f5cux548c-hubgenes-ux7b5bux9009}}

\hypertarget{ux86cbux767dux4e92ux4f5c}{%
\subsubsection{蛋白互作}\label{ux86cbux767dux4e92ux4f5c}}

R 包\texttt{STRINGdb}提供网站\url{https://www.string-db.org/}的API,用以绘制蛋白质互作网络。

可视化的蛋白质互作网络图为见图\ref{fig:fig4}
\textbf{(对应文件为 \texttt{figs/protein\_interaction.pdf})}

\begin{figure}
\centering
\includegraphics{./figs/protein_interaction.pdf}
\caption{\label{fig:fig4}药物敏感性基因的蛋白质互作图}
\end{figure}

\hypertarget{hubgenes-ux7b5bux9009}{%
\subsubsection{Hubgenes 筛选}\label{hubgenes-ux7b5bux9009}}

利用 Cytoscape 的插件 CytoHubba\textsuperscript{\protect\hyperlink{ref-CytohubbaIdenChin2014}{1}} 提供的 MCC 算法计算
Hub 基因得分(这里 MCC 算法被集成到 R 中,独立计算)。

以下为结果概览:
\textbf{(对应文件为 \texttt{./tanshinone.iMCC\_score.xlsx})}

\begin{verbatim}
## # A tibble: 18 x 15
##    genes  MCC_score STRING_id    cor p.value components Description Category Uniprot.ID Gifts GC.Id Relevance.score GeneCards.Link
##    <chr>      <dbl> <chr>      <dbl>   <dbl> <chr>      <chr>       <chr>    <chr>      <int> <chr>           <dbl> <chr>         
##  1 PPARG         13 9606.ENS~ -0.396 0.00173 tanshinon~ Peroxisome~ Protein~ P37231        60 GC03~           47.0  https://www.g~
##  2 ITGAM         12 9606.ENS~  0.338 0.00834 tanshinon~ Integrin S~ Protein~ P11215        54 GC16~            9.38 https://www.g~
##  3 ST14           9 9606.ENS~ -0.290 0.0247  tanshinon~ ST14 Trans~ Protein~ Q9Y5Y6        55 GC11~           17.6  https://www.g~
##  4 PRSS8          9 9606.ENS~ -0.297 0.0210  tanshinon~ Serine Pro~ Protein~ Q16651        52 GC16~            6.83 https://www.g~
##  5 TMPRS~         9 9606.ENS~ -0.350 0.00604 tanshinon~ Transmembr~ Protein~ O15393        54 GC21~           17.5  https://www.g~
##  6 ELANE          9 9606.ENS~  0.310 0.0161  tanshinon~ Elastase, ~ Protein~ P08246        58 GC19~           11.6  https://www.g~
##  7 TMPRS~         8 9606.ENS~ -0.361 0.00462 tanshinon~ Transmembr~ Protein~ Q9NRS4        46 GC11~            8.03 https://www.g~
##  8 HGF            7 9606.ENS~  0.360 0.00475 tanshinon~ Hepatocyte~ Protein~ P14210        58 GC07~           30.4  https://www.g~
##  9 ITGA6          7 9606.ENS~ -0.365 0.00417 tanshinon~ Integrin S~ Protein~ P23229        57 GC02~           20.0  https://www.g~
## 10 CTRL           7 9606.ENS~  0.401 0.00148 tanshinon~ Chymotryps~ Protein~ P40313        47 GC16~            5.83 https://www.g~
## 11 TMPRS~         3 9606.ENS~  0.323 0.0118  tanshinon~ Transmembr~ Protein~ Q6ZMR5        43 GC04~            5.10 https://www.g~
## 12 GRB7           3 9606.ENS~ -0.366 0.00401 tanshinon~ Growth Fac~ Protein~ Q14451        50 GC17~           16.8  https://www.g~
## 13 KLK6           2 9606.ENS~ -0.359 0.00491 tanshinon~ Kallikrein~ Protein~ Q92876        51 GC19~           12.0  https://www.g~
## 14 MYBL2          1 9606.ENS~  0.280 0.0305  tanshinon~ MYB Proto-~ Protein~ P10244        49 GC20~            8.81 https://www.g~
## 15 BAX            1 9606.ENS~  0.267 0.0393  tanshinon~ BCL2 Assoc~ Protein~ Q07812        57 GC19~           52.9  https://www.g~
## 16 CST6           1 9606.ENS~ -0.272 0.0355  tanshinon~ Cystatin E~ Protein~ Q15828        48 GC11~            6.68 https://www.g~
## 17 CTRC           1 9606.ENS~  0.325 0.0112  tanshinon~ Chymotryps~ Protein~ Q99895        52 GC01~            6.27 https://www.g~
## 18 ERCC1         NA 9606.ENS~  0.260 0.0444  tanshinon~ ERCC Excis~ Protein~ P07992        53 GC19~           35.5  https://www.g~
## # i 2 more variables: ensembl_gene_id <chr>, entrezgene_id <int>
\end{verbatim}

将结果可视化,见 Figure \ref{fig:mcc}

\begin{figure}
\centering
\includegraphics{./figs/MCC_top10.pdf}
\caption{\label{fig:mcc}MCC score of gene targets of Tanshinone i}
\end{figure}

\hypertarget{ux4f7fux7528clusterprofilerux5bccux96c6ux5206ux6790}{%
\subsection{\texorpdfstring{使用\texttt{clusterProfiler}富集分析}{使用clusterProfiler富集分析}}\label{ux4f7fux7528clusterprofilerux5bccux96c6ux5206ux6790}}

tanshinone i 的富集分析用 MCC top 10 的靶点基因进行,其他化合物的靶点基因直接以关联分析筛选过的基因富集。

以下为tanshinone i的 KEGG 和 GO 富集图:

\begin{figure}
\centering
\includegraphics{./enrichKEGG/71_tanshinone i.pdf}
\caption{\label{fig:sigKEGG}KEGG enrichment}
\end{figure}

\begin{figure}
\centering
\includegraphics{./enrichGO/71_tanshinone i.pdf}
\caption{\label{fig:sigGO}GO enrichment}
\end{figure}

图片数量较多,不一一展示(KEGG 富集共73个,GO富集共73)。
\textbf{(对应文件为 \texttt{./enrichGO}, \texttt{./enrichKEGG})}
对富集图的解释,可以参考文献\textsuperscript{\protect\hyperlink{ref-IntegrativeAnaLiuY2020}{2}}。

说明:KEGG富集分析都有结果;但是对于GO 富集分析(BP,CC 或 MF)中,个别化合物有靶点基因,但未映射到通路中的基因,所以无结果, 这些是(TRUE表示有结果,而FALSE表示无结果):

\begin{verbatim}
## $`(2R)-3-(3,4-dihydroxyphenyl)-2-[(Z)-3-(3,4-dihydroxyphenyl)acryloyl]oxy-propionic acid`
##   BP   CC   MF 
## TRUE TRUE TRUE 
## 
## $`(2S,3S)-2-(3,4-dihydroxyphenyl)-7-hydroxy-4-[(E)-3-hydroxy-3-oxoprop-1-enyl]-2,3-dihydrobenzofuran-3-carboxylic acid`
##   BP   CC   MF 
## TRUE TRUE TRUE 
## 
## $`(6S)-6-(hydroxymethyl)-1,6-dimethyl-8,9-dihydro-7H-naphtho[8,7-g]benzofuran-10,11-dione`
##   BP   CC   MF 
## TRUE TRUE TRUE 
## 
## $`(E)-3-(3-hydroxy-4,5-dimethoxy-phenyl)acrylic acid`
##   BP   CC   MF 
## TRUE TRUE TRUE 
## 
## $`(E)-3-[2-(3,4-dihydroxyphenyl)-7-hydroxy-benzofuran-4-yl]acrylic acid`
##   BP   CC   MF 
## TRUE TRUE TRUE 
## 
## $`(R)-p-Menth-1-en-4-ol`
##   BP   CC   MF 
## TRUE TRUE TRUE 
## 
## $`1-methyl-8,9-dihydro-7H-naphtho[5,6-g]benzofuran-6,10,11-trione`
##    BP    CC    MF 
##  TRUE FALSE  TRUE 
## 
## $`1,2-DT-Quinone`
##   BP   CC   MF 
## TRUE TRUE TRUE 
## 
## $`1,2,5,6-tetrahydrotanshinone`
##   BP   CC   MF 
## TRUE TRUE TRUE 
## 
## $`15,16-dihydrotanshinone i`
##   BP   CC   MF 
## TRUE TRUE TRUE 
## 
## $`2-isopropyl-8-methylphenanthrene-3,4-dione`
##   BP   CC   MF 
## TRUE TRUE TRUE 
## 
## $`3-beta-Hydroxymethyllenetanshiquinone`
##   BP   CC   MF 
## TRUE TRUE TRUE 
## 
## $`3beta-Hydroxytanshinone IIA`
##   BP   CC   MF 
## TRUE TRUE TRUE 
## 
## $`3α-hydroxytanshinoneⅡa`
##   BP   CC   MF 
## TRUE TRUE TRUE 
## 
## $`4-methylenemiltirone`
##   BP   CC   MF 
## TRUE TRUE TRUE 
## 
## $`7-oxoroyleanone2`
##    BP    CC    MF 
##  TRUE FALSE  TRUE 
## 
## $aloeemodin
##   BP   CC   MF 
## TRUE TRUE TRUE 
## 
## $`alpha-amyrin`
##   BP   CC   MF 
## TRUE TRUE TRUE 
## 
## $aucubin
##   BP   CC   MF 
## TRUE TRUE TRUE 
## 
## $`beta-sitosterol`
##   BP   CC   MF 
## TRUE TRUE TRUE 
## 
## $`caffeic acid`
##   BP   CC   MF 
## TRUE TRUE TRUE 
## 
## $carnosol
##   BP   CC   MF 
## TRUE TRUE TRUE 
## 
## $`chlorogenic acid`
##    BP    CC    MF 
##  TRUE FALSE  TRUE 
## 
## $cryptotanshinone
##   BP   CC   MF 
## TRUE TRUE TRUE 
## 
## $cyanidol
##   BP   CC   MF 
## TRUE TRUE TRUE 
## 
## $`dan-shexinkum b`
##   BP   CC   MF 
## TRUE TRUE TRUE 
## 
## $`dan-shexinkum d`
##   BP   CC   MF 
## TRUE TRUE TRUE 
## 
## $`Danshenol A`
##   BP   CC   MF 
## TRUE TRUE TRUE 
## 
## $danshensu
##   BP   CC   MF 
## TRUE TRUE TRUE 
## 
## $`Dehydrotanshinone II A`
##   BP   CC   MF 
## TRUE TRUE TRUE 
## 
## $dihydroisotanshinoneⅠ
##   BP   CC   MF 
## TRUE TRUE TRUE 
## 
## $dihydrotanshinlactone
##    BP    CC    MF 
##  TRUE  TRUE FALSE 
## 
## $`dihydrotanshinone i`
##   BP   CC   MF 
## TRUE TRUE TRUE 
## 
## $dihydrotanshinoneⅠ
##   BP   CC   MF 
## TRUE TRUE TRUE 
## 
## $dimethyllithospermate
##   BP   CC   MF 
## TRUE TRUE TRUE 
## 
## $DTY
##   BP   CC   MF 
## TRUE TRUE TRUE 
## 
## $EIC
##   BP   CC   MF 
## TRUE TRUE TRUE 
## 
## $`ferulic acid`
##   BP   CC   MF 
## TRUE TRUE TRUE 
## 
## $formyltanshinone
##   BP   CC   MF 
## TRUE TRUE TRUE 
## 
## $GLY
##   BP   CC   MF 
## TRUE TRUE TRUE 
## 
## $`isoferulic acid`
##   BP   CC   MF 
## TRUE TRUE TRUE 
## 
## $`isotanshinone i`
##   BP   CC   MF 
## TRUE TRUE TRUE 
## 
## $kaempferol
##   BP   CC   MF 
## TRUE TRUE TRUE 
## 
## $labiatenicacid
##    BP    CC    MF 
##  TRUE FALSE  TRUE 
## 
## $Methylenetanshinquinone
##   BP   CC   MF 
## TRUE TRUE TRUE 
## 
## $methylrosmarinate
##   BP   CC   MF 
## TRUE TRUE TRUE 
## 
## $methyltanshinonate
##   BP   CC   MF 
## TRUE TRUE TRUE 
## 
## $`Mono-O-methylwightin`
##   BP   CC   MF 
## TRUE TRUE TRUE 
## 
## $Nortrachelogenin
##   BP   CC   MF 
## TRUE TRUE TRUE 
## 
## $`oleanolic acid`
##   BP   CC   MF 
## TRUE TRUE TRUE 
## 
## $`palmitic acid`
##   BP   CC   MF 
## TRUE TRUE TRUE 
## 
## $PHA
##   BP   CC   MF 
## TRUE TRUE TRUE 
## 
## $Poriferasterol
##    BP    CC    MF 
##  TRUE FALSE  TRUE 
## 
## $`prolithospermic acid`
##   BP   CC   MF 
## TRUE TRUE TRUE 
## 
## $`protocatechuic acid`
##   BP   CC   MF 
## TRUE TRUE TRUE 
## 
## $`przewalskin a`
##    BP    CC    MF 
##  TRUE FALSE  TRUE 
## 
## $`przewalskin b`
##    BP    CC    MF 
##  TRUE FALSE  TRUE 
## 
## $`Przewaquinone B`
##   BP   CC   MF 
## TRUE TRUE TRUE 
## 
## $`przewaquinone f`
##   BP   CC   MF 
## TRUE TRUE TRUE 
## 
## $rhein
##   BP   CC   MF 
## TRUE TRUE TRUE 
## 
## $`Rosemary acid`
##    BP    CC    MF 
##  TRUE FALSE  TRUE 
## 
## $rutin
##   BP   CC   MF 
## TRUE TRUE TRUE 
## 
## $`Sal A`
##   BP   CC   MF 
## TRUE TRUE TRUE 
## 
## $`salvianic acid c`
##   BP   CC   MF 
## TRUE TRUE TRUE 
## 
## $`salvianolic acid a`
##   BP   CC   MF 
## TRUE TRUE TRUE 
## 
## $Salvigenin
##   BP   CC   MF 
## TRUE TRUE TRUE 
## 
## $`salvilenone Ⅰ`
##   BP   CC   MF 
## TRUE TRUE TRUE 
## 
## $Tanshilactone
##   BP   CC   MF 
## TRUE TRUE TRUE 
## 
## $tanshinaldehyde
##   BP   CC   MF 
## TRUE TRUE TRUE 
## 
## $`Tanshinol A`
##   BP   CC   MF 
## TRUE TRUE TRUE 
## 
## $`tanshinone i`
##   BP   CC   MF 
## TRUE TRUE TRUE 
## 
## $`tanshinone Ⅵ`
##   BP   CC   MF 
## TRUE TRUE TRUE 
## 
## $`Z-8-Hexadecen-1-ol acetate`
##   BP   CC   MF 
## TRUE TRUE TRUE
\end{verbatim}

\hypertarget{bibliography}{%
\section*{Reference}\label{bibliography}}
\addcontentsline{toc}{section}{Reference}

\hypertarget{refs}{}
\begin{cslreferences}
\leavevmode\hypertarget{ref-CytohubbaIdenChin2014}{}%
1. Chin, C.-H. \emph{et al.} CytoHubba: Identifying hub objects and sub-networks from complex interactome. \emph{BMC Systems Biology} \textbf{8}, S11 (2014).

\leavevmode\hypertarget{ref-IntegrativeAnaLiuY2020}{}%
2. Liu, Y. \emph{et al.} Integrative analyses of biomarkers and pathways for adipose tissue after bariatric surgery. \emph{Adipocyte} \textbf{9}, 384--400 (2020).
\end{cslreferences}

\end{document}
