% Options for packages loaded elsewhere
\PassOptionsToPackage{unicode}{hyperref}
\PassOptionsToPackage{hyphens}{url}
%
\documentclass[
]{article}
\usepackage{lmodern}
\usepackage{amssymb,amsmath}
\usepackage{ifxetex,ifluatex}
\ifnum 0\ifxetex 1\fi\ifluatex 1\fi=0 % if pdftex
  \usepackage[T1]{fontenc}
  \usepackage[utf8]{inputenc}
  \usepackage{textcomp} % provide euro and other symbols
\else % if luatex or xetex
  \usepackage{unicode-math}
  \defaultfontfeatures{Scale=MatchLowercase}
  \defaultfontfeatures[\rmfamily]{Ligatures=TeX,Scale=1}
\fi
% Use upquote if available, for straight quotes in verbatim environments
\IfFileExists{upquote.sty}{\usepackage{upquote}}{}
\IfFileExists{microtype.sty}{% use microtype if available
  \usepackage[]{microtype}
  \UseMicrotypeSet[protrusion]{basicmath} % disable protrusion for tt fonts
}{}
\makeatletter
\@ifundefined{KOMAClassName}{% if non-KOMA class
  \IfFileExists{parskip.sty}{%
    \usepackage{parskip}
  }{% else
    \setlength{\parindent}{0pt}
    \setlength{\parskip}{6pt plus 2pt minus 1pt}}
}{% if KOMA class
  \KOMAoptions{parskip=half}}
\makeatother
\usepackage{xcolor}
\IfFileExists{xurl.sty}{\usepackage{xurl}}{} % add URL line breaks if available
\IfFileExists{bookmark.sty}{\usepackage{bookmark}}{\usepackage{hyperref}}
\hypersetup{
  pdftitle={Report of Analysis},
  pdfauthor={Huang LiChuang of Wie-Biotech},
  hidelinks,
  pdfcreator={LaTeX via pandoc}}
\urlstyle{same} % disable monospaced font for URLs
\usepackage[margin=1in]{geometry}
\usepackage{longtable,booktabs}
% Correct order of tables after \paragraph or \subparagraph
\usepackage{etoolbox}
\makeatletter
\patchcmd\longtable{\par}{\if@noskipsec\mbox{}\fi\par}{}{}
\makeatother
% Allow footnotes in longtable head/foot
\IfFileExists{footnotehyper.sty}{\usepackage{footnotehyper}}{\usepackage{footnote}}
\makesavenoteenv{longtable}
\usepackage{graphicx}
\makeatletter
\def\maxwidth{\ifdim\Gin@nat@width>\linewidth\linewidth\else\Gin@nat@width\fi}
\def\maxheight{\ifdim\Gin@nat@height>\textheight\textheight\else\Gin@nat@height\fi}
\makeatother
% Scale images if necessary, so that they will not overflow the page
% margins by default, and it is still possible to overwrite the defaults
% using explicit options in \includegraphics[width, height, ...]{}
\setkeys{Gin}{width=\maxwidth,height=\maxheight,keepaspectratio}
% Set default figure placement to htbp
\makeatletter
\def\fps@figure{htbp}
\makeatother
\setlength{\emergencystretch}{3em} % prevent overfull lines
\providecommand{\tightlist}{%
  \setlength{\itemsep}{0pt}\setlength{\parskip}{0pt}}
\setcounter{secnumdepth}{5}
\usepackage{caption} \captionsetup{font={footnotesize},width=6in} \renewcommand{\dblfloatpagefraction}{.9} \makeatletter \renewenvironment{figure} {\def\@captype{figure}} \makeatother \definecolor{shadecolor}{RGB}{242,242,242} \usepackage{xeCJK} \usepackage{setspace} \setstretch{1.3} \usepackage{tcolorbox}
\newlength{\cslhangindent}
\setlength{\cslhangindent}{1.5em}
\newenvironment{cslreferences}%
  {}%
  {\par}

\title{Report of Analysis}
\author{Huang LiChuang of Wie-Biotech}
\date{}

\begin{document}
\maketitle

{
\setcounter{tocdepth}{3}
\tableofcontents
}
\listoffigures

\listoftables

\hypertarget{test}{%
\section{test}\label{test}}

Figure \ref{fig:test}为图test概览。

\textbf{(对应文件为 \texttt{../2023\_06\_30\_eval/figs/370\_into\_1cxw.png})}

\def\@captype{figure}
\begin{center}
\includegraphics[width = 0.9\linewidth]{../2023_06_30_eval/figs/370_into_1cxw.png}
\caption{Test}\label{fig:test}
\end{center}

Figure \ref{fig:report-test}为图report test概览。

\textbf{(对应文件为 \texttt{../2023\_06\_25\_fix/figs/MCC\_top10.pdf})}

\def\@captype{figure}
\begin{center}
\includegraphics[width = 0.9\linewidth]{../2023_06_25_fix/figs/MCC_top10.pdf}
\caption{Report test}\label{fig:report-test}
\end{center}

Figure \ref{fig:test-figure-out-of-line}为图test figure out of line概览。

\textbf{(对应文件为 \texttt{../2023\_06\_30\_eval/figs/5280343\_into\_5th6.png})}

\def\@captype{figure}
\begin{center}
\includegraphics[width = 0.9\linewidth]{../2023_06_30_eval/figs/5280343_into_5th6.png}
\caption{Test figure out of line}\label{fig:test-figure-out-of-line}
\end{center}

Figure \ref{fig:plot-with-ggplot2}为图plot with ggplot2概览。

\textbf{(对应文件为 \texttt{figs/plot-with-ggplot2.pdf})}

\def\@captype{figure}
\begin{center}
\includegraphics[width = 0.9\linewidth]{figs/plot-with-ggplot2.pdf}
\caption{Plot with ggplot2}\label{fig:plot-with-ggplot2}
\end{center}

Table \ref{tab:mtcars}为表格mtcars概览。
test

\textbf{(对应文件为 \texttt{tabs/mtcars.csv})}

\begin{center}\begin{tcolorbox}[colback=gray!10, colframe=gray!50, width=0.9\linewidth, arc=1mm, boxrule=0.5pt]注:表格共有32行11列,以下预览的表格可能省略部分数据;表格含有25个唯一`mpg'。
show time\end{tcolorbox}
\end{center}

\begin{longtable}[]{@{}lllllllllll@{}}
\caption{\label{tab:mtcars}Mtcars}\tabularnewline
\toprule
mpg & cyl & disp & hp & drat & wt & qsec & vs & am & gear & carb\tabularnewline
\midrule
\endfirsthead
\toprule
mpg & cyl & disp & hp & drat & wt & qsec & vs & am & gear & carb\tabularnewline
\midrule
\endhead
21 & 6 & 160 & 110 & 3.9 & 2.62 & 16.46 & 0 & 1 & 4 & 4\tabularnewline
21 & 6 & 160 & 110 & 3.9 & 2.875 & 17.02 & 0 & 1 & 4 & 4\tabularnewline
22.8 & 4 & 108 & 93 & 3.85 & 2.32 & 18.61 & 1 & 1 & 4 & 1\tabularnewline
21.4 & 6 & 258 & 110 & 3.08 & 3.215 & 19.44 & 1 & 0 & 3 & 1\tabularnewline
18.7 & 8 & 360 & 175 & 3.15 & 3.44 & 17.02 & 0 & 0 & 3 & 2\tabularnewline
18.1 & 6 & 225 & 105 & 2.76 & 3.46 & 20.22 & 1 & 0 & 3 & 1\tabularnewline
14.3 & 8 & 360 & 245 & 3.21 & 3.57 & 15.84 & 0 & 0 & 3 & 4\tabularnewline
24.4 & 4 & 146.7 & 62 & 3.69 & 3.19 & 20 & 1 & 0 & 4 & 2\tabularnewline
22.8 & 4 & 140.8 & 95 & 3.92 & 3.15 & 22.9 & 1 & 0 & 4 & 2\tabularnewline
19.2 & 6 & 167.6 & 123 & 3.92 & 3.44 & 18.3 & 1 & 0 & 4 & 4\tabularnewline
17.8 & 6 & 167.6 & 123 & 3.92 & 3.44 & 18.9 & 1 & 0 & 4 & 4\tabularnewline
16.4 & 8 & 275.8 & 180 & 3.07 & 4.07 & 17.4 & 0 & 0 & 3 & 3\tabularnewline
17.3 & 8 & 275.8 & 180 & 3.07 & 3.73 & 17.6 & 0 & 0 & 3 & 3\tabularnewline
15.2 & 8 & 275.8 & 180 & 3.07 & 3.78 & 18 & 0 & 0 & 3 & 3\tabularnewline
10.4 & 8 & 472 & 205 & 2.93 & 5.25 & 17.98 & 0 & 0 & 3 & 4\tabularnewline
\ldots{} & \ldots{} & \ldots{} & \ldots{} & \ldots{} & \ldots{} & \ldots{} & \ldots{} & \ldots{} & \ldots{} & \ldots{}\tabularnewline
\bottomrule
\end{longtable}

\begin{verbatim}
## Warning: Removed 63 rows containing missing values (`geom_point()`).
\end{verbatim}

Figure \ref{fig:heatdata}为图heatdata概览。

\textbf{(对应文件为 \texttt{figs/heatdata.pdf})}

\def\@captype{figure}
\begin{center}
\includegraphics[width = 0.9\linewidth]{figs/heatdata.pdf}
\caption{Heatdata}\label{fig:heatdata}
\end{center}

这是自己的文件
`Custom function' 数据已全部提供。

\textbf{(对应文件为 \texttt{../../ahr\_sig})}

\begin{center}\begin{tcolorbox}[colback=gray!10, colframe=gray!50, width=0.9\linewidth, arc=1mm, boxrule=0.5pt]注:文件夹../../ahr\_sig共包含6个文件。

\begin{enumerate}\tightlist
\item analysis\_data.1.R
\item analysis\_data.10.R
\item analysis\_data.11.R
\item analysis\_data.12.R
\item analysis\_data.13.R
\item ...
\end{enumerate}\end{tcolorbox}
\end{center}

`List of mtcars' 数据已全部提供。

\textbf{(对应文件为 \texttt{list-of-mtcars})}

\begin{center}\begin{tcolorbox}[colback=gray!10, colframe=gray!50, width=0.9\linewidth, arc=1mm, boxrule=0.5pt]注:文件夹list-of-mtcars共包含3个文件。

\begin{enumerate}\tightlist
\item 1\_mtcars.csv
\item 2\_mtcars.csv
\item 3\_iris.csv
\end{enumerate}\end{tcolorbox}
\end{center}

`List of test ggplot' 数据已全部提供。

\textbf{(对应文件为 \texttt{list-of-test-ggplot})}

\begin{center}\begin{tcolorbox}[colback=gray!10, colframe=gray!50, width=0.9\linewidth, arc=1mm, boxrule=0.5pt]注:文件夹list-of-test-ggplot共包含1个文件。

\begin{enumerate}\tightlist
\item 1\_point.pdf
\end{enumerate}\end{tcolorbox}
\end{center}

Figure \ref{fig:grob}为图grob概览。

\textbf{(对应文件为 \texttt{figs/grob.pdf})}

\def\@captype{figure}
\begin{center}
\includegraphics[width = 0.9\linewidth]{figs/grob.pdf}
\caption{Grob}\label{fig:grob}
\end{center}

\hypertarget{ux751fux4fe1ux8bc4ux4f30}{%
\section{生信评估}\label{ux751fux4fe1ux8bc4ux4f30}}

关于转录组数据库筛选肌少症、癌症(结直肠癌)、化疗共同的通路,是否要串连三个要素(肌少症,结直肠癌,化疗)?

\begin{enumerate}
\def\labelenumi{\arabic{enumi}.}
\tightlist
\item
  使用公共数据库,可行:

  \begin{enumerate}
  \def\labelenumii{\arabic{enumii}.}
  \tightlist
  \item
    筛选 GEO 至少两个数据集,一个肌少症,另一个结直肠癌症。后者最好包含化疗前后的两组数据。如果 GEO 数据库不存在结直肠癌化疗前后的合适数据,可能需要筛选三个数据集供处理。
  \item
    消除批次效应(不同来源的数据的各种无关因素带来的影响)。
  \item
    筛选共通基因,有两种方法:

    \begin{enumerate}
    \def\labelenumiii{\arabic{enumiii}.}
    \tightlist
    \item
      一般法,多重比对,取交集。
    \item
      WGCNA\textsuperscript{\protect\hyperlink{ref-WgcnaAnRPacLangfe2008}{1}} 法,对多个数据集进行一致性分析,寻找关键基因模块。
    \item
      联合法,联合上述两种方法。
    \end{enumerate}
  \item
    根据基因筛选结果,通路富集分析。
  \end{enumerate}
\item
  使用自备数据集,可行。可以更有针对性的设计实验分组,避免批次效应,分析结果更可靠;但针对性越强,成本越高。
\end{enumerate}

工作量:视数据集的多少和分析复杂程度,需要2-3天。

\hypertarget{bibliography}{%
\section*{Reference}\label{bibliography}}
\addcontentsline{toc}{section}{Reference}

\hypertarget{refs}{}
\begin{cslreferences}
\leavevmode\hypertarget{ref-WgcnaAnRPacLangfe2008}{}%
1. Langfelder, P. \& Horvath, S. WGCNA: An r package for weighted correlation network analysis. \emph{BMC Bioinformatics} \textbf{9}, (2008).
\end{cslreferences}

\end{document}
