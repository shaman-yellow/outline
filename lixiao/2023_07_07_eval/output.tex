% Options for packages loaded elsewhere
\PassOptionsToPackage{unicode}{hyperref}
\PassOptionsToPackage{hyphens}{url}
%
\documentclass[
]{article}
\usepackage{lmodern}
\usepackage{amssymb,amsmath}
\usepackage{ifxetex,ifluatex}
\ifnum 0\ifxetex 1\fi\ifluatex 1\fi=0 % if pdftex
  \usepackage[T1]{fontenc}
  \usepackage[utf8]{inputenc}
  \usepackage{textcomp} % provide euro and other symbols
\else % if luatex or xetex
  \usepackage{unicode-math}
  \defaultfontfeatures{Scale=MatchLowercase}
  \defaultfontfeatures[\rmfamily]{Ligatures=TeX,Scale=1}
\fi
% Use upquote if available, for straight quotes in verbatim environments
\IfFileExists{upquote.sty}{\usepackage{upquote}}{}
\IfFileExists{microtype.sty}{% use microtype if available
  \usepackage[]{microtype}
  \UseMicrotypeSet[protrusion]{basicmath} % disable protrusion for tt fonts
}{}
\makeatletter
\@ifundefined{KOMAClassName}{% if non-KOMA class
  \IfFileExists{parskip.sty}{%
    \usepackage{parskip}
  }{% else
    \setlength{\parindent}{0pt}
    \setlength{\parskip}{6pt plus 2pt minus 1pt}}
}{% if KOMA class
  \KOMAoptions{parskip=half}}
\makeatother
\usepackage{xcolor}
\IfFileExists{xurl.sty}{\usepackage{xurl}}{} % add URL line breaks if available
\IfFileExists{bookmark.sty}{\usepackage{bookmark}}{\usepackage{hyperref}}
\hypersetup{
  pdftitle={Analysis},
  pdfauthor={Huang LiChuang of Wie-Biotech},
  hidelinks,
  pdfcreator={LaTeX via pandoc}}
\urlstyle{same} % disable monospaced font for URLs
\usepackage[margin=1in]{geometry}
\usepackage{longtable,booktabs}
% Correct order of tables after \paragraph or \subparagraph
\usepackage{etoolbox}
\makeatletter
\patchcmd\longtable{\par}{\if@noskipsec\mbox{}\fi\par}{}{}
\makeatother
% Allow footnotes in longtable head/foot
\IfFileExists{footnotehyper.sty}{\usepackage{footnotehyper}}{\usepackage{footnote}}
\makesavenoteenv{longtable}
\usepackage{graphicx}
\makeatletter
\def\maxwidth{\ifdim\Gin@nat@width>\linewidth\linewidth\else\Gin@nat@width\fi}
\def\maxheight{\ifdim\Gin@nat@height>\textheight\textheight\else\Gin@nat@height\fi}
\makeatother
% Scale images if necessary, so that they will not overflow the page
% margins by default, and it is still possible to overwrite the defaults
% using explicit options in \includegraphics[width, height, ...]{}
\setkeys{Gin}{width=\maxwidth,height=\maxheight,keepaspectratio}
% Set default figure placement to htbp
\makeatletter
\def\fps@figure{htbp}
\makeatother
\setlength{\emergencystretch}{3em} % prevent overfull lines
\providecommand{\tightlist}{%
  \setlength{\itemsep}{0pt}\setlength{\parskip}{0pt}}
\setcounter{secnumdepth}{5}
\usepackage{caption} \captionsetup{font={footnotesize},width=6in} \renewcommand{\dblfloatpagefraction}{.9} \makeatletter \renewenvironment{figure} {\def\@captype{figure}} \makeatother \definecolor{shadecolor}{RGB}{242,242,242} \usepackage{xeCJK} \usepackage{setspace} \setstretch{1.3} \usepackage{tcolorbox}
\newlength{\cslhangindent}
\setlength{\cslhangindent}{1.5em}
\newenvironment{cslreferences}%
  {}%
  {\par}

\title{Analysis}
\author{Huang LiChuang of Wie-Biotech}
\date{}

\begin{document}
\maketitle

{
\setcounter{tocdepth}{3}
\tableofcontents
}
\listoffigures

\listoftables

\hypertarget{abstract}{%
\section{摘要}\label{abstract}}

根据客户需求和提供的数据,筛出(瘢痕增生)能够与 TCF-AS1 结合又能与 TCF4 结合的 RNA 结合蛋白。
结果请参考 \ref{dis}

\hypertarget{route}{%
\section{研究设计流程图}\label{route}}

\includegraphics[width=\linewidth]{output_files/figure-latex/unnamed-chunk-4-1}

\hypertarget{methods}{%
\section{材料和方法}\label{methods}}

\begin{enumerate}
\def\labelenumi{\arabic{enumi}.}
\tightlist
\item
  获取参考注释基因。
\item
  初步处理客户提供的数据(fastp 质控、kallisto\textsuperscript{\protect\hyperlink{ref-NearOptimalPrBray2016}{1}} 对比到参考基因座等)。
\item
  使用 limma 差异分析\textsuperscript{\protect\hyperlink{ref-LimmaPowersDiRitchi2015}{2}}。
\item
  使用 WGCNA 方法\textsuperscript{\protect\hyperlink{ref-WgcnaAnRPacLangfe2008}{3}},从差异基因中筛选与 TCF4-AS1 lncRNA 和 TCF4mRNA 具有共表达关系的基
  因。
\item
  视情况选择合适的预测工具\textsuperscript{\protect\hyperlink{ref-ICatIRapidArmaos2021}{4}--\protect\hyperlink{ref-RckAccurateAOrenst2016}{7}},预测蛋白和 RNA 的结合程度,并可视化为图表。
\end{enumerate}

\hypertarget{results}{%
\section{分析结果}\label{results}}

\hypertarget{ux4e0bux8f7dux53c2ux8003ux57faux56e0ux7ec4ux6ce8ux91caux6587ux4ef6}{%
\subsection{下载参考基因组注释文件}\label{ux4e0bux8f7dux53c2ux8003ux57faux56e0ux7ec4ux6ce8ux91caux6587ux4ef6}}

下载 cDNA 和 ncRNA 参考基因注释。
\url{https://ftp.ensembl.org/pub/release-110/fasta/homo_sapiens/}

\hypertarget{fastq-ux9884ux5904ux7406}{%
\subsection{fastq 预处理}\label{fastq-ux9884ux5904ux7406}}

\hypertarget{ux6570ux636eux8d28ux63a7}{%
\subsubsection{数据质控}\label{ux6570ux636eux8d28ux63a7}}

使用fastp去接头和去低质量的碱基

此为 fastp 处理时生成的报告文件。
`Reports fastq files processed with fastp' 数据已全部提供。

\textbf{(对应文件为 \texttt{./fastp\_report})}

\begin{center}\begin{tcolorbox}[colback=gray!10, colframe=gray!50, width=0.9\linewidth, arc=1mm, boxrule=0.5pt]注:文件夹./fastp\_report共包含6个文件。

\begin{enumerate}\tightlist
\item CT1-CT1\_combined\_R.html
\item CT2-CT2\_combined\_R.html
\item CT3-CT3\_combined\_R.html
\item CUR1\_R.html
\item CUR2\_R.html
\item ...
\end{enumerate}\end{tcolorbox}
\end{center}

\hypertarget{ux4f7fux7528-kallisto-ux6bd4ux5bf9-fastq-ux5230ux53c2ux8003ux57faux56e0ux5ea7}{%
\subsection{使用 kallisto 比对 fastq 到参考基因座}\label{ux4f7fux7528-kallisto-ux6bd4ux5bf9-fastq-ux5230ux53c2ux8003ux57faux56e0ux5ea7}}

kallisto 提供了快速且准确的 fastq 比对到参考基因座的方法\textsuperscript{\protect\hyperlink{ref-NearOptimalPrBray2016}{1}} (\url{http://pachterlab.github.io/kallisto/manual.html})。

\hypertarget{ux9274ux5b9a-mrna}{%
\subsubsection{鉴定 mRNA}\label{ux9274ux5b9a-mrna}}

使用 kallisto 将 fastq 与 hg38 的 cDNA 数据比对。

主要为子目录下的 abundance.tsv 文件。
`Refer to mRNA' 数据已全部提供。

\textbf{(对应文件为 \texttt{./quant\_hg38\_mrna})}

\begin{center}\begin{tcolorbox}[colback=gray!10, colframe=gray!50, width=0.9\linewidth, arc=1mm, boxrule=0.5pt]注:文件夹./quant\_hg38\_mrna共包含6个文件。

\begin{enumerate}\tightlist
\item CT1-CT1\_combined\_R
\item CT2-CT2\_combined\_R
\item CT3-CT3\_combined\_R
\item CUR1\_R
\item CUR2\_R
\item ...
\end{enumerate}\end{tcolorbox}
\end{center}

\hypertarget{ux9274ux5b9a-ncrna}{%
\subsubsection{鉴定 ncRNA}\label{ux9274ux5b9a-ncrna}}

使用 kallisto 将 fastq 与 hg38 的 ncRNA 数据比对。

主要为子目录下的 abundance.tsv 文件。
`Refer to ncRNA' 数据已全部提供。

\textbf{(对应文件为 \texttt{./quant\_hg38\_ncrna})}

\begin{center}\begin{tcolorbox}[colback=gray!10, colframe=gray!50, width=0.9\linewidth, arc=1mm, boxrule=0.5pt]注:文件夹./quant\_hg38\_ncrna共包含6个文件。

\begin{enumerate}\tightlist
\item CT1-CT1\_combined\_R
\item CT2-CT2\_combined\_R
\item CT3-CT3\_combined\_R
\item CUR1\_R
\item CUR2\_R
\item ...
\end{enumerate}\end{tcolorbox}
\end{center}

\hypertarget{ux5deeux5f02ux5206ux6790}{%
\subsection{差异分析}\label{ux5deeux5f02ux5206ux6790}}

\hypertarget{ux8bfbux53d6ux5e76ux5408ux5e76ux4e0dux540cux6837ux672c-rna-ux5b9aux91cfux6570ux636e}{%
\subsubsection{读取并合并不同样本 RNA 定量数据}\label{ux8bfbux53d6ux5e76ux5408ux5e76ux4e0dux540cux6837ux672c-rna-ux5b9aux91cfux6570ux636e}}

Table \ref{tab:merged-mrna}为表格merged mrna概览。

\textbf{(对应文件为 \texttt{Figure+Table/merged-mrna.csv})}

\begin{center}\begin{tcolorbox}[colback=gray!10, colframe=gray!50, width=0.9\linewidth, arc=1mm, boxrule=0.5pt]注:表格共有207249行7列,以下预览的表格可能省略部分数据;表格含有207249个唯一`target\_id'。
\end{tcolorbox}
\end{center}

\begin{longtable}[]{@{}lllllll@{}}
\caption{\label{tab:merged-mrna}Merged mrna}\tabularnewline
\toprule
targe\ldots{} & CT1-C\ldots{} & CT2-C\ldots{} & CT3-C\ldots{} & CUR1\_R & CUR2\_R & CUR3\_R\tabularnewline
\midrule
\endfirsthead
\toprule
targe\ldots{} & CT1-C\ldots{} & CT2-C\ldots{} & CT3-C\ldots{} & CUR1\_R & CUR2\_R & CUR3\_R\tabularnewline
\midrule
\endhead
ENST0\ldots{} & 0 & 0 & 0 & 0 & 0 & 0\tabularnewline
ENST0\ldots{} & 0 & 0 & 0 & 0 & 0 & 0\tabularnewline
ENST0\ldots{} & 0 & 0 & 0 & 0 & 0 & 0\tabularnewline
ENST0\ldots{} & 0 & 0 & 0 & 0 & 0 & 0\tabularnewline
ENST0\ldots{} & 0 & 0 & 0 & 0 & 0 & 0\tabularnewline
ENST0\ldots{} & 0 & 0 & 0 & 0 & 0 & 0\tabularnewline
ENST0\ldots{} & 0 & 0 & 0 & 0 & 0 & 0\tabularnewline
ENST0\ldots{} & 0 & 0 & 0 & 0 & 0 & 0\tabularnewline
ENST0\ldots{} & 0 & 0 & 0 & 0 & 0 & 0\tabularnewline
ENST0\ldots{} & 0 & 0 & 0 & 0 & 0 & 0\tabularnewline
ENST0\ldots{} & 0 & 0 & 0 & 0 & 0 & 0\tabularnewline
ENST0\ldots{} & 0 & 0 & 0 & 0 & 0 & 0\tabularnewline
ENST0\ldots{} & 0 & 0 & 0 & 0 & 0 & 0\tabularnewline
ENST0\ldots{} & 0 & 0 & 0 & 0 & 0 & 0\tabularnewline
ENST0\ldots{} & 0 & 0 & 0 & 0 & 0 & 0\tabularnewline
\ldots{} & \ldots{} & \ldots{} & \ldots{} & \ldots{} & \ldots{} & \ldots{}\tabularnewline
\bottomrule
\end{longtable}

Table \ref{tab:merged-ncrna}为表格merged ncrna概览。

\textbf{(对应文件为 \texttt{Figure+Table/merged-ncrna.csv})}

\begin{center}\begin{tcolorbox}[colback=gray!10, colframe=gray!50, width=0.9\linewidth, arc=1mm, boxrule=0.5pt]注:表格共有68492行7列,以下预览的表格可能省略部分数据;表格含有68492个唯一`target\_id'。
\end{tcolorbox}
\end{center}

\begin{longtable}[]{@{}lllllll@{}}
\caption{\label{tab:merged-ncrna}Merged ncrna}\tabularnewline
\toprule
targe\ldots{} & CT1-C\ldots{} & CT2-C\ldots{} & CT3-C\ldots{} & CUR1\_R & CUR2\_R & CUR3\_R\tabularnewline
\midrule
\endfirsthead
\toprule
targe\ldots{} & CT1-C\ldots{} & CT2-C\ldots{} & CT3-C\ldots{} & CUR1\_R & CUR2\_R & CUR3\_R\tabularnewline
\midrule
\endhead
ENST0\ldots{} & 0 & 0 & 0 & 0 & 0 & 0\tabularnewline
ENST0\ldots{} & 0 & 0 & 0 & 0 & 0 & 0\tabularnewline
ENST0\ldots{} & 0 & 0 & 0 & 0 & 0 & 0\tabularnewline
ENST0\ldots{} & 0 & 0 & 0 & 0 & 0 & 0\tabularnewline
ENST0\ldots{} & 2496.34 & 3798.65 & 11014.8 & 2845.62 & 3811.47 & 10121.3\tabularnewline
ENST0\ldots{} & 0 & 0 & 0 & 0 & 0 & 0\tabularnewline
ENST0\ldots{} & 0 & 0 & 0.166667 & 0 & 0.333333 & 0\tabularnewline
ENST0\ldots{} & 0 & 0 & 0 & 0 & 0 & 0\tabularnewline
ENST0\ldots{} & 0 & 0 & 0 & 0 & 0 & 0\tabularnewline
ENST0\ldots{} & 0 & 0 & 0 & 0 & 0 & 0\tabularnewline
ENST0\ldots{} & 0 & 0 & 0 & 0 & 0 & 0\tabularnewline
ENST0\ldots{} & 0 & 0 & 0 & 0 & 0 & 0\tabularnewline
ENST0\ldots{} & 0 & 0.5 & 0 & 0 & 0 & 0\tabularnewline
ENST0\ldots{} & 3 & 7 & 5 & 4 & 6 & 9\tabularnewline
ENST0\ldots{} & 0 & 0 & 0 & 0 & 0 & 0\tabularnewline
\ldots{} & \ldots{} & \ldots{} & \ldots{} & \ldots{} & \ldots{} & \ldots{}\tabularnewline
\bottomrule
\end{longtable}

\hypertarget{ux5408ux5e76-mrna-ux548c-ncrna-ux6570ux636e}{%
\subsubsection{合并 mRNA 和 ncRNA 数据}\label{ux5408ux5e76-mrna-ux548c-ncrna-ux6570ux636e}}

在这里,将 mRNA 数据和 ncRNA 数据按照列(样品)合并。

Table \ref{tab:merged-data-of-mRNA-and-ncRNA}为表格merged data of mRNA and ncRNA概览。

\textbf{(对应文件为 \texttt{Figure+Table/merged-data-of-mRNA-and-ncRNA.csv})}

\begin{center}\begin{tcolorbox}[colback=gray!10, colframe=gray!50, width=0.9\linewidth, arc=1mm, boxrule=0.5pt]注:表格共有275741行7列,以下预览的表格可能省略部分数据;表格含有275741个唯一`target\_id'。
\end{tcolorbox}
\end{center}

\begin{longtable}[]{@{}lllllll@{}}
\caption{\label{tab:merged-data-of-mRNA-and-ncRNA}Merged data of mRNA and ncRNA}\tabularnewline
\toprule
targe\ldots{} & CT1-C\ldots{} & CT2-C\ldots{} & CT3-C\ldots{} & CUR1\_R & CUR2\_R & CUR3\_R\tabularnewline
\midrule
\endfirsthead
\toprule
targe\ldots{} & CT1-C\ldots{} & CT2-C\ldots{} & CT3-C\ldots{} & CUR1\_R & CUR2\_R & CUR3\_R\tabularnewline
\midrule
\endhead
ENST0\ldots{} & 0 & 0 & 0 & 0 & 0 & 0\tabularnewline
ENST0\ldots{} & 0 & 0 & 0 & 0 & 0 & 0\tabularnewline
ENST0\ldots{} & 0 & 0 & 0 & 0 & 0 & 0\tabularnewline
ENST0\ldots{} & 0 & 0 & 0 & 0 & 0 & 0\tabularnewline
ENST0\ldots{} & 0 & 0 & 0 & 0 & 0 & 0\tabularnewline
ENST0\ldots{} & 0 & 0 & 0 & 0 & 0 & 0\tabularnewline
ENST0\ldots{} & 0 & 0 & 0 & 0 & 0 & 0\tabularnewline
ENST0\ldots{} & 0 & 0 & 0 & 0 & 0 & 0\tabularnewline
ENST0\ldots{} & 0 & 0 & 0 & 0 & 0 & 0\tabularnewline
ENST0\ldots{} & 0 & 0 & 0 & 0 & 0 & 0\tabularnewline
ENST0\ldots{} & 0 & 0 & 0 & 0 & 0 & 0\tabularnewline
ENST0\ldots{} & 0 & 0 & 0 & 0 & 0 & 0\tabularnewline
ENST0\ldots{} & 0 & 0 & 0 & 0 & 0 & 0\tabularnewline
ENST0\ldots{} & 0 & 0 & 0 & 0 & 0 & 0\tabularnewline
ENST0\ldots{} & 0 & 0 & 0 & 0 & 0 & 0\tabularnewline
\ldots{} & \ldots{} & \ldots{} & \ldots{} & \ldots{} & \ldots{} & \ldots{}\tabularnewline
\bottomrule
\end{longtable}

\hypertarget{ux4f7fux7528-biomart-ux83b7ux53d6ux57faux56e0ux6ce8ux91ca}{%
\subsubsection{\texorpdfstring{使用 \texttt{biomaRt} 获取基因注释}{使用 biomaRt 获取基因注释}}\label{ux4f7fux7528-biomart-ux83b7ux53d6ux57faux56e0ux6ce8ux91ca}}

使用 R包 \texttt{biomaRt} 获取 mRNA 和 ncRNA 的注释。

Table \ref{tab:annotation-mRNA}为表格annotation mRNA概览。

\textbf{(对应文件为 \texttt{Figure+Table/annotation-mRNA.tsv})}

\begin{center}\begin{tcolorbox}[colback=gray!10, colframe=gray!50, width=0.9\linewidth, arc=1mm, boxrule=0.5pt]注:表格共有275741行8列,以下预览的表格可能省略部分数据;表格含有275741个唯一`ensembl\_transcript\_id'。
\end{tcolorbox}
\end{center}

\begin{longtable}[]{@{}llllllll@{}}
\caption{\label{tab:annotation-mRNA}Annotation mRNA}\tabularnewline
\toprule
ensem\ldots\ldots1 & ensem\ldots\ldots2 & entre\ldots{} & hgnc\_\ldots{} & chrom\ldots{} & start\ldots{} & end\_p\ldots{} & descr\ldots{}\tabularnewline
\midrule
\endfirsthead
\toprule
ensem\ldots\ldots1 & ensem\ldots\ldots2 & entre\ldots{} & hgnc\_\ldots{} & chrom\ldots{} & start\ldots{} & end\_p\ldots{} & descr\ldots{}\tabularnewline
\midrule
\endhead
ENST0\ldots{} & ENSG0\ldots{} & 4535 & MT-ND1 & MT & 3307 & 4262 & mitoc\ldots{}\tabularnewline
ENST0\ldots{} & ENSG0\ldots{} & 4536 & MT-ND2 & MT & 4470 & 5511 & mitoc\ldots{}\tabularnewline
ENST0\ldots{} & ENSG0\ldots{} & 4512 & MT-CO1 & MT & 5904 & 7445 & mitoc\ldots{}\tabularnewline
ENST0\ldots{} & ENSG0\ldots{} & 4513 & MT-CO2 & MT & 7586 & 8269 & mitoc\ldots{}\tabularnewline
ENST0\ldots{} & ENSG0\ldots{} & 4509 & MT-ATP8 & MT & 8366 & 8572 & mitoc\ldots{}\tabularnewline
ENST0\ldots{} & ENSG0\ldots{} & 4508 & MT-ATP6 & MT & 8527 & 9207 & mitoc\ldots{}\tabularnewline
ENST0\ldots{} & ENSG0\ldots{} & 4514 & MT-CO3 & MT & 9207 & 9990 & mitoc\ldots{}\tabularnewline
ENST0\ldots{} & ENSG0\ldots{} & 4537 & MT-ND3 & MT & 10059 & 10404 & mitoc\ldots{}\tabularnewline
ENST0\ldots{} & ENSG0\ldots{} & 4539 & MT-ND4L & MT & 10470 & 10766 & mitoc\ldots{}\tabularnewline
ENST0\ldots{} & ENSG0\ldots{} & 4538 & MT-ND4 & MT & 10760 & 12137 & mitoc\ldots{}\tabularnewline
ENST0\ldots{} & ENSG0\ldots{} & 4540 & MT-ND5 & MT & 12337 & 14148 & mitoc\ldots{}\tabularnewline
ENST0\ldots{} & ENSG0\ldots{} & 4541 & MT-ND6 & MT & 14149 & 14673 & mitoc\ldots{}\tabularnewline
ENST0\ldots{} & ENSG0\ldots{} & 4519 & MT-CYB & MT & 14747 & 15887 & mitoc\ldots{}\tabularnewline
ENST0\ldots{} & ENSG0\ldots{} & 10272\ldots{} & & KI270\ldots{} & 4612 & 29626 &\tabularnewline
ENST0\ldots{} & ENSG0\ldots{} & 10272\ldots{} & & KI270\ldots{} & 4612 & 29626 &\tabularnewline
\ldots{} & \ldots{} & \ldots{} & \ldots{} & \ldots{} & \ldots{} & \ldots{} & \ldots{}\tabularnewline
\bottomrule
\end{longtable}

\hypertarget{diff}{%
\subsubsection{\texorpdfstring{使用 \texttt{limma} 差异分析}{使用 limma 差异分析}}\label{diff}}

Figure \ref{fig:filter-low-expression-genes}为图filter low expression genes概览。

\textbf{(对应文件为 \texttt{Figure+Table/filter-low-expression-genes.pdf})}

\def\@captype{figure}
\begin{center}
\includegraphics[width = 0.9\linewidth]{Figure+Table/filter-low-expression-genes.pdf}
\caption{Filter low expression genes}\label{fig:filter-low-expression-genes}
\end{center}

Figure \ref{fig:nomalize-genes-expression}为图nomalize genes expression概览。

\textbf{(对应文件为 \texttt{Figure+Table/nomalize-genes-expression.pdf})}

\def\@captype{figure}
\begin{center}
\includegraphics[width = 0.9\linewidth]{Figure+Table/nomalize-genes-expression.pdf}
\caption{Nomalize genes expression}\label{fig:nomalize-genes-expression}
\end{center}

线形回归拟合 \texttt{model.matrix(\textasciitilde{}\ 0\ +\ group)}\textsuperscript{\protect\hyperlink{ref-LimmaPowersDiRitchi2015}{2},\protect\hyperlink{ref-AGuideToCreaLawC2020}{8}},并统计检验。

Table \ref{tab:linear-regression-and-contrast-fit-results}为表格linear regression and contrast fit results概览。
根据 P.Value (0.05) 和 \textbar log2FC\textbar{} (0.3) 过滤得到结果。

\textbf{(对应文件为 \texttt{Figure+Table/linear-regression-and-contrast-fit-results.xlsx})}

\begin{center}\begin{tcolorbox}[colback=gray!10, colframe=gray!50, width=0.9\linewidth, arc=1mm, boxrule=0.5pt]注:表格共有7204行14列,以下预览的表格可能省略部分数据;表格含有7204个唯一`ensembl\_transcript\_id'。
\end{tcolorbox}
\end{center}

\begin{longtable}[]{@{}lllllllllll@{}}
\caption{\label{tab:linear-regression-and-contrast-fit-results}Linear regression and contrast fit results}\tabularnewline
\toprule
ensem\ldots\ldots1 & ensem\ldots\ldots2 & entre\ldots{} & hgnc\_\ldots{} & chrom\ldots{} & start\ldots{} & end\_p\ldots{} & descr\ldots{} & logFC & AveExpr & \ldots{}\tabularnewline
\midrule
\endfirsthead
\toprule
ensem\ldots\ldots1 & ensem\ldots\ldots2 & entre\ldots{} & hgnc\_\ldots{} & chrom\ldots{} & start\ldots{} & end\_p\ldots{} & descr\ldots{} & logFC & AveExpr & \ldots{}\tabularnewline
\midrule
\endhead
ENST0\ldots{} & ENSG0\ldots{} & 23657 & SLC7A11 & 4 & 13816\ldots{} & 13824\ldots{} & solut\ldots{} & 1.096\ldots{} & 6.693\ldots{} & \ldots{}\tabularnewline
ENST0\ldots{} & ENSG0\ldots{} & 2316 & FLNA & X & 15434\ldots{} & 15437\ldots{} & filam\ldots{} & -0.60\ldots{} & 10.38\ldots{} & \ldots{}\tabularnewline
ENST0\ldots{} & ENSG0\ldots{} & 1728 & NQO1 & 16 & 69706996 & 69726668 & NAD(P\ldots{} & 1.540\ldots{} & 7.379\ldots{} & \ldots{}\tabularnewline
ENST0\ldots{} & ENSG0\ldots{} & 3486 & IGFBP3 & 7 & 45912245 & 45921874 & insul\ldots{} & -1.77\ldots{} & 5.925\ldots{} & \ldots{}\tabularnewline
ENST0\ldots{} & ENSG0\ldots{} & 3880 & KRT19 & 17 & 41523617 & 41528308 & kerat\ldots{} & -1.40\ldots{} & 5.649\ldots{} & \ldots{}\tabularnewline
ENST0\ldots{} & ENSG0\ldots{} & NA & & 16 & 69709874 & 69710583 & novel\ldots{} & 1.519\ldots{} & 4.998\ldots{} & \ldots{}\tabularnewline
ENST0\ldots{} & ENSG0\ldots{} & 682 & BSG & 19 & 571277 & 583494 & basig\ldots{} & -1.89\ldots{} & 6.649\ldots{} & \ldots{}\tabularnewline
ENST0\ldots{} & ENSG0\ldots{} & 3488 & IGFBP5 & 2 & 21667\ldots{} & 21669\ldots{} & insul\ldots{} & -1.63\ldots{} & 6.279\ldots{} & \ldots{}\tabularnewline
ENST0\ldots{} & ENSG0\ldots{} & 128239 & IQGAP3 & HG251\ldots{} & 83962 & 131161 & IQ mo\ldots{} & -1.08\ldots{} & 5.548\ldots{} & \ldots{}\tabularnewline
ENST0\ldots{} & ENSG0\ldots{} & 128239 & IQGAP3 & 1 & 15652\ldots{} & 15657\ldots{} & IQ mo\ldots{} & -1.08\ldots{} & 5.548\ldots{} & \ldots{}\tabularnewline
ENST0\ldots{} & ENSG0\ldots{} & 1728 & NQO1 & 16 & 69706996 & 69726668 & NAD(P\ldots{} & 1.374\ldots{} & 5.529\ldots{} & \ldots{}\tabularnewline
ENST0\ldots{} & ENSG0\ldots{} & 4176 & MCM7 & 7 & 10009\ldots{} & 10010\ldots{} & minic\ldots{} & -1.04\ldots{} & 6.143\ldots{} & \ldots{}\tabularnewline
ENST0\ldots{} & ENSG0\ldots{} & 9537 & TP53I11 & 11 & 44885903 & 44951306 & tumor\ldots{} & -1.01\ldots{} & 5.866\ldots{} & \ldots{}\tabularnewline
ENST0\ldots{} & ENSG0\ldots{} & 1728 & NQO1 & 16 & 69706996 & 69726668 & NAD(P\ldots{} & 1.757\ldots{} & 3.238\ldots{} & \ldots{}\tabularnewline
ENST0\ldots{} & ENSG0\ldots{} & 994 & CDC25B & 20 & 3786772 & 3806121 & cell \ldots{} & -0.80\ldots{} & 6.245\ldots{} & \ldots{}\tabularnewline
\ldots{} & \ldots{} & \ldots{} & \ldots{} & \ldots{} & \ldots{} & \ldots{} & \ldots{} & \ldots{} & \ldots{} & \ldots{}\tabularnewline
\bottomrule
\end{longtable}

Figure \ref{fig:volcano-plot-of-differential-expression-genes}为图volcano plot of differential expression genes概览。

\textbf{(对应文件为 \texttt{Figure+Table/volcano-plot-of-differential-expression-genes.pdf})}

\def\@captype{figure}
\begin{center}
\includegraphics[width = 0.9\linewidth]{Figure+Table/volcano-plot-of-differential-expression-genes.pdf}
\caption{Volcano plot of differential expression genes}\label{fig:volcano-plot-of-differential-expression-genes}
\end{center}

\hypertarget{ux57faux56e0ux5171ux8868ux8fbeux5206ux6790}{%
\subsection{基因共表达分析}\label{ux57faux56e0ux5171ux8868ux8fbeux5206ux6790}}

\hypertarget{ux5efaux7acbux57faux56e0ux5171ux8868ux8fbeux6a21ux5757}{%
\subsubsection{建立基因共表达模块}\label{ux5efaux7acbux57faux56e0ux5171ux8868ux8fbeux6a21ux5757}}

将上述(\ref{diff}, Fig. \ref{fig:nomalize-genes-expression})
标准化过的\textbf{差异表达}基因数据
(Tab. \ref{tab:linear-regression-and-contrast-fit-results})
用于 WGCNA 分析\textsuperscript{\protect\hyperlink{ref-WgcnaAnRPacLangfe2008}{3}}。

Figure \ref{fig:cluster-sample}为图cluster sample概览。

\textbf{(对应文件为 \texttt{Figure+Table/cluster-sample.pdf})}

\def\@captype{figure}
\begin{center}
\includegraphics[width = 0.9\linewidth]{Figure+Table/cluster-sample.pdf}
\caption{Cluster sample}\label{fig:cluster-sample}
\end{center}

由于样本数量较少,没有明显合适的 `soft threshold'。这里,选择 `soft threshold' 为 3。

Figure \ref{fig:pick-soft-thereshold}为图pick soft thereshold概览。

\textbf{(对应文件为 \texttt{Figure+Table/pick-soft-thereshold.pdf})}

\def\@captype{figure}
\begin{center}
\includegraphics[width = 0.9\linewidth]{Figure+Table/pick-soft-thereshold.pdf}
\caption{Pick soft thereshold}\label{fig:pick-soft-thereshold}
\end{center}

Figure \ref{fig:gene-modules}为图gene modules概览。

\textbf{(对应文件为 \texttt{Figure+Table/gene-modules.pdf})}

\def\@captype{figure}
\begin{center}
\includegraphics[width = 0.9\linewidth]{Figure+Table/gene-modules.pdf}
\caption{Gene modules}\label{fig:gene-modules}
\end{center}

\hypertarget{ux5171ux8868ux8fbeux6a21ux5757ux548cux57faux56e0ux7684ux5173ux8054ux6027}{%
\subsubsection{共表达模块和基因的关联性}\label{ux5171ux8868ux8fbeux6a21ux5757ux548cux57faux56e0ux7684ux5173ux8054ux6027}}

计算 `gene module' 和 genes 之间的关联性(module membership)。

Table \ref{tab:module-membership}为表格module membership概览。

\textbf{(对应文件为 \texttt{Figure+Table/module-membership.xlsx})}

\begin{center}\begin{tcolorbox}[colback=gray!10, colframe=gray!50, width=0.9\linewidth, arc=1mm, boxrule=0.5pt]注:表格共有11853行15列,以下预览的表格可能省略部分数据;表格含有6009个唯一`gene'。
\end{tcolorbox}
\end{center}

\begin{longtable}[]{@{}llllllllllllll@{}}
\caption{\label{tab:module-membership}Module membership}\tabularnewline
\toprule
\begin{minipage}[b]{0.05\columnwidth}\raggedright
gene\strut
\end{minipage} & \begin{minipage}[b]{0.04\columnwidth}\raggedright
module\strut
\end{minipage} & \begin{minipage}[b]{0.04\columnwidth}\raggedright
cor\strut
\end{minipage} & \begin{minipage}[b]{0.04\columnwidth}\raggedright
pvalue\strut
\end{minipage} & \begin{minipage}[b]{0.05\columnwidth}\raggedright
-log2\ldots{}\strut
\end{minipage} & \begin{minipage}[b]{0.05\columnwidth}\raggedright
signi\ldots{}\strut
\end{minipage} & \begin{minipage}[b]{0.03\columnwidth}\raggedright
sign\strut
\end{minipage} & \begin{minipage}[b]{0.05\columnwidth}\raggedright
p.adjust\strut
\end{minipage} & \begin{minipage}[b]{0.05\columnwidth}\raggedright
ensem\ldots{}\strut
\end{minipage} & \begin{minipage}[b]{0.05\columnwidth}\raggedright
entre\ldots{}\strut
\end{minipage} & \begin{minipage}[b]{0.05\columnwidth}\raggedright
hgnc\_\ldots{}\strut
\end{minipage} & \begin{minipage}[b]{0.05\columnwidth}\raggedright
chrom\ldots{}\strut
\end{minipage} & \begin{minipage}[b]{0.05\columnwidth}\raggedright
start\ldots{}\strut
\end{minipage} & \begin{minipage}[b]{0.02\columnwidth}\raggedright
\ldots{}\strut
\end{minipage}\tabularnewline
\midrule
\endfirsthead
\toprule
\begin{minipage}[b]{0.05\columnwidth}\raggedright
gene\strut
\end{minipage} & \begin{minipage}[b]{0.04\columnwidth}\raggedright
module\strut
\end{minipage} & \begin{minipage}[b]{0.04\columnwidth}\raggedright
cor\strut
\end{minipage} & \begin{minipage}[b]{0.04\columnwidth}\raggedright
pvalue\strut
\end{minipage} & \begin{minipage}[b]{0.05\columnwidth}\raggedright
-log2\ldots{}\strut
\end{minipage} & \begin{minipage}[b]{0.05\columnwidth}\raggedright
signi\ldots{}\strut
\end{minipage} & \begin{minipage}[b]{0.03\columnwidth}\raggedright
sign\strut
\end{minipage} & \begin{minipage}[b]{0.05\columnwidth}\raggedright
p.adjust\strut
\end{minipage} & \begin{minipage}[b]{0.05\columnwidth}\raggedright
ensem\ldots{}\strut
\end{minipage} & \begin{minipage}[b]{0.05\columnwidth}\raggedright
entre\ldots{}\strut
\end{minipage} & \begin{minipage}[b]{0.05\columnwidth}\raggedright
hgnc\_\ldots{}\strut
\end{minipage} & \begin{minipage}[b]{0.05\columnwidth}\raggedright
chrom\ldots{}\strut
\end{minipage} & \begin{minipage}[b]{0.05\columnwidth}\raggedright
start\ldots{}\strut
\end{minipage} & \begin{minipage}[b]{0.02\columnwidth}\raggedright
\ldots{}\strut
\end{minipage}\tabularnewline
\midrule
\endhead
\begin{minipage}[t]{0.05\columnwidth}\raggedright
ENST0\ldots{}\strut
\end{minipage} & \begin{minipage}[t]{0.04\columnwidth}\raggedright
ME1\strut
\end{minipage} & \begin{minipage}[t]{0.04\columnwidth}\raggedright
-0.84\strut
\end{minipage} & \begin{minipage}[t]{0.04\columnwidth}\raggedright
0.0346\strut
\end{minipage} & \begin{minipage}[t]{0.05\columnwidth}\raggedright
4.853\ldots{}\strut
\end{minipage} & \begin{minipage}[t]{0.05\columnwidth}\raggedright
\textless{} 0.05\strut
\end{minipage} & \begin{minipage}[t]{0.03\columnwidth}\raggedright
*\strut
\end{minipage} & \begin{minipage}[t]{0.05\columnwidth}\raggedright
0.078\ldots{}\strut
\end{minipage} & \begin{minipage}[t]{0.05\columnwidth}\raggedright
ENSG0\ldots{}\strut
\end{minipage} & \begin{minipage}[t]{0.05\columnwidth}\raggedright
2101\strut
\end{minipage} & \begin{minipage}[t]{0.05\columnwidth}\raggedright
ESRRA\strut
\end{minipage} & \begin{minipage}[t]{0.05\columnwidth}\raggedright
11\strut
\end{minipage} & \begin{minipage}[t]{0.05\columnwidth}\raggedright
64305497\strut
\end{minipage} & \begin{minipage}[t]{0.02\columnwidth}\raggedright
\ldots{}\strut
\end{minipage}\tabularnewline
\begin{minipage}[t]{0.05\columnwidth}\raggedright
ENST0\ldots{}\strut
\end{minipage} & \begin{minipage}[t]{0.04\columnwidth}\raggedright
ME1\strut
\end{minipage} & \begin{minipage}[t]{0.04\columnwidth}\raggedright
0.89\strut
\end{minipage} & \begin{minipage}[t]{0.04\columnwidth}\raggedright
0.0166\strut
\end{minipage} & \begin{minipage}[t]{0.05\columnwidth}\raggedright
5.912\ldots{}\strut
\end{minipage} & \begin{minipage}[t]{0.05\columnwidth}\raggedright
\textless{} 0.05\strut
\end{minipage} & \begin{minipage}[t]{0.03\columnwidth}\raggedright
*\strut
\end{minipage} & \begin{minipage}[t]{0.05\columnwidth}\raggedright
0.061\ldots{}\strut
\end{minipage} & \begin{minipage}[t]{0.05\columnwidth}\raggedright
ENSG0\ldots{}\strut
\end{minipage} & \begin{minipage}[t]{0.05\columnwidth}\raggedright
64847\strut
\end{minipage} & \begin{minipage}[t]{0.05\columnwidth}\raggedright
SPATA20\strut
\end{minipage} & \begin{minipage}[t]{0.05\columnwidth}\raggedright
17\strut
\end{minipage} & \begin{minipage}[t]{0.05\columnwidth}\raggedright
50543058\strut
\end{minipage} & \begin{minipage}[t]{0.02\columnwidth}\raggedright
\ldots{}\strut
\end{minipage}\tabularnewline
\begin{minipage}[t]{0.05\columnwidth}\raggedright
ENST0\ldots{}\strut
\end{minipage} & \begin{minipage}[t]{0.04\columnwidth}\raggedright
ME2\strut
\end{minipage} & \begin{minipage}[t]{0.04\columnwidth}\raggedright
-0.85\strut
\end{minipage} & \begin{minipage}[t]{0.04\columnwidth}\raggedright
0.0319\strut
\end{minipage} & \begin{minipage}[t]{0.05\columnwidth}\raggedright
4.970\ldots{}\strut
\end{minipage} & \begin{minipage}[t]{0.05\columnwidth}\raggedright
\textless{} 0.05\strut
\end{minipage} & \begin{minipage}[t]{0.03\columnwidth}\raggedright
*\strut
\end{minipage} & \begin{minipage}[t]{0.05\columnwidth}\raggedright
0.076\ldots{}\strut
\end{minipage} & \begin{minipage}[t]{0.05\columnwidth}\raggedright
ENSG0\ldots{}\strut
\end{minipage} & \begin{minipage}[t]{0.05\columnwidth}\raggedright
64847\strut
\end{minipage} & \begin{minipage}[t]{0.05\columnwidth}\raggedright
SPATA20\strut
\end{minipage} & \begin{minipage}[t]{0.05\columnwidth}\raggedright
17\strut
\end{minipage} & \begin{minipage}[t]{0.05\columnwidth}\raggedright
50543058\strut
\end{minipage} & \begin{minipage}[t]{0.02\columnwidth}\raggedright
\ldots{}\strut
\end{minipage}\tabularnewline
\begin{minipage}[t]{0.05\columnwidth}\raggedright
ENST0\ldots{}\strut
\end{minipage} & \begin{minipage}[t]{0.04\columnwidth}\raggedright
ME1\strut
\end{minipage} & \begin{minipage}[t]{0.04\columnwidth}\raggedright
-0.82\strut
\end{minipage} & \begin{minipage}[t]{0.04\columnwidth}\raggedright
0.0471\strut
\end{minipage} & \begin{minipage}[t]{0.05\columnwidth}\raggedright
4.408\ldots{}\strut
\end{minipage} & \begin{minipage}[t]{0.05\columnwidth}\raggedright
\textless{} 0.05\strut
\end{minipage} & \begin{minipage}[t]{0.03\columnwidth}\raggedright
*\strut
\end{minipage} & \begin{minipage}[t]{0.05\columnwidth}\raggedright
0.088\ldots{}\strut
\end{minipage} & \begin{minipage}[t]{0.05\columnwidth}\raggedright
ENSG0\ldots{}\strut
\end{minipage} & \begin{minipage}[t]{0.05\columnwidth}\raggedright
57414\strut
\end{minipage} & \begin{minipage}[t]{0.05\columnwidth}\raggedright
RHBDD2\strut
\end{minipage} & \begin{minipage}[t]{0.05\columnwidth}\raggedright
7\strut
\end{minipage} & \begin{minipage}[t]{0.05\columnwidth}\raggedright
75842602\strut
\end{minipage} & \begin{minipage}[t]{0.02\columnwidth}\raggedright
\ldots{}\strut
\end{minipage}\tabularnewline
\begin{minipage}[t]{0.05\columnwidth}\raggedright
ENST0\ldots{}\strut
\end{minipage} & \begin{minipage}[t]{0.04\columnwidth}\raggedright
ME2\strut
\end{minipage} & \begin{minipage}[t]{0.04\columnwidth}\raggedright
0.89\strut
\end{minipage} & \begin{minipage}[t]{0.04\columnwidth}\raggedright
0.0171\strut
\end{minipage} & \begin{minipage}[t]{0.05\columnwidth}\raggedright
5.869\ldots{}\strut
\end{minipage} & \begin{minipage}[t]{0.05\columnwidth}\raggedright
\textless{} 0.05\strut
\end{minipage} & \begin{minipage}[t]{0.03\columnwidth}\raggedright
*\strut
\end{minipage} & \begin{minipage}[t]{0.05\columnwidth}\raggedright
0.062\ldots{}\strut
\end{minipage} & \begin{minipage}[t]{0.05\columnwidth}\raggedright
ENSG0\ldots{}\strut
\end{minipage} & \begin{minipage}[t]{0.05\columnwidth}\raggedright
57414\strut
\end{minipage} & \begin{minipage}[t]{0.05\columnwidth}\raggedright
RHBDD2\strut
\end{minipage} & \begin{minipage}[t]{0.05\columnwidth}\raggedright
7\strut
\end{minipage} & \begin{minipage}[t]{0.05\columnwidth}\raggedright
75842602\strut
\end{minipage} & \begin{minipage}[t]{0.02\columnwidth}\raggedright
\ldots{}\strut
\end{minipage}\tabularnewline
\begin{minipage}[t]{0.05\columnwidth}\raggedright
ENST0\ldots{}\strut
\end{minipage} & \begin{minipage}[t]{0.04\columnwidth}\raggedright
ME3\strut
\end{minipage} & \begin{minipage}[t]{0.04\columnwidth}\raggedright
0.85\strut
\end{minipage} & \begin{minipage}[t]{0.04\columnwidth}\raggedright
0.0305\strut
\end{minipage} & \begin{minipage}[t]{0.05\columnwidth}\raggedright
5.035\ldots{}\strut
\end{minipage} & \begin{minipage}[t]{0.05\columnwidth}\raggedright
\textless{} 0.05\strut
\end{minipage} & \begin{minipage}[t]{0.03\columnwidth}\raggedright
*\strut
\end{minipage} & \begin{minipage}[t]{0.05\columnwidth}\raggedright
0.075\ldots{}\strut
\end{minipage} & \begin{minipage}[t]{0.05\columnwidth}\raggedright
ENSG0\ldots{}\strut
\end{minipage} & \begin{minipage}[t]{0.05\columnwidth}\raggedright
3675\strut
\end{minipage} & \begin{minipage}[t]{0.05\columnwidth}\raggedright
ITGA3\strut
\end{minipage} & \begin{minipage}[t]{0.05\columnwidth}\raggedright
17\strut
\end{minipage} & \begin{minipage}[t]{0.05\columnwidth}\raggedright
50055968\strut
\end{minipage} & \begin{minipage}[t]{0.02\columnwidth}\raggedright
\ldots{}\strut
\end{minipage}\tabularnewline
\begin{minipage}[t]{0.05\columnwidth}\raggedright
ENST0\ldots{}\strut
\end{minipage} & \begin{minipage}[t]{0.04\columnwidth}\raggedright
ME1\strut
\end{minipage} & \begin{minipage}[t]{0.04\columnwidth}\raggedright
0.96\strut
\end{minipage} & \begin{minipage}[t]{0.04\columnwidth}\raggedright
0.002\strut
\end{minipage} & \begin{minipage}[t]{0.05\columnwidth}\raggedright
8.965\ldots{}\strut
\end{minipage} & \begin{minipage}[t]{0.05\columnwidth}\raggedright
\textless{} 0.05\strut
\end{minipage} & \begin{minipage}[t]{0.03\columnwidth}\raggedright
*\strut
\end{minipage} & \begin{minipage}[t]{0.05\columnwidth}\raggedright
0.039\ldots{}\strut
\end{minipage} & \begin{minipage}[t]{0.05\columnwidth}\raggedright
ENSG0\ldots{}\strut
\end{minipage} & \begin{minipage}[t]{0.05\columnwidth}\raggedright
2067\strut
\end{minipage} & \begin{minipage}[t]{0.05\columnwidth}\raggedright
ERCC1\strut
\end{minipage} & \begin{minipage}[t]{0.05\columnwidth}\raggedright
19\strut
\end{minipage} & \begin{minipage}[t]{0.05\columnwidth}\raggedright
45407334\strut
\end{minipage} & \begin{minipage}[t]{0.02\columnwidth}\raggedright
\ldots{}\strut
\end{minipage}\tabularnewline
\begin{minipage}[t]{0.05\columnwidth}\raggedright
ENST0\ldots{}\strut
\end{minipage} & \begin{minipage}[t]{0.04\columnwidth}\raggedright
ME2\strut
\end{minipage} & \begin{minipage}[t]{0.04\columnwidth}\raggedright
-0.82\strut
\end{minipage} & \begin{minipage}[t]{0.04\columnwidth}\raggedright
0.0435\strut
\end{minipage} & \begin{minipage}[t]{0.05\columnwidth}\raggedright
4.522\ldots{}\strut
\end{minipage} & \begin{minipage}[t]{0.05\columnwidth}\raggedright
\textless{} 0.05\strut
\end{minipage} & \begin{minipage}[t]{0.03\columnwidth}\raggedright
*\strut
\end{minipage} & \begin{minipage}[t]{0.05\columnwidth}\raggedright
0.086\ldots{}\strut
\end{minipage} & \begin{minipage}[t]{0.05\columnwidth}\raggedright
ENSG0\ldots{}\strut
\end{minipage} & \begin{minipage}[t]{0.05\columnwidth}\raggedright
2067\strut
\end{minipage} & \begin{minipage}[t]{0.05\columnwidth}\raggedright
ERCC1\strut
\end{minipage} & \begin{minipage}[t]{0.05\columnwidth}\raggedright
19\strut
\end{minipage} & \begin{minipage}[t]{0.05\columnwidth}\raggedright
45407334\strut
\end{minipage} & \begin{minipage}[t]{0.02\columnwidth}\raggedright
\ldots{}\strut
\end{minipage}\tabularnewline
\begin{minipage}[t]{0.05\columnwidth}\raggedright
ENST0\ldots{}\strut
\end{minipage} & \begin{minipage}[t]{0.04\columnwidth}\raggedright
ME3\strut
\end{minipage} & \begin{minipage}[t]{0.04\columnwidth}\raggedright
-0.86\strut
\end{minipage} & \begin{minipage}[t]{0.04\columnwidth}\raggedright
0.0287\strut
\end{minipage} & \begin{minipage}[t]{0.05\columnwidth}\raggedright
5.122\ldots{}\strut
\end{minipage} & \begin{minipage}[t]{0.05\columnwidth}\raggedright
\textless{} 0.05\strut
\end{minipage} & \begin{minipage}[t]{0.03\columnwidth}\raggedright
*\strut
\end{minipage} & \begin{minipage}[t]{0.05\columnwidth}\raggedright
0.073\ldots{}\strut
\end{minipage} & \begin{minipage}[t]{0.05\columnwidth}\raggedright
ENSG0\ldots{}\strut
\end{minipage} & \begin{minipage}[t]{0.05\columnwidth}\raggedright
2067\strut
\end{minipage} & \begin{minipage}[t]{0.05\columnwidth}\raggedright
ERCC1\strut
\end{minipage} & \begin{minipage}[t]{0.05\columnwidth}\raggedright
19\strut
\end{minipage} & \begin{minipage}[t]{0.05\columnwidth}\raggedright
45407334\strut
\end{minipage} & \begin{minipage}[t]{0.02\columnwidth}\raggedright
\ldots{}\strut
\end{minipage}\tabularnewline
\begin{minipage}[t]{0.05\columnwidth}\raggedright
ENST0\ldots{}\strut
\end{minipage} & \begin{minipage}[t]{0.04\columnwidth}\raggedright
ME2\strut
\end{minipage} & \begin{minipage}[t]{0.04\columnwidth}\raggedright
0.86\strut
\end{minipage} & \begin{minipage}[t]{0.04\columnwidth}\raggedright
0.0294\strut
\end{minipage} & \begin{minipage}[t]{0.05\columnwidth}\raggedright
5.088\ldots{}\strut
\end{minipage} & \begin{minipage}[t]{0.05\columnwidth}\raggedright
\textless{} 0.05\strut
\end{minipage} & \begin{minipage}[t]{0.03\columnwidth}\raggedright
*\strut
\end{minipage} & \begin{minipage}[t]{0.05\columnwidth}\raggedright
0.074\ldots{}\strut
\end{minipage} & \begin{minipage}[t]{0.05\columnwidth}\raggedright
ENSG0\ldots{}\strut
\end{minipage} & \begin{minipage}[t]{0.05\columnwidth}\raggedright
8635\strut
\end{minipage} & \begin{minipage}[t]{0.05\columnwidth}\raggedright
RNASET2\strut
\end{minipage} & \begin{minipage}[t]{0.05\columnwidth}\raggedright
6\strut
\end{minipage} & \begin{minipage}[t]{0.05\columnwidth}\raggedright
16692\ldots{}\strut
\end{minipage} & \begin{minipage}[t]{0.02\columnwidth}\raggedright
\ldots{}\strut
\end{minipage}\tabularnewline
\begin{minipage}[t]{0.05\columnwidth}\raggedright
ENST0\ldots{}\strut
\end{minipage} & \begin{minipage}[t]{0.04\columnwidth}\raggedright
ME1\strut
\end{minipage} & \begin{minipage}[t]{0.04\columnwidth}\raggedright
0.99\strut
\end{minipage} & \begin{minipage}[t]{0.04\columnwidth}\raggedright
3e-04\strut
\end{minipage} & \begin{minipage}[t]{0.05\columnwidth}\raggedright
11.70\ldots{}\strut
\end{minipage} & \begin{minipage}[t]{0.05\columnwidth}\raggedright
\textless{} 0.001\strut
\end{minipage} & \begin{minipage}[t]{0.03\columnwidth}\raggedright
**\strut
\end{minipage} & \begin{minipage}[t]{0.05\columnwidth}\raggedright
0.022\ldots{}\strut
\end{minipage} & \begin{minipage}[t]{0.05\columnwidth}\raggedright
ENSG0\ldots{}\strut
\end{minipage} & \begin{minipage}[t]{0.05\columnwidth}\raggedright
5010\strut
\end{minipage} & \begin{minipage}[t]{0.05\columnwidth}\raggedright
CLDN11\strut
\end{minipage} & \begin{minipage}[t]{0.05\columnwidth}\raggedright
3\strut
\end{minipage} & \begin{minipage}[t]{0.05\columnwidth}\raggedright
17041\ldots{}\strut
\end{minipage} & \begin{minipage}[t]{0.02\columnwidth}\raggedright
\ldots{}\strut
\end{minipage}\tabularnewline
\begin{minipage}[t]{0.05\columnwidth}\raggedright
ENST0\ldots{}\strut
\end{minipage} & \begin{minipage}[t]{0.04\columnwidth}\raggedright
ME2\strut
\end{minipage} & \begin{minipage}[t]{0.04\columnwidth}\raggedright
-0.9\strut
\end{minipage} & \begin{minipage}[t]{0.04\columnwidth}\raggedright
0.015\strut
\end{minipage} & \begin{minipage}[t]{0.05\columnwidth}\raggedright
6.058\ldots{}\strut
\end{minipage} & \begin{minipage}[t]{0.05\columnwidth}\raggedright
\textless{} 0.05\strut
\end{minipage} & \begin{minipage}[t]{0.03\columnwidth}\raggedright
*\strut
\end{minipage} & \begin{minipage}[t]{0.05\columnwidth}\raggedright
0.059\ldots{}\strut
\end{minipage} & \begin{minipage}[t]{0.05\columnwidth}\raggedright
ENSG0\ldots{}\strut
\end{minipage} & \begin{minipage}[t]{0.05\columnwidth}\raggedright
5010\strut
\end{minipage} & \begin{minipage}[t]{0.05\columnwidth}\raggedright
CLDN11\strut
\end{minipage} & \begin{minipage}[t]{0.05\columnwidth}\raggedright
3\strut
\end{minipage} & \begin{minipage}[t]{0.05\columnwidth}\raggedright
17041\ldots{}\strut
\end{minipage} & \begin{minipage}[t]{0.02\columnwidth}\raggedright
\ldots{}\strut
\end{minipage}\tabularnewline
\begin{minipage}[t]{0.05\columnwidth}\raggedright
ENST0\ldots{}\strut
\end{minipage} & \begin{minipage}[t]{0.04\columnwidth}\raggedright
ME3\strut
\end{minipage} & \begin{minipage}[t]{0.04\columnwidth}\raggedright
-0.95\strut
\end{minipage} & \begin{minipage}[t]{0.04\columnwidth}\raggedright
0.0038\strut
\end{minipage} & \begin{minipage}[t]{0.05\columnwidth}\raggedright
8.039\ldots{}\strut
\end{minipage} & \begin{minipage}[t]{0.05\columnwidth}\raggedright
\textless{} 0.05\strut
\end{minipage} & \begin{minipage}[t]{0.03\columnwidth}\raggedright
*\strut
\end{minipage} & \begin{minipage}[t]{0.05\columnwidth}\raggedright
0.046\ldots{}\strut
\end{minipage} & \begin{minipage}[t]{0.05\columnwidth}\raggedright
ENSG0\ldots{}\strut
\end{minipage} & \begin{minipage}[t]{0.05\columnwidth}\raggedright
5010\strut
\end{minipage} & \begin{minipage}[t]{0.05\columnwidth}\raggedright
CLDN11\strut
\end{minipage} & \begin{minipage}[t]{0.05\columnwidth}\raggedright
3\strut
\end{minipage} & \begin{minipage}[t]{0.05\columnwidth}\raggedright
17041\ldots{}\strut
\end{minipage} & \begin{minipage}[t]{0.02\columnwidth}\raggedright
\ldots{}\strut
\end{minipage}\tabularnewline
\begin{minipage}[t]{0.05\columnwidth}\raggedright
ENST0\ldots{}\strut
\end{minipage} & \begin{minipage}[t]{0.04\columnwidth}\raggedright
ME1\strut
\end{minipage} & \begin{minipage}[t]{0.04\columnwidth}\raggedright
-0.9\strut
\end{minipage} & \begin{minipage}[t]{0.04\columnwidth}\raggedright
0.014\strut
\end{minipage} & \begin{minipage}[t]{0.05\columnwidth}\raggedright
6.158\ldots{}\strut
\end{minipage} & \begin{minipage}[t]{0.05\columnwidth}\raggedright
\textless{} 0.05\strut
\end{minipage} & \begin{minipage}[t]{0.03\columnwidth}\raggedright
*\strut
\end{minipage} & \begin{minipage}[t]{0.05\columnwidth}\raggedright
0.058\ldots{}\strut
\end{minipage} & \begin{minipage}[t]{0.05\columnwidth}\raggedright
ENSG0\ldots{}\strut
\end{minipage} & \begin{minipage}[t]{0.05\columnwidth}\raggedright
84957\strut
\end{minipage} & \begin{minipage}[t]{0.05\columnwidth}\raggedright
RELT\strut
\end{minipage} & \begin{minipage}[t]{0.05\columnwidth}\raggedright
11\strut
\end{minipage} & \begin{minipage}[t]{0.05\columnwidth}\raggedright
73376399\strut
\end{minipage} & \begin{minipage}[t]{0.02\columnwidth}\raggedright
\ldots{}\strut
\end{minipage}\tabularnewline
\begin{minipage}[t]{0.05\columnwidth}\raggedright
ENST0\ldots{}\strut
\end{minipage} & \begin{minipage}[t]{0.04\columnwidth}\raggedright
ME2\strut
\end{minipage} & \begin{minipage}[t]{0.04\columnwidth}\raggedright
0.85\strut
\end{minipage} & \begin{minipage}[t]{0.04\columnwidth}\raggedright
0.0311\strut
\end{minipage} & \begin{minipage}[t]{0.05\columnwidth}\raggedright
5.006\ldots{}\strut
\end{minipage} & \begin{minipage}[t]{0.05\columnwidth}\raggedright
\textless{} 0.05\strut
\end{minipage} & \begin{minipage}[t]{0.03\columnwidth}\raggedright
*\strut
\end{minipage} & \begin{minipage}[t]{0.05\columnwidth}\raggedright
0.075\ldots{}\strut
\end{minipage} & \begin{minipage}[t]{0.05\columnwidth}\raggedright
ENSG0\ldots{}\strut
\end{minipage} & \begin{minipage}[t]{0.05\columnwidth}\raggedright
84957\strut
\end{minipage} & \begin{minipage}[t]{0.05\columnwidth}\raggedright
RELT\strut
\end{minipage} & \begin{minipage}[t]{0.05\columnwidth}\raggedright
11\strut
\end{minipage} & \begin{minipage}[t]{0.05\columnwidth}\raggedright
73376399\strut
\end{minipage} & \begin{minipage}[t]{0.02\columnwidth}\raggedright
\ldots{}\strut
\end{minipage}\tabularnewline
\begin{minipage}[t]{0.05\columnwidth}\raggedright
\ldots{}\strut
\end{minipage} & \begin{minipage}[t]{0.04\columnwidth}\raggedright
\ldots{}\strut
\end{minipage} & \begin{minipage}[t]{0.04\columnwidth}\raggedright
\ldots{}\strut
\end{minipage} & \begin{minipage}[t]{0.04\columnwidth}\raggedright
\ldots{}\strut
\end{minipage} & \begin{minipage}[t]{0.05\columnwidth}\raggedright
\ldots{}\strut
\end{minipage} & \begin{minipage}[t]{0.05\columnwidth}\raggedright
\ldots{}\strut
\end{minipage} & \begin{minipage}[t]{0.03\columnwidth}\raggedright
\ldots{}\strut
\end{minipage} & \begin{minipage}[t]{0.05\columnwidth}\raggedright
\ldots{}\strut
\end{minipage} & \begin{minipage}[t]{0.05\columnwidth}\raggedright
\ldots{}\strut
\end{minipage} & \begin{minipage}[t]{0.05\columnwidth}\raggedright
\ldots{}\strut
\end{minipage} & \begin{minipage}[t]{0.05\columnwidth}\raggedright
\ldots{}\strut
\end{minipage} & \begin{minipage}[t]{0.05\columnwidth}\raggedright
\ldots{}\strut
\end{minipage} & \begin{minipage}[t]{0.05\columnwidth}\raggedright
\ldots{}\strut
\end{minipage} & \begin{minipage}[t]{0.02\columnwidth}\raggedright
\ldots{}\strut
\end{minipage}\tabularnewline
\bottomrule
\end{longtable}

\hypertarget{tcf4-ux6240ux5728ux7684ux57faux56e0ux8868ux8fbeux6a21ux5757}{%
\subsubsection{TCF4 所在的基因表达模块}\label{tcf4-ux6240ux5728ux7684ux57faux56e0ux8868ux8fbeux6a21ux5757}}

确认TCF4 或 TCF-AS1 所在基因模块。

Table \ref{tab:TCF4-in-modules-memberships}为表格TCF4 in modules memberships概览。
TCF4 所在基因模块为 `ME1' 和 `ME2'(TCF4 和 TCF-AS1不存在共表达关系)。

\textbf{(对应文件为 \texttt{Figure+Table/TCF4-in-modules-memberships.csv})}

\begin{center}\begin{tcolorbox}[colback=gray!10, colframe=gray!50, width=0.9\linewidth, arc=1mm, boxrule=0.5pt]注:表格共有4行15列,以下预览的表格可能省略部分数据;表格含有3个唯一`gene'。
\end{tcolorbox}
\end{center}

\begin{longtable}[]{@{}llllllllllllll@{}}
\caption{\label{tab:TCF4-in-modules-memberships}TCF4 in modules memberships}\tabularnewline
\toprule
\begin{minipage}[b]{0.05\columnwidth}\raggedright
gene\strut
\end{minipage} & \begin{minipage}[b]{0.04\columnwidth}\raggedright
module\strut
\end{minipage} & \begin{minipage}[b]{0.04\columnwidth}\raggedright
cor\strut
\end{minipage} & \begin{minipage}[b]{0.04\columnwidth}\raggedright
pvalue\strut
\end{minipage} & \begin{minipage}[b]{0.05\columnwidth}\raggedright
-log2\ldots{}\strut
\end{minipage} & \begin{minipage}[b]{0.05\columnwidth}\raggedright
signi\ldots{}\strut
\end{minipage} & \begin{minipage}[b]{0.03\columnwidth}\raggedright
sign\strut
\end{minipage} & \begin{minipage}[b]{0.05\columnwidth}\raggedright
p.adjust\strut
\end{minipage} & \begin{minipage}[b]{0.05\columnwidth}\raggedright
ensem\ldots{}\strut
\end{minipage} & \begin{minipage}[b]{0.05\columnwidth}\raggedright
entre\ldots{}\strut
\end{minipage} & \begin{minipage}[b]{0.05\columnwidth}\raggedright
hgnc\_\ldots{}\strut
\end{minipage} & \begin{minipage}[b]{0.05\columnwidth}\raggedright
chrom\ldots{}\strut
\end{minipage} & \begin{minipage}[b]{0.05\columnwidth}\raggedright
start\ldots{}\strut
\end{minipage} & \begin{minipage}[b]{0.02\columnwidth}\raggedright
\ldots{}\strut
\end{minipage}\tabularnewline
\midrule
\endfirsthead
\toprule
\begin{minipage}[b]{0.05\columnwidth}\raggedright
gene\strut
\end{minipage} & \begin{minipage}[b]{0.04\columnwidth}\raggedright
module\strut
\end{minipage} & \begin{minipage}[b]{0.04\columnwidth}\raggedright
cor\strut
\end{minipage} & \begin{minipage}[b]{0.04\columnwidth}\raggedright
pvalue\strut
\end{minipage} & \begin{minipage}[b]{0.05\columnwidth}\raggedright
-log2\ldots{}\strut
\end{minipage} & \begin{minipage}[b]{0.05\columnwidth}\raggedright
signi\ldots{}\strut
\end{minipage} & \begin{minipage}[b]{0.03\columnwidth}\raggedright
sign\strut
\end{minipage} & \begin{minipage}[b]{0.05\columnwidth}\raggedright
p.adjust\strut
\end{minipage} & \begin{minipage}[b]{0.05\columnwidth}\raggedright
ensem\ldots{}\strut
\end{minipage} & \begin{minipage}[b]{0.05\columnwidth}\raggedright
entre\ldots{}\strut
\end{minipage} & \begin{minipage}[b]{0.05\columnwidth}\raggedright
hgnc\_\ldots{}\strut
\end{minipage} & \begin{minipage}[b]{0.05\columnwidth}\raggedright
chrom\ldots{}\strut
\end{minipage} & \begin{minipage}[b]{0.05\columnwidth}\raggedright
start\ldots{}\strut
\end{minipage} & \begin{minipage}[b]{0.02\columnwidth}\raggedright
\ldots{}\strut
\end{minipage}\tabularnewline
\midrule
\endhead
\begin{minipage}[t]{0.05\columnwidth}\raggedright
ENST0\ldots{}\strut
\end{minipage} & \begin{minipage}[t]{0.04\columnwidth}\raggedright
ME1\strut
\end{minipage} & \begin{minipage}[t]{0.04\columnwidth}\raggedright
0.85\strut
\end{minipage} & \begin{minipage}[t]{0.04\columnwidth}\raggedright
0.033\strut
\end{minipage} & \begin{minipage}[t]{0.05\columnwidth}\raggedright
4.921\ldots{}\strut
\end{minipage} & \begin{minipage}[t]{0.05\columnwidth}\raggedright
\textless{} 0.05\strut
\end{minipage} & \begin{minipage}[t]{0.03\columnwidth}\raggedright
*\strut
\end{minipage} & \begin{minipage}[t]{0.05\columnwidth}\raggedright
0.077\ldots{}\strut
\end{minipage} & \begin{minipage}[t]{0.05\columnwidth}\raggedright
ENSG0\ldots{}\strut
\end{minipage} & \begin{minipage}[t]{0.05\columnwidth}\raggedright
6925\strut
\end{minipage} & \begin{minipage}[t]{0.05\columnwidth}\raggedright
TCF4\strut
\end{minipage} & \begin{minipage}[t]{0.05\columnwidth}\raggedright
18\strut
\end{minipage} & \begin{minipage}[t]{0.05\columnwidth}\raggedright
55222185\strut
\end{minipage} & \begin{minipage}[t]{0.02\columnwidth}\raggedright
\ldots{}\strut
\end{minipage}\tabularnewline
\begin{minipage}[t]{0.05\columnwidth}\raggedright
ENST0\ldots{}\strut
\end{minipage} & \begin{minipage}[t]{0.04\columnwidth}\raggedright
ME1\strut
\end{minipage} & \begin{minipage}[t]{0.04\columnwidth}\raggedright
0.89\strut
\end{minipage} & \begin{minipage}[t]{0.04\columnwidth}\raggedright
0.0171\strut
\end{minipage} & \begin{minipage}[t]{0.05\columnwidth}\raggedright
5.869\ldots{}\strut
\end{minipage} & \begin{minipage}[t]{0.05\columnwidth}\raggedright
\textless{} 0.05\strut
\end{minipage} & \begin{minipage}[t]{0.03\columnwidth}\raggedright
*\strut
\end{minipage} & \begin{minipage}[t]{0.05\columnwidth}\raggedright
0.062\ldots{}\strut
\end{minipage} & \begin{minipage}[t]{0.05\columnwidth}\raggedright
ENSG0\ldots{}\strut
\end{minipage} & \begin{minipage}[t]{0.05\columnwidth}\raggedright
6925\strut
\end{minipage} & \begin{minipage}[t]{0.05\columnwidth}\raggedright
TCF4\strut
\end{minipage} & \begin{minipage}[t]{0.05\columnwidth}\raggedright
18\strut
\end{minipage} & \begin{minipage}[t]{0.05\columnwidth}\raggedright
55222185\strut
\end{minipage} & \begin{minipage}[t]{0.02\columnwidth}\raggedright
\ldots{}\strut
\end{minipage}\tabularnewline
\begin{minipage}[t]{0.05\columnwidth}\raggedright
ENST0\ldots{}\strut
\end{minipage} & \begin{minipage}[t]{0.04\columnwidth}\raggedright
ME2\strut
\end{minipage} & \begin{minipage}[t]{0.04\columnwidth}\raggedright
-0.86\strut
\end{minipage} & \begin{minipage}[t]{0.04\columnwidth}\raggedright
0.0292\strut
\end{minipage} & \begin{minipage}[t]{0.05\columnwidth}\raggedright
5.097\ldots{}\strut
\end{minipage} & \begin{minipage}[t]{0.05\columnwidth}\raggedright
\textless{} 0.05\strut
\end{minipage} & \begin{minipage}[t]{0.03\columnwidth}\raggedright
*\strut
\end{minipage} & \begin{minipage}[t]{0.05\columnwidth}\raggedright
0.074\ldots{}\strut
\end{minipage} & \begin{minipage}[t]{0.05\columnwidth}\raggedright
ENSG0\ldots{}\strut
\end{minipage} & \begin{minipage}[t]{0.05\columnwidth}\raggedright
6925\strut
\end{minipage} & \begin{minipage}[t]{0.05\columnwidth}\raggedright
TCF4\strut
\end{minipage} & \begin{minipage}[t]{0.05\columnwidth}\raggedright
18\strut
\end{minipage} & \begin{minipage}[t]{0.05\columnwidth}\raggedright
55222185\strut
\end{minipage} & \begin{minipage}[t]{0.02\columnwidth}\raggedright
\ldots{}\strut
\end{minipage}\tabularnewline
\begin{minipage}[t]{0.05\columnwidth}\raggedright
ENST0\ldots{}\strut
\end{minipage} & \begin{minipage}[t]{0.04\columnwidth}\raggedright
ME2\strut
\end{minipage} & \begin{minipage}[t]{0.04\columnwidth}\raggedright
-0.83\strut
\end{minipage} & \begin{minipage}[t]{0.04\columnwidth}\raggedright
0.0412\strut
\end{minipage} & \begin{minipage}[t]{0.05\columnwidth}\raggedright
4.601\ldots{}\strut
\end{minipage} & \begin{minipage}[t]{0.05\columnwidth}\raggedright
\textless{} 0.05\strut
\end{minipage} & \begin{minipage}[t]{0.03\columnwidth}\raggedright
*\strut
\end{minipage} & \begin{minipage}[t]{0.05\columnwidth}\raggedright
0.084\ldots{}\strut
\end{minipage} & \begin{minipage}[t]{0.05\columnwidth}\raggedright
ENSG0\ldots{}\strut
\end{minipage} & \begin{minipage}[t]{0.05\columnwidth}\raggedright
6925\strut
\end{minipage} & \begin{minipage}[t]{0.05\columnwidth}\raggedright
TCF4\strut
\end{minipage} & \begin{minipage}[t]{0.05\columnwidth}\raggedright
18\strut
\end{minipage} & \begin{minipage}[t]{0.05\columnwidth}\raggedright
55222185\strut
\end{minipage} & \begin{minipage}[t]{0.02\columnwidth}\raggedright
\ldots{}\strut
\end{minipage}\tabularnewline
\bottomrule
\end{longtable}

过滤 Tab. \ref{tab:module-membership} 数据,根据 p.adjust \textless{} 0.05, 以及 module 为 `ME1' 和 `ME2'。
随后,使用 \texttt{biomaRt} 获取基因对应的蛋白质的序列,同时,获取 TCF4 和 TCF-AS1 的序列;将这些序列转化为 `fasta'
格式(数量大于 500 个的 `fasta' 文件被切分)。

\hypertarget{ux4f7fux7528-catrapid-omics-v2.1-ux9884ux6d4b-rbps}{%
\subsubsection{使用 `catRAPID omics v2.1' 预测 RBPs}\label{ux4f7fux7528-catrapid-omics-v2.1-ux9884ux6d4b-rbps}}

\hypertarget{ux4e0aux4f20-catrapid-ux670dux52a1ux5668}{%
\paragraph{上传 catRAPID 服务器}\label{ux4e0aux4f20-catrapid-ux670dux52a1ux5668}}

catRAPID omics v2.1\textsuperscript{\protect\hyperlink{ref-ICatIRapidArmaos2021}{4}} 可同时计算多对 RNA 和蛋白质的结合(一次最多接受 500 个序列)。

结果可见于服务器:

\begin{itemize}
\tightlist
\item
  \url{http://crg-webservice.s3.amazonaws.com/submissions/2023-08/729560/output/index.html?unlock=c9f3fccec3}
\item
  \url{http://crg-webservice.s3.amazonaws.com/submissions/2023-08/729563/output/index.html?unlock=77c11a2b6a}
\item
  \url{http://crg-webservice.s3.amazonaws.com/submissions/2023-08/729565/output/index.html?unlock=6449ff7496}
\end{itemize}

\hypertarget{ux7ed3ux679cux6574ux7406}{%
\paragraph{结果整理}\label{ux7ed3ux679cux6574ux7406}}

Table \ref{tab:all-results-include-positive-or-negtive}为表格all results include positive or negtive概览。

\textbf{(对应文件为 \texttt{Figure+Table/all-results-include-positive-or-negtive.tsv})}

\begin{center}\begin{tcolorbox}[colback=gray!10, colframe=gray!50, width=0.9\linewidth, arc=1mm, boxrule=0.5pt]注:表格共有162666行13列,以下预览的表格可能省略部分数据;表格含有1291个唯一`Protein\_ID'。
\end{tcolorbox}
\end{center}

\begin{longtable}[]{@{}lllllllllll@{}}
\caption{\label{tab:all-results-include-positive-or-negtive}All results include positive or negtive}\tabularnewline
\toprule
Prote\ldots{} & RNA\_ID & rnaFr\ldots\ldots3 & rnaFr\ldots\ldots4 & Annot\ldots{} & Inter\ldots{} & Z\_score & RBP\_P\ldots{} & RNA\_B\ldots{} & numof\ldots\ldots10 & \ldots{}\tabularnewline
\midrule
\endfirsthead
\toprule
Prote\ldots{} & RNA\_ID & rnaFr\ldots\ldots3 & rnaFr\ldots\ldots4 & Annot\ldots{} & Inter\ldots{} & Z\_score & RBP\_P\ldots{} & RNA\_B\ldots{} & numof\ldots\ldots10 & \ldots{}\tabularnewline
\midrule
\endhead
ERCC5 & TCF4 & 6973 & 7306 & - & 119.58 & 1.47 & 0.43 & PF007\ldots{} & 2 & \ldots{}\tabularnewline
DLG3 & TCF4 & 6973 & 7306 & - & 115.51 & 1.34 & 0.5 & PF006\ldots{} & 7 & \ldots{}\tabularnewline
NRDC & TCF4 & 6973 & 7306 & - & 114.44 & 1.3 & 0.51 & PF161\ldots{} & 3 & \ldots{}\tabularnewline
INCENP & TCF4 & 6973 & 7306 & - & 112.79 & 1.25 & 0.63 & PF121\ldots{} & 2 & \ldots{}\tabularnewline
ERC1 & TCF4 & 6973 & 7306 & - & 107.5 & 1.08 & 0.29 & PF101\ldots{} & 2 & \ldots{}\tabularnewline
DLG3 & TCF4.AS1 & 251 & 302 & - & 105.42 & 5.79 & 0.5 & PF006\ldots{} & 7 & \ldots{}\tabularnewline
KIF2C & TCF4 & 2201 & 2534 & - & 105.19 & 1.01 & 0.41 & PF002\ldots{} & 2 & \ldots{}\tabularnewline
ERCC5 & TCF4.AS1 & 251 & 302 & - & 103.23 & 5.65 & 0.43 & PF007\ldots{} & 2 & \ldots{}\tabularnewline
GTSE1 & TCF4 & 6973 & 7306 & - & 103.09 & 0.94 & 0.41 & PF15259 & 1 & \ldots{}\tabularnewline
LIG1 & TCF4 & 6973 & 7306 & - & 102.39 & 0.92 & 0.43 & PF010\ldots{} & 3 & \ldots{}\tabularnewline
DLG3 & TCF4.AS1 & 276 & 327 & - & 102.28 & 5.59 & 0.5 & PF006\ldots{} & 7 & \ldots{}\tabularnewline
FILIP1L & TCF4 & 6973 & 7306 & - & 101.81 & 0.9 & 0.29 & PF09727 & 1 & \ldots{}\tabularnewline
TPX2 & TCF4 & 6973 & 7306 & - & 101.34 & 0.89 & 1 & PF122\ldots{} & 3 & \ldots{}\tabularnewline
KIF15 & TCF4 & 6973 & 7306 & - & 101.24 & 0.88 & 0.37 & PF002\ldots{} & 3 & \ldots{}\tabularnewline
NUP107 & TCF4 & 6973 & 7306 & - & 100.24 & 0.85 & 0.23 & PF04121 & 1 & \ldots{}\tabularnewline
\ldots{} & \ldots{} & \ldots{} & \ldots{} & \ldots{} & \ldots{} & \ldots{} & \ldots{} & \ldots{} & \ldots{} & \ldots{}\tabularnewline
\bottomrule
\end{longtable}

关于结果表格和各类评分的解释可以参考: \url{http://service.tartaglialab.com/static_files/shared/documentation_omics2.html}。

接下来,按照不同条件筛选结果:

\begin{itemize}
\tightlist
\item
  \texttt{RBP\_Propensity\ ==\ 1},
\item
  \texttt{Interaction\_Propensity\ \textgreater{}\ 0},
\item
  \texttt{numof.RNA.Binding\_Domains\_Instances\ \textgreater{}\ 0},
\item
  \texttt{numof.RNA\_Binding\_Motifs\_Instances\ \textgreater{}\ 0}
\end{itemize}

Figure \ref{fig:intersects-of-sets-of-filtering-conditions}为图intersects of sets of filtering conditions概览。

\textbf{(对应文件为 \texttt{Figure+Table/intersects-of-sets-of-filtering-conditions.pdf})}

\def\@captype{figure}
\begin{center}
\includegraphics[width = 0.9\linewidth]{Figure+Table/intersects-of-sets-of-filtering-conditions.pdf}
\caption{Intersects of sets of filtering conditions}\label{fig:intersects-of-sets-of-filtering-conditions}
\end{center}

可以发现,将四个数据集取交集,能得到包含少量数据的结果。

Table \ref{tab:top-candidates}为表格top candidates概览。

\textbf{(对应文件为 \texttt{Figure+Table/top-candidates.xlsx})}

\begin{center}\begin{tcolorbox}[colback=gray!10, colframe=gray!50, width=0.9\linewidth, arc=1mm, boxrule=0.5pt]注:表格共有47行13列,以下预览的表格可能省略部分数据;表格含有10个唯一`Protein\_ID'。
\end{tcolorbox}
\end{center}

\begin{longtable}[]{@{}lllllllllll@{}}
\caption{\label{tab:top-candidates}Top candidates}\tabularnewline
\toprule
Prote\ldots{} & RNA\_ID & rnaFr\ldots\ldots3 & rnaFr\ldots\ldots4 & Annot\ldots{} & Inter\ldots{} & Z\_score & RBP\_P\ldots{} & RNA\_B\ldots{} & numof\ldots\ldots10 & \ldots{}\tabularnewline
\midrule
\endfirsthead
\toprule
Prote\ldots{} & RNA\_ID & rnaFr\ldots\ldots3 & rnaFr\ldots\ldots4 & Annot\ldots{} & Inter\ldots{} & Z\_score & RBP\_P\ldots{} & RNA\_B\ldots{} & numof\ldots\ldots10 & \ldots{}\tabularnewline
\midrule
\endhead
PPIG & TCF4 & 6973 & 7306 & - & 70.11 & -0.1 & 1 & PF00160 & 1 & \ldots{}\tabularnewline
LARP4 & TCF4 & 2367 & 2700 & - & 46.53 & -0.85 & 1 & PF05383 & 1 & \ldots{}\tabularnewline
PPIG & TCF4 & 5313 & 5646 & - & 33.58 & -1.26 & 1 & PF00160 & 1 & \ldots{}\tabularnewline
LARP4 & TCF4 & 2325 & 2658 & - & 26.88 & -1.47 & 1 & PF05383 & 1 & \ldots{}\tabularnewline
LARP4 & TCF4 & 6973 & 7306 & - & 26.8 & -1.47 & 1 & PF05383 & 1 & \ldots{}\tabularnewline
CPEB2 & TCF4 & 3363 & 3696 & - & 25.41 & -1.52 & 1 & PF163\ldots{} & 3 & \ldots{}\tabularnewline
CPEB2 & TCF4 & 167 & 500 & - & 23.38 & -1.58 & 1 & PF163\ldots{} & 3 & \ldots{}\tabularnewline
CPEB2 & TCF4 & 3031 & 3364 & - & 19.65 & -1.7 & 1 & PF163\ldots{} & 3 & \ldots{}\tabularnewline
CPEB2 & TCF4 & 209 & 542 & - & 16.25 & -1.81 & 1 & PF163\ldots{} & 3 & \ldots{}\tabularnewline
PPIG & TCF4 & 5355 & 5688 & - & 8.39 & -2.06 & 1 & PF00160 & 1 & \ldots{}\tabularnewline
CPEB2 & TCF4 & 3321 & 3654 & - & 7.93 & -2.07 & 1 & PF163\ldots{} & 3 & \ldots{}\tabularnewline
PPIG & TCF4 & 5479 & 5812 & - & 5.62 & -2.15 & 1 & PF00160 & 1 & \ldots{}\tabularnewline
IGF2BP1 & TCF4 & 2325 & 2658 & - & 42.85 & -0.96 & 1 & PF000\ldots{} & 2 & \ldots{}\tabularnewline
IGF2BP1 & TCF4 & 2159 & 2492 & - & 37.92 & -1.12 & 1 & PF000\ldots{} & 2 & \ldots{}\tabularnewline
PCBP2 & TCF4 & 2159 & 2492 & - & 34.91 & -1.22 & 1 & PF00013 & 1 & \ldots{}\tabularnewline
\ldots{} & \ldots{} & \ldots{} & \ldots{} & \ldots{} & \ldots{} & \ldots{} & \ldots{} & \ldots{} & \ldots{} & \ldots{}\tabularnewline
\bottomrule
\end{longtable}

Tab. \ref{tab:top-candidates} 包含 RBPs 与 TCF4 结合或 TCF-AS1 结合的可能性,以下取它们的交集。

Figure \ref{fig:unique-candidate-of-RBP-binding-with-TCF4-and-TCF-AS1}为图unique candidate of RBP binding with TCF4 and TCF AS1概览。

\textbf{(对应文件为 \texttt{Figure+Table/unique-candidate-of-RBP-binding-with-TCF4-and-TCF-AS1.pdf})}

\def\@captype{figure}
\begin{center}
\includegraphics[width = 0.9\linewidth]{Figure+Table/unique-candidate-of-RBP-binding-with-TCF4-and-TCF-AS1.pdf}
\caption{Unique candidate of RBP binding with TCF4 and TCF AS1}\label{fig:unique-candidate-of-RBP-binding-with-TCF4-and-TCF-AS1}
\end{center}

Fig. \ref{fig:unique-candidate-of-RBP-binding-with-TCF4-and-TCF-AS1} 中包含的蛋白的 `symbol' 为:

\begin{verbatim}
## $TCF4
##  [1] "PPIG"    "LARP4"   "CPEB2"   "IGF2BP1" "PCBP2"   "YTHDF3"  "HNRNPH1" "KHDRBS3" "MBNL1"   "DAZAP1" 
## 
## $TCF4.AS1
## [1] "HNRNPH1"
\end{verbatim}

\hypertarget{dis}{%
\section{结论}\label{dis}}

将 RNA-seq 数据结合差异分析、基因共表达分析,并利用 catRAPID 工具预测 RBPs,成功筛选出一批 RBPs。
随后,根据 RBP 倾向(\texttt{RBP\_Propensity})、结合倾向(\texttt{Interaction\_Propensity})等条件筛选,
获得唯一 RBP:HNRNPH1

\hypertarget{bibliography}{%
\section*{Reference}\label{bibliography}}
\addcontentsline{toc}{section}{Reference}

\hypertarget{refs}{}
\begin{cslreferences}
\leavevmode\hypertarget{ref-NearOptimalPrBray2016}{}%
1. Bray, N. L., Pimentel, H., Melsted, P. \& Pachter, L. Near-optimal probabilistic rna-seq quantification. \emph{Nature Biotechnology} \textbf{34}, (2016).

\leavevmode\hypertarget{ref-LimmaPowersDiRitchi2015}{}%
2. Ritchie, M. E. \emph{et al.} Limma powers differential expression analyses for rna-sequencing and microarray studies. \emph{Nucleic Acids Research} \textbf{43}, e47 (2015).

\leavevmode\hypertarget{ref-WgcnaAnRPacLangfe2008}{}%
3. Langfelder, P. \& Horvath, S. WGCNA: An r package for weighted correlation network analysis. \emph{BMC Bioinformatics} \textbf{9}, (2008).

\leavevmode\hypertarget{ref-ICatIRapidArmaos2021}{}%
4. Armaos, A., Colantoni, A., Proietti, G., Rupert, J. \& Tartaglia, G. G. \textless i\textgreater cat\textless/i\textgreater RAPID\textless i\textgreater omics v2.0\textless/i\textgreater: Going deeper and wider in the prediction of proteinRNA interactions. \emph{Nucleic Acids Research} \textbf{49}, (2021).

\leavevmode\hypertarget{ref-RbpTstlIsATPeng2022}{}%
5. Peng, X. \emph{et al.} RBP-tstl is a two-stage transfer learning framework for genome-scale prediction of rna-binding proteins. \emph{Briefings in Bioinformatics} \textbf{23}, (2022).

\leavevmode\hypertarget{ref-IntegratingTheSuYu2019}{}%
6. Su, Y., Luo, Y., Zhao, X., Liu, Y. \& Peng, J. Integrating thermodynamic and sequence contexts improves protein-rna binding prediction. \emph{PLOS Computational Biology} \textbf{15}, (2019).

\leavevmode\hypertarget{ref-RckAccurateAOrenst2016}{}%
7. Orenstein, Y., Wang, Y. \& Berger, B. RCK: Accurate and efficient inference of sequence- and structure-based proteinRNA binding models from rnacompete data. \emph{Bioinformatics} \textbf{32}, (2016).

\leavevmode\hypertarget{ref-AGuideToCreaLawC2020}{}%
8. Law, C. W. \emph{et al.} A guide to creating design matrices for gene expression experiments. \emph{F1000Research} \textbf{9}, 1444 (2020).
\end{cslreferences}

\end{document}
