% Options for packages loaded elsewhere
\PassOptionsToPackage{unicode}{hyperref}
\PassOptionsToPackage{hyphens}{url}
%
\documentclass[
]{article}
\usepackage{lmodern}
\usepackage{amssymb,amsmath}
\usepackage{ifxetex,ifluatex}
\ifnum 0\ifxetex 1\fi\ifluatex 1\fi=0 % if pdftex
  \usepackage[T1]{fontenc}
  \usepackage[utf8]{inputenc}
  \usepackage{textcomp} % provide euro and other symbols
\else % if luatex or xetex
  \usepackage{unicode-math}
  \defaultfontfeatures{Scale=MatchLowercase}
  \defaultfontfeatures[\rmfamily]{Ligatures=TeX,Scale=1}
\fi
% Use upquote if available, for straight quotes in verbatim environments
\IfFileExists{upquote.sty}{\usepackage{upquote}}{}
\IfFileExists{microtype.sty}{% use microtype if available
  \usepackage[]{microtype}
  \UseMicrotypeSet[protrusion]{basicmath} % disable protrusion for tt fonts
}{}
\makeatletter
\@ifundefined{KOMAClassName}{% if non-KOMA class
  \IfFileExists{parskip.sty}{%
    \usepackage{parskip}
  }{% else
    \setlength{\parindent}{0pt}
    \setlength{\parskip}{6pt plus 2pt minus 1pt}}
}{% if KOMA class
  \KOMAoptions{parskip=half}}
\makeatother
\usepackage{xcolor}
\IfFileExists{xurl.sty}{\usepackage{xurl}}{} % add URL line breaks if available
\IfFileExists{bookmark.sty}{\usepackage{bookmark}}{\usepackage{hyperref}}
\hypersetup{
  hidelinks,
  pdfcreator={LaTeX via pandoc}}
\urlstyle{same} % disable monospaced font for URLs
\usepackage[margin=1in]{geometry}
\usepackage{longtable,booktabs}
% Correct order of tables after \paragraph or \subparagraph
\usepackage{etoolbox}
\makeatletter
\patchcmd\longtable{\par}{\if@noskipsec\mbox{}\fi\par}{}{}
\makeatother
% Allow footnotes in longtable head/foot
\IfFileExists{footnotehyper.sty}{\usepackage{footnotehyper}}{\usepackage{footnote}}
\makesavenoteenv{longtable}
\usepackage{graphicx}
\makeatletter
\def\maxwidth{\ifdim\Gin@nat@width>\linewidth\linewidth\else\Gin@nat@width\fi}
\def\maxheight{\ifdim\Gin@nat@height>\textheight\textheight\else\Gin@nat@height\fi}
\makeatother
% Scale images if necessary, so that they will not overflow the page
% margins by default, and it is still possible to overwrite the defaults
% using explicit options in \includegraphics[width, height, ...]{}
\setkeys{Gin}{width=\maxwidth,height=\maxheight,keepaspectratio}
% Set default figure placement to htbp
\makeatletter
\def\fps@figure{htbp}
\makeatother
\setlength{\emergencystretch}{3em} % prevent overfull lines
\providecommand{\tightlist}{%
  \setlength{\itemsep}{0pt}\setlength{\parskip}{0pt}}
\setcounter{secnumdepth}{5}
\usepackage{caption} \captionsetup{font={footnotesize},width=6in} \renewcommand{\dblfloatpagefraction}{.9} \makeatletter \renewenvironment{figure} {\def\@captype{figure}} \makeatother \@ifundefined{Shaded}{\newenvironment{Shaded}} \@ifundefined{snugshade}{\newenvironment{snugshade}} \renewenvironment{Shaded}{\begin{snugshade}}{\end{snugshade}} \definecolor{shadecolor}{RGB}{230,230,230} \usepackage{xeCJK} \usepackage{setspace} \setstretch{1.3} \usepackage{tcolorbox} \setcounter{secnumdepth}{4} \setcounter{tocdepth}{4} \usepackage{wallpaper} \usepackage[absolute]{textpos} \tcbuselibrary{breakable} \renewenvironment{Shaded} {\begin{tcolorbox}[colback = gray!10, colframe = gray!40, width = 16cm, arc = 1mm, auto outer arc, title = {R input}]} {\end{tcolorbox}} \usepackage{titlesec} \titleformat{\paragraph} {\fontsize{10pt}{0pt}\bfseries} {\arabic{section}.\arabic{subsection}.\arabic{subsubsection}.\arabic{paragraph}} {1em} {} []
\newlength{\cslhangindent}
\setlength{\cslhangindent}{1.5em}
\newenvironment{cslreferences}%
  {}%
  {\par}

\author{}
\date{\vspace{-2.5em}}

\begin{document}

\begin{titlepage} \newgeometry{top=7.5cm}
\ThisCenterWallPaper{1.12}{~/outline/lixiao//cover_page.pdf}
\begin{center} \textbf{\Huge 乙酰化酶分析筛选}
\vspace{4em} \begin{textblock}{10}(3,5.9) \huge
\textbf{\textcolor{white}{2024-04-17}}
\end{textblock} \begin{textblock}{10}(3,7.3)
\Large \textcolor{black}{LiChuang Huang}
\end{textblock} \begin{textblock}{10}(3,11.3)
\Large \textcolor{black}{@立效研究院}
\end{textblock} \end{center} \end{titlepage}
\restoregeometry

\pagenumbering{roman}

\tableofcontents

\listoffigures

\listoftables

\newpage

\pagenumbering{arabic}

\hypertarget{abstract}{%
\section{摘要}\label{abstract}}

\hypertarget{ux9700ux6c42}{%
\subsection{需求}\label{ux9700ux6c42}}

利用生物信息学分析结合已有文献资料,筛选并验证与XX相关的乙酰化酶AA

具体要求为: 利用开源数据库,筛选心肌梗死机体的心脏细胞中关键差异表达基因XX以及与乙酰化相关酶基因的关联性。

\begin{enumerate}
\def\labelenumi{\arabic{enumi}.}
\tightlist
\item
  因客户之前所做基因为FKBP5,故XX初步定为FKBP5。假设FKBP5在心肌梗死机体心肌细胞中高表达,抑制FKBP5后可缓解心肌梗死。
\item
  乙酰化酶AA备选:去乙酰化酶sirtuin 1(SIRT1)可以直接与FKBP5相互作用。
\item
  若方案中的AA选择为HDAC6(客户之前发表过LncRNA NORAD-HDAC6-H3K9 -VEGF),那么方案中的XX选择不一定非要是FKBP5,若有创新点的更好的基因也可。
\end{enumerate}

\hypertarget{ux7ed3ux679c}{%
\subsection{结果}\label{ux7ed3ux679c}}

\begin{itemize}
\tightlist
\item
  结合数据库 MI 靶点 和 MI 小鼠数据集获取一批 MI 基因 Fig. \ref{fig:Intersection-of-MI-DEGs-with-MI-targets}
\item
  从 epiFactor 数据库获取乙酰酶 (CoA) (Tab. \ref{tab:All-protein-of-CoA}) ,筛选了 MI 中为差异表达的 CoA (Fig. \ref{fig:Intersection-of-All-CoA-with-MI-DEGs}) 。
\item
  根据 Fig. \ref{fig:Intersection-of-MI-DEGs-with-MI-targets}
  和 Fig. \ref{fig:Intersection-of-All-CoA-with-MI-DEGs}
  建立 PPI 网络 (有实验基础的蛋白物理直接互作) ,见 Fig. \ref{fig:Filtered-and-formated-PPI-network}
\item
  筛选 CoA 与 DEGs 显著关联的组合,Fig. \ref{fig:MI-correlation-heatmap},Tab. \ref{tab:MI-significant-correlation}
\item
  筛选上述关系:存在 PPI 关联且关联分析显著的组合 Tab. \ref{tab:PPI-interact-and-significant-correlated-in-MI}
\item
  将上述 DEGs GO 富集分析,Fig. \ref{fig:GO-enrichment},BP 结果指向了 MI。
\item
  建立 CoA-XX-pathways 网络关系图,Fig. \ref{fig:CoA-XX-GOpathways},
  数据见 Tab. \ref{tab:All-candidates-and-enriched-GO-BP-pathways}。
\item
  最后,推荐 CoA-XX 组合为:

  \begin{itemize}
  \tightlist
  \item
    CoA:BRCA1, DEG:FLNA
  \item
    CoA:HDAC9, DEG:PIK3CG
  \item
    以上 DEG 相关 GO 通路:cardiac muscle contraction; coagulation; muscle system process; regulation of body fluid levels; striated muscle contraction; wound healing
  \end{itemize}
\item
  其它候选见 Tab. \ref{tab:All-candidates-and-enriched-GO-BP-pathways}
\end{itemize}

注:

\begin{itemize}
\tightlist
\item
  FKBP5 (Fkbp5) 在 MI 中属于显著高表达,见 Tab. \ref{tab:Fkbp5-expression}。
\item
  FKBP5 在 Fig. \ref{fig:Intersection-of-MI-DEGs-with-MI-targets} 被筛离。
\item
  尝试单独建立 PPI,未发现 SIRT1 与 FKBP5 的直接结合作用。
\end{itemize}

\hypertarget{introduction}{%
\section{前言}\label{introduction}}

\hypertarget{methods}{%
\section{材料和方法}\label{methods}}

\hypertarget{ux6750ux6599}{%
\subsection{材料}\label{ux6750ux6599}}

All used GEO expression data and their design:

\begin{itemize}
\tightlist
\item
  \textbf{GSE236374}: Nine 8-week-old male C57BL/6JR mice were included in the experiment. The experiment was divided into 3 groups. Each group contained 3 mice, 2 groups of which required surgery to make models, called\ldots{}
\end{itemize}

\hypertarget{ux65b9ux6cd5}{%
\subsection{方法}\label{ux65b9ux6cd5}}

Mainly used method:

\begin{itemize}
\tightlist
\item
  The \texttt{biomart} was used for mapping genes between organism (e.g., mgi\_symbol to hgnc\_symbol)\textsuperscript{\protect\hyperlink{ref-MappingIdentifDurinc2009}{1}}.
\item
  R package \texttt{ClusterProfiler} used for gene enrichment analysis\textsuperscript{\protect\hyperlink{ref-ClusterprofilerWuTi2021}{2}}.
\item
  Database \texttt{EpiFactors} used for screening epigenetic regulators\textsuperscript{\protect\hyperlink{ref-Epifactors2022Maraku2023}{3}}.
\item
  GEO \url{https://www.ncbi.nlm.nih.gov/geo/} used for expression dataset aquisition.
\item
  Databses of \texttt{DisGeNet}, \texttt{GeneCards}, \texttt{PharmGKB} used for collating disease related targets\textsuperscript{\protect\hyperlink{ref-TheDisgenetKnPinero2019}{4}--\protect\hyperlink{ref-PharmgkbAWorBarbar2018}{6}}.
\item
  The Human Gene Database \texttt{GeneCards} used for disease related genes prediction\textsuperscript{\protect\hyperlink{ref-TheGenecardsSStelze2016}{5}}.
\item
  R package \texttt{ClusterProfiler} used for GSEA enrichment\textsuperscript{\protect\hyperlink{ref-ClusterprofilerWuTi2021}{2}}.
\item
  R package \texttt{Limma} and \texttt{edgeR} used for differential expression analysis\textsuperscript{\protect\hyperlink{ref-LimmaPowersDiRitchi2015}{7},\protect\hyperlink{ref-EdgerDifferenChen}{8}}.
\item
  R package \texttt{STEINGdb} used for PPI network construction\textsuperscript{\protect\hyperlink{ref-TheStringDataSzklar2021}{9},\protect\hyperlink{ref-CytohubbaIdenChin2014}{10}}.
\item
  R version 4.3.2 (2023-10-31); Other R packages (eg., \texttt{dplyr} and \texttt{ggplot2}) used for statistic analysis or data visualization.
\end{itemize}

\hypertarget{results}{%
\section{分析结果}\label{results}}

\hypertarget{dis}{%
\section{结论}\label{dis}}

\hypertarget{workflow}{%
\section{附:分析流程}\label{workflow}}

\hypertarget{mi-targets}{%
\subsection{MI targets}\label{mi-targets}}

使用以下合集:

Figure \ref{fig:Overall-targets-number-of-datasets} (下方图) 为图Overall targets number of datasets概览。

\textbf{(对应文件为 \texttt{Figure+Table/Overall-targets-number-of-datasets.pdf})}

\def\@captype{figure}
\begin{center}
\includegraphics[width = 0.9\linewidth]{Figure+Table/Overall-targets-number-of-datasets.pdf}
\caption{Overall targets number of datasets}\label{fig:Overall-targets-number-of-datasets}
\end{center}

\begin{center}\begin{tcolorbox}[colback=gray!10, colframe=gray!50, width=0.9\linewidth, arc=1mm, boxrule=0.5pt]
\textbf{
The GeneCards data was obtained by querying
:}

\vspace{0.5em}

    myocardial infarction

\vspace{2em}


\textbf{
Restrict (with quotes)
:}

\vspace{0.5em}

    TRUE

\vspace{2em}


\textbf{
Filtering by Score:
:}

\vspace{0.5em}

    Score > 5

\vspace{2em}
\end{tcolorbox}
\end{center}

Table \ref{tab:GeneCards-used-data} (下方表格) 为表格GeneCards used data概览。

\textbf{(对应文件为 \texttt{Figure+Table/GeneCards-used-data.xlsx})}

\begin{center}\begin{tcolorbox}[colback=gray!10, colframe=gray!50, width=0.9\linewidth, arc=1mm, boxrule=0.5pt]注:表格共有567行7列,以下预览的表格可能省略部分数据;含有567个唯一`Symbol'。
\end{tcolorbox}
\end{center}

\begin{longtable}[]{@{}lllllll@{}}
\caption{\label{tab:GeneCards-used-data}GeneCards used data}\tabularnewline
\toprule
Symbol & Description & Category & UniProt\_ID & GIFtS & GC\_id & Score\tabularnewline
\midrule
\endfirsthead
\toprule
Symbol & Description & Category & UniProt\_ID & GIFtS & GC\_id & Score\tabularnewline
\midrule
\endhead
ACE & Angiotensi\ldots{} & Protein Co\ldots{} & P12821 & 60 & GC17P063477 & 75.08\tabularnewline
MIAT & Myocardial\ldots{} & RNA Gene (\ldots{} & & 32 & GC22P026646 & 71.09\tabularnewline
F7 & Coagulatio\ldots{} & Protein Co\ldots{} & P08709 & 56 & GC13P113105 & 54.33\tabularnewline
ITGB3 & Integrin S\ldots{} & Protein Co\ldots{} & P05106 & 61 & GC17P112532 & 48.15\tabularnewline
LTA & Lymphotoxi\ldots{} & Protein Co\ldots{} & P01374 & 52 & GC06P134818 & 44.63\tabularnewline
OLR1 & Oxidized L\ldots{} & Protein Co\ldots{} & P78380 & 51 & GC12M029495 & 44.32\tabularnewline
PLAT & Plasminoge\ldots{} & Protein Co\ldots{} & P00750 & 58 & GC08M042174 & 39.78\tabularnewline
MCI2 & Myocardial\ldots{} & Genetic Locus & & 4 & GC13U900611 & 39.39\tabularnewline
F13A1 & Coagulatio\ldots{} & Protein Co\ldots{} & P00488 & 56 & GC06M006144 & 39.35\tabularnewline
CDKN2B-AS1 & CDKN2B Ant\ldots{} & RNA Gene (\ldots{} & & 31 & GC09P021994 & 39.31\tabularnewline
LGALS2 & Galectin 2 & Protein Co\ldots{} & P05162 & 47 & GC22M037570 & 38.25\tabularnewline
MEF2A & Myocyte En\ldots{} & Protein Co\ldots{} & Q02078 & 54 & GC15P099565 & 38.14\tabularnewline
MIR499A & MicroRNA 499a & RNA Gene (\ldots{} & & 29 & GC20P034990 & 37.65\tabularnewline
ESR1 & Estrogen R\ldots{} & Protein Co\ldots{} & P03372 & 62 & GC06P151656 & 37.58\tabularnewline
MIR208B & MicroRNA 208b & RNA Gene (\ldots{} & & 27 & GC14M023417 & 35.34\tabularnewline
\ldots{} & \ldots{} & \ldots{} & \ldots{} & \ldots{} & \ldots{} & \ldots{}\tabularnewline
\bottomrule
\end{longtable}

\hypertarget{MI}{%
\subsection{MI mice DEGs}\label{MI}}

\hypertarget{ux6570ux636eux6765ux6e90}{%
\subsubsection{数据来源}\label{ux6570ux636eux6765ux6e90}}

\begin{center}\begin{tcolorbox}[colback=gray!10, colframe=gray!50, width=0.9\linewidth, arc=1mm, boxrule=0.5pt]
\textbf{
Data Source ID
:}

\vspace{0.5em}

    GSE236374

\vspace{2em}


\textbf{
data\_processing
:}

\vspace{0.5em}

    Raw reads were trimmed adaptor sequences and removed
low-quality reads using TrimGalore with default settings

\vspace{2em}


\textbf{
data\_processing.1
:}

\vspace{0.5em}

    Trimmed reads were aligned to the mm10 reference genome
by STAR with default settings

\vspace{2em}


\textbf{
data\_processing.2
:}

\vspace{0.5em}

    Read count extraction were performed using
FeatureCounts

\vspace{2em}


\textbf{
data\_processing.3
:}

\vspace{0.5em}

    Assembly: mm10

\vspace{2em}


\textbf{
(Others)
:}

\vspace{0.5em}

    ...

\vspace{2em}
\end{tcolorbox}
\end{center}

\textbf{(上述信息框内容已保存至 \texttt{Figure+Table/GSE236374-content})}

\hypertarget{ux5deeux5f02ux5206ux6790}{%
\subsubsection{差异分析}\label{ux5deeux5f02ux5206ux6790}}

\begin{itemize}
\tightlist
\item
  MI-7d (7 day) vs Control
\end{itemize}

Figure \ref{fig:MI-MI-7d-vs-MI-sham-DEGs} (下方图) 为图MI MI 7d vs MI sham DEGs概览。

\textbf{(对应文件为 \texttt{Figure+Table/MI-MI-7d-vs-MI-sham-DEGs.pdf})}

\def\@captype{figure}
\begin{center}
\includegraphics[width = 0.9\linewidth]{Figure+Table/MI-MI-7d-vs-MI-sham-DEGs.pdf}
\caption{MI MI 7d vs MI sham DEGs}\label{fig:MI-MI-7d-vs-MI-sham-DEGs}
\end{center}
\begin{center}\begin{tcolorbox}[colback=gray!10, colframe=gray!50, width=0.9\linewidth, arc=1mm, boxrule=0.5pt]
\textbf{
adj.P.Val cut-off
:}

\vspace{0.5em}

    0.05

\vspace{2em}


\textbf{
Log2(FC) cut-off
:}

\vspace{0.5em}

    1

\vspace{2em}
\end{tcolorbox}
\end{center}

\textbf{(上述信息框内容已保存至 \texttt{Figure+Table/MI-MI-7d-vs-MI-sham-DEGs-content})}

Table \ref{tab:MI-data-MI-7d-vs-MI-sham-DEGs} (下方表格) 为表格MI data MI 7d vs MI sham DEGs概览。

\textbf{(对应文件为 \texttt{Figure+Table/MI-data-MI-7d-vs-MI-sham-DEGs.csv})}

\begin{center}\begin{tcolorbox}[colback=gray!10, colframe=gray!50, width=0.9\linewidth, arc=1mm, boxrule=0.5pt]注:表格共有5724行8列,以下预览的表格可能省略部分数据;含有5724个唯一`Genesymbol'。
\end{tcolorbox}
\end{center}
\begin{center}\begin{tcolorbox}[colback=gray!10, colframe=gray!50, width=0.9\linewidth, arc=1mm, boxrule=0.5pt]\begin{enumerate}\tightlist
\item logFC:  estimate of the log2-fold-change corresponding to the effect or contrast (for ‘topTableF’ there may be several columns of log-fold-changes)
\item AveExpr:  average log2-expression for the probe over all arrays and channels, same as ‘Amean’ in the ‘MarrayLM’ object
\item t:  moderated t-statistic (omitted for ‘topTableF’)
\item P.Value:  raw p-value
\item B:  log-odds that the gene is differentially expressed (omitted for ‘topTreat’)
\end{enumerate}\end{tcolorbox}
\end{center}

\begin{longtable}[]{@{}llllllll@{}}
\caption{\label{tab:MI-data-MI-7d-vs-MI-sham-DEGs}MI data MI 7d vs MI sham DEGs}\tabularnewline
\toprule
rownames & Genesy\ldots{} & logFC & AveExpr & t & P.Value & adj.P.Val & B\tabularnewline
\midrule
\endfirsthead
\toprule
rownames & Genesy\ldots{} & logFC & AveExpr & t & P.Value & adj.P.Val & B\tabularnewline
\midrule
\endhead
7514 & Ctss & 4.8320\ldots{} & 7.4632\ldots{} & 50.601\ldots{} & 5.8181\ldots{} & 8.3472\ldots{} & 19.814\ldots{}\tabularnewline
14679 & Adamts2 & 3.9541\ldots{} & 7.5930\ldots{} & 37.675\ldots{} & 9.6997\ldots{} & 3.8153\ldots{} & 17.557\ldots{}\tabularnewline
23411 & Col14a1 & 4.5612\ldots{} & 7.5634\ldots{} & 37.311\ldots{} & 1.0637\ldots{} & 3.8153\ldots{} & 17.437\ldots{}\tabularnewline
11851 & Lox & 5.9882\ldots{} & 7.0429\ldots{} & 37.907\ldots{} & 9.1490\ldots{} & 3.8153\ldots{} & 17.419\ldots{}\tabularnewline
21619 & Fstl1 & 3.9252\ldots{} & 9.4422\ldots{} & 33.841\ldots{} & 2.6934\ldots{} & 6.6702\ldots{} & 16.550\ldots{}\tabularnewline
1261 & Ctsh & 2.6147\ldots{} & 6.6709\ldots{} & 31.959\ldots{} & 4.6403\ldots{} & 6.6702\ldots{} & 16.144\ldots{}\tabularnewline
13487 & Pla2g7 & 4.0298\ldots{} & 4.6625\ldots{} & 32.933\ldots{} & 3.4885\ldots{} & 6.6702\ldots{} & 16.129\ldots{}\tabularnewline
22176 & Laptm5 & 3.3558\ldots{} & 6.9162\ldots{} & 31.874\ldots{} & 4.7596\ldots{} & 6.6702\ldots{} & 16.105\ldots{}\tabularnewline
1490 & Sparc & 3.2522\ldots{} & 11.160\ldots{} & 32.579\ldots{} & 3.8660\ldots{} & 6.6702\ldots{} & 16.079\ldots{}\tabularnewline
6315 & Hexb & 3.1220\ldots{} & 6.3869\ldots{} & 31.264\ldots{} & 5.7173\ldots{} & 6.6702\ldots{} & 15.929\ldots{}\tabularnewline
5004 & Ctsz & 3.0952\ldots{} & 6.9421\ldots{} & 30.777\ldots{} & 6.6372\ldots{} & 6.6702\ldots{} & 15.790\ldots{}\tabularnewline
21174 & Fbln5 & 3.7685\ldots{} & 7.2452\ldots{} & 30.367\ldots{} & 7.5384\ldots{} & 6.6702\ldots{} & 15.656\ldots{}\tabularnewline
1805 & Litaf & 2.3676\ldots{} & 5.9412\ldots{} & 30.219\ldots{} & 7.8956\ldots{} & 6.6702\ldots{} & 15.624\ldots{}\tabularnewline
12260 & Nckap1l & 3.3359\ldots{} & 5.8304\ldots{} & 29.954\ldots{} & 8.5853\ldots{} & 6.6702\ldots{} & 15.519\ldots{}\tabularnewline
3893 & Gusb & 2.3568\ldots{} & 6.0931\ldots{} & 29.740\ldots{} & 9.1894\ldots{} & 6.6702\ldots{} & 15.480\ldots{}\tabularnewline
\ldots{} & \ldots{} & \ldots{} & \ldots{} & \ldots{} & \ldots{} & \ldots{} & \ldots{}\tabularnewline
\bottomrule
\end{longtable}

\hypertarget{ux57faux56e0ux6620ux5c04}{%
\subsubsection{基因映射}\label{ux57faux56e0ux6620ux5c04}}

将小鼠基因映射到人类

Table \ref{tab:Mapped-genes} (下方表格) 为表格Mapped genes概览。

\textbf{(对应文件为 \texttt{Figure+Table/Mapped-genes.csv})}

\begin{center}\begin{tcolorbox}[colback=gray!10, colframe=gray!50, width=0.9\linewidth, arc=1mm, boxrule=0.5pt]注:表格共有5274行2列,以下预览的表格可能省略部分数据;含有5123个唯一`mgi\_symbol;含有5146个唯一`hgnc\_symbol'。
\end{tcolorbox}
\end{center}
\begin{center}\begin{tcolorbox}[colback=gray!10, colframe=gray!50, width=0.9\linewidth, arc=1mm, boxrule=0.5pt]\begin{enumerate}\tightlist
\item hgnc\_symbol:  基因名 (Human)
\item mgi\_symbol:  基因名 (Mice)
\end{enumerate}\end{tcolorbox}
\end{center}

\begin{longtable}[]{@{}ll@{}}
\caption{\label{tab:Mapped-genes}Mapped genes}\tabularnewline
\toprule
mgi\_symbol & hgnc\_symbol\tabularnewline
\midrule
\endfirsthead
\toprule
mgi\_symbol & hgnc\_symbol\tabularnewline
\midrule
\endhead
Tmsb4x & TMSB4Y\tabularnewline
Hopx & HOPX\tabularnewline
Cyth4 & CYTH4\tabularnewline
Col6a2 & COL6A2\tabularnewline
Pacsin2 & PACSIN2\tabularnewline
Fbln1 & FBLN1\tabularnewline
Sh3bp2 & SH3BP2\tabularnewline
Abcg1 & ABCG1\tabularnewline
Mipep & MIPEP\tabularnewline
Itgb2 & ITGB2\tabularnewline
Pmepa1 & PMEPA1\tabularnewline
Maged2 & MAGED2\tabularnewline
Postn & POSTN\tabularnewline
Slc39a6 & SLC39A6\tabularnewline
Sirpa & SIRPG\tabularnewline
\ldots{} & \ldots{}\tabularnewline
\bottomrule
\end{longtable}

\hypertarget{fkbp5-ux7684ux8868ux8fbe}{%
\subsubsection{FKBP5 的表达}\label{fkbp5-ux7684ux8868ux8fbe}}

FKBP5 (Fkbp5) 在 MI 中属于显著高表达。

Table \ref{tab:Fkbp5-expression} (下方表格) 为表格Fkbp5 expression概览。

\textbf{(对应文件为 \texttt{Figure+Table/Fkbp5-expression.csv})}

\begin{center}\begin{tcolorbox}[colback=gray!10, colframe=gray!50, width=0.9\linewidth, arc=1mm, boxrule=0.5pt]注:表格共有1行10列,以下预览的表格可能省略部分数据;含有1个唯一`hgnc\_symbol'。
\end{tcolorbox}
\end{center}
\begin{center}\begin{tcolorbox}[colback=gray!10, colframe=gray!50, width=0.9\linewidth, arc=1mm, boxrule=0.5pt]\begin{enumerate}\tightlist
\item hgnc\_symbol:  基因名 (Human)
\item mgi\_symbol:  基因名 (Mice)
\item logFC:  estimate of the log2-fold-change corresponding to the effect or contrast (for ‘topTableF’ there may be several columns of log-fold-changes)
\item AveExpr:  average log2-expression for the probe over all arrays and channels, same as ‘Amean’ in the ‘MarrayLM’ object
\item t:  moderated t-statistic (omitted for ‘topTableF’)
\item P.Value:  raw p-value
\item B:  log-odds that the gene is differentially expressed (omitted for ‘topTreat’)
\end{enumerate}\end{tcolorbox}
\end{center}

\begin{longtable}[]{@{}llllllllll@{}}
\caption{\label{tab:Fkbp5-expression}Fkbp5 expression}\tabularnewline
\toprule
hgnc\_s\ldots{} & mgi\_sy\ldots{} & rownames & Genesy\ldots{} & logFC & AveExpr & t & P.Value & adj.P.Val & B\tabularnewline
\midrule
\endfirsthead
\toprule
hgnc\_s\ldots{} & mgi\_sy\ldots{} & rownames & Genesy\ldots{} & logFC & AveExpr & t & P.Value & adj.P.Val & B\tabularnewline
\midrule
\endhead
FKBP5 & Fkbp5 & 9124 & Fkbp5 & 1.5635\ldots{} & 5.3072\ldots{} & 5.7027\ldots{} & 0.0002\ldots{} & 0.0005\ldots{} & 0.0172\ldots{}\tabularnewline
\bottomrule
\end{longtable}

\hypertarget{mi-intersection-mi_key_degs}{%
\subsection{\texorpdfstring{MI intersection (\texttt{MI\_key\_DEGs})}{MI intersection (MI\_key\_DEGs)}}\label{mi-intersection-mi_key_degs}}

Figure \ref{fig:Intersection-of-MI-DEGs-with-MI-targets} (下方图) 为图Intersection of MI DEGs with MI targets概览。

\textbf{(对应文件为 \texttt{Figure+Table/Intersection-of-MI-DEGs-with-MI-targets.pdf})}

\def\@captype{figure}
\begin{center}
\includegraphics[width = 0.9\linewidth]{Figure+Table/Intersection-of-MI-DEGs-with-MI-targets.pdf}
\caption{Intersection of MI DEGs with MI targets}\label{fig:Intersection-of-MI-DEGs-with-MI-targets}
\end{center}
\begin{center}\begin{tcolorbox}[colback=gray!10, colframe=gray!50, width=0.9\linewidth, arc=1mm, boxrule=0.5pt]
\textbf{
Intersection
:}

\vspace{0.5em}

    ABCG1, ITGB2, POSTN, EGLN3, PPARGC1A, LTBP2, CYBB,
C3AR1, THBS1, SERPINE1, CLU, SFRP2, TGFB3, IGFBP4, TNC,
LCP1, GAS6, CTSZ, HPGDS, BGN, VLDLR, GUCY1A1, CYP4F3, LIPA,
NCAM1, GLA, HLA-DMB, FERMT3, LGALS3, TLR2, MMP2, GPNMB,
CYBA, ALCAM, KDR, TNNI3, ARNTL, IGFBP7, ANPEP, PPM1L,
TNFRSF1B, SERPINF1, ...

\vspace{2em}
\end{tcolorbox}
\end{center}

\textbf{(上述信息框内容已保存至 \texttt{Figure+Table/Intersection-of-MI-DEGs-with-MI-targets-content})}

\hypertarget{ux4e59ux9170ux5316ux9176}{%
\subsection{乙酰化酶}\label{ux4e59ux9170ux5316ux9176}}

\hypertarget{ux4f7fux7528ux7684ux4e59ux9170ux5316ux9176ux53caux5176ux76f8ux5173ux4fe1ux606f}{%
\subsubsection{使用的乙酰化酶及其相关信息}\label{ux4f7fux7528ux7684ux4e59ux9170ux5316ux9176ux53caux5176ux76f8ux5173ux4fe1ux606f}}

Table \ref{tab:All-protein-of-CoA} (下方表格) 为表格All protein of CoA概览。

\textbf{(对应文件为 \texttt{Figure+Table/All-protein-of-CoA.xlsx})}

\begin{center}\begin{tcolorbox}[colback=gray!10, colframe=gray!50, width=0.9\linewidth, arc=1mm, boxrule=0.5pt]注:表格共有145行25列,以下预览的表格可能省略部分数据;含有142个唯一`HGNC\_symbol'。
\end{tcolorbox}
\end{center}

\begin{longtable}[]{@{}llllllllll@{}}
\caption{\label{tab:All-protein-of-CoA}All protein of CoA}\tabularnewline
\toprule
HGNC\_s\ldots{} & Modifi\ldots{} & Id & Status & HGNC\_ID & HGNC\_name & GeneID & UniPro\ldots\ldots8 & UniPro\ldots\ldots9 & Domain\tabularnewline
\midrule
\endfirsthead
\toprule
HGNC\_s\ldots{} & Modifi\ldots{} & Id & Status & HGNC\_ID & HGNC\_name & GeneID & UniPro\ldots\ldots8 & UniPro\ldots\ldots9 & Domain\tabularnewline
\midrule
\endhead
ARID4A & Histon\ldots{} & 36 & \# & 9885 & AT ric\ldots{} & 5926 & P29374 & ARI4A\_\ldots{} & ARID P\ldots{}\tabularnewline
ARID4B & Histon\ldots{} & 37 & \# & 15550 & AT ric\ldots{} & 51742 & Q4LE39 & ARI4B\_\ldots{} & ARID P\ldots{}\tabularnewline
ATF2 & Histon\ldots{} & 49 & \# & 784 & activa\ldots{} & 1386 & P15336 & ATF2\_H\ldots{} & bZIP\_1\ldots{}\tabularnewline
ATXN7 & Histon\ldots{} & 55 & \# & 10560 & ataxin 7 & 6314 & O15265 & ATX7\_H\ldots{} & Pfam-B\ldots{}\tabularnewline
BANP & Histon\ldots{} & 62 & \# & 13450 & BTG3 a\ldots{} & 54971 & Q8N9N5 & BANP\_H\ldots{} & BEN PF\ldots{}\tabularnewline
BAZ2A & Histon\ldots{} & 67 & \# & 962 & bromod\ldots{} & 11176 & Q9UIF9 & BAZ2A\_\ldots{} & Bromod\ldots{}\tabularnewline
BCORL1 & Histon\ldots{} & 70 & \# & 25657 & BCL6 c\ldots{} & 63035 & Q5H9F3 & BCORL\_\ldots{} & Ank\_2 \ldots{}\tabularnewline
BRCA1 & Histon\ldots{} & 73 & \# & 1100 & breast\ldots{} & 672 & P38398 & BRCA1\_\ldots{} & BRCT P\ldots{}\tabularnewline
BRCA2 & Histon\ldots{} & 74 & \# & 1101 & breast\ldots{} & 675 & P51587 & BRCA2\_\ldots{} & BRCA-2\ldots{}\tabularnewline
BRMS1L & Histon\ldots{} & 86 & \# & 20512 & breast\ldots{} & 84312 & Q5PSV4 & BRM1L\_\ldots{} & Sds3 P\ldots{}\tabularnewline
BRPF3 & Histon\ldots{} & 88 & \# & 14256 & bromod\ldots{} & 27154 & Q9ULD4 & BRPF3\_\ldots{} & Bromod\ldots{}\tabularnewline
CDY1 & Histon\ldots{} & 115 & \# & 1809 & chromo\ldots{} & 9085 & Q9Y6F8 & CDY1\_H\ldots{} & Chromo\ldots{}\tabularnewline
CDY1B & Histon\ldots{} & 116 & \# & 23920 & chromo\ldots{} & 253175 & Q9Y6F8 & CDY1\_H\ldots{} & Chromo\ldots{}\tabularnewline
CDY2A & Histon\ldots{} & 117 & \# & 1810 & chromo\ldots{} & 9426 & Q9Y6F7 & CDY2\_H\ldots{} & Chromo\ldots{}\tabularnewline
CDY2B & Histon\ldots{} & 118 & \# & 23921 & chromo\ldots{} & 203611 & Q9Y6F7 & CDY2\_H\ldots{} & Chromo\ldots{}\tabularnewline
\ldots{} & \ldots{} & \ldots{} & \ldots{} & \ldots{} & \ldots{} & \ldots{} & \ldots{} & \ldots{} & \ldots{}\tabularnewline
\bottomrule
\end{longtable}

\hypertarget{ux7b5bux9009ux5deeux5f02ux8868ux8fbeux7684ux4e59ux9170ux5316ux9176-coa_degs}{%
\subsubsection{\texorpdfstring{筛选差异表达的乙酰化酶 (\texttt{CoA\_DEGs})}{筛选差异表达的乙酰化酶 (CoA\_DEGs)}}\label{ux7b5bux9009ux5deeux5f02ux8868ux8fbeux7684ux4e59ux9170ux5316ux9176-coa_degs}}

使用 MI 数据 (\ref{MI}) 的 DEGs,筛选差异表达的乙酰化酶。

以 \texttt{mgi\_symbol} 取交集。

Figure \ref{fig:Intersection-of-All-CoA-with-MI-DEGs} (下方图) 为图Intersection of All CoA with MI DEGs概览。

\textbf{(对应文件为 \texttt{Figure+Table/Intersection-of-All-CoA-with-MI-DEGs.pdf})}

\def\@captype{figure}
\begin{center}
\includegraphics[width = 0.9\linewidth]{Figure+Table/Intersection-of-All-CoA-with-MI-DEGs.pdf}
\caption{Intersection of All CoA with MI DEGs}\label{fig:Intersection-of-All-CoA-with-MI-DEGs}
\end{center}
\begin{center}\begin{tcolorbox}[colback=gray!10, colframe=gray!50, width=0.9\linewidth, arc=1mm, boxrule=0.5pt]
\textbf{
Intersection
:}

\vspace{0.5em}

    Brca1, Eid1, Eid2b, Hdac11, Hdac9, Hif1an, Jdp2,
Morf4l2, Ncoa1, Nsl1, Sirt7, Smarca1, Taf7, Zbtb16

\vspace{2em}
\end{tcolorbox}
\end{center}

\textbf{(上述信息框内容已保存至 \texttt{Figure+Table/Intersection-of-All-CoA-with-MI-DEGs-content})}

\hypertarget{ux5176ux5b83ux5019ux9009}{%
\subsection{其它候选}\label{ux5176ux5b83ux5019ux9009}}

\hypertarget{ppi}{%
\subsubsection{\texorpdfstring{以 PPI 网络筛选与 \texttt{CoA\_DEGs} 相关的 \texttt{MI\_key\_DEGs}}{以 PPI 网络筛选与 CoA\_DEGs 相关的 MI\_key\_DEGs}}\label{ppi}}

根据 Fig. \ref{fig:Intersection-of-MI-DEGs-with-MI-targets}
和 Fig. \ref{fig:Intersection-of-All-CoA-with-MI-DEGs}
建立 PPI 网络 (有实验基础的蛋白物理直接互作) 。

\begin{center}\begin{tcolorbox}[colback=gray!10, colframe=gray!50, width=0.9\linewidth, arc=1mm, boxrule=0.5pt]
\textbf{
STRINGdb network type:
:}

\vspace{0.5em}

    physical

\vspace{2em}


\textbf{
Filter experiments score:
:}

\vspace{0.5em}

    At least score 100

\vspace{2em}


\textbf{
Filter textmining score:
:}

\vspace{0.5em}

    At least score 0

\vspace{2em}
\end{tcolorbox}
\end{center}

Table \ref{tab:PPI-annotation} (下方表格) 为表格PPI annotation概览。

\textbf{(对应文件为 \texttt{Figure+Table/PPI-annotation.csv})}

\begin{center}\begin{tcolorbox}[colback=gray!10, colframe=gray!50, width=0.9\linewidth, arc=1mm, boxrule=0.5pt]注:表格共有1364行10列,以下预览的表格可能省略部分数据;含有381个唯一`from'。
\end{tcolorbox}
\end{center}
\begin{center}\begin{tcolorbox}[colback=gray!10, colframe=gray!50, width=0.9\linewidth, arc=1mm, boxrule=0.5pt]\begin{enumerate}\tightlist
\item experiments:  相关实验。
\end{enumerate}\end{tcolorbox}
\end{center}

\begin{longtable}[]{@{}llllllllll@{}}
\caption{\label{tab:PPI-annotation}PPI annotation}\tabularnewline
\toprule
from & to & homology & experi\ldots\ldots4 & experi\ldots\ldots5 & database & databa\ldots{} & textmi\ldots\ldots8 & textmi\ldots\ldots9 & \ldots{}\tabularnewline
\midrule
\endfirsthead
\toprule
from & to & homology & experi\ldots\ldots4 & experi\ldots\ldots5 & database & databa\ldots{} & textmi\ldots\ldots8 & textmi\ldots\ldots9 & \ldots{}\tabularnewline
\midrule
\endhead
TNFRSF1A & RIPK3 & 0 & 292 & 0 & 0 & 0 & 473 & 0 & \ldots{}\tabularnewline
DCN & PLAT & 0 & 205 & 0 & 0 & 0 & 0 & 0 & \ldots{}\tabularnewline
DCN & TGFB1 & 0 & 457 & 0 & 500 & 0 & 979 & 60 & \ldots{}\tabularnewline
MMP2 & TGFB1 & 0 & 548 & 0 & 0 & 0 & 118 & 0 & \ldots{}\tabularnewline
PLAT & SERPINE1 & 0 & 955 & 0 & 700 & 0 & 982 & 0 & \ldots{}\tabularnewline
MYH9 & ACTA2 & 0 & 205 & 97 & 900 & 0 & 0 & 91 & \ldots{}\tabularnewline
MMP2 & COL1A1 & 0 & 292 & 0 & 0 & 0 & 0 & 0 & \ldots{}\tabularnewline
TGFB1 & VDR & 0 & 292 & 0 & 0 & 0 & 0 & 0 & \ldots{}\tabularnewline
COL1A1 & VDR & 0 & 292 & 0 & 0 & 0 & 0 & 0 & \ldots{}\tabularnewline
MMP2 & LOX & 0 & 238 & 0 & 0 & 0 & 0 & 0 & \ldots{}\tabularnewline
COL1A1 & LOX & 0 & 230 & 0 & 0 & 0 & 0 & 0 & \ldots{}\tabularnewline
COL1A1 & SPARC & 0 & 457 & 0 & 0 & 0 & 89 & 90 & \ldots{}\tabularnewline
VDR & IL12B & 0 & 292 & 0 & 0 & 0 & 0 & 0 & \ldots{}\tabularnewline
ACTA2 & CTSD & 0 & 229 & 0 & 0 & 0 & 0 & 0 & \ldots{}\tabularnewline
VDR & EGR1 & 0 & 292 & 0 & 0 & 0 & 0 & 0 & \ldots{}\tabularnewline
\ldots{} & \ldots{} & \ldots{} & \ldots{} & \ldots{} & \ldots{} & \ldots{} & \ldots{} & \ldots{} & \ldots{}\tabularnewline
\bottomrule
\end{longtable}

获取 CoA -\textgreater{} DEGs 的网络:

Figure \ref{fig:Filtered-and-formated-PPI-network} (下方图) 为图Filtered and formated PPI network概览。

\textbf{(对应文件为 \texttt{Figure+Table/Filtered-and-formated-PPI-network.pdf})}

\def\@captype{figure}
\begin{center}
\includegraphics[width = 0.9\linewidth]{Figure+Table/Filtered-and-formated-PPI-network.pdf}
\caption{Filtered and formated PPI network}\label{fig:Filtered-and-formated-PPI-network}
\end{center}

\hypertarget{cor}{%
\subsubsection{关联分析}\label{cor}}

根据 Fig. \ref{fig:Filtered-and-formated-PPI-network},以小鼠数据集 (\ref{MI}) 进行关联分析。

Figure \ref{fig:MI-correlation-heatmap} (下方图) 为图MI correlation heatmap概览。

\textbf{(对应文件为 \texttt{Figure+Table/MI-correlation-heatmap.pdf})}

\def\@captype{figure}
\begin{center}
\includegraphics[width = 0.9\linewidth]{Figure+Table/MI-correlation-heatmap.pdf}
\caption{MI correlation heatmap}\label{fig:MI-correlation-heatmap}
\end{center}

Table \ref{tab:MI-significant-correlation} (下方表格) 为表格MI significant correlation概览。

\textbf{(对应文件为 \texttt{Figure+Table/MI-significant-correlation.csv})}

\begin{center}\begin{tcolorbox}[colback=gray!10, colframe=gray!50, width=0.9\linewidth, arc=1mm, boxrule=0.5pt]注:表格共有738行7列,以下预览的表格可能省略部分数据;含有13个唯一`CoA\_DEGs\_ppi'。
\end{tcolorbox}
\end{center}
\begin{center}\begin{tcolorbox}[colback=gray!10, colframe=gray!50, width=0.9\linewidth, arc=1mm, boxrule=0.5pt]\begin{enumerate}\tightlist
\item cor:  皮尔逊关联系数,正关联或负关联。
\item pvalue:  显著性 P。
\item -log2(P.value):  P 的对数转化。
\item significant:  显著性。
\item sign:  人为赋予的符号,参考 significant。
\end{enumerate}\end{tcolorbox}
\end{center}

\begin{longtable}[]{@{}lllllll@{}}
\caption{\label{tab:MI-significant-correlation}MI significant correlation}\tabularnewline
\toprule
CoA\_DEGs\_ppi & MI\_key\_DEG\ldots{} & cor & pvalue & -log2(P.va\ldots{} & significant & sign\tabularnewline
\midrule
\endfirsthead
\toprule
CoA\_DEGs\_ppi & MI\_key\_DEG\ldots{} & cor & pvalue & -log2(P.va\ldots{} & significant & sign\tabularnewline
\midrule
\endhead
Morf4l2 & Ppargc1a & -0.95 & 1e-04 & 13.2877123\ldots{} & \textless{} 0.001 & **\tabularnewline
Hdac9 & Ppargc1a & 0.98 & 0 & 16.6096404\ldots{} & \textless{} 0.001 & **\tabularnewline
Sirt7 & Ppargc1a & -0.92 & 5e-04 & 10.9657842\ldots{} & \textless{} 0.001 & **\tabularnewline
Nsl1 & Ppargc1a & -0.94 & 2e-04 & 12.2877123\ldots{} & \textless{} 0.001 & **\tabularnewline
Taf7 & Ppargc1a & 0.9 & 0.001 & 9.96578428\ldots{} & \textless{} 0.001 & **\tabularnewline
Ncoa1 & Ppargc1a & 0.91 & 7e-04 & 10.4803574\ldots{} & \textless{} 0.001 & **\tabularnewline
Jdp2 & Ppargc1a & -0.74 & 0.0217 & 5.52616114\ldots{} & \textless{} 0.05 & *\tabularnewline
Hif1an & Ppargc1a & 0.99 & 0 & 16.6096404\ldots{} & \textless{} 0.001 & **\tabularnewline
Brca1 & Ppargc1a & -0.93 & 3e-04 & 11.7027498\ldots{} & \textless{} 0.001 & **\tabularnewline
Smarca1 & Ppargc1a & -0.95 & 1e-04 & 13.2877123\ldots{} & \textless{} 0.001 & **\tabularnewline
Hdac11 & Ppargc1a & 0.96 & 0 & 16.6096404\ldots{} & \textless{} 0.001 & **\tabularnewline
Eid1 & Ppargc1a & -0.87 & 0.0024 & 8.70274987\ldots{} & \textless{} 0.05 & *\tabularnewline
Zbtb16 & Ppargc1a & 0.67 & 0.0483 & 4.37183300\ldots{} & \textless{} 0.05 & *\tabularnewline
Morf4l2 & Il18r1 & 0.88 & 0.0018 & 9.11778737\ldots{} & \textless{} 0.05 & *\tabularnewline
Hdac9 & Il18r1 & -0.78 & 0.013 & 6.26534456\ldots{} & \textless{} 0.05 & *\tabularnewline
\ldots{} & \ldots{} & \ldots{} & \ldots{} & \ldots{} & \ldots{} & \ldots{}\tabularnewline
\bottomrule
\end{longtable}

\hypertarget{ux5b58ux5728-ppi-ux5173ux8054ux4e14ux5173ux8054ux5206ux6790ux663eux8457ux7684ux7ec4ux5408}{%
\subsubsection{存在 PPI 关联且关联分析显著的组合}\label{ux5b58ux5728-ppi-ux5173ux8054ux4e14ux5173ux8054ux5206ux6790ux663eux8457ux7684ux7ec4ux5408}}

结合 \ref{ppi} 和 \ref{cor} 筛选 CoA 与 XX

Table \ref{tab:PPI-interact-and-significant-correlated-in-MI} (下方表格) 为表格PPI interact and significant correlated in MI概览。

\textbf{(对应文件为 \texttt{Figure+Table/PPI-interact-and-significant-correlated-in-MI.csv})}

\begin{center}\begin{tcolorbox}[colback=gray!10, colframe=gray!50, width=0.9\linewidth, arc=1mm, boxrule=0.5pt]注:表格共有64行9列,以下预览的表格可能省略部分数据;含有13个唯一`CoA\_DEGs\_ppi'。
\end{tcolorbox}
\end{center}
\begin{center}\begin{tcolorbox}[colback=gray!10, colframe=gray!50, width=0.9\linewidth, arc=1mm, boxrule=0.5pt]\begin{enumerate}\tightlist
\item cor:  皮尔逊关联系数,正关联或负关联。
\item pvalue:  显著性 P。
\item -log2(P.value):  P 的对数转化。
\item significant:  显著性。
\item sign:  人为赋予的符号,参考 significant。
\end{enumerate}\end{tcolorbox}
\end{center}

\begin{longtable}[]{@{}lllllllll@{}}
\caption{\label{tab:PPI-interact-and-significant-correlated-in-MI}PPI interact and significant correlated in MI}\tabularnewline
\toprule
CoA\_DE\ldots{} & MI\_key\ldots{} & cor & pvalue & -log2(\ldots{} & signif\ldots{} & sign & CoA\_hg\ldots{} & DEG\_hg\ldots{}\tabularnewline
\midrule
\endfirsthead
\toprule
CoA\_DE\ldots{} & MI\_key\ldots{} & cor & pvalue & -log2(\ldots{} & signif\ldots{} & sign & CoA\_hg\ldots{} & DEG\_hg\ldots{}\tabularnewline
\midrule
\endhead
Brca1 & Casp1 & 0.91 & 8e-04 & 10.287\ldots{} & \textless{} 0.001 & ** & BRCA1 & CASP1\tabularnewline
Brca1 & Ccna2 & 0.9 & 0.0011 & 9.8282\ldots{} & \textless{} 0.05 & * & BRCA1 & CCNA2\tabularnewline
Brca1 & Ccnd1 & -0.87 & 0.0024 & 8.7027\ldots{} & \textless{} 0.05 & * & BRCA1 & CCND1\tabularnewline
Brca1 & Cdk1 & 0.95 & 1e-04 & 13.287\ldots{} & \textless{} 0.001 & ** & BRCA1 & CDK1\tabularnewline
Brca1 & E2f1 & 0.95 & 1e-04 & 13.287\ldots{} & \textless{} 0.001 & ** & BRCA1 & E2F1\tabularnewline
Brca1 & Esr1 & 0.7 & 0.0356 & 4.8119\ldots{} & \textless{} 0.05 & * & BRCA1 & ESR1\tabularnewline
Brca1 & Ezh2 & 0.88 & 0.002 & 8.9657\ldots{} & \textless{} 0.05 & * & BRCA1 & EZH2\tabularnewline
Brca1 & Fancd2 & 0.94 & 2e-04 & 12.287\ldots{} & \textless{} 0.001 & ** & BRCA1 & FANCD2\tabularnewline
Brca1 & Flna & 0.92 & 4e-04 & 11.287\ldots{} & \textless{} 0.001 & ** & BRCA1 & FLNA\tabularnewline
Brca1 & Jup & -0.94 & 2e-04 & 12.287\ldots{} & \textless{} 0.001 & ** & BRCA1 & JUP\tabularnewline
Brca1 & Kif2c & 0.89 & 0.0011 & 9.8282\ldots{} & \textless{} 0.05 & * & BRCA1 & KIF2C\tabularnewline
Brca1 & Lgals3 & 0.89 & 0.0013 & 9.5872\ldots{} & \textless{} 0.05 & * & BRCA1 & LGALS3\tabularnewline
Brca1 & Lmna & 0.89 & 0.0015 & 9.3808\ldots{} & \textless{} 0.05 & * & BRCA1 & LMNA\tabularnewline
Brca1 & Mapt & -0.91 & 8e-04 & 10.287\ldots{} & \textless{} 0.001 & ** & BRCA1 & MAPT\tabularnewline
Brca1 & Mefv & 0.97 & 0 & 16.609\ldots{} & \textless{} 0.001 & ** & BRCA1 & MEFV\tabularnewline
\ldots{} & \ldots{} & \ldots{} & \ldots{} & \ldots{} & \ldots{} & \ldots{} & \ldots{} & \ldots{}\tabularnewline
\bottomrule
\end{longtable}

\hypertarget{ux5bccux96c6ux5206ux6790}{%
\subsubsection{富集分析}\label{ux5bccux96c6ux5206ux6790}}

将 Tab. \ref{tab:PPI-interact-and-significant-correlated-in-MI} 中的 DEGs 进行富集分析

Figure \ref{fig:GO-enrichment} (下方图) 为图GO enrichment概览。

\textbf{(对应文件为 \texttt{Figure+Table/GO-enrichment.pdf})}

\def\@captype{figure}
\begin{center}
\includegraphics[width = 0.9\linewidth]{Figure+Table/GO-enrichment.pdf}
\caption{GO enrichment}\label{fig:GO-enrichment}
\end{center}

\hypertarget{coa-xx-pathways}{%
\subsubsection{CoA-XX-pathways}\label{coa-xx-pathways}}

Figure \ref{fig:CoA-XX-GOpathways} (下方图) 为图CoA XX GOpathways概览。

\textbf{(对应文件为 \texttt{Figure+Table/CoA-XX-GOpathways.pdf})}

\def\@captype{figure}
\begin{center}
\includegraphics[width = 0.9\linewidth]{Figure+Table/CoA-XX-GOpathways.pdf}
\caption{CoA XX GOpathways}\label{fig:CoA-XX-GOpathways}
\end{center}

Table \ref{tab:All-candidates-and-enriched-GO-BP-pathways} (下方表格) 为表格All candidates and enriched GO BP pathways概览。

\textbf{(对应文件为 \texttt{Figure+Table/All-candidates-and-enriched-GO-BP-pathways.csv})}

\begin{center}\begin{tcolorbox}[colback=gray!10, colframe=gray!50, width=0.9\linewidth, arc=1mm, boxrule=0.5pt]注:表格共有64行4列,以下预览的表格可能省略部分数据;含有13个唯一`CoA\_hgnc\_symbol;含有51个唯一`DEG\_hgnc\_symbol'。
\end{tcolorbox}
\end{center}

\begin{longtable}[]{@{}llll@{}}
\caption{\label{tab:All-candidates-and-enriched-GO-BP-pathways}All candidates and enriched GO BP pathways}\tabularnewline
\toprule
CoA\_hgnc\_symbol & DEG\_hgnc\_symbol & Hit\_pathway\_number & Enriched\_pathways\tabularnewline
\midrule
\endfirsthead
\toprule
CoA\_hgnc\_symbol & DEG\_hgnc\_symbol & Hit\_pathway\_number & Enriched\_pathways\tabularnewline
\midrule
\endhead
BRCA1 & FLNA & 6 & cardiac muscle co\ldots{}\tabularnewline
BRCA1 & SRC & 6 & coagulation; regu\ldots{}\tabularnewline
HDAC9 & PIK3CG & 6 & cardiac muscle co\ldots{}\tabularnewline
NCOA1 & SRC & 6 & coagulation; regu\ldots{}\tabularnewline
ZBTB16 & CASP3 & 5 & response to corti\ldots{}\tabularnewline
BRCA1 & CCND1 & 4 & regulation of bod\ldots{}\tabularnewline
BRCA1 & TTN & 4 & cardiac muscle co\ldots{}\tabularnewline
JDP2 & FOS & 4 & response to corti\ldots{}\tabularnewline
MORF4L2 & ACTG1 & 4 & coagulation; regu\ldots{}\tabularnewline
MORF4L2 & TNNT2 & 4 & cardiac muscle co\ldots{}\tabularnewline
NCOA1 & CCND1 & 4 & regulation of bod\ldots{}\tabularnewline
NCOA1 & FOS & 4 & response to corti\ldots{}\tabularnewline
NCOA1 & PPARA & 4 & muscle system pro\ldots{}\tabularnewline
BRCA1 & JUP & 3 & cardiac muscle co\ldots{}\tabularnewline
BRCA1 & PLAUR & 3 & coagulation; regu\ldots{}\tabularnewline
\ldots{} & \ldots{} & \ldots{} & \ldots{}\tabularnewline
\bottomrule
\end{longtable}

\hypertarget{bibliography}{%
\section*{Reference}\label{bibliography}}
\addcontentsline{toc}{section}{Reference}

\hypertarget{refs}{}
\begin{cslreferences}
\leavevmode\hypertarget{ref-MappingIdentifDurinc2009}{}%
1. Durinck, S., Spellman, P. T., Birney, E. \& Huber, W. Mapping identifiers for the integration of genomic datasets with the r/bioconductor package biomaRt. \emph{Nature protocols} \textbf{4}, 1184--1191 (2009).

\leavevmode\hypertarget{ref-ClusterprofilerWuTi2021}{}%
2. Wu, T. \emph{et al.} ClusterProfiler 4.0: A universal enrichment tool for interpreting omics data. \emph{The Innovation} \textbf{2}, (2021).

\leavevmode\hypertarget{ref-Epifactors2022Maraku2023}{}%
3. Marakulina, D. \emph{et al.} EpiFactors 2022: Expansion and enhancement of a curated database of human epigenetic factors and complexes. \emph{Nucleic acids research} \textbf{51}, D564--D570 (2023).

\leavevmode\hypertarget{ref-TheDisgenetKnPinero2019}{}%
4. Piñero, J. \emph{et al.} The disgenet knowledge platform for disease genomics: 2019 update. \emph{Nucleic Acids Research} (2019) doi:\href{https://doi.org/10.1093/nar/gkz1021}{10.1093/nar/gkz1021}.

\leavevmode\hypertarget{ref-TheGenecardsSStelze2016}{}%
5. Stelzer, G. \emph{et al.} The genecards suite: From gene data mining to disease genome sequence analyses. \emph{Current protocols in bioinformatics} \textbf{54}, 1.30.1--1.30.33 (2016).

\leavevmode\hypertarget{ref-PharmgkbAWorBarbar2018}{}%
6. Barbarino, J. M., Whirl-Carrillo, M., Altman, R. B. \& Klein, T. E. PharmGKB: A worldwide resource for pharmacogenomic information. \emph{Wiley interdisciplinary reviews. Systems biology and medicine} \textbf{10}, (2018).

\leavevmode\hypertarget{ref-LimmaPowersDiRitchi2015}{}%
7. Ritchie, M. E. \emph{et al.} Limma powers differential expression analyses for rna-sequencing and microarray studies. \emph{Nucleic Acids Research} \textbf{43}, e47 (2015).

\leavevmode\hypertarget{ref-EdgerDifferenChen}{}%
8. Chen, Y., McCarthy, D., Ritchie, M., Robinson, M. \& Smyth, G. EdgeR: Differential analysis of sequence read count data user's guide. 119.

\leavevmode\hypertarget{ref-TheStringDataSzklar2021}{}%
9. Szklarczyk, D. \emph{et al.} The string database in 2021: Customizable proteinprotein networks, and functional characterization of user-uploaded gene/measurement sets. \emph{Nucleic Acids Research} \textbf{49}, D605--D612 (2021).

\leavevmode\hypertarget{ref-CytohubbaIdenChin2014}{}%
10. Chin, C.-H. \emph{et al.} CytoHubba: Identifying hub objects and sub-networks from complex interactome. \emph{BMC Systems Biology} \textbf{8}, S11 (2014).
\end{cslreferences}

\end{document}
