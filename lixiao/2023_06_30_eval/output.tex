% Options for packages loaded elsewhere
\PassOptionsToPackage{unicode}{hyperref}
\PassOptionsToPackage{hyphens}{url}
%
\documentclass[
]{article}
\usepackage{lmodern}
\usepackage{amssymb,amsmath}
\usepackage{ifxetex,ifluatex}
\ifnum 0\ifxetex 1\fi\ifluatex 1\fi=0 % if pdftex
  \usepackage[T1]{fontenc}
  \usepackage[utf8]{inputenc}
  \usepackage{textcomp} % provide euro and other symbols
\else % if luatex or xetex
  \usepackage{unicode-math}
  \defaultfontfeatures{Scale=MatchLowercase}
  \defaultfontfeatures[\rmfamily]{Ligatures=TeX,Scale=1}
\fi
% Use upquote if available, for straight quotes in verbatim environments
\IfFileExists{upquote.sty}{\usepackage{upquote}}{}
\IfFileExists{microtype.sty}{% use microtype if available
  \usepackage[]{microtype}
  \UseMicrotypeSet[protrusion]{basicmath} % disable protrusion for tt fonts
}{}
\makeatletter
\@ifundefined{KOMAClassName}{% if non-KOMA class
  \IfFileExists{parskip.sty}{%
    \usepackage{parskip}
  }{% else
    \setlength{\parindent}{0pt}
    \setlength{\parskip}{6pt plus 2pt minus 1pt}}
}{% if KOMA class
  \KOMAoptions{parskip=half}}
\makeatother
\usepackage{xcolor}
\IfFileExists{xurl.sty}{\usepackage{xurl}}{} % add URL line breaks if available
\IfFileExists{bookmark.sty}{\usepackage{bookmark}}{\usepackage{hyperref}}
\hypersetup{
  pdftitle={Results of Molecular Docking},
  pdfauthor={Huang Lichuang in Wie-Biotech},
  hidelinks,
  pdfcreator={LaTeX via pandoc}}
\urlstyle{same} % disable monospaced font for URLs
\usepackage[margin=1in]{geometry}
\usepackage{longtable,booktabs}
% Correct order of tables after \paragraph or \subparagraph
\usepackage{etoolbox}
\makeatletter
\patchcmd\longtable{\par}{\if@noskipsec\mbox{}\fi\par}{}{}
\makeatother
% Allow footnotes in longtable head/foot
\IfFileExists{footnotehyper.sty}{\usepackage{footnotehyper}}{\usepackage{footnote}}
\makesavenoteenv{longtable}
\usepackage{graphicx}
\makeatletter
\def\maxwidth{\ifdim\Gin@nat@width>\linewidth\linewidth\else\Gin@nat@width\fi}
\def\maxheight{\ifdim\Gin@nat@height>\textheight\textheight\else\Gin@nat@height\fi}
\makeatother
% Scale images if necessary, so that they will not overflow the page
% margins by default, and it is still possible to overwrite the defaults
% using explicit options in \includegraphics[width, height, ...]{}
\setkeys{Gin}{width=\maxwidth,height=\maxheight,keepaspectratio}
% Set default figure placement to htbp
\makeatletter
\def\fps@figure{htbp}
\makeatother
\setlength{\emergencystretch}{3em} % prevent overfull lines
\providecommand{\tightlist}{%
  \setlength{\itemsep}{0pt}\setlength{\parskip}{0pt}}
\setcounter{secnumdepth}{5}
\usepackage{caption} \captionsetup{font={footnotesize},width=6in} \renewcommand{\dblfloatpagefraction}{.9} \makeatletter \renewenvironment{figure} {\def\@captype{figure}} \makeatletter \definecolor{shadecolor}{RGB}{242,242,242} \usepackage{xeCJK} \usepackage{setspace} \setstretch{1.3}
\newlength{\cslhangindent}
\setlength{\cslhangindent}{1.5em}
\newenvironment{cslreferences}%
  {}%
  {\par}

\title{Results of Molecular Docking}
\author{Huang Lichuang in Wie-Biotech}
\date{}

\begin{document}
\maketitle

{
\setcounter{tocdepth}{3}
\tableofcontents
}
\hypertarget{ux8bbeux8ba1ux6d41ux7a0b}{%
\section{设计流程}\label{ux8bbeux8ba1ux6d41ux7a0b}}

\hypertarget{ux5df2ux6709ux6570ux636e}{%
\subsection{已有数据}\label{ux5df2ux6709ux6570ux636e}}

已筛选的Hubgenes;药方的中药和相应成分,以及对应靶点。

\hypertarget{ux5206ux5b50ux5bf9ux63a5}{%
\subsection{分子对接}\label{ux5206ux5b50ux5bf9ux63a5}}

步骤:

\begin{itemize}
\tightlist
\item
  从已有数据得到成分和靶点数据,根据筛选的Hubgenes预处理,得到可能的药物分子和靶点的结合。
\item
  从Uniprot 网站(\url{https://www.uniprot.org/})或 RCSB PDB(\url{https://www.rcsb.org/})获取靶点蛋白的结构。
\item
  从PubChem \url{https://pubchem.ncbi.nlm.nih.gov/} 获取药物分子结构。
\item
  以AutoDock Vina 1.2.0\textsuperscript{\protect\hyperlink{ref-AutodockVina1Eberha2021}{1}} (\url{http://vina.scripps.edu}),在Python下使用,批量对接多个分子和靶点。
\item
  将结果可视化或输出成表格。
\end{itemize}

\hypertarget{ux5206ux6790ux548cux7ed3ux679c}{%
\section{分析和结果}\label{ux5206ux6790ux548cux7ed3ux679c}}

以下分析和上述设计思路可能有所不同。

除了AutoDock Vina (\url{http://vina.scripps.edu}), 还会用到 ADFR 工具组(\url{https://ccsb.scripps.edu/adfr/}),Python 的 Meeko ,或其它工具。

流程请参考文献\textsuperscript{\protect\hyperlink{ref-ComputationalPForli2016}{2}} 或者 \url{https://autodock-vina.readthedocs.io/en/latest/docking_basic.html}。

\hypertarget{ux4e2dux836fux548cux6210ux5206ux548cux9776ux70b9ux6570ux636eux7684ux9884ux5904ux7406}{%
\subsection{中药和成分和靶点数据的预处理}\label{ux4e2dux836fux548cux6210ux5206ux548cux9776ux70b9ux6570ux636eux7684ux9884ux5904ux7406}}

\hypertarget{ux6574ux7406ux6765ux81ea-herb-ux7f51ux7ad9ux7684ux6570ux636e}{%
\subsubsection{整理来自 HERB 网站的数据}\label{ux6574ux7406ux6765ux81ea-herb-ux7f51ux7ad9ux7684ux6570ux636e}}

从网站 \url{http://herb.ac.cn/Download/} 获取成分信息。

药方的化合物成分信息概览,共有1097个唯一成分:
\textbf{(对应文件为 \texttt{./all\_ingredients\_info.xlsx})}

\begin{verbatim}
## # A tibble: 1,231 x 4
##    herb_id    Ingredient.id Ingredient.name                                                                       Ingredient.alias
##    <chr>      <chr>         <chr>                                                                                 <chr>           
##  1 HERB000872 HBIN004123    2,3-isopropylidene cyasterone                                                         NA              
##  2 HERB000872 HBIN004124    2,3-isopropylidene isocyasterone                                                      NA              
##  3 HERB000872 HBIN004406    24-hydroxycyasterone                                                                  NA              
##  4 HERB000872 HBIN009103    3-O-[α-L-rhamnopyranosyl-(1→3)β-D-glucopyranosiduronic acid]-28-O-β-D-glucopyranosyl~ NA              
##  5 HERB000872 HBIN009104    3-O-[α-L-rhamnopyranosyl-(1→3)β-D-glucopyranosiduronic acid] oleanolic acid           NA              
##  6 HERB000872 HBIN009108    3-O-[α-L-rhamnopyranosyl-(1→3)-(n-butyl-β-D-glucopyranosiduronate}-28-O-β-D-glucopyr~ NA              
##  7 HERB000872 HBIN009160    3-O-[β-D-glucopyranosiduronic acid]-28-O-β-D-glucopyranosyl oleanolic acid            NA              
##  8 HERB000872 HBIN009161    3-O-(β-D-glucopyranosiduronic acid) oleanolic acid                                    oleanolic acid-~
##  9 HERB000872 HBIN009164    3-O-[β-D-glucopyranosyl-(1→2)-α-L-rhamnopyranosyl-(1→3)β-D-glucopyranosiduronic acid~ NA              
## 10 HERB000872 HBIN009222    3-O-β-D-glucopyranosyl oleanolic acid                                                 28-O-β-D-glucop~
## # i 1,221 more rows
\end{verbatim}

\hypertarget{ux6839ux636e-bindingdb-ux6570ux636eux5e93ux7b5bux9009ux836fux7269ux548cux9776ux70b9ux7684ux7ed3ux5408ux53efux80fd}{%
\subsubsection{根据 BindingDB 数据库筛选药物和靶点的结合可能}\label{ux6839ux636e-bindingdb-ux6570ux636eux5e93ux7b5bux9009ux836fux7269ux548cux9776ux70b9ux7684ux7ed3ux5408ux53efux80fd}}

BindingDB 数据库记录了化合物和成分的亲和信息: \url{https://www.bindingdb.org/rwd/bind/chemsearch/marvin/Download.jsp}。

下载相关数据,以供后续筛选化合物和靶点。

\hypertarget{ux83b7ux53d6ux836fux7269ux5206ux5b50ux7ed3ux6784ux6570ux636eux548cux9884ux5904ux7406}{%
\subsection{获取药物分子结构数据和预处理}\label{ux83b7ux53d6ux836fux7269ux5206ux5b50ux7ed3ux6784ux6570ux636eux548cux9884ux5904ux7406}}

\hypertarget{par}{%
\subsubsection{根据 BindingDB 提供的结合可能筛选药物分子}\label{par}}

PubChem CID 连接到 PubChem 数据库\textsuperscript{\protect\hyperlink{ref-PubchemSubstanKimS2015}{3}}。

根据整理的药方成分的 PubChem CID 信息和 BindingDB 数据的 PubChem CID 信息过滤化合物。

\textbf{注意:} 药方统计有1097个唯一成分,而包含PubChem ID 的化合物共有7个。

以下为包含PubChem ID的化合物的概览:

\begin{verbatim}
## # A tibble: 796 x 3
##    Ingredient.id Ingredient.name                        PubChem_id
##    <chr>         <chr>                                       <int>
##  1 HBIN000177    10-o-acetylgeniposide                     6324916
##  2 HBIN000280    11,13-Eicosadienoic acid, methyl ester    5365674
##  3 HBIN000328    1,1,6-trimethyl-2H-naphthalene             121677
##  4 HBIN000391    11-deoxyglycyrrhetic acid                12305517
##  5 HBIN000392    11-deoxyglycyrrhetinic acid              12305517
##  6 HBIN000573    1,2,3,4,6-pentagalloylglucose               65238
##  7 HBIN000643    1,2,4-benzenetriol                          10787
##  8 HBIN001070    1,3,5-trihydroxyxanthone                  5281663
##  9 HBIN001080    13657-68-6                               14106072
## 10 HBIN001307    13-Tetradecenyl acetate                    521718
## # i 786 more rows
\end{verbatim}

这些数据结合 BindingDB 数据进一步过滤。

以下为 BindingDB 中记录有上述化合物的条目,包含了靶点蛋白的信息:
\textbf{(对应文件为 \texttt{BindingDB\_data\_filter\_by\_herb\_compounds.csv})}

\begin{verbatim}
## # A tibble: 3,909 x 2
##    PubChem_id pdb_id                                                                                                              
##         <int> <chr>                                                                                                               
##  1     689043 "1IVO,1M14,1M17,1MOX,1XKK,1Z9I,2GS2,2GS6,2ITW,2ITX,2ITY,2J5E,2J5F,2J6M,2N5S,2RF9,2RGP,3B2V,3BEL,3GOP,3POZ,3QWQ,3VJO~
##  2    5280443 "1G3N,1JOW,1XO2,2EUF,2F2C,3NUP,3NUX,4AUA,4EZ5,4TTH,5L2I,5L2S,5L2T,6OQL,6OQO"                                        
##  3    5280343 "1G3N,1JOW,1XO2,2EUF,2F2C,3NUP,3NUX,4AUA,4EZ5,4TTH,5L2I,5L2S,5L2T,6OQL,6OQO"                                        
##  4    5281607 "1G3N,1JOW,1XO2,2EUF,2F2C,3NUP,3NUX,4AUA,4EZ5,4TTH,5L2I,5L2S,5L2T,6OQL,6OQO"                                        
##  5    5280863 "1G3N,1JOW,1XO2,2EUF,2F2C,3NUP,3NUX,4AUA,4EZ5,4TTH,5L2I,5L2S,5L2T,6OQL,6OQO"                                        
##  6    5280443 "1UNG,1UNH,1UNL,3O0G,7VDP,7VDQ,7VDR,7VDS"                                                                           
##  7    5280343 "1UNG,1UNH,1UNL,3O0G,7VDP,7VDQ,7VDR,7VDS"                                                                           
##  8    5281607 "1UNG,1UNH,1UNL,3O0G,7VDP,7VDQ,7VDR,7VDS"                                                                           
##  9    5280863 "1UNG,1UNH,1UNL,3O0G,7VDP,7VDQ,7VDR,7VDS"                                                                           
## 10    5280443 ""                                                                                                                  
## # i 3,899 more rows
\end{verbatim}

将上述数据结合 MCC top 10 靶点蛋白的注释数据(包含PDB ID)进一步过滤。

(靶点蛋白的 PDB ID 的获取,参考 \ref{pdbID})

现在,已有化合物(PubChem CID)和对应的靶点蛋白(PDB ID)的信息:

\begin{verbatim}
## $`370`
## [1] "1CXW" "1GEN" "1RTG"
## 
## $`370`
## [1] "1C5G" "1DVN" "1LJ5" "4AQH"
## 
## $`73111`
## [1] "1C5G" "1DVN" "1LJ5" "4AQH"
## 
## $`5280704`
##  [1] "2AZ5" "3ALQ" "3IT8" "3L9J" "4G3Y" "4TWT" "5M2I" "5M2J" "5M2M" "5MU8" "5WUX" "5YOY" "6OOY" "6OOZ" "6X81" "6X82" "6X83" "6X85"
## [19] "6X86" "7JRA"
## 
## $`66065`
## [1] "1IL6" "1P9M" "2IL6" "4CNI" "4J4L" "4NI7" "4NI9" "4O9H" "4ZS7"
## 
## $`66065`
##  [1] "2AZ5" "3ALQ" "3IT8" "3L9J" "4G3Y" "4TWT" "5M2I" "5M2J" "5M2M" "5MU8" "5WUX" "5YOY" "6OOY" "6OOZ" "6X81" "6X82" "6X83" "6X85"
## [19] "6X86" "7JRA"
## 
## $`6476139`
## [1] "1CXW" "1GEN" "1RTG"
## 
## $`5280343`
## [1] "1CXW" "1GEN" "1RTG"
## 
## $`5280343`
##  [1] "1L6J" "2OVX" "2OVZ" "2OW0" "2OW1" "2OW2" "4H1Q" "4H82" "4HMA" "4WZV" "5TH6" "5TH9" "6ESM"
## 
## $`689043`
##  [1] "1L6J" "2OVX" "2OVZ" "2OW0" "2OW1" "2OW2" "4H1Q" "4H82" "4HMA" "4WZV" "5TH6" "5TH9" "6ESM"
## 
## $`689043`
## [1] "1CXW" "1GEN" "1RTG"
## 
## $`689043`
##  [1] "1L6J" "2OVX" "2OVZ" "2OW0" "2OW1" "2OW2" "4H1Q" "4H82" "4HMA" "4WZV" "5TH6" "5TH9" "6ESM"
## 
## $`5280343`
##  [1] "1L6J" "2OVX" "2OVZ" "2OW0" "2OW1" "2OW2" "4H1Q" "4H82" "4HMA" "4WZV" "5TH6" "5TH9" "6ESM"
\end{verbatim}

\hypertarget{ux83b7ux53d6ux5316ux5408ux7269ux5206ux5b50ux7684-sdf-ux6570ux636e}{%
\subsubsection{获取化合物分子的 SDF 数据}\label{ux83b7ux53d6ux5316ux5408ux7269ux5206ux5b50ux7684-sdf-ux6570ux636e}}

通过 PubChem ID 使用 PubChem API 获取官方 .sdf 文件(\url{https://pubchem.ncbi.nlm.nih.gov/docs/pug-rest/})\textsuperscript{\protect\hyperlink{ref-AnUpdateOnPuKimS2018}{4}}。

所有 .sdf 文件被整合。
\textbf{(对应文件为 \texttt{./SDFs/all\_compounds.sdf})}

\hypertarget{ux9884ux5904ux7406ux5316ux5408ux7269ux7684ux7ed3ux6784ux6570ux636e}{%
\subsubsection{预处理化合物的结构数据}\label{ux9884ux5904ux7406ux5316ux5408ux7269ux7684ux7ed3ux6784ux6570ux636e}}

使用 Python 的 Meeko 转化 .sdf 数据为 .pdbqt。

共有 7 个化合物被成功转化。
\textbf{(对应文件为 \texttt{./pdbqt})}

\hypertarget{ux83b7ux53d6ux9776ux70b9ux86cbux767dux6570ux636e}{%
\subsection{获取靶点蛋白数据}\label{ux83b7ux53d6ux9776ux70b9ux86cbux767dux6570ux636e}}

\hypertarget{pdbID}{%
\subsubsection{获取 MCC top 10 蛋白的PDB ID}\label{pdbID}}

使用 R \texttt{BiomaRt} 获取MCC 筛选的Top 10 蛋白的 PDB ID。

结果如下:
\textbf{(对应文件为 \texttt{MCC\_tops\_PDB\_ID.csv})}

\begin{verbatim}
## # A tibble: 215 x 2
##    hgnc_symbol pdb  
##    <chr>       <chr>
##  1 TNF         4Y6O 
##  2 TNF         3L9J 
##  3 TNF         1A8M 
##  4 TNF         1TNF 
##  5 TNF         2AZ5 
##  6 TNF         2E7A 
##  7 TNF         2TUN 
##  8 TNF         2ZJC 
##  9 TNF         2ZPX 
## 10 TNF         3ALQ 
## # i 205 more rows
\end{verbatim}

\textbf{注意},一个蛋白对应多种结构,对应有多个 PDB ID:

\begin{verbatim}
## Data of 10 
##   +++  Protein   1 +++
##   IL6
##     Sum: 12
##       pdb: 1P9M, 5FUC, 1ALU, 1IL6, ...
## 
##   +++  Protein   2 +++
##   IL1B
##     Sum: 59
##       pdb: 3O4O, 4DEP, 1HIB, 1I1B, ...
## 
##   +++  Protein   3 +++
##   TNF
##     Sum: 39
##       pdb: 4Y6O, 3L9J, 1A8M, 1TNF, ...
## 
##   +++  Protein   4 +++
##   MMP9
##     Sum: 25
##       pdb: 1GKC, 1GKD, 1ITV, 1L6J, ...
## 
##   +++  Protein   5 +++
##   CXCL8
##     Sum: 18
##       pdb: 1ILP, 1ILQ, 6XMN, 6LFM, ...
## 
##   +++  Protein   6 +++
##   TGFB1
##     Sum: 
##       No data.
##   +++  Protein   7 +++
##   MMP2
##     Sum: 11
##       pdb: 3AYU, 1CK7, 1CXW, 1EAK, ...
## 
##   +++  Protein   8 +++
##   IL10
##     Sum: 8
##       pdb: 1J7V, 1Y6K, 6X93, 1ILK, ...
## 
##   +++  Protein   9 +++
##   ICAM1
##     Sum: 14
##       pdb: 1MQ8, 3TCX, 1D3E, 1D3I, ...
## 
##   +++  Protein   10 +++
##   SERPINE1
##     Sum: 29
##       pdb: 5BRR, 5ZLZ, 3PB1, 1OC0, ...
\end{verbatim}

\hypertarget{ux6839ux636e-bindingdb-ux6570ux636eux63d0ux4f9bux7684ux7ed3ux5408ux53efux80fdux7b5bux9009ux9776ux70b9ux86cbux767dux7ed3ux6784}{%
\subsubsection{根据 BindingDB 数据提供的结合可能筛选靶点蛋白结构}\label{ux6839ux636e-bindingdb-ux6570ux636eux63d0ux4f9bux7684ux7ed3ux5408ux53efux80fdux7b5bux9009ux9776ux70b9ux86cbux767dux7ed3ux6784}}

由于同一个蛋白对应多个名称,根据 BindingDB 提供的结合可能筛选,减少计算量。

此步骤已经在 @ref\{par\} 中同步实现。

\hypertarget{ux83b7ux53d6ux9776ux70b9ux86cbux767dux7684-pdb-ux6587ux4ef6}{%
\subsubsection{获取靶点蛋白的 PDB 文件}\label{ux83b7ux53d6ux9776ux70b9ux86cbux767dux7684-pdb-ux6587ux4ef6}}

使用 RCSB PDB 提供的 API 获取 .pdb 文件。共有49个。
\url{https://www.rcsb.org/docs/programmatic-access/batch-downloads-with-shell-script}

\textbf{(对应文件为 \texttt{./protein\_pdb})}

\hypertarget{ux6839ux636eux79cdux65cfux8fc7ux6ee4ux9776ux70b9ux86cbux767d}{%
\subsubsection{根据种族过滤靶点蛋白}\label{ux6839ux636eux79cdux65cfux8fc7ux6ee4ux9776ux70b9ux86cbux767d}}

PDB 文件中记录有种族信息,根据种族(人种)过滤靶点蛋白(Regex match: ``ORGANISM\_SCIENTIFIC: HOMO SAPIENS;'')。

过滤前有 215 个文件,过滤后有 49 个文件。

\hypertarget{ux9884ux5904ux7406ux9776ux70b9ux86cbux767d-pdb-ux6587ux4ef6}{%
\subsubsection{预处理靶点蛋白 PDB 文件}\label{ux9884ux5904ux7406ux9776ux70b9ux86cbux767d-pdb-ux6587ux4ef6}}

使用 ADFR 工具给受体蛋白加氢并转化为 .pdbqt 文件。成功获取48个文件。

\textbf{(对应文件为 \texttt{./protein\_pdbqt})}

\hypertarget{ux4f7fux7528autodock-vinaux5206ux5b50ux5bf9ux63a5}{%
\subsection{使用AutoDock Vina分子对接}\label{ux4f7fux7528autodock-vinaux5206ux5b50ux5bf9ux63a5}}

\hypertarget{ux6240ux6709ux5206ux5b50ux548cux9776ux70b9ux86cbux767dux7ed3ux5408ux7684ux53efux80fdux6027}{%
\subsubsection{所有分子和靶点蛋白结合的可能性}\label{ux6240ux6709ux5206ux5b50ux548cux9776ux70b9ux86cbux767dux7ed3ux5408ux7684ux53efux80fdux6027}}

结合 \ref{par} 得到的对应关系以及最终获得的化合物和靶点蛋白的 .pdbqt 文件,共有以下121结合可能:

\begin{verbatim}
## # A tibble: 121 x 2
##    Ligand Receptor
##    <chr>  <chr>   
##  1 370    1cxw    
##  2 370    1gen    
##  3 370    1rtg    
##  4 370    1c5g    
##  5 370    1dvn    
##  6 370    1lj5    
##  7 370    4aqh    
##  8 73111  1c5g    
##  9 73111  1dvn    
## 10 73111  1lj5    
## # i 111 more rows
\end{verbatim}

\hypertarget{ux4f7fux7528-autodock-vina-ux5206ux5b50ux5bf9ux63a5}{%
\subsubsection{使用 AutoDock Vina 分子对接}\label{ux4f7fux7528-autodock-vina-ux5206ux5b50ux5bf9ux63a5}}

该步骤包括使用 ADFR 工具计算 affinity maps \url{https://ccsb.scripps.edu/adfr/}。

尽管已经通过多种方式筛选了化合物和蛋白的结合,依然有 121 种可能性。

vina 的一次计算时间约 0.5 分钟到数小时不等;此处设定了计算时间限制(3600 秒),超出时间限制将被强制取消。

所有可能都被计算,中途可能强制取消。结果文件和计算需要的分子或蛋白信息都被存储。
\textbf{(对应文件为 \texttt{./vina\_space})}
子目录的命名规则为:``PubChem ID'' + ``\_into\_'' + ``PDB ID''。
子目录下的更多文件的解释请参考:\url{https://autodock-vina.readthedocs.io/en/latest/docking_basic.html}

在121次计算中:

\begin{itemize}
\tightlist
\item
  成功计算(76次)
\item
  或时间限制或软件原因失败(45次)。
\end{itemize}

对接的结果概览(Affinity 单位为 kcal/mol,值越低,结合程度越好):
\textbf{(对应文件为 \texttt{./results\_of\_batch\_docking.csv})}

\begin{verbatim}
## # A tibble: 76 x 9
##    PDB_ID PubChem_id Affinity dir                          file                    Combn Ingredient.id Ingredient.name hgnc_symbol
##    <chr>       <int>    <dbl> <chr>                        <chr>                   <chr> <chr>         <chr>           <chr>      
##  1 1lj5        73111    -8.94 vina_space/73111_into_1lj5   vina_space/73111_into_~ 7311~ HBIN043755    sennoside a     SERPINE1   
##  2 6x81      5280704    -7.81 vina_space/5280704_into_6x81 vina_space/5280704_int~ 5280~ HBIN021590    Cosmetin        TNF        
##  3 6x82      5280704    -7.2  vina_space/5280704_into_6x82 vina_space/5280704_int~ 5280~ HBIN021590    Cosmetin        TNF        
##  4 6ooz      5280704    -7.05 vina_space/5280704_into_6ooz vina_space/5280704_int~ 5280~ HBIN021590    Cosmetin        TNF        
##  5 6ooy      5280704    -6.90 vina_space/5280704_into_6ooy vina_space/5280704_int~ 5280~ HBIN021590    Cosmetin        TNF        
##  6 6x81        66065    -6.13 vina_space/66065_into_6x81   vina_space/66065_into_~ 6606~ HBIN017919    bergenin        TNF        
##  7 3it8      5280704    -6.10 vina_space/5280704_into_3it8 vina_space/5280704_int~ 5280~ HBIN021590    Cosmetin        TNF        
##  8 6ooz        66065    -6.02 vina_space/66065_into_6ooz   vina_space/66065_into_~ 6606~ HBIN017919    bergenin        TNF        
##  9 6x83      5280704    -6.00 vina_space/5280704_into_6x83 vina_space/5280704_int~ 5280~ HBIN021590    Cosmetin        TNF        
## 10 6x85      5280704    -6.00 vina_space/5280704_into_6x85 vina_space/5280704_int~ 5280~ HBIN021590    Cosmetin        TNF        
## # i 66 more rows
\end{verbatim}

可视化如下(根据靶点蛋白去重复):
\textbf{(对应文件为 \texttt{./figs/Docking\_Affinity.pdf})}

\begin{figure}
\centering
\includegraphics{./figs/Docking_Affinity.pdf}
\caption{\label{fig:fig2}Molecular Docking Affinity}
\end{figure}

\hypertarget{ux5bf9ux63a5ux53efux89c6ux5316}{%
\subsubsection{对接可视化}\label{ux5bf9ux63a5ux53efux89c6ux5316}}

使用 PyMol 工具将结果可视化\textsuperscript{\protect\hyperlink{ref-LigandDockingSeelig2010}{5}}。
(对应文件存储在 \texttt{vina\_space} 目录下,png 文件,共 76 个)。
在 Figure \ref{fig:fig2} 中展示的结果被保存在 \texttt{./figs} 文件夹。
\textbf{(对应文件为 \texttt{./figs/66065\_into\_6x81.png}, \texttt{./figs})}

以下 Figure \ref{fig:fig2} 展示 Figure \ref{fig:fig3} 中排名最高的结果:

\def\@captype{figure}
\includegraphics[width=9.29in]{thesis_fig/73111_into_1lj5} \caption{Visualization of Molecular docking}\label{fig:fig3}
\makeatletter

\hypertarget{bibliography}{%
\section*{Reference}\label{bibliography}}
\addcontentsline{toc}{section}{Reference}

\hypertarget{refs}{}
\begin{cslreferences}
\leavevmode\hypertarget{ref-AutodockVina1Eberha2021}{}%
1. Eberhardt, J., Santos-Martins, D., Tillack, A. F. \& Forli, S. AutoDock vina 1.2.0: New docking methods, expanded force field, and python bindings. \emph{Journal of Chemical Information and Modeling} \textbf{61}, 3891--3898 (2021).

\leavevmode\hypertarget{ref-ComputationalPForli2016}{}%
2. Forli, S. \emph{et al.} Computational proteinligand docking and virtual drug screening with the autodock suite. \emph{Nature Protocols} \textbf{11}, (2016).

\leavevmode\hypertarget{ref-PubchemSubstanKimS2015}{}%
3. Kim, S. \emph{et al.} PubChem substance and compound databases. \emph{Nucleic Acids Research} (2015).

\leavevmode\hypertarget{ref-AnUpdateOnPuKimS2018}{}%
4. Kim, S., Thiessen, P. A., Cheng, T., Yu, B. \& Bolton, E. E. An update on pug-rest: RESTful interface for programmatic access to pubchem. \emph{Nucleic Acids Research} \textbf{46}, (2018).

\leavevmode\hypertarget{ref-LigandDockingSeelig2010}{}%
5. Seeliger, D. \& Groot, B. L. de. Ligand docking and binding site analysis with pymol and autodock/vina. \emph{Journal of Computer-Aided Molecular Design} \textbf{24}, (2010).
\end{cslreferences}

\end{document}
