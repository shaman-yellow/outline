% Options for packages loaded elsewhere
\PassOptionsToPackage{unicode}{hyperref}
\PassOptionsToPackage{hyphens}{url}
%
\documentclass[
]{article}
\usepackage{lmodern}
\usepackage{amssymb,amsmath}
\usepackage{ifxetex,ifluatex}
\ifnum 0\ifxetex 1\fi\ifluatex 1\fi=0 % if pdftex
  \usepackage[T1]{fontenc}
  \usepackage[utf8]{inputenc}
  \usepackage{textcomp} % provide euro and other symbols
\else % if luatex or xetex
  \usepackage{unicode-math}
  \defaultfontfeatures{Scale=MatchLowercase}
  \defaultfontfeatures[\rmfamily]{Ligatures=TeX,Scale=1}
\fi
% Use upquote if available, for straight quotes in verbatim environments
\IfFileExists{upquote.sty}{\usepackage{upquote}}{}
\IfFileExists{microtype.sty}{% use microtype if available
  \usepackage[]{microtype}
  \UseMicrotypeSet[protrusion]{basicmath} % disable protrusion for tt fonts
}{}
\makeatletter
\@ifundefined{KOMAClassName}{% if non-KOMA class
  \IfFileExists{parskip.sty}{%
    \usepackage{parskip}
  }{% else
    \setlength{\parindent}{0pt}
    \setlength{\parskip}{6pt plus 2pt minus 1pt}}
}{% if KOMA class
  \KOMAoptions{parskip=half}}
\makeatother
\usepackage{xcolor}
\IfFileExists{xurl.sty}{\usepackage{xurl}}{} % add URL line breaks if available
\IfFileExists{bookmark.sty}{\usepackage{bookmark}}{\usepackage{hyperref}}
\hypersetup{
  pdftitle={Report of Analysis},
  pdfauthor={Huang LiChuang of Wie-Biotech},
  hidelinks,
  pdfcreator={LaTeX via pandoc}}
\urlstyle{same} % disable monospaced font for URLs
\usepackage[margin=1in]{geometry}
\usepackage{longtable,booktabs}
% Correct order of tables after \paragraph or \subparagraph
\usepackage{etoolbox}
\makeatletter
\patchcmd\longtable{\par}{\if@noskipsec\mbox{}\fi\par}{}{}
\makeatother
% Allow footnotes in longtable head/foot
\IfFileExists{footnotehyper.sty}{\usepackage{footnotehyper}}{\usepackage{footnote}}
\makesavenoteenv{longtable}
\usepackage{graphicx}
\makeatletter
\def\maxwidth{\ifdim\Gin@nat@width>\linewidth\linewidth\else\Gin@nat@width\fi}
\def\maxheight{\ifdim\Gin@nat@height>\textheight\textheight\else\Gin@nat@height\fi}
\makeatother
% Scale images if necessary, so that they will not overflow the page
% margins by default, and it is still possible to overwrite the defaults
% using explicit options in \includegraphics[width, height, ...]{}
\setkeys{Gin}{width=\maxwidth,height=\maxheight,keepaspectratio}
% Set default figure placement to htbp
\makeatletter
\def\fps@figure{htbp}
\makeatother
\setlength{\emergencystretch}{3em} % prevent overfull lines
\providecommand{\tightlist}{%
  \setlength{\itemsep}{0pt}\setlength{\parskip}{0pt}}
\setcounter{secnumdepth}{5}
\usepackage{caption} \captionsetup{font={footnotesize},width=6in} \renewcommand{\dblfloatpagefraction}{.9} \makeatletter \renewenvironment{figure} {\def\@captype{figure}} \makeatother \definecolor{shadecolor}{RGB}{242,242,242} \usepackage{xeCJK} \usepackage{setspace} \setstretch{1.3} \usepackage{tcolorbox}
\newlength{\cslhangindent}
\setlength{\cslhangindent}{1.5em}
\newenvironment{cslreferences}%
  {}%
  {\par}

\title{Report of Analysis}
\author{Huang LiChuang of Wie-Biotech}
\date{}

\begin{document}
\maketitle

{
\setcounter{tocdepth}{3}
\tableofcontents
}
\listoffigures

\listoftables

\hypertarget{abstract}{%
\section{摘要}\label{abstract}}

根据客户提供的材料分析基因突变及信号通路,筛选出研究的对象基因。相关疾病是``高胆汁酸血症''或是``妊娠期肝内胆汁淤积症(Intrahepatic cholestasis of pregnancy,ICP)

\hypertarget{fastp-ux8d28ux63a7}{%
\subsection{fastp 质控}\label{fastp-ux8d28ux63a7}}

\begin{itemize}
\tightlist
\item
  去低质量碱基
\item
  去接头
\item
  生成报告
\end{itemize}

\hypertarget{workflow}{%
\subsection{全外显子分析流程}\label{workflow}}

WES 一般分析流程为:

\url{https://gatk.broadinstitute.org/hc/en-us/sections/360007226651-Best-Practices-Workflows}:

\begin{itemize}
\tightlist
\item
  Preprocessing \url{https://gatk.broadinstitute.org/hc/en-us/articles/360035535912-Data-pre-processingfor-variant-discovery}

  \begin{itemize}
  \tightlist
  \item
    比对到参考基因组
  \item
    标记重复
  \item
    基础校准(Base (Quality Score) Recalibration)
  \end{itemize}
\item
  Variant discovery \url{https://gatk.broadinstitute.org/hc/en-us/articles/360035535932-Germline-shortvariant-discovery-SNPs-Indels-}

  \begin{itemize}
  \tightlist
  \item
    获取变异注释文件
  \item
    变异检测
  \item
    变异质控和过滤
  \item
    变异注释
  \end{itemize}
\end{itemize}

\hypertarget{ux7ed3ux679cux53efux89c6ux5316}{%
\subsection{结果可视化}\label{ux7ed3ux679cux53efux89c6ux5316}}

使用 maftools 对变异注释结果可视化。

\hypertarget{route}{%
\section{研究设计流程图}\label{route}}

\includegraphics[width=\linewidth]{output_files/figure-latex/unnamed-chunk-4-1}

\hypertarget{methods}{%
\section{材料和方法}\label{methods}}

\begin{itemize}
\tightlist
\item
  fastp (\url{https://github.com/OpenGene/fastp})
\end{itemize}

以下可以通过 \url{https://gatk.broadinstitute.org/hc/en-us/articles/360041320571--How-to-Install-all-software-packages-required-to-follow-the-GATK-Best-Practices} 获取安装。

\begin{itemize}
\tightlist
\item
  bwa
\item
  \ldots{}
\end{itemize}

使用 elPrep\textsuperscript{\protect\hyperlink{ref-MultithreadedV2021}{1}} 替代 GATK4 做 WES 分析(见 \ref{workflow})。

使用 bcftools\textsuperscript{\protect\hyperlink{ref-TwelveYearsOfDanece2021}{2}} 过滤 vcf。

使用 ANNOVAR 变异注释。

使用 R maftools 可视化 ANNOVAR 注释结果。

使用 clusterProfiler 富集分析(KEGG)。

参考基因组:

\begin{itemize}
\tightlist
\item
  \url{https://hgdownload.soe.ucsc.edu/goldenPath/hg38/bigZips/latest/hg38.fa.gz}
\end{itemize}

SNPs 和 Indels:

(\url{https://console.cloud.google.com/storage/browser/genomics-public-data/resources/broad/hg38/v0})

\begin{itemize}
\tightlist
\item
  \texttt{1000G\_phase1.snps.high\_confidence.hg38.vcf}
\item
  \texttt{Mills\_and\_1000G\_gold\_standard.indels.hg38.vcf}
\end{itemize}

\hypertarget{results}{%
\section{分析结果}\label{results}}

\hypertarget{fastp-ux8d28ux63a7-1}{%
\subsection{fastp 质控}\label{fastp-ux8d28ux63a7-1}}

`Fastp report files' 数据已全部提供。

\textbf{(对应文件为 \texttt{./fastp\_report})}

\begin{center}\begin{tcolorbox}[colback=gray!10, colframe=gray!50, width=0.9\linewidth, arc=1mm, boxrule=0.5pt]注:文件夹./fastp\_report共包含6个文件。

\begin{enumerate}\tightlist
\item V350065014\_L01\_93\_.html
\item V350065014\_L01\_94\_.html
\item V350065014\_L02\_94\_.html
\item V350065026\_L03\_86\_.html
\item V350065026\_L04\_85\_.html
\item ...
\end{enumerate}\end{tcolorbox}
\end{center}

注:客户提供 8 个病人的数据,每个病人的目录下有 2 个子文件,因此共 16 个样本数据。硬盘中有个别 fastq 文件有损坏。损坏的文件未纳入分析流程中(没有报告生成的为损坏的文件)。

\hypertarget{wes-ux53d8ux5f02ux7b5bux9009}{%
\subsection{WES 变异筛选}\label{wes-ux53d8ux5f02ux7b5bux9009}}

以下流程相较于 GATK4 Best Practice (\url{https://gatk.broadinstitute.org/hc/en-us/sections/360007226651-Best-Practices-Workflows}) 有所变化。

\begin{itemize}
\tightlist
\item
  使用 elPrep 5 完成检测流程(流程类似于 GATK4,但速度更快)\textsuperscript{\protect\hyperlink{ref-MultithreadedV2021}{1}}。
\end{itemize}

得到变异信息文件(vcf)后,使用 bcftools 过滤(QUAL\textgreater10 \&\& GQ\textgreater10 \&\& FORMAT/DP\textgreater10 \&\& INFO/DP\textgreater100)。

\hypertarget{annovar-res}{%
\subsubsection{ANNOVAR 注释}\label{annovar-res}}

使用 ANNOVAR (\url{https://annovar.openbioinformatics.org/en/latest/}) 注释后,滤除同义突变。

`Exonic annotation by ANNOVAR' 数据已全部提供。

\textbf{(对应文件为 \texttt{exonic-annotation-by-ANNOVAR})}

\begin{center}\begin{tcolorbox}[colback=gray!10, colframe=gray!50, width=0.9\linewidth, arc=1mm, boxrule=0.5pt]注:文件夹exonic-annotation-by-ANNOVAR共包含6个文件。

\begin{enumerate}\tightlist
\item 1\_X220325\_I26\_V350065014\_L1\_22L01298712.93.csv
\item 10\_X220325\_M038\_V350065181\_L04\_22L01298713.24.csv
\item 11\_X220325\_M038\_V350065181\_L04\_22L01298716.25.csv
\item 12\_X220325\_M120\_V350065026\_L03\_22L01298714.86.csv
\item 13\_X220325\_M120\_V350065026\_L04\_22L01298711.85.csv
\item ...
\end{enumerate}\end{tcolorbox}
\end{center}

\hypertarget{maftools-ux53efux89c6ux5316}{%
\subsubsection{\texorpdfstring{\texttt{maftools} 可视化}{maftools 可视化}}\label{maftools-ux53efux89c6ux5316}}

参考 \url{https://www.bioconductor.org/packages/release/bioc/vignettes/maftools/inst/doc/maftools.html\#910_Mutational_Signatures}

Figure \ref{fig:summary-of-mutations-in-samples}为图summary of mutations in samples概览。

\textbf{(对应文件为 \texttt{Figure+Table/summary-of-mutations-in-samples.pdf})}

\def\@captype{figure}
\begin{center}
\includegraphics[width = 0.9\linewidth]{Figure+Table/summary-of-mutations-in-samples.pdf}
\caption{Summary of mutations in samples}\label{fig:summary-of-mutations-in-samples}
\end{center}

Figure \ref{fig:proportion-of-SNPs-mutation}为图proportion of SNPs mutation概览。

\textbf{(对应文件为 \texttt{Figure+Table/proportion-of-SNPs-mutation.pdf})}

\def\@captype{figure}
\begin{center}
\includegraphics[width = 0.9\linewidth]{Figure+Table/proportion-of-SNPs-mutation.pdf}
\caption{Proportion of SNPs mutation}\label{fig:proportion-of-SNPs-mutation}
\end{center}

\hypertarget{ux4e0bux6e38ux5206ux6790}{%
\subsection{下游分析}\label{ux4e0bux6e38ux5206ux6790}}

\hypertarget{ux83b7ux53d6-genecards-ux4e0eux80c6ux6c41ux76f8ux5173ux75beux75c5ux7684ux57faux56e0}{%
\subsubsection{获取 Genecards 与胆汁相关疾病的基因}\label{ux83b7ux53d6-genecards-ux4e0eux80c6ux6c41ux76f8ux5173ux75beux75c5ux7684ux57faux56e0}}

Table \ref{tab:Genecards-genes-relative-with-bild-acids}为表格Genecards genes relative with bild acids概览。

\textbf{(对应文件为 \texttt{Figure+Table/Genecards-genes-relative-with-bild-acids.xlsx})}

\begin{center}\begin{tcolorbox}[colback=gray!10, colframe=gray!50, width=0.9\linewidth, arc=1mm, boxrule=0.5pt]注:表格共有1494行7列,以下预览的表格可能省略部分数据;表格含有1494个唯一`Symbol'。
\end{tcolorbox}
\end{center}

\begin{longtable}[]{@{}lllllll@{}}
\caption{\label{tab:Genecards-genes-relative-with-bild-acids}Genecards genes relative with bild acids}\tabularnewline
\toprule
Symbol & Descr\ldots{} & Category & UniPr\ldots{} & GIFtS & GC\_id & Score\tabularnewline
\midrule
\endfirsthead
\toprule
Symbol & Descr\ldots{} & Category & UniPr\ldots{} & GIFtS & GC\_id & Score\tabularnewline
\midrule
\endhead
BAAT & Bile \ldots{} & Prote\ldots{} & Q14032 & 47 & GC09M\ldots{} & 93.50\tabularnewline
AKR1D1 & Aldo-\ldots{} & Prote\ldots{} & P51857 & 48 & GC07P\ldots{} & 88.14\tabularnewline
ABCB11 & ATP B\ldots{} & Prote\ldots{} & O95342 & 50 & GC02M\ldots{} & 85.41\tabularnewline
SLC10A1 & Solut\ldots{} & Prote\ldots{} & Q14973 & 47 & GC14M\ldots{} & 83.40\tabularnewline
NR1H4 & Nucle\ldots{} & Prote\ldots{} & Q96RI1 & 52 & GC12P\ldots{} & 83.18\tabularnewline
AMACR & Alpha\ldots{} & Prote\ldots{} & Q9UHK6 & 49 & GC05M\ldots{} & 76.57\tabularnewline
HSD3B7 & Hydro\ldots{} & Prote\ldots{} & Q9H2F3 & 45 & GC16P\ldots{} & 70.66\tabularnewline
CYP7B1 & Cytoc\ldots{} & Prote\ldots{} & O75881 & 50 & GC08M\ldots{} & 65.39\tabularnewline
GPBAR1 & G Pro\ldots{} & Prote\ldots{} & Q8TDU6 & 44 & GC02P\ldots{} & 63.83\tabularnewline
CYP7A1 & Cytoc\ldots{} & Prote\ldots{} & P22680 & 47 & GC08M\ldots{} & 60.99\tabularnewline
ACOX2 & Acyl-\ldots{} & Prote\ldots{} & Q99424 & 47 & GC03M\ldots{} & 59.23\tabularnewline
SLC51B & Solut\ldots{} & Prote\ldots{} & Q86UW2 & 38 & GC15P\ldots{} & 58.25\tabularnewline
ABCB4 & ATP B\ldots{} & Prote\ldots{} & P21439 & 51 & GC07M\ldots{} & 56.92\tabularnewline
SLC27A5 & Solut\ldots{} & Prote\ldots{} & Q9Y2P5 & 45 & GC19M\ldots{} & 55.16\tabularnewline
ALB & Albumin & Prote\ldots{} & P02768 & 53 & GC04P\ldots{} & 51.11\tabularnewline
\ldots{} & \ldots{} & \ldots{} & \ldots{} & \ldots{} & \ldots{} & \ldots{}\tabularnewline
\bottomrule
\end{longtable}

\hypertarget{ux901aux8defux5bccux96c6ux5206ux6790}{%
\subsubsection{通路富集分析}\label{ux901aux8defux5bccux96c6ux5206ux6790}}

取 Tab. \ref{tab:Genecards-genes-relative-with-bild-acids} 的基因与 \ref{tab:annovar-res} 的所有基因的交集。

Figure \ref{fig:intersect-of-variants-with-Genecards-prediction}为图intersect of variants with Genecards prediction概览。

\textbf{(对应文件为 \texttt{Figure+Table/intersect-of-variants-with-Genecards-prediction.pdf})}

\def\@captype{figure}
\begin{center}
\includegraphics[width = 0.9\linewidth]{Figure+Table/intersect-of-variants-with-Genecards-prediction.pdf}
\caption{Intersect of variants with Genecards prediction}\label{fig:intersect-of-variants-with-Genecards-prediction}
\end{center}

以交集基因通路富集。

Figure \ref{fig:KEGG-enrichment}为图KEGG enrichment概览。

\textbf{(对应文件为 \texttt{Figure+Table/KEGG-enrichment.pdf})}

\def\@captype{figure}
\begin{center}
\includegraphics[width = 0.9\linewidth]{Figure+Table/KEGG-enrichment.pdf}
\caption{KEGG enrichment}\label{fig:KEGG-enrichment}
\end{center}

取 `Bile secretion' 和 `Cholesterol metabolism' 相关的基因。

Figure \ref{fig:Intersection-of-filtered-variants-with-KEGG-pathway}为图Intersection of filtered variants with KEGG pathway概览。

\textbf{(对应文件为 \texttt{Figure+Table/Intersection-of-filtered-variants-with-KEGG-pathway.pdf})}

\def\@captype{figure}
\begin{center}
\includegraphics[width = 0.9\linewidth]{Figure+Table/Intersection-of-filtered-variants-with-KEGG-pathway.pdf}
\caption{Intersection of filtered variants with KEGG pathway}\label{fig:Intersection-of-filtered-variants-with-KEGG-pathway}
\end{center}

取 Fig. \ref{fig:Intersection-of-filtered-variants-with-KEGG-pathway} 所示的两条通路的基因交集。

将这些交集基因回归到所有样本的变异数据中,取共同发生的突变结果。

Figure \ref{fig:intersects-of-the-pathways-related-variants-in-all-samples}为图intersects of the pathways related variants in all samples概览。

\textbf{(对应文件为 \texttt{Figure+Table/intersects-of-the-pathways-related-variants-in-all-samples.pdf})}

\def\@captype{figure}
\begin{center}
\includegraphics[width = 0.9\linewidth]{Figure+Table/intersects-of-the-pathways-related-variants-in-all-samples.pdf}
\caption{Intersects of the pathways related variants in all samples}\label{fig:intersects-of-the-pathways-related-variants-in-all-samples}
\end{center}

有 7 个变异同时发生在所有样本中。

Table \ref{tab:Bile-acids-related-variants-occurs-in-all-ICP-samples}为表格Bile acids related variants occurs in all ICP samples概览。

\textbf{(对应文件为 \texttt{Figure+Table/Bile-acids-related-variants-occurs-in-all-ICP-samples.xlsx})}

\begin{center}\begin{tcolorbox}[colback=gray!10, colframe=gray!50, width=0.9\linewidth, arc=1mm, boxrule=0.5pt]注:表格共有7行14列,以下预览的表格可能省略部分数据;表格含有7个唯一`hgnc\_symbol'。
\end{tcolorbox}
\end{center}

\begin{longtable}[]{@{}llllllllllllll@{}}
\caption{\label{tab:Bile-acids-related-variants-occurs-in-all-ICP-samples}Bile acids related variants occurs in all ICP samples}\tabularnewline
\toprule
hgnc\_\ldots{} & prote\ldots{} & Chr & Start & End & Ref & Alt & Func\ldots. & Gene\ldots. & GeneD\ldots{} & Exoni\ldots{} & AACha\ldots{} & cytoBand & \ldots{}\tabularnewline
\midrule
\endfirsthead
\toprule
hgnc\_\ldots{} & prote\ldots{} & Chr & Start & End & Ref & Alt & Func\ldots. & Gene\ldots. & GeneD\ldots{} & Exoni\ldots{} & AACha\ldots{} & cytoBand & \ldots{}\tabularnewline
\midrule
\endhead
LRP1 & p.Q2900P & chr12 & 57196001 & 57196001 & A & C & exonic & LRP1 & & nonsy\ldots{} & LRP1:\ldots{} & 12q13.3 & \ldots{}\tabularnewline
SLC10A1 & p.S267F & chr14 & 69778476 & 69778476 & G & A & exonic & SLC10A1 & & nonsy\ldots{} & SLC10\ldots{} & 14q24.1 & \ldots{}\tabularnewline
AQP9 & p.T214A & chr15 & 58184082 & 58184082 & A & G & exonic & AQP9 & & nonsy\ldots{} & AQP9:\ldots{} & 15q21.3 & \ldots{}\tabularnewline
APOH & p.V266L & chr17 & 66214639 & 66214639 & C & A & exonic & APOH & & nonsy\ldots{} & APOH:\ldots{} & 17q24.2 & \ldots{}\tabularnewline
ABCB11 & p.V444A & chr2 & 16897\ldots{} & 16897\ldots{} & A & G & exonic & ABCB11 & & nonsy\ldots{} & ABCB1\ldots{} & 2q31.1 & \ldots{}\tabularnewline
LRP2 & p.A2872T & chr2 & 16919\ldots{} & 16919\ldots{} & C & T & exonic & LRP2 & & nonsy\ldots{} & LRP2:\ldots{} & 2q31.1 & \ldots{}\tabularnewline
TSPO & p.T147A & chr22 & 43162920 & 43162920 & A & G & exonic & TSPO & & nonsy\ldots{} & TSPO:\ldots{} & 22q13.2 & \ldots{}\tabularnewline
\bottomrule
\end{longtable}

\hypertarget{dis}{%
\section{结论}\label{dis}}

见 Tab. \ref{tab:Bile-acids-related-variants-occurs-in-all-ICP-samples}。

ICP 相关对象基因:

LRP1, SLC10A1, AQP9, APOH, ABCB11, LRP2, TSPO。

突变形式 (hgvs) 为:

p.Q2900P, p.S267F, p.T214A, p.V266L, p.V444A, p.A2872T, p.T147A。

\hypertarget{ux5176ux5b83}{%
\section{其它}\label{ux5176ux5b83}}

\hypertarget{ux65b0ux751fux513fux5fc3ux810fux9aa4ux505c}{%
\subsection{新生儿心脏骤停}\label{ux65b0ux751fux513fux5fc3ux810fux9aa4ux505c}}

\hypertarget{ux6570ux636eux6765ux6e90}{%
\subsubsection{数据来源}\label{ux6570ux636eux6765ux6e90}}

检索:neonatal cardiac arrest

数据来源于\textsuperscript{\protect\hyperlink{ref-TranscriptomePTuLa2019}{3}} (piglets mRNA-seq):

\begin{itemize}
\tightlist
\item
  PMID: 31005300
\item
  GSE120863
\end{itemize}

\begin{center}\begin{tcolorbox}[colback=gray!10, colframe=gray!50, width=0.9\linewidth, arc=1mm, boxrule=0.5pt]
\textbf{
data\_processing
:}

\vspace{0.5em}

    RNA sequencing reads were aligned to the Pig genome
(Sscrofa11.1) using Star.

\vspace{2em}


\textbf{
data\_processing.1
:}

\vspace{0.5em}

    featureCounts was used to map the reads to the exons of
genes.

\vspace{2em}


\textbf{
data\_processing.2
:}

\vspace{0.5em}

    DESeq2 was used to normalize the data using
regularized-logarithm.

\vspace{2em}


\textbf{
data\_processing.3
:}

\vspace{0.5em}

    Genome\_build: Sscrofa11.1

\vspace{2em}


\textbf{
data\_processing.4
:}

\vspace{0.5em}

    Supplementary\_files\_format\_and\_content:
0618\_striatum.feature\_counts.all\_samples.txt includes the
counts.

\vspace{2em}


\textbf{
data\_processing.5
:}

\vspace{0.5em}

    Supplementary\_files\_format\_and\_content:
sham\_DHCA\_striatum\_mRNA\_seq.txt includes the fold changes
and statistics.

\vspace{2em}
\end{tcolorbox}
\end{center}

\hypertarget{ux65b0ux751fux513fux80ceux7caaux6027ux8179ux819cux708eux5deeux5f02ux57faux56e0}{%
\subsection{新生儿胎粪性腹膜炎差异基因}\label{ux65b0ux751fux513fux80ceux7caaux6027ux8179ux819cux708eux5deeux5f02ux57faux56e0}}

检索:neonatal meconium peritonitis

无相关数据。

\hypertarget{ux80ceux513fux5babux5185ux7a98ux8feb}{%
\subsection{胎儿宫内窘迫}\label{ux80ceux513fux5babux5185ux7a98ux8feb}}

检索:fetal distress

无相关数据。

\hypertarget{ux6b7bux80ce}{%
\subsection{死胎}\label{ux6b7bux80ce}}

检索:stillbirth

无相关数据。

\hypertarget{ux65b0ux751fux513fux547cux5438ux7a98ux8febux7efcux5408ux5f81}{%
\subsection{新生儿呼吸窘迫综合征}\label{ux65b0ux751fux513fux547cux5438ux7a98ux8febux7efcux5408ux5f81}}

neonatal respiratory distress syndrome

无相关数据。

\hypertarget{bibliography}{%
\section*{Reference}\label{bibliography}}
\addcontentsline{toc}{section}{Reference}

\hypertarget{refs}{}
\begin{cslreferences}
\leavevmode\hypertarget{ref-MultithreadedV2021}{}%
1. Multithreaded variant calling in elPrep 5. \emph{PLOS ONE} \textbf{16}, 1--13 (2021).

\leavevmode\hypertarget{ref-TwelveYearsOfDanece2021}{}%
2. Danecek, P. \emph{et al.} Twelve years of samtools and bcftools. \emph{GigaScience} \textbf{10}, (2021).

\leavevmode\hypertarget{ref-TranscriptomePTuLa2019}{}%
3. Tu, L. N. \emph{et al.} Transcriptome profiling reveals activation of inflammation and apoptosis in the neonatal striatum after deep hypothermic circulatory arrest. \emph{The Journal of Thoracic and Cardiovascular Surgery} \textbf{158}, (2019).
\end{cslreferences}

\end{document}
