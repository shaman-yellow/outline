% Options for packages loaded elsewhere
\PassOptionsToPackage{unicode}{hyperref}
\PassOptionsToPackage{hyphens}{url}
%
\documentclass[
]{article}
\usepackage{lmodern}
\usepackage{amssymb,amsmath}
\usepackage{ifxetex,ifluatex}
\ifnum 0\ifxetex 1\fi\ifluatex 1\fi=0 % if pdftex
  \usepackage[T1]{fontenc}
  \usepackage[utf8]{inputenc}
  \usepackage{textcomp} % provide euro and other symbols
\else % if luatex or xetex
  \usepackage{unicode-math}
  \defaultfontfeatures{Scale=MatchLowercase}
  \defaultfontfeatures[\rmfamily]{Ligatures=TeX,Scale=1}
\fi
% Use upquote if available, for straight quotes in verbatim environments
\IfFileExists{upquote.sty}{\usepackage{upquote}}{}
\IfFileExists{microtype.sty}{% use microtype if available
  \usepackage[]{microtype}
  \UseMicrotypeSet[protrusion]{basicmath} % disable protrusion for tt fonts
}{}
\makeatletter
\@ifundefined{KOMAClassName}{% if non-KOMA class
  \IfFileExists{parskip.sty}{%
    \usepackage{parskip}
  }{% else
    \setlength{\parindent}{0pt}
    \setlength{\parskip}{6pt plus 2pt minus 1pt}}
}{% if KOMA class
  \KOMAoptions{parskip=half}}
\makeatother
\usepackage{xcolor}
\IfFileExists{xurl.sty}{\usepackage{xurl}}{} % add URL line breaks if available
\IfFileExists{bookmark.sty}{\usepackage{bookmark}}{\usepackage{hyperref}}
\hypersetup{
  hidelinks,
  pdfcreator={LaTeX via pandoc}}
\urlstyle{same} % disable monospaced font for URLs
\usepackage[margin=1in]{geometry}
\usepackage{longtable,booktabs}
% Correct order of tables after \paragraph or \subparagraph
\usepackage{etoolbox}
\makeatletter
\patchcmd\longtable{\par}{\if@noskipsec\mbox{}\fi\par}{}{}
\makeatother
% Allow footnotes in longtable head/foot
\IfFileExists{footnotehyper.sty}{\usepackage{footnotehyper}}{\usepackage{footnote}}
\makesavenoteenv{longtable}
\usepackage{graphicx}
\makeatletter
\def\maxwidth{\ifdim\Gin@nat@width>\linewidth\linewidth\else\Gin@nat@width\fi}
\def\maxheight{\ifdim\Gin@nat@height>\textheight\textheight\else\Gin@nat@height\fi}
\makeatother
% Scale images if necessary, so that they will not overflow the page
% margins by default, and it is still possible to overwrite the defaults
% using explicit options in \includegraphics[width, height, ...]{}
\setkeys{Gin}{width=\maxwidth,height=\maxheight,keepaspectratio}
% Set default figure placement to htbp
\makeatletter
\def\fps@figure{htbp}
\makeatother
\setlength{\emergencystretch}{3em} % prevent overfull lines
\providecommand{\tightlist}{%
  \setlength{\itemsep}{0pt}\setlength{\parskip}{0pt}}
\setcounter{secnumdepth}{5}
\usepackage{tikz} \usepackage{auto-pst-pdf} \usepackage{pgfornament} \usepackage{pstricks-add} \usepackage{caption} \captionsetup{font={footnotesize},width=6in} \renewcommand{\dblfloatpagefraction}{.9} \makeatletter \renewenvironment{figure} {\def\@captype{figure}} \makeatother \@ifundefined{Shaded}{\newenvironment{Shaded}} \@ifundefined{snugshade}{\newenvironment{snugshade}} \renewenvironment{Shaded}{\begin{snugshade}}{\end{snugshade}} \definecolor{shadecolor}{RGB}{230,230,230} \usepackage{xeCJK} \usepackage{setspace} \setstretch{1.3} \usepackage{tcolorbox} \setcounter{secnumdepth}{4} \setcounter{tocdepth}{4} \usepackage{wallpaper} \usepackage[absolute]{textpos} \tcbuselibrary{breakable} \renewenvironment{Shaded} {\begin{tcolorbox}[colback = gray!10, colframe = gray!40, width = 16cm, arc = 1mm, auto outer arc, title = {R input}]} {\end{tcolorbox}} \usepackage{titlesec} \titleformat{\paragraph} {\fontsize{10pt}{0pt}\bfseries} {\arabic{section}.\arabic{subsection}.\arabic{subsubsection}.\arabic{paragraph}} {1em} {} []

\author{}
\date{\vspace{-2.5em}}

\begin{document}

\begin{titlepage} \newgeometry{top=7.5cm}
\ThisCenterWallPaper{1.12}{~/outline/lixiao//cover_page.pdf}
\begin{center} \textbf{\Huge } \vspace{4em}
\begin{textblock}{10}(3,5.9) \huge
\textbf{\textcolor{white}{2024-06-24}}
\end{textblock} \begin{textblock}{10}(3,7.3)
\Large \textcolor{black}{LiChuang Huang}
\end{textblock} \begin{textblock}{10}(3,11.3)
\Large \textcolor{black}{@立效研究院}
\end{textblock} \end{center} \end{titlepage}
\restoregeometry

\pagenumbering{roman}

\begin{center}\vspace{1.5cm}\pgfornament[anchor=center,ydelta=0pt,width=8cm]{84}\end{center}\tableofcontents

\newpage

\pagenumbering{arabic}

\hypertarget{ux6458ux8981}{%
\section{摘要}\label{ux6458ux8981}}

使用 Rmarkdown + Latex + 自定义的 R 程序,在分析的同时,生成美观、规范的报告文档。

\hypertarget{ux5177ux4f53ux4f18ux5316}{%
\section{具体优化}\label{ux5177ux4f53ux4f18ux5316}}

\hypertarget{ux76eeux5f55}{%
\subsection{目录}\label{ux76eeux5f55}}

\begin{itemize}
\tightlist
\item
  除了内容目录,还提供图片索引、表格索引。
\end{itemize}

\def\@captype{figure}
\begin{center}
\includegraphics[width = 0.9\linewidth]{~/Pictures/Screenshots/Screenshot from 2024-06-24 16-00-24.png}
\caption{Unnamed chunk 5}\label{fig:unnamed-chunk-5}
\end{center}

\def\@captype{figure}
\begin{center}
\includegraphics[width = 0.9\linewidth]{~/Pictures/Screenshots/Screenshot from 2024-06-24 16-01-26.png}
\caption{Unnamed chunk 6}\label{fig:unnamed-chunk-6}
\end{center}

\def\@captype{figure}
\begin{center}
\includegraphics[width = 0.9\linewidth]{~/Pictures/Screenshots/Screenshot from 2024-06-24 16-01-32.png}
\caption{Unnamed chunk 7}\label{fig:unnamed-chunk-7}
\end{center}

\hypertarget{figure}{%
\subsection{Figure}\label{figure}}

\hypertarget{ux7ec4ux56fe}{%
\subsubsection{组图}\label{ux7ec4ux56fe}}

\begin{itemize}
\tightlist
\item
  图片所在定位。
\item
  组图所用材料定位。
\item
  自动化 figure 标记。
\item
  注释分界线 (避免内容混淆)
\end{itemize}

\def\@captype{figure}
\begin{center}
\includegraphics[width = 0.9\linewidth]{~/Pictures/Screenshots/Screenshot from 2024-06-24 16-03-13.png}
\caption{Unnamed chunk 8}\label{fig:unnamed-chunk-8}
\end{center}

\hypertarget{ux5206ux56fe}{%
\subsubsection{分图}\label{ux5206ux56fe}}

在组图的特点的基础上:

\begin{itemize}
\tightlist
\item
  特定 Figure,将自动触发类似如下参数注释
\end{itemize}

\def\@captype{figure}
\begin{center}
\includegraphics[width = 0.9\linewidth]{~/Pictures/Screenshots/Screenshot from 2024-06-24 16-06-22.png}
\caption{Unnamed chunk 9}\label{fig:unnamed-chunk-9}
\end{center}

\def\@captype{figure}
\begin{center}
\includegraphics[width = 0.9\linewidth]{~/Pictures/Screenshots/Screenshot from 2024-06-24 16-08-49.png}
\caption{Unnamed chunk 10}\label{fig:unnamed-chunk-10}
\end{center}

\hypertarget{table}{%
\subsection{Table}\label{table}}

\begin{itemize}
\tightlist
\item
  类似 Figure 的文件定位。
\item
  所有提供的表格,都将在报告中提供概览。
\item
  必要内容,可触发自动列名称注释。
\end{itemize}

\def\@captype{figure}
\begin{center}
\includegraphics[width = 0.9\linewidth]{~/Pictures/Screenshots/Screenshot from 2024-06-24 16-09-51.png}
\caption{Unnamed chunk 11}\label{fig:unnamed-chunk-11}
\end{center}

\hypertarget{ux9644ux52a0ux6587ux4ef6}{%
\subsection{附加文件}\label{ux9644ux52a0ux6587ux4ef6}}

包含附加文件时,触发对文件注释说明。

\def\@captype{figure}
\begin{center}
\includegraphics[width = 0.9\linewidth]{~/Pictures/Screenshots/Screenshot from 2024-06-24 16-12-39.png}
\caption{Unnamed chunk 12}\label{fig:unnamed-chunk-12}
\end{center}

\hypertarget{ux5206ux6790ux6d41ux7a0b}{%
\subsection{分析流程}\label{ux5206ux6790ux6d41ux7a0b}}

\hypertarget{ux81eaux52a8ux5316ux751fux6210ux6807ux9898}{%
\subsubsection{自动化生成标题}\label{ux81eaux52a8ux5316ux751fux6210ux6807ux9898}}

\begin{itemize}
\tightlist
\item
  标题格式为:分级+方法+分析内容+标记
\item
  分析与标题生成同步,分析结束后,可借程序生成如下内容,发送 AI 获取注释 (见 \ref{ai})
\end{itemize}

\def\@captype{figure}
\begin{center}
\includegraphics[width = 0.9\linewidth]{~/Pictures/Screenshots/Screenshot from 2024-06-24 16-22-04.png}
\caption{Unnamed chunk 13}\label{fig:unnamed-chunk-13}
\end{center}

\hypertarget{ai}{%
\subsubsection{AI 注释 (ChatGPT 4 分析流程说明)}\label{ai}}

结合自动标题,可对分析流程进行 AI 注释,嵌入文档。

\def\@captype{figure}
\begin{center}
\includegraphics[width = 0.9\linewidth]{~/Pictures/Screenshots/Screenshot from 2024-06-24 16-18-25.png}
\caption{Unnamed chunk 14}\label{fig:unnamed-chunk-14}
\end{center}

\hypertarget{ux539fux4ee3ux7801-ux5982ux5fc5ux8981}{%
\subsection{原代码 (如必要)}\label{ux539fux4ee3ux7801-ux5982ux5fc5ux8981}}

\begin{itemize}
\tightlist
\item
  如必要,可生成代码块,与分析内容一一对应。
\end{itemize}

\def\@captype{figure}
\begin{center}
\includegraphics[width = 0.9\linewidth]{~/Pictures/Screenshots/Screenshot from 2024-06-24 16-26-21.png}
\caption{Unnamed chunk 15}\label{fig:unnamed-chunk-15}
\end{center}

\end{document}
