% Options for packages loaded elsewhere
\PassOptionsToPackage{unicode}{hyperref}
\PassOptionsToPackage{hyphens}{url}
%
\documentclass[
]{article}
\usepackage{lmodern}
\usepackage{amssymb,amsmath}
\usepackage{ifxetex,ifluatex}
\ifnum 0\ifxetex 1\fi\ifluatex 1\fi=0 % if pdftex
  \usepackage[T1]{fontenc}
  \usepackage[utf8]{inputenc}
  \usepackage{textcomp} % provide euro and other symbols
\else % if luatex or xetex
  \usepackage{unicode-math}
  \defaultfontfeatures{Scale=MatchLowercase}
  \defaultfontfeatures[\rmfamily]{Ligatures=TeX,Scale=1}
\fi
% Use upquote if available, for straight quotes in verbatim environments
\IfFileExists{upquote.sty}{\usepackage{upquote}}{}
\IfFileExists{microtype.sty}{% use microtype if available
  \usepackage[]{microtype}
  \UseMicrotypeSet[protrusion]{basicmath} % disable protrusion for tt fonts
}{}
\makeatletter
\@ifundefined{KOMAClassName}{% if non-KOMA class
  \IfFileExists{parskip.sty}{%
    \usepackage{parskip}
  }{% else
    \setlength{\parindent}{0pt}
    \setlength{\parskip}{6pt plus 2pt minus 1pt}}
}{% if KOMA class
  \KOMAoptions{parskip=half}}
\makeatother
\usepackage{xcolor}
\IfFileExists{xurl.sty}{\usepackage{xurl}}{} % add URL line breaks if available
\IfFileExists{bookmark.sty}{\usepackage{bookmark}}{\usepackage{hyperref}}
\hypersetup{
  hidelinks,
  pdfcreator={LaTeX via pandoc}}
\urlstyle{same} % disable monospaced font for URLs
\usepackage[margin=1in]{geometry}
\usepackage{longtable,booktabs}
% Correct order of tables after \paragraph or \subparagraph
\usepackage{etoolbox}
\makeatletter
\patchcmd\longtable{\par}{\if@noskipsec\mbox{}\fi\par}{}{}
\makeatother
% Allow footnotes in longtable head/foot
\IfFileExists{footnotehyper.sty}{\usepackage{footnotehyper}}{\usepackage{footnote}}
\makesavenoteenv{longtable}
\usepackage{graphicx}
\makeatletter
\def\maxwidth{\ifdim\Gin@nat@width>\linewidth\linewidth\else\Gin@nat@width\fi}
\def\maxheight{\ifdim\Gin@nat@height>\textheight\textheight\else\Gin@nat@height\fi}
\makeatother
% Scale images if necessary, so that they will not overflow the page
% margins by default, and it is still possible to overwrite the defaults
% using explicit options in \includegraphics[width, height, ...]{}
\setkeys{Gin}{width=\maxwidth,height=\maxheight,keepaspectratio}
% Set default figure placement to htbp
\makeatletter
\def\fps@figure{htbp}
\makeatother
\setlength{\emergencystretch}{3em} % prevent overfull lines
\providecommand{\tightlist}{%
  \setlength{\itemsep}{0pt}\setlength{\parskip}{0pt}}
\setcounter{secnumdepth}{5}
\usepackage{tikz} \usepackage{auto-pst-pdf} \usepackage{pgfornament} \usepackage{pstricks-add} \usepackage{caption} \captionsetup{font={footnotesize},width=6in} \renewcommand{\dblfloatpagefraction}{.9} \makeatletter \renewenvironment{figure} {\def\@captype{figure}} \makeatother \@ifundefined{Shaded}{\newenvironment{Shaded}} \@ifundefined{snugshade}{\newenvironment{snugshade}} \renewenvironment{Shaded}{\begin{snugshade}}{\end{snugshade}} \definecolor{shadecolor}{RGB}{230,230,230} \usepackage{xeCJK} \usepackage{setspace} \setstretch{1.3} \usepackage{tcolorbox} \setcounter{secnumdepth}{4} \setcounter{tocdepth}{4} \usepackage{wallpaper} \usepackage[absolute]{textpos} \tcbuselibrary{breakable} \renewenvironment{Shaded} {\begin{tcolorbox}[colback = gray!10, colframe = gray!40, width = 16cm, arc = 1mm, auto outer arc, title = {R input}]} {\end{tcolorbox}} \usepackage{titlesec} \titleformat{\paragraph} {\fontsize{10pt}{0pt}\bfseries} {\arabic{section}.\arabic{subsection}.\arabic{subsubsection}.\arabic{paragraph}} {1em} {} []
\newlength{\cslhangindent}
\setlength{\cslhangindent}{1.5em}
\newenvironment{cslreferences}%
  {}%
  {\par}

\author{}
\date{\vspace{-2.5em}}

\begin{document}

\begin{titlepage} \newgeometry{top=6.5cm}
\ThisCenterWallPaper{1.12}{~/outline/bosai//cover_page_analysis.pdf}
\begin{center} \textbf{\huge 骨肉瘤}
\vspace{4em} \begin{textblock}{10}(3,4.85) \Large
\textbf{\textcolor{black}{BSZD231122}}
\end{textblock} \begin{textblock}{10}(3,5.8)
\Large \textbf{\textcolor{black}{黄礼闯}}
\end{textblock} \begin{textblock}{10}(3,6.75)
\Large
\textbf{\textcolor{black}{分析优化}}
\end{textblock} \begin{textblock}{10}(3,7.7)
\Large
\textbf{\textcolor{black}{杨立宇}}
\end{textblock} \end{center} \end{titlepage}
\restoregeometry

\pagenumbering{roman}

\begin{center}\vspace{1.5cm}\pgfornament[anchor=center,ydelta=0pt,width=8cm]{84}\end{center}\tableofcontents

\begin{center}\vspace{1.5cm}\pgfornament[anchor=center,ydelta=0pt,width=8cm]{88}\end{center}\listoffigures

\begin{center}\vspace{1.5cm}\pgfornament[anchor=center,ydelta=0pt,width=8cm]{89}\end{center}\listoftables

\newpage

\pagenumbering{arabic}

\hypertarget{abstract}{%
\section{分析流程}\label{abstract}}

\begin{center}\vspace{1.5cm}\pgfornament[anchor=center,ydelta=0pt,width=9cm]{88}\end{center}
\def\@captype{figure}
\begin{center}
\includegraphics[width = 0.9\linewidth]{Figure+Table/Route.pdf}
\caption{Route}\label{fig:Route}
\end{center}

\begin{center}\pgfornament[anchor=center,ydelta=0pt,width=9cm]{88}\vspace{1.5cm}\end{center}

\hypertarget{introduction}{%
\section{材料和方法}\label{introduction}}

\hypertarget{ux6570ux636eux5206ux6790ux5e73ux53f0}{%
\subsection{数据分析平台}\label{ux6570ux636eux5206ux6790ux5e73ux53f0}}

在 Linux pop-os x86\_64 (6.9.3-76060903-generic) 上,使用 R version 4.4.2 (2024-10-31) (\url{https://www.r-project.org/}) 对数据统计分析与整合分析。

\hypertarget{seurat-ux96c6ux6210ux5355ux7ec6ux80deux6570ux636eux5206ux6790-dataset-os}{%
\subsection{Seurat 集成单细胞数据分析 (Dataset: OS)}\label{seurat-ux96c6ux6210ux5355ux7ec6ux80deux6570ux636eux5206ux6790-dataset-os}}

使用 Seurat R 包 (5.1.0) 进行单细胞数据质量控制 (QC) 和下游分析。依据 \url{https://satijalab.org/seurat/articles/integration_introduction} 为指导对单细胞数据预处理。
一个细胞至少应有 1000 个基因,并且基因数量小于 7000。线粒体基因的比例小于 10\%。根据上述条件,获得用于下游分析的高质量细胞。
执行标准 Seurat 分析工作流 (\texttt{NormalizeData}, \texttt{FindVariableFeatures}, \texttt{ScaleData}, \texttt{RunPCA})。以 \texttt{ElbowPlot} 判断后续分析的 PC 维度。
以 \texttt{Seurat::IntegrateLayers} 集成数据,去除批次效应 (使用 HarmonyIntegration 方法)。在 1-10 PC 维度下,以 \texttt{Seurat::FindNeighbors} 构建 Nearest-neighbor Graph。随后在 1.2 分辨率下,以 \texttt{Seurat::FindClusters} 函数识别细胞群并以 \texttt{Seurat::RunUMAP} 进行 UMAP 聚类。
以 \texttt{Seurat::FindAllMarkers} (LogFC 阈值 0.25; 最小检出率 0.01) 为所有细胞群寻找 Markers。
以 Python 工具 \texttt{SCSA} ((2020, \textbf{IF:2.8}, Q2, Frontiers in genetics)\textsuperscript{\protect\hyperlink{ref-ScsaACellTyCaoY2020}{1}}) (\url{https://github.com/bioinfo-ibms-pumc/SCSA}) 对细胞群注释。

\hypertarget{copykat-ux764cux7ec6ux80deux9274ux5b9a-dataset-os}{%
\subsection{CopyKAT 癌细胞鉴定 (Dataset: OS)}\label{copykat-ux764cux7ec6ux80deux9274ux5b9a-dataset-os}}

R 包 \texttt{CopyKAT} 用于鉴定恶性细胞 (2021, \textbf{IF:33.1}, Q1, Nature Biotechnology)\textsuperscript{\protect\hyperlink{ref-DelineatingCopGaoR2021}{2}}。\texttt{CopyKAT} 可以区分整倍体与非整倍体,其中非整倍体被认为是肿瘤细胞,而整倍体是正常细胞 (2012, \textbf{IF:39.1}, Q1, Nature Reviews Genetics)\textsuperscript{\protect\hyperlink{ref-CausesAndConsGordon2012}{3}}。由于 \texttt{CopyKAT} 不适用于多样本数据 (批次效应的存在) ,因此,对各个样本独立鉴定。

\hypertarget{scfea-ux5355ux7ec6ux80deux6570ux636eux7684ux4ee3ux8c22ux901aux91cfux9884ux6d4b-dataset-os_sample}{%
\subsection{scFEA 单细胞数据的代谢通量预测 (Dataset: OS\_SAMPLE)}\label{scfea-ux5355ux7ec6ux80deux6570ux636eux7684ux4ee3ux8c22ux901aux91cfux9884ux6d4b-dataset-os_sample}}

将 Seurat 的 \texttt{RNA} Assay (`counts') 作为输入数据,以 \texttt{scFEA} 预测细胞的代谢通量 (2021, \textbf{IF:6.2}, Q1, Genome research)\textsuperscript{\protect\hyperlink{ref-AGraphNeuralAlgham2021}{4}}。参考 \url{https://github.com/changwn/scFEA/blob/master/scFEA_tutorial1.ipynb} 和 \url{https://github.com/changwn/scFEA/blob/master/scFEA_tutorial2.ipynb}。

\hypertarget{seurat-ux7ec6ux80deux4e9aux7fa4ux5206ux6790-dataset-os_cancer}{%
\subsection{Seurat 细胞亚群分析 (Dataset: OS\_CANCER)}\label{seurat-ux7ec6ux80deux4e9aux7fa4ux5206ux6790-dataset-os_cancer}}

执行标准 Seurat 分析工作流 (\texttt{NormalizeData}, \texttt{FindVariableFeatures}, \texttt{ScaleData}, \texttt{RunPCA})。以 \texttt{ElbowPlot} 判断后续分析的 PC 维度。
在 1-15 PC 维度下,以 Seurat::FindNeighbors 构建 Nearest-neighbor Graph。随后在 1.2 分辨率下,以 Seurat::FindClusters 函数识别细胞群并以 Seurat::RunUMAP 进行 UMAP 聚类。

\hypertarget{limma-ux4ee3ux8c22ux901aux91cfux5deeux5f02ux5206ux6790-dataset-os_cancer_flux}{%
\subsection{Limma 代谢通量差异分析 (Dataset: OS\_CANCER\_FLUX)}\label{limma-ux4ee3ux8c22ux901aux91cfux5deeux5f02ux5206ux6790-dataset-os_cancer_flux}}

以 \texttt{limma} (3.62.2) (2005)\textsuperscript{\protect\hyperlink{ref-LimmaLinearMSmyth2005}{5}} 差异分析。分析方法参考 \url{https://bioconductor.org/packages/release/workflows/vignettes/RNAseq123/inst/doc/limmaWorkflow.html}。创建设计矩阵,对比矩阵,差异分析:Malignant\_cell\_BC vs Benign\_cell\_BC。使用 \texttt{limma::lmFit}, \texttt{limma::contrasts.fit}, \texttt{limma::eBayes} 拟合线形模型。以 \texttt{limma::topTable} 提取所有结果,并过滤得到 adj.P.Val 小于 0.05,\textbar Log2(FC)\textbar{} 大于 0.5 的统计结果。

\hypertarget{tcga-ux6570ux636eux83b7ux53d6-dataset-os}{%
\subsection{TCGA 数据获取 (Dataset: OS)}\label{tcga-ux6570ux636eux83b7ux53d6-dataset-os}}

以 R 包 \texttt{TCGAbiolinks} (2.35.1) (2015, \textbf{IF:16.6}, Q1, Nucleic Acids Research)\textsuperscript{\protect\hyperlink{ref-TcgabiolinksAColapr2015}{6}} 获取 TARGET-OS 数据集。

\hypertarget{cox-ux56deux5f52-dataset-tcga_os}{%
\subsection{COX 回归 (Dataset: TCGA\_OS)}\label{cox-ux56deux5f52-dataset-tcga_os}}

以 R 包 \texttt{survival} (3.8.3) 进行单因素 COX 回归 (\texttt{survival::coxph})。筛选 \texttt{Pr(\textgreater{}\textbar{}z\textbar{})} \textless{} .05\texttt{的基因。\ 以\ R\ 包}glmnet\texttt{(4.1.8)\ 作\ lasso\ 处罚的\ cox\ 回归,以}cv.glmnet` 函数作 5 交叉验证获得模型。

\hypertarget{survival-ux751fux5b58ux5206ux6790-dataset-tcga_os}{%
\subsection{Survival 生存分析 (Dataset: TCGA\_OS)}\label{survival-ux751fux5b58ux5206ux6790-dataset-tcga_os}}

以 R 包 \texttt{survival} (3.8.3) 生存分析,以 R 包 \texttt{survminer} (0.5.0) 绘制生存曲线。

\hypertarget{gse-ux6570ux636eux641cux7d22-dataset-os}{%
\subsection{GSE 数据搜索 (Dataset: OS)}\label{gse-ux6570ux636eux641cux7d22-dataset-os}}

使用 Entrez Direct (EDirect) \url{https://www.ncbi.nlm.nih.gov/books/NBK3837/} 搜索 GEO 数据库 (\texttt{esearch\ -db\ gds}),查询信息为: ((Osteosarcoma{[}Description{]}) AND ((6:300{[}Number of Samples{]}) AND (GSE{[}Entry Type{]}) AND (Homo sapiens{[}Organism{]})),转化为数据表格。
以正则匹配,滤除 `summary' 或 `title' 中包含 `single cell' 或 `scRNA' 的数据例。仅查询临床数据,因此滤除匹配到关键词 in vitro, cell line, CD{[}0-9{]}+, vehicle, vector, DMSO, /ml, nm 的数据例。
(注:以上仅为查找合适的 GEO 数据所做的数据筛选,与实际分析无关) 。仅获取类型包含 `Expression profiling by high throughput sequencing' 或 `Expression profiling by array' 的数据例。此外,排除 summary 或 title 中匹配到字符集 EX (KO, WT, \emph{WT}, \emph{KO}, wildtype, mutant, knock, deficien, absen, SuperSeries, transgenic, CD{[}0-9{]}+) 的数据例。上述得到 147 个 GSE 数据集。以 R 抓取网页 (例如,\texttt{RCurl::getURL} 抓取 ),解析 `Overall design' 和 `Samples' 模块,匹配字符集 EX,排除匹配到的数据例。排除 `Overall design' 中包含 in vitro, cell line, CD{[}0-9{]}+, vehicle, vector, DMSO, /ml, nm 的数据例。仅获取包含 `protein coding' 测序的数据集,排除 `Samples' 和 `Overall design' 中包含 siRNA, miRNA, miR, lncRNA 字符的数据例。余下共 73 个。
以 \texttt{GEOquery} 获取 GSE 数据集 (n=73)。
从元数据中匹配包含关键词的数据:`Survival\textbar Event\textbar Dead\textbar Alive\textbar Status\textbar Day\textbar Time',共得到 17 个数据集。

\hypertarget{geo-ux6570ux636eux83b7ux53d6-dataset-os_gse39057}{%
\subsection{GEO 数据获取 (Dataset: OS\_GSE39057)}\label{geo-ux6570ux636eux83b7ux53d6-dataset-os_gse39057}}

以 R 包 \texttt{GEOquery} (2.74.0) 获取 GSE39057 数据集。

\hypertarget{geo-ux6570ux636eux83b7ux53d6-dataset-os_gse39055}{%
\subsection{GEO 数据获取 (Dataset: OS\_GSE39055)}\label{geo-ux6570ux636eux83b7ux53d6-dataset-os_gse39055}}

以 R 包 \texttt{GEOquery} (2.74.0) 获取 GSE39055 数据集。

\hypertarget{geo-ux6570ux636eux83b7ux53d6-dataset-os_gse16091}{%
\subsection{GEO 数据获取 (Dataset: OS\_GSE16091)}\label{geo-ux6570ux636eux83b7ux53d6-dataset-os_gse16091}}

以 R 包 \texttt{GEOquery} (2.74.0) 获取 GSE16091 数据集。

\hypertarget{geo-ux6570ux636eux83b7ux53d6-dataset-os_gse21257}{%
\subsection{GEO 数据获取 (Dataset: OS\_GSE21257)}\label{geo-ux6570ux636eux83b7ux53d6-dataset-os_gse21257}}

以 R 包 \texttt{GEOquery} (2.74.0) 获取 GSE21257 数据集。

\hypertarget{survival-ux751fux5b58ux5206ux6790-dataset-os_outer}{%
\subsection{Survival 生存分析 (Dataset: OS\_OUTER)}\label{survival-ux751fux5b58ux5206ux6790-dataset-os_outer}}

以 R 包 \texttt{survival} (3.8.3) 生存分析,以 R 包 \texttt{survminer} (0.5.0) 绘制生存曲线。

\hypertarget{clusterprofiler-ux5bccux96c6ux5206ux6790-dataset-prog}{%
\subsection{ClusterProfiler 富集分析 (Dataset: PROG)}\label{clusterprofiler-ux5bccux96c6ux5206ux6790-dataset-prog}}

以 ClusterProfiler R 包 (4.15.0.2) (2021, \textbf{IF:33.2}, Q1, The Innovation)\textsuperscript{\protect\hyperlink{ref-ClusterprofilerWuTi2021}{7}}进行 KEGG 和 GO 富集分析。以 p.adjust 表示显著水平。

\hypertarget{workflow}{%
\section{分析结果}\label{workflow}}

\hypertarget{seurat-ux96c6ux6210ux5355ux7ec6ux80deux6570ux636eux5206ux6790-os}{%
\subsection{Seurat 集成单细胞数据分析 (OS)}\label{seurat-ux96c6ux6210ux5355ux7ec6ux80deux6570ux636eux5206ux6790-os}}

读取 BC10, BC11, BC16, BC17, BC2, BC20, BC21, BC22, BC3, BC5, BC6 样本的数据集。前期质量控制,一个细胞至少应有 1000 个基因,并且基因数量小于 7000。线粒体基因的比例小于 10\%。数据归一化,PCA 聚类 (Seurat 标准工作流,见方法章节) 后,绘制 PC standard deviations 图。去除批次效应后 (详见方法章节) ,在 1-10 PC 维度,1.2 分辨率下,对细胞群 UMAP 聚类。计算所有细胞群的 Marker。使用特异性 Marker,以 SCSA 对细胞群注释。

(鉴定所用 Markers 来源于原作文献 \url{PMID:33303760} (2020, \textbf{IF:14.7}, Q1, Nature communications)\textsuperscript{\protect\hyperlink{ref-Single_cell_RNA_Zhou_2020}{8}})

\begin{center}\vspace{1.5cm}\pgfornament[anchor=center,ydelta=0pt,width=9cm]{88}\end{center}
\def\@captype{figure}
\begin{center}
\includegraphics[width = 0.9\linewidth]{Figure+Table/Pre-Quality-control.pdf}
\caption{Pre Quality control}\label{fig:Pre-Quality-control}
\end{center}

\begin{center}\pgfornament[anchor=center,ydelta=0pt,width=9cm]{88}\vspace{1.5cm}\end{center}

Fig. \ref{fig:Pre-Quality-control}

\begin{center}\vspace{1.5cm}\pgfornament[anchor=center,ydelta=0pt,width=9cm]{88}\end{center}
\def\@captype{figure}
\begin{center}
\includegraphics[width = 0.9\linewidth]{Figure+Table/OS-After-Quality-control.pdf}
\caption{OS After Quality control}\label{fig:OS-After-Quality-control}
\end{center}

\begin{center}\pgfornament[anchor=center,ydelta=0pt,width=9cm]{88}\vspace{1.5cm}\end{center}

Fig. \ref{fig:OS-After-Quality-control}

\begin{center}\vspace{1.5cm}\pgfornament[anchor=center,ydelta=0pt,width=9cm]{88}\end{center}
\def\@captype{figure}
\begin{center}
\includegraphics[width = 0.9\linewidth]{Figure+Table/OS-Standard-deviations-of-PCs.pdf}
\caption{OS Standard deviations of PCs}\label{fig:OS-Standard-deviations-of-PCs}
\end{center}

\begin{center}\pgfornament[anchor=center,ydelta=0pt,width=9cm]{88}\vspace{1.5cm}\end{center}

Fig. \ref{fig:OS-Standard-deviations-of-PCs}

\begin{center}\vspace{1.5cm}\pgfornament[anchor=center,ydelta=0pt,width=9cm]{88}\end{center}
\def\@captype{figure}
\begin{center}
\includegraphics[width = 0.9\linewidth]{Figure+Table/OS-UMAP-Unintegrated.pdf}
\caption{OS UMAP Unintegrated}\label{fig:OS-UMAP-Unintegrated}
\end{center}

\begin{center}\pgfornament[anchor=center,ydelta=0pt,width=9cm]{88}\vspace{1.5cm}\end{center}

Fig. \ref{fig:OS-UMAP-Unintegrated}

\begin{center}\vspace{1.5cm}\pgfornament[anchor=center,ydelta=0pt,width=9cm]{88}\end{center}
\def\@captype{figure}
\begin{center}
\includegraphics[width = 0.9\linewidth]{Figure+Table/OS-UMAP-Integrated.pdf}
\caption{OS UMAP Integrated}\label{fig:OS-UMAP-Integrated}
\end{center}

\begin{center}\pgfornament[anchor=center,ydelta=0pt,width=9cm]{88}\vspace{1.5cm}\end{center}

Fig. \ref{fig:OS-UMAP-Integrated}

\begin{center}\vspace{1.5cm}\pgfornament[anchor=center,ydelta=0pt,width=9cm]{89}\end{center}

\begin{longtable}[]{@{}lllll@{}}
\caption{\label{tab:OS-significant-markers-of-cell-clusters}OS significant markers of cell clusters}\tabularnewline
\toprule
rownames & p\_val & avg\_log2FC & pct.1 & pct.2\tabularnewline
\midrule
\endfirsthead
\toprule
rownames & p\_val & avg\_log2FC & pct.1 & pct.2\tabularnewline
\midrule
\endhead
ALPL & 0 & 1.494 & 0.915 & 0.412\tabularnewline
PANX3 & 0 & 2.181 & 0.645 & 0.156\tabularnewline
IFITM5 & 0 & 2.458 & 0.802 & 0.323\tabularnewline
LY6K & 0 & 1.191 & 0.77 & 0.314\tabularnewline
RHBDL2 & 0 & 1.733 & 0.644 & 0.21\tabularnewline
\ldots{} & \ldots{} & \ldots{} & \ldots{} & \ldots{}\tabularnewline
\bottomrule
\end{longtable}

\begin{center}\pgfornament[anchor=center,ydelta=0pt,width=9cm]{89}\vspace{1.5cm}\end{center}

Tab. \ref{tab:OS-significant-markers-of-cell-clusters}

\begin{center}\vspace{1.5cm}\pgfornament[anchor=center,ydelta=0pt,width=9cm]{88}\end{center}
\def\@captype{figure}
\begin{center}
\includegraphics[width = 0.9\linewidth]{Figure+Table/OS-Marker-Validation.pdf}
\caption{OS Marker Validation}\label{fig:OS-Marker-Validation}
\end{center}

\begin{center}\pgfornament[anchor=center,ydelta=0pt,width=9cm]{88}\vspace{1.5cm}\end{center}

Fig. \ref{fig:OS-Marker-Validation}

\begin{center}\vspace{1.5cm}\pgfornament[anchor=center,ydelta=0pt,width=9cm]{88}\end{center}
\def\@captype{figure}
\begin{center}
\includegraphics[width = 0.9\linewidth]{Figure+Table/OS-SCSA-Cell-type-annotation.pdf}
\caption{OS SCSA Cell type annotation}\label{fig:OS-SCSA-Cell-type-annotation}
\end{center}

\begin{center}\pgfornament[anchor=center,ydelta=0pt,width=9cm]{88}\vspace{1.5cm}\end{center}

Fig. \ref{fig:OS-SCSA-Cell-type-annotation}

\begin{center}\vspace{1.5cm}\pgfornament[anchor=center,ydelta=0pt,width=9cm]{88}\end{center}
\def\@captype{figure}
\begin{center}
\includegraphics[width = 0.9\linewidth]{Figure+Table/OS-SCSA-Cell-Proportions-in-each-sample.pdf}
\caption{OS SCSA Cell Proportions in each sample}\label{fig:OS-SCSA-Cell-Proportions-in-each-sample}
\end{center}

\begin{center}\pgfornament[anchor=center,ydelta=0pt,width=9cm]{88}\vspace{1.5cm}\end{center}

Fig. \ref{fig:OS-SCSA-Cell-Proportions-in-each-sample}

\hypertarget{copykat-ux764cux7ec6ux80deux9274ux5b9a-os}{%
\subsection{CopyKAT 癌细胞鉴定 (OS)}\label{copykat-ux764cux7ec6ux80deux9274ux5b9a-os}}

以 \texttt{CopyKAT} 鉴定恶质细胞。

\begin{center}\vspace{1.5cm}\pgfornament[anchor=center,ydelta=0pt,width=9cm]{88}\end{center}
\def\@captype{figure}
\begin{center}
\includegraphics[width = 0.9\linewidth]{Figure+Table/OS-proportions-of-aneuploid-and-diploid.pdf}
\caption{OS proportions of aneuploid and diploid}\label{fig:OS-proportions-of-aneuploid-and-diploid}
\end{center}

\begin{center}\pgfornament[anchor=center,ydelta=0pt,width=9cm]{88}\vspace{1.5cm}\end{center}

Fig. \ref{fig:OS-proportions-of-aneuploid-and-diploid}

\begin{center}\vspace{1.5cm}\pgfornament[anchor=center,ydelta=0pt,width=9cm]{89}\end{center}

\begin{longtable}[]{@{}llll@{}}
\caption{\label{tab:BC10-copyKAT-prediction-data}BC10 copyKAT prediction data}\tabularnewline
\toprule
orig.ident & cell.names & copykat.pred & copykat\_cell\tabularnewline
\midrule
\endfirsthead
\toprule
orig.ident & cell.names & copykat.pred & copykat\_cell\tabularnewline
\midrule
\endhead
BC10 & AAACCTGAGACTCGGA-1\_1 & aneuploid & Cancer cell\tabularnewline
BC10 & AAACCTGAGGAACTGC-1\_1 & diploid & Normal cell\tabularnewline
BC10 & AAACCTGAGGATGGAA-1\_1 & diploid & Normal cell\tabularnewline
BC10 & AAACCTGAGGTGCTTT-1\_1 & diploid & Normal cell\tabularnewline
BC10 & AAACCTGAGTAGCGGT-1\_1 & diploid & Normal cell\tabularnewline
\ldots{} & \ldots{} & \ldots{} & \ldots{}\tabularnewline
\bottomrule
\end{longtable}

\begin{center}\pgfornament[anchor=center,ydelta=0pt,width=9cm]{89}\vspace{1.5cm}\end{center}

Tab. \ref{tab:BC10-copyKAT-prediction-data}

\begin{center}\pgfornament[anchor=center,ydelta=0pt,width=9cm]{85}\vspace{1.5cm}\end{center}

\begin{center}\pgfornament[anchor=center,ydelta=0pt,width=9cm]{85}\vspace{1.5cm}\end{center}

\hypertarget{scfea-ux5355ux7ec6ux80deux6570ux636eux7684ux4ee3ux8c22ux901aux91cfux9884ux6d4b-os_sample}{%
\subsection{scFEA 单细胞数据的代谢通量预测 (OS\_SAMPLE)}\label{scfea-ux5355ux7ec6ux80deux6570ux636eux7684ux4ee3ux8c22ux901aux91cfux9884ux6d4b-os_sample}}

根据样本和细胞类型分组,将细胞随机抽样 (各组比例为:0.5) (细胞数量较多,通过随机抽样的方式减少计算负担) (随机种子:987456)。
将 \texttt{Seurat} (所有细胞) 以 \texttt{scFEA} 预测代谢通量。

\begin{center}\vspace{1.5cm}\pgfornament[anchor=center,ydelta=0pt,width=9cm]{88}\end{center}
\def\@captype{figure}
\begin{center}
\includegraphics[width = 0.9\linewidth]{scfea_OS_SAMPLE/loss_20250408-165632.png}
\caption{OS SAMPLE Convergency of the loss terms during training}\label{fig:OS-SAMPLE-Convergency-of-the-loss-terms-during-training}
\end{center}

\begin{center}\pgfornament[anchor=center,ydelta=0pt,width=9cm]{88}\vspace{1.5cm}\end{center}

Fig. \ref{fig:OS-SAMPLE-Convergency-of-the-loss-terms-during-training}

\begin{center}\vspace{1.5cm}\pgfornament[anchor=center,ydelta=0pt,width=9cm]{89}\end{center}

\begin{longtable}[]{@{}lllll@{}}
\caption{\label{tab:OS-SAMPLE-annotation-of-metabolic-flux}OS SAMPLE annotation of metabolic flux}\tabularnewline
\toprule
V1 & Module\_id & Compound\_IN\_name & Compound\_IN\_ID & Compound\_OUT\_name\tabularnewline
\midrule
\endfirsthead
\toprule
V1 & Module\_id & Compound\_IN\_name & Compound\_IN\_ID & Compound\_OUT\_name\tabularnewline
\midrule
\endhead
M\_1 & 1 & Glucose & C00267 & G6P\tabularnewline
M\_2 & 2 & G6P & C00668 & G3P\tabularnewline
M\_3 & 3 & G3P & C00118 & 3PD\tabularnewline
M\_4 & 4 & 3PD & C00197 & Pyruvate\tabularnewline
M\_5 & 5 & Pyruvate & C00022 & Acetyl-Coa\tabularnewline
\ldots{} & \ldots{} & \ldots{} & \ldots{} & \ldots{}\tabularnewline
\bottomrule
\end{longtable}

\begin{center}\pgfornament[anchor=center,ydelta=0pt,width=9cm]{89}\vspace{1.5cm}\end{center}

Tab. \ref{tab:OS-SAMPLE-annotation-of-metabolic-flux}

\begin{center}\vspace{1.5cm}\pgfornament[anchor=center,ydelta=0pt,width=9cm]{89}\end{center}

\begin{longtable}[]{@{}lllll@{}}
\caption{\label{tab:OS-SAMPLE-metabolic-flux-matrix}OS SAMPLE metabolic flux matrix}\tabularnewline
\toprule
V1 & M\_1 & M\_2 & M\_3 & M\_4\tabularnewline
\midrule
\endfirsthead
\toprule
V1 & M\_1 & M\_2 & M\_3 & M\_4\tabularnewline
\midrule
\endhead
AAACCTGAGGATGGAA-1\_1 & 0.01224 & 0.01727 & 0.0649 & 0.1126\tabularnewline
AAACCTGCACAACGCC-1\_1 & 0.01224 & 0.02175 & 0.06084 & 0.07629\tabularnewline
AAACCTGTCATCATTC-1\_1 & 0.01618 & 0.02072 & 0.02948 & 0.03468\tabularnewline
AAACCTGTCGTCCGTT-1\_1 & 0.01832 & 0.07996 & 0.1203 & 0.2113\tabularnewline
AAACCTGTCTTGCATT-1\_1 & 0.01683 & 0.06143 & 0.09314 & 0.1331\tabularnewline
\ldots{} & \ldots{} & \ldots{} & \ldots{} & \ldots{}\tabularnewline
\bottomrule
\end{longtable}

\begin{center}\pgfornament[anchor=center,ydelta=0pt,width=9cm]{89}\vspace{1.5cm}\end{center}

Tab. \ref{tab:OS-SAMPLE-metabolic-flux-matrix}

\begin{center}\vspace{1.5cm}\pgfornament[anchor=center,ydelta=0pt,width=9cm]{88}\end{center}
\def\@captype{figure}
\begin{center}
\includegraphics[width = 0.9\linewidth]{Figure+Table/OS-SAMPLE-cells-metabolic-flux.pdf}
\caption{OS SAMPLE cells metabolic flux}\label{fig:OS-SAMPLE-cells-metabolic-flux}
\end{center}

\begin{center}\pgfornament[anchor=center,ydelta=0pt,width=9cm]{88}\vspace{1.5cm}\end{center}

Fig. \ref{fig:OS-SAMPLE-cells-metabolic-flux}

\hypertarget{seurat-ux7ec6ux80deux4e9aux7fa4ux5206ux6790-os_cancer}{%
\subsection{Seurat 细胞亚群分析 (OS\_CANCER)}\label{seurat-ux7ec6ux80deux4e9aux7fa4ux5206ux6790-os_cancer}}

成骨细胞和软骨细胞骨肉瘤是临床上常见的两种主要骨肉瘤类型(2020, \textbf{IF:14.7}, Q1, Nature communications)\textsuperscript{\protect\hyperlink{ref-Single_cell_RNA_Zhou_2020}{8}}。
在这里,聚焦于注释结果中的 Proliferating\_osteoblastic\_OS, Chondroblastic\_OS, Osteoblastic\_OS 细胞,重新聚类分析。
匹配 scsa\_cell 中包含"\_OS\$"的描述,最终得到 34230 例数据。分析其亚群。数据归一化,PCA 聚类 (Seurat 标准工作流,见方法章节) 后,绘制 PC standard deviations 图。在 1-15 PC 维度,1.2 分辨率下,对细胞群 UMAP 聚类。

\begin{center}\vspace{1.5cm}\pgfornament[anchor=center,ydelta=0pt,width=9cm]{88}\end{center}
\def\@captype{figure}
\begin{center}
\includegraphics[width = 0.9\linewidth]{Figure+Table/OS-CANCER-The-scsa-cell.pdf}
\caption{OS CANCER The scsa cell}\label{fig:OS-CANCER-The-scsa-cell}
\end{center}

\begin{center}\pgfornament[anchor=center,ydelta=0pt,width=9cm]{88}\vspace{1.5cm}\end{center}

Fig. \ref{fig:OS-CANCER-The-scsa-cell}

\hypertarget{seurat-copykat-ux764cux7ec6ux80deux6ce8ux91ca-os_cancer}{%
\subsubsection{Seurat-copyKAT 癌细胞注释 (OS\_CANCER)}\label{seurat-copykat-ux764cux7ec6ux80deux6ce8ux91ca-os_cancer}}

将 \texttt{CopyKAT} 的预测结果映射细胞注释中。

\begin{center}\vspace{1.5cm}\pgfornament[anchor=center,ydelta=0pt,width=9cm]{88}\end{center}
\def\@captype{figure}
\begin{center}
\includegraphics[width = 0.9\linewidth]{Figure+Table/OS-CANCER-Cancer-Cell-type-annotation.pdf}
\caption{OS CANCER Cancer Cell type annotation}\label{fig:OS-CANCER-Cancer-Cell-type-annotation}
\end{center}

\begin{center}\pgfornament[anchor=center,ydelta=0pt,width=9cm]{88}\vspace{1.5cm}\end{center}

\begin{center}\vspace{1.5cm}\pgfornament[anchor=center,ydelta=0pt,width=9cm]{88}\end{center}
\def\@captype{figure}
\begin{center}
\includegraphics[width = 0.9\linewidth]{Figure+Table/OS-CANCER-cancer-cell-proportions.pdf}
\caption{OS CANCER cancer cell proportions}\label{fig:OS-CANCER-cancer-cell-proportions}
\end{center}

\begin{center}\pgfornament[anchor=center,ydelta=0pt,width=9cm]{88}\vspace{1.5cm}\end{center}

Fig. \ref{fig:OS-CANCER-cancer-cell-proportions}

\hypertarget{limma-ux4ee3ux8c22ux901aux91cfux5deeux5f02ux5206ux6790-os_cancer_flux}{%
\subsubsection{Limma 代谢通量差异分析 (OS\_CANCER\_FLUX)}\label{limma-ux4ee3ux8c22ux901aux91cfux5deeux5f02ux5206ux6790-os_cancer_flux}}

匹配 scsa\_cell 中包含"\_OS\$"的描述,最终得到 17122 例数据。以 公式 \textasciitilde{} 0 + group 创建设计矩阵 (design matrix) 。差异分析:Malignant\_cell\_BC vs Benign\_cell\_BC。(若 A vs B,则为前者比后者,LogFC 大于 0 时,A 表达量高于 B)。上调或下调 DMFs 统计:up (n=29) , down (n=4)

\begin{center}\vspace{1.5cm}\pgfornament[anchor=center,ydelta=0pt,width=9cm]{88}\end{center}
\def\@captype{figure}
\begin{center}
\includegraphics[width = 0.9\linewidth]{Figure+Table/OS-CANCER-FLUX-Malignant-cell-BC-vs-Benign-cell-BC.pdf}
\caption{OS CANCER FLUX Malignant cell BC vs Benign cell BC}\label{fig:OS-CANCER-FLUX-Malignant-cell-BC-vs-Benign-cell-BC}
\end{center}

\begin{center}\pgfornament[anchor=center,ydelta=0pt,width=9cm]{88}\vspace{1.5cm}\end{center}\begin{center}\begin{tcolorbox}[colback=gray!10, colframe=gray!50, width=0.9\linewidth, arc=1mm, boxrule=0.5pt]
\textbf{
adj.P.Val cut-off
:}

\vspace{0.5em}

    0.05

\vspace{2em}


\textbf{
Log2(FC) cut-off
:}

\vspace{0.5em}

    0.5

\vspace{2em}
\end{tcolorbox}
\end{center}

\textbf{(See: \texttt{Figure+Table/3.4.2\_Limma\_代谢通量差异分析\_(OS\_CANCER\_FLUX)/OS-CANCER-FLUX-Malignant-cell-BC-vs-Benign-cell-BC-content})}

Fig. \ref{fig:OS-CANCER-FLUX-Malignant-cell-BC-vs-Benign-cell-BC}

\begin{center}\vspace{1.5cm}\pgfornament[anchor=center,ydelta=0pt,width=9cm]{89}\end{center}

\begin{longtable}[]{@{}lllll@{}}
\caption{\label{tab:OS-CANCER-FLUX-data-Malignant-cell-BC-vs-Benign-cell-BC}OS CANCER FLUX data Malignant cell BC vs Benign cell BC}\tabularnewline
\toprule
name & logFC & adj.P.Val & rownames & Module\_id\tabularnewline
\midrule
\endfirsthead
\toprule
name & logFC & adj.P.Val & rownames & Module\_id\tabularnewline
\midrule
\endhead
3PD -\textgreater{} Pyruvate & 1.227 & 0 & M\_4 & 4\tabularnewline
G3P -\textgreater{} 3PD & 1.211 & 0 & M\_3 & 3\tabularnewline
ADP -\textgreater{} Deoxyadeno\ldots{} & 1.015 & 0 & M\_140 & 140\tabularnewline
lysine -\textgreater{} Acetyl-CoA & 1.07 & 0 & M\_60 & 60\tabularnewline
Pyruvate -\textgreater{} Lactate & 1.147 & 0 & M\_6 & 6\tabularnewline
\ldots{} & \ldots{} & \ldots{} & \ldots{} & \ldots{}\tabularnewline
\bottomrule
\end{longtable}

\begin{center}\pgfornament[anchor=center,ydelta=0pt,width=9cm]{89}\vspace{1.5cm}\end{center}

Tab. \ref{tab:OS-CANCER-FLUX-data-Malignant-cell-BC-vs-Benign-cell-BC}

\begin{center}\vspace{1.5cm}\pgfornament[anchor=center,ydelta=0pt,width=9cm]{88}\end{center}
\def\@captype{figure}
\begin{center}
\includegraphics[width = 0.9\linewidth]{Figure+Table/OS-SAMPLE-Malignant-cell-Benign-cell-Cell-flux-ridge-plot.pdf}
\caption{OS SAMPLE Malignant cell Benign cell Cell flux ridge plot}\label{fig:OS-SAMPLE-Malignant-cell-Benign-cell-Cell-flux-ridge-plot}
\end{center}

\begin{center}\pgfornament[anchor=center,ydelta=0pt,width=9cm]{88}\vspace{1.5cm}\end{center}

Fig. \ref{fig:OS-SAMPLE-Malignant-cell-Benign-cell-Cell-flux-ridge-plot}

\hypertarget{tcga-ux6570ux636eux83b7ux53d6-os}{%
\subsection{TCGA 数据获取 (OS)}\label{tcga-ux6570ux636eux83b7ux53d6-os}}

获取 TARGET-OS 数据。

\hypertarget{cox-ux56deux5f52-tcga_os}{%
\subsection{COX 回归 (TCGA\_OS)}\label{cox-ux56deux5f52-tcga_os}}

将\textbf{基因集} (Malignant\_cell\_Benign\_cell, 来自于scFEA 单细胞数据的代谢通量预测{[}Section: OS\_SAMPLE{]}) 用于模型建立。共 298 个基因在数据集 TARGET-OS 中找到 (根据基因名匹配)。所有数据生存状态 (去除生存状态未知的数据),(Alive (n=57) , Dead (n=29) )。执行单因素 COX 回归,筛选 P 值 \textless{} 0.05,共筛选到 25 个基因。在单因素回归得到的基因 (P \textless{} 0.01) 的基础上,使用 \texttt{glmnet::cv.glmnet} 作 5 倍交叉验证 (评估方式为 C-index),筛选 lambda 值。lambda.min, lambda.1se 值分别为 0.006, 0.07 (R 随机种子为 987456)。对应的特征数 (基因数) 分别为 10, 10。

\begin{center}\vspace{1.5cm}\pgfornament[anchor=center,ydelta=0pt,width=9cm]{89}\end{center}

\begin{longtable}[]{@{}lllll@{}}
\caption{\label{tab:TCGA-OS-sig-Univariate-Cox-Coefficients}TCGA OS sig Univariate Cox Coefficients}\tabularnewline
\toprule
feature & coef & exp(coef) & se(coef) & z\tabularnewline
\midrule
\endfirsthead
\toprule
feature & coef & exp(coef) & se(coef) & z\tabularnewline
\midrule
\endhead
ACAT1 & 0.4221 & 1.525 & 0.2025 & 2.084\tabularnewline
UPRT & -0.6027 & 0.5473 & 0.208 & -2.897\tabularnewline
UGT2B10 & 0.3296 & 1.39 & 0.1409 & 2.339\tabularnewline
PCCB & 0.4687 & 1.598 & 0.175 & 2.679\tabularnewline
PGLS & -0.3914 & 0.6761 & 0.1773 & -2.207\tabularnewline
\ldots{} & \ldots{} & \ldots{} & \ldots{} & \ldots{}\tabularnewline
\bottomrule
\end{longtable}

\begin{center}\pgfornament[anchor=center,ydelta=0pt,width=9cm]{89}\vspace{1.5cm}\end{center}

\begin{center}\vspace{1.5cm}\pgfornament[anchor=center,ydelta=0pt,width=9cm]{88}\end{center}
\def\@captype{figure}
\begin{center}
\includegraphics[width = 0.9\linewidth]{Figure+Table/TCGA-OS-lasso-COX-model.pdf}
\caption{TCGA OS lasso COX model}\label{fig:TCGA-OS-lasso-COX-model}
\end{center}

\begin{center}\pgfornament[anchor=center,ydelta=0pt,width=9cm]{88}\vspace{1.5cm}\end{center}

\begin{center}\vspace{1.5cm}\pgfornament[anchor=center,ydelta=0pt,width=9cm]{88}\end{center}
\def\@captype{figure}
\begin{center}
\includegraphics[width = 0.9\linewidth]{Figure+Table/TCGA-OS-lasso-COX-coeffients-lambda-min.pdf}
\caption{TCGA OS lasso COX coeffients lambda min}\label{fig:TCGA-OS-lasso-COX-coeffients-lambda-min}
\end{center}

\begin{center}\pgfornament[anchor=center,ydelta=0pt,width=9cm]{88}\vspace{1.5cm}\end{center}

\begin{center}\vspace{1.5cm}\pgfornament[anchor=center,ydelta=0pt,width=9cm]{88}\end{center}
\def\@captype{figure}
\begin{center}
\includegraphics[width = 0.9\linewidth]{Figure+Table/TCGA-OS-lasso-COX-coeffients-lambda-1se.pdf}
\caption{TCGA OS lasso COX coeffients lambda 1se}\label{fig:TCGA-OS-lasso-COX-coeffients-lambda-1se}
\end{center}

\begin{center}\pgfornament[anchor=center,ydelta=0pt,width=9cm]{88}\vspace{1.5cm}\end{center}

\begin{center}\vspace{1.5cm}\pgfornament[anchor=center,ydelta=0pt,width=9cm]{88}\end{center}
\def\@captype{figure}
\begin{center}
\includegraphics[width = 0.9\linewidth]{Figure+Table/TCGA-OS-lasso-COX-ROC-lambda-min.pdf}
\caption{TCGA OS lasso COX ROC lambda min}\label{fig:TCGA-OS-lasso-COX-ROC-lambda-min}
\end{center}

\begin{center}\pgfornament[anchor=center,ydelta=0pt,width=9cm]{88}\vspace{1.5cm}\end{center}

\begin{center}\vspace{1.5cm}\pgfornament[anchor=center,ydelta=0pt,width=9cm]{88}\end{center}
\def\@captype{figure}
\begin{center}
\includegraphics[width = 0.9\linewidth]{Figure+Table/TCGA-OS-lasso-COX-ROC-lambda-1se.pdf}
\caption{TCGA OS lasso COX ROC lambda 1se}\label{fig:TCGA-OS-lasso-COX-ROC-lambda-1se}
\end{center}

\begin{center}\pgfornament[anchor=center,ydelta=0pt,width=9cm]{88}\vspace{1.5cm}\end{center}

\hypertarget{survival-ux751fux5b58ux5206ux6790-tcga_os}{%
\subsection{Survival 生存分析 (TCGA\_OS)}\label{survival-ux751fux5b58ux5206ux6790-tcga_os}}

选择 lambda.min 时得到的特征集,包含 10 个基因,
分别为: UPRT, PCCB, CYP2C8, UCK2, ALDH4A1, G6PD, FDPS, CTPS1, HSD11B2, ABAT。以回归系数构建风险评分模型。

\[ Score = \sum(expr(Gene) \times coef) \]
按 \texttt{survminer::surv\_cutpoint} 计算的 cutoff,
将样本分为 Low 和 High 风险组 (cutoff: 0.55168695257864)
(High (n=28) , Low (n=58) ), 随后进行生存分析。

\begin{center}\pgfornament[anchor=center,ydelta=0pt,width=9cm]{85}\vspace{1.5cm}\end{center}

\begin{center}\pgfornament[anchor=center,ydelta=0pt,width=9cm]{85}\vspace{1.5cm}\end{center}

\begin{center}\pgfornament[anchor=center,ydelta=0pt,width=9cm]{85}\vspace{1.5cm}\end{center}

\begin{center}\pgfornament[anchor=center,ydelta=0pt,width=9cm]{85}\vspace{1.5cm}\end{center}

\hypertarget{ux5916ux90e8ux6570ux636eux96c6ux9a8cux8bc1}{%
\subsection{外部数据集验证}\label{ux5916ux90e8ux6570ux636eux96c6ux9a8cux8bc1}}

\hypertarget{gse-ux6570ux636eux641cux7d22-os}{%
\subsubsection{GSE 数据搜索 (OS)}\label{gse-ux6570ux636eux641cux7d22-os}}

以 Entrez Direct (EDirect) 搜索 GEO 数据库 (检索条件见方法章节) 。
在检索匹配后,经人工确认,全部带有生存数据的 Osteosarcoma 为:GSE16091, GSE39055, GSE39057, GSE21257

\hypertarget{geo-ux6570ux636eux83b7ux53d6-os_gse39057}{%
\subsubsection{GEO 数据获取 (OS\_GSE39057)}\label{geo-ux6570ux636eux83b7ux53d6-os_gse39057}}

以 \texttt{GEOquery} 获取 GSE39057 的数据信息。

\hypertarget{geo-ux6570ux636eux83b7ux53d6-os_gse39055}{%
\subsubsection{GEO 数据获取 (OS\_GSE39055)}\label{geo-ux6570ux636eux83b7ux53d6-os_gse39055}}

以 \texttt{GEOquery} 获取 GSE39055 的数据信息。

\hypertarget{geo-ux6570ux636eux83b7ux53d6-os_gse16091}{%
\subsubsection{GEO 数据获取 (OS\_GSE16091)}\label{geo-ux6570ux636eux83b7ux53d6-os_gse16091}}

以 \texttt{GEOquery} 获取 GSE16091 的数据信息。

\hypertarget{geo-ux6570ux636eux83b7ux53d6-os_gse21257}{%
\subsubsection{GEO 数据获取 (OS\_GSE21257)}\label{geo-ux6570ux636eux83b7ux53d6-os_gse21257}}

以 \texttt{GEOquery} 获取 GSE21257 的数据信息。

\hypertarget{survival-ux751fux5b58ux5206ux6790-os_outer}{%
\subsubsection{Survival 生存分析 (OS\_OUTER)}\label{survival-ux751fux5b58ux5206ux6790-os_outer}}

对在 GEO 找到的所有具备生存信息的 Osteosarcoma 数据集做了外部验证。
合并数据集 (GSE16091, GSE39055, GSE39057, GSE21257)。对于不同注释来源的基因名,以 \texttt{org.Hs.eg.db::org.Hs.eg.db} 获取基因的别名 (ALIAS) ,根据 (ALIAS) 的一致性合并。查找预后模型中基因的 ALIAS,在未找到对应基因的情况下,使用该基因的 ALIAS 查找。 (原模型基因:UPRT, PCCB, CYP2C8, UCK2, ALDH4A1, G6PD, FDPS, CTPS1, HSD11B2, ABAT;以 ALIAS 匹配后,基因为:UPRT, PCCB, CYP2C8, UCK2, ALDH4A1, G6PD, FDPS, CTPS1, HSD11B2, ABAT) 。随后,将基因表达数据归一化 (Z-score)。按 \texttt{survminer::surv\_cutpoint} 计算的 cutoff,
将样本分为 Low 和 High 风险组 (cutoff: 1.12270604442518)
(High (n=15) , Low (n=94) ), 随后进行生存分析。此外,对于未合并前的各个数据集,以相同的方式生存分析。

\begin{center}\vspace{1.5cm}\pgfornament[anchor=center,ydelta=0pt,width=9cm]{88}\end{center}
\def\@captype{figure}
\begin{center}
\includegraphics[width = 0.9\linewidth]{Figure+Table/OS-OUTER-all-datasets-survival-plot.pdf}
\caption{OS OUTER all datasets survival plot}\label{fig:OS-OUTER-all-datasets-survival-plot}
\end{center}

\begin{center}\pgfornament[anchor=center,ydelta=0pt,width=9cm]{88}\vspace{1.5cm}\end{center}

Fig. \ref{fig:OS-OUTER-all-datasets-survival-plot}

\begin{center}\vspace{1.5cm}\pgfornament[anchor=center,ydelta=0pt,width=9cm]{88}\end{center}
\def\@captype{figure}
\begin{center}
\includegraphics[width = 0.9\linewidth]{Figure+Table/OS-OUTER-all-datasets-ROC-validation.pdf}
\caption{OS OUTER all datasets ROC validation}\label{fig:OS-OUTER-all-datasets-ROC-validation}
\end{center}

\begin{center}\pgfornament[anchor=center,ydelta=0pt,width=9cm]{88}\vspace{1.5cm}\end{center}

Fig. \ref{fig:OS-OUTER-all-datasets-ROC-validation}

\hypertarget{clusterprofiler-ux5bccux96c6ux5206ux6790-prog}{%
\subsection{ClusterProfiler 富集分析 (PROG)}\label{clusterprofiler-ux5bccux96c6ux5206ux6790-prog}}

对\textbf{基因集} (UPRT, PCCB, CYP2C8, \ldots{[}n = 10{]}, 来自于Survival 生存分析{[}Section: TCGA\_OS{]}) 进行ClusterProfiler 富集分析。

\begin{center}\vspace{1.5cm}\pgfornament[anchor=center,ydelta=0pt,width=9cm]{88}\end{center}
\def\@captype{figure}
\begin{center}
\includegraphics[width = 0.9\linewidth]{Figure+Table/PROG-KEGG-enrichment.pdf}
\caption{PROG KEGG enrichment}\label{fig:PROG-KEGG-enrichment}
\end{center}

\begin{center}\pgfornament[anchor=center,ydelta=0pt,width=9cm]{88}\vspace{1.5cm}\end{center}

Fig. \ref{fig:PROG-KEGG-enrichment}

\begin{center}\vspace{1.5cm}\pgfornament[anchor=center,ydelta=0pt,width=9cm]{88}\end{center}
\def\@captype{figure}
\begin{center}
\includegraphics[width = 0.9\linewidth]{Figure+Table/PROG-GO-enrichment.pdf}
\caption{PROG GO enrichment}\label{fig:PROG-GO-enrichment}
\end{center}

\begin{center}\pgfornament[anchor=center,ydelta=0pt,width=9cm]{88}\vspace{1.5cm}\end{center}

Fig. \ref{fig:PROG-GO-enrichment}

\hypertarget{conclusion}{%
\section{总结}\label{conclusion}}

癌症的本质特征与癌细胞自身代谢的改变息息相关 (2008, \textbf{IF:48.8}, Q1, Cancer cell)\textsuperscript{\protect\hyperlink{ref-Tumor_cell_meta_Kroeme_2008}{9}}。
对于骨肉瘤,目前仍缺少研究从单细胞水平探究癌症的代谢变化。
本分析从单细胞水平鉴定的恶质细胞 (肿瘤细胞) 出发,分析正常细胞与癌症细胞之间的代谢通量差异,
进而获取对应代谢模块的基因,建立预后模型,以代谢改变的角度,预测疾病的进展。
模型以 TARGET-OS 数据集建立,进而在 GEO 数据库搜索了所有可用的带有生存信息的基因表达数据集,
用以验证预后模型的可靠性。

\hypertarget{bibliography}{%
\section*{Reference}\label{bibliography}}
\addcontentsline{toc}{section}{Reference}

\hypertarget{refs}{}
\begin{cslreferences}
\leavevmode\hypertarget{ref-ScsaACellTyCaoY2020}{}%
1. Cao, Y., Wang, X. \& Peng, G. SCSA: A cell type annotation tool for single-cell rna-seq data. \emph{Frontiers in genetics} \textbf{11}, (2020).

\leavevmode\hypertarget{ref-DelineatingCopGaoR2021}{}%
2. Gao, R. \emph{et al.} Delineating copy number and clonal substructure in human tumors from single-cell transcriptomes. \emph{Nature Biotechnology} \textbf{39}, 599--608 (2021).

\leavevmode\hypertarget{ref-CausesAndConsGordon2012}{}%
3. Gordon, D. J., Resio, B. \& Pellman, D. Causes and consequences of aneuploidy in cancer. \emph{Nature Reviews Genetics} \textbf{13}, 189--203 (2012).

\leavevmode\hypertarget{ref-AGraphNeuralAlgham2021}{}%
4. Alghamdi, N. \emph{et al.} A graph neural network model to estimate cell-wise metabolic flux using single-cell rna-seq data. \emph{Genome research} \textbf{31}, 1867--1884 (2021).

\leavevmode\hypertarget{ref-LimmaLinearMSmyth2005}{}%
5. Smyth, G. K. Limma: Linear models for microarray data. in \emph{Bioinformatics and Computational Biology Solutions Using R and Bioconductor} (eds. Gentleman, R., Carey, V. J., Huber, W., Irizarry, R. A. \& Dudoit, S.) 397--420 (Springer-Verlag, 2005). doi:\href{https://doi.org/10.1007/0-387-29362-0_23}{10.1007/0-387-29362-0\_23}.

\leavevmode\hypertarget{ref-TcgabiolinksAColapr2015}{}%
6. Colaprico, A. \emph{et al.} TCGAbiolinks: An r/bioconductor package for integrative analysis of tcga data. \emph{Nucleic Acids Research} \textbf{44}, (2015).

\leavevmode\hypertarget{ref-ClusterprofilerWuTi2021}{}%
7. Wu, T. \emph{et al.} ClusterProfiler 4.0: A universal enrichment tool for interpreting omics data. \emph{The Innovation} \textbf{2}, (2021).

\leavevmode\hypertarget{ref-Single_cell_RNA_Zhou_2020}{}%
8. Zhou, Y. \emph{et al.} Single-cell rna landscape of intratumoral heterogeneity and immunosuppressive microenvironment in advanced osteosarcoma. \emph{Nature communications} \textbf{11}, (2020).

\leavevmode\hypertarget{ref-Tumor_cell_meta_Kroeme_2008}{}%
9. Kroemer, G. \& Pouyssegur, J. Tumor cell metabolism: Cancers achillesheel. \emph{Cancer cell} \textbf{13}, 472--482 (2008).
\end{cslreferences}

\end{document}
