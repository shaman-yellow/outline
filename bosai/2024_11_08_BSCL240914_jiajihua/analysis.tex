% Options for packages loaded elsewhere
\PassOptionsToPackage{unicode}{hyperref}
\PassOptionsToPackage{hyphens}{url}
%
\documentclass[
  ignorenonframetext,
]{beamer}
\usepackage{pgfpages}
\setbeamertemplate{caption}[numbered]
\setbeamertemplate{caption label separator}{: }
\setbeamercolor{caption name}{fg=normal text.fg}
\beamertemplatenavigationsymbolsempty
% Prevent slide breaks in the middle of a paragraph
\widowpenalties 1 10000
\raggedbottom
\setbeamertemplate{part page}{
  \centering
  \begin{beamercolorbox}[sep=16pt,center]{part title}
    \usebeamerfont{part title}\insertpart\par
  \end{beamercolorbox}
}
\setbeamertemplate{section page}{
  \centering
  \begin{beamercolorbox}[sep=12pt,center]{part title}
    \usebeamerfont{section title}\insertsection\par
  \end{beamercolorbox}
}
\setbeamertemplate{subsection page}{
  \centering
  \begin{beamercolorbox}[sep=8pt,center]{part title}
    \usebeamerfont{subsection title}\insertsubsection\par
  \end{beamercolorbox}
}
\AtBeginPart{
  \frame{\partpage}
}
\AtBeginSection{
  \ifbibliography
  \else
    \frame{\sectionpage}
  \fi
}
\AtBeginSubsection{
  \frame{\subsectionpage}
}
\usepackage{lmodern}
\usepackage{amssymb,amsmath}
\usepackage{ifxetex,ifluatex}
\ifnum 0\ifxetex 1\fi\ifluatex 1\fi=0 % if pdftex
  \usepackage[T1]{fontenc}
  \usepackage[utf8]{inputenc}
  \usepackage{textcomp} % provide euro and other symbols
\else % if luatex or xetex
  \usepackage{unicode-math}
  \defaultfontfeatures{Scale=MatchLowercase}
  \defaultfontfeatures[\rmfamily]{Ligatures=TeX,Scale=1}
\fi
% Use upquote if available, for straight quotes in verbatim environments
\IfFileExists{upquote.sty}{\usepackage{upquote}}{}
\IfFileExists{microtype.sty}{% use microtype if available
  \usepackage[]{microtype}
  \UseMicrotypeSet[protrusion]{basicmath} % disable protrusion for tt fonts
}{}
\makeatletter
\@ifundefined{KOMAClassName}{% if non-KOMA class
  \IfFileExists{parskip.sty}{%
    \usepackage{parskip}
  }{% else
    \setlength{\parindent}{0pt}
    \setlength{\parskip}{6pt plus 2pt minus 1pt}}
}{% if KOMA class
  \KOMAoptions{parskip=half}}
\makeatother
\usepackage{xcolor}
\IfFileExists{xurl.sty}{\usepackage{xurl}}{} % add URL line breaks if available
\IfFileExists{bookmark.sty}{\usepackage{bookmark}}{\usepackage{hyperref}}
\hypersetup{
  hidelinks,
  pdfcreator={LaTeX via pandoc}}
\urlstyle{same} % disable monospaced font for URLs
\newif\ifbibliography
\usepackage{longtable,booktabs}
\usepackage{caption}
% Make caption package work with longtable
\makeatletter
\def\fnum@table{\tablename~\thetable}
\makeatother
\setlength{\emergencystretch}{3em} % prevent overfull lines
\providecommand{\tightlist}{%
  \setlength{\itemsep}{0pt}\setlength{\parskip}{0pt}}
\setcounter{secnumdepth}{-\maxdimen} % remove section numbering

\author{}
\date{\vspace{-2.5em}}

\begin{document}

\begin{frame}
\begin{titlepage} \newgeometry{top=6.5cm}
\ThisCenterWallPaper{1.12}{~/outline/bosai//cover_page_analysis.pdf}
\begin{center} \textbf{\huge 预测甲基化调控因子}
\vspace{4em} \begin{textblock}{10}(3,4.85) \Large
\textbf{\textcolor{black}{BSCL240914}}
\end{textblock} \begin{textblock}{10}(3,5.8)
\Large \textbf{\textcolor{black}{黄礼闯}}
\end{textblock} \begin{textblock}{10}(3,6.75)
\Large
\textbf{\textcolor{black}{生信分析}}
\end{textblock} \begin{textblock}{10}(3,7.7)
\Large
\textbf{\textcolor{black}{邱美婷}}
\end{textblock} \end{center} \end{titlepage}
\restoregeometry

\pagenumbering{roman}

\begin{center}\vspace{1.5cm}\pgfornament[anchor=center,ydelta=0pt,width=8cm]{84}\end{center}\tableofcontents

\begin{center}\vspace{1.5cm}\pgfornament[anchor=center,ydelta=0pt,width=8cm]{88}\end{center}\listoffigures

\begin{center}\vspace{1.5cm}\pgfornament[anchor=center,ydelta=0pt,width=8cm]{89}\end{center}\listoftables



\pagenumbering{arabic}
\end{frame}

\begin{frame}{分析流程}
\protect\hypertarget{abstract}{}
\begin{block}{需求}
\protect\hypertarget{ux9700ux6c42}{}
通过软件预测甲基化调控因子(如METTL14)的靶基因,并通过数据库筛选于PCOS患者中表达水平具有显著差异性的基因,合并交集,并对该交集中的基因进行功能富集和KEGG通路富集分析,筛选PCOS患者中可能的METTL14甲基化调控基因及其相关通路;
\end{block}

\begin{block}{实际流程}
\protect\hypertarget{ux5b9eux9645ux6d41ux7a0b}{}
从 EpiFactors 获取表观遗传调控因子,筛出甲基化相关调控因子 (A 集合) 。
获取 PCOS GEO 数据,差异分析得到 DEGs,与 m6A-Atlas
数据库比对,发现可能存在甲基化修饰位点的基因 B 集合。 在 PCOS
中筛选出差异表达的甲基化调控因子 (C集合) ,与 B
集合关联分析,随后富集分析。
\end{block}
\end{frame}

\begin{frame}[fragile]{材料和方法}
\protect\hypertarget{introduction}{}
\begin{block}{数据分析平台}
\protect\hypertarget{ux6570ux636eux5206ux6790ux5e73ux53f0}{}
在 Linux pop-os x86\_64 (6.9.3-76060903-generic) 上,使用 R version
4.4.2 (2024-10-31) (\url{https://www.r-project.org/})
对数据统计分析与整合分析。
\end{block}

\begin{block}{EpiFactors 表观遗传调控因子数据获取 (Dataset: METHY)}
\protect\hypertarget{epifactors-ux8868ux89c2ux9057ux4f20ux8c03ux63a7ux56e0ux5b50ux6570ux636eux83b7ux53d6-dataset-methy}{}
从数据库 \texttt{EpiFactors} (2023, \textbf{IF:16.6}, Q1, Nucleic acids
research){[}@Epifactors2022Maraku2023{]} 获取表观遗传调控蛋白的数据。
\end{block}

\begin{block}{GEO 数据获取 (Dataset: PCOS)}
\protect\hypertarget{geo-ux6570ux636eux83b7ux53d6-dataset-pcos}{}
以 R 包 \texttt{GEOquery} (2.74.0) 获取 GSE277906 数据集。
\end{block}

\begin{block}{Limma 差异分析 (Dataset: PCOS)}
\protect\hypertarget{limma-ux5deeux5f02ux5206ux6790-dataset-pcos}{}
以 R 包 \texttt{limma} (3.62.1) (2005, \textbf{IF:}, ,
){[}@LimmaLinearMSmyth2005{]} \texttt{edgeR} (4.4.0) (, \textbf{IF:}, ,
){[}@EdgerDifferenChen{]} 进行差异分析。以 \texttt{edgeR::filterByExpr}
过滤 count 数量小于 10 的基因。以
\texttt{edgeR::calcNormFactors},\texttt{limma::voom} 转化 count 数据为
log2 counts-per-million (logCPM)。分析方法参考
\url{https://bioconductor.org/packages/release/workflows/vignettes/RNAseq123/inst/doc/limmaWorkflow.html}。随后,以
公式 \textasciitilde{} 0 + group 创建设计矩阵 (design matrix)
用于线性分析。 使用 \texttt{limma::lmFit},
\texttt{limma::contrasts.fit}, \texttt{limma::eBayes}
差异分析对比组:pcos vs control。以 \texttt{limma::topTable}
提取所有结果,并过滤得到 P.Value 小于 0.05,\textbar Log2(FC)\textbar{}
大于 0.5 的统计结果。 对 GSE277906 的 mRNA 数据 (protein\_coding)
差异分析
\end{block}

\begin{block}{富集分析 (Dataset: SIGCOR\_05)}
\protect\hypertarget{ux5bccux96c6ux5206ux6790-dataset-sigcor_05}{}
以 ClusterProfiler R 包 (4.15.0.2) (2021, \textbf{IF:33.2}, Q1, The
Innovation){[}@ClusterprofilerWuTi2021{]}进行 KEGG 和 GO 富集分析。 以
\texttt{pathview} R 包 (1.46.0) 对选择的 KEGG 通路可视化。
\end{block}
\end{frame}

\begin{frame}[fragile]{分析结果}
\protect\hypertarget{workflow}{}
\begin{block}{EpiFactors 表观遗传调控因子数据获取 (METHY)}
\protect\hypertarget{epifactors-ux8868ux89c2ux9057ux4f20ux8c03ux63a7ux56e0ux5b50ux6570ux636eux83b7ux53d6-methy}{}
从所有 表观调控因子 Fig.
@ref(fig:Distribution-all-protein-of-epigenetic-regulators)
中筛选出甲基化修饰调控因子,见 Tab. @ref(tab:METHY-regulators)

\begin{center}\vspace{1.5cm}\pgfornament[anchor=center,ydelta=0pt,width=9cm]{88}\end{center}
\def\@captype{figure}
\begin{center}
\includegraphics[width = 0.9\linewidth]{Figure+Table/Distribution-all-protein-of-epigenetic-regulators.pdf}
\caption{Distribution all protein of epigenetic regulators}\label{fig:Distribution-all-protein-of-epigenetic-regulators}
\end{center}

Figure @ref(fig:Distribution-all-protein-of-epigenetic-regulators)
(下方图) 为图Distribution all protein of epigenetic regulators概览。

\textbf{(File path:
\texttt{Figure+Table/Distribution-all-protein-of-epigenetic-regulators.pdf})}

\begin{center}\pgfornament[anchor=center,ydelta=0pt,width=9cm]{88}\vspace{1.5cm}\end{center}

\begin{center}\vspace{1.5cm}\pgfornament[anchor=center,ydelta=0pt,width=9cm]{89}\end{center}

\begin{longtable}[]{@{}llllllllll@{}}
\caption{METHY regulators}\tabularnewline
\toprule
Id & HGNC\_s\ldots{} & Status & HGNC\_ID & HGNC\_name & GeneID &
UniPro\ldots\ldots7 & UniPro\ldots\ldots8 & Domain &
MGI\_sy\ldots{}\tabularnewline
\midrule
\endfirsthead
\toprule
Id & HGNC\_s\ldots{} & Status & HGNC\_ID & HGNC\_name & GeneID &
UniPro\ldots\ldots7 & UniPro\ldots\ldots8 & Domain &
MGI\_sy\ldots{}\tabularnewline
\midrule
\endhead
11 & AEBP2 & \# & 24051 & AE bin\ldots{} & 121536 & Q6ZN18 &
AEBP2\_\ldots{} & Pfam-B\ldots{} & Aebp2\tabularnewline
12 & AICDA & \# & 13203 & activa\ldots{} & 57379 & Q9GZX7 &
AICDA\_\ldots{} & APOBEC\ldots{} & Aicda\tabularnewline
15 & ALKBH1 & New & 17911 & Nuclei\ldots{} & 8846 & Q13686 &
ALKB1\_\ldots{} & PF13532 & Alkbh1\tabularnewline
16 & ALKBH4 & New & 21900 & Alpha-\ldots{} & 54784 & Q9NXW9 &
ALKB4\_\ldots{} & PF13532 & Alkbh4\tabularnewline
17 & ALKBH5 & New & 25996 & alkB h\ldots{} & 54890 & Q6P6C2 &
ALKB5\_\ldots{} & PF13532 & Alkbh5\tabularnewline
\ldots{} & \ldots{} & \ldots{} & \ldots{} & \ldots{} & \ldots{} &
\ldots{} & \ldots{} & \ldots{} & \ldots{}\tabularnewline
\bottomrule
\end{longtable}

Table @ref(tab:METHY-regulators) (下方表格) 为表格METHY regulators概览。

\textbf{(File path: \texttt{Figure+Table/METHY-regulators.xlsx})}

\begin{center}\begin{tcolorbox}[colback=gray!10, colframe=gray!50, width=0.9\linewidth, arc=1mm, boxrule=0.5pt]注:表格共有175行25列,以下预览的表格可能省略部分数据;含有175个唯一`Id'。
\end{tcolorbox}
\end{center}

\begin{center}\pgfornament[anchor=center,ydelta=0pt,width=9cm]{89}\vspace{1.5cm}\end{center}
\end{block}

\begin{block}{GEO 数据获取 (PCOS)}
\protect\hypertarget{geo-ux6570ux636eux83b7ux53d6-pcos}{}
获取 GEO PCOS 数据,用于筛选差异表达基因。

\begin{center}\begin{tcolorbox}[colback=gray!10, colframe=gray!50, width=0.9\linewidth, arc=1mm, boxrule=0.5pt]
\textbf{
Data Source ID
:}

\vspace{0.5em}

    GSE277906

\vspace{2em}


\textbf{
data\_processing
:}

\vspace{0.5em}

    Illumina Casava1.7 software used for basecalling.

\vspace{2em}


\textbf{
data\_processing.1
:}

\vspace{0.5em}

    Raw reads of fastq format were firstly processed using
fastp and the low quality reads were removed to obtain the
clean reads.

\vspace{2em}


\textbf{
data\_processing.2
:}

\vspace{0.5em}

    The clean reads were mapped to the reference genome
using HISAT2. FPKM of each gene was calculated and the read
counts of each gene were obtained by HTSeq-count

\vspace{2em}


\textbf{
data\_processing.3
:}

\vspace{0.5em}

    Assembly: GRCh38

\vspace{2em}


\textbf{
(Others)
:}

\vspace{0.5em}

    ...

\vspace{2em}
\end{tcolorbox}
\end{center}

\textbf{(见 \texttt{Figure+Table/PCOS-GSE277906-content})}
\end{block}

\begin{block}{Limma 差异分析 (PCOS)}
\protect\hypertarget{limma-ux5deeux5f02ux5206ux6790-pcos}{}
差异分析,得到 DEGs 见 Fig. @ref(fig:PCOS-pcos-vs-control)

\begin{center}\vspace{1.5cm}\pgfornament[anchor=center,ydelta=0pt,width=9cm]{88}\end{center}
\def\@captype{figure}
\begin{center}
\includegraphics[width = 0.9\linewidth]{Figure+Table/PCOS-Filter-low-counts.pdf}
\caption{PCOS Filter low counts}\label{fig:PCOS-Filter-low-counts}
\end{center}

Figure @ref(fig:PCOS-Filter-low-counts) (下方图) 为图PCOS Filter low
counts概览。

\textbf{(File path: \texttt{Figure+Table/PCOS-Filter-low-counts.pdf})}

\begin{center}\pgfornament[anchor=center,ydelta=0pt,width=9cm]{88}\vspace{1.5cm}\end{center}

\begin{center}\vspace{1.5cm}\pgfornament[anchor=center,ydelta=0pt,width=9cm]{88}\end{center}
\def\@captype{figure}
\begin{center}
\includegraphics[width = 0.9\linewidth]{Figure+Table/PCOS-Normalization.pdf}
\caption{PCOS Normalization}\label{fig:PCOS-Normalization}
\end{center}

Figure @ref(fig:PCOS-Normalization) (下方图) 为图PCOS
Normalization概览。

\textbf{(File path: \texttt{Figure+Table/PCOS-Normalization.pdf})}

\begin{center}\pgfornament[anchor=center,ydelta=0pt,width=9cm]{88}\vspace{1.5cm}\end{center}

\begin{center}\vspace{1.5cm}\pgfornament[anchor=center,ydelta=0pt,width=9cm]{88}\end{center}
\def\@captype{figure}
\begin{center}
\includegraphics[width = 0.9\linewidth]{Figure+Table/PCOS-pcos-vs-control.pdf}
\caption{PCOS pcos vs control}\label{fig:PCOS-pcos-vs-control}
\end{center}

Figure @ref(fig:PCOS-pcos-vs-control) (下方图) 为图PCOS pcos vs
control概览。

\textbf{(File path: \texttt{Figure+Table/PCOS-pcos-vs-control.pdf})}

\begin{center}\pgfornament[anchor=center,ydelta=0pt,width=9cm]{88}\vspace{1.5cm}\end{center}\begin{center}\begin{tcolorbox}[colback=gray!10, colframe=gray!50, width=0.9\linewidth, arc=1mm, boxrule=0.5pt]
\textbf{
P.Value cut-off
:}

\vspace{0.5em}

    0.05

\vspace{2em}


\textbf{
Log2(FC) cut-off
:}

\vspace{0.5em}

    0.5

\vspace{2em}
\end{tcolorbox}
\end{center}

\textbf{(See: \texttt{Figure+Table/PCOS-pcos-vs-control-content})}

\begin{center}\vspace{1.5cm}\pgfornament[anchor=center,ydelta=0pt,width=9cm]{89}\end{center}

\begin{longtable}[]{@{}llllllllll@{}}
\caption{PCOS data pcos vs control}\tabularnewline
\toprule
rownames & id & gene\_D\ldots{} & coding\ldots{} & descri\ldots{} &
pathway & pathwa\ldots{} & GO\_ID & GO\_term & wiki\_ID\tabularnewline
\midrule
\endfirsthead
\toprule
rownames & id & gene\_D\ldots{} & coding\ldots{} & descri\ldots{} &
pathway & pathwa\ldots{} & GO\_ID & GO\_term & wiki\_ID\tabularnewline
\midrule
\endhead
PRPS2 & PRPS2 & 5634 & protei\ldots{} & phosph\ldots{} & hsa000\ldots{}
& Pentos\ldots{} & \url{GO:000}\ldots{} & magnes\ldots{}
&\tabularnewline
FXYD6 & FXYD6 & 53826 & protei\ldots{} & FXYD d\ldots{} & & &
\url{GO:000}\ldots{} & molecu\ldots{} &\tabularnewline
MMP15 & MMP15 & 4324 & protei\ldots{} & matrix\ldots{} & hsa04928 &
Parath\ldots{} & \url{GO:000}\ldots{} & metall\ldots{} &
WP5283\ldots{}\tabularnewline
CXCL16 & CXCL16 & 58191 & protei\ldots{} & C-X-C \ldots{} &
hsa040\ldots{} & Cytoki\ldots{} & \url{GO:000}\ldots{} & low-de\ldots{}
& WP5115\ldots{}\tabularnewline
MX1 & MX1 & 4599 & protei\ldots{} & MX dyn\ldots{} & hsa032\ldots{} &
Viral \ldots{} & \url{GO:000}\ldots{} & GTPase\ldots{} &
WP5115\ldots{}\tabularnewline
\ldots{} & \ldots{} & \ldots{} & \ldots{} & \ldots{} & \ldots{} &
\ldots{} & \ldots{} & \ldots{} & \ldots{}\tabularnewline
\bottomrule
\end{longtable}

Table @ref(tab:PCOS-data-pcos-vs-control) (下方表格) 为表格PCOS data
pcos vs control概览。

\textbf{(File path:
\texttt{Figure+Table/PCOS-data-pcos-vs-control.xlsx})}

\begin{center}\begin{tcolorbox}[colback=gray!10, colframe=gray!50, width=0.9\linewidth, arc=1mm, boxrule=0.5pt]注:表格共有177行19列,以下预览的表格可能省略部分数据;含有177个唯一`rownames'。
\end{tcolorbox}
\end{center}
\begin{center}\begin{tcolorbox}[colback=gray!10, colframe=gray!50, width=0.9\linewidth, arc=1mm, boxrule=0.5pt]\begin{enumerate}\tightlist
\item logFC:  estimate of the log2-fold-change corresponding to the effect or contrast (for ‘topTableF’ there may be several columns of log-fold-changes)
\item AveExpr:  average log2-expression for the probe over all arrays and channels, same as ‘Amean’ in the ‘MarrayLM’ object
\item t:  moderated t-statistic (omitted for ‘topTableF’)
\item P.Value:  raw p-value
\item B:  log-odds that the gene is differentially expressed (omitted for ‘topTreat’)
\end{enumerate}\end{tcolorbox}
\end{center}

\begin{center}\pgfornament[anchor=center,ydelta=0pt,width=9cm]{89}\vspace{1.5cm}\end{center}
\end{block}

\begin{block}{差异表达的 Methylation Factors}
\protect\hypertarget{ux5deeux5f02ux8868ux8fbeux7684-methylation-factors}{}
将差异表达基因与 Tab. @ref(tab:METHY-regulators) 中的因子取交集, 见
Fig. @ref(fig:Intersection-of-Methy-factor-with-DEGs) 。

\begin{center}\vspace{1.5cm}\pgfornament[anchor=center,ydelta=0pt,width=9cm]{88}\end{center}
\def\@captype{figure}
\begin{center}
\includegraphics[width = 0.9\linewidth]{Figure+Table/Intersection-of-Methy-factor-with-DEGs.pdf}
\caption{Intersection of Methy factor with DEGs}\label{fig:Intersection-of-Methy-factor-with-DEGs}
\end{center}

Figure @ref(fig:Intersection-of-Methy-factor-with-DEGs) (下方图)
为图Intersection of Methy factor with DEGs概览。

\textbf{(File path:
\texttt{Figure+Table/Intersection-of-Methy-factor-with-DEGs.pdf})}

\begin{center}\pgfornament[anchor=center,ydelta=0pt,width=9cm]{88}\vspace{1.5cm}\end{center}\begin{center}\begin{tcolorbox}[colback=gray!10, colframe=gray!50, width=0.9\linewidth, arc=1mm, boxrule=0.5pt]
\textbf{
All\_intersection
:}

\vspace{0.5em}

    PRDM6

\vspace{2em}
\end{tcolorbox}
\end{center}

\textbf{(See:
\texttt{Figure+Table/Intersection-of-Methy-factor-with-DEGs-content})}

\begin{center}\vspace{1.5cm}\pgfornament[anchor=center,ydelta=0pt,width=9cm]{89}\end{center}

\begin{longtable}[]{@{}llllllllll@{}}
\caption{Intersection METHY epigenetic regulators}\tabularnewline
\toprule
Id & HGNC\_s\ldots{} & Status & HGNC\_ID & HGNC\_name & GeneID &
UniPro\ldots{} & UniPro\ldots1 & Domain & MGI\_sy\ldots{}\tabularnewline
\midrule
\endfirsthead
\toprule
Id & HGNC\_s\ldots{} & Status & HGNC\_ID & HGNC\_name & GeneID &
UniPro\ldots{} & UniPro\ldots1 & Domain & MGI\_sy\ldots{}\tabularnewline
\midrule
\endhead
510 & PRDM6 & \# & 9350 & PR dom\ldots{} & 93166 & Q9NQX0 &
PRDM6\_\ldots{} & SET PF\ldots{} & Prdm6\tabularnewline
\bottomrule
\end{longtable}

Table @ref(tab:Intersection-METHY-epigenetic-regulators) (下方表格)
为表格Intersection METHY epigenetic regulators概览。

\textbf{(File path:
\texttt{Figure+Table/Intersection-METHY-epigenetic-regulators.xlsx})}

\begin{center}\begin{tcolorbox}[colback=gray!10, colframe=gray!50, width=0.9\linewidth, arc=1mm, boxrule=0.5pt]注:表格共有1行25列,以下预览的表格可能省略部分数据;含有1个唯一`Id'。
\end{tcolorbox}
\end{center}

\begin{center}\pgfornament[anchor=center,ydelta=0pt,width=9cm]{89}\vspace{1.5cm}\end{center}
\end{block}

\begin{block}{Methylation Factors 与 DEGs 关联分析}
\protect\hypertarget{methylation-factors-ux4e0e-degs-ux5173ux8054ux5206ux6790}{}
为了寻找 Fig. @ref(fig:Intersection-of-Methy-factor-with-DEGs)
中发现的差异表达的 Methylation Factors 可能调控的 DEGs
修饰,将两个数据集作关联分析,结果见 Tab.
@ref(tab:All-correlation-results) 。 以 pvalue \textless{} 0.05
为条件筛选,见 Tab. @ref(tab:correlation-results-05) , Fig.
@ref(fig:Significant-correlation) 。 其中,pvalue \textless{} 0.001
的见Fig. @ref(fig:correlation-results-001)。

\begin{center}\vspace{1.5cm}\pgfornament[anchor=center,ydelta=0pt,width=9cm]{89}\end{center}

\begin{longtable}[]{@{}llllllll@{}}
\caption{All correlation results}\tabularnewline
\toprule
From & To & cor & pvalue & model & -log2(\ldots{} & signif\ldots{} &
sign\tabularnewline
\midrule
\endfirsthead
\toprule
From & To & cor & pvalue & model & -log2(\ldots{} & signif\ldots{} &
sign\tabularnewline
\midrule
\endhead
PRDM6 & A2M & 0.1286\ldots{} & 0.49596 & c(Cont\ldots{} & 1.0117\ldots{}
& \textgreater{} 0.05 & -\tabularnewline
PRDM6 & AARD & -0.195\ldots{} & 0.35308 & c(Cont\ldots{} &
1.5019\ldots{} & \textgreater{} 0.05 & -\tabularnewline
PRDM6 & AATK & -0.035\ldots{} & 0.74928 & c(Cont\ldots{} &
0.4164\ldots{} & \textgreater{} 0.05 & -\tabularnewline
PRDM6 & ABCC9 & -0.146\ldots{} & 0.45304 & c(Cont\ldots{} &
1.1422\ldots{} & \textgreater{} 0.05 & -\tabularnewline
PRDM6 & ADAMTSL2 & 0.1576\ldots{} & 0.30011 & c(Cont\ldots{} &
1.7364\ldots{} & \textgreater{} 0.05 & -\tabularnewline
\ldots{} & \ldots{} & \ldots{} & \ldots{} & \ldots{} & \ldots{} &
\ldots{} & \ldots{}\tabularnewline
\bottomrule
\end{longtable}

Table @ref(tab:All-correlation-results) (下方表格) 为表格All correlation
results概览。

\textbf{(File path: \texttt{Figure+Table/All-correlation-results.xlsx})}

\begin{center}\begin{tcolorbox}[colback=gray!10, colframe=gray!50, width=0.9\linewidth, arc=1mm, boxrule=0.5pt]注:表格共有177行8列,以下预览的表格可能省略部分数据;含有1个唯一`From'。
\end{tcolorbox}
\end{center}

\begin{center}\pgfornament[anchor=center,ydelta=0pt,width=9cm]{89}\vspace{1.5cm}\end{center}

\begin{center}\vspace{1.5cm}\pgfornament[anchor=center,ydelta=0pt,width=9cm]{88}\end{center}
\def\@captype{figure}
\begin{center}
\includegraphics[width = 0.9\linewidth]{Figure+Table/Significant-correlation.pdf}
\caption{Significant correlation}\label{fig:Significant-correlation}
\end{center}

Figure @ref(fig:Significant-correlation) (下方图) 为图Significant
correlation概览。

\textbf{(File path: \texttt{Figure+Table/Significant-correlation.pdf})}

\begin{center}\pgfornament[anchor=center,ydelta=0pt,width=9cm]{88}\vspace{1.5cm}\end{center}

\begin{center}\vspace{1.5cm}\pgfornament[anchor=center,ydelta=0pt,width=9cm]{89}\end{center}

\begin{longtable}[]{@{}llllllll@{}}
\caption{Correlation results 05}\tabularnewline
\toprule
From & To & cor & pvalue & model & -log2(\ldots{} & signif\ldots{} &
sign\tabularnewline
\midrule
\endfirsthead
\toprule
From & To & cor & pvalue & model & -log2(\ldots{} & signif\ldots{} &
sign\tabularnewline
\midrule
\endhead
PRDM6 & ARL17A & -0.401\ldots{} & 0.019322 & c(Cont\ldots{} &
5.6936\ldots{} & \textless{} 0.05 & *\tabularnewline
PRDM6 & BSN & -0.295\ldots{} & 0.0035095 & c(Cont\ldots{} &
8.1545\ldots{} & \textless{} 0.05 & *\tabularnewline
PRDM6 & C1orf115 & 0.3889\ldots{} & 0.0074019 & c(Cont\ldots{} &
7.0778\ldots{} & \textless{} 0.05 & *\tabularnewline
PRDM6 & C2CD4C & 0.4359\ldots{} & 0.0027875 & c(Cont\ldots{} &
8.4868\ldots{} & \textless{} 0.05 & *\tabularnewline
PRDM6 & CLDN22 & -0.862\ldots{} & 0.0002\ldots{} & c(Cont\ldots{} &
11.862\ldots{} & \textless{} 0.001 & **\tabularnewline
\ldots{} & \ldots{} & \ldots{} & \ldots{} & \ldots{} & \ldots{} &
\ldots{} & \ldots{}\tabularnewline
\bottomrule
\end{longtable}

Table @ref(tab:correlation-results-05) (下方表格) 为表格correlation
results 05概览。

\textbf{(File path: \texttt{Figure+Table/correlation-results-05.xlsx})}

\begin{center}\begin{tcolorbox}[colback=gray!10, colframe=gray!50, width=0.9\linewidth, arc=1mm, boxrule=0.5pt]注:表格共有25行8列,以下预览的表格可能省略部分数据;含有1个唯一`From'。
\end{tcolorbox}
\end{center}

\begin{center}\pgfornament[anchor=center,ydelta=0pt,width=9cm]{89}\vspace{1.5cm}\end{center}

\begin{center}\vspace{1.5cm}\pgfornament[anchor=center,ydelta=0pt,width=9cm]{88}\end{center}
\def\@captype{figure}
\begin{center}
\includegraphics[width = 0.9\linewidth]{Figure+Table/correlation-results-001.pdf}
\caption{Correlation results 001}\label{fig:correlation-results-001}
\end{center}

Figure @ref(fig:correlation-results-001) (下方图) 为图correlation
results 001概览。

\textbf{(File path: \texttt{Figure+Table/correlation-results-001.pdf})}

\begin{center}\pgfornament[anchor=center,ydelta=0pt,width=9cm]{88}\vspace{1.5cm}\end{center}
\end{block}

\begin{block}{富集分析 (SIGCOR\_05)}
\protect\hypertarget{ux5bccux96c6ux5206ux6790-sigcor_05}{}
将 Tab. @ref(tab:correlation-results-05) 中的基因富集分析 (包含 PRDM6),

KEGG,GO 结果见 Fig. @ref(fig:SIGCOR-05-KEGG-enrichment), Fig.
@ref(fig:SIGCOR-05-GO-enrichment) 。 Fig.
@ref(fig:SIGCOR-05-hsa04024-visualization) 为 KEGG 中最为显著的 cAMP
通路,可能与 PCOS 中甲基化调控相关。 富集分析的数据表格见 Tab.
@ref(tab:SIGCOR-05-KEGG-enrichment-data), Tab.
@ref(tab:SIGCOR-05-GO-enrichment-data) 。

\begin{center}\vspace{1.5cm}\pgfornament[anchor=center,ydelta=0pt,width=9cm]{88}\end{center}
\def\@captype{figure}
\begin{center}
\includegraphics[width = 0.9\linewidth]{Figure+Table/SIGCOR-05-KEGG-enrichment.pdf}
\caption{SIGCOR 05 KEGG enrichment}\label{fig:SIGCOR-05-KEGG-enrichment}
\end{center}

Figure @ref(fig:SIGCOR-05-KEGG-enrichment) (下方图) 为图SIGCOR 05 KEGG
enrichment概览。

\textbf{(File path:
\texttt{Figure+Table/SIGCOR-05-KEGG-enrichment.pdf})}

\begin{center}\pgfornament[anchor=center,ydelta=0pt,width=9cm]{88}\vspace{1.5cm}\end{center}

\begin{center}\vspace{1.5cm}\pgfornament[anchor=center,ydelta=0pt,width=9cm]{88}\end{center}
\def\@captype{figure}
\begin{center}
\includegraphics[width = 0.9\linewidth]{Figure+Table/SIGCOR-05-GO-enrichment.pdf}
\caption{SIGCOR 05 GO enrichment}\label{fig:SIGCOR-05-GO-enrichment}
\end{center}

Figure @ref(fig:SIGCOR-05-GO-enrichment) (下方图) 为图SIGCOR 05 GO
enrichment概览。

\textbf{(File path: \texttt{Figure+Table/SIGCOR-05-GO-enrichment.pdf})}

\begin{center}\pgfornament[anchor=center,ydelta=0pt,width=9cm]{88}\vspace{1.5cm}\end{center}

\begin{center}\vspace{1.5cm}\pgfornament[anchor=center,ydelta=0pt,width=9cm]{88}\end{center}
\def\@captype{figure}
\begin{center}
\includegraphics[width = 0.9\linewidth]{pathview2024-11-29_17_49_18.568122/hsa04024.pathview.png}
\caption{SIGCOR 05 hsa04024 visualization}\label{fig:SIGCOR-05-hsa04024-visualization}
\end{center}

Figure @ref(fig:SIGCOR-05-hsa04024-visualization) (下方图) 为图SIGCOR 05
hsa04024 visualization概览。

\textbf{(File path:
\texttt{Figure+Table/SIGCOR-05-hsa04024-visualization.png})}

\begin{center}\begin{tcolorbox}[colback=gray!10, colframe=gray!50, width=0.9\linewidth, arc=1mm, boxrule=0.5pt]
\textbf{
Interactive figure
:}

\vspace{0.5em}

    \url{https://www.genome.jp/pathway/hsa04024}

\vspace{2em}


\textbf{
Enriched genes
:}

\vspace{0.5em}

    SSTR2, FXYD1, RAC3

\vspace{2em}
\end{tcolorbox}
\end{center}

\begin{center}\pgfornament[anchor=center,ydelta=0pt,width=9cm]{88}\vspace{1.5cm}\end{center}

\begin{center}\vspace{1.5cm}\pgfornament[anchor=center,ydelta=0pt,width=9cm]{89}\end{center}

\begin{longtable}[]{@{}llllllllll@{}}
\caption{SIGCOR 05 KEGG enrichment data}\tabularnewline
\toprule
category & subcat\ldots{} & ID & Descri\ldots{} & GeneRatio & BgRatio &
pvalue & p.adjust & qvalue & geneID\tabularnewline
\midrule
\endfirsthead
\toprule
category & subcat\ldots{} & ID & Descri\ldots{} & GeneRatio & BgRatio &
pvalue & p.adjust & qvalue & geneID\tabularnewline
\midrule
\endhead
Enviro\ldots{} & Signal\ldots{} & hsa04024 & cAMP s\ldots{} & 3/12 &
226/8868 & 0.0030\ldots{} & 0.1085\ldots{} & 0.1009\ldots{} &
5348/5\ldots{}\tabularnewline
Cellul\ldots{} & Cellul\ldots{} & hsa04520 & Adhere\ldots{} & 2/12 &
93/8868 & 0.0067\ldots{} & 0.1085\ldots{} & 0.1009\ldots{} &
5795/5881\tabularnewline
Human \ldots{} & Cancer\ldots{} & hsa05231 & Cholin\ldots{} & 2/12 &
99/8868 & 0.0075\ldots{} & 0.1085\ldots{} & 0.1009\ldots{} &
9468/5881\tabularnewline
Metabo\ldots{} & Carboh\ldots{} & hsa00030 & Pentos\ldots{} & 1/12 &
31/8868 & 0.0411\ldots{} & 0.2895\ldots{} & 0.2693\ldots{} &
5634\tabularnewline
Metabo\ldots{} & Carboh\ldots{} & hsa00051 & Fructo\ldots{} & 1/12 &
34/8868 & 0.0450\ldots{} & 0.2895\ldots{} & 0.2693\ldots{} &
3795\tabularnewline
\ldots{} & \ldots{} & \ldots{} & \ldots{} & \ldots{} & \ldots{} &
\ldots{} & \ldots{} & \ldots{} & \ldots{}\tabularnewline
\bottomrule
\end{longtable}

Table @ref(tab:SIGCOR-05-KEGG-enrichment-data) (下方表格) 为表格SIGCOR
05 KEGG enrichment data概览。

\textbf{(File path:
\texttt{Figure+Table/SIGCOR-05-KEGG-enrichment-data.xlsx})}

\begin{center}\begin{tcolorbox}[colback=gray!10, colframe=gray!50, width=0.9\linewidth, arc=1mm, boxrule=0.5pt]注:表格共有43行13列,以下预览的表格可能省略部分数据;含有6个唯一`category'。
\end{tcolorbox}
\end{center}

\begin{center}\pgfornament[anchor=center,ydelta=0pt,width=9cm]{89}\vspace{1.5cm}\end{center}

\begin{center}\vspace{1.5cm}\pgfornament[anchor=center,ydelta=0pt,width=9cm]{89}\end{center}

\begin{longtable}[]{@{}llllllllll@{}}
\caption{SIGCOR 05 GO enrichment data}\tabularnewline
\toprule
ont & ID & Descri\ldots{} & GeneRatio & BgRatio & pvalue & p.adjust &
qvalue & geneID & Count\tabularnewline
\midrule
\endfirsthead
\toprule
ont & ID & Descri\ldots{} & GeneRatio & BgRatio & pvalue & p.adjust &
qvalue & geneID & Count\tabularnewline
\midrule
\endhead
BP & \url{GO:003}\ldots{} & regula\ldots{} & 4/20 & 307/18986 &
0.0002\ldots{} & 0.0821\ldots{} & 0.0605\ldots{} & 1191/7\ldots{} &
4\tabularnewline
BP & \url{GO:200}\ldots{} & negati\ldots{} & 2/20 & 25/18986 &
0.0003\ldots{} & 0.0821\ldots{} & 0.0605\ldots{} & 25927/\ldots{} &
2\tabularnewline
BP & \url{GO:009}\ldots{} & granul\ldots{} & 3/20 & 156/18986 &
0.0005\ldots{} & 0.0821\ldots{} & 0.0605\ldots{} & 4354/5\ldots{} &
3\tabularnewline
BP & \url{GO:004}\ldots{} & positi\ldots{} & 3/20 & 159/18986 &
0.0005\ldots{} & 0.0821\ldots{} & 0.0605\ldots{} & 1191/7\ldots{} &
3\tabularnewline
BP & \url{GO:190}\ldots{} & regula\ldots{} & 2/20 & 47/18986 &
0.0011\ldots{} & 0.1115\ldots{} & 0.0822\ldots{} & 4354/5881 &
2\tabularnewline
\ldots{} & \ldots{} & \ldots{} & \ldots{} & \ldots{} & \ldots{} &
\ldots{} & \ldots{} & \ldots{} & \ldots{}\tabularnewline
\bottomrule
\end{longtable}

Table @ref(tab:SIGCOR-05-GO-enrichment-data) (下方表格) 为表格SIGCOR 05
GO enrichment data概览。

\textbf{(File path:
\texttt{Figure+Table/SIGCOR-05-GO-enrichment-data.xlsx})}

\begin{center}\begin{tcolorbox}[colback=gray!10, colframe=gray!50, width=0.9\linewidth, arc=1mm, boxrule=0.5pt]注:表格共有732行12列,以下预览的表格可能省略部分数据;含有3个唯一`ont'。
\end{tcolorbox}
\end{center}

\begin{center}\pgfornament[anchor=center,ydelta=0pt,width=9cm]{89}\vspace{1.5cm}\end{center}
\end{block}

\begin{block}{MusiteDeep 蛋白质转录后修饰位点预测 (SIGCOR\_05)}
\protect\hypertarget{musitedeep-ux86cbux767dux8d28ux8f6cux5f55ux540eux4feeux9970ux4f4dux70b9ux9884ux6d4b-sigcor_05}{}
\end{block}
\end{frame}

\begin{frame}{总结}
\protect\hypertarget{conclusion}{}
筛选的甲基化调控因子为 PRDM6,可能调控的基因见 Tab.
@ref(tab:correlation-results-05) , 富集分析结果中,cAMP
通路最为显著,Fig. @ref(fig:SIGCOR-05-hsa04024-visualization) 。
\end{frame}

\end{document}
