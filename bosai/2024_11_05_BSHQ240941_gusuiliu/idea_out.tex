% Options for packages loaded elsewhere
\PassOptionsToPackage{unicode}{hyperref}
\PassOptionsToPackage{hyphens}{url}
%
\documentclass[
]{article}
\usepackage{lmodern}
\usepackage{amssymb,amsmath}
\usepackage{ifxetex,ifluatex}
\ifnum 0\ifxetex 1\fi\ifluatex 1\fi=0 % if pdftex
  \usepackage[T1]{fontenc}
  \usepackage[utf8]{inputenc}
  \usepackage{textcomp} % provide euro and other symbols
\else % if luatex or xetex
  \usepackage{unicode-math}
  \defaultfontfeatures{Scale=MatchLowercase}
  \defaultfontfeatures[\rmfamily]{Ligatures=TeX,Scale=1}
\fi
% Use upquote if available, for straight quotes in verbatim environments
\IfFileExists{upquote.sty}{\usepackage{upquote}}{}
\IfFileExists{microtype.sty}{% use microtype if available
  \usepackage[]{microtype}
  \UseMicrotypeSet[protrusion]{basicmath} % disable protrusion for tt fonts
}{}
\makeatletter
\@ifundefined{KOMAClassName}{% if non-KOMA class
  \IfFileExists{parskip.sty}{%
    \usepackage{parskip}
  }{% else
    \setlength{\parindent}{0pt}
    \setlength{\parskip}{6pt plus 2pt minus 1pt}}
}{% if KOMA class
  \KOMAoptions{parskip=half}}
\makeatother
\usepackage{xcolor}
\IfFileExists{xurl.sty}{\usepackage{xurl}}{} % add URL line breaks if available
\IfFileExists{bookmark.sty}{\usepackage{bookmark}}{\usepackage{hyperref}}
\hypersetup{
  hidelinks,
  pdfcreator={LaTeX via pandoc}}
\urlstyle{same} % disable monospaced font for URLs
\usepackage[margin=1in]{geometry}
\usepackage{longtable,booktabs}
% Correct order of tables after \paragraph or \subparagraph
\usepackage{etoolbox}
\makeatletter
\patchcmd\longtable{\par}{\if@noskipsec\mbox{}\fi\par}{}{}
\makeatother
% Allow footnotes in longtable head/foot
\IfFileExists{footnotehyper.sty}{\usepackage{footnotehyper}}{\usepackage{footnote}}
\makesavenoteenv{longtable}
\usepackage{graphicx}
\makeatletter
\def\maxwidth{\ifdim\Gin@nat@width>\linewidth\linewidth\else\Gin@nat@width\fi}
\def\maxheight{\ifdim\Gin@nat@height>\textheight\textheight\else\Gin@nat@height\fi}
\makeatother
% Scale images if necessary, so that they will not overflow the page
% margins by default, and it is still possible to overwrite the defaults
% using explicit options in \includegraphics[width, height, ...]{}
\setkeys{Gin}{width=\maxwidth,height=\maxheight,keepaspectratio}
% Set default figure placement to htbp
\makeatletter
\def\fps@figure{htbp}
\makeatother
\setlength{\emergencystretch}{3em} % prevent overfull lines
\providecommand{\tightlist}{%
  \setlength{\itemsep}{0pt}\setlength{\parskip}{0pt}}
\setcounter{secnumdepth}{5}
\usepackage{tikz} \usepackage{auto-pst-pdf} \usepackage{pgfornament} \usepackage{pstricks-add} \usepackage{caption} \captionsetup{font={footnotesize},width=6in} \renewcommand{\dblfloatpagefraction}{.9} \makeatletter \renewenvironment{figure} {\def\@captype{figure}} \makeatother \@ifundefined{Shaded}{\newenvironment{Shaded}} \@ifundefined{snugshade}{\newenvironment{snugshade}} \renewenvironment{Shaded}{\begin{snugshade}}{\end{snugshade}} \definecolor{shadecolor}{RGB}{230,230,230} \usepackage{xeCJK} \usepackage{setspace} \setstretch{1.3} \usepackage{tcolorbox} \setcounter{secnumdepth}{4} \setcounter{tocdepth}{4} \usepackage{wallpaper} \usepackage[absolute]{textpos} \tcbuselibrary{breakable} \renewenvironment{Shaded} {\begin{tcolorbox}[colback = gray!10, colframe = gray!40, width = 16cm, arc = 1mm, auto outer arc, title = {R input}]} {\end{tcolorbox}} \usepackage{titlesec} \titleformat{\paragraph} {\fontsize{10pt}{0pt}\bfseries} {\arabic{section}.\arabic{subsection}.\arabic{subsubsection}.\arabic{paragraph}} {1em} {} []
\newlength{\cslhangindent}
\setlength{\cslhangindent}{1.5em}
\newenvironment{cslreferences}%
  {}%
  {\par}

\author{}
\date{\vspace{-2.5em}}

\begin{document}

\begin{titlepage} \newgeometry{top=7.5cm}
\ThisCenterWallPaper{1.12}{~/outline/bosai//cover_page.pdf}
\begin{center} \textbf{\huge 骨髓瘤思路设计}
\vspace{4em} \begin{textblock}{10}(3.2,9.25)
\huge \textbf{\textcolor{black}{2024-11-07}}
\end{textblock} \end{center} \end{titlepage}
\restoregeometry

\pagenumbering{roman}

\begin{center}\vspace{1.5cm}\pgfornament[anchor=center,ydelta=0pt,width=8cm]{84}\end{center}\tableofcontents

\begin{center}\vspace{1.5cm}\pgfornament[anchor=center,ydelta=0pt,width=8cm]{88}\end{center}\listoffigures

\begin{center}\vspace{1.5cm}\pgfornament[anchor=center,ydelta=0pt,width=8cm]{89}\end{center}\listoftables

\newpage

\pagenumbering{arabic}

\hypertarget{abstract}{%
\section{研究背景}\label{abstract}}

Multiple myeloma (MM) 是一种基因复杂、异质性高的疾病,其发展是一个多步骤的过程,涉及肿瘤细胞基因改变的获得和骨髓微环境的变化 (2024, Nature reviews. Disease primers, \textbf{IF:76.9}, Q1)\textsuperscript{\protect\hyperlink{ref-MultipleMyelomMalard2024}{1}}。

\hypertarget{introduction}{%
\subsection{思路}\label{introduction}}

结合 MM 的 GWAS 研究 (变异与疾病的关系) ,预测基因表达变化水平 (即TWAS,基因与疾病的关系) ;MM 的 scRNA-seq 肿瘤细胞分析,并进一步预测肿瘤细胞的代谢变化; 最后,聚焦于基因对肿瘤细胞的代谢改变,以及对应的功能基因。

思路为: TWAS (GWAS + eQTL) + scRNA-seq + metabolic

(TWAS 部分可能会相对耗时,因为该部分的方法为首次接触,需要配置程序)

\hypertarget{methods}{%
\section{可行性}\label{methods}}

\hypertarget{ux4ee5-multiple-myeloma-and-metabolic-ux641cux7d22ux6587ux732eux53d1ux73b0-mm-ux4e0eux4ee3ux8c22ux5173ux8054ux5bc6ux5207}{%
\subsection{\texorpdfstring{以 \texttt{"Multiple\ myeloma"\ AND\ "metabolic"} 搜索文献,发现 MM 与代谢关联密切。}{以 "Multiple myeloma" AND "metabolic" 搜索文献,发现 MM 与代谢关联密切。}}\label{ux4ee5-multiple-myeloma-and-metabolic-ux641cux7d22ux6587ux732eux53d1ux73b0-mm-ux4e0eux4ee3ux8c22ux5173ux8054ux5bc6ux5207}}

\begin{center}\vspace{1.5cm}\pgfornament[anchor=center,ydelta=0pt,width=9cm]{88}\end{center}
\def\@captype{figure}
\begin{center}
\includegraphics[width = 0.9\linewidth]{~/Pictures/Screenshots/Screenshot from 2024-11-07 14-37-56.png}
\caption{Unnamed chunk 6}\label{fig:unnamed-chunk-6}
\end{center}

\begin{center}\pgfornament[anchor=center,ydelta=0pt,width=9cm]{88}\vspace{1.5cm}\end{center}

\hypertarget{ux4ee5-multiple-myeloma-and-twas-ux641cux7d22ux6587ux732eux5df2ux6709ux501fux52a9-twas-ux7814ux7a76-mm-ux7684ux6587ux7ae0}{%
\subsection{\texorpdfstring{以 \texttt{"Multiple\ myeloma"\ AND\ "TWAS"} 搜索文献,已有借助 TWAS 研究 MM 的文章。}{以 "Multiple myeloma" AND "TWAS" 搜索文献,已有借助 TWAS 研究 MM 的文章。}}\label{ux4ee5-multiple-myeloma-and-twas-ux641cux7d22ux6587ux732eux5df2ux6709ux501fux52a9-twas-ux7814ux7a76-mm-ux7684ux6587ux7ae0}}

\begin{center}\vspace{1.5cm}\pgfornament[anchor=center,ydelta=0pt,width=9cm]{88}\end{center}
\def\@captype{figure}
\begin{center}
\includegraphics[width = 0.9\linewidth]{~/Pictures/Screenshots/Screenshot from 2024-11-07 14-36-52.png}
\caption{Unnamed chunk 7}\label{fig:unnamed-chunk-7}
\end{center}

\begin{center}\pgfornament[anchor=center,ydelta=0pt,width=9cm]{88}\vspace{1.5cm}\end{center}

\hypertarget{ux4ee5-multiple-myeloma-and-metabolic-and-gwas-ux641cux7d22ux6587ux732eux53d1ux73b0ux4e00ux7bc7ux5b5fux5fb7ux5c14ux968fux673aux5316ux7814ux7a76mm-ux57faux56e0ux4e0eux4ee3ux8c22ux7684ux5173ux7cfb}{%
\subsection{\texorpdfstring{以 \texttt{"Multiple\ myeloma"\ AND\ "metabolic"\ AND\ "GWAS"} 搜索文献,发现一篇孟德尔随机化研究,MM 基因与代谢的关系。}{以 "Multiple myeloma" AND "metabolic" AND "GWAS" 搜索文献,发现一篇孟德尔随机化研究,MM 基因与代谢的关系。}}\label{ux4ee5-multiple-myeloma-and-metabolic-and-gwas-ux641cux7d22ux6587ux732eux53d1ux73b0ux4e00ux7bc7ux5b5fux5fb7ux5c14ux968fux673aux5316ux7814ux7a76mm-ux57faux56e0ux4e0eux4ee3ux8c22ux7684ux5173ux7cfb}}

\begin{center}\vspace{1.5cm}\pgfornament[anchor=center,ydelta=0pt,width=9cm]{88}\end{center}
\def\@captype{figure}
\begin{center}
\includegraphics[width = 0.9\linewidth]{~/Pictures/Screenshots/Screenshot from 2024-11-07 14-39-32.png}
\caption{Unnamed chunk 8}\label{fig:unnamed-chunk-8}
\end{center}

\begin{center}\pgfornament[anchor=center,ydelta=0pt,width=9cm]{88}\vspace{1.5cm}\end{center}

\hypertarget{results}{%
\section{创新性}\label{results}}

\hypertarget{ux4ee5-multiple-myeloma-and-metabolic-and-twas-ux641cux7d22ux6587ux732eux672aux53d1ux73b0ux76f8ux5173ux7814ux7a76}{%
\subsection{\texorpdfstring{以 \texttt{"Multiple\ myeloma"\ AND\ "metabolic"\ AND\ "TWAS"} 搜索文献,未发现相关研究。}{以 "Multiple myeloma" AND "metabolic" AND "TWAS" 搜索文献,未发现相关研究。}}\label{ux4ee5-multiple-myeloma-and-metabolic-and-twas-ux641cux7d22ux6587ux732eux672aux53d1ux73b0ux76f8ux5173ux7814ux7a76}}

\begin{center}\vspace{1.5cm}\pgfornament[anchor=center,ydelta=0pt,width=9cm]{88}\end{center}
\def\@captype{figure}
\begin{center}
\includegraphics[width = 0.9\linewidth]{~/Pictures/Screenshots/Screenshot from 2024-11-07 14-40-57.png}
\caption{Unnamed chunk 9}\label{fig:unnamed-chunk-9}
\end{center}

\begin{center}\pgfornament[anchor=center,ydelta=0pt,width=9cm]{88}\vspace{1.5cm}\end{center}

\hypertarget{ux4ee5-multiple-myeloma-and-scrna-seq-and-metabolic-and-gwas-ux641cux7d22-pubmedux672aux53d1ux73b0ux76f8ux5173ux7814ux7a76}{%
\subsection{\texorpdfstring{以 \texttt{"Multiple\ myeloma"\ AND\ "scRNA-seq"\ AND\ "metabolic"\ AND\ "GWAS"} 搜索 PubMed,未发现相关研究。}{以 "Multiple myeloma" AND "scRNA-seq" AND "metabolic" AND "GWAS" 搜索 PubMed,未发现相关研究。}}\label{ux4ee5-multiple-myeloma-and-scrna-seq-and-metabolic-and-gwas-ux641cux7d22-pubmedux672aux53d1ux73b0ux76f8ux5173ux7814ux7a76}}

\begin{center}\vspace{1.5cm}\pgfornament[anchor=center,ydelta=0pt,width=9cm]{88}\end{center}
\def\@captype{figure}
\begin{center}
\includegraphics[width = 0.9\linewidth]{~/Pictures/Screenshots/Screenshot from 2024-11-07 14-41-29.png}
\caption{Unnamed chunk 10}\label{fig:unnamed-chunk-10}
\end{center}

\begin{center}\pgfornament[anchor=center,ydelta=0pt,width=9cm]{88}\vspace{1.5cm}\end{center}

\hypertarget{workflow}{%
\section{参考文献和数据集}\label{workflow}}

\hypertarget{twas-ux65b9ux6cd5}{%
\subsection{TWAS 方法}\label{twas-ux65b9ux6cd5}}

\begin{itemize}
\tightlist
\item
  FUSION (2016, Nature Genetics, \textbf{IF:31.7}, Q1)\textsuperscript{\protect\hyperlink{ref-IntegrativeAppGusev2016}{2}}
\item
  FOCUS (2020, Human genetics, \textbf{IF:3.8}, Q2)\textsuperscript{\protect\hyperlink{ref-APowerfulFineWuCh2020}{3}}
\end{itemize}

\hypertarget{ux5355ux7ec6ux80deux6570ux636eux9884ux6d4bux4ee3ux8c22ux901aux91cfux7684ux65b9ux6cd5}{%
\subsection{单细胞数据预测代谢通量的方法}\label{ux5355ux7ec6ux80deux6570ux636eux9884ux6d4bux4ee3ux8c22ux901aux91cfux7684ux65b9ux6cd5}}

\begin{itemize}
\tightlist
\item
  scFEA 通过scRNA-seq 预测代谢通量 (2021, Genome research, \textbf{IF:6.2}, Q1)\textsuperscript{\protect\hyperlink{ref-AGraphNeuralAlgham2021}{4}}
\item
  scFEA 的应用实例 (2023, Frontiers in endocrinology, \textbf{IF:3.9}, Q2)\textsuperscript{\protect\hyperlink{ref-SingleCellCorAgoro2023}{5}}
\end{itemize}

\hypertarget{gwas-ux6570ux636e}{%
\subsection{GWAS 数据}\label{gwas-ux6570ux636e}}

\begin{center}\vspace{1.5cm}\pgfornament[anchor=center,ydelta=0pt,width=9cm]{89}\end{center}

\begin{longtable}[]{@{}llllllllll@{}}
\caption{\label{tab:GWAS}GWAS}\tabularnewline
\toprule
id & trait & ncase & group\_\ldots{} & year & author & consor\ldots{} & sex & pmid & popula\ldots{}\tabularnewline
\midrule
\endfirsthead
\toprule
id & trait & ncase & group\_\ldots{} & year & author & consor\ldots{} & sex & pmid & popula\ldots{}\tabularnewline
\midrule
\endhead
ieu-b-\ldots{} & Multip\ldots{} & 601 & public & 2021 & Burrows & UK Bio\ldots{} & Males \ldots{} & NA & European\tabularnewline
finn-b\ldots{} & Multip\ldots{} & 598 & public & 2021 & NA & NA & Males \ldots{} & NA & European\tabularnewline
finn-b\ldots{} & Multip\ldots{} & 598 & public & 2021 & NA & NA & Males \ldots{} & NA & European\tabularnewline
\bottomrule
\end{longtable}

\begin{center}\pgfornament[anchor=center,ydelta=0pt,width=9cm]{89}\vspace{1.5cm}\end{center}

\hypertarget{scrna-seq}{%
\subsection{scRNA-seq}\label{scrna-seq}}

GEO 上有多数 MM 的 scRNA-seq 数据集,以下举一例。

\begin{itemize}
\tightlist
\item
  GSE271107
\end{itemize}

\begin{center}\begin{tcolorbox}[colback=gray!10, colframe=gray!50, width=0.9\linewidth, arc=1mm, boxrule=0.5pt]
\textbf{
Data Source ID
:}

\vspace{0.5em}

    GSE271107

\vspace{2em}


\textbf{
data\_processing
:}

\vspace{0.5em}

    Raw scRNA-seq data were preprocessed using the Cell
Ranger analysis pipelines (10x Genomics) version 6 with
reference genome of human genome (GRCh38) to demultiplex
for cell and transcript and generate count table.

\vspace{2em}


\textbf{
data\_processing.1
:}

\vspace{0.5em}

    The count table was loaded into R through Seurat
version 4 package for further analysis. Cells that have
gene numbers lesser than 200, greater than 7,000, and more
than 10% of unique molecular identifiers stemming from
mitochondrial genes were discarded from the analysis.

\vspace{2em}


\textbf{
data\_processing.2
:}

\vspace{0.5em}

    For individual sample, a principal component analysis
(PCA) was performed on significantly variable genes for
remained high-quality cells. Results of individual samples
were used for data integration across samples using
reciprocal PCA method to minimize technical differences
between samples.

\vspace{2em}


\textbf{
data\_processing.3
:}

\vspace{0.5em}

    The integration results were employed as input for
clustering using Louvain algorithm with multilevel
refinement and the Uniform Manifold Approximation and
Projection for Dimension Reduction (UMAP).

\vspace{2em}


\textbf{
(Others)
:}

\vspace{0.5em}

    ...

\vspace{2em}
\end{tcolorbox}
\end{center}

\textbf{(上述信息框内容已保存至 \texttt{Figure+Table/prods-content})}

\begin{center}\vspace{1.5cm}\pgfornament[anchor=center,ydelta=0pt,width=9cm]{89}\end{center}

\begin{longtable}[]{@{}llll@{}}
\caption{\label{tab:sample}Sample}\tabularnewline
\toprule
rownames & title & disease.state.ch1 & tissue.ch1\tabularnewline
\midrule
\endfirsthead
\toprule
rownames & title & disease.state.ch1 & tissue.ch1\tabularnewline
\midrule
\endhead
GSM8369863 & Healthy donor\_1 & Healthy & Bone marrow aspirate\tabularnewline
GSM8369864 & Healthy donor\_2 & Healthy & Bone marrow aspirate\tabularnewline
GSM8369865 & Healthy donor\_3 & Healthy & Bone marrow aspirate\tabularnewline
GSM8369866 & Healthy donor\_4 & Healthy & Bone marrow aspirate\tabularnewline
GSM8369867 & Healthy donor\_5 & Healthy & Bone marrow aspirate\tabularnewline
GSM8369868 & MGUS\_1 & MGUS & Bone marrow aspirate\tabularnewline
GSM8369869 & MGUS\_2 & MGUS & Bone marrow aspirate\tabularnewline
GSM8369870 & MGUS\_3 & MGUS & Bone marrow aspirate\tabularnewline
GSM8369871 & MGUS\_4 & MGUS & Bone marrow aspirate\tabularnewline
GSM8369872 & MGUS\_5 & MGUS & Bone marrow aspirate\tabularnewline
GSM8369873 & MGUS\_6 & MGUS & Bone marrow aspirate\tabularnewline
GSM8369874 & SMM\_1 & SMM & Bone marrow aspirate\tabularnewline
GSM8369875 & SMM\_2 & SMM & Bone marrow aspirate\tabularnewline
GSM8369876 & SMM\_3 & SMM & Bone marrow aspirate\tabularnewline
GSM8369877 & SMM\_4 & SMM & Bone marrow aspirate\tabularnewline
\ldots{} & \ldots{} & \ldots{} & \ldots{}\tabularnewline
\bottomrule
\end{longtable}

\begin{center}\pgfornament[anchor=center,ydelta=0pt,width=9cm]{89}\vspace{1.5cm}\end{center}

\hypertarget{bibliography}{%
\section*{Reference}\label{bibliography}}
\addcontentsline{toc}{section}{Reference}

\hypertarget{refs}{}
\begin{cslreferences}
\leavevmode\hypertarget{ref-MultipleMyelomMalard2024}{}%
1. Malard, F. \emph{et al.} Multiple myeloma. \emph{Nature reviews. Disease primers} \textbf{10}, (2024).

\leavevmode\hypertarget{ref-IntegrativeAppGusev2016}{}%
2. Gusev, A. \emph{et al.} Integrative approaches for large-scale transcriptome-wide association studies. \emph{Nature Genetics} \textbf{48}, 245--252 (2016).

\leavevmode\hypertarget{ref-APowerfulFineWuCh2020}{}%
3. Wu, C. \& Pan, W. A powerful fine-mapping method for transcriptome-wide association studies. \emph{Human genetics} \textbf{139}, 199--213 (2020).

\leavevmode\hypertarget{ref-AGraphNeuralAlgham2021}{}%
4. Alghamdi, N. \emph{et al.} A graph neural network model to estimate cell-wise metabolic flux using single-cell rna-seq data. \emph{Genome research} \textbf{31}, 1867--1884 (2021).

\leavevmode\hypertarget{ref-SingleCellCorAgoro2023}{}%
5. Agoro, R. \emph{et al.} Single cell cortical bone transcriptomics define novel osteolineage gene sets altered in chronic kidney disease. \emph{Frontiers in endocrinology} \textbf{14}, (2023).
\end{cslreferences}

\end{document}
