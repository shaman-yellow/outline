% Options for packages loaded elsewhere
\PassOptionsToPackage{unicode}{hyperref}
\PassOptionsToPackage{hyphens}{url}
%
\documentclass[
]{article}
\usepackage{lmodern}
\usepackage{amssymb,amsmath}
\usepackage{ifxetex,ifluatex}
\ifnum 0\ifxetex 1\fi\ifluatex 1\fi=0 % if pdftex
  \usepackage[T1]{fontenc}
  \usepackage[utf8]{inputenc}
  \usepackage{textcomp} % provide euro and other symbols
\else % if luatex or xetex
  \usepackage{unicode-math}
  \defaultfontfeatures{Scale=MatchLowercase}
  \defaultfontfeatures[\rmfamily]{Ligatures=TeX,Scale=1}
\fi
% Use upquote if available, for straight quotes in verbatim environments
\IfFileExists{upquote.sty}{\usepackage{upquote}}{}
\IfFileExists{microtype.sty}{% use microtype if available
  \usepackage[]{microtype}
  \UseMicrotypeSet[protrusion]{basicmath} % disable protrusion for tt fonts
}{}
\makeatletter
\@ifundefined{KOMAClassName}{% if non-KOMA class
  \IfFileExists{parskip.sty}{%
    \usepackage{parskip}
  }{% else
    \setlength{\parindent}{0pt}
    \setlength{\parskip}{6pt plus 2pt minus 1pt}}
}{% if KOMA class
  \KOMAoptions{parskip=half}}
\makeatother
\usepackage{xcolor}
\IfFileExists{xurl.sty}{\usepackage{xurl}}{} % add URL line breaks if available
\IfFileExists{bookmark.sty}{\usepackage{bookmark}}{\usepackage{hyperref}}
\hypersetup{
  hidelinks,
  pdfcreator={LaTeX via pandoc}}
\urlstyle{same} % disable monospaced font for URLs
\usepackage[margin=1in]{geometry}
\usepackage{color}
\usepackage{fancyvrb}
\newcommand{\VerbBar}{|}
\newcommand{\VERB}{\Verb[commandchars=\\\{\}]}
\DefineVerbatimEnvironment{Highlighting}{Verbatim}{commandchars=\\\{\}}
% Add ',fontsize=\small' for more characters per line
\usepackage{framed}
\definecolor{shadecolor}{RGB}{248,248,248}
\newenvironment{Shaded}{\begin{snugshade}}{\end{snugshade}}
\newcommand{\AlertTok}[1]{\textcolor[rgb]{0.94,0.16,0.16}{#1}}
\newcommand{\AnnotationTok}[1]{\textcolor[rgb]{0.56,0.35,0.01}{\textbf{\textit{#1}}}}
\newcommand{\AttributeTok}[1]{\textcolor[rgb]{0.77,0.63,0.00}{#1}}
\newcommand{\BaseNTok}[1]{\textcolor[rgb]{0.00,0.00,0.81}{#1}}
\newcommand{\BuiltInTok}[1]{#1}
\newcommand{\CharTok}[1]{\textcolor[rgb]{0.31,0.60,0.02}{#1}}
\newcommand{\CommentTok}[1]{\textcolor[rgb]{0.56,0.35,0.01}{\textit{#1}}}
\newcommand{\CommentVarTok}[1]{\textcolor[rgb]{0.56,0.35,0.01}{\textbf{\textit{#1}}}}
\newcommand{\ConstantTok}[1]{\textcolor[rgb]{0.00,0.00,0.00}{#1}}
\newcommand{\ControlFlowTok}[1]{\textcolor[rgb]{0.13,0.29,0.53}{\textbf{#1}}}
\newcommand{\DataTypeTok}[1]{\textcolor[rgb]{0.13,0.29,0.53}{#1}}
\newcommand{\DecValTok}[1]{\textcolor[rgb]{0.00,0.00,0.81}{#1}}
\newcommand{\DocumentationTok}[1]{\textcolor[rgb]{0.56,0.35,0.01}{\textbf{\textit{#1}}}}
\newcommand{\ErrorTok}[1]{\textcolor[rgb]{0.64,0.00,0.00}{\textbf{#1}}}
\newcommand{\ExtensionTok}[1]{#1}
\newcommand{\FloatTok}[1]{\textcolor[rgb]{0.00,0.00,0.81}{#1}}
\newcommand{\FunctionTok}[1]{\textcolor[rgb]{0.00,0.00,0.00}{#1}}
\newcommand{\ImportTok}[1]{#1}
\newcommand{\InformationTok}[1]{\textcolor[rgb]{0.56,0.35,0.01}{\textbf{\textit{#1}}}}
\newcommand{\KeywordTok}[1]{\textcolor[rgb]{0.13,0.29,0.53}{\textbf{#1}}}
\newcommand{\NormalTok}[1]{#1}
\newcommand{\OperatorTok}[1]{\textcolor[rgb]{0.81,0.36,0.00}{\textbf{#1}}}
\newcommand{\OtherTok}[1]{\textcolor[rgb]{0.56,0.35,0.01}{#1}}
\newcommand{\PreprocessorTok}[1]{\textcolor[rgb]{0.56,0.35,0.01}{\textit{#1}}}
\newcommand{\RegionMarkerTok}[1]{#1}
\newcommand{\SpecialCharTok}[1]{\textcolor[rgb]{0.00,0.00,0.00}{#1}}
\newcommand{\SpecialStringTok}[1]{\textcolor[rgb]{0.31,0.60,0.02}{#1}}
\newcommand{\StringTok}[1]{\textcolor[rgb]{0.31,0.60,0.02}{#1}}
\newcommand{\VariableTok}[1]{\textcolor[rgb]{0.00,0.00,0.00}{#1}}
\newcommand{\VerbatimStringTok}[1]{\textcolor[rgb]{0.31,0.60,0.02}{#1}}
\newcommand{\WarningTok}[1]{\textcolor[rgb]{0.56,0.35,0.01}{\textbf{\textit{#1}}}}
\usepackage{longtable,booktabs}
% Correct order of tables after \paragraph or \subparagraph
\usepackage{etoolbox}
\makeatletter
\patchcmd\longtable{\par}{\if@noskipsec\mbox{}\fi\par}{}{}
\makeatother
% Allow footnotes in longtable head/foot
\IfFileExists{footnotehyper.sty}{\usepackage{footnotehyper}}{\usepackage{footnote}}
\makesavenoteenv{longtable}
\usepackage{graphicx}
\makeatletter
\def\maxwidth{\ifdim\Gin@nat@width>\linewidth\linewidth\else\Gin@nat@width\fi}
\def\maxheight{\ifdim\Gin@nat@height>\textheight\textheight\else\Gin@nat@height\fi}
\makeatother
% Scale images if necessary, so that they will not overflow the page
% margins by default, and it is still possible to overwrite the defaults
% using explicit options in \includegraphics[width, height, ...]{}
\setkeys{Gin}{width=\maxwidth,height=\maxheight,keepaspectratio}
% Set default figure placement to htbp
\makeatletter
\def\fps@figure{htbp}
\makeatother
\setlength{\emergencystretch}{3em} % prevent overfull lines
\providecommand{\tightlist}{%
  \setlength{\itemsep}{0pt}\setlength{\parskip}{0pt}}
\setcounter{secnumdepth}{5}
\usepackage{tikz} \usepackage{auto-pst-pdf} \usepackage{pgfornament} \usepackage{pstricks-add} \usepackage{caption} \captionsetup{font={footnotesize},width=6in} \renewcommand{\dblfloatpagefraction}{.9} \makeatletter \renewenvironment{figure} {\def\@captype{figure}} \makeatother \@ifundefined{Shaded}{\newenvironment{Shaded}} \@ifundefined{snugshade}{\newenvironment{snugshade}} \renewenvironment{Shaded}{\begin{snugshade}}{\end{snugshade}} \definecolor{shadecolor}{RGB}{230,230,230} \usepackage{xeCJK} \usepackage{setspace} \setstretch{1.3} \usepackage{tcolorbox} \setcounter{secnumdepth}{4} \setcounter{tocdepth}{4} \usepackage{wallpaper} \usepackage[absolute]{textpos} \tcbuselibrary{breakable} \renewenvironment{Shaded} {\begin{tcolorbox}[colback = gray!10, colframe = gray!40, width = 16cm, arc = 1mm, auto outer arc, title = {R input}]} {\end{tcolorbox}} \usepackage{titlesec} \titleformat{\paragraph} {\fontsize{10pt}{0pt}\bfseries} {\arabic{section}.\arabic{subsection}.\arabic{subsubsection}.\arabic{paragraph}} {1em} {} []

\author{}
\date{\vspace{-2.5em}}

\begin{document}

\begin{titlepage} \newgeometry{top=7.5cm}
\begin{center} \textbf{\Huge GEO
数据快速查询与获取} \vspace{4em}
\begin{textblock}{10}(3,5.9) \huge
\textbf{\textcolor{black}{2025-01-02}}
\end{textblock} \begin{textblock}{10}(3,7.3)
\Large \textcolor{black}{LiChuang Huang}
\end{textblock} \end{center} \end{titlepage}
\restoregeometry

\hypertarget{ux5b89ux88c5ux4f9dux8d56}{%
\section{安装依赖}\label{ux5b89ux88c5ux4f9dux8d56}}

对于该文档所述的功能,只需要两个工具,\texttt{EDirect},以及我的R包 \texttt{utils.tool} (从 github 获取)。
\texttt{utils.tool} 需要很多 R 包依赖,例如这里主要的 \texttt{GEOquery} R 包。
还可能存在一些你没有安装过的R包。

对于该功能,我已经在服务器 (账号:

\texttt{HostName\ ssh.cn-zhongwei-1.paracloud.com} \newline
\texttt{User\ t0s000324@BSCC-T} \newline

) 中部署完毕 (conda: \texttt{r4-base}),可直接使用,无需再安装了。

\begin{Shaded}
\begin{Highlighting}[]
\CommentTok{\#\# 这是一个 Linux 命令行工具}
\KeywordTok{system}\NormalTok{(}\StringTok{\textquotesingle{}sh {-}c "$(wget {-}q https://ftp.ncbi.nlm.nih.gov/entrez/entrezdirect/install{-}edirect.sh {-}O {-})"\textquotesingle{}}\NormalTok{)}
\CommentTok{\#\# 获取 github 中的 R 包。}
\KeywordTok{system}\NormalTok{(}\StringTok{"git clone {-}{-}depth 1 https://github.com/shaman{-}yellow/utils.tool \textasciitilde{}/utils.tool"}\NormalTok{)}
\CommentTok{\#\# 由于该包很多方法没有导出,所以无法通过 install 来使用。请使用以下方式加载:}
\NormalTok{devtools}\OperatorTok{::}\KeywordTok{load\_all}\NormalTok{(}\StringTok{"\textasciitilde{}/utils.tool"}\NormalTok{)}
\end{Highlighting}
\end{Shaded}

\hypertarget{ux57faux672cux65b9ux6cd5}{%
\section{基本方法}\label{ux57faux672cux65b9ux6cd5}}

\hypertarget{ux52a0ux8f7dux5305}{%
\subsection{加载包}\label{ux52a0ux8f7dux5305}}

\begin{Shaded}
\begin{Highlighting}[]
\CommentTok{\#\# devtools::load\_all, 相当于 \textasciigrave{}library\textasciigrave{} 这个命令}
\NormalTok{devtools}\OperatorTok{::}\KeywordTok{load\_all}\NormalTok{(}\StringTok{"\textasciitilde{}/utils.tool"}\NormalTok{)}
\CommentTok{\#\# 设置缓存路径}
\KeywordTok{set\_prefix}\NormalTok{(}\StringTok{"\textasciitilde{}/cache"}\NormalTok{)}
\end{Highlighting}
\end{Shaded}

\hypertarget{ux4e3bux8981ux65b9ux6cd5}{%
\subsection{主要方法}\label{ux4e3bux8981ux65b9ux6cd5}}

以下所有,除了 \texttt{job\_gds}, 都是 S4 泛型方法。

\begin{Shaded}
\begin{Highlighting}[]
\CommentTok{\#\# 以 EDirect 查询 GEO 信息,整理成数据框导入}
\NormalTok{job\_gds}
\CommentTok{\#\# 预设的一些过滤条件}
\NormalTok{step1}
\CommentTok{\#\# 预设的一些过滤方法}
\NormalTok{step2}
\CommentTok{\#\# 一种交互式操作,快速格式化元数据的列,整理出 sample, group 列}
\NormalTok{expect}
\CommentTok{\#\# 针对 group 列,形成 markdown 格式文本}
\NormalTok{anno}
\end{Highlighting}
\end{Shaded}

\hypertarget{ux67e5ux770bux65b9ux6cd5ux672cux4f53}{%
\subsection{查看方法本体}\label{ux67e5ux770bux65b9ux6cd5ux672cux4f53}}

S4 在查看函数本体上不如普通的 function 方便,但你可以用以下方法查看:

\begin{Shaded}
\begin{Highlighting}[]
\KeywordTok{selectMethod}\NormalTok{(step1, }\StringTok{"job\_gds"}\NormalTok{)}
\KeywordTok{selectMethod}\NormalTok{(step2, }\StringTok{"job\_gds"}\NormalTok{)}
\KeywordTok{selectMethod}\NormalTok{(expect, }\StringTok{"job\_gds"}\NormalTok{)}
\KeywordTok{selectMethod}\NormalTok{(anno, }\StringTok{"job\_gds"}\NormalTok{)}
\end{Highlighting}
\end{Shaded}

或者,也可以直接查看 R 包中的源代码。

\begin{Shaded}
\begin{Highlighting}[]
\KeywordTok{readLines}\NormalTok{(}\StringTok{"\textasciitilde{}/utils.tool/R/workflow\_88\_gds.R"}\NormalTok{)}
\end{Highlighting}
\end{Shaded}

\hypertarget{ux4f7fux7528ux793aux4f8b}{%
\section{使用示例}\label{ux4f7fux7528ux793aux4f8b}}

\hypertarget{ux6700ux521d}{%
\subsection{最初}\label{ux6700ux521d}}

先加载这个包。

\begin{Shaded}
\begin{Highlighting}[]
\NormalTok{devtools}\OperatorTok{::}\KeywordTok{load\_all}\NormalTok{(}\StringTok{"\textasciitilde{}/utils.tool"}\NormalTok{)}
\KeywordTok{set\_prefix}\NormalTok{(}\StringTok{"\textasciitilde{}/cache"}\NormalTok{)}
\end{Highlighting}
\end{Shaded}

\hypertarget{ux67e5ux8be2ux7761ux7720ux547cux5438ux6682ux505cux75c7}{%
\subsection{查询睡眠呼吸暂停症}\label{ux67e5ux8be2ux7761ux7720ux547cux5438ux6682ux505cux75c7}}

以 ``睡眠呼吸暂停症'' 为例子,查询 GEO 可用数据。

\begin{Shaded}
\begin{Highlighting}[]
\CommentTok{\#\# org 参数可以指定物种,例如 Humo Sapiens,这里不指定}
\CommentTok{\#\# 这里还有一些默认参数,例如 n, 指定样本数量,默认是 6:1000}
\CommentTok{\#\# 第一个参数,即 c("Sleep apnea"),可以指定多个关键词,例如 c("Sleep apnea", "Healthy")}
\NormalTok{gds.sa \textless{}{-}}\StringTok{ }\KeywordTok{job\_gds}\NormalTok{(}\KeywordTok{c}\NormalTok{(}\StringTok{"Sleep apnea"}\NormalTok{), }\DataTypeTok{org =} \OtherTok{NULL}\NormalTok{)}
\end{Highlighting}
\end{Shaded}

注意,\texttt{job\_gds} 会形成本地缓存,下次免于联网搜索,如果你一定要重新搜索,请指定参数:\texttt{force\ =\ TRUE}。

\hypertarget{ux5f97ux5230ux7684ux7ed3ux679c-ux5df2ux7ecfux5f97ux5230ux4e86ux6570ux636eux8868}{%
\subsection{得到的结果 (已经得到了数据表)}\label{ux5f97ux5230ux7684ux7ed3ux679c-ux5df2ux7ecfux5f97ux5230ux4e86ux6570ux636eux8868}}

可以从 \texttt{gds.sa@object} 中查看运行结果。

\begin{Shaded}
\begin{Highlighting}[]
\NormalTok{gds.sa}\OperatorTok{@}\NormalTok{object}
\end{Highlighting}
\end{Shaded}

\begin{verbatim}
## # A tibble: 12 x 7
##    GSE       title                     summary taxon gdsType n_samples PubMedIds
##    <chr>     <chr>                     <chr>   <chr> <chr>       <int> <chr>    
##  1 GSE242668 Long-term intermittent h~ Obstru~ Mus ~ Expres~        10 PRJNA101~
##  2 GSE235867 Low testosterone and int~ Interm~ Mus ~ Expres~        20 PRJNA987~
##  3 GSE215935 mRNA, lncRNA, and circRN~ Backgr~ Mus ~ Expres~         6 PRJNA891~
##  4 GSE189958 Combined intermittent an~ Study ~ Mus ~ Expres~        16 PRJNA785~
##  5 GSE145435 Short-term exposure to i~ Obstru~ Mus ~ Expres~         6 PRJNA607~
##  6 GSE145434 Short-term exposure to i~ Obstru~ Mus ~ Expres~        12 PRJNA607~
##  7 GSE145221 Expression data from mou~ Athero~ Mus ~ Expres~        23 PRJNA606~
##  8 GSE21409  Chronic Intermittent Hyp~ Obstru~ Mus ~ Expres~        10 PRJNA126~
##  9 GSE14981  Distinct Mechanisms Unde~ Backgr~ Dros~ Expres~         9 PRJNA111~
## 10 GSE2271   Gene expression and phen~ Chroni~ Mus ~ Expres~        18 PRJNA913~
## 11 GSE1873   The effect of intermitte~ All an~ Mus ~ Expres~        10 PRJNA905~
## 12 GSE480    Sleep apnea and glucose ~ This s~ Mus ~ Expres~        20 PRJNA851~
\end{verbatim}

\hypertarget{ux6b63ux5219ux5339ux914dux8fc7ux6ee4-ux53efux9009ux7684}{%
\subsection{正则匹配过滤 (可选的)}\label{ux6b63ux5219ux5339ux914dux8fc7ux6ee4-ux53efux9009ux7684}}

\begin{Shaded}
\begin{Highlighting}[]
\CommentTok{\#\# 函数 \textasciigrave{}grpl\textasciigrave{} 是 \textasciigrave{}grepl\textasciigrave{} 的封装,只是改变了参数顺序。 }
\CommentTok{\#\# clinical 会按照预设的一些条件,过滤掉一些数据,请查看 \textasciigrave{}selectMethod(step1, "job\_gds")\textasciigrave{}}
\CommentTok{\#\# 这里筛选了包含 \textasciigrave{}Intermittent hypoxia\textasciigrave{} 的数据。}
\NormalTok{gds.sa \textless{}{-}}\StringTok{ }\KeywordTok{step1}\NormalTok{(}
\NormalTok{  gds.sa, }\DataTypeTok{clinical =} \OtherTok{FALSE}\NormalTok{, }\OperatorTok{!}\KeywordTok{grpl}\NormalTok{(taxon, }\StringTok{"Homo Sapiens"}\NormalTok{, }\OtherTok{TRUE}\NormalTok{),}
  \KeywordTok{grpl}\NormalTok{(summary, }\StringTok{"Intermittent hypoxia"}\NormalTok{, }\OtherTok{TRUE}\NormalTok{)}
\NormalTok{)}
\end{Highlighting}
\end{Shaded}

可以以下方式,跳过这一步。

\begin{Shaded}
\begin{Highlighting}[]
\NormalTok{gds.sa}\OperatorTok{@}\NormalTok{step \textless{}{-}}\StringTok{ }\NormalTok{1L}
\end{Highlighting}
\end{Shaded}

\hypertarget{ux83b7ux53d6ux5143ux6570ux636e}{%
\subsection{获取元数据}\label{ux83b7ux53d6ux5143ux6570ux636e}}

\begin{Shaded}
\begin{Highlighting}[]
\CommentTok{\#\# 会下载数据集,请注意,尽量避免一次性下载过多,所以过滤 \textasciigrave{}ges.sa@object\textasciigrave{} 中的数据是必要的}
\CommentTok{\#\# 需要等待一会儿}
\NormalTok{ges.sa \textless{}{-}}\StringTok{ }\KeywordTok{step2}\NormalTok{(gds.sa)}
\end{Highlighting}
\end{Shaded}

\hypertarget{ux67e5ux770bux5143ux6570ux636e}{%
\subsection{查看元数据}\label{ux67e5ux770bux5143ux6570ux636e}}

基本上,该数据集能否用于你的分析,看一下这个结果就能知道了。
例如生存分析,你至少要在结果中找到 survival 对应的数据记录。

\begin{Shaded}
\begin{Highlighting}[]
\KeywordTok{head}\NormalTok{(gds.sa}\OperatorTok{@}\NormalTok{params}\OperatorTok{$}\NormalTok{res}\OperatorTok{$}\NormalTok{metas, }\DataTypeTok{n =} \DecValTok{1}\NormalTok{)}
\end{Highlighting}
\end{Shaded}

\begin{verbatim}
## $GSE242668
## # A tibble: 10 x 11
##    sample     group  rownames title age.at_the_treatment~1 diet.ch1 genotype.ch1
##    <chr>      <chr>  <chr>    <chr> <chr>                  <chr>    <chr>       
##  1 GSM7766596 inter~ GSM7766~ ih36~ 16 weeks               regular~ C57BL/6JRj ~
##  2 GSM7766597 inter~ GSM7766~ ih37~ 16 weeks               regular~ C57BL/6JRj ~
##  3 GSM7766598 inter~ GSM7766~ ih38~ 16 weeks               regular~ C57BL/6JRj ~
##  4 GSM7766599 inter~ GSM7766~ ih39~ 16 weeks               regular~ C57BL/6JRj ~
##  5 GSM7766600 inter~ GSM7766~ ih40~ 16 weeks               regular~ C57BL/6JRj ~
##  6 GSM7766601 normo~ GSM7766~ no31~ 16 weeks               regular~ C57BL/6JRj ~
##  7 GSM7766602 normo~ GSM7766~ no32~ 16 weeks               regular~ C57BL/6JRj ~
##  8 GSM7766603 normo~ GSM7766~ no33~ 16 weeks               regular~ C57BL/6JRj ~
##  9 GSM7766604 normo~ GSM7766~ no34~ 16 weeks               regular~ C57BL/6JRj ~
## 10 GSM7766605 normo~ GSM7766~ no35~ 16 weeks               regular~ C57BL/6JRj ~
## # i abbreviated name: 1: age.at_the_treatment_onset.ch1
## # i 4 more variables: Sex.ch1 <chr>, tissue.ch1 <chr>,
## #   treatment.duration.ch1 <chr>, treatment.ch1 <chr>
\end{verbatim}

\hypertarget{ux5febux901fux683cux5f0fux5316ux5206ux7ec4ux4fe1ux606f-ux53efux9009}{%
\subsection{快速格式化分组信息 (可选)}\label{ux5febux901fux683cux5f0fux5316ux5206ux7ec4ux4fe1ux606f-ux53efux9009}}

这是一个极其方便的工具,查找可能存在的 ``group'' 列,交互式 (并生成本地记录) 提示,可能让你手动指定。
请自行探索。

\begin{Shaded}
\begin{Highlighting}[]
\NormalTok{gds.sa \textless{}{-}}\StringTok{ }\KeywordTok{expect}\NormalTok{(gds.sa, }\KeywordTok{geo\_cols}\NormalTok{())}
\CommentTok{\#\# 结果请查看}
\NormalTok{gds.sa}\OperatorTok{@}\NormalTok{params}\OperatorTok{$}\NormalTok{res}\OperatorTok{$}\NormalTok{metas}
\end{Highlighting}
\end{Shaded}

\hypertarget{ux5bf9-group-ux5217ux751fux6210ux603bux7ed3-ux53efux9009}{%
\subsection{对 Group 列生成总结 (可选)}\label{ux5bf9-group-ux5217ux751fux6210ux603bux7ed3-ux53efux9009}}

\begin{Shaded}
\begin{Highlighting}[]
\NormalTok{gds.sa \textless{}{-}}\StringTok{ }\KeywordTok{anno}\NormalTok{(gds.sa)}
\end{Highlighting}
\end{Shaded}

可以通过以下方式查看结果

\begin{Shaded}
\begin{Highlighting}[]
\NormalTok{gds.sa}\OperatorTok{@}\NormalTok{snap}
\end{Highlighting}
\end{Shaded}

或者:

\begin{Shaded}
\begin{Highlighting}[]
\KeywordTok{writeLines}\NormalTok{(}\KeywordTok{snap}\NormalTok{(gds.sa, }\StringTok{"a"}\NormalTok{))}
\end{Highlighting}
\end{Shaded}

\hypertarget{ux6700ux7ec8ux6548ux679cux5c55ux73b0}{%
\subsection{最终效果展现}\label{ux6700ux7ec8ux6548ux679cux5c55ux73b0}}

上述步骤都运行后,可得到:

\begin{Shaded}
\begin{Highlighting}[]
\KeywordTok{writeLines}\NormalTok{(}\KeywordTok{snap}\NormalTok{(gds.sa, }\StringTok{"a"}\NormalTok{))}
\end{Highlighting}
\end{Shaded}

\begin{itemize}
\tightlist
\item
  \textbf{GSE242668}, \textbf{Type}: RNA-seq

  \begin{itemize}
  \tightlist
  \item
    intermittent\_hypoxia (n = 5)
  \item
    normoxic\_control (n = 5)
  \end{itemize}
\item
  \textbf{GSE235867}, \textbf{Type}: RNA-seq

  \begin{itemize}
  \tightlist
  \item
    ORX-IH (n = 5)
  \item
    ORX-Nx (n = 5)
  \item
    Sham-IH (n = 5)
  \item
    Sham-Nx (n = 5)
  \end{itemize}
\item
  \textbf{GSE215935}, \textbf{Type}: Microarray; Non-coding RNA-seq

  \begin{itemize}
  \tightlist
  \item
    chronic intermittent hypoxia (CIH) system combined with Ang II (n = 3)
  \item
    normal saline (n = 3)
  \end{itemize}
\item
  \textbf{GSE189958}, \textbf{Type}: RNA-seq

  \begin{itemize}
  \tightlist
  \item
    Intermittent hypoxia (n = 4)
  \item
    Overlap hypoxia (n = 4)
  \item
    Room air (n = 4)
  \item
    Sustained hypoxia (n = 4)
  \end{itemize}
\item
  \textbf{GSE145435}, \textbf{Type}: (scRNA-seq) RNA-seq

  \begin{itemize}
  \tightlist
  \item
    Ctrl (n = 3)
  \item
    Hypo (n = 3)
  \end{itemize}
\item
  \textbf{GSE145434}, \textbf{Type}: RNA-seq

  \begin{itemize}
  \tightlist
  \item
    CTRL (n = 6)
  \item
    HYPO (n = 6)
  \end{itemize}
\item
  \textbf{GSE145221}, \textbf{Type}: Microarray

  \begin{itemize}
  \tightlist
  \item
    CIH for 12 (n = 4)
  \item
    CIH for 8 (n = 5)
  \item
    CIH for 8 weeks followed by normoxia for 4 (n = 4)
  \item
    normoxia for 12 (n = 5)
  \item
    normoxia for 8 (n = 5)
  \end{itemize}
\item
  \textbf{GSE21409}, \textbf{Type}: Microarray

  \begin{itemize}
  \tightlist
  \item
    Interm Hypoxia (n = 5)
  \item
    Normoxia (n = 5)
  \end{itemize}
\item
  \textbf{GSE14981}, \textbf{Type}: Microarray

  \begin{itemize}
  \tightlist
  \item
    CH (n = 3)
  \item
    IH (n = 3)
  \item
    NC (n = 3)
  \end{itemize}
\item
  \textbf{GSE2271}, \textbf{Type}: Microarray

  \begin{itemize}
  \tightlist
  \item
    mouse subjected (n = 2)
  \item
    mouse subjected 1 week to chronic constant hypoxia (n = 1)
  \item
    mouse subjected 1 week to chronic intermittent hypoxia (n = 2)
  \item
    mouse subjected 2 week to chronic constant hypoxia (n = 1)
  \item
    mouse subjected 2 week to chronic intermittent hypoxia (n = 2)
  \item
    mouse subjected 4 week to chronic constant hypoxia (n = 2)
  \item
    mouse subjected 4 week to chronic intermittent hypoxia (n = 2)
  \item
    mouse, 1 week control (n = 2)
  \item
    mouse, 2 week control (n = 2)
  \item
    mouse, 4 week control (n = 2)
  \end{itemize}
\item
  \textbf{GSE1873}, \textbf{Type}: Microarray

  \begin{itemize}
  \tightlist
  \item
    Liver, Intermittent Hypoxia (n = 5)
  \item
    Liver, Normoxia (n = 5)
  \end{itemize}
\item
  \textbf{GSE480}, \textbf{Type}: Microarray

  \begin{itemize}
  \tightlist
  \item
    PGA-MGM-ConBrain (n = 1)
  \item
    PGA-MGM-ConHeart (n = 1)
  \item
    PGA-MGM-ConHyp (n = 2)
  \item
    PGA-MGM-ConLung (n = 1)
  \item
    PGA-MGM-ConMuscle (n = 1)
  \item
    PGA-MGM-ConNonHyp (n = 2)
  \item
    PGA-MGM-FragBrain (n = 1)
  \item
    PGA-MGM-FragHeart (n = 1)
  \item
    PGA-MGM-FragLung (n = 1)
  \item
    PGA-MGM-FragMuscle (n = 1)
  \item
    PGA-MGM-GlucoseHyp (n = 2)
  \item
    PGA-MGM-GlucoseNonHyp (n = 2)
  \item
    PGA-MGM-HypoxiaBrain (n = 1)
  \item
    PGA-MGM-HypoxiaHeart (n = 1)
  \item
    PGA-MGM-HypoxiaLung (n = 1)
  \item
    PGA-MGM-HypoxiaMuscle (n = 1)
  \end{itemize}
\end{itemize}

\hypertarget{ux8865ux5145ux8bf4ux660e}{%
\section{补充说明}\label{ux8865ux5145ux8bf4ux660e}}

\hypertarget{ux5173ux4e8e-job_gds}{%
\subsection{\texorpdfstring{关于 \texttt{job\_gds}}{关于 job\_gds}}\label{ux5173ux4e8e-job_gds}}

\texttt{gds.sa\ \textless{}-\ job\_gds(c("Sleep\ apnea"),\ org\ =\ NULL)} 的运行效果,相当于以下:

\begin{Shaded}
\begin{Highlighting}[]
\KeywordTok{system}\NormalTok{(}\StringTok{"esearch {-}db gds {-}query \textquotesingle{}(Sleep apnea[Description]) AND ((6:1000[Number of Samples]) AND (GSE[Entry Type]))\textquotesingle{} |efetch {-}format docsum |xtract {-}pattern DocumentSummary {-}sep \textquotesingle{}|\textquotesingle{} {-}element GSE title summary taxon gdsType n\_samples PubMedIds BioProject \textgreater{} \textasciitilde{}/cache/query\_gds/X.Sleep.apnea.Description...AND...6.1000.Number.of.Samples...AND..GSE.Entry.Type..."}\NormalTok{)}
\end{Highlighting}
\end{Shaded}

请参考 \url{https://www.ncbi.nlm.nih.gov/books/NBK3837/} 官方文档说明。

\hypertarget{ux5173ux4e8e-step2-ux83b7ux53d6ux6570ux636e}{%
\subsection{\texorpdfstring{关于 \texttt{step2} 获取数据}{关于 step2 获取数据}}\label{ux5173ux4e8e-step2-ux83b7ux53d6ux6570ux636e}}

是以下函数的封装:

\begin{Shaded}
\begin{Highlighting}[]
\NormalTok{GEOquery}\OperatorTok{::}\NormalTok{getGEO}
\end{Highlighting}
\end{Shaded}


\end{document}
